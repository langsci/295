\title{A grammar of Japhug}  %look no further, you can change those things right here.
\subtitle{}
\BackTitle{A grammar of Japhug} % Change if BackTitle != Title
\BackBody{Japhug is a vulnerable Gyalrongic language, which belongs to the Trans-Himalayan (Sino-Tibetan) family. It is spoken by several thousand speakers in Mbarkham county, Rngaba district, Sichuan province, China. This grammar is the result of nearly 20 years of fieldwork on one variety of Japhug, based on a corpus of narratives and conversations, a large part of which is available from the Pangloss Collection. It covers the whole grammar of the language, and the text examples provide a unique insight into Gyalrong culture. It was written with a general linguistics audience in mind, and should prove useful not only to specialists of Trans-Himalayan historical linguistics and typologists, but also to anthropologists doing research in Gyalrong areas. It is also hoped that some readers will use it to learn Japhug and pursue research on this fascinating language in the future.}
\dedication{Pour mon amour, Mirabelle \\ \zh{题献给爱妻米哈}}
\typesetter{Guillaume Jacques}
\proofreader{Anton Antonov, Curtis Bartosik, Aimée Lahaussois, Laura Arnold, Tom Bossuyt, Michael Daniel, Valérie Guérin, Ariel Gutman, Andreas Hölzl, Katarzyna Janic, Seppo Kittilä, Joseph Lovestrand, Lachlan Mackenzie, Bruno Olsson, Bastian Persohn, Yvonne Treis, Jeroen van de Weijer and Ye Jingting \zh{叶婧婷}}
\author{Guillaume Jacques}
 \BookDOI{10.5281/zenodo.4548232}%ask coordinator for DOI
\renewcommand{\lsISBNdigital}{978-3-96110-305-8}
\renewcommand{\lsISBNhardcover}{978-3-98554-001-3}
\renewcommand{\lsSeries}{cgl}
\renewcommand{\lsSeriesNumber}{1}
\renewcommand{\lsURL}{http://langsci-press.org/catalog/book/295} 
\renewcommand{\lsID}{295}
 
 
 
 
  
