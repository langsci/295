\chapter{Valency-decreasing derivations}  \label{chap:valency.decreasing.derivation}

 
\section{Passive} \label{sec:passive}
The passive derivation is marked by the prefix  \forme{a\trt}, a formative also found in the reciprocal derivation (§\ref{sec:redp.reciprocal}) and probably of denominal origin (§\ref{sec:applicative.history}). It derives an intransitive verb from a transitive one, whose intransitive subject corresponds to the object of the base verb (\japhug{tɤscoz}{letter} in \ref{ex:rAt} and \ref{ex:arAt}).\footnote{The derivation also affects the dynamicity of the verb and the TAME (§\ref{sec:passive.stative}). } The transitive subject cannot be expressed: passive verbs do not take overt agents.

\begin{exe}
	\ex 
	\begin{xlist}	
		\ex \label{ex:rAt}
		\gll ɯʑo kɯ tɤscoz rɤt \\
		\textsc{3sg} \textsc{erg} letter write:\textsc{fact} \\
		\glt `S/he will write a/the letter.' (elicited)
		\ex \label{ex:arAt}
		\gll tɤscoz a-rɤt \\
		letter \textsc{pass}-write:\textsc{fact} \\
		\glt `The letter is written/has been written.' (elicited)
	\end{xlist}	
\end{exe}


Like other verbs in \forme{a\trt}, passive verbs present vowel contraction and insertion of the peg circumfix \forme{k-...-ci} in Inferential forms (§\ref{sec:ifr.morphology}, §\ref{sec:preverbs.contracting.verbs}), as in (\ref{ex:pjAkAGAtWGci}) (where it is realized as \forme{-ɤ\trt}, following the rule in §\ref{sec:contraction}). In the Kamnyu dialect, the peg circumfix, although originally a secondary marker, has almost become the main exponent of the passive in the Inferential tenses.

\begin{exe}
\ex \label{ex:pjAkAGAtWGci}
 \gll kɯm ɲɤ-stʰoʁ ri, kɯm pjɤ-k-ɤ-ɣɤtɯɣ-ci ɕti tɕe,  \\
 door \textsc{ifr}-push \textsc{lnk} \textsc{dem} \textsc{ifr}.\textsc{ipfv}-\textsc{peg}-\textsc{pass}-lock-\textsc{peg} be:\textsc{aff}:\textsc{fact} \textsc{lnk} \\
\glt  `He pushed the door, but the door was locked.' (tWJo 2012, 22)
\end{exe} 

\tabref{tab:passive} presents a few representative examples of passive verbs, including  lexicalized ones whose meaning has become slightly shifted from that of the base verb (§\ref{sec:passive.lexicalized}). The presence of Tibetan loanwords such as \japhug{rtsi}{count} (from \tibet{རྩི་}{rtsi}{count}) among the base verbs shows that the passive derivation is still productive.
 
\begin{table}
\caption{Examples of the passive \forme{a-} prefix} \label{tab:passive}
\begin{tabular}{llllll}
\lsptoprule
Base verb & Derived verb \\
\midrule
\japhug{mpʰɯr}{wrap} & \japhug{ampʰɯr}{be wrapped}  \\
\japhug{βraʁ}{attach to} & \japhug{aβraʁ}{be attached to}  \\
\japhug{rtsi}{count} & \japhug{artsi}{be counted as}  \\
\japhug{ɲɟoʁ}{paste} & \japhug{aɲɟoʁ}{be glued} \\
\japhug{tsʰoʁ}{attach} & \japhug{atsʰoʁ}{be attached}  \\
\japhug{sti}{block} & \japhug{asti}{be blocked} \\
\midrule
\japhug{pa}{do} & \japhug{apa}{become}  \\
\japhug{βzu}{make} & \japhug{aβzu}{become, grow}  \\
\japhug{ta}{put} & \japhug{ata}{be on}  \\
\japhug{rku}{put in} & \japhug{arku}{be in}  \\
\lspbottomrule
\end{tabular}
\end{table}

The passive derivation only removes transitive subjects, and passivized verbs preserve oblique arguments or adjuncts. For instance, like the verb \japhug{βraʁ}{attach}, which includes a locative argument in its argument structure, the passive form \japhug{aβraʁ}{be attached} also selects a locative phrase, with a locative case as in (\ref{ex:xtWrkW.aBraR}) or in absolutive form as in (\ref{ex:kWmqhu.pjAkABraRci}).

 \begin{exe}
\ex \label{ex:xtWrkW.aBraR}
 \gll stuxsi ɣɯ ɯ-χcɤl ri xtɯrkɯ a-βraʁ, \\
 double.yoke \textsc{gen} \textsc{3sg}.\textsc{poss}-middle \textsc{loc} hide.ripe \textsc{pass}-attach:\textsc{fact} \\
\glt `The hide rope (connecting the yoke to the plough) is attached in the middle of the double yoke.'  (24-mbGo, 107)
\end{exe}

 \begin{exe}
\ex \label{ex:kWmqhu.pjAkABraRci}
 \gll tɤ-mu nɯ ɣɯ ɯ-tɕɯ nɯ kɯm-qʰu pjɤ-k-ɤ-βraʁ-ci tɕe,\\
 \textsc{indef}.\textsc{poss}-mother \textsc{dem} \textsc{gen} \textsc{3sg}.\textsc{poss}-son \textsc{dem} door-back \textsc{ipfv}-\textsc{peg}-\textsc{pass}-attach-\textsc{peg} \textsc{lnk} \\
\glt `The old woman's son was attached on the back of the door.' (tWJo 2012, 63)
  \end{exe}

The passive derivation also applies to some noun-verb collocations. For instance, in (\ref{ex:sAcW.malAt}),  passivized verb \forme{a-lɤt}, meaning `be locked' in collocation with the noun \japhug{sɤcɯ}{key}, has a meaning directly derived from that of the collocation of \forme{sɤcɯ} with the base verb \japhug{lɤt}{release}, which means `lock (the door)' (§\ref{sec:lexicalized.oblique.participle}). The noun \japhug{sɤcɯ}{key} is a semi-object (§\ref{sec:semi.object}) in this collocation. The intransitive subject of \forme{(sɤcɯ) a-lɤt} `be locked' in (\ref{ex:sAcW.malAt}) is \japhug{kɯm}{door}.
 

\begin{exe}
\ex \label{ex:sAcW.malAt}
 \gll  kɯm sɤcɯ mɤ-a-lɤt \\
door key \textsc{neg}-\textsc{pass}-release:\textsc{fact} \\
\glt `The door is not locked/has not been locked.' (140428 xiaohongmao-zh, 77)
 \end{exe}
 
 
Passive verb forms are almost never attested in first or second person forms, except for the highly lexicalized passives \japhug{apa}{become} and \japhug{aβzu}{become, grow} (§\ref{sec:passive.lexicalized}). Example (\ref{ex:parkua}) with \japhug{arku}{be in}\footnote{Although this passive verb is used as a semi-lexicalized existential verb (§\ref{sec:passive.lexicalized}), in this particular context it can be interpreted as expressing a resultative state with a definite agent. } shows that there is not absolute constraints against first or second person indexation on passive verbs. 
 
\begin{exe}
\ex \label{ex:parkua}
 \gll  nɯnɯ pʰoŋ ɯ-ŋgɯ nɯtɕu pɯ-a-rku-a \\
 %nɯ mɯ́j-tɯ-stu ɲɯ-ŋu tɕe \\
 \textsc{dem} bottle \textsc{3sg}.\textsc{poss}-in \textsc{dem}:\textsc{loc} \textsc{pst}.\textsc{ipfv}-\textsc{pass}-put.in-\textsc{1sg} \\
 %\textsc{dem} \textsc{neg}:\textsc{sens}-2-believe \textsc{sens}-be \textsc{lnk} \\
 \glt `(You don't believe that) I had been put inside this bottle.' (140512 yufu yu mogui-zh, 119)
\end{exe}

The historical hypothesis in (§\ref{sec:portmanteau.prefixes.history}) proposes that the 1\fl{}2 portmanteau prefix \forme{ta-} originates from the fusion of  the passive \forme{a-} with the second person \forme{tɯ-} prefix indexing the object. If correct, this scenario implies that first or second person indexation on passive verbs may have been more common in proto-Gyalrong than in Japhug.

\subsection{The interaction of the Passive derivation with Dynamicity and TAME} \label{sec:passive.stative} 

With the exception of \japhug{apa}{become}  and \japhug{aβzu}{become, grow}  (§\ref{sec:passive.lexicalized}), passive verbs express resultative states. They are mainly attested in the Past Imperfective (\ref{ex:parkua}) and in the Inferential Imperfective (\ref{ex:kWmqhu.pjAkABraRci}, see also \ref{ex:pjAkAmphWrci} in §\ref{sec:passive.agent}). In combination with the Factual Non-Past the passive means `$X$ has already been $Y$ed' or `$X$ is $Y$ed' (where $X$ represents the object of the base verb $Y$) as in (\ref{ex:marAt}) (see also \ref{ex:arAt} and \ref{ex:sAcW.malAt} above).

\begin{exe}
\ex \label{ex:marAt}
 \gll  jaʁmɤzdoʁzdoʁ nɯ kɯnɤ pɣa ŋu, mɤ-a-rɤt \\
 bird.sp. \textsc{dem} also bird be:\textsc{fact} \textsc{neg}-\textsc{pass}-write:\textsc{fact} \\
 \glt `The \forme{jaʁmɤzdoʁzdoʁ} is also a bird, it is not written (in the list of bird's names that had been prepared beforehand to ask about all known bird species).' (23-RmWrcWftsa, 86)
\end{exe}

 Examples of passive verbs in Perfective or Inferential forms are very rare. In (\ref{ex:koKAYJoRci}), the form \forme{ko-k-ɤ-ɲɟoʁ-ci} can only be interpreted as the Inferential of the passive of \japhug{ɲɟoʁ}{paste}, with a meaning `(the spit) got glued (on her)' that does not seem interpretable as resultative.
 
 \begin{exe}
\ex \label{ex:koKAYJoRci}
 \gll   tɕelo kutɕu a-mci ci s-cʰɯ-βde-a, nɯ ɕɯ ɣɯ ndʑi-taʁ kɤ-kɯ-ɤ-ɲɟoʁ nɯ a-rʑaβ a-pɯ-tɯ-ŋu-ndʑi to-ti ri [...] tɕe ci s-cʰo-βde ri, rŋɯlɤsmɤn ɯ-taʁ ko-k-ɤ-ɲɟoʁ-ci ɕti ri, jaʁmɤcʰɯqa kɯ ɲɤ-nɯ-pɕiz qʰe ɯʑo ɯ-taʁ ko-nɯ-ɲɟoʁ. \\
 upstream \textsc{dem}.\textsc{prox}:\textsc{loc} \textsc{1sg}.\textsc{poss}-saliva once \textsc{tral}-\textsc{ipfv}:\textsc{downstream}-throw-\textsc{1sg}, \textsc{dem} who \textsc{gen} \textsc{2du}.\textsc{poss}-on \textsc{aor}-\textsc{sbj}:\textsc{pcp}-\textsc{pass}-glue \textsc{dem} \textsc{1sg}.\textsc{poss}-wife \textsc{irr}-\textsc{ipfv}-2-be-\textsc{du} \textsc{ifr}-say \textsc{lnk} {  } \textsc{lnk} once \textsc{tral}-\textsc{ifr}:\textsc{downstream}-throw \textsc{lnk}  \textsc{anthr} \textsc{3sg}.\textsc{poss}-on \textsc{ifr}-\textsc{peg}-\textsc{pass}-glue-\textsc{peg} be.\textsc{aff}:\textsc{fact} \textsc{lnk}   \textsc{anthr} \textsc{erg} \textsc{ifr}-\textsc{auto}-wipe \textsc{lnk} \textsc{3sg} \textsc{3sg}.\textsc{poss}-on  \textsc{ifr}-\textsc{auto}-glue \\
\glt `He said `I will spit from here, and whoever of you two gets glued by my spit will be my wife.' He spat, and although (his spit) got glued on Rngulasman, Yagmakhyiqa wiped it off and pasted it onto herself.' (2003-kWBRa, 48-50)
\end{exe}

The participle \ipa{kɤkɤɲɟoʁ} in this example could in principle either be parsed as a an perfective object participle \forme{kɤ-kɤ-ɲɟoʁ} (\textsc{aor}-\textsc{obj}:\textsc{pcp}-glue) or perfective subject passive participle \forme{kɤ-kɯ-ɤ-ɲɟoʁ} (\textsc{aor}-\textsc{sbj}:\textsc{pcp}-\textsc{pass}-glue), but only the second option is likely, due to the fact that the gluing action is spontaneous and lacks an agent.\footnote{Note in addition that the relativized element is the referent marked with the oblique relator noun  \japhug{ɯ-taʁ}{on, above} (§\ref{sec:WtaR}). } The fact that \japhug{ɲɟoʁ}{paste} lacks an anticausative (§\ref{sec:anticausative}) and that the passive form \forme{aɲɟoʁ} is used with an anticausative-like meaning (§\ref{sec:passive.agent}) may explain why the perfective forms are possible with this verb.

 
\subsection{Lexicalized passives} \label{sec:passive.lexicalized}
The verbs \japhug{pa}{do} (§\ref{sec:pa.lv}) and \japhug{βzu}{make} (§\ref{sec:Bzu.lv}) have lexicalized passive forms that are used as quasi-copulas  \japhug{apa}{become} and \japhug{aβzu}{become, grow} taking a semi-object serving as nominal predicate (§\ref{sec:copula.basic}), for instance \japhug{tɕʰi}{what} in (\ref{ex:anWtABzu}) or \forme{tɯrme (ci)} `a human' in (\ref{ex:anApa}) (for additional examples, see \ref{ex:nWnWra.pGa.nWra} in §\ref{sec:demonstrative.determiners}, \ref{ex:YAkABzuaci} and  \ref{ex:YAtABzu} in §\ref{sec:preverbs.contracting.verbs}). In addition, \forme{aβzu} is also used as a plain intransitive verb meaning `grow'.
 
 \begin{exe}
\ex \label{ex:anWtABzu}
 \gll nɤʑo tɕʰi a-nɯ-tɯ-ɤβzu ra? \\
 \textsc{2sg} what \textsc{irr}-\textsc{pfv}-2-become be.needed:\textsc{fact} \\
 \glt `What do you (want to) become?' (2003kandZislama, 36)
\end{exe}

 \begin{exe}
\ex \label{ex:anApa}
 \gll tɯrme ci a-nɯ-ɤpa-a kɯ, tɯrme ɲɯ-ɤβzu-a ci a-pɯ-kʰɯ kɯ \\
 human \textsc{indef} \textsc{irr}-\textsc{pfv}-become-\textsc{1sg} \textsc{sfp} human \textsc{ipfv}-become-\textsc{1sg} a.little \textsc{irr}-\textsc{ipfv}-be.possible \textsc{sfp} \\
 \glt `If only I could become a human!' (150819 haidenver-zh, 242)
 \end{exe}
 
In the perfective, the third person forms of  \japhug{aβzu}{become, grow} have an identical surface form with those of the transitive \japhug{βzu}{make}: for instance \ipa{naβzu} can be either parsed as \forme{nɯ-aβzu} (\textsc{aor}-become) or as \forme{na-βzu} (\textsc{aor}:3\flobv{}-make). Since the transitive \forme{βzu} has dummy subject functions (§\ref{sec:transitive.dummy}), in examples such as (\ref{ex:Wmat.thaBzu}), it is possible to analyze  the form \ipa{tʰaβzu} as  \forme{tʰa-βzu}  (\textsc{aor}:3\flobv{}-make) instead.
 
\begin{exe}
\ex \label{ex:Wmat.thaBzu}
 \gll ɯ-mat tʰɯ-aβzu tɕe tɕe \\
 \textsc{3sg}.\textsc{poss}-fruit \textsc{aor}-grow \textsc{lnk} \textsc{lnk} \\
 \glt `When its fruits grow...' (06-zrantcu; 25)
\end{exe}
  
The verbs \japhug{ta}{put} and \japhug{rku}{put in} have passive forms whose meaning is still predictable from the base verb,  \japhug{ata}{be on}  and \japhug{arku}{be in}.  However, these passive forms generally serve as existential verbs (§\ref{sec:existential.basic}) as in (\ref{ex:Wsno.ata}) and (\ref{ex:WNgW.arku}), without the implication that the state results from a manipulative action. While in (\ref{ex:Wsno.ata}) one could argue that an alternative translation such as `a saddle has been put there' would be possible,\footnote{In this example, the Factual actually can be construed as having a future interpretation (§\ref{sec:fact.main.clauses}): a character describes to another character what he will see when he arrives at a place. I translate here \forme{a-ta} as `there is' in the present tense rather than `there will be' because at the time of utterance, the wooden saddle is already placed next to the horse, and there is no implication that a change of state will occur. }  in the case of (\ref{ex:WNgW.arku}) it is obvious the presence of a sweet substance inside of a plant leaf is not attributable to an external agent. 
 
\begin{exe}
\ex \label{ex:Wsno.ata}
 \gll ɯ-rkɯ nɯtɕu si ɯ-sno ci a-ta tɕe, \\
\textsc{3sg}.\textsc{poss}-side \textsc{dem}:\textsc{loc} wood \textsc{3sg}.\textsc{poss}-saddle \textsc{indef} \textsc{pass}-put:\textsc{fact} \textsc{lnk} \\
\glt `Next to (the horse), there is a wooden saddle.' (2012 qachGa, 51)
\end{exe}

\begin{exe}
\ex \label{ex:WNgW.arku}
 \gll %tɕe kú-wɣ-sɯ-ɤsɯɣ tɕe, 
 nɯ ɯ-ŋgɯ nɯ kɯ-cʰi ɣʑɤzga kɯ-fse a-rku tɕe \\
 % nɯ kú-wɣ-nɯnɯ ŋgrɤl. \\
 %\textsc{lnk} \textsc{ipfv}-\textsc{caus}-be.pressed \textsc{lnk} 
 \textsc{dem} \textsc{3sg}.\textsc{poss}-in \textsc{dem} \textsc{sbj}:\textsc{pcp}-be.sweet honey \textsc{sbj}:\textsc{pcp}-be.like \textsc{pass}-put.in:\textsc{fact} \textsc{lnk} \\
 % \textsc{dem} \textsc{ipfv}-\textsc{inv}-suck be.usually.the.case:\textsc{fact} \\
 \glt `(When one presses on the flower of the \textit{Habenaria glaucifolia}), inside of it there is something sweet like honey.' (16-CWrNgo, 204)
  \end{exe}

A piece of evidence suggesting that the verbs \japhug{ata}{be on}  and \japhug{arku}{be in} are in the process of becoming specialized existential verbs (§\ref{sec:existential.basic}) is the fact that they alternate with the suppletive Sensory forms of the existential verbs (\japhug{ɣɤʑu}{exist} and \japhug{maŋe}{not exist}, §\ref{sec:intr.person.irregular}), as in (\ref{ex:pata.ri.maNe}). 

\begin{exe}
\ex \label{ex:pata.ri.maNe}
 \gll nɯtɕu pɯ-a-ta ri maŋe \\
 \textsc{dem}:\textsc{loc} \textsc{pst}.\textsc{ipfv}-\textsc{pass}-put \textsc{lnk} not.exist:\textsc{sens} \\
\glt `(The meat) used to be there, but now it is not there (any more).' (meimeidegushi, 74)
  \end{exe}

However, unlike the existential verbs \japhug{tu}{exist} and \japhug{me}{not exist}, both  \japhug{ata}{be on}  and \japhug{arku}{be in} have regular sensory forms, as in (\ref{ex:tWtaNGoR.YArku}).

\begin{exe}
\ex \label{ex:tWtaNGoR.YArku}
 \gll sɤtɕʰa ɯ-ŋgɯ nɯtɕu rŋɯl tɯ-taɴɢoʁ ɲɯ-ɤ-rku tɕe, \\
 ground \textsc{3sg}.\textsc{poss}-in \textsc{dem}:\textsc{loc} silver  one-basket \textsc{sens}-\textsc{pass}-put.in \textsc{lnk} \\
\glt `Inside the ground, there is a basketful of silver.' (2003 tamukatsa, 67)
\end{exe} 

\subsection{Agent demotion} \label{sec:passive.agent}
The \forme{a-} derivation demotes the transitive subject, and the agent is not expressible in the same clause. However, the agent is not necessary semantically deleted by the passive derivation. In (\ref{ex:pjAkAmphWrci}) for instance, the agent of the wrapping action is clear from the context both to the narrator of the story and to the person discovering the silver ingots inside the pieces of bread -- it is therefore semantically recoverable.

\begin{exe}
\ex \label{ex:pjAkAmphWrci}
 \gll ɯ-ŋgɯ nɯtɕu rŋɯl qʰoʁqʰoʁ tɯ-rdoʁ pjɤ-k-ɤ-mpʰɯr-ci, \\
 \textsc{3sg}.\textsc{poss}-in \textsc{dem}:\textsc{loc} silver ingot one-piece \textsc{ifr}.\textsc{ipfv}-\textsc{peg}-\textsc{pass}-wrap-\textsc{peg} \\
 \glt `A silver ingot had been wrapped inside.' (qajdoskAt 2002, 112)
\end{exe}

 
Many of the verbs that have an anticausative form (built by prenasalizing the onset, see Tables \ref{tab:anticausative.unaspirated} and \ref{tab:anticausative.aspirated}, §\ref{sec:anticausative} below) also have a passive. The passive is required in these cases when the agent is implicit and recoverable (for instance, \japhug{aprɤt}{have been broken} from \japhug{prɤt}{break} (vt), example \ref{ex:pjAkAprAtci} in §\ref{sec:anticausative.function}), while the anticausative is selected when the action occurs spontaneously without external agent (\japhug{mbrɤt}{break} (vi), example \ref{ex:nWmbrAt.pWmbrAt} in §\ref{sec:anticausative.function}).
 
 In some contexts, both passive and anticausative forms are possible. For instance, to express the fact that a piece of iron is attached to an object, both the passive \japhug{atsʰoʁ}{be attached} (\ref{ex:Com.atshoR}) and the anticausative \japhug{ndzoʁ}{be attached}  (\ref{ex:Com.kWndzoR}) (from 
\japhug{tsʰoʁ}{attach}) are attested.


\begin{exe}
\ex \label{ex:Com.atshoR}
 \gll ɯ-pa, tɕʰɯŋkʰɤr ɣɯ ɯ-spjɯŋ tu-kɯ-ɣi nɯnɯre ri li ɕom a-tsʰoʁ tɕe \\
  \textsc{3sg}.\textsc{poss}-down water.wheel \textsc{gen} \textsc{3sg}.\textsc{poss}-axle \textsc{ipfv}:\textsc{up}-\textsc{sbj}:\textsc{pcp}-come \textsc{dem}:\textsc{loc} \textsc{loc} again iron \textsc{pass}-attach:\textsc{fact} \textsc{lnk} \\
\glt `On the (top extremity of) the axle of the mill coming up from below, a piece of iron is attached.' (06-BGa, 66-67)
\end{exe}

\begin{exe}
\ex \label{ex:Com.kWndzoR}
 \gll nɯ ɯ-tʰɤcu ri ɕom kɯ-ndzoʁ ci tu tɕe, nɯ tʰaʁmu rmi. \\
 \textsc{dem} \textsc{3sg}.\textsc{poss}-downstream \textsc{loc} iron \textsc{sbj}:\textsc{pcp}-\textsc{acaus}:attach \textsc{indef} exist:\textsc{fact} \textsc{lnk} \textsc{dem} weaving.blade call:\textsc{fact} \\
 \glt `Below, there is (a weaving implement) that has a piece of iron attached to it, it is called \forme{tʰaʁmu}.' (thaXtsa 2002, 63)
\end{exe}

The anticausative \japhug{ndzoʁ}{be attached} is required in cases when the entity that is attached on a surface has grown on it, rather than having been attached by someone, as in (\ref{ex:WmWntoR.kundzoR}). In this function, it can be considered to be a subtype of existential verb (§\ref{sec:existential.basic}).

\begin{exe}
\ex \label{ex:WmWntoR.kundzoR}
 \gll ɯ-mɯntoʁ nɯ kɯnɤ cʰɯ-ɤʑirja tɕe tɯ-kʰɤl nɯtɕu ʁnɯz, χsɯm kɯβde jamar ku-ndzoʁ. \\
 \textsc{3sg}.\textsc{poss}-flower \textsc{dem} also \textsc{ipfv}:\textsc{downstream}-be.aligned \textsc{lnk} one-place \textsc{dem}:\textsc{loc} two three four about \textsc{ipfv}-\textsc{acaus}:attach \\
 \glt `Its flowers are also aligned, in each place they are attached in (groups) of about two, three or four.' (16-RlWmsWsi, 8-9)
\end{exe}

Another difference between \japhug{ndzoʁ}{be attached} and \japhug{atsʰoʁ}{be attached} is that the former only occurs as a resultative (like most passive verbs in Japhug, see §\ref{sec:passive.stative}), while \japhug{ndzoʁ}{be attached} can also mean `cling onto, lean on, grab' as in (\ref{ex:WtaR.kondzoR}) (§\ref{sec:anticausative.volitionality}).

In the case of verbs lacking an anticausative form however, such as  \japhug{ɲɟoʁ}{paste}, the passive forms can be used with an anticausative function, with a semantically deleted agent, as shown by examples such as (\ref{ex:koKAYJoRci}) in §\ref{sec:passive.stative}.

Passive verbs are as a rule incompatible with an overt agent, even when semantically recoverable. Other morphosyntactic devices such as a direct-inverse marking are used to express the relative saliency of the agent and the patient in Japhug (see the discussion in §\ref{sec:inverse.3.3.saliency}).

In example (\ref{ex:tAse.kW.pjAkAmarci}) with a passive verb, the ergative postpositional phrase \forme{tɤ-se kɯ} could appear to be an example of overt agent.\footnote{In the Chinese original from which this story is translated, the corresponding passage is \ch{满嘴鲜血}{mǎnzuǐ xiānxuè}{the mouth covered in blood}; the presence of ergative in (\ref{ex:tAse.kW.pjAkAmarci}) cannot be explained as calquing from Chinese. } However, note that the base verb \japhug{mar}{smear} can encode the material used to smear a surface as a semi-object (\japhug{taʁɟaz}{soot} in \ref{ex:taRJAz.tomar}) rather than as an instrument. Likewise, the ergative is optional on \japhug{tɤ-se}{blood} (compare with \ref{ex:tAse.ra.pjAkAmarci}).

\begin{exe}
\ex \label{ex:tAse.kW.pjAkAmarci}
 \gll xɕiri ɣɯ ɯ-mtɕʰi ra [tɤ-se kɯ] pjɤ-k-ɤ-mar-ci ʑo \\
 weasel \textsc{gen} \textsc{3sg}.\textsc{poss}-mouth \textsc{pl} \textsc{indef}.\textsc{poss}-blood \textsc{erg} \textsc{ifr}.\textsc{ipfv}-\textsc{peg}-\textsc{pass}-smear-\textsc{peg} \textsc{emph} \\
\glt  `The weasel's mouth was smeared with blood.' (140518 xuezhe he huangshulang-zh, 24)
\end{exe}

\begin{exe}
\ex \label{ex:taRJAz.tomar}
 \gll tɕʰeme nɯ kɯ  [...] ɯ-rŋa taʁɟaz to-mar \\
 girl \textsc{dem} \textsc{erg} { } \textsc{3sg}.\textsc{poss}-face soot \textsc{ifr}-smear \\
 \glt `The lady smeared her face with soot.' (2002 qaCpa, 267)
\end{exe}

\begin{exe}
\ex \label{ex:tAse.ra.pjAkAmarci}
 \gll  ɯ-mtɕʰi ɯ-taʁ tɕe tɤ-se ra pjɤ-k-ɤ-mar-ci ʑo. \\
 \textsc{3sg}.\textsc{poss}-mouth \textsc{3sg}.\textsc{poss}-on \textsc{loc} \textsc{indef}.\textsc{poss}-blood \textsc{pl} \textsc{ifr}.\textsc{ipfv}-\textsc{peg}-\textsc{pass}-smear-\textsc{peg} \textsc{emph} \\
\glt  `Its mouth was smeared with blood.' (140518 xuezhe he huangshulang-zh, 19)
\end{exe}

The apparently puzzling optional use of the ergative in (\ref{ex:tAse.kW.pjAkAmarci}) is in fact the same as that observed with the partitive argument of the verb \japhug{mtsʰɤt}{be full} (§\ref{sec:kW.mtshAt}), and has to be distinguished from either agent or instrument.

The agent can at best be expressed as a possessive prefix on the intransitive subject of the passive verb, as in (\ref{ex:Wjandza}), though in this example the implication that the agent is \textsc{2sg} is only contextual.

\begin{exe}
\ex \label{ex:Wjandza}
 \gll nɤ-smɤn ɯ-j-á-ndza? \\
 \textsc{2sg}.\textsc{poss}-medicine \textsc{qu}-\textsc{peg}-\textsc{pass}-eat:\textsc{fact} \\
 \glt `Has your medicine been eaten?' (elicited)
\end{exe}

The only case of a passive verb taking as argument an overt noun phrase corresponding to the transitive subject of the base verb is \japhug{afskɤr}{be surrounded} from \japhug{fskɤr}{go around}. As shown in (\ref{ex:jAGAt.kofskAr}) and (\ref{ex:tWji.afskAr}), an absolutive locative noun phrase referring to the surrounding entity (in square brackets) can appear with this verb after the intransitive subject (the surrounded entity/location).

\begin{exe}
\ex \label{ex:jAGAt.kofskAr}
 \gll iʑɤra ji-kʰa nɯ [jɤɣɤt tɯ-tɤkʰrɤz nɯ] ku-o-fskɤr. \\
 \textsc{1pl} \textsc{1pl}.\textsc{poss}-house \textsc{dem} balcony one-row \textsc{dem} \textsc{ipfv}:\textsc{east}-\textsc{pass}-surround \\
\glt `Our house is surrounded by one row of balcony.' (2011-11-kha2, 4)
\end{exe}

\begin{exe}
\ex \label{ex:tWji.afskAr}
 \gll a-kʰa ɯ-rkɯ nɯnɯ [tɯ-ji] a-fskɤr.\\
\textsc{1sg}.\textsc{poss}-house \textsc{3sg}.\textsc{poss}-side \textsc{dem} \textsc{indef}.\textsc{poss}-field \textsc{pass}-surround:\textsc{fact}\\
\glt `(The sides of) my house are surrounded by fields.' (elicited)
\end{exe}

However, the base verb  \japhug{fskɤr}{go around} in the meaning `surround' can be constructed with the surrounding entity as its transitive subject (in \ref{ex:praR.kW.kufskAr}, \japhug{praʁ}{cliff}). This ergatively-marked subject however is not a volitional agent and rather semantically corresponds to a locative argument, which may explain why it is not removed by passivization (though ergative marking is lost).

\begin{exe}
\ex \label{ex:praR.kW.kufskAr}
 \gll kɯki sɤtɕʰa ki, ɯ-rkɯ tʰamtɕɤt [praʁ] kɯ ku-fskɤr ɲɯ-ɕti \\
 \textsc{dem}.\textsc{prox} place \textsc{dem}.\textsc{prox} \textsc{3sg}.\textsc{poss}-side all cliff \textsc{erg} \textsc{ipfv}:\textsc{east}-surround \textsc{sens}-be.\textsc{aff} \\
 \glt `(The sides of) this place are all surrounded by cliffs.' (elicited)
 \end{exe}
 
\subsection{Passive from ditransitive verbs} \label{passive.ditransitive}
Secundative verbs such as \japhug{mbi}{give} and \japhug{jtsʰi}{give to drink} (a lexicalized causative of \japhug{tsʰi}{drink}), which select as object the recipient rather than the theme (§\ref{sec:ditransitive.secundative}), can be subjected to the passive derivation. The resulting verbs  \japhug{ambi}{be given} and  \japhug{ajtsʰi}{be given to drink}  however encode as subjects not the recipient, but the theme; in particular, they never take indexation affixes even when the recipient is a first or second person. The recipient can only be encoded as possessor of the subject (\ref{ex:WsmAn.ambi}) or as left-dislocated focus constituent (\ref{ex:tAlu.ajtshi}).

\begin{exe}
\ex \label{ex:WsmAn.ambi}
 \gll tɤ-pɤtso ɯ-smɤn a-mbi  \\
  \textsc{indef}.\textsc{poss}-child \textsc{3sg}.\textsc{poss}-medicine \textsc{pass}-give:\textsc{fact} \\
  \glt `The child has been given his medicine.'  (elicited)
\end{exe} 

\begin{exe}
\ex \label{ex:tAlu.ajtshi}
 \gll tɤ-pɤtso ra tɤ-lu a-jtsʰi \\
\textsc{indef}.\textsc{poss}-child \textsc{pl} \textsc{indef}.\textsc{poss}-milk \textsc{pass}-give.to.drink:\textsc{fact} \\
\glt `The children have been given milk to drink.' (elicited)
\end{exe} 
 
 The causativization of these passive verbs yields the rogative derivation (§\ref{sec:rogative.derivation}).
 
\subsection{Reduplicated passive} \label{passive.redp}
The intransitive verb \japhug{aβzdoʁβzdɯ}{be in good order} derives from the transitive verb \japhug{βzdɯ}{collect} (itself borrowed from \tibet{བསྡུ་}{bsdu}{collect}), and can be interpreted as a lexicalized passive derivation (`have been collected and put in order' \fl{} `be in good order'). However, unlike other passive verbs, it presents reduplication with the rare \forme{-oʁ} replicant (on this type of reduplication, see also §\ref{sec.distributed.action.oR}).

The verb \japhug{aβdoʁβdi}{be in good health} (\ref{ex:koBdoRBdij}) apparently has the same morphological structure as \japhug{aβzdoʁβzdɯ}{be in good order}, but since its base verb \japhug{βdi}{be well} (from \tibet{བདེ་}{bde}{be well}) is stative intransitive, this example cannot be analyzed as a passive.

\begin{exe}
\ex \label{ex:koBdoRBdij}
\gll iʑora ɕɤxɕo ku-oβdoʁβdi-j \\
\textsc{1pl} these.days \textsc{prs}-be.in.good.health-\textsc{1pl} \\
\glt `These days we are in good health.' (elicited)
\end{exe} 

Another interesting difference between \japhug{aβzdoʁβzdɯ}{be in good order}  and \japhug{aβdoʁβdi}{be in good health} is that the former has a sigmatic causative \japhug{sɤβzdoʁβzdɯ}{put in order}, while the latter has a velar causative  \japhug{ɣɤβdoʁβdi}{tidy up, put in order} whose meaning is not derivable from that of \japhug{aβdoʁβdi}{be in good health}.

\subsection{Compatibility with other derivations} \label{sec:passive.other.derivations}
There are no examples in the corpus of passive derivations taking as input causative or applicative verbs, except for some lexicalized causatives such as \japhug{jtsʰi}{give to drink} (§\ref{passive.ditransitive}). However, it is likely that the progressive prefix \forme{asɯ-} originates from the combination of the passive and the sigmatic causative (§\ref{sec:sig.caus.other.derivations}, §\ref{sec:progressive.history}).

Passive verbs however can be subjected to sigmatic causative derivation. The fusion of the sigmatic causative \forme{sɯ-} with the passive \forme{a-} yield \ipa{sɤ}, a surface form identical to the antipassive (§\ref{sec:antipassive.sA}) and the proprietive (§\ref{sec:proprietive}) prefixes.
 
First, the causative-passive double derivations from secundative verbs (§\ref{passive.ditransitive}) has become the rogative derivation (§\ref{sec:rogative.derivation}).  Second, the lexicalized passives \japhug{apa}{become} and \japhug{aβzu}{become, grow} (§\ref{sec:passive.lexicalized}) have the causative forms \forme{sɤpa} and \forme{sɤβzu}, both of which predictably mean `cause to become, turn into, transform into'. These causative verbs are ditransitive, as in (\ref{ex:sApaj}), taking as semi-object the entity in which the object is transformed.

\begin{exe}
\ex \label{ex:sApaj}
\gll tɯ-ci tʰamtɕɤt tɤ-se sɯ-ɤpa-j \\
\textsc{indef}.\textsc{poss}-water all \textsc{indef}.\textsc{poss}-blood \textsc{caus}-become:\textsc{fact}-\textsc{1sg} \\
\glt `Let us turn all the water into blood.' (2003smanmi, 91)
\end{exe} 

The object indexed on the verb is the entity that undergoes transformation, such as the \textsc{1pl} in (\ref{ex:nWwGsABzuj}), not the one that it is transformed into. Only the object is targeted by reflexivization (§\ref{sec:refl.caus}). 

\begin{exe}
\ex \label{ex:nWwGsABzuj}
\gll kɯki ji-βdaʁmu kɯ ki kɯ-fse, nɤki, sɯŋgɯ pɣa nɯ́-wɣ-sɯ-ɤβzu-j \\
\textsc{dem}.\textsc{prox} \textsc{1pl}.\textsc{poss}-queen \textsc{erg} \textsc{dem}.\textsc{prox} \textsc{sbj}:\textsc{pcp}-be.like \textsc{filler} forest bird \textsc{aor}-\textsc{inv}-\textsc{caus}-become-\textsc{1pl} \\
\glt `Our queen has transformed us into bird from the forest.' (140520 ye
tiane-zh, 127)
\end{exe} 

The verbs \forme{sɤpa} and \forme{sɤβzu} `cause to become, turn into, transform' occur with a variety of semi-objects. In (\ref{ex:tWmpCar.YWwGGsABzu}), the semi-object \forme{tɯ-mpɕar tɯ-mpɕar} is a repeated counted noun (§\ref{sec:CN.repetition}) expressing literally the meaning `turn (the rock) into sheets one by one'. 

\begin{exe}
\ex \label{ex:tWmpCar.YWwGGsABzu}
\gll cupa kɤ-tɕɤt tɕe ɲɯ-ɴqa ma praʁ ɕ-pjɯ́-wɣ-pʰɯt ɲɯ-ra […]  tɕe nɯ tɯ-mpɕar tɯ-mpɕar ɲɯ́-wɣ-sɯ-ɤβzu ɲɯ-ra \\
stone.slab \textsc{inf}-take.out \textsc{lnk} \textsc{sens}-be.hard \textsc{lnk} cliff \textsc{tral}-\textsc{ipfv}-\textsc{inv}-take.off \textsc{sens}-be.needed { } 
\textsc{lnk} \textsc{dem} one-sheet one-sheet \textsc{ipfv}-\textsc{inv}-\textsc{caus}-become \textsc{sens}-be.needed \\
\glt `To extract stone slabs, one has to chop the rocks sheet by sheet (layer by layer).’ (14-siblings, 173-175)
\end{exe} 

The semi-object of  \forme{sɤpa} and \forme{sɤβzu} `cause to become' can also be an infinitival or participial clause. This combination is a common type of periphrastic causative construction (§\ref{sec:sWpa.sABzu}), illustrated in example (\ref{ex:kWmpCWmpCAr.nasABzu}) with the two relative clauses \forme{ɯ-ɕɣa kɯ-xtɕɯ\redp{}xtɕi} `who is young, whose age is small' and \forme{kɯ-mpɕɯ\redp{}mpɕɤr} `who is beautiful' as semi objects. A more literal translation could be `he turned his mother into (someone) whose age was small and who was beautiful'.
 
\begin{exe}
\ex \label{ex:kWmpCWmpCAr.nasABzu}
\gll ɯ-mu [ɯ-ɕɣa kɯ-xtɕɯ\redp{}xtɕi] [kɯ-mpɕɯ\redp{}mpɕɤr] ʑo na-sɯ-ɤβzu ɲɯ-ŋu. \\
\textsc{3sg}.\textsc{poss}-mother \textsc{3sg}.\textsc{poss}-age \textsc{sbj}:\textsc{pcp}-\textsc{emph}\redp{}be.small \textsc{sbj}:\textsc{pcp}-\textsc{emph}\redp{}be.beautiful \textsc{emph} \textsc{aor}:3\flobv{}-\textsc{caus}-become \textsc{sens}-be \\
\glt `He made his mother become young and beautiful.' (Norbzang 2005, 253)
\end{exe} 
 
This periphrastic construction is also possible with the reflexivized forms of \japhug{sɤpa}{transform} and \japhug{sɤβzu}{transform}   (example \ref{ex:kWZo.YAZGAsApa}, §\ref{sec:refl.caus})

The verb \japhug{sɤrtsi}{consider as}  (see \ref{ex:tutanWsArtsi}, §\ref{sec:essive.abs}) is the sigmatic causative of \japhug{artsi}{count as} (\ref{ex:mWYArtsi2}), itself passive of the transitive \japhug{rtsi}{count}  (from \tibet{རྩི་}{rtsi}{count}).

 \begin{exe}
\ex \label{ex:mWYArtsi2}
\gll xɕiri kɯ-fse nɯra kɯrŋi mɯ-ɲɯ-ɤ-rtsi lo. \\
 weasel \textsc{sbj}:\textsc{pcp}-be.like \textsc{dem}:\textsc{pl}  beast.of.prey \textsc{neg}-\textsc{sens}-\textsc{pass}-count \textsc{sfp} \\
\glt `Weasel and the like do not count as beasts of prey (I think).' 
\end{exe} 

Passive verbs can occur as second element of Noun-Verb nominal compounds, but in this case the \forme{a-} prefix is absorbed by the previous nominal root, resulting in a surface form undistinguishable from that of the base verb. For instance, the compound \japhug{ɕnɤsti}{person with a stuffy nose} can be parsed as \forme{ɕnɤ-} (\textit{status constructus} of  \japhug{tɯ-ɕna}{nose}) and \forme{-sti}, a syllable ambiguous between the passive \japhug{asti}{be blocked} and the base verb  \japhug{sti}{block} (\ref{ex:pjAkAstici}, §\ref{sec:object.verb.compounds}). . The former is more probable, as this noun should be analyzed as meaning `(person) whose nose is blocked rather than as `blocking noses'.
 
\section{Rogative} \label{sec:rogative.derivation}
The rogative\footnote{This term is from Latin \textit{rogo} `I ask, I require'. } \forme{sɤ-} prefix is attested on the secundative verbs \japhug{mbi}{give} and \japhug{jtsʰi}{give to drink}, and produces intransitive verbs meaning `ask (someone) to $X$ (something) to oneself’ (where $X$ stands for the meaning of the base verb), \japhug{sɤmbi}{ask for} and  \japhug{sɤjtsʰi}{ask for something to drink}, respectively.

The \forme{sɤ-} prefix is homophonous to that of the antipassive (§\ref{sec:antipassive.sA})\footnote{The rogative prefix on \japhug{sɤmbi}{ask for} was mistakenly glossed as `antipassive’ in \citet[215]{jacques12demotion}, example (32). The  antipassive forms of \japhug{mbi}{give} and  \japhug{jtsʰi}{give to drink} are \japhug{rɤmbi}{give to someone} and \japhug{rɤjtsʰi}{give to someone to drink}, respectively. These forms have the \forme{rɤ-} antipassive prefix instead of \forme{sɤ-} (§\ref{sec:antipassive.ditransitive}). } and of the proprietive (§\ref{sec:proprietive}) derivations.

Rogative verbs are morphologically intransitive, as shown by (\ref{ex:nWsAmbia}) (the past transitive suffix \forme{-t} would be expected here if this verb were transitive, see §\ref{sec:transitivity.morphology}) and (\ref{ex:nWsAmbinW}) (here a type-C orientation prefix would have be expected §\ref{sec:indexation.non.local}, §\ref{sec:kamnyu.preverbs}). 

However, while the rogative derivation affects transitivity, it does not decrease valency: \forme{sɤmbi} and  \forme{sɤjtsʰi} are trivalent like their base verb: they have two arguments in addition to the intransitive subject, a semi-object (the entity that is asked for) and a dative argument (§\ref{sec:semi.transitive.dative}).

\begin{exe}
\ex \label{ex:nWsAmbia}
\gll aʑo a-pi ɯ-ɕki kɯmtɕʰɯ ci nɯ-sɤ-mbi-a \\
\textsc{1sg} \textsc{1sg}.\textsc{poss}-elder.sibling \textsc{3sg}.\textsc{poss}-\textsc{dat} toy \textsc{indef} \textsc{aor}-\textsc{rog}-give-\textsc{1sg} \\
\glt `I asked my elder brother/sister to give me a toy.' (elicited)
\end{exe}

The semi-object of \japhug{sɤmbi}{ask for} can be a human (including a first or second person), in contexts like asking for someone in the context of arranged marriage (\ref{ex:nWsAmbinW}) or asking for a child to become recognized as a reincarnated lama.

\begin{exe}
\ex \label{ex:nWsAmbinW}
\gll tɕe a-ʁi nɯre ri, nɯ-sɤ-mbi-nɯ \\
\textsc{lnk} \textsc{1sg}.\textsc{poss}-younger.sibling \textsc{dem}:\textsc{loc} \textsc{loc} \textsc{aor}-\textsc{rog}-give-\textsc{pl} \\
\glt `They asked for my younger brother (to come to marry their daughter) there.' (14-siblings, 210)
\end{exe}

The dative argument is optional, and rarely expressed. Rogative verbs with non-overt dative can be used as a polite way to ask for something, as in  (\ref{ex:kAsAjtshi.WBrAtu}), where neither the person asking (the child) nor the addressee (the grandfather, that would take the dative) are expressed within the same clause.

\begin{exe}
\ex \label{ex:kAsAjtshi.WBrAtu}
\gll  a-wɯ, tɯ-mgo kɤ-sɤ-mbi, tɯ-ci kɤ-sɤ-jtshi ɯβrɤ-tu \\
\textsc{1sg}.\textsc{poss}-grandfather \textsc{indef}.\textsc{poss}-food \textsc{obj}:\textsc{pcp}-\textsc{rog}-give \textsc{indef}.\textsc{poss}-water \textsc{obj}:\textsc{pcp}-\textsc{rog}-give.to.drink opt-exist:\textsc{fact} \\
\glt `Grandfather, would there be food or water that (we could) ask for (you to give us)?' (2011-05-nyima, 81)
\end{exe}

The rogative \forme{sɤ-} is likely to have originated from the combination of the sigmatic causative  \forme{sɯ-} and passive  \forme{a-}  prefixes (§\ref{sec:passive.other.derivations}). The verbs \japhug{mbi}{give} and \japhug{jtsʰi}{give to drink}  have the passive forms \japhug{ambi}{be given} and  \japhug{ajtsʰi}{be given to drink} (§\ref{passive.ditransitive}), which select as intransitive subject the theme. A causative derivation from these passive verbs `cause $X$ to be given' could yield the meaning of the rogative derivation (note the semantic similarity with the inversive and rogative functions of the causative discussed in §\ref{sec:sig.caus.inversive} and §\ref{sec:sig.caus.rogative}) . However, this analysis is not completely straightforward, as one would expect causative verbs to be transitive (not semi-transitive like the rogative verbs) and the dative marking of the addressee is also unexplained.


\section{Reflexive} \label{sec:reflexive}
While Japhug has a generic pronoun \japhug{tɯʑo}{one} (§\ref{sec:genr.pro}) and an emphatic pronoun \japhug{raŋ}{oneself} (§\ref{sec:pronouns.emph}), it lacks any reflexive pronoun, and the only way to express reflexive meaning is by means of a dedicated prefix \forme{ʑɣɤ\trt}, different from the autive (§\ref{sec:autobenefactive}), the reciprocal (§\ref{sec:reciprocal}) and other valency-decreasing derivations.

\subsection{The reflexive prefix: form and basic function}
\tabref{tab:ZGA.refl} presents several examples of the reflexive prefix \forme{ʑɣɤ\trt},  deriving reflexive verbs from transitive ones.

The reflexive derivation turns the base verb into an intransitive verb whose subject acts upon itself, and corresponds to both the transitive subject and the object of the base verb. For instance, in (\ref{ex:CpjWZGAGAlanW.kWZGAnWsmAn}), the subjects of the reflexive verbs \forme{ɕ-pjɯ-ʑɣɤ-ɣɤ-la-nɯ} `they bathe in it=they cause themselves to soak in it' and \forme{kɯ-ʑɣɤ-nɯsmɤn} to heal themselves' are semantically both agents and patients.  When the subject is non-singular, each member of the group corresponding to the subject is acting on himself, as opposed to the reciprocal derivation, where every member performs the same action to other members of the groups, and is agent and patient either simultaneously or in turn (§\ref{sec:reciprocal}).

\begin{exe}
\ex \label{ex:CpjWZGAGAlanW.kWZGAnWsmAn}
 \gll   %tɕʰismɤn nɯnɯ iɕqʰa tɯ-ci kɯ-sɤ-ɕke. ... 
  tɕe nɯ ɯ-ŋgɯ ri ɕ-pjɯ-ʑɣɤ-ɣɤ-la-nɯ tɕe li kɯ-ʑɣɤ-nɯsmɤn ju-ɕe-nɯ tɕe, \\
%warm.springs \textsc{dem} filler \textsc{indef}.\textsc{poss}-water \textsc{sbj}:\textsc{pcp}-\textsc{prop}-burn { } 
\textsc{lnk} \textsc{dem} \textsc{3sg}.\textsc{poss}-in \textsc{loc} \textsc{tral}-\textsc{ipfv}-\textsc{refl}-\textsc{caus}-soak-\textsc{pl} \textsc{lnk} again \textsc{sbj}:\textsc{pcp}-\textsc{refl}-heal \textsc{ipfv}-go-\textsc{pl} \textsc{lnk} \\
\glt `People go and bathe in (warm springs), they go (there) to heal themselves.' (20-ldWGi, 52;56)
 \end{exe}
  
\begin{table}
\caption{Examples of reflexive verbs in Japhug} \label{tab:ZGA.refl}
\begin{tabular}{lllllll}
\lsptoprule
Base verb & Reflexive verb \\
\midrule
\japhug{χtɕi}{wash} & \japhug{ʑɣɤχtɕi}{wash oneself} \\
\japhug{tsʰi}{strangle} & \japhug{ʑɣɤtsʰi}{hang oneself} \\
\japhug{sat}{kill} & \japhug{ʑɣɤsat}{commit suicide} \\ 
\japhug{rku}{put in} & \japhug{ʑɣɤrku}{put oneself in} \\
\japhug{nɤstu}{believe in} & \japhug{ʑɣɤnɤstu}{believe in oneself} \\
\japhug{fstɯn}{take care of} & \japhug{ʑɣɤfstɯn}{take care of oneself} \\
\lspbottomrule
\end{tabular}
\end{table}
 
%ʑɣɤfstɤt
The prefix \forme{ʑɣɤ-} is highly productive, but only targets objects.\footnote{The only exceptions to this rule are a handful of intransitive verbs which take the \forme{ʑɣɤ-} prefix (§\ref{sec:refl.intr}). } Other syntactic functions such as possessors of objects for instance cannot be reflexivized using \forme{ʑɣɤ-}. The form  $\dagger$\forme{ʑɣɤ-sɤɕɤt} (intended meaning `comb oneself')  found in one example in the corpus (\ref{ex:pjAZGAsACAtndZi}) is a mistake, as pointed out by the consultant who told the story (Tshendzin) herself, since \japhug{sɤɕɤt}{comb} can only take the noun \japhug{tɯ-ku}{head} as object, not the person whose hair is combed. The only way to express the meaning `comb oneself' is by the autive prefix (§\ref{sec:autoben.proper}) as in (\ref{ex:pjAnWsACAtndZi}).

\begin{exe}
\ex 
\begin{xlist}
\ex \label{ex:pjAZGAsACAtndZi}
\gll $\dagger$pjɤ-ʑɣɤ-sɤɕɤt-ndʑi to-rɤmpɕoʁ\redp{}mpɕɤr-ndʑi, \\
  \textsc{ifr}-\textsc{refl}-comb-\textsc{du} \textsc{ifr}-\textsc{emph}\redp{}make.up-\textsc{du} \\
  \glt Intended meaning: `They combed themselves and made themselves beautiful.' (150830 baihe jiemei-zh, 85)
  \ex \label{ex:pjAnWsACAtndZi}
\gll ndʑi-ku pjɤ-nɯ-sɤɕɤt-ndʑi  to-rɤmpɕoʁ\redp{}mpɕɤr-ndʑi,  \\
 \textsc{3du}.\textsc{poss}-head \textsc{ifr}-\textsc{auto}-comb-\textsc{du} \textsc{ifr}-\textsc{emph}\redp{}make.up-\textsc{du} \\
\glt `They combed their hair and made themselves beautiful.' (correction of the previous example) 
\end{xlist}
\end{exe}
 
 Similarly, in the reflexive causative derivation \japhug{ʑɣɤsɯntɕʰɤr}{cause oneself to appear} from \japhug{ntɕʰɤr}{appear} (§\ref{sec:refl.caus}), reflexivization by \textit{ʑɣɤ-} only targets the intransitive subject of the base verb (the entity which appears in the dream, \japhug{sɯŋgi}{lion} in \ref{ex:sWNgi.kontChAr}), not the experiencer, which is encoded as possessor of the noun \japhug{tɯ-jmŋo}{dream}. Thus, example (\ref{ex:koZGAsWntChAr}) cannot be interpreted as `he caused her to appear to himself'.
 
\begin{exe}
\ex 
\begin{xlist}
\ex \label{ex:sWNgi.kontChAr}
\gll ɯ-jmŋo ɯ-ŋgɯ sɯŋgi ko-ntɕʰɤr \\
\textsc{3sg}.\textsc{poss}-dream \textsc{3sg}.\textsc{poss}-in lion \textsc{ifr}-appear \\
\glt `He dreamt of a tiger / A tiger appeared to him in a dream.' (elicited)
\ex \label{ex:koZGAsWntChAr}
\gll  ɯ-jmŋo ɯ-ŋgɯ ko-ʑɣɤ-sɯ-ntɕʰɤr    \\
\textsc{3sg}.\textsc{poss}-dream \textsc{3sg}.\textsc{poss}-in \textsc{ifr}-\textsc{refl}-\textsc{caus}-appear  \\
\glt `He made himself appear to her in a dream.' (2012 Norbzang, 116)
\end{xlist}
 \end{exe}
 
  \subsubsection{Transitivity} \label{sec:refl.erg}
All reflexive verbs are morphologically intransitive (§\ref{sec:transitivity.morphology}). In terms of case marking however, they are not prototypical intransitive verbs. When their subject is overt, it is in absolutive form in most cases as in (\ref{ex:kunWZGAfstWn}), but subjects of reflexive verbs in the ergative (§\ref{sec:S.kW}) are also attested (\ref{ex:tWZo.kW.tWZo}, §\ref{sec:genr.pro}).

\begin{exe}
\ex \label{ex:kunWZGAfstWn}
 \gll ɯʑo ku-nɯ-ʑɣɤ-fstɯn cʰa \\
 \textsc{3sg} \textsc{ipfv}-\textsc{auto}-\textsc{refl}-serve can:\textsc{fact} \\
 \glt `It (will) be able to take care of itself on its own.' (150822 laoye zuoshi zongshi duide-zh, 149)
 \end{exe}

In particular, the semi-transitive lexicalized reflexive \japhug{ʑɣɤpa}{pretend} (derived from \japhug{pa}{do}, §\ref{sec:lexicalized.refl}) often occurs with subjects marked with the ergative, such as the right-dislocated constituent \forme{<tangseng> nɯ kɯ} in (\ref{ex:mAkWtso.toZGApa})  (see also example \ref{ex:kW.toZGApa}, §\ref{sec:semi.transitive}).
 
 \begin{exe}
\ex \label{ex:mAkWtso.toZGApa}
 \gll  mɤ-kɯ-tso to-ʑɣɤpa, <tangseng> nɯ kɯ. \\
 \textsc{neg}-\textsc{sbj}:\textsc{pcp}-understand \textsc{ifr}-pretend  \textsc{anthr} \textsc{dem} \textsc{erg} \\
 \glt  `He pretended not to have understood, Tangseng.' (180503 xiyouji 12-zh, 46)
 \end{exe}
% 
%  \end{exe}
% \begin{exe}
%\ex \label{ex:mWpWkWsWXsAl.toZGApa}
% \gll kɯ-lɤɣ tɤ-pɤtso nɯ kɯ nɯ kóʁmɯz nɯ mɤɕtʂa mɯ-pɯ-kɯ-sɯχsɤl to-ʑɣɤpa tɕe kɯ-nɤmtsʰɤr to-ʑɣɤpa \\
% \textsc{sbj}:\textsc{pcp}-graze \textsc{indef}.\textsc{poss}-child \textsc{dem} \textsc{erg} \textsc{dem} only.after \textsc{dem} until \textsc{neg}-\textsc{aor}-\textsc{sbj}:\textsc{pcp}-realize \textsc{ifr}-pretend \textsc{lnk} \textsc{sbj}:\textsc{pcp}-be.surprised.by  \textsc{ifr}-pretend \\
% \glt `The young shepherd pretended not to have noticed (the presence of the noble) until just (that time) and pretended to be surprized by it.' (140513 mutong de disheng-zh, 123)
% \end{exe}
 
  \subsubsection{Reflexivization from ditransitive verbs} \label{sec:refl.ditransitive}
The reflexive is possible with indirective verbs of giving or speech, whose objects corresponds to the theme (§\ref{sec:ditransitive.indirective}): \japhug{kʰo}{give} and \japhug{fɕɤt}{tell} yield the reflexive verbs \japhug{ʑɣɤkʰo}{give oneself up} and \japhug{ʑɣɤfɕɤt}{tell about oneself}, as in (\ref{ex:YWtWnWZGAkho}). 

\begin{exe}
\ex \label{ex:YWtWnWZGAkho}
 \gll nɤʑo ʁgra ɯ-jaʁ nɯtɕu ɲɯ-tɯ-nɯ-ʑɣɤ-kʰo ʑo ɯmɤ-kɯ-ɕti-ci? \\
 \textsc{2sg} enemy \textsc{3sg}.\textsc{poss}-hand \textsc{dem}:\textsc{loc} \textsc{ipfv}-2-\textsc{auto}-\textsc{refl}-give \textsc{emph} \textsc{prob}-\textsc{peg}-be.\textsc{aff}-\textsc{peg} \\
 \glt `(By doing that), aren't you handing yourself over to your own enemy?' (2014niulan li de lu-zh, 11)
 \end{exe}
 
The \forme{ʑɣɤ-} prefix can only express reflexivization of the theme of indirective verbs, never the dative-marked recipient; for instance, it cannot be prefixed to an indirective verb such as \japhug{tʰu}{ask} (§\ref{sec:ditransitive.indirective}) to convey the meaning `ask oneself'. With the secundative verb \japhug{mbi}{give}, the reflexive prefix is not possible either, though for pragmatic rather than syntactic reasons. 
%ʑɣɤrpu
 %ʑɣɤntsɣe
   \subsubsection{Permissive function} \label{sec:refl.permissive}
The reflexive prefix can in some cases have the permissive function  `let oneself be $X$', the most common case with the verb \japhug{ʑɣɤnɯβlu}{be fooled} from the transitive denominal verb \japhug{nɯβlu}{cheat} (`let oneself be cheated'). With most transitive verbs, this meaning is expressed by means of the reflexive+causative double derivation (§\ref{sec:refl.caus}).

In (\ref{ex:toZGAnWmbrApW}), an overt agent (corresponding to the causee of the causative construction) marked with the ergative \forme{kɯ} appears with the verb \japhug{ʑɣɤnɯmbrɤpɯ}{let oneself be mounted}. Although taken from a text translated from Chinese, it is not considered incorrect after rechecking with Tshendzin. 

 \begin{exe}
\ex \label{ex:toZGAnWmbrApW}
 \gll  tɕendɤre mbro kɯ ``ɲɯ-pe" to-ti ɲɯ-ŋu. tɕendɤre tɯrme nɯ kɯ to-ʑɣɤ-nɯmbrɤpɯ ɲɯ-ŋu tɕe tɕe tɯrme nɯ mbro ɯ-taʁ nɯtɕu to-ɕe ɲɯ-ŋu tɕe, \\
 \textsc{lnk} horse \textsc{erg} \textsc{sens}-be.good \textsc{ifr}-say \textsc{sens}-be \textsc{lnk}  man \textsc{dem} \textsc{erg} \textsc{ifr}-\textsc{refl}-ride \textsc{sens}-be \textsc{lnk} \textsc{lnk} man \textsc{dem} horse \textsc{3sg}.\textsc{poss}-on \textsc{dem}:\textsc{loc} \textsc{ifr}:\textsc{up}-go \textsc{sens}-be \textsc{lnk} \\
 \glt  `The horse said `good' and let himself be mounted by the man, and the man rode on him.'  (ma he lu-zh, 221-23)
 \end{exe}
 %ʑɣɤfsraŋ


\subsubsection{Reflexivization of intransitive verbs} \label{sec:refl.intr}
A handful of intransitive verbs can undergo reflexivization with the \forme{ʑɣɤ-} prefix.

With intransitive deadverbial (§\ref{sec:verbs.relative.location}) and denominal verbs of relative locations in \forme{mɤ\trt}, such as \japhug{maŋlo}{be upstream} and \japhug{mɤpɤrtʰɤβ}{be in the middle}, the reflexive prefix derives  volitional motion verbs such as \japhug{ʑɣɤmaŋlo}{put oneself upstream} (\ref{ex:loZGAmaNlo}) and \japhug{ʑɣɤmɤpɤrtʰɤβ}{put oneself in between}. It is possible that these forms come from the simplification of former reflexive+causative (§\ref{sec:refl.caus}) by loss of the causative prefix.

\begin{exe}
\ex \label{ex:loZGAmaNlo}
\gll  ɯ-lɤcu nɯtɕu jo-ɕe tɕe lo-ʑɣɤ-maŋlo. \\
\textsc{3sg}.\textsc{poss}-upstream \textsc{dem}:\textsc{loc} \textsc{ifr}-go \textsc{lnk} \textsc{ifr}:\textsc{upstream}-\textsc{refl}-be.upstream \\
\glt `(The wolf) went to a place upstream (from the lamb), he placed himself upstream.' (2014 lang he yang, 8-9)
\end{exe}

The pair of intransitive verbs \japhug{ɣɤtɕa}{be wrong} and \japhug{ɣɤŋgi}{be right} have the reflexive forms \japhug{ʑɣɤɣɤtɕa}{recognize one's mistake} and \japhug{ʑɣɤɣɤŋgi}{consider oneself to be right}, as shown by (\ref{ex:pjWkWZGAGAtCa}). Note that the causative of these verbs has a tropative meaning, for  instance the causative \japhug{zɣɤtɕa}{consider to be wrong} in the second clause of (\ref{ex:pjWkWZGAGAtCa}); the reflexive forms thus have the expected meaning of reflexive+causative double derivations (`consider oneself to be right/wrong').

\begin{exe}
\ex \label{ex:pjWkWZGAGAtCa}
\gll tɯʑo pjɯ-kɯ-ʑɣɤ-ɣɤtɕa ra ma, tɯ-zda pjɯ́-wɣ-z-ɣɤtɕa, tɯʑo %ntsɯ pjɯ-kɯ-ʑɣɤ-ɣɤŋgi tɕe, 
(...) pɯ-kɯ-nɯ-ɣɤtɕa kɯnɤ pjɯ-kɯ-ʑɣɤ-ɣɤŋgi tɕe,  ɯ-mbrɤzɯ kɯ-tu me tu-kɯ-ti ɲɯ-ŋu. \\
\textsc{genr} \textsc{ipfv}-\textsc{genr}:S/O-\textsc{refl}-be.wrong be.needed:\textsc{fact} \textsc{lnk} \textsc{genr}.\textsc{poss}-companion \textsc{ipfv}-\textsc{inv}-\textsc{caus}-be.wrong \textsc{genr} {  } \textsc{ipfv}-\textsc{genr}:S/O-\textsc{refl}-be.wrong also  \textsc{ipfv}-\textsc{genr}:S/O-\textsc{auto}-be.right  \textsc{lnk} \textsc{3sg}.\textsc{poss}-result \textsc{sbj}:\textsc{pcp}-exist not.exist:\textsc{fact} \textsc{ipfv}-\textsc{genr}:S/O-say \textsc{sens}-be \\ 
\glt `One has to recognize one's mistakes; if one considers one's companions to be wrong, and if one consider oneself to be always right even if one is wrong, there will be no (good) result.' (lWlu, 80-82)
\end{exe}

The reflexive verb \japhug{ʑɣɤsɤɕke}{burn oneself} appears to derive from the proprietive verb \japhug{sɤɕke}{be burning} (from \japhug{ɕke}{burn}, §\ref{sec:proprietive}), though from its meaning it is tempting to wonder whether this prefix \forme{sɤ-} here is not rather analyzable as an irregular causative (§\ref{sec:sig.caus.irregular}).

\begin{exe}
\ex \label{ex:mapWtWZGAsACke}
\gll ma-pɯ-tɯ-ʑɣɤ-sɤ-ɕke \\
\textsc{neg}-\textsc{imp}-2-\textsc{refl}-\textsc{prop}?-burn \\
\glt `Don't burn yourself!' (elicited)
\end{exe}
  
One reflexive verb seems to derived from an ideophone rather than from a verb root: \japhug{ʑɣɤɕpʰɤβ}{lay flat} (\ref{ex:pWZGACphaBa})\footnote{The vowel alternation in (\ref{ex:pWZGACphaBa}) is regular, see §\ref{sec:intr.1}. } shares the same root \idroot{ɕpʰɤβ} as \japhug{ɕpʰɤβɕpʰɤβ}{laying flat} (\ref{ex:CphABCphAB.Zo.YWrAZi}). It is possible that this verb originates from a deideophonic derivation (§\ref{sec:voice.deideophonic}), with subsequent deletion of the denominal prefix.

\begin{exe}
\ex \label{ex:pWZGACphaBa}
\gll  ɯ-tʰoʁ zɯ pɯ-ʑɣɤɕpʰaβ-a \\
\textsc{3sg}.\textsc{poss}-ground \textsc{loc} \textsc{aor}-lay.flat-\textsc{1sg} \\
\glt `I laid flat on the ground.' (elicited)
\end{exe}

\begin{exe}
\ex \label{ex:CphABCphAB.Zo.YWrAZi}
\gll   ɯ-tʰoʁ ɕpʰɤβɕpʰɤβ ʑo ɲɯ-rɤʑi, kɤ-nɯqambɯmbjom mɯ́j-cʰa \\
\textsc{3sg}.\textsc{poss}-ground \textsc{idph}(II):laying.flat \textsc{emph} \textsc{sens}-stay \textsc{inf}-fly \textsc{neg}:\textsc{sens}-can \\
\glt `The bird is staying on the ground without moving, it cannot fly.' (elicited)
\end{exe}
 
  \subsubsection{Other reflexive constructions} \label{sec:refl.other}
The reflexive \forme{ʑɣɤ-} is almost the only way to express reflexivity in Japhug. However, the antipassive verb \forme{ra-χtɕi} (from the transitive verb \japhug{χtɕi}{wash}), has the meaning `wash one's face' or `have a shower' (§\ref{sec:antipassive.reflexive}), almost like that of the regular reflexive \japhug{ʑɣɤχtɕi}{wash oneself}. Similarly, the prefix \forme{rɤ-} in  \japhug{rɤmpɕɤr}{make up} from \japhug{mpɕɤr}{be beautiful} is reflexive-like (`make oneself beautiful', see §\ref{sec:rA.non.apass}).
 
 \subsection{Reflexive vs. anticausative}  \label{sec:refl.acaus}
 
Transitive verbs such as \japhug{tʂaβ}{cause to fall/roll}, \japhug{kɤɣ}{bend} or \japhug{xtʰom}{put horizontally} which have a corresponding  prenasalized anticausatives (§\ref{sec:anticausative}) use the reflexive derivation to express the corresponding intransitive volitional action. 
 
The action of the transitive verb itself can be either volitional (\ref{ex:luxthomnW.kuCWrNgWnW})  or non-volitional (including involuntary actions of animate beings and actions of inanimate referents, as in \ref{ex:WthoR.pjAwGtsxaB}) depending on the context.
 
 \begin{exe}
\ex \label{ex:luxthomnW.kuCWrNgWnW}
 \gll  tɤ-pɤtso nɯ ki kɯ-fse lu-xtʰom-nɯ ku-ɕɯ-rŋgɯ-nɯ pɯ-maʁ, \\
 \textsc{indef}.\textsc{poss}-child \textsc{dem} \textsc{dem}.\textsc{prox} \textsc{sbj}:\textsc{pcp}-be.like \textsc{ipfv}:\textsc{upstream}-put.horizontally-\textsc{pl} \textsc{ipfv}-\textsc{caus}-lie.down-\textsc{pl} \textsc{pst}.\textsc{ipfv}-not.be \\
\glt `People would not lie the babies down horizontally like this.' (140426 tApAtso kAnWBdaR1, 57)
\end{exe}

 \begin{exe}
\ex \label{ex:WthoR.pjAwGtsxaB}
 \gll  rdɤstaʁ nɯ ɯ-tɯ-rʑi kɯ ɯ-tʰoʁ pjɤ́-wɣ-tʂaβ tɕe pjɤ́-wɣ-sat. \\
 stone \textsc{dem} \textsc{3sg}.\textsc{poss}-\textsc{nmlz}:\textsc{deg}-be.heavy \textsc{erg} \textsc{3sg}.\textsc{poss}-ground \textsc{ifr}:\textsc{down}-\textsc{inv}-cause.to.fall \textsc{lnk} \textsc{ifr}:\textsc{down}-\textsc{inv}-kill \\
\glt `The stones (in the wolf's belly) were so heavy that they caused him to fall down and die.' (140428 xiaohongmao-zh, 171)
 \end{exe}
 
 The anticausative verbs are nearly always non-volitionally (see §\ref{sec:anticausative.volitionality}), as illustrated by (\ref{ex:rNgW.nWni.pjAndzxaBndZi}) and (\ref{ex:londom}). 
 
 \begin{exe}
\ex \label{ex:rNgW.nWni.pjAndzxaBndZi}
\gll tɕe rŋgɯ nɯni (...) to-k-ɤmɯ-rpu-ndʑi tɕe pjɤ-ɴɢrɯ-ndʑi tɕe tʂɤndo pjɤ-ndʐaβ-ndʑi. \\
\textsc{lnk} boulder \textsc{dem}:\textsc{du} {  } \textsc{ifr}-\textsc{peg}-\textsc{recip}-bump-\textsc{du} \textsc{lnk} \textsc{ifr}:\textsc{down}-\textsc{acaus}:shatter-\textsc{du} \textsc{lnk} side.of.the.road \textsc{ifr}:\textsc{down}-\textsc{acaus}:cause.to.roll-\textsc{du} \\
\glt `The two boulders bumped into each other, shattered, and rolled down the side of the road.' (28-smAnmi, 143)
 \end{exe}

In (\ref{ex:londom}), the verb  \forme{lo-ndom} means `lie down horizontally after falling down (out of exhaustion)', as shown by (\ref{ex:londom.explanation}), the gloss provided in Japhug for this verb form, which also involves the anticausative verb \japhug{ndʐaβ}{fall/roll}.
 
 \begin{exe}
\ex \label{ex:londom}
\gll tɕendɤre ɯ-ʁi nɯ kɤ-ŋke mɯ-ɲɤ-cʰa tɕe tɕendɤre lo-ndom ɕti tɕe \\
\textsc{lnk} \textsc{3sg}.\textsc{poss}-younger.sibling \textsc{dem} \textsc{inf}-walk \textsc{neg}-\textsc{ifr}-can \textsc{lnk} \textsc{lnk} \textsc{ifr}:\textsc{upstream}-\textsc{acaus}:put.horizontally be.\textsc{aff}:\textsc{fact} \textsc{lnk} \\
\glt `His younger brother was not able to walk anymore, and lay down.' (2011-05-nyima, 58-59)
 \end{exe}
 
\begin{exe}
\ex \label{ex:londom.explanation}
\gll ɯ-ku lo lo-ru tɕe pjɤ-ndʐaβ, tɕe kɤ-nɯ-rɤru mɤ-kɯ-cʰa nɯ, tɕe lo-ndom tu-kɯ-ti ɲɯ-ŋu \\
\textsc{3sg}.\textsc{poss}-head upstream \textsc{ifr}:\textsc{upstream}-look \textsc{lnk} \textsc{ifr}:\textsc{down}-\textsc{acaus}:cause.to.fall \textsc{lnk} \textsc{inf}-\textsc{auto}-get.up \textsc{neg}-\textsc{sbj}:\textsc{pcp}-can \textsc{dem} \textsc{lnk}  \textsc{ifr}:\textsc{upstream}-\textsc{acaus}:put.horizontally \textsc{ipfv}-\textsc{genr}:S/O-say \textsc{sens}-be \\
\glt `He fell down, head looking upstream, not able to get up by himself, so one says \forme{lo-ndom}.' (elicited, explanation of example \ref{ex:londom})
 \end{exe}
 
On other hand, the combination of the transitive verb with the reflexive expresses a volitional meaning implying that the action was done on purposive by an animate agent. In (\ref{ex:YAZGAxthom}), the subject of \forme{ɲɤ-ʑɣɤ-xtʰom} `he put himself horizontally=he laid down horizontally' did not fall down (unlike that of example \ref{ex:londom}), but on the contrary lays down on the ground by himself voluntarily to pretend to have fallen down and have died.
 
\begin{exe}
\ex \label{ex:YAZGAxthom}
\gll tʂu nɯtɕu ɲɤ-ʑɣɤ-xtʰom tɕe pɯ-kɯ-si to-ʑɣɤpa. \\
road \textsc{dem}:\textsc{loc} \textsc{ifr}:\textsc{west}-\textsc{refl}-put.horizontally \textsc{lnk} \textsc{aor}-\textsc{sbj}:\textsc{pcp}-die \textsc{ifr}-pretend \\
\glt `He laid down on the road horizontally and pretended to be dead.' (140517 huli he lang-zh, 12)
 \end{exe}
 
In (\ref{ex:koZGAtsxaB}) , the verb \japhug{ʑɣɤtʂaβ}{cause oneself to fall/roll} expresses the same action as \japhug{mtsaʁ}{jump}, a clearly volitional motion verb.
 
\begin{exe}
\ex \label{ex:koZGAtsxaB}
\gll tɯ-ci kɯ-wxtɯ\redp{}wxti nɯ ɯ-ŋgɯ nɯtɕu ko-mtsaʁ. tɕe nɯtɕu ko-ʑɣɤ-tʂaβ qʰe     \\
\textsc{indef}.\textsc{poss}-water \textsc{sbj}:\textsc{pcp}-\textsc{emph}\redp{}be.big \textsc{dem} \textsc{3sg}.\textsc{poss}-in \textsc{dem}:\textsc{loc} \textsc{ifr}:\textsc{east}-jump \textsc{lnk} \textsc{dem}:\textsc{loc} \textsc{ifr}:\textsc{east}-\textsc{refl}-cause.to.roll \textsc{lnk}  \\
\glt `He jumped into the river, he let himself roll into it.' (150830 baihe jiemei-zh, 209-2012)
 \end{exe}

When no prenasalized anticausative verb exists, the reflexive can still be used to insist on the volitional character of a usually non-volitional action in combination with a causative prefix (§\ref{sec:refl.caus.volitional}).

\subsection{Lexicalized reflexives} \label{sec:lexicalized.refl}
A few reflexive verbs have meanings that are unpredictable from their base verbs and have become highly lexicalized.

The semi-transitive verb \japhug{ʑɣɤpa}{pretend} (§\ref{sec:refl.erg}, §\ref{sec:semi.transitive}, §\ref{sec:subject.participle.complementation}) transparently comes from the reflexive of the light verb \japhug{pa}{do} (§\ref{sec:pa.lv}), probably through a meaning such as `make oneself into $X$'. This verb is also attested as a stative verb meaning `be arrogant'.

The intransitive auxiliary \forme{ʑɣɤstu}, which occurs as a light verb with ideophones (§\ref{sec:stu.idph}) to express voluntary actions, is the reflexive of the similative verb \japhug{stu}{do like} (§\ref{sec:ditransitive.secundative}, §\ref{sec:svc.similative.verb}).

The verb \japhug{ʑɣɤɕtʰɯz}{reveal one's true nature} (\ref{ex:koZGACthWz}) derives from the orienting verb \japhug{ɕtʰɯz}{turn towards} (§\ref{sec:orienting.verbs}).

\begin{exe}
\ex \label{ex:koZGACthWz}
\gll  ɯʑo nɯstʰɯci pjɯ-sɯzdɯɣ-a mɯ́j-pe ɲɤ-sɯso tɕe ko-ʑɣɤɕtʰɯz. \\
\textsc{3sg} so.much \textsc{ipfv}-cause.to.worry-\textsc{1sg} \textsc{neg}:\textsc{sens}-be.good \textsc{ifr}-think \textsc{lnk} \textsc{ifr}-reveal.oneself \\
\glt `She thought: ``It is not good of me to cause him so much worries'', and she revealed her true identity to him.' (2003kAndzwsqhaj2, 125)
 \end{exe}
 
\subsection{Reflexive causative} \label{sec:refl.caus}
Reflexive derivations from causativized verbs are extremely common in Japhug, and have a considerable diversity of uses.

\subsubsection{Reflexive causative and volitionality}  \label{sec:refl.caus.volitional}
When applied to non-volitional intransitive verbs, the double (causative+reflexive) derivation is a strategy to express volitional action. For instance, a non-volitional verb like \japhug{fka}{be full} is not attested in the Imperative (§\ref{sec:imp.function}). The doubly derived \forme{ʑɣɤ-ɕɯ-fka}, which conveys the meaning `cause oneself to become full=eat to one's full', can on the other hand occur in the Imperative, as in (\ref{ex:tAZGACWfka}).

\begin{exe}
\ex \label{ex:tAZGACWfka}
\gll tɤ-ʑɣɤ-ɕɯ-fka \\
\textsc{imp}-\textsc{refl}-\textsc{caus}-be.full \\
\glt `Eat to your full!' (a common polite expression)
 \end{exe}
 
Examples (\ref{ex:tAZGACWfka}) and (\ref{ex:atAtWGAZo.AtAZGAGAmbjom}), illustrate that the reflexive prefix is compatible with both the velar  (§\ref{sec:velar.caus.other}) and the sigmatic (§\ref{sec:sig.causative}) causative prefixes.  

 Note that the meaning of \japhug{ʑɣɤɣɤʑo}{make oneself light} can also be expressed with a periphrastic construction as in (\ref{ex:kWZo.YAZGAsApa}) below.
 
\begin{exe}
\ex \label{ex:atAtWGAZo.AtAZGAGAmbjom}
\gll nɤʑo nɯ, tɤ-muj stʰɯci a-tɤ-tɯ-ʑɣɤ-ɣɤ-ʑo,  nɤ-mbro nɯnɯ qale stʰɯci a-tɤ-ʑɣɤ-ɣɤ-mbjom tɕe, \\
\textsc{2sg} \textsc{dem} \textsc{indef}.\textsc{poss}-feather so.much \textsc{irr}-\textsc{pfv}-2-\textsc{refl}-\textsc{caus}-be.light \textsc{2sg}.\textsc{poss}-horse \textsc{dem} wind so.much \textsc{irr}-\textsc{pfv}-\textsc{refl}-\textsc{caus}-be.quick \textsc{lnk} \\
\glt  `(If) you make yourself as light as a feather, and your horse makes itself as quick as the wind, (you will succeed).'  (2011-04-smanmi, 65-66)
 \end{exe}

This use of the double causative+reflexive derivation is similar to the reflexivization of transitive verbs that have prenasalized anticausative counterparts to express the volitional intransitive (§\ref {sec:refl.acaus}).

When the meaning of the causative verb is not completely predictable from that of the base verb, the reflexivized verb always follows the meaning of the causative, rather than that of the base verb. For instance, \forme{ʑɣɤ-ɣɤ-la} `bathe into, immerse oneself into' (example \ref{ex:CpjWZGAGAlanW.kWZGAnWsmAn} above), derives from the intransitive verb \japhug{la}{soak} through the causative \japhug{ɣɤ-la}{immerse, dip in}. However, even in this case the doubly derived verb \forme{ʑɣɤ-ɣɤ-la} is volitional and the base verb  \japhug{la}{soak} non-volitional (and essentially only takes inanimate subjects).

When the base verb is an intransitive verb that can be used volitionally, the doubly derived verb is used to put emphasis on the fact that the subject took active measures to perform the action. For instance  the verb \japhug{mbɣom}{be in a hurry} (a verb whose Imperative form \forme{tɤ-mbɣom} `hurry up!' exists) has the double derivation \forme{ʑɣɤ-ɕɯ-mbɣom} `hasten, hurry, strive to do $X$ as fast as possible', as in (\ref{ex:toZGACWmbGom}).

\begin{exe}
\ex \label{ex:toZGACWmbGom}
\gll to-ʑɣɤ-ɕɯ-mbɣom ʑo jo-nɯ-ɕe tɕe ɯ-jilco ra nɯ-ɕki to-ti. \\
\textsc{ifr}-\textsc{refl}-\textsc{caus}-be.in.a.hurry \textsc{emph} \textsc{ifr}-\textsc{vert}-go \textsc{lnk} \textsc{3sg}.\textsc{poss}-villager \textsc{pl} \textsc{3pl}.\textsc{poss}-\textsc{dat} \textsc{ifr}-say \\
\glt `He hastened to go back (to the village), and told it to the other villagers.' (150902 hailibu-zh, 109)
\end{exe}

Some intransitive verbs have both volitional and non-volitional uses, with slightly different meanings. For instance, the intransitive \forme{ɴqoʁ} has two meanings: `hang, be hanging' or `grab and cling onto' (see \ref{ex:WtaR.kANqoRnW} in §\ref{sec:WtaR}, and \ref{ex:jWfCWndzximWr} in §\ref{sec:time.ordinals}). The doubly derived verb \forme{ʑɣɤ-ɕɯ-ɴqoʁ} `let oneself hang' (as in \ref{ex:pjWZGACWNqoR}) only reflects the first meaning of the base verb  (the non-volitional `be hanging').

\begin{exe}
\ex \label{ex:pjWZGACWNqoR}
\gll  porɤt nɯ kɯnɤ tɕetu kʰɤrka nɯtɕu tɕe pjɯ-ʑɣɤ-ɕɯ-ɴqoʁ ŋgrɤl. \\
small.spider \textsc{dem} also up.there ceiling \textsc{dem}:\textsc{loc} \textsc{loc} \textsc{ipfv}:\textsc{down}-\textsc{refl}-\textsc{caus}-hang be.usually.the.case:\textsc{fact} \\
\glt `The small spider also lets itself hang down from the ceiling.' (26-mYaRmtsaR, 103)
 \end{exe}
 
 

%\begin{exe}
%\ex \label{ex:WjmNo.kontChAr}
%\gll ɯ-jmŋo ko-ntɕʰɤr  \\
%\textsc{3sg}.\textsc{poss}-dream  \textsc{ifr}-\textsc{refl}-\textsc{caus}-appear \\
%\glt `He had a dream.' (140426 xiaohaizi he hua de shizi-zh, 7)
% \end{exe}
% 
 
 
 
\subsubsection{Reflexive of tropative causatives} \label{sec:refl.trop.caus}
Some causative verbs have tropative (`consider $X$ to be $Y$', §\ref{sec:tropative}) or causative tropative (`cause people to consider $X$ to be $Y$'), for instance \japhug{ɣɤkʰe}{depreciate, demean} from \japhug{kʰe}{be stupid} (§\ref{sec:tropative.other.construction}). Reflexive verbs deriving from this type of causative mean `(cause people to) consider oneself to be $Y$'; for example, \japhug{ʑɣɤɣɤ\-kʰe}{depreciate, demean} is not used in meaning `cause oneself to become stupid'.
 
 
 \subsubsection{Motion verbs} \label{sec:refl.caus.motion}

With the stative verbs of relative location \japhug{armbat}{be near} and  \japhug{arqʰi}{be far}, the reflexive+causative double derivations produce the motion verbs \japhug{ʑɣɤsɤrmbat}{move closer} and \japhug{ʑɣɤsɤrqʰi}{move further away}, whose goal is marked in the dative, as in (\ref{ex:WCki.koZGAsArmbat}) (see also \ref{ex:tWCki} in §\ref{sec:indef.genr.poss}). Although zero conversion of these stative verbs into motion verbs is also attested (§\ref {sec:convertion.orientable.verbs}), the double derivation is by far the most common way of expressing these meanings.

\begin{exe}
\ex \label{ex:WCki.koZGAsArmbat}
\gll ɯʑo daltsɯtsa ʑo li, masɤrɯrju ʑo, srɯtpʰu nɯ ɯ-ɕki ko-ʑɣɤ-sɯ-ɤrmbat. \\
\textsc{3sg} slowly \textsc{emph} again in.secret \textsc{emph} ogre \textsc{dem} \textsc{3sg}.\textsc{poss}-\textsc{dat} \textsc{ifr}:\textsc{east}-\textsc{refl}-\textsc{caus}-be.near \\
\glt `He moved closer to the ogre slowly in secret.'  (160706 poucet6, 91)
\end{exe}

   
When the base verb is a motion verb, the reflexive+causative double derivation also yields a motion verb, as in the case of  \forme{ʑɣɤ-sɯ-ɤzɣɯt} `manage to reach' from \japhug{zɣɯt}{reach} (on the \forme{ɤ-} prefixal element, see §\ref{sec:intr.person.irregular}). The verb \forme{ʑɣɤ-sɯ-ɤzɣɯt} differs from its base verb in that it expresses that the subject  had to strive hard to reach the goal, as example (\ref{ex:nW.kWnA.joZGAsAzGWt}) illustrates.

\begin{exe}
\ex \label{ex:nW.kWnA.joZGAsAzGWt}
\gll ɯ-tɯ-ɤrqʰi pjɤ-saχaʁ ʑo ri, nɯ kɯnɤ jo-ʑɣɤ-sɯ-ɤzɣɯt. \\
\textsc{3sg}.\textsc{poss}-\textsc{nmlz}:degree-be.far \textsc{ifr}.\textsc{ipfv}-be.extremely \textsc{emph} \textsc{lnk} \textsc{dem} also \textsc{ifr}-\textsc{refl}-\textsc{caus}-reach \\
\glt `(China)$_i$ was very far, but even so he managed to reach it$_i$.' (140511 alading-zh, 22)
\end{exe}
%ʑɣɤsɯxtso
\subsubsection{Reflexive  causatives from transitive verbs}
When the base verb is transitive, the result of the reflexive+causative double derivation is still an intransitive verb. 

The double derivation can have a plain compositional meaning, as in \forme{ʑɣɤ-sɯ-rtoʁ} `have oneself examined' in (\ref{ex:atAtWZGAsWrtoR}). The reflexivized argument corresponds to the causer and object of the base verb. The causee \japhug{smɤnba}{doctor} does not received ergative marking, and is not indexed on the verb, as an inverse 3\fl{}2 configuration (§\ref{sec:indexation.mixed}) would be nonsensical in (\ref{ex:atAtWZGAsWrtoR}).


\begin{exe}
\ex \label{ex:atAtWZGAsWrtoR}
\gll smɤnba a-tɤ-tɯ-ʑɣɤ-sɯ-rtoʁ \\
doctor \textsc{irr}-\textsc{pfv}-2-\textsc{refl}-\textsc{caus}-see \\
\glt `You should have yourself examined by a doctor.' (elicited)
\end{exe}

Doubly derived verbs can also express involuntary actions `get oneself $X$ed'. For instance, the reflexive+causative \forme{ʑɣɤ-sɯ-sat} from \japhug{sat}{kill} can have the meaning `get oneself killed', as in (\ref{ex:pjWZGAsWsat}).

\begin{exe}
\ex \label{ex:pjWZGAsWsat}
\gll ma βɣɯz ŋu tɕe, kʰɤrkɯ ra wuma ʑo ɣi. tɕe nɯnɯ pjɯ-ʑɣɤ-sɯ-sat ŋgrɤl. \\ 
\textsc{lnk} badger be:\textsc{fact} \textsc{lnk} side.of.the.house \textsc{pl} really \textsc{emph} come:\textsc{fact} \textsc{lnk} \textsc{dem} \textsc{ipfv}-\textsc{refl}-\textsc{caus}-kill be.usually.the.case:\textsc{fact} \\ 
\glt `The badger often comes near houses, and gets itself killed. (The dogs chase it and kill it, and people also kill it).' (27-spjaNkW, 127-129)
\end{exe}

With some transitive verbs, the double derivation has the same meaning `take pains to $X$, strive to $X$' as with intransitive verbs above (examples \ref{ex:toZGACWmbGom} and \ref{ex:nW.kWnA.joZGAsAzGWt}). For instance, the reflexive causative \forme{ʑɣɤ-sɯ-pjɤl} of the transitive motion verb \japhug{pjɤl}{go around, cross, avoid} (§\ref{sec:motion.verbs}) means `strive to avoid $X$, take the necessary measures to avoid $X$' rather than `get oneself avoided': with this verb, the reflexivization targets the causee rather than the object (`cause oneself to avoid $X$' rather than `cause $X$ to avoid oneself').

\begin{exe}
\ex \label{ex:tukAZGAsWpjAl.kowa}
\gll tɕendɤre nɯ tu-kɤ-ʑɣɤ-sɯ-pjɤl kowa ntsɯ tu-βzu-ndʑi pjɤ-ŋu. \\
\textsc{lnk} \textsc{dem} \textsc{ipfv}-\textsc{inf}-\textsc{refl}-\textsc{caus}-go.around manner always \textsc{ipfv}-make-\textsc{du} \textsc{ifr}.\textsc{ipfv}-be \\
\glt `(The mouse and the sparrow) took every measure to avoid (the cat).' (lWlu, 20)
\end{exe}

 %ʑɣɤsɯndza
%ʑɣɤɕɯɣmu
%ʑɣɤɕɯrga
%ʑɣɤsɯβzi
%ʑɣɤɕɯmbɣom
%ʑɣɤsɯmtɕɯr

The causative verbs \japhug{sɤpa}{transform} and \japhug{sɤβzu}{transform} (§\ref{sec:passive.other.derivations}) derived from the lexicalized passive verbs \japhug{apa}{become} and \japhug{aβzu}{become, grow} (§\ref{sec:passive.lexicalized}) are ditransitive, selecting as object the entity that undergoes transformation, and as semi-object the entity into which the change occurs. Reflexivization targets the object, and the corresponding reflexive forms \forme{ʑɣɤsɤpa}  and \forme{ʑɣɤsɤβzu} mean `transform oneself into $X$' (rather than `transform $X$ into oneself'). They select as semi-object the entity into which the subject is transformed, for instance \japhug{pjɤrgɤt}{vulture} in (\ref{ex:pjArgAt.nWnWZGAsABzu}).

\begin{exe}
\ex \label{ex:pjArgAt.nWnWZGAsABzu}
\gll sloχpɯn nɯ ci nɯ-ʑɣɤ-sɤphɤr nɤ pjɤrgɤt ci nɯ-nɯ-ʑɣɤ-sɯ-ɤβzu ɲɯ-ŋu. \\
teacher \textsc{dem} a.little \textsc{aor}-\textsc{refl}-shake \textsc{add} vulture \textsc{indef} \textsc{aor}-\textsc{auto}-\textsc{refl}-\textsc{caus}-become \textsc{sens}-be \\
 \glt `The teacher shook himself, and transformed himself into a vulture.'  (2003kandZislama, 61-62)
\end{exe}

Just like the causative verbs \forme{sɤpa}  and \forme{sɤβzu} from which they are derived (example \ref{ex:kWmpCWmpCAr.nasABzu}, §\ref{sec:passive.other.derivations}, §\ref{sec:sig.caus.other.derivations}), \forme{ʑɣɤsɤpa} and \forme{ʑɣɤsɤβzu} can select as semi-object participial clauses, and be used as periphrastic causative constructions (§\ref{sec:sWpa.sABzu}), as in (\ref{ex:kWZo.YAZGAsApa}), where instead of the reflexive causative verb \japhug{ʑɣɤɣɤʑo}{make oneself light} (example \ref{ex:atAtWGAZo.AtAZGAGAmbjom} above), the participle \forme{kɯ-ʑo} `the one who is light' occurs as semi-object of \japhug{ʑɣɤsɤpa}{transform oneself into}.

 
\begin{exe}
\ex \label{ex:kWZo.YAZGAsApa}
\gll tɯrme ɯʑo nɯnɯ rcanɯ [tɤ-muj ʑo kɯ-fse kɯ-ʑo] ɲɤ-ʑɣɤ-sɯ-ɤpa, \\
man \textsc{3sg} \textsc{dem} \textsc{unexp}:\textsc{deg} \textsc{indef}.\textsc{poss}-feather \textsc{emph} \textsc{sbj}:\textsc{pcp}-be.like \textsc{sbj}:\textsc{pcp}-be.light \textsc{ifr}-\textsc{refl}-\textsc{caus}-become \\
\glt `The man, him, made himself as light as a feather.' (04-smanmi, 84)
\end{exe}

 \subsection{Reflexive and autive} \label{sec:refl.autoben}
Reflexive and autive (§\ref{sec:autobenefactive}) derivations have completely different semantics, and distinct morphosyntactic properties, since the latter does not change the verb transitivity. With a handful of verbs such as \japhug{ʁmɯɣ}{intend to, decide}, the reflexive form has an autobenefactive meaning (\japhug{ʑɣɤʁmɯɣ}{decide for oneself}), but the transitivity is changed. 
 
The autive \forme{nɯ-} (§\ref{sec:autobenefactive}) is compatible with the reflexive \forme{ʑɣɤ-}. It is the only derivational prefix that precedes the reflexive prefix in the prefixal chain (§\ref{sec:inner.prefixal.chain}). The autive+reflexive combination has four different meanings.

First, it can express and involuntary action that the subject both causes and suffers from (as in \ref{ex:YWtWnWZGAkho} above).

Second, the \forme{nɯ-} prefix can also be used to put emphasis on the fact that the subject perform the reflexive action by him/her/itself without external help, as in (\ref{ex:kunWZGAfstWn2}).

\begin{exe}
\ex \label{ex:kunWZGAfstWn2}
\gll  tɕetʰa tɯ-ji ɯ-ŋgɯ nɯnɯra, iɕqʰa tɯ-rdoʁ kɯ-fse nɯra ɕɯ-ndze, tɕendɤre, ɯʑo ku-nɯ-ʑɣɤ-fstɯn cʰa   \\
later \textsc{indef}.\textsc{poss}-field \textsc{3sg}.\textsc{poss}-in \textsc{dem}:\textsc{pl} \textsc{filler} one-piece \textsc{sbj}:\textsc{pcp}-be.like \textsc{dem}:\textsc{pl} \textsc{tral}-eat[III]:\textsc{fact} \textsc{lnk} \textsc{3sg} \textsc{ipfv}-\textsc{auto}-\textsc{refl}-take.care can:\textsc{fact}   \\
\glt `(This hen) will go and eat the grains in the fields, it will be able to take care of itself on its own.' (150822 laoye zuoshi zongshi duide-zh, 149)
\end{exe} 

Third, the \forme{nɯ-} prefix is also found with the autobenefactive meaning `for oneself' in these forms. In (\ref{ex:tonWZGACWfkandZi}) for instance, the autive prefix occurs to insist on the fact that the parents (dual subject) ate to their full in the absence of their children.

\begin{exe}
\ex \label{ex:tonWZGACWfkandZi}
\gll ndʑi-pɤri kɯ-pɯ\redp{}pe to-βzu-ndʑi qʰe, ʑɤni to-nɯ-ʑɣɤ-ɕɯ-fka-ndʑi. \\
\textsc{3du}.\textsc{poss}-diner \textsc{sbj}:\textsc{pcp}-\textsc{emph}\redp{}be.good \textsc{ifr}-make-\textsc{du} \textsc{lnk} \textsc{3du} \textsc{ifr}-\textsc{auto}-\textsc{refl}-\textsc{caus}-be.full-\textsc{du} \\
\glt `They made a nice diner for themselves, and ate to their fill.' (160701 poucet2, 16-17)
\end{exe}

Fourth, it can occur to express an action taking place spontaneously without external agent, as in (\ref{ex:chAnWZGAsAtsa}).

\begin{exe}
\ex \label{ex:chAnWZGAsAtsa}
\gll  kɯm cʰɤ-nɯ-ʑɣɤ-sɯ-ɤtsa \\
door \textsc{ifr}-\textsc{auto}-\textsc{refl}-\textsc{caus}-be.locked \\
\glt `The dock locked itself.' (elicited; describes house doors with an automatic locking system)
\end{exe}

The autive is common on verbs with reflexive and causative prefixes (§\ref{sec:refl.caus}), resulting in a triple derivation, as in (\ref{ex:tonWZGACWfkandZi}) and (\ref{ex:chAnWZGAsAtsa}) above.

 \subsection{Reflexive, tropative and applicative} \label{sec:refl.tropative}
Apart from the causative, the reflexive derivation can be added after other valency-increasing derivations such as the tropative and the applicative. 

As an example of reflexivized tropative, \japhug{nɤmpɕɤr}{consider to be beautiful} (from \japhug{mpɕɤr}{be beautiful}) can be reflexivized as \japhug{ʑɣɤnɤmpɕɤr}{consider oneself to be beautiful} (§\ref{sec:tropative.other.derivations}).

The applicative verbs \japhug{nɯɣmu}{be afraid of} and \japhug{nɯrga}{like, love} (from \japhug{mu}{be afraid} and \japhug{rga}{like}) have reflexive form \forme{ʑɣɤ-nɯ-rga} and \forme{ʑɣɤ-nɯɣ-mu}  which can either mean `like/be afraid of oneself' or have a reflexive+causative meaning `have people like/be afraid of oneself'. The latter meaning can also be expressed with a triple derivation such as \forme{ʑɣɤ-z-nɯɣ-mu}  with the causative \forme{z-} between the reflexive and the applicative prefixes.

The applicative \japhug{nɤstu}{believe in}, which selects as object the person one believes in (example \ref{ex:WYWkWnAstua}, §\ref{sec:applicative.promoted}), can also be reflexivized as \japhug{ʑɣɤnɤstu}{believe in oneself}, as in (\ref{ex:mWYAZGAnAstu}).\footnote{This example is from a translated story, but the presence of a second object (the relative \forme{nɯ pa-mto nɯra}) is not due to calquing from the original (which has \ch{不敢相信}{bùgǎnxiāngxìn}{he did not dare to believe it}), and was not considered to be clumsy upon rechecking. }

\begin{exe}
\ex \label{ex:mWYAZGAnAstu}
\gll nɯ pa-mto nɯra mɯ-ɲɤ-ʑɣɤ-nɤ-stu.  \\
\textsc{dem} \textsc{aor}:3\flobv{}-see \textsc{dem}:\textsc{pl} \textsc{neg}-\textsc{ifr}-\textsc{refl}-\textsc{appl}-believe \\
\glt `He did not believe the things that he himself had (just) seen.' (150830 baihe jiemei-zh, 106)
\end{exe}

 
\subsection{Historical origin}  \label{sec:reflexive.origin}
The reflexive \forme{ʑɣɤ-} is cognate to forms found in other Gyalrong languages, including Tshobdun \forme{oɟɐ-} \citep{jackson14morpho}, Zbu \forme{vjɐ-} (\citealt[9]{gong18these}) and Situ \forme{wjɐ-}.\footnote{Some dialects of Zbu however have a divergent reflexive prefix \forme{ɲɐ\trt}, which cannot be cognate to the forms discussed here (\citealt[9]{gong18these}). } These prefixes go back to a proto-Gyalrong form \forme{*wjɐ-} (with a variant \forme{*wəjɐ-} for Tshobdun) with metathesis in Japhug. However, the cluster \forme{ʑɣ-} is otherwise only attested in Japhug in ideophones such as \japhug{ʑɣɤrʑɣɤr}{having some (pieces) coming out (out of a bundle)} (§\ref{sec:idph.onsets}).

This prefix itself is possibly the incorporated \textit{status constructus} form of the third singular pronoun \forme{*wəjaŋ} (corresponding to Japhug \japhug{ɯʑo}{he}, §\ref{sec:pers.pronouns}, whose irregular phonology is discussed in §\ref{sec:3sg.possessive.form}); typological parallels of the \textsc{personal pronoun} $\Rightarrow$ \textsc{reflexive} change proposed here also exist in Yukaghir (\citealt{jacques10refl}, \citealt[§5.2]{maslova07yukaghir}).\footnote{For an alternative etymology, see \citet{jackson14morpho}.}
 
The reflexive prefix is a core Gyalrong innovation, not shared even by Khroskyabs. The reflexive prefix \forme{ʁjæ̂-} in Khroskyabs is not directly related to its core Gyalrong equivalent, as it is based on the denominal \forme{ʁ-} (\citealt[300]{lai17khroskyabs}), cognate of the Japhug stative denominal \forme{a-} (§\ref{sec:denom.contracting}). Both Khroskyabs \forme{ʁjæ̂-} and proto-Gyalrong \forme{*wjɐ-} reflexive prefixes share the \textit{status constructus} of the pronominal base \forme{*jaŋ-} (Japhug \forme{-ʑo}, §\ref{sec:pers.pronouns}), but result from independent grammaticalizations.

\section{Reciprocal} \label{sec:reciprocal}

\subsection{Reduplicated reciprocal} \label{sec:redp.reciprocal}
The reduplicated reciprocal derivation is the productive way of expressing mutual action in Japhug. It combines the \forme{a-} prefixal element found in the passive (§\ref{sec:passive}) and several denominal derivations (§\ref{sec:denom.contracting}) with the partially reduplicated stem of the base verb. If the verb has a polysyllabic stem (whether or the the non-final syllable are derivational prefixes), only the last syllable is reduplicated. For instance \japhug{rqoʁ}{hug},  \japhug{nɯpoʁ}{kiss} and \japhug{nɯrɯtʂa}{envy}\footnote{The verb \japhug{nɯrɯtʂa}{envy} is denominal from \japhug{rɯtʂa}{envy} (n). } yield \forme{a-rqɯ\redp{}rqoʁ} `hug each other'  (\ref{ex:kokArqWrqoRnWci}), \forme{a-nɯpɯ\redp{}poʁ} `kiss each other' and \forme{a-nɯrɯtʂɯ\redp{}tʂa} `envy each other', respectively. 

As in other cases of partial derivation, the reduplication disregards morpheme boundaries: when the reciprocal derivation is applied to verbs with the \forme{sɯɣ-} and \forme{nɯɣ-} allomorphs of the sigmative causative (§\ref{sec:caus.sWG}) and applicative (§\ref{sec:allomorphy.applicative}), the final \forme{ɣ-} is reduplicated together with the monosyllabic verb root. 

For instance, the reciprocal of the applicative \japhug{nɯɣbɯɣ}{miss, long for} is \forme{anɯɣbɯ\redp{}ɣbɯɣ} `miss each other' (§\ref{sec:reciprocal.other}), despite the fact that the verb root is \japhug{bɯɣ}{miss home} (\tabref{tab:applicative}, §\ref{sec:applicative}) and that the preinitial \forme{ɣ-} belongs to the applicative prefix. 

Since reduplicated reciprocal has two exponents (\forme{a-} and reduplication), in the glosses only the \forme{a-} prefix is glossed as \textsc{recip}, while the reduplication left unglossed to avoid unnecessary redundancy: thus \forme{a-rqɯ\redp{}rqoʁ} is glossed as \textsc{recip}-hug rather than as \textsc{recip}-\textsc{recip}\redp{}hug. 

The \forme{ɤ/a/o} allomorphy of the \forme{a-} element, and the presence of the peg \forme{k-...-ci} (§\ref{sec:peg.circumfix}) in the Inferential (as in \ref{ex:kokArqWrqoRnWci}) follows the same rules as other contracting verbs (§\ref{sec:contraction}, §\ref{sec:preverbs.contracting.verbs}).

Reciprocal verbs are morphologically intransitive, and their subject is necessarily non-singular except in the case of verbs expressing naturally collective action \citep[123--127]{kemmer93middle} (as in \ref{ex:Wqa.YAndWndo} below), in generic forms, where number is neutralized, or when the subject is a group of inanimate and poorly distinguishable entities. 

When the plural or dual subject of the reciprocal verb is the combination of a previously mentioned entity or group with another group, the comitative \forme{cʰo} (§\ref{sec:comitative})  can be used to specify the second group. 

For instance, in (\ref{ex:kokArqWrqoRnWci}), the subject of the verb \forme{ko-k-ɤ-rqɯ\redp{}rqoʁ-nɯ-ci} `they hug each other' corresponds to the sum of referent of the subject of the first verb \forme{jo-nɯ-ɕe} `he went back' with the group of people referred to by the postpositional phrase \forme{ɯ-ɣi ra cʰo} `with his relatives' (see §\ref{sec:semi.transitive.dative} for examples of the same phenomenon with non-reciprocal verbs).

\begin{exe}
\ex \label{ex:kokArqWrqoRnWci}
\gll   kʰa jo-nɯ-ɕe tɕendɤre [ɯ-ɣi ra cʰo] ko-k-ɤ-rqɯ\redp{}rqoʁ-nɯ-ci ʑo ɲɤ-ɣɤwu-nɯ. \\
home \textsc{ifr}-\textsc{vert}-go \textsc{lnk} \textsc{3sg}.\textsc{poss}-relative \textsc{pl} \textsc{comit} \textsc{ifr}-\textsc{peg}-\textsc{recip}-hug-\textsc{pl}-\textsc{peg} \textsc{emph} \textsc{ifr}-cry-\textsc{pl} \\
\glt `He went home and he and his relatives hug each other and cried.' (140512 fushang he yaomo1, 35)
\end{exe}

Expectedly, the reciprocal of secundative verbs (§\ref{sec:ditransitive.secundative}) targets the recipient object; for instance, \japhug{mbi}{give} yields \japhug{ambɯmbi}{give to each other} (\ref{ex:YAmbWmbinW}).

\begin{exe}
\ex \label{ex:YAmbWmbinW}
\gll tɕe nɯ ɯ-ŋgɯ srɯsmɤn ra pjɯ-lɤt-nɯ, tɕe nɯra pjɯ-nɯ-lɤt-nɯ tɕe nɯra ɲɯ-ɤmbɯ\redp{}mbi-nɯ ra ŋu \\
\textsc{lnk} \textsc{dem} \textsc{3sg}.\textsc{poss}-in medicine \textsc{pl} \textsc{ipfv}:\textsc{down}-release-\textsc{pl} \textsc{lnk} \textsc{dem}:\textsc{pl} \textsc{ipfv}:\textsc{down}-\textsc{auto}-release-\textsc{pl} \textsc{lnk} \textsc{dem}:\textsc{pl} \textsc{ipfv}-\textsc{recip}-give-\textsc{pl} \textsc{pl} be:\textsc{fact} \\
\glt `They put medicine (against sinus inflammation) into the (snuff tobacco)$_i$, and they give it$_i$ to each other.' (30-CnAto, 17-18)
\end{exe}
 
The reduplicated reciprocal derivation can also express naturally collective events without clearly distinct agents and patients, such as \japhug{awɯwum}{gather together} (\ref{ex:chAwWwumnW}) from \japhug{wum}{gather}.

\begin{exe}
\ex \label{ex:chAwWwumnW}
\gll qartsɯ tɤ-mda qʰe tɕe cʰɯ-ɤ-wɯ\redp{}wum-nɯ qʰe kɯ-dɯ\redp{}dɤn ʑo tɯtɯrca ku-rɤʑi-nɯ ɲɯ-ŋu \\
winter \textsc{aor}-be.the.time \textsc{lnk} \textsc{lnk} \textsc{ipfv}:\textsc{downstream}-\textsc{recip}-gather-\textsc{pl} \textsc{lnk} \textsc{sbj}:\textsc{pcp}-\textsc{emph}\redp{}be.many \textsc{emph} together \textsc{ipfv}-stay-\textsc{pl} \textsc{sens}-be \\
\glt `When winter arrives, they gather and stay together in great numbers.' (23-qapGAmtWmtW, 107-108)
\end{exe}

The meaning of this reciprocal form is also slightly different from that of the base verb, since \japhug{wum}{gather} has a wide range of extended meanings, such as `take as a X' (as in example \ref{ex:naslama.kukWwuma}, §\ref{sec:essive.abs}) or `fold wings' (as in \ref{ex:YWqAt.nA.kuwum}, §\ref{sec:centripetal.centrifugal}) which are completely absent from the reciprocal form. In addition, the verb  \forme{wum} typically takes non-human entities as objects when meaning `gather' as in (\ref{ex:aCtAtWwum}).

\begin{exe}
\ex \label{ex:aCtAtWwum}
\gll tɯrsa ɯ-rkɯ χpɯn-ŋga tʰamtɕɤt a-ɕ-tɤ-tɯ-wum tɕe \\
cemetery \textsc{3sg}.\textsc{poss}-side monk-clothes all \textsc{irr}-\textsc{tral}-\textsc{pfv}-2-gather \textsc{lnk} \\
\glt `Go and collect all the monk robes near the cemetery.' (2003qachGa, 155)
\end{exe}
 
The intransitive verb \forme{andɯndo} derived from \japhug{ndo}{catch} can have a prototypical reciprocal meaning `grab each other' (especially relative to fighting, see example \ref{ex:YWsAtWtea} below, §\ref{sec:reciprocal.other}). It also occurs with the naturally collective event meaning `be clustered together' as in (\ref{ex:Wqa.YAndWndo}) with a singular verb form, due to the fact that mushrooms are poorly differentiable inanimate referents (§\ref{sec:optional.indexation}).

\begin{exe}
\ex \label{ex:Wqa.YAndWndo}
\gll tɕe tɕe ɯ-qa nɯ ɲɯ-ɤ-ndɯ\redp{}ndo, ɯ-taʁ nɯ ki kɯ-fse kɯ-dɯ\redp{}dɤn ɲɯ-ŋu tɕe, \\
 \textsc{lnk} \textsc{lnk} \textsc{3sg}.\textsc{poss}-foot \textsc{dem} \textsc{sens}-\textsc{recip}-take \textsc{3sg}.\textsc{poss}-top \textsc{dem} \textsc{dem}.\textsc{prox} \textsc{sbj}:\textsc{pcp}-be.like \textsc{sbj}:\textsc{pcp}-\textsc{emph}\redp{}be.many \textsc{sens}-be \textsc{lnk} \\
\glt `The base (of the mushroom \ch{刷把菌}{shuābǎjūn}{Ramaria formosa}) is all clustered together, but it has many top parts.' (23-tshAYCAnW, 12)
\end{exe}

A handful of intransitive verbs can derive reciprocal forms. The verb of speech \japhug{rɯɕmi}{talk} has the derived form \forme{a-rɯɕmɯ\redp{}ɕmi} `exchange words, talk to each other' (see example \ref{ex:YAmJAkhondZi}, §\ref{sec:co.participation}), with reciprocalization of the dative-marked recipient (compare with \ref{ex:WCki.torWCmi}, §\ref{sec:dative}).

\subsubsection{Reciprocal and noun-verb collocations} \label{sec:reciprocal.collocation}
In some noun-verb collocations (§\ref{sec:tr.light.verbs}), the verb can undergo the reduplicated reciprocal derivation, expressing mutual action between the subject and the possessor of the object of the base construction.

For instance, the collocation meaning `braid hair' (\ref{ex:Wku.nWBzuta}) including the verb \japhug{βzu}{make} and the body part \japhug{tɯ-ku}{head} (§\ref{sec:body.part}) yields in (\ref{ex:Wku.YABzWBzundZi}) a reciprocalized construction with the reduplicated verb \forme{aβzɯβzu} and the noun \forme{tɯ-ku} demoted as semi-object.

\begin{exe}
\ex \label{ex:Wku.nWBzuta}
\gll  a-pi ɯ-ku nɯ-βzu-t-a \\
\textsc{1sg}.\textsc{poss}-elder.sibling \textsc{3sg}.\textsc{poss}-head \textsc{aor}-make-\textsc{pst}:\textsc{tr}-\textsc{1sg} \\
\glt `I braided my sister's hair.' (elicited)
\end{exe}

\begin{exe}
\ex \label{ex:Wku.YABzWBzundZi}
\gll ɯ-pi cʰo ndʑi-ku ɲɯ-ɤ-βzɯ\redp{}βzu-ndʑi ndɤre, \\
\textsc{3sg}.\textsc{poss}-elder.sibling \textsc{comit} \textsc{3du}.\textsc{poss}-head \textsc{ipfv}-\textsc{recip}-make-\textsc{du} \textsc{lnk} \\
\glt `She and her sister braided each other's hair.' (2005 Kunbzang, 260)
\end{exe}

Example (\ref{ex:nWBra.YABzWBzunW}) illustrates the same phenomenon with the collocation meaning `share with', combining the noun \japhug{tɯ-βra}{share} with \japhug{βzu}{make}.

\begin{exe}
	\ex \label{ex:nWBra.YABzWBzunW}
	\gll nɤʑo kɯ ɯ-pɯ-tɯ-sat nɤ, nɤʑo kɯ aʑo a-βra ɲɯ-tɯ-βze, aʑo pɯ\redp{}pɯ-sat-a nɤ, nɤj nɤ-βra ɲɯ-βze-a, nɯ kɯ-fse, nɯ-βra ɲɯ-ɤ-βzɯ\redp{}βzu-nɯ pjɤ-ŋɡrɤl. \\
	\textsc{2sg} \textsc{erg} \textsc{qu}-\textsc{aor}-2-kill \textsc{add} 	\textsc{2sg} \textsc{erg} \textsc{1sg} \textsc{1sg}.\textsc{poss}-share \textsc{ipfv}-2-make[III] \textsc{1sg} \textsc{cond}\redp{}\textsc{aor}-kill-\textsc{1sg} \textsc{add} \textsc{2sg} \textsc{2sg}.\textsc{poss}-share \textsc{ipfv}-make[III]-\textsc{1sg} \textsc{dem} \textsc{sbj}:\textsc{pcp}-be.like \textsc{3pl}.\textsc{poss}-share \textsc{ipfv}-\textsc{recip}-make-\textsc{pl} \textsc{ifr}.\textsc{ipfv}-be.usually.the.case-\textsc{pl} \\
	\glt `If it was you who had killed (game during a hunt), you would share some of it with me, if it was me who had killed (game), I would share some of it with you, this way, people used to share (their game meat) with each other.' (160714 XsWmsna, 15-17)
\end{exe}

\subsubsection{Reciprocal and other derivations} \label{sec:reciprocal.other}
The reciprocal derivation is highly productive, can be applied to verbs that have undergone a valency-increasing derivation such as causative (§\ref{sec:sig.causative}, §\ref{sec:velar.causative}), applicative (§\ref{sec:applicative}) or tropative (§\ref{sec:tropative}). 

Reciprocalized causatives are found with both sigmatic and velar causative verbs. For instance, the velar causative \japhug{ɣɤrlaʁ}{destroy} (from the intransitive verb \japhug{rlaʁ}{disappear}, borrowed from \tibet{བརླག་}{brlag}{lose}) has a reciprocal form \japhug{aɣɤrlɯrlaʁ}{destroy each other} (\ref{ex:chAGArlWrlaRnW}).

 \begin{exe}
\ex \label{ex:chAGArlWrlaRnW}
\gll kɯɕɯŋgɯ tɕe, tɤru ra tu-o-nɯsnɯɲɯ\redp{}ɲaʁ-nɯ tɕe cʰɯ-ɤ-ɣɤ-rlɯ\redp{}rlaʁ-nɯ pjɤ-ŋgrɤl \\
former.times \textsc{lnk} chieftain \textsc{pl} \textsc{ipfv}-\textsc{recip}-do.harm-\textsc{pl} \textsc{lnk} \textsc{ipfv}-\textsc{recip}-\textsc{caus}-disappear-\textsc{pl} \textsc{ipfv}.\textsc{ifr}-be.usually.the.case \\
\glt `In former times, chieftains used to harm (murder) each other and destroy each other's families.' (elicited)
\end{exe}

Reciprocalization of sigmatic causatives is productive. Example (\ref{ex:YAkAsWGYWGYaRnW}) illustrates two verbs with this double derivation: \japhug{asɯɴqʰɯɴqʰi}{make each other dirty} (from the causative  \japhug{sɯɴqʰi}{make dirty} derived from \japhug{ɴqʰi}{be dirty}) and  \japhug{asɯɣɲɯɣɲaʁ}{blacken each other} (from the causative  \japhug{sɯɣɲaʁ}{blacken} from \japhug{ɲaʁ}{be black}). In the latter, note that partial reduplication targets the syllable \ipa{ɣɲaʁ}, disregarding morpheme boundaries (the \ipa{ɣ} is part of the causative prefix, §\ref{sec:caus.sWG}).

 \begin{exe}
\ex \label{ex:YAkAsWGYWGYaRnW}
\gll ɲɤ-k-ɤ-sɯ-ɴqʰɯ\redp{}ɴqʰi-ndʑi tɕe, ɲɤ-k-ɤ-sɯɣ-ɲɯ\redp{}ɣɲaʁ-ndʑi-ci ʑo \\
\textsc{ifr}-\textsc{peg}-\textsc{recip}-\textsc{caus}-be.dirty-\textsc{du} \textsc{lnk} \textsc{ifr}-\textsc{peg}-\textsc{recip}-\textsc{caus}-be.black-\textsc{du}-\textsc{peg} \textsc{emph} \\
\glt `They$_{DU}$ made each other dirty, they blackened each other.' (elicited, can be said of children playing in a dirty place)
\end{exe}

The lexicalized verb \japhug{nɯsɯkʰo}{rob, extort} which etymologically derives from \japhug{kʰo}{give} with the causative and autive prefixes (§\ref{sec:sig.caus.lexicalized}) can also undergo reciprocalization to \japhug{anɯsɯkʰɯkʰo}{extort each other} as in (\ref{ex:pjAkAnWswkhWkhonWci}).
 
 \begin{exe}
\ex \label{ex:pjAkAnWswkhWkhonWci}
\gll nɯnɯ jɤ-kɤ-ɣɤrɤt nɯ pjɤ-k-ɤ-nɯsɯkʰɯ\redp{}kʰo-nɯ-ci. \\
\textsc{dem} \textsc{aor}-\textsc{obj}:\textsc{pcp}-throw \textsc{dem} \textsc{ifr}.\textsc{ipfv}-\textsc{peg}-\textsc{recip}-rob-\textsc{pl}-\textsc{peg} \\
\glt `The beasts fought with each other to get the (piece of cloth) that he had thrown (at them).' (150825 huluwa-zh, 150)
\end{exe}

Reciprocalized tropatives are not commonly found in the corpus, but potentially any tropative verb can undergo reciprocal derivation. For instance, the slightly lexicalized tropative verb \japhug{nɤpe}{consider to be good, like} (from \japhug{pe}{be good}) yields \japhug{anɤpɯpe}{like each other}, as in (\ref{ex:YAkAnApWpendZici}).\footnote{The other verb \japhug{anɤntsʰɯntsʰi}{love each other} in (\ref{ex:YAkAnApWpendZici}) derives from \japhug{nɤntsʰi}{love}, which is a lexicalized and synchronically non-analyzable tropative (§\ref{sec:tropative.lexicalized}). }

\begin{exe}
\ex \label{ex:YAkAnApWpendZici}
\gll  ɲɤ-k-ɤ-nɤntsʰɯ\redp{}ntsʰi-ndʑi-ci, ɲɤ-k-ɤ-nɤ-pɯ\redp{}pe-ndʑi tɕe \\
\textsc{ifr}-\textsc{peg}-\textsc{recip}-love-\textsc{du}-\textsc{peg} \textsc{ifr}-\textsc{peg}-\textsc{recip}-\textsc{trop}-be.good-\textsc{du} \textsc{lnk}\\
\glt `They fell in love with each other.' (150827 mengjiangnv-zh, 104-105)
\end{exe}

The reciprocal derivation is also completely productive with applicative verbs. For instance \japhug{nɯɣbɯɣ}{miss, long for} (from \japhug{bɯɣ}{miss home}), \japhug{nɤkʰɤzŋga}{shout at} (from \japhug{akʰu}{call}) and \japhug{nɯrga}{like} (from \japhug{rga}{like, be happy}) yield  the reciprocal verbs \japhug{anɯɣbɯɣbɯɣ}{miss each other}, \forme{anɤkʰɤzŋgɯ\redp{}zŋga} `shout at each other'  and \forme{anɯrgɯ\redp{}rga} `like each other', respectively. In the case of \forme{a-nɯɣ-bɯ\-\redp{}ɣbɯɣ}, partial reduplication disregards morpheme boundaries (the \ipa{ɣ} that belongs to the applicative prefixes is reduplicated together with the verb root), as in the case of the reciprocal of causative \japhug{asɯɣɲɯɣɲaʁ}{blacken each other} discussed above (§\ref{sec:reciprocal.other}).
 
A sigmatic causative derivation can be applied to a reciprocal verb. For instance, \japhug{awɯwum}{gather together} (see \ref{ex:chAwWwumnW} in §\ref{sec:redp.reciprocal} above) yields the verb \japhug{sɤwɯ\-wum}{gather}, whose meaning is close to that of the base verb \japhug{wum}{gather} (see example \ref{ex:aCtAtWwum} above and the related discussion), but lacking the extended meanings of this verb and more commonly used to express the meaning `gather' with human objects.

\begin{exe}
\ex \label{ex:chAsAwWwum}
\gll  srɯnmɯ ra cʰɤ-sɯ-ɤ-wɯ\redp{}wum tɕe, \\
 râkshasî \textsc{pl} \textsc{ifr}-\textsc{caus}-\textsc{recip}-gather \textsc{lnk} \\
\glt `She gathered the râkhasîs together.' (2011-05-nyima, 33)
\end{exe}

Such double derivations are by no means rare. The  causative+reciprocal \japhug{sɤtɯta}{separate} (of two persons that are fighting with each other, as in \ref{ex:YWsAtWtea}) is considerably more commonly used than the simple reciprocal \japhug{atɯta}{release each other} (from \japhug{ta}{put}). 

\begin{exe}
\ex \label{ex:YWsAtWtea}
\gll ɲɯ-sɯ-ɤ-tɯ\redp{}te-a ri tɕe mɤʑɯ ku-o-ndɯ\redp{}ndo-ndʑi ɕti \\
\textsc{ipfv}-\textsc{caus}-\textsc{recip}-put-\textsc{1sg} \textsc{lnk} \textsc{lnk} again \textsc{ipfv}:\textsc{east}-\textsc{recip}-take-\textsc{du} be.\textsc{aff}:\textsc{fact} \\
\glt `I am (repeatedly) separating them (two fighting ants), but they grab each other again (each time).' (conversation 14-05-01)
\end{exe}

The verb \japhug{asɤmɯmtsʰɯmtsʰɤm}{inform each other}, which presents two instances of the reciprocal derivation, is treated in (§\ref{sec:amW.reciprocal}).
 
\subsubsection{Lexicalized reciprocal} \label{sec:redp.lexicalized}
The meaning of reciprocal verbs is not always fully predictable from that of the base verb; some reciprocal verbs express naturally collective action (§\ref{sec:redp.reciprocal}) and may have a meaning that is more restricted than that of the base verb.

In some cases, the meanings of the reciprocal form has changed to such an extent that the two verbs have become synchronically unrelated. The clearest example is the intransitive verb \japhug{alɯlɤt}{fight}, which requires a non-singular subject and can select a comitative phrase (§\ref{sec:comitative}) as in (\ref{ex:talalWlAtndZi}), like regular reciprocal verbs (§\ref{sec:redp.reciprocal}). 

\begin{exe}
\ex \label{ex:talalWlAtndZi}
\gll ʁdɯxpa kɤrpu ɣɯ ɯ-tɕɯ cʰo aʑo a-tɕɯ nɯ tɤ-alɯlɤt-ndʑi tɕe, tɕe a-tɕɯ ɣɯ-sat pjɤ-ŋu ri, \\
\textsc{anthr}  \textsc{anthr} \textsc{gen} \textsc{3sg}.\textsc{poss}-son \textsc{comit} \textsc{1sg} \textsc{1sg}.\textsc{poss}-son \textsc{dem} \textsc{aor}-fight-\textsc{du}  \textsc{lnk} \textsc{lnk} \textsc{1sg}.\textsc{poss}-son \textsc{inv}-kill:\textsc{fact} \textsc{ifr}.\textsc{ipfv}-be \textsc{lnk} \\
\glt `Gdugpa dkarpo's son and my son fought with each other, and my son was about to be killed.' (28-smAnmi, 253)
\end{exe}

This verb originates from the transitive verb \japhug{lɤt}{release}, which is used as a light verb in several collocations related to fight. In these constructions, the syntactic object of \forme{lɤt} is the instrument used to hit or shoot, such as weapons (\japhug{scapa}{sword}, \japhug{tɯdi}{arrow}, \japhug{ɕɤmɯɣdɯ}{gun} etc) or body parts (\japhug{tɤŋkʰɯt}{fist} etc) as objects (\ref{ex:CAmWGdW.tulAtnW}, \ref{ex:qachGa.WtaR.tWdi.tolAt} and also \ref{ex:scoRqhu.tulAt} in §\ref{sec:WtaR}), while the semantic patient (the entity that is hit or shot at) is marked by the relator noun \japhug{ɯ-taʁ}{on} (§\ref{sec:WtaR}).

\begin{exe}
\ex \label{ex:CAmWGdW.tulAtnW}
\gll nɯnɯ ɕɤmɯɣdɯ tu-lɤt-nɯ tɕe pjɯ-sat-nɯ ɕti. \\
\textsc{dem} gun \textsc{ipfv}-release-\textsc{pl} \textsc{lnk} \textsc{ipfv}-kill-\textsc{pl} be.\textsc{aff}:\textsc{kill} \\
\glt `They shoot at it with guns and kill it.' (28-qapar, 20)
\end{exe}

\begin{exe}
\ex \label{ex:qachGa.WtaR.tWdi.tolAt}
\gll qacʰɣa ɯ-taʁ ʑo li tɯdi nɯ ci to-lɤt. \\
fox \textsc{3sg}.\textsc{poss}-on \textsc{emph} again arrow \textsc{dem} \textsc{indef} \textsc{ifr}-release \\
\glt `He too shot an arrow at the fox.' (140507 jinniao-zh, 82)
\end{exe}

The reciprocal derivation here originally expressed mutual action between the subject and the oblique argument (marked by \japhug{ɯ-taʁ}{on}) of the base verb; the original meaning may have been `shoot at/hit each other (with $X$)', and it can be surmised that the verb \japhug{alɯlɤt}{fight} used to require a semi-object corresponding to the instrument used for hitting/shooting at an earlier stage. The verb \forme{alɯlɤt} became fully lexicalized when it ceased to co-occur with a semi-object, and when its meaning became narrowed to the meaning `fight', as opposed to all the other possible meanings of the light verb \forme{lɤt} (§\ref{sec:lAt.lv}).
 
In some cases, the base verb does not exist anymore but its possible form can be easily recovered. For instance, the reciprocal \japhug{anɯrŋɤrɯru}{look at each other's face} (example \ref{ex:kanWrNArWrundZi}) is derived from a lost base verb $\dagger$\forme{nɯrŋɤru}, an incorporating verb made from the intransitive \japhug{ru}{look at} (§\ref{sec:orienting.verbs})  and the nominal root of \japhug{tɯ-rŋa}{face}, with the denominal prefix \forme{nɯ-} (§\ref{sec:incorp.other}).

\begin{exe}
\ex \label{ex:kanWrNArWrundZi}
\gll   nɯnɯ cʰo ci kɤ-anɯrŋɤrɯru-ndʑi tɕe \\
\textsc{dem} \textsc{comit} a.little \textsc{aor}-look.at.each.other-\textsc{du} \textsc{lnk} \\
\glt  `They exchanged a look with each other'. (2010-07, pear story)
\end{exe}

Finally, we find a few intransitive verbs that resemble reciprocal verbs formally (presence of \forme{a-} prefix and verb stem reduplication) and syntactically (non-singular subject, select comitative phrases), but whose verb root is not otherwise attested in Japhug: \japhug{amɯmi}{be in good terms with} (§\ref{sec:comitative}), \japhug{aʑɯʑu}{wrestle} and \japhug{asɯsu}{copulate} (the latter perhaps related to \japhug{sɯsu}{live}).

\subsection{Reciprocal \forme{amɯ-} prefix} \label{sec:amW.reciprocal}  
In addition to the reduplicated reciprocal (§\ref{sec:redp.reciprocal}), a second reciprocal pattern is attested in Japhug: the prefix  \forme{amɯ-}. As shown by the examples in \tabref{tab:amW.reciprocal}, the \forme{amɯ-} prefix can derive reciprocal verbs from ditransitive, transitive, semi-transitive and intransitive verbs. When prefixed to \forme{a-} initial verbs such as \japhug{atɯɣ}{meet}, no vowel contraction takes place (the reciprocal is \japhug{amɯtɯɣ}{meet each other} rather than $\dagger$\forme{amɤtɯɣ} as could have been expected).


\begin{table}
\caption{Examples of the \forme{amɯ-} reciprocal prefix} \label{tab:amW.reciprocal}
\begin{tabular}{lllllllllll}
\lsptoprule
&Base verb & Reciprocal verb \\
\midrule
indirective&\japhug{ti}{say} & \japhug{amɯti}{say to each other} \\
&\japhug{stʰaβ}{put against} & \japhug{amɯstʰaβ}{be one against the other} \\
&\japhug{rpu}{bump} & \japhug{amɯrpu}{bump against each other} \\
\midrule
mono-&\japhug{mto}{see} & \japhug{amɯmto}{see each other} \\
transitive&\japhug{mtsʰɤm}{hear} & \japhug{amɯmtsʰɤm}{hear each other} \\
\midrule
semi-&\japhug{tso}{know, understand} & \japhug{amɯtso}{understand each other} \\
transitive&\japhug{atɯɣ}{meet} & \japhug{amɯtɯɣ}{meet each other} \\
\midrule
intransitive&\japhug{fse}{be like} & \japhug{amɯfse}{know each other} \\
&\japhug{armbat}{be near} & \japhug{amɯrmbat}{be close to each other} \\
&\japhug{arqʰi}{be far} & \japhug{amɯrqʰi}{be far from each other} \\
\lspbottomrule
\end{tabular}
\end{table}

The verbs that are compatible with \forme{amɯ-} derivation can be divided into four groups: indirective, perception, semi-transitive and stative.

\subsubsection{\forme{amɯ-} reciprocalization of indirective verbs} \label{sec:recip.amW.indirective}
Indirective verbs (§\ref{sec:ditransitive.indirective}) such as \japhug{ti}{say} become semi-transitive (§\ref{sec:semi.transitive}) when subjected to the \forme{amɯ-} reciprocal derivation. 

The reciprocal verb \japhug{amɯti}{say to each other} is compatible with a semi-object (noun phrase or reported speech complement clause as in \ref{ex:tokAmWtindZi} and \ref{ex:amAtsa.tukAmWti}) corresponding to the object of the base verb. The reciprocal derivation here expresses mutual action between the subject and the dative recipient of \japhug{ti}{say}. 

\begin{exe}
\ex \label{ex:tokAmWtindZi}
\gll `nɯtɕu tɯ-ci z-ɲɯ-kɯ-sɯ-ɤ-j-tsʰi ɲɯ-ntsʰi' to-k-ɤmɯ-ti-ndʑi \\
\textsc{dem}:\textsc{loc} \textsc{indef}.\textsc{poss}-water \textsc{tral}-\textsc{ipfv}-\textsc{genr}:S/O-\textsc{caus}-\textsc{pass}-\textsc{caus}-drink \textsc{sens}-be.better \textsc{ifr}-\textsc{peg}-\textsc{recip}-say-\textsc{du} \\
\glt `They said to each other `we should go there and ask for water to drink.'' (Nyima wodzer 2002, 59)
\end{exe}

Example (\ref{ex:amAtsa.tukAmWti}) illustrates the use of this reciprocal verb with a generic person form, with neutralization of number marking (the verb otherwise always has non-singular number indexation in finite forms).\footnote{Concerning the kinship rule described in (\ref{ex:amAtsa.tukAmWti}), see §\ref{sec:ego.cousins}. }

\begin{exe}
\ex \label{ex:amAtsa.tukAmWti}
\gll nɯnɯ kɤndʑi-sqʰaj ɯ-rɟit nɯ tɕe ``a-mɤtsa" tu-kɯ-ɤmɯ-ti ɲɯ-ŋu. \\
\textsc{dem} \textsc{coll}-sister \textsc{3sg}.\textsc{poss}-offspring \textsc{dem} \textsc{lnk} \textsc{1sg}.\textsc{poss}-MZCh \textsc{ipfv}-\textsc{genr}:S/O-\textsc{recip}-say \textsc{sens}-be \\
\glt `Children of sisters call each other `my maternal parallel cousin'.' (140425 kWmdza4, 5)
\end{exe}

The reciprocal \forme{amɯ-} prefix expresses reciprocity between the subject and other oblique arguments,  for instance those marked by the relator noun \japhug{ɯ-taʁ}{on, above} (§\ref{sec:WtaR}). For instance, the verb \japhug{amɯstʰaβ}{be one against the other} (\ref{ex:kAmWsthWsthaB}) is derived from \japhug{stʰaβ}{put against}, a ditransitive verb which selects a phrase in \forme{ɯ-taʁ} (\ref{ex:WtaR.kuwGsthaB}).

\begin{exe}
\ex \label{ex:kAmWsthWsthaB}
\gll tɕe nɯnɯ li bɤβbɤβ ʑo kɯ-pa tɕe, kɯ-ɤmɯ-stʰɯ\redp{}stʰaβ ʑo kɯ-dɤn tu-ɬoʁ ŋu. \\
\textsc{lnk} \textsc{dem} again \textsc{idph}(II):growing.in.clumps \textsc{emph} \textsc{sbj}:\textsc{pcp}-\textsc{aux} \textsc{lnk} \textsc{sbj}:\textsc{pcp}-\textsc{recip}-\textsc{emph}\redp{}put.against \textsc{emph} \textsc{sbj}:\textsc{pcp}-be.many \textsc{ipfv}-come.out be:\textsc{fact} \\
\glt `(These mushrooms) grow in clumps, one against the other in great numbers.' (23-mbrAZim, 9)
\end{exe}

\begin{exe}
\ex \label{ex:WtaR.kuwGsthaB}
\gll ma nɯnɯ ɯʑo ʁɟa [tɯ-mdʑu ɯ-taʁ] kú-wɣ-stʰaβ tɕe mɤrtsaβ, \\
 \textsc{lnk} \textsc{dem} \textsc{3sg} completely \textsc{genr}.\textsc{poss}-tongue \textsc{3sg}.\textsc{poss}-on \textsc{ipfv}-\textsc{inv}-put.against \textsc{lnk} be.spicy:\textsc{fact} \\
\glt `If one puts it (this plant) on one's tongue (without anything else), it is spicy.' (13-tCamu, 13)
\end{exe}

The verb \japhug{amɯrpu}{bump one against the other} is a similar case (see example \ref{ex:rNgW.tokAmWrpundZici}, §\ref{sec:pers.pronouns}), but its base verb \japhug{rpu}{bump against} is labile (§\ref{sec:goal.labile}).

\subsubsection{\forme{amɯ-} reciprocalization of perception verbs}
The perception verbs \japhug{mto}{see}, \japhug{mtsʰɤm}{hear} are also compatible with the \forme{amɯ-} prefix. Unlike the indirective verbs discussed in §\ref{sec:recip.amW.indirective} their reciprocal forms  \japhug{amɯmto}{see each other} and \japhug{amɯmtsʰɤm}{hear from each other}  (or `hear each other') express reciprocal action between the subject (experiencer) and the object (stimulus) of the base verb. As shown by (\ref{ex:pjAmWmtondZi}), these verbs also select the comitative like other reciprocal forms.

\begin{exe}
\ex \label{ex:pjAmWmtondZi}
\gll <liangshanbo> cʰondɤre pjɯ-ɤmɯ-mto-ndʑi mɯ-pjɤ-jɤɣ tɕe \\
\textsc{anthr} \textsc{comit} \textsc{ipfv}-\textsc{recip}-see-\textsc{du} \textsc{neg}-\textsc{ifr}.\textsc{ipfv}-be.allowed \textsc{lnk} \\
\glt `She and Liang Shanbo were not allowed to see each other.' (150826 liangshanbo zhuyingtai-zh, 164)
\end{exe}


\subsubsection{\forme{amɯ-} reciprocalization of semi-transitive verbs}
In the case of the semi-transitive verbs \japhug{tso}{know, understand} and \japhug{atɯɣ}{meet}, the \forme{amɯ-} derivation targets the semi-object.

The verb \forme{amɯtso} can be used with the reciprocal meaning of `understand each other' (\ref{ex:pjWkAmWtso.YWra}), but also has an additional meaning `be clear, be understandable (of speech)', reflecting the homophonous \forme{amɯ-} distributed property derivation (§\ref{sec:distributed.amW}). As discussed in §\ref{sec:distributed.amW}, the reciprocal and distributed property \forme{amɯ-} prefixes are historically related, and the verb \forme{amɯtso} is one of the pivot forms between them.

\begin{exe}
\ex \label{ex:pjWkAmWtso.YWra}
 \gll kupa-skɤt tú-wɣ-βzu tɕe nɯnɯ a-pɯ-ŋu, iʑo ɣɯ <guoyu> ɲɯ-ŋu tɕe, nɯnɯ kɤsɯfse ɣɯ ji-rju ɲɯ-ŋu tɕe, pjɯ-kɯ-ɤmɯ-tso ɲɯ-ra ri, li nɯ koŋla mɯ́j-tso-nɯ. \\
 Chinese-language \textsc{ipfv}-\textsc{inv}-make \textsc{lnk} \textsc{dem} \textsc{irr}-\textsc{ipfv}-be \textsc{1sg} \textsc{gen} national.language \textsc{sens}-be \textsc{lnk} \textsc{dem} all \textsc{gen} \textsc{1sg}.\textsc{poss}-speech \textsc{sens}-be \textsc{lnk} \textsc{ipfv}-\textsc{genr}:S/O-\textsc{recip}-understand \textsc{sens}-be.needed \textsc{lnk} again \textsc{dem} completely \textsc{neg}.\textsc{sens}-understand-\textsc{pl} \\
\glt `When we speak Chinese, since it is our national language, everybody's language, we should be able to understand each other, but (the people from Tshobdun) do not understand it completely either.' (150901 tshuBdWnskAt, 15-17)
\end{exe}

\subsubsection{\forme{amɯ-} reciprocalization of stative verbs} \label{sec:amW.stative}
The \forme{amɯ-} prefix occurs with a few stative verbs. The stative verbs of relative location \japhug{armbat}{be near} and  \japhug{arqʰi}{be far} have the reciprocal forms  \japhug{amɯrmbat}{be close to each other} and \japhug{amɯrqʰi}{be far from each other}. In (\ref{ex:YAmWrmbat}), \forme{amɯrmbat} appears in singular form due to the inanimate character of the stars and their poor differentiability from each other with a naked eye.

\begin{exe}
\ex \label{ex:YAmWrmbat}
\gll ʑŋgri wuma kɯ-tʂot ɲɯ-maʁ ri, kɯ-ɤmɯ-rmbɯ\redp{}rmbat ʑo ɲɯ-ŋu tɕe, nɯ wuma ɲɯ-saχsɤl. [...] ɯ-tɯ-ɤʑirja ɯ-tsʰɯɣa nɯ, [...] ʁnɤmchi tsa ɲɯ-fse, [...] ri ɲɯ-ɤmɯ-rmbat. \\
star really  \textsc{sbj}:\textsc{pcp}-be.bright \textsc{sens}-not.be \textsc{lnk} \textsc{sbj}:\textsc{pcp}-\textsc{recip}-\textsc{emph}\redp{}be.near \textsc{emph} be:\textsc{fact} \textsc{lnk} \textsc{dem} really \textsc{sens}-be.clear { } \textsc{3sg}.\textsc{poss}-\textsc{nmlz}:\textsc{deg}-be.aligned \textsc{3sg}.\textsc{poss}-shape \textsc{dem} {  } gnam.khyi a.little be.like:\textsc{fact} {  } \textsc{lnk} \textsc{sens}-\textsc{recip}-be.near \\
\glt `(There six stars in the Pleiades.) They are not very bright stars, but they are close to each other, and for this reason they are quite visible. The way they are aligned is a bit similar to that of the constellation Gnam.khyi, but closer to each other.' (29-mWBZi, 15-20)
\end{exe}

The verb \japhug{amɯfse}{know each other} (example \ref{ex:kamWfsetCi.pamWmitCi}) historically originates from the \forme{amɯ-} reciprocal form of a base \forme{-fse}, which is not related to the intransitive stative verb \japhug{fse}{be like}, but rather to a lost verb corresponding to the Tshobdun verb \forme{fsɛʔ} `hear' \citep[213]{jackson19tshobdun}. The isolated derivation \japhug{nɯfse}{know} (a person) (possibly a lexicalized autive §\ref{sec:autoben.lexicalized}) is derived from the same base with the same meaning; it also has a Tshobdun cognate: \forme{nə́fsɛ} `know well' \citep[121]{jackson19tshobdun}.

\begin{exe}
\ex \label{ex:kamWfsetCi.pamWmitCi}
\gll  nɯre ri tɕe kɤ-amɯfse-tɕi tɕe, tɕendɤre wuma ʑo pɯ-amɯmi-tɕi \\
\textsc{dem}:\textsc{loc} \textsc{loc} \textsc{loc} \textsc{aor}-know.each.other-\textsc{1du} \textsc{lnk} \textsc{lnk} really \textsc{emph} \textsc{pst}.\textsc{ipfv}-be.in.good.terms-\textsc{1du} \\
\glt  `We got to know each other there, and we were in very good terms.' (12-BzaNsa, 6)
\end{exe}

Some reciprocal verbs derived either from dynamic transitive verbs (\japhug{amɯstʰaβ}{be one against the other} in \ref{ex:kAmWsthWsthaB}) or from stative verbs (\japhug{amɯrmbat}{be close to each other} in \ref{ex:YAmWrmbat}) are stative, and often appear with emphatic reduplication. This reduplication is different from that of reduplicated reciprocal verbs (§\ref{sec:redp.reciprocal}), which is one of the morphological exponents of that reciprocal formation.

\subsubsection{Causativivization of \forme{amɯ-} reciprocal verbs} \label{sec:sAmW}
The sigmatic causative can be added to \forme{amɯ-} reciprocal verbs. For instance, \japhug{amɯmto}{see each other} discussed above yields the causative form \japhug{sɤmɯmto}{cause to see each other}, which also selects the comitative like its base verb,\footnote{The Chinese original passage from which (\ref{ex:cho.pjAsAmWmto}) is translated is \ch{老翁便领着那两个少男女出来与耿去病见面}{lǎowēng biàn lǐngzhe liǎngge shàonánnǚ chūlái yǔ Gěng Qùbìng jiànmiàn}{The old man then brought the girl and the boy to meet with Geng Qubing}, and the presence of comitative could in principle be an effect of calquing of the preposition \ch{与}{yǔ}{with}, but additional elicitation has confirmed that this construction is grammatically correct. }
 as shown by (\ref{ex:cho.pjAsAmWmto}).\footnote{The absence of plural indexation in (\ref{ex:cho.pjAsAmWmto}) is expected since in direct 3\flobv{} forms only the number of the subject is indexed on the verb (§\ref{sec:indexation.non.local}). }


\begin{exe}
\ex \label{ex:cho.pjAsAmWmto}
\gll jo-sɯ-ɣe tɕe iɕqʰa <gengqubing> nɯ cʰo pjɤ-sɯ-ɤmɯ-mto.  \\
\textsc{ifr}-\textsc{caus}-come \textsc{lnk} the.aforementioned  \textsc{anthr} \textsc{dem} \textsc{comit} \textsc{ifr}-\textsc{caus}-\textsc{recip}-see \\
\glt `The (old man) brought them in and had them meet Geng Qubing.' (150906 qingfeng-zh, 58)
\end{exe}

The verb \japhug{asɤmɯmtsʰɯmtsʰɤm}{inform each other} underwent three derivations from the base verb \japhug{mtsʰɤm}{hear}: \forme{amɯ-} reciprocal derivation (\japhug{amɯmtsʰɤm}{hear from each other}), sigmatic causative (\japhug{sɤmɯmtsʰɤm}{cause to hear from each other}) and then finally the reduplicated reciprocal derivation (\forme{a-sɤmɯmtsʰɯ\redp{}mtsʰɤm}, §\ref{sec:reciprocal.other}). Unlike the non-volitional reciprocal verb \japhug{amɯmtsʰɤm}{hear from each other} from which it is derived, \forme{asɤmɯmtsʰɯmtsʰɤm} is a verb of speech, and expresses a volitional action (see \ref{ex:ZnasAmWmtshWmtshAmnW} below and \ref{ex:Wzda.tWrdoR} in §\ref{sec:possessive.prefix.obv.def}). 

\begin{exe}
\ex \label{ex:ZnasAmWmtshWmtshAmnW}
\gll ɯʑo kɯ ɯ-zda tɯ-rdoʁ ɯ-pʰe ta-tɯt, tɯ-zda kɯ li ci ɯ-zda nɯ ɯ-pʰe kɯ-fse ɕ-ta-tɯt nɤ,
[...] ɣurʑa kɯrcat nɯ ʑ-nɯ-a-sɯ-ɤmɯ-mtsʰɯ\redp{}mtsʰɤm-nɯ ɲɯ-ŋu, \\
\textsc{3sg} \textsc{erg} \textsc{3sg}.\textsc{poss}-companion one-piece \textsc{3sg}.\textsc{poss}-\textsc{dat} \textsc{aor}:3\flobv{}-say \textsc{indef}.\textsc{poss}-companion \textsc{erg} again \textsc{indef} \textsc{3sg}.\textsc{poss}-companion \textsc{dem} \textsc{3sg}.\textsc{poss}-\textsc{dat}  \textsc{sbj}:\textsc{pcp}-be.like \textsc{tral}-\textsc{aor}:3\flobv{}-say \textsc{add} { } hundred eight \textsc{dem} \textsc{tral}-\textsc{aor}-\textsc{recip}-\textsc{caus}-\textsc{recip}-hear-\textsc{pl} \textsc{sens}-be \\
\glt `The boy told one of his companions, and that one went and told another one, and all one hundred and eight (boys) went and informed each other.' (2005 Norbzang, 89-90)
\end{exe}

The first two clauses in (\ref{ex:ZnasAmWmtshWmtshAmnW}) provide a native gloss on the meaning of this reciprocal verb.


 \subsection{Reciprocal \forme{andʑɯ-} prefix} \label{sec:andZW.reciprocal}
The transitive verb \japhug{βri}{protect} (an irregular causative of  \japhug{ri}{remain}, see §\ref{sec:causative.m}), has the reciprocal \japhug{andʑɯβri}{protect each other} with the unique \forme{andʑɯ-} prefix, historically related to the denominal \forme{andʑi-} (§\ref{sec:denom.andZi}; on the difficult question of the \ipa{i} vs. \ipa{ɯ} contrast in this context, see §\ref{sec:W.i.contrast}).

\begin{exe}
\ex \label{ex:andWBritCi}
\gll tɕiʑo andʑɯ-βri-tɕi ra \\
\textsc{1du} \textsc{recip}-protect:\textsc{fact}-\textsc{1du} be.needed:\textsc{fact} \\
\glt `The two of us have to look out for each other.' (elicited)
\end{exe}

\subsection{Verbs of co-participation} \label{sec:co.participation} 
The compound verb \japhug{amɟɤkʰo}{give and take}, which derives from the transitive verbs \japhug{mɟa}{take} and \japhug{kʰo}{give} (§\ref{sec:antipassive.t}, §\ref{sec:ditransitive.indirective}) is not formally reciprocal but implies an action performed by more than one person. Contrary to a reciprocal or a reflexive, this collective action is not mutual or directed towards oneself: rather, it expresses that two distinct actions performed by different referents take place (near-)simultaneously and are linked with one another. Example (\ref{ex:YAmJAkhondZi}) illustrates that one of the two people referred to by the third dual subject hands over the child (a semi-object, §\ref{sec:semi.object}) and that the other person takes the child from her hands. Note the non-iconic order in the compound, where  the root \forme{mɟa} occurs before \forme{kʰo}, also the action of giving necessarily temporally precedes that of taking.

\begin{exe}
\ex \label{ex:YAmJAkhondZi}
\gll  tɕeri tɤ-rɟit nɯ ɲɯ-ɤmɟɤkʰo-ndʑi qʰe koŋla tu-o-nɯ-rɯɕmɯ\redp{}ɕmi-ndʑi kɯnɤ mɯ́j-tsu ma tɕe li tú-wɣ-nɯ-tsɯm ɲɯ-ɕti. \\
\textsc{lnk} \textsc{indef}.\textsc{poss}-child \textsc{dem} \textsc{ipfv}-give.and.take-\textsc{du} \textsc{lnk} really \textsc{ipfv}-\textsc{recip}-talk-\textsc{du} also \textsc{neg}:\textsc{sens}-have.time.to \textsc{lnk} \textsc{lnk} again \textsc{ipfv}:\textsc{up}-\textsc{inv}-\textsc{vert}-take.away \textsc{sens}-be:\textsc{aff} \\
\glt `She$_i$ hands the child$_j$ to him$_k$ and he$_k$ takes him$_j$, but they$_{i+k}$ don't get the time to exchange any words and she$_i$ is taken back to heaven.' (150828 donglang, 174-175)
\end{exe}

Despite the presence of a reciprocal verb \forme{a-rɯɕmɯ\redp{}ɕmi} `talk to each other' (§\ref{sec:redp.reciprocal}) in this passage, since there is no exchange of roles in the compound action described by the verb \japhug{amɟɤkʰo}{give and take}, it is preferrable to refer to this type of construction as `co-participation' (more precisely, `unspecified co-participation' in Creissels and Voisin's \citeyear{creissels08coparticipation} terminology). This verb is isolated, as none of the other compound verbs recorded up to now have a meaning of this type.

\section{Anticausative} \label{sec:anticausative}

\subsection{Morphology} \label{sec:anticausative.morphology}  


\subsubsection{Prenasalized-unvoiced alternation} \label{sec:anticausative.pairs}  
Voice derivations in Japhug are mainly concatenative and prefixal. An important exception is the alternation between unvoiced stops/affricates and their voiced prenasalized counterparts (which are to be analyzed as single phonemes, §\ref{sec:consonant.phonemes}), reflected in verb pairs whose unvoiced member is transitive, and whose voiced prenasalized member is intransitive.
 

\begin{table}
\caption{Prenasalized anticausative verbs from unaspirated roots (21 examples)}\label{tab:anticausative.unaspirated}
\begin{tabular}{llll} 
\lsptoprule
transitive verb  & intransitive  verb &\\
\midrule
\japhug{plɯt}{destroy} & \japhug{mblɯt}{be destroyed} \\
\japhug{prɤt}{break} (vt, of thread) & \japhug{mbrɤt}{break} (vi) \\
\japhug{pri}{tear} & \japhug{mbri}{be torn} \\
\japhug{pɣaʁ}{turn over} (vt) & \japhug{mbɣaʁ}{turn over} (vi) \\
\midrule
\japhug{χtɤr}{scatter} & \japhug{ʁndɤr}{be scattered} \\
\midrule
\japhug{tɕɤβ}{burn} (vt) & \japhug{ndʑɤβ}{be burned} \\
\japhug{tɕɣaʁ}{squeeze out} & \japhug{ndʑɣaʁ}{be squeezed out} \\ 
\japhug{tʂaβ}{cause to fall/roll} & \japhug{ndʐaβ}{fall/roll} (vi) \\
\japhug{ftʂi}{melt} (vt) & \japhug{ndʐi}{melt} (vi) \\
\midrule
\japhug{cɯ}{open} (vt) & \japhug{ɲɟɯ}{open} (vi) \\ 
\midrule
\japhug{kɤɣ}{bend} & \japhug{ŋgɤɣ}{be bent} \\ 
\japhug{kio}{cause to glide} & \japhug{ŋgio}{slip}, `glide' \\
\japhug{kra}{cause to fall} & \japhug{ŋgra}{fall} \\
\midrule
\japhug{qaʁ}{peel off} (vt) & \japhug{ɴɢaʁ}{peel off} (vi)  \\
\japhug{qɤt}{separate} (vt) & \japhug{nɯɴɢɤt}{part ways} \\
\japhug{qia}{tear down} & \japhug{ɴɢia}{come loose} \\
\japhug{qlɯt}{break} (vt, of long objects) & \japhug{ɴɢlɯt}{break} (vi) \\
\japhug{qraʁ}{tear} & \japhug{ɴɢraʁ}{be torn} \\
\japhug{qrɤz}{shave} & \japhug{ɴɢrɤz}{break} (vi, of hair, dry leaves etc) \\ 
\japhug{qrɯ}{break} (vt, of hard objects) & \japhug{ɴɢrɯ}{break} (vi) \\
\midrule 
\japhug{sar}{filter out} (vt) & \japhug{ndzar}{drip dry} (vi) \\
 \lspbottomrule
\end{tabular}
\end{table}

\begin{table}
\caption{Prenasalized anticausative verbs from aspirated roots (8 examples)}\label{tab:anticausative.aspirated}
\begin{tabular}{llll} 
\lsptoprule
transitive verb  & intransitive  verb &\\
\midrule
\japhug{pʰaʁ}{split} (vt) & \japhug{mbaʁ}{split, break} (vi)	 	\\
\japhug{ɯ-ʁo+pʰi}{be disappointed by} & \japhug{ɯ-ʁo+mbi}{be discouraged}  	\\
\japhug{sɤpʰɤr}{wipe off}	&		\japhug{mbɤr}{be wiped off}	 	\\
\japhug{tʰɯ}{built} (road, bridge)	&	\japhug{ndɯ}{be spread} (road, bridge)		\\
\japhug{xtʰom}{put horizontally} & \japhug{ndom}{lie horizontally}  \\
\japhug{tsʰoʁ}{attach} & \japhug{ndzoʁ}{be attached}  \\
\japhug{cʰɤβ}{flatten, crush} & \japhug{ɲɟɤβ}{be crushed, flattened} 	 	\\ 
\japhug{qʰrɯt}{completely scratch}	& \japhug{ɴɢrɯt}{be completely scratched}		\\
 \lspbottomrule
\end{tabular}
\end{table}

There are 29 known examples of this alternation, involving both unvoiced unaspirated obstruents (\tabref{tab:anticausative.unaspirated}) and aspirated stops/affricates (\tabref{tab:anticausative.aspirated}).

In \tabref{tab:anticausative.unaspirated}, the verb \japhug{nɯɴɢɤt}{part ways} has a lexicalized autive \forme{nɯ-} integrated in the verb stem (§\ref{sec:autoben.lexicalized}), but the bare stem \forme{ɴɢɤt} is found in nominalized forms such as \japhug{ɯ-sɤ-ɴɢɤt}{place where X part ways} (§\ref{sec:determinative.n.n}).\footnote{The verb pairs \japhug{xtʰom}{put horizontally} / \japhug{ndom}{lie horizontally} and  \japhug{ftʂi}{melt} (vt) / \japhug{ndʐi}{melt}, which have a cluster in the transitive form but a single prenasalized stop in its intransitive counterpart, are discussed in §\ref{sec:anticausative.fx}. }
 
 
In the absence of a clearly identifiable derivational affix, the direction of the derivation is not completely obvious. It is conceivable in principle that the intransitive verbs in Tables \ref{tab:anticausative.unaspirated} and \ref{tab:anticausative.aspirated} are derived from their transitive counterparts, but the opposite direction is equally possible. 

The latter direction could even seem more likely when looking at the meaning of some of the transitive verbs in these tables from a West European-cum-Chinese perspective: for instance, the meaning of \japhug{kra}{cause to fall} has to be glossed in a way that makes it seem like it is derived from the intransitive verb \japhug{ŋgra}{fall}.

A cognate phenomenon is well-known in other branches of the Trans-Hima\-la\-yan family such as Old Chinese, Tibetan and Lolo-Burmese, and several traditions of research analyse these cases as devoicing of the voiced initial by the sigmatic causative prefix (for instance \citealt{shefts1971causative, daiqx94biguan, gerner07caus}) while other scholars argue for the opposite direction (\citealt{sagart12sprefix, jacques12internal}; see a summary of several opinions on this matter in \citealt{handel12valence}). 


\subsubsection{Evidence for the directionality of the anticausative derivation} \label{sec:anticausative.direction}
In Japhug (and other Gyalrong languages), three independent pieces of evidence clearly indicate that the direction of derivation must be from the transitive verb to the intransitive one.

First, the transitive verbs in these pairs can have either unaspirated onset (see the examples in \tabref{tab:anticausative.unaspirated}), an aspirated onset (\tabref{tab:anticausative.aspirated}) or even a fricative onset (the pair \japhug{sar}{filter out} vs. \japhug{ndzar}{drip dry}). In the hypothesis that the intransitive verbs derive from their transitive counterpart, this observation can be trivially explained: the aspiration contrast is neutralized by the prenasalization, as illustrated in \tabref{tab:prenasalization} (some of the aspirated affricates are indicated in brackets in this table, as no examples are attested).

\begin{table}
\caption{Prenasalization and aspiration neutralization}\label{tab:prenasalization}
\begin{tabular}{llll} 
\lsptoprule
\forme{p\trt}, \forme{pʰ\trt} \fl{} \forme{mb-} \\
\forme{t\trt}, \forme{tʰ\trt} \fl{} \forme{nd-} \\
\forme{ts\trt}, \forme{tsʰ\trt}, \forme{s\trt} \fl{} \forme{ndz-} \\
\forme{tɕ\trt}, (\forme{tɕʰ\trt}) \fl{} \forme{ndʑ-} \\
\forme{tʂ\trt}, (\forme{tʂʰ\trt}) \fl{} \forme{ndʐ-} \\
\forme{c\trt}, \forme{cʰ\trt} \fl{} \forme{ɲɟ-} \\
\forme{k\trt}, \forme{kʰ\trt} \fl{} \forme{ŋg-} \\
\lspbottomrule
\end{tabular}
\end{table}

On the other hand, in the hypothesis that the intransitive verbs are primary, the origin of aspiration contrast on the transitive counterparts requires an additional set of explanations.

Second, the verb \japhug{χtɤr}{scatter} (\tabref{tab:anticausative.unaspirated}) is borrowed from Tibetan \tibet{གཏོར་}{gtor}{scatter}. The prenasalized form \japhug{ʁndɤr}{be scattered} has no Tibetan equivalent, and its onset \forme{ʁnd-} is incompatible with the phonotactics of Tibetan consonant clusters. Thus, this intransitive verb must be a Gyalrong-internal creation from a Tibetan base,\footnote{Cognate pairs also exist in Zbu (\forme{χtór} / \forme{ʁⁿdór}, \citealt[271]{gong18these}) and in Tshobdun (\forme{χtor} / \forme{ʁⁿdor}, \citealt[345; 241]{jackson19tshobdun}), showing that this derivation goes back at least to the common ancestor of these three languages.  } and it follows that the direction of derivation should be from the transitive verb to the intransitive one.  

Third, all scholars favouring the hypothesis that the intransitive verb is primary suppose that the onset of transitive verbs has been devoiced by the addition of a sigmatic causative. In Japhug, this hypothesis makes no sense, because the sigmatic causative (§\ref{sec:sig.causative}) is not only attested but fully productive (§\ref{sec:sig.caus.allomorphs}), with a considerable number of allomorphs but without ever devoicing either sonorant nor obstruents onsets.\footnote{The same is incidentally true of various other languages of the Trans-Himalayan family, including Tibetan \citep{jacques12internal,hill14voicing} and Jinghpo \citep[78]{dai92yufa}, where anticausative derivation also exists. } In addition, causativization of prenasalized verbs is attested in Japhug (§\ref{sec:anticausative.other.derivations}) and other Gyalrong languages such as Tshobdun \citep{jackson14morpho}, for instance \forme{sɯɣ-ndʐi} `melt (vt)' from \japhug{ndʐi}{melt} (vi) (compare with the transitive \japhug{ftʂi}{melt} vt).

Since the intransitive verbs in Tables \ref{tab:anticausative.unaspirated} and \ref{tab:anticausative.aspirated} have a non-volitional meaning (§\ref{sec:anticausative.function}), it is likely that the prenasalization is a fossilized form of the autive \forme{nɯ-} prefix (§\ref{sec:autoben.historical}) in its `spontaneous event' function, like the isolated case of prenasalization in the verb pair \japhug{sqlɯm}{collapse} vs. \japhug{arɴɢlɯm}{be caved in} (§\ref{sec:fossil.prenasalization}).

While traces of the voicing (prenasalization) alternation can be brought to light in most languages of the Trans-Himalayan family, Japhug and the other Gyalrong languages are the only branch of the family where the origin of this alternation is still visible. The study of the prenasalization derivation in Japhug is thus of considerable interest for comparative Trans-Himalayan.\footnote{Independent evidence against the hypothesis that the voicing alternation originates from sigmatic prefixation is also found in Tibetan \citep{jacques20alternation}. Further evidence in Gyalrongic is provided by \citet{gates22voicing}. }


\subsubsection{Absence of clusters in the anticausative form} \label{sec:anticausative.fx}
Among the pairs in §\ref{sec:anticausative.pairs}, two verbs stand out in having a preinitial consonant in the transitive form without equivalent in the intransitive one: \japhug{ftʂi}{melt} (vt), with a \forme{f-} (phonologically \ipa{w}) prefixal element (the expected form of the intransitive \japhug{ndʐi}{melt} would be $\dagger$\forme{mdʐi}) and \japhug{xtʰom}{put horizontally} with a \forme{x-} element (the expected form of \japhug{ndom}{lie horizontally} would be $\dagger$\forme{ɣndom}). 

No decisive explanation can be provided to account for this idiosyncrasy, found in other Gyalrongic languages including Tangut. Two mutually incompatible hypotheses can be considered. First, it is possible that in these two pairs both the intransitive and the transitive verbs are derived from a common root with different fossil derivational prefixes. Second, the reconstructed nasal prefix responsible for the anticausative prenasalization might have caused cluster simplification (\forme{*N-ptri} $\rightarrow$ \forme{*N-tri} $\rightarrow$ \forme{ndʐi}).

In the second hypothesis, the anticausative form \japhug{ʁndɤr}{be scattered} from the Tibetan loanword \japhug{χtɤr}{scatter} would be phonetically irregular, possibly a clue of it being analogically created on the basis of other anticausative derivations.

\subsection{Function} \label{sec:anticausative.function}
The discussion in the previous section has shown that the transitivity alternation exhibited by the verb pairs in Tables \ref{tab:anticausative.unaspirated} and \ref{tab:anticausative.aspirated} was a valency-decreasing derivation, turning a transitive verb with unvoiced obstruent onset into an intransitive verb with voiced prenasalized onset (following the rules in \tabref{tab:prenasalization}).

Like the passive (§\ref{sec:passive}), the only argument of the prenasalized intransitive verb corresponds to the object of the base verb. Unlike the passive derivation however, prenasalized intransitive verbs have a dynamic meaning (rather than expressing a resultative state) and also imply that the action took place spontaneously, semantically removing the agent. For instance, while the passive \forme{a-prɤt} of the verb \japhug{prɤt}{break} (of a thread) implies the existence of an agent (\ref{ex:pjAkAprAtci}), the prenasalized intransitive \forme{mbrɤt} expresses a spontaneous action without external agent (\ref{ex:nWmbrAt.pWmbrAt}).\footnote{The anticausative meaning of the prenasalization derivation is a plot device in (\ref{ex:nWmbrAt.pWmbrAt}). The context of this sentence is that the queen arrives in a room whose floor is tiled with turquoise and coral (see example \ref{ex:WBrAsqlWm}, §\ref{sec:fossil.prenasalization}), and unsure whether it is safe to walk on it (worrying that it might yield under her weight), she willfully breaks her necklace, spreading the pearls on the floor. Her servants, unaware that she did it on purpose, enter the room first to pick up the pearls, and seeing that the floor does not collapse, she then follows them. } For this reason, this derivation is henceforth referred to as `anticausative'.
 

\begin{exe}
\ex \label{ex:pjAkAprAtci}
\gll pjɤ-k-ɤ-prɤt-ci \\
\textsc{ifr}-\textsc{peg}-\textsc{pass}-break-\textsc{peg} \\
\glt `It has been broken (by someone).' 
\end{exe}

\begin{exe}
\ex \label{ex:nWmbrAt.pWmbrAt}
\gll  wo a-ʑi ra nɯ-mkɤɣɯr pɯ-mbrɤt \\
\textsc{interj} \textsc{1sg}.\textsc{poss}-lady \textsc{pl} \textsc{3pl}.\textsc{poss}-necklace \textsc{aor}-\textsc{acaus}:break \\
\glt `My lady, your necklace broke!' (2003 Kunbzang, 255)
\end{exe}
 
 %{ex:WZo.tonWYJW} {sec:autoben.spontaneous} to-nɯ-ɲɟɯ
While anticausativized verbs are not compatible with external volitional agents, they are however attested with an explicit expression of the cause and/or of an involuntary and indirect agent. For instance in (\ref{ex:ndZimbrW.chAmbGAR}) the capsizing of the ship (expressed by the anticausative \japhug{mbɣaʁ}{turn over} (vi) from \japhug{pɣaʁ}{turn over} (vt)) is due to a storm mentioned in the previous clause, however without explicit marking of the causal relationship.  

\begin{exe}
\ex \label{ex:ndZimbrW.chAmbGAR}
\gll   ndʑi-ʑmbrɯ cʰɤ-mbɣaʁ. \\
 \textsc{3du}.\textsc{poss}-boat \textsc{ifr}:\textsc{downstream}-\textsc{acaus}:turn.over \\
\glt `(One day, there was a terrible storm  on the ocean, and) their boat capsized.' (140511 xinbada-zh, 18)
\end{exe}

In (\ref{ex:pjANWClWG.pjANGrW}), the anticausative \japhug{ɴɢrɯ}{break} (vi) occurs even though the human subject of the preceding clause is the identified agent of the verb \forme{pjɤ-nɯ-ɕlɯɣ} `she dropped it' and the involuntary indirect cause of the breaking action.

\begin{exe}
\ex \label{ex:pjANWClWG.pjANGrW}
\gll popo pjɤ-nɯ-ɕlɯɣ tɕe, pjɤ-ɴɢrɯ. \\
earthenware \textsc{ifr}-\textsc{auto}-\textsc{drop} \textsc{lnk} \textsc{ifr}-\textsc{acaus}:break \\
\glt `She dropped the earthenware and it broke.' (2003gesar, 329)
\end{exe}

Anticausatives verbs can also follow their corresponding base transitive verbs, as illustrated by the pair \japhug{qlɯt}{break} (of long objects, vt) and \japhug{ɴɢlɯt}{break} (vi) in (\ref{ex:pjWtWqlWt.pjWNGlWt}). Instead of non-volitional action, what the anticausative \forme{ɴɢlɯt}  expresses in this case is the successful realization of the action: the subject of \forme{qlɯt} controls his decision to \textit{attempt} at breaking an object, but cannot control his success in performing this action.

\begin{exe}
\ex \label{ex:pjWtWqlWt.pjWNGlWt} 
\gll pjɯ-tɯ-qlɯt qʰe pjɯ-ɴɢlɯt ɲɯ-ɕti. \\
\textsc{ipfv}-\textsc{conv}:\textsc{imm}-break \textsc{lnk} \textsc{ipfv}-\textsc{acaus}:break \textsc{sens}-be.\textsc{aff} \\
\glt `(Twigs of willow that grow in lower altitude) break as soon as one breaks it.' (07-Zmbri, 6)
\end{exe}

\subsection{Anticausative and dummy subject constructions} \label{sec:anticausative.dummy}
The verb \forme{tsʰoʁ} can be used as a prototypical transitive verb with the meaning `attach, plant', as in (\ref{ex:CtutshoRnW}) (see also \ref{ex:junWqambWmbjomndZi.junACWCendZi}, §\ref{sec:distributed.action}) or `fix' (something on something else).

\begin{exe}
\ex \label{ex:CtutshoRnW}
\gll  fsaŋ ɕ-pjɯ-ta-nɯ ŋu. tɕe loŋrta ra ɕ-tu-tsʰoʁ-nɯ \\
fumigation   \textsc{tral}-\textsc{ipfv}-put-\textsc{pl} be:\textsc{fact} \textsc{lnk} prayer.flag \textsc{pl} \textsc{tral}-\textsc{ipfv}-attach-\textsc{pl} \\
\glt `(In the morning of the first day of the year) People go (there) and make fumigation, plant prayer flags...' (140522 Kamnyu zgo, 308-309)
\end{exe}

It is also one of the few transitive verbs to occur in the dummy subject construction (§\ref{sec:transitive.dummy}) in the meaning `grow' (of fruits, leaves and flowers), as in (\ref{ex:kWBdecAB.kutshoR}).

\begin{exe}
\ex \label{ex:kWBdecAB.kutshoR}
\gll tɕe tɯ-kʰɤl nɯtɕu, χsɯ-cɤβ, kɯβde-cɤβ jamar ku-tsʰoʁ \\
\textsc{lnk} one-place \textsc{dem}:\textsc{loc} three-pod four-pod about \textsc{ipfv}-attach \\
\glt `In each place (in each section on the stalk of the plant), three or four pods grow.' (09-stoR, 42)
\end{exe} 

The anticausative \japhug{ndzoʁ}{be attached} occurs with exactly the same meaning as \japhug{tsʰoʁ}{attach} in the dummy subject construction. In (\ref{ex:XsWcAB.kundzoR}), the intransitive subject \forme{ʁnɯ-cɤβ, χsɯ-cɤβ} `two or three pods' of \forme{ndzoʁ} corresponds to the object \forme{χsɯ-cɤβ, kɯβde-cɤβ} of \forme{tsʰoʁ} in (\ref{ex:kWBdecAB.kutshoR}).

\begin{exe}
\ex \label{ex:XsWcAB.kundzoR}
\gll tɯ-kʰɤl ri, ʁnɯ-cɤβ, χsɯ-cɤβ jamar ku-ndzoʁ cʰa \\
one-place \textsc{lnk} two-pod  three-pod about \textsc{ipfv}-\textsc{acaus}:attach can:\textsc{fact} \\
\glt `In each place (section on its stalk), two or three pods can grow.' (09-stoR, 35)
\end{exe}

This is not the only use of \forme{ndzoʁ}, which is one of the few anticausatives that are compatible with a volitional meaning (compare with \ref{ex:WtaR.kondzoR}, §\ref{sec:anticausative.volitionality}).

%ɯ-χpɯm dzɯr ʑo ta-nɯ-tshoʁ
%dzoʁ ʑo pjɤ-tshoʁ


\subsection{Collocation} \label{sec:anticausative.collocation}
Among the pairs in \tabref{tab:anticausative.unaspirated}, the verbs \japhug{ɯ-ʁo+pʰi}{be disappointed by} and \japhug{ɯ-ʁo+mbi}{be discouraged} are remarkable in that both are noun-verb collocations, taking the same inalienably possessed noun \forme{ɯ-ʁo} (otherwise unattested) as object or intransitive subject,\footnote{The noun and the verb are glossed with the same expression, but using the indices (1) and (2) (§\ref{sec:orphan.verb}).   } Showing that the anticausative prenasalization, like several other derivations (§\ref{sec:light.verb}), affects collocations as a whole despite being only morphologically expressed on the verb stem.

The intransitive \forme{mbi} `be discouraged, feel frustrated, lose heart' is always in \textsc{3sg} form (§\ref{sec:intransitive.invariable}), and the possessor on \forme{ɯ-ʁo} indicates the experiencer, as in (\ref{ex:ndZi.Ro.amAnWmbi}) where the form \forme{ndʑi-ʁo} takes a \textsc{2du} possessive prefix coreferent with the subject of the previous verb (see also \ref{ex:nWRo.amAnWmbi} in §\ref{sec:intransitive.invariable}).\footnote{This collocation has an exact Tshobdun cognate \forme{o-ʁeʔ+ⁿbi} `lose morale' \citep[708]{jackson19tshobdun}. }

\begin{exe}
\ex \label{ex:ndZi.Ro.amAnWmbi}
\gll stɤβtsʰɤt mɯ-pɯ-tɯ-nɯ-cʰa-ndʑi ɕti tɕe, ndʑi-ʁo a-mɤ-nɯ-mbi  \\
contest \textsc{neg}-\textsc{aor}-2-\textsc{auto}-can-\textsc{du} be.\textsc{aff}:\textsc{fact} \textsc{lnk} \textsc{2du}.\textsc{poss}-disappoint(1) \textsc{irr}-\textsc{neg}-\textsc{pfv}-\textsc{acaus}:disappoint(2) \\
\glt `(It is not that I don't want to give her to you), it is that you failed in the contest, don't feel frustrated.' (2003sras, 118)
\end{exe}

The transitive verb \forme{pʰi} occurs with the meaning `disappoint', encoding the stimulus as transitive subject and the experiencer as possessor of the object as in (\ref{ex:aRo.YWtWphi}). The argument structure of the two verbs thus only differs by the loss of the subject (stimulus) in the intransitive form \forme{mbi} meaning `be disappointed'.

\begin{exe}
\ex \label{ex:aRo.YWtWphi}
\gll a-ʁo ɲɯ-tɯ-pʰi \\
 \textsc{1sg}.\textsc{poss}-disappoint(1) \textsc{sens}-2-disappoint(2) \\
 \glt `I am disappointed by you.' (elicited)
\end{exe}

A reflexive meaning `be disappointed in oneself'  can be expressed by combining the same person as subject of \forme{pʰi} and possessor of \forme{ɯ-ʁo}  (\textsc{1sg} in  \ref{ex:aRo.YWnWphia}) with the autive prefix \forme{nɯ-} in its `self-affectedness' function (§\ref{sec:autoben.proper}).

 \begin{exe}
\ex \label{ex:aRo.YWnWphia}
\gll kɯki si ki ɯ-qa cʰɯ-tɯ-tɕɤt, ju-tɯ-tsɯm ɯ-tɯ-cʰa nɤ,  [...] tɕe aʑo a-ʁo ɲɯ-nɯ-pʰi-a ŋu  \\
\textsc{dem}.\textsc{prox} tree \textsc{dem}.\textsc{prox} \textsc{3sg}.\textsc{poss}-root \textsc{ipfv}-2-take.out \textsc{ipfv}-2-take.away \textsc{qu}-2-can:\textsc{fact} \textsc{add} {  } \textsc{lnk} \textsc{1sg} \textsc{1sg}.\textsc{poss}-disappoint(1) \textsc{ipfv}-\textsc{auto}-disappoint(2)-\textsc{1sg} be:\textsc{fact} \\
\glt `If you succeed in uprooting this tree and carrying it away, I will admit defeat.' (140428 yonggan de xiaocaifeng-zh, 82-84)
\end{exe}

The anticausativized collocation \forme{ɯ-ʁo+mbi} can undergo additional derivations, such as the facilitative (see \ref{ex:nARo.WtWGAmbi} in §\ref{sec:facilitative.GA}), and the incorporating verbs \japhug{sɤʁombi}{be discouraging}, `be hopeless' and \japhug{nɤʁombi}{lose hope} also derived from it (§\ref{sec:incorp.denom}).

\subsection{Volitionality} \label{sec:anticausative.volitionality} 
An important proportion of anticausative verbs are only compatible with inanimate subjects, for instance \japhug{ɴɢrɯ}{break} (example \ref{ex:pjANWClWG.pjANGrW} in §\ref{sec:anticausative.function}), \japhug{ndʐi}{melt} as in (\ref{ex:tundzxi.YWcha}) or \japhug{ndʑɣaʁ}{be squeezed out} (example \ref{ex:WtWciste}, §\ref{sec:body.part}), and therefore express non-volitional actions.

\begin{exe}
\ex \label{ex:tundzxi.YWcha}
\gll tɤjpa kɯ-xtɕɯ\redp{}xtɕi ka-lɤt ri, mɯ́j-ʁdɯɣ, pɤjkʰu tu-ndʐi ɲɯ-cʰa. \\
snow \textsc{sbj}:\textsc{pcp}-\textsc{emph}\redp{}be.small \textsc{aor}:3\flobv{}-release \textsc{lnk} \textsc{neg}:\textsc{sens}-be.serious still \textsc{ipfv}-\textsc{acaus}:melt \textsc{sens}-can \\
\glt `There was a bit of snow, but it is not serious, it can still melt.' (conversation 15-12-17)
\end{exe} 

Even anticausative verbs that are compatible with human or animal subjects are poorly compatible with volitional meaning. For instance, \japhug{ndʐaβ}{fall/roll} (from \japhug{tʂaβ}{cause to fall/roll}) expresses involuntary fall as in (\ref{ex:pWndzxaB.panWlwoR}), but cannot be used for voluntary rolling motion. The reflexive form of the base transitive verb \japhug{ʑɣɤtʂaβ}{cause oneself to fall/roll} is required instead for this meaning (§\ref{sec:refl.acaus}).

\begin{exe}
\ex \label{ex:pWndzxaB.panWlwoR}
\gll pɯ-ndʐaβ qʰe, paχɕi ra pa-nɯ-lwoʁ tɕe, \\
\textsc{aor}:\textsc{down}-\textsc{acaus}:cause.to.fall \textsc{lnk} apple \textsc{pl} \textsc{aor}:3\flobv{}-\textsc{auto}-spill \textsc{lnk} \\
\glt `He fell down and spilled the apples.' (2010 Tshendzin pear story, 9)
\end{exe} 

However, a few anticausative verbs are compatible with various degrees of volitionality. The verb \japhug{mbɣaʁ}{turn over} (vi) (from \japhug{pɣaʁ}{turn over}) can be used for controllable actions such as tossing over one's bed (\ref{ex:kombGaR.YAmbGaR}), and its distributed action derivation (§\ref{sec:distributed.action}) \japhug{nɤmbɣaʁlaʁ}{turn over here and there} occurs to express voluntary actions as in (\ref{ex:nWnAmbGaRlaR}).

\begin{exe}
\ex \label{ex:kombGaR.YAmbGaR}
\gll kʰri ɯ-taʁ nɯtɕu ko-mbɣaʁ nɤ ɲɤ-mbɣaʁ tɕe \\
bed \textsc{3sg}.\textsc{poss}-on \textsc{dem}:\textsc{loc} \textsc{ifr}:\textsc{east}-\textsc{acaus}:turn.over \textsc{add} \textsc{ifr}:\textsc{west}-\textsc{acaus}:turn.over \textsc{lnk} \\
\glt `She tossed over her bed.' (140430 yufu he tade qizi-zh, 237)
\end{exe} 

 \begin{exe}
\ex \label{ex:nWnAmbGaRlaR}
\gll  nɯ χsɯ-ɣjɤn nɯ-nɤmbɣaʁlaʁ ɲɯ-ŋu. \\ \textsc{dem} three-times  \textsc{aor}-\textsc{distr}:turn.over  \textsc{sens}-be \\
\glt  `(The horse Rtamchog Rinpoche) rolled over on its back three times (on the beach).' (2012 Norbzang, 105)
\end{exe}

The anticausative \japhug{nɯɴɢɤt}{part ways} (from \japhug{qɤt}{separate}, with a lexicalized autive \forme{nɯ\trt}, §\ref{sec:anticausative.direction}, §\ref{sec:anticausative.other.derivations}) stands out in having no restriction on volitionality, as in (\ref{ex:pWnWNGAtndZi}) and (\ref{ex:nWNGAttCi}) where it occurs in the meaning `divorce'. 

\begin{exe}
\ex \label{ex:pWnWNGAtndZi}
\gll ɯ-χti ci na-nɯ-ɕar ri, tɕendɤre kɯ-maqʰu qʰe pɯ-nɯɴɢɤt-ndʑi \\
\textsc{3sg}.\textsc{poss}-companion \textsc{indef} \textsc{aor}:3\flobv{}-\textsc{auto}-search \textsc{lnk} \textsc{lnk} \textsc{sbj}:\textsc{pcp}-be.after \textsc{lnk} \textsc{aor}-\textsc{acaus}:separate-\textsc{du} \\
\glt `She found a husband, but they eventually divorced.' (14-siblings, 99-100)
\end{exe}

\begin{exe}
\ex \label{ex:nWNGAttCi}
\gll nɤʑo jɤ-ɕe, tɕiʑo nɯɴɢɤt-tɕi ma mɤ-jɤɣ \\
\textsc{2sg} \textsc{imp}-go \textsc{1du}  \textsc{acaus}:separate:\textsc{fact}-\textsc{1du} apart.from \textsc{neg}-be.allowed:\textsc{fact} \\
\glt `Go away, we absolutely have to part ways.' (2002qajdoskAt, 75)
\end{exe}

The anticausative \japhug{ndzoʁ}{be attached} (§\ref{sec:anticausative.dummy}) is also attested as a synonym of the intransitive \forme{ɴqoʁ} in its special meaning `cling onto, lean on, grab' (see example \ref{ex:WtaR.kANqoRnW}, §\ref{sec:WtaR}) and with a clear volitional meaning, as shown by (\ref{ex:WtaR.kondzoR}).\footnote{In this excerpt, the same action is described twice, the first time with \japhug{ndzoʁ}{be attached}, the second time with \japhug{ɴqoʁ}{hang}.}
 
\begin{exe}
\ex \label{ex:WtaR.kondzoR}
\gll  ɯ-fsomɯr qʰendɤre, tɯlɤt nɯ ɯ-taʁ ko-ndzoʁ qʰe [...] tɯrmɯkʰa tɕe tɯlɤt nɯ ɯ-taʁ ko-ɴqoʁ tɕe, \\
\textsc{3sg}.\textsc{poss}-tomorrow.evening \textsc{lnk} second.sibling \textsc{dem} \textsc{3sg}.\textsc{poss}-on \textsc{ifr}-\textsc{acaus}:attach \textsc{lnk} { } dusk \textsc{lnk} second.sibling \textsc{dem} \textsc{3sg}.\textsc{poss}-on \textsc{ifr}-hang \textsc{lnk} \\
\glt `The next day in the evening, he grabbed (clung onto) the second sister; (...) at dusk, he grabbed the second sister.' (07-deluge, 41;45)
\end{exe}

\subsection{Compatibility with other derivations} \label{sec:anticausative.other.derivations}

Even though the anticausative probably originates from the autive prefix (§\ref{sec:anticausative.direction}), both derivations are compatible, as shown by examples such as \forme{pɯ-nɯ-ŋgra} (\ref{ex:pWnWNgra.kWnA}) (from \japhug{kra}{cause to fall}) in a concessive clause (§\ref{sec:autoben.spontaneous}, §\ref{sec:concessive.conditional}) or with the spontaneous function (example \ref{ex:WZo.tonWYJW} in §\ref{sec:autoben.spontaneous}).

\begin{exe}
\ex \label{ex:pWnWNgra.kWnA}
\gll ɣɯjpa ɯ-mɯntoʁ nɯ-kɤ-lɤt nɯ pɯ-nnɯ-ŋgra kɯnɤ fsaqʰe qʰe nɯ ɯ-sta nɯ li nɯ jamar tɕʰi kɯ-tu nɯ ɯ-mat ɲɯ-βze ɕti  \\
this.year \textsc{3sg}.\textsc{poss}-flower \textsc{aor}-\textsc{sbj}:\textsc{pcp}-release \textsc{dem} \textsc{aor}:\textsc{down}-\textsc{auto}-\textsc{acaus}:cause.to.fall also next.year \textsc{lnk} \textsc{dem} \textsc{3sg}.\textsc{poss}-place \textsc{dem} again \textsc{dem} about what \textsc{sbj}:\textsc{pcp}-exist \textsc{dem} \textsc{3sg}.\textsc{poss}-fruit \textsc{ipfv}-make[III] be.\textsc{aff}:\textsc{fact} \\
\glt `Even if the flowers that have blossomed this year and fall down, the next year it makes at that place as many fruits (as there were flowers the previous year).' (11-qarGW, 62)
\end{exe}

Like most intransitive verbs,  anticausative verbs can undergo the subject-orien\-ted facilitative (§\ref{sec:facilitative.GA}) \forme{ɣɤ-} derivation. For instance, \japhug{ɴɢlɯt}{break} (vi), \japhug{ɴɢrɯ}{break} (vi) and \japhug{mbɣaʁ}{turn over} (vi) have the derived forms  \japhug{ɣɤɴɢlɯt}{breaking easily} (\ref{ex:mAGANGlWt}), \japhug{ɣɤɴɢrɯ}{break easily} and \japhug{ɣɤmbɣaʁ}{turning over easily} (of cars on a slippery road). 

\begin{exe}
\ex \label{ex:mAGANGlWt}
\gll nɯ-rom kɯnɤ mɤ-ɣɤ-ɴɢlɯt  \\
\textsc{aor}-be.dry also \textsc{neg}-\textsc{facil}-\textsc{acaus}:break:\textsc{fact} \\
\glt `Even after it has dried up, (the wood of high mountain willow twig) does not break easily.' (07-Zmbri, 59)
\end{exe}

The subject-oriented facilitative of the anticausative and the object oriented facilitative \forme{nɯɣɯ-}  (§\ref{sec:facilitative.nWGW}) of the base transitive verb has very close meanings: for instance from \japhug{prɤt}{break} (vt) and \japhug{mbrɤt}{break} (vi) have the facilitative forms \forme{nɯɣɯ-prɤt} and \forme{ɣɤ-mbrɤt}, both of which can be translated as `break easily'; the former implies however the presence of an external agent, while the latter expresses a spontaneous action.

The proprietive \forme{sɤ-} derivation (§\ref{sec:proprietive}) is also attested with some anticausative verbs, in particular \japhug{sɤŋgio}{be slippery} from \japhug{ŋgio}{slip}.

The distributed action derivation (§\ref{sec:distributed.action}) occurs with anticausative verbs expressing a motion event, such as \forme{nɤndʐaβlaβ} `roll again and again/in all directions' and \forme{nɤmbɣaʁlaʁ} `turn over again and again'  from \japhug{ndʐaβ}{fall/roll} \japhug{mbɣaʁ}{turn over} (vi), with a repeated motion; \forme{nɤmbɣaʁlaʁ} can describe a (possibly volitional) rolling motion in both lateral directions (remaining at the same place, as in \ref{ex:zrWG.nWnAmbGaRlaR} below and \ref{ex:nWnAmbGaRlaR} in §\ref{sec:anticausative.volitionality}), as opposed to \forme{nɤndʐaβlaβ}, used for a rolling motion either in one direction (\ref{ex:tWnAndzxaBlaB.joZa}), or  rolling motion in disorderly fashion.

\begin{exe}
\ex \label{ex:zrWG.nWnAmbGaRlaR}
\gll  zrɯɣ nɯ-nɤmbɣaʁlaʁ rdɯl mɤ-tɕɤt \\
louse \textsc{aor}-\textsc{distr}:\textsc{acaus}:turn.over dust \textsc{neg}-take.out:\textsc{fact} \\
\glt `When a louse rolls around, it does not raise dust.' (proverb)
\end{exe}

\begin{exe}
\ex \label{ex:tWnAndzxaBlaB.joZa}
\gll nɯnɯ rgoŋlu nɯ nɯɕimɯma ʑo tɯ-nɤndʐaβlaβ jo-ʑa tɕe jo-ndʐaβ nɤ jo-ndʐaβ. \\
\textsc{dem} ball \textsc{dem} immediately \textsc{emph} \textsc{inf}:II-\textsc{distr}:\textsc{acaus}:roll \textsc{ifr}-start \textsc{lnk} \textsc{ifr}-\textsc{acaus}:roll add \textsc{ifr}-\textsc{acaus}:roll \\
\glt `The ball immediately started rolling over and over.' (140514 huishuohua de niao-zh, 136)
\end{exe}


The anticausative verb \japhug{ŋgio}{slip} has two distributed action forms, the regular one \forme{nɤŋgiolo} which can be translated as `slip/glide/move around over and over' (\ref{ex:mAnANgiolo}) and \forme{nɯŋgiolɯlo}, which rather has a volitional meaning `glide, slide (in no particular direction)' (\ref{ex:YWnWNgiolWlo}).
 
\begin{exe}
\ex \label{ex:mAnANgiolo}
\gll tɕe nɯnɯ pjɯ́-wɣ-ta tɕe, tɕe snama tɤ-ŋke tɕe tɕe, mɤ-nɤŋgiolo.  \\
\textsc{lnk} \textsc{dem} \textsc{ipfv}:\textsc{down}-\textsc{inv}-put \textsc{lnk} \textsc{lnk} beast.of.burden \textsc{aor}-walk \textsc{lnk} \textsc{lnk} \textsc{neg}-\textsc{distr}:\textsc{acaus}:cause.to.glide:\textsc{fact} \\
\glt `One puts (belly and neck bands on the burdens), so that when the beast of burden walks, (the burden) does not move around (on its back).' (30-tAsno, 98)
 \end{exe}
 
\begin{exe}
\ex \label{ex:YWnWNgiolWlo}
\gll tɤ-pɤtso tɤjpɣom ɯ-taʁ ɲɯ-nɯŋgiolɯlo \\
\textsc{indef}.\textsc{poss}-child ice \textsc{3sg}.\textsc{poss}-on \textsc{sens}-\textsc{distr}:\textsc{acaus}:cause.to.glide \\
\glt `The child slides on the ice.' (elicited)
\end{exe}

A few anticausative verbs can take the sigmatic causative prefix. For instance, the causative \japhug{sɯɣndʐi}{melt} (vt) of \japhug{ndʐi}{melt} (vi) can be elicited. This form can express indirect causation, as in (\ref{ex:chAsWGndzxita}),\footnote{The same semantic contrast appears to be found in Zbu (Gong Xun, p.c.) and Tshobdun \citep{jackson14morpho}. } as opposed to the base transitive verb \japhug{ftʂi}{melt} which is used for volitional activities (\ref{ex:khru.chWftsxinW}).

\begin{exe}
\ex \label{ex:chAsWGndzxita}
\gll ta-mar cʰɤ-sɯɣ-ndʐi-t-a \\
\textsc{indef}.\textsc{poss}-butter \textsc{ifr}-\textsc{caus}-\textsc{acaus}:melt-\textsc{pst}:\textsc{tr}-\textsc{1sg} \\
\glt `I let the butter melt (by forgetting it next to a source of heat).' (elicited)
\end{exe}

\begin{exe}
\ex \label{ex:khru.chWftsxinW}
\gll kʰru nɯ cʰɯ-ftʂi-nɯ tɕe, tɕe nɯnɯ pjɯ-lɤt-nɯ ɲɯ-ɕti tɕe, \\
pig.iron \textsc{dem} \textsc{ipfv}-melt-\textsc{pl} \textsc{lnk} \textsc{lnk} \textsc{dem} \textsc{ipfv}:\textsc{down}-release-\textsc{pl} \textsc{sens}-be.\textsc{aff} \textsc{lnk} \\
\glt `They melt the pig iron and pour it into (the mold).' (25-qraR, 33)
\end{exe}

%Example (\ref{ex:tusWNGAR}) illustrates the uses of the anticausative \japhug{ɴɢaʁ}{peel off} (vi), its causative \japhug{sɯɴɢaʁ}{cause to peel off} and the base verb \japhug{qaʁ}{peel off} (vt).
% 
%\begin{exe}
%\ex \label{ex:tusWNGAR}
% \gll tɯrgi laŋlaŋ nɯ tɕe  [...]
%nɯfse, tʰɯ́-wɣ-qaʁ qʰe, tɯ-rdoʁ nɯ tú-wɣ-sɯ-ɴɢaʁ qʰe kɯ-xtɕɯ\redp{}xtɕi ma mɤ-ɴɢaʁ. \\
%fir cone \textsc{dem} \textsc{lnk} { } just.like.that \textsc{aor}:\textsc{downstream}-\textsc{inv}-peel \textsc{lnk} one-piece \textsc{dem} \textsc{ipfv}-\textsc{inv}-\textsc{caus}-\textsc{acaus}:peel \textsc{lnk} \textsc{sbj}:\textsc{pcp}-\textsc{emph}\redp{}be.small apart.from \textsc{neg}-\textsc{acaus}:peel:\textsc{fact} \\
%\glt `The fir cone, when one peels (its scales), one (can) make one (of the scales) come off, (but)  only a little will come off.' (unlike corn cobs, whose hull covers the whole cob)  (08-tWrgi, 84)
% \end{exe}
 %tɕe kɯ-xtɕɯ\redp{}xtɕi ʑo kɯ-ɤrtɯm kɯ-fse lɤ-kɯ-ɤɣɯŋgɯŋgɯ kɯ-fse ɕti ma,
%\textsc{lnk} \textsc{sbj}:\textsc{pcp}-\textsc{emph}\redp{}be.small \textsc{emph} \textsc{sbj}:\textsc{pcp}-be.round \textsc{sbj}:\textsc{pcp}-be.like \textsc{ipfv}:\textsc{upstream}-\textsc{sbj}:\textsc{pcp}-have.concentric.layers \textsc{sbj}:\textsc{pcp}-be.like be.\textsc{aff}:\textsc{fact} \textsc{lnk}
%(its scales) are a bit round and are in several layers that are one one the other,

The anticausative \japhug{nɯɴɢɤt}{part ways} has the causative form \japhug{znɯɴɢɤt}{separate}. This verb is not specifically used for indirection causation, but it is restricted to express separation of two entities from each other, as in (\ref{ex:YWwGznWNGAtndZi2}), unlike the base verb \japhug{qɤt}{separate} which has a broader range of meanings, including `spread' (of limbs, hair, feathers) as in (\ref{ex:Wjme.YWqAt}).

\begin{exe}
\ex \label{ex:YWwGznWNGAtndZi2}
\gll rasti cʰo rɤjndoʁ ni, pjɯ́-wɣ-ʁndzɤr-ndʑi tɕe ɲɯ́-wɣ-z-nɯɴɢɤt-ndʑi ŋu.\\
turnip \textsc{comit} turnip.root \textsc{du} \textsc{ipfv}-\textsc{inv}-cut-\textsc{du} \textsc{lnk} \textsc{ipfv}-\textsc{inv}-\textsc{caus}-\textsc{acaus}:separate-\textsc{du} be:\textsc{fact}\\
\glt `One separates the turnip from its root by cutting them.' (150903 kAJar, 7-8)
\end{exe}

\begin{exe}
\ex \label{ex:Wjme.YWqAt}
\gll ɯ-jme nɯ [...] ki tu-fse tu-z-nɯndzi tɕe tɕe ɲɯ-qɤt ɲɯ-ŋu. \\
\textsc{3sg}.\textsc{poss}-tail \textsc{dem}  {  }  \textsc{dem}.\textsc{prox} \textsc{ipfv}-be.like \textsc{ipfv}-\textsc{caus}-be.vertical[III] \textsc{lnk} \textsc{lnk} \textsc{ipfv}-separate \textsc{sens}-be \\
\glt `It puts its tail vertically like this and spreads (the tail feathers).' (24-ZmbrWpGa, 73-79)
\end{exe}

\subsection{Other cases of voicing alternation} \label{sec:voicing.alternation.non.anticausative}
The anticausative derivation is not the only type of voicing alternation in Japhug. 

Among verbs, two cases of non-anticausative voicing alternations are found, the isolated pair treated in §\ref{sec:fossil.prenasalization} and the irregular sigmatic causative \japhug{ʑɴɢoʁ}{hang} (vt) from \japhug{ɴqoʁ}{hang} (vi). The onset \forme{ʑɴɢ-} in \japhug{ʑɴɢoʁ}{hang} (containing the irregular \forme{ʑ-} allomorph of the sigmatic causative prefix, §\ref{sec:caus.Z}) apparently results from the voicing of an earlier cluster like \forme{*ɕ-ɴq-}  due to phonotactic constraints (§\ref{sec:shC.clusters}). 

In ideophones, voicing alternations with or without prenasalization are also attested, as shown by the pair \forme{qɯqli} and \forme{ɴɢɯɴɢli}, both meaning `eyes wide open' (§\ref{sec:idph.gradation}). In addition, at least one ideophone, \japhug{dzoʁ}{kneeling suddenly and respectfully}, which appears in collocation with the transitive verb \forme{tsʰoʁ}, originates from the anticausative verb \japhug{ndzoʁ}{be attached} (§\ref{sec:genesis.idph}).

\section{Antipassive} \label{sec:antipassive}
The antipassive derivation converts a morphologically transitive verb into an intransitive one, removing the object and preserving the subject. As in the closely related Tshobdun language (\citealt[8]{jackson06paisheng}), two antipassive prefixes are found in Japhug: \forme{rɤ-} and \forme{sɤ-}.


\subsection{\forme{rɤ-} antipassive} \label{sec:antipassive.rA}
The \forme{rɤ-} antipassive prefix is productive, and it is thus impossible to provide a complete list of all examples. \tabref{tab:antipassive1} provides a representative sample of this derivation, which includes a number of verbs of Tibetan origin (for instance \japhug{fsoʁ}{earn} and \japhug{βzjoz}{learn} from \tibet{བསོགས་}{bsogs}{accumulate} and \tibet{སྦྱངས་}{sbʲaŋs}{learn}, respectively). As in Tshobdun (\citealt[8]{jackson06paisheng}), this prefix is typically used when the suppressed argument is non-human (§\ref{sec:antipassive.rA.sA}), though a few exceptions exist (§\ref{sec:antipassive.ditransitive}). 
 
 This derivation takes as input mono- or ditransitive verbs (§\ref{sec:antipassive.ditransitive}), and one labile verb (\japhug{sɯso}{think}, see §\ref{sec:lability.apass}).

\begin{table}
\caption{Examples of the antipassive prefix  \forme{rɤ-}}\label{tab:antipassive1}
\begin{tabular}{lllll} 
\lsptoprule
Base verb  & Derived  verb \\
\midrule
\japhug{roʁ}{carve} &  	\forme{rɤroʁ} `carve things'  &	 \\  
\japhug{ɕpʰɤt}{patch} &  	\forme{rɤɕpʰɤt} `patch clothes'  &	 \\  
\japhug{ɕtʂat}{spare} &  	\forme{rɤɕtʂat} `spare things, managing without wasting'  &	 \\  
\japhug{fse}{whet} &  	\forme{rɤfse} `whet things'  &	 \\  
\japhug{ftɕɤz}{castrate} &  	\forme{rɤftɕɤz} `castrate animals'  &	 \\  
\japhug{ntɕʰa}{butcher} &  	\forme{rɤntɕʰa} `butcher animals'  &	 \\  
\japhug{mɲo}{prepare} &  	\forme{rɤmɲo} `prepare things'  &	 \\  
\japhug{ndɯn}{read aloud} &  	\forme{rɤndɯn} `read sutras/formulas'  &	 \\  
\japhug{rkɤz}{carve} &  	\forme{rɤrkɤz} `carve things'  &	 \\  
\japhug{rɤt}{write, draw} &  	\forme{rɤrɤt} `write/draw things'  &	 \\  
\japhug{βzjoz}{learn} &  	\forme{rɤβzjoz} `study, learn about things, go to school' \\  
\japhug{skɤr}{weigh} &  	\forme{rɤskɤr} `weigh things'  &	 \\  
\japhug{tʂɯβ}{sew} &  	\forme{rɤtʂɯβ} `sew clothes'   &	 \\   
\japhug{scɤt}{move} &  	\forme{rɤscɤt} `move one's house'  &	 \\  
\japhug{fsoʁ}{earn} &  	\forme{rɤfsoʁ} `earn money'  &	 \\  
\japhug{ɕar}{search} &  	\forme{rɤɕar} `search for things'  &	 \\  
\japhug{χtɯ}{buy} &  	\forme{raχtɯ} `do shopping, buy things'  &	 \\  
\japhug{χtɕi}{wash} &  	\forme{raχtɕi} `wash, have a bath'  &	 \\  
\midrule
\japhug{fɕɤt}{tell} &  	\japhug{rɤfɕɤt}{report}   &	 \\ 
\japhug{ŋa}{owe} (money) &  	\japhug{rɤnŋa}{have a debt}   &	 \\  
\japhug{tɕɤβ}{burn} &  	\japhug{rɤtɕɤβ}{burn land} \\  
\japhug{pɣaʁ}{turn over} &  	\japhug{rɤpɣaʁ}{reclaim land} \\  
\japhug{ntsɣe}{sell} &  	\japhug{rɤtsɣe}{do business}   &	 \\ 
\japhug{raʁrɯz}{sweep} &  \forme{rɤroʁrɯz}  `sweep the ground and tidy things up' \\
\midrule
\japhug{sɯso}{think} & \japhug{rɯsɯso}{think}, `ponder'   &	 \\ 
\midrule
\japhug{tʰu}{ask} &  \japhug{rɤtʰu}{ask questions}  \\ 
\japhug{ɕtʂɯ}{entrust with} &  	\japhug{rɤɕtʂɯ}{entrust someone with to}  &	 \\  
\japhug{mbi}{give} &  \japhug{rɤmbi}{give to someone}  \\  
\lspbottomrule
\end{tabular}
\end{table}

The main morphosyntactic differences between transitive verbs and their corresponding antipassive forms are illustrated the following examples. In  (\ref{ex:tWsNaR.paBzjoz}), the base verb \japhug{βzjoz}{learn} is fully transitive, selecting the type-C preverbs in the non-local direct Aorist (§\ref{sec:transitivity.morphology}, §\ref{sec:kamnyu.preverbs}). The object is the topic studied (in  \ref{ex:tWsNaR.paBzjoz}, the nominalized verb \japhug{tɯ-sŋaʁ}{sorcery}) and the transitive subject, the person learning, is taking the ergative case.
 
 
\begin{exe}
\ex \label{ex:tWsNaR.paBzjoz}
\gll  aʑɯɣ  a-me ci tu tɕe, nɯnɯ kɯ tɯ-sŋaʁ pa-βzjoz tɕe, \\
\textsc{1sg}.\textsc{gen} \textsc{1sg}.\textsc{poss}-daughter \textsc{indef} exist:\textsc{fact} \textsc{lnk} \textsc{dem} \textsc{erg} \textsc{nmlz}:\textsc{action}-cast.spells \textsc{aor}:3\flobv{}-learn \textsc{lnk} \\
\glt `I have a daughter, and she has learned sorcery.' (140512 fushang he yaomo-zh, 143)
\end{exe}

The antipassive verb \japhug{rɤβzjoz}{learn things} in (\ref{ex:Wme.pWrABzjoz}) is morphologically intransitive, selecting the type-A preverbs (§\ref{sec:transitivity.morphology}) in the Aorist (\forme{pɯ-} instead of \forme{pa\trt}, §\ref{sec:kamnyu.preverbs}). It is used to avoid mentioning a specific topic of study, and it is often better to translate this verb as `go to school/university'. Like the transitive subject of \japhug{βzjoz}{learn}, the subject of \japhug{rɤβzjoz}{learn things}  corresponds to the person(s) acquiring knowledge. However, when overt, its subject occurs in absolutive form as \forme{ɯ-me nɯ} `her daughter' in (\ref{ex:Wme.pWrABzjoz}).


\begin{exe}
\ex \label{ex:Wme.pWrABzjoz}
\gll ɯ-me nɯ pɯ-rɤ-βzjoz ri tʰam ɯ-<gongzuo> ɯ-ma me \\
\textsc{3sg}.\textsc{poss}-daughter \textsc{dem} \textsc{aor}-\textsc{apass}-learn \textsc{lnk} now \textsc{3sg}.\textsc{poss}-job \textsc{3sg}.\textsc{poss}-work not.exist:\textsc{fact} \\
\glt `Her daughter went to school (studied things) but has no job now.' (17-lhazgron, 60)
\end{exe}

Although example (\ref{ex:Wme.pWrABzjoz}) only illustrates one of the seven morphological criteria for transitivity (§\ref{sec:transitivity.morphology}), all have been successfully tested with antipassive prefixes.

The non-orientable antipassive verbs select as lexicalized orientation the same one as that of their base verb. For instance, \japhug{rɤmbi}{give to someone} selects the \textsc{eastwards} preverbs like its base verb \japhug{mbi}{give} (§\ref{sec:preverb.giving}).

The object affected by the \forme{rɤ-} derivation is not only demoted morphologically but also removed syntactically, and verbs with the antipassive prefix cannot take an overt patient corresponding to the object of the base verb, even as semi-object or with an oblique case (the antipassive \japhug{rɤfɕɤt}{report} is however an exception, see §\ref{sec:antipassive.lexicalized}).

The antipassive verbs are generally understood as having a generic/indefinite patient. In \tabref{tab:antipassive1}, their meaning is translated using the most usual patient associated with a particular activity, for instance `clothes' in the case of \forme{rɤ-tʂɯβ} `sew' and \forme{rɤ-ɕpʰɤt} `patch'.  The patient that has been demoted from object status by the \forme{rɤ-} prefix is nearly always an inanimate entity, but there are some examples of verbs (such as \japhug{rɤntɕʰa} {butcher}) with animal patient, and even demoted human recipient in the case of some secundative verbs (§\ref{sec:antipassive.ditransitive}). The semantic contrast between the \forme{rɤ-} and \forme{sɤ-} antipassive derivations is described in (§\ref{sec:antipassive.function}), and morphosyntactic constructions competing with antipassive derivations to express indefinite objects are discussed in §\ref{sec:non.antipassive.indef.patient}.


Although the meaning of the derived verbs is not fully predictable in each case (§\ref{sec:antipassive.lexicalized}), the formation of the \forme{rɤ-} antipassive is almost perfectly regular. There are only three \forme{rɤ-} antipassive verbs with irregular morphology. First, \japhug{rɤnŋa}{have a debt}  from \japhug{ŋa}{owe}  (a verb selecting as object the amount of money owed, \ref{ex:sqWmpCar.kANata}) has an additional \forme{-n} element. 

\begin{exe}
\ex \label{ex:sqWmpCar.kANata}
\gll sqɯ-mpɕar kɤ-ŋa-t-a \\
\textsc{ten}-money.unit \textsc{aor}-owe-\textsc{pst}:\textsc{tr}-\textsc{1sg} \\
\glt `I bought (it) on credit and owe (him) ten renminbi.' (elicited)
\end{exe}

Second,  \japhug{rɤtsɣe}{do business} from \japhug{ntsɣe}{sell} lacks the \forme{n-} preinitial found in the base verb. Third, \japhug{rɯsɯso}{think} from \japhug{sɯso}{think} (about)  (§\ref{sec:lability.apass}) has \forme{rɯ-} instead of \forme{rɤ-}. These cases are accounted for in §\ref{sec:antipassive.history} .
 
Apart from these irregularities, the allomorphy of the \forme{rɤ-} antipassive is limited to the variant \forme{ra-} found with verb roots with an onset with uvular preinitial (§\ref{sec:A.vs.a.prefixes}) as in \japhug{raχtɯ}{do shopping}. The antipassive \forme{rɤ-} prefix is formally similar to the denominative \forme{rɤ-} (§\ref{sec:denom.intr.rA}),\footnote{In addition, there is a residue of \forme{rɤ-} prefixed verb that can neither be analyzed as antipassive nor as denominal derivations (§\ref{sec:rA.non.apass}). } to the extent that synchronic ambiguity exists between these two prefixes. In particular, the verb \japhug{rɤznde}{make a wall} (vi) can be synchronically analysed either as a denominal form of \japhug{znde}{stone wall} or of the transitive verb \japhug{znde}{make a wall}.  Note that both options are equally likely, as the prefix \forme{rɤ-} is attested with other denominal expressing the building of the base noun, for instance \japhug{tɤ-loʁ}{nest} \fl{} \japhug{rɤloʁ}{make a nest} (vi). The historical implications of this observation are explored in §\ref{sec:antipassive.history}.  
 
Not all transitive verbs with non-human patients can be antipassivized. In particular, when an intransitive verb having the meaning of the expected antipassive verb already exists, antipassivization is less likely. For instance, \japhug{ti}{say} lacks a \forme{rɤ-} antipassive form, as the intransitive \japhug{rɯɕmi}{speak} (which can have an overt dative recipient, \ref{ex:WCki.turWCmia}, §\ref{sec:intr.goal}, but no reported speech complement clause) already serves as its `lexical antipassive'. The verbs of ingestion \japhug{ndza}{eat} and \japhug{tsʰi}{drink} share the compound verb \japhug{rɯndzɤtsʰi}{have a meal} (§\ref{sec:denom.intr.rA}, §\ref{sec:denom.compound.verbs}) as their intransitive counterpart.

\subsection{\forme{sɤ-} antipassive} \label{sec:antipassive.sA}
Like the \forme{rɤ-} antipassive, the \forme{sɤ-} antipassive is productive. \tabref{tab:antipassive2} presents a list of representative examples, including Tibetan loanwords such as \japhug{fstɤt}{praise}  from  \tibet{བསྟོད་}{bstod}{praise}. Unlike the \forme{rɤ-} antipassive, the  \forme{sɤ-} prefix is used when the demoted patient is  human or equivalent (§\ref{sec:antipassive.rA.sA}, see \citealt[8]{jackson06paisheng} on Tshobdun), except in the case of some secundative verbs (§\ref{sec:antipassive.ditransitive}). The \forme{sɤ-} antipassive derivation can also demote animal patients in specific contexts (§\ref{sec:antipassive.rA.sA}).

The only cases of allomorphy with the antipassive \forme{sɤ-} are the allomorph \forme{sa-} occurring when the prefix is followed by a complex onset whose first  element is a uvular fricative (§\ref{sec:A.vs.a.prefixes}), and the allomorph \forme{sɤz-} which appears with some polysyllabic stems whose first syllable has a sonorant initial \japhug{sɤzɣɤmɯ}{praise people}.

The \forme{sɤ-} prefix is homophonous with the rogative (§\ref{sec:rogative.derivation}), the proprietive (§\ref{sec:proprietive}) and denominal derivations (§\ref{sec:sigmatic.denominal}), as well as with the oblique participle \forme{sɤ(z)-} (§\ref{sec:oblique.participle}). Potential ambiguity exists with the proprietive derivation. For instance, the transitive \japhug{nɯzdɯɣ}{worry about} has two homophonous intransitive derived verbs \forme{sɤ-nɯzdɯɣ}, an antipassive meaning `worry about people', and a proprietive `causing worry to people'.

\begin{table}[h]
\caption{Examples of the antipassive prefix \forme{sɤ-} }\label{tab:antipassive2}
\begin{tabular}{lllllllll}
 \lsptoprule
Base verb  & Derived  verb &\\
\midrule
\japhug{tɕʰɯ}{gore} &  	\japhug{sɤtɕʰɯ}{gore people}  &	 \\  
\japhug{mtsɯɣ}{bite} &  	\japhug{sɤmtsɯɣ}{bite people}   &	 \\  
\japhug{nɤmtsioʁ}{peck} &  	\japhug{sɤnɤmtsioʁ}{peck people}   &	 \\   
\midrule
\japhug{ɣɤmɯ}{praise} &  	\forme{sɤɣɤmɯ}, \forme{sɤzɣɤmɯ} `praise people'  &	 \\   
\japhug{fstɤt}{praise} &  	\japhug{sɤfstɤt}{praise people}  &	 \\ 
\japhug{nɯrtɕa}{tease} &  	\japhug{sɤnɯrtɕa}{tease people}   &	 \\ 
\japhug{ʁndɯ}{hit} &  	\japhug{saʁndɯ}{hit people}   &	 \\  
\japhug{nɤkʰe}{bully} &  	\japhug{sɤnɤkʰe}{bully people}   &	 \\  
\japhug{nɤsɤɣ}{be jealous of}   &  	\japhug{sɤnɤsɤɣ}{be jealous of people}   &	 \\  
\japhug{nɯrɯtʂa}{envy} &  	\japhug{sɤnɯrɯtʂa}{envy people}   &	 \\  
\japhug{ɕar}{search} &  	\japhug{sɤɕar}{search someone}   \\ 
\midrule
\japhug{sɯxɕɤt}{teach} &  	\japhug{sɤsɯxɕɤt}{teach people}  &	 \\  
\japhug{tʰu}{ask} &  \japhug{sɤtʰu}{ask in marriage}  \\  
\midrule
\japhug{nɤre}{laugh} &  	\japhug{sɤnɤre}{laugh at people}   &	  \\  
\japhug{sɤŋo}{listen} &  	\japhug{sɤsɤŋo}{listen to advice}	 \\  
\lspbottomrule
\end{tabular}
\end{table}

Most \forme{sɤ-} antipassives derive from monotransitive verbs, but a handful of them are based on ditransitive verbs (§\ref{sec:antipassive.ditransitive}). In addition, the labile verbs \japhug{nɤre}{laugh} and \japhug{sɤŋo}{listen} are also compatible with the \forme{sɤ-} prefix. The antipassive forms \japhug{sɤsɤŋo}{be obedient} and \japhug{sɤnɤre}{laugh at people} derive from the meanings `listen to' and `laugh at, mock' that these two verbs have when  conjugates transitively (§\ref{sec:lability.apass}).

Some of \forme{sɤ-} antipassive verbs can either be dynamic verbs or stative verbs, in the latter case expressing a general tendency/propensity of the subject to do the action. For instance, \forme{sɤ-ndza} from \japhug{ndza}{eat} can mean `eat people, eat someone' (as in \ref{ex:kWsAndza.toGWGu}, §\ref{sec:antipassive.indefinite.patient}), but also `be a man-eater', `prick' (\ref{ex:mAsAndza}) or `be carnivorous' (animal eating other animals, \ref{ex:kWrNi1}, §\ref{sec:antipassive.rA.sA}).

\begin{exe}
\ex \label{ex:mAsAndza}
\gll tɕe nɯnɯ ɯ-rme tu ma mɤ-sɤ-ndza ma ɯ-mdzu me.  \\
\textsc{lnk} \textsc{dem} \textsc{3sg}.\textsc{poss}-hair exist:\textsc{fact} \textsc{lnk} \textsc{neg}-\textsc{apass}-eat:\textsc{fact} \textsc{lnk} \textsc{3sg}.\textsc{poss}-thorn not.exist:\textsc{fact} \\
\glt `It has hair, but does not prick, because it has no thorns.' (15-babW, 97)
\end{exe}

The propensity antipassives have a meaning very close to that of generic human objects, as demonstrated by example (\ref{ex:sAndza.kWndza}), where the same meaning is expressed by the antipassive \forme{sɤ-ndza} and the generic object \forme{kɯ-ndza}  (see additional examples in §\ref{sec:antipassive.vs.generic}).
 
\begin{exe}
\ex \label{ex:sAndza.kWndza}
\gll  mdzadi nɯ wuma ʑo sɤ-ndza. wuma ʑo kɯ-ndza tɕe rɤʑa. \\
flea \textsc{dem} really \textsc{emph} \textsc{apass}-eat:\textsc{fact} really \textsc{emph} \textsc{genr}:S/O-eat:\textsc{fact} \textsc{lnk} itch:\textsc{fact} \\
\glt `Fleas bite a lot. They bite people a lot, and it itches.' (21-mdzadi, 16-17)
\end{exe}

\subsection{Lexicalized antipassive} \label{sec:antipassive.lexicalized} 
 
Most antipassive verbs have meanings that are predictable from that of their base verbs. However, in some cases antipassivization not only demotes the patient from object status, but restricts the range of meanings of the verb root.

The meaning of the transitive verb \japhug{pɣaʁ}{turn over} differs depending on the  objects it occurs with; its possible meanings include `turning (clothes) inside out', `open (the cover of a box)', `cross (mountains, rivers)' (\ref{ex:jannWpGaRndZi.jannWmGlandZi}, §\ref{sec:motion.verbs}) or `plough (fields)' (see \ref{ex:nWji.lopGaRnW} below and \ref{ex:lukWpGaR.nW.cinA} in §\ref{sec:cinA}). 

\begin{exe}
\ex \label{ex:nWji.lopGaRnW}
\gll nɯ-ji ra kɯ-dɯ\redp{}dɤn kɯ-jɯ\redp{}jom lo-pɣaʁ-nɯ \\
\textsc{3pl}.\textsc{poss}-field \textsc{pl} \textsc{sbj}:\textsc{pcp}-\textsc{emph}\redp{}be.many \textsc{sbj}:\textsc{pcp}-\textsc{emph}\redp{}be.wide \textsc{ifr}:\textsc{upstream}-turn.over-\textsc{pl} \\
\glt `They had ploughed many wide fields for them.' (2002 qajdoskAt, 90)
\end{exe}

By contrast, its antipassive \japhug{rɤpɣaʁ}{reclaim land}\footnote{The meaning of this verb corresponds to Chinese \ch{开垦}{kāikěn}{reclaim land}, `clear a wild area for cultivation'.}  only preserves the last meaning of the base verb, with `land' or `field' as implicit patient as in (\ref{ex:lurApGaR}). Note in addition that the antipassive selects the \textsc{upstream} preverbs (\forme{lu-} in \ref{ex:lurApGaR}) like the verb \forme{pɣaʁ} in the meaning `plough' (\forme{lo-} in \ref{ex:nWji.lopGaRnW}).

\begin{exe}
\ex \label{ex:lurApGaR}
\gll zgoku tu-ɕe qʰe lu-rɤ-pɣaʁ nɤ, tɤ-rɤku cʰo ra pjɯ-ji qʰe,  tɕe nɯ kɯ-fse ku-rɤʑi pjɤ-ŋu. \\
mountain \textsc{ipfv}:\textsc{up}-go \textsc{lnk} \textsc{ipfv}:\textsc{upstream}-\textsc{apass}-turn.over \textsc{add} \textsc{indef}.\textsc{poss}-crops \textsc{comit} \textsc{pl} \textsc{ipfv}-plant \textsc{lnk} \textsc{lnk} \textsc{dem} \textsc{sbj}:\textsc{pcp}-be.like \textsc{ipfv}-stay \textsc{ifr}.\textsc{ipfv}-be \\
\glt `(The old man) lived by going to the mountain, clearing fields and  planting crops.' (150831 jubaopen-zh, 3-5)
\end{exe}

Similarly, while the transitive verb \japhug{tɕɤβ}{burn} can take various types of referents as objects (including humans, as in \ref{ex:Padma.katCABnW}, §\ref{sec:obviation.saliency}), the antipassive \japhug{rɤtɕɤβ}{burn land}\footnote{The meaning of this verb is translated as \ch{烧荒}{shāohuāng}{clear land by fire} in Chinese. } is also restricted to its use in agriculture, with `land' as demoted patient.

The expected meaning of \japhug{rɤpɣaʁ}{reclaim land} and \japhug{rɤtɕɤβ}{burn land} if they had been regular antipassives would have been `turn things over' and `burn things', respectively. Despite the absence of an incorporated noun, these two verbs include information about the demoted patient. This semantic irregularity has implications for the study of the origin of the antipassive prefixes in Japhug, as discussed in §\ref{sec:antipassive.irr.semantic} .

 The verb \japhug{rɤfɕɤt}{report} from \japhug{fɕɤt}{tell} stands out among antipassive verbs from the point of view syntax. While \forme{rɤfɕɤt} is morphologically intransitive, as is shown by the form \forme{kɯ-rɤfɕɤt} in (\ref{ex:kWrAfCAt.YWra}) with the intransitive subject generic prefix \forme{kɯ-} (§\ref{sec:intr.23}), the patient of this verb can nevertheless be overt as a semi-object (in \ref{ex:kWrAfCAt.YWra}, the demonstrative \forme{nɯnɯra}, anaphorically referring to the previous complement clauses). The object of the base verb \japhug{fɕɤt}{tell} is thus only demoted to semi-object status (§\ref{sec:semi.object}), and not syntactically removed. The dative recipient is also preserved by the antipassive derivation.

\begin{exe}
\ex \label{ex:kWrAfCAt.YWra}
\gll slama ra ɣɯ tʰɯtʰɤci kɯ-fse, nɯ kɤ-rɤ-βzjoz ra ɲɯ-stu mɯ́j-stu nɯ,
[...] nɯnɯra nɯ-pʰama ra nɯ-ɕki kɯ-rɤfɕɤt ɲɯ-ra. \\
student \textsc{pl} \textsc{gen} something \textsc{sbj}:\textsc{pcp}-be.like \textsc{dem} \textsc{inf}-\textsc{apass}-study \textsc{pl} \textsc{sens}-be.serious \textsc{neg}:\textsc{sens}-be.serious \textsc{dem} {  }  \textsc{dem}:\textsc{pl} \textsc{3pl}.\textsc{poss}-parents \textsc{pl} \textsc{3pl}.\textsc{poss}-\textsc{dat} \textsc{genr}:S/O-report:\textsc{fact} \textsc{sens}-be.needed \\
\glt `(As a teacher), one has to report to the parents all sorts of things about the student, whether they study seriously or not etc.' (150901 tshuBdWnskAt, 20)
\end{exe}

 
 \subsection{Antipassive forms of ditransitive verbs } \label{sec:antipassive.ditransitive}
 Among ditransitive verbs, indirective verbs (§\ref{sec:ditransitive.indirective}) build their antipassive derivations the same way as monotransitive verbs.\footnote{Note however the morphosynactic peculiarities of the lexicalized antipassive  \japhug{rɤfɕɤt}{report}  from the indirective verb \japhug{fɕɤt}{tell} (§\ref{sec:antipassive.lexicalized}). } For instance,  \japhug{tʰu}{ask} has the two antipassive forms \japhug{rɤtʰu}{ask questions} and \japhug{sɤtʰu}{ask in marriage} depending on whether the demoted patient is inanimate or human (see examples \ref{ex:korAthunW} and \ref{ex:kWsAthu.Cea}, §\ref{sec:antipassive.function}).


Three secundative verbs, \japhug{mbi}{give}, \japhug{ɕtʂɯ}{entrust with} and  \japhug{jtsʰi}{give to drink}, behave differently in this regard. Their object (semantically the recipient) is generally human, but the \forme{sɤ-} antipassive prefix does not occur with these verbs: the corresponding forms (for instance \japhug{sɤmbi}{ask for}) exist, but do not have antipassive meaning  (they exemplify the rogative derivation, §\ref{sec:rogative.derivation}). Instead, their antipassive forms \japhug{rɤmbi}{give to someone}, \japhug{rɤɕtʂɯ}{entrust someone with} and \japhug{rɤjtsʰi}{give to someone to drink} take the \forme{rɤ-} prefix. These verbs are henceforth referred to as `\textit{rɤ-} antipassivized secundative verbs'.

The demoted recipient of \textit{rɤ-} antipassivized secundative verbs can be indefinite and non-specific as in the case of most antipassive verbs (§\ref{sec:antipassive.function}), as  in (\ref{ex:lonba.YWrAmbi}).

\begin{exe}
\ex \label{ex:lonba.YWrAmbi}
\gll  ɯʑɤɣ ɯ-βra kɯ-rkɯ\redp{}rkɯn ntsɯ ma mɯ-pjɤ-nɯ-ta. ɯ-ro ra lonba ɲɯ-rɤ-mbi pjɤ-ŋu. \\
\textsc{3sg}:\textsc{gen} \textsc{3sg}.\textsc{poss}-share \textsc{sbj}:\textsc{pcp}-\textsc{emph}\redp{}be.few always apart.from \textsc{neg}-\textsc{ifr}-\textsc{auto}-put \textsc{3sg}.\textsc{poss}-rest \textsc{pl} all \textsc{ipfv}-\textsc{apass}-give \textsc{ifr}.\textsc{ipfv}-be \\
\glt  `He would only keep a little for himself, and give away the rest.' (150902 hailibu-zh, 8)
\end{exe}
%\begin{exe}
%\ex \label{ex:aRAndWndAt.YArAmbi}
%\gll  nɯ-<xigua> ɣɯ ɯ-rɣi nɯnɯ li aʁɤndɯndɤt ʑo ɲɤ-rɤ-mbi. \\
%3pl.\textsc{poss}-watermelon \textsc{gen} \textsc{3sg}.\textsc{poss}-seed \textsc{dem} again everywhere \textsc{emph} \textsc{ifr}-\textsc{apass}-give \\
%\glt `He gave (distributed) the watermelon's seeds everywhere.'  (150824 yuanding-zh, 131)
%\end{exe}

However, often, as in (\ref{ex:nWrAmbitCi}) and (\ref{ex:CkArACtsxWa}), the identify of the recipient is known to the speaker, but the antipassive is chosen to leave it unspecified, either because the information is irrelevant (because the addressee does not know the people in question) or to hide information.

\begin{exe}
\ex \label{ex:nWrAmbitCi}
\gll   kɯmŋu ma mɯ-nɯ-ʁaʁ ɯ-qʰu tɕe, tɕe χsɯm nɯ nɯ-rɤ-mbi-tɕi tɕe \\
five apart.from \textsc{neg}-\textsc{aor}-hatch \textsc{3sg}.\textsc{poss}-after \textsc{lnk} \textsc{lnk} three \textsc{dem} \textsc{aor}-\textsc{apass}-give-\textsc{du} \textsc{lnk} \\
\glt `Only five (of the twelve eggs) hatches, and we gave three (of the chicks) (to other people).' (22-kumpGa, 81-82)
\end{exe}

\begin{exe}
\ex \label{ex:CkArACtsxWa}
\gll  a-kumpɣa ɯ-pɯ kɯ-sɤjndɯ\redp{}jndɤt ʑo pɯ-ŋu ri, tɕeri <xingqitian> <fangjia> tɕe tu-nɯ-ɕe-a pɯ-ra tɕe ɕ-kɤ-rɤ-ɕtʂɯ-a. \\
\textsc{1sg}.\textsc{poss}-hen \textsc{3sg}.\textsc{poss}-young \textsc{sbj}:\textsc{pcp}-\textsc{emph}\redp{}be.cute \textsc{emph} \textsc{pst}.\textsc{ipfv}-be \textsc{lnk} \textsc{lnk} sunday holidays \textsc{loc} \textsc{ipfv}:\textsc{up}-\textsc{vert}-go-\textsc{1sg} \textsc{pst}.\textsc{ipfv}-be.needed \textsc{lnk} \textsc{tral}-\textsc{aor}-\textsc{apass}-entrust.with-\textsc{1sg} \\
\glt `My chicks were very cute, but on sunday, I had to go back on holiday, and I went (to someone$_i$) and entrusted him/her/them$_i$ with them. (150819 kumpGa, 48-49)
\end{exe}

In addition, antipassivization of secundative verbs does not removes the recipient: it can be demoted to oblique argument status, with dative marking, as in (\ref{ex:YWrAjtshia.NgrAl}) and (\ref{ex:aCki.kArACtsxW}). 
 
\begin{exe}
\ex \label{ex:YWrAjtshia.NgrAl}
\gll  aʑo tɤ-tɕɯ ra nɯ-ɕki cʰa ɲɯ-rɤ-jtsʰi-a ŋgrɤl \\
 \textsc{1sg} \textsc{indef}.\textsc{poss}-son \textsc{pl} \textsc{3pl}.\textsc{poss}-\textsc{dat} alcohol \textsc{ipfv}-\textsc{apass}-give.to.drink be.usually.the.case:\textsc{fact} \\
 \glt `I (usually) give a drink to the guys.' (elicited)
 \end{exe}

Although the three \textit{rɤ-} antipassivized secundative verbs \japhug{rɤmbi}{give to someone}, \japhug{rɤɕtʂɯ}{entrust someone with} and \japhug{rɤjtsʰi}{give to someone to drink} are morphologically intransitive like all other antipassives, as shown by the absence of past transitive \forme{-t-} suffix in (\ref{ex:CkArACtsxWa}), and of C-type orientation preverb in (\ref{ex:aCki.kArACtsxW}), they do preserve more transitivity features than most antipassive verbs. 

First, the three \textit{rɤ-} antipassivized secundative verbs differ from other antipassive verbs in terms of case marking: their subject can receive ergative marking in some contexts.  

With third person arguments and no overt recipient as in (\ref{ex:akhWna.YArAmbindZi}), the subject \forme{a-mu a-wa ni} is in absolutive form.

\begin{exe}
\ex \label{ex:akhWna.YArAmbindZi}
\gll a-mu a-wa ni a-kʰɯna ɲɤ-rɤ-mbi-ndʑi \\
\textsc{1sg}.\textsc{poss}-mother \textsc{1sg}.\textsc{poss}-father \textsc{du} \textsc{1sg}.\textsc{poss}-dog  \textsc{ifr}-\textsc{apass}-give-\textsc{du} \\
\glt `My parents gave my dog away.' (elicitation, 2019-11-30)
 \end{exe}
 
However, ergative marking on the subject occurs in (\ref{ex:aCki.kArACtsxW}) with a \textsc{1sg} recipient, and in (\ref{ex:nWwGrAmbia2}) below with a \textsc{1sg} semi-object (theme).
 
\begin{exe}
\ex \label{ex:aCki.kArACtsxW}
\gll a-wa kɯ a-ʁi a-ɕki kɤ-rɤ-ɕtʂɯ. \\
 \textsc{1sg}.\textsc{poss}-father \textsc{erg} \textsc{1sg}.\textsc{poss}-younger.brother \textsc{1sg}.\textsc{poss}-\textsc{dat} \textsc{aor}-\textsc{apass}-entrust.with \\
 \glt `My father entrusted me with my younger brother.' (elicited)
 \end{exe} 
 
Second, an even more unusual feature of the the three \textit{rɤ-} antipassivized secundative verbs is that the morphosyntactic status of the theme of the giving action. In all the examples above from (\ref{ex:lonba.YWrAmbi}) to (\ref{ex:YWrAjtshia.NgrAl}), the theme is a third person semi-object (§\ref{sec:semi.object}), and its number cannot be indexed even with a \textsc{1sg} subject  (§\ref{sec:double.number.indexation}).

However, in the very rare cases when the theme is first or second person, the verb  \japhug{rɤmbi}{give to someone}  exceptionally occurs with inverse \forme{wɣ-} or local scenario indexation affixes (\forme{ta-} 1\fl{}2 and \forme{kɯ-}  2\fl{}1, see §\ref{sec:indexation.local}) prefixes, as if the verb were morphologically transitive, and the theme were a direct object. No such example is found in the corpus, but forms such as (\ref{ex:nWwGrAmbia2}) can be elicited (§\ref{sec:secundative.theme}).
 
 \begin{exe}
\ex \label{ex:nWwGrAmbia2}
\gll  a-wa kɯ aʑo nɯ́-wɣ-rɤ-mbi-a \\
\textsc{1sg}.\textsc{poss}-father \textsc{erg} \textsc{1sg} \textsc{aor}-\textsc{inv}-\textsc{apass}-give-\textsc{1sg} \\
\glt `My father gave me away.' (elicited)
\end{exe}

This puzzling construction is not only possible, but it is actually the only way to express a non-third person theme with the verb \japhug{mbi}{give}.
 
Not all secundative verbs however select the \forme{rɤ-} antipassive like \japhug{mbi}{give} and  \japhug{jtsʰi}{give to drink}. For instance, \japhug{sɯxɕɤt}{teach}, which encodes the recipient/addressee as object (the \textsc{2sg} in \ref{ex:pjWtasWxCAt}, as indicated by the \forme{ta-} portmanteau prefix), is not compatible with \forme{rɤ-} prefix: the only antipassive form of this verb is \japhug{sɤsɯxɕɤt}{teach people}, `work as a teacher', as in (\ref{ex:pjWsAsWxCata}).

\begin{exe}
\ex \label{ex:pjWtasWxCAt}
\gll tɤfsɤri kɤ-βzu ci pjɯ-ta-sɯxɕɤt. \\
thread \textsc{inf}-make a.little \textsc{ipfv}-1\fl{}2-teach \\
\glt `Let me teach you how to make threads.' (vid-20140506043657, 43)
\end{exe}

\begin{exe}
\ex \label{ex:pjWsAsWxCata}
\gll li sloχpɯn ta-ndo-t-a tɕe, nɯre pjɯ-sɤ-sɯxɕat-a pɯ-ŋu. \\
again teacher \textsc{aor}:3\flobv{}-take-\textsc{pst}:\textsc{tr}-\textsc{1sg} \textsc{lnk} \textsc{dem}:\textsc{loc} \textsc{ipfv}-\textsc{apass}-teach-\textsc{1sg} \textsc{pst}.\textsc{ipfv}-be \\
\glt `I too became a teacher, and I was teaching there.' (12-BzaNsa, 19)
\end{exe} 

Similarly, the secundative verb \japhug{nɯsɯkʰo}{rob, extort} has the antipassive form \japhug{sɤnɯsɯkʰo}{rob people}, also with the \forme{sɤ-} prefix (example \ref{ex:zYAsAnWsWkhonW}, §\ref{sec:antipassive.compatibility}).

 \subsection{Reduplicated antipassive} \label{sec:antipassive.redp}
The intransitive verb \japhug{rɤtʰutʰe}{inquire}, `ask for information' derives from \japhug{tʰu}{ask} like \japhug{rɤtʰu}{ask questions} (\ref{ex:korAthunW}, §\ref{sec:antipassive.function}, §\ref{sec:antipassive.ditransitive}), with the rare reduplication in \forme{-e} in addition to the \forme{rɤ-} prefix. This reduplicated antipassive form is isolated, but this type of reduplication is attested in a few other forms (§\ref{sec.distributed.action.l}).

As shown by (\ref{ex:aCki.GWnWrAthuthe}), \japhug{rɤtʰutʰe}{inquire} is an intransitive verb: the type A orientation preverb \forme{nɯ-} is selected instead of type C if the verb were morphologically transitive (§\ref{sec:kamnyu.preverbs}). This verb also selects an oblique argument in the dative, like its base verb (§\ref{sec:ditransitive.indirective}).


\begin{exe}
\ex \label{ex:aCki.GWnWrAthuthe}
\gll  a-ɕki ɣɯ-nɯ-rɤtʰutʰe \\
\textsc{1sg}.\textsc{poss}-\textsc{dat} \textsc{cisl}-\textsc{aor}-ask.permission \\
\glt `He came and asked me (for information).' (elicited)
\end{exe}

This verb is mainly attested as a negative infinitive converb \forme{mɤ-kɤ-rɤtʰutʰe} `without asking (for permission)' (as in Chinese \zh{问都没有问就 ……}) as in (\ref{ex:mAkArAthuthe}).\footnote{The ergative on the subject \forme{mbro nɯ kɯ} `the horse' is due to the main verb \forme{pjɤ-qʰa} `he hated that...'. }

\begin{exe}
	\ex \label{ex:mAkArAthuthe}
	\gll mbro nɯ kɯ ɕkɤrɯ nɯnɯ [maka mɤ-kɤ-rɤtʰutʰe] kɯ nɯfse ju-kɤ-ɣi nɯ wuma ʑo pjɤ-qʰa ɲɯ-ŋu.  \\
	horse \textsc{dem} \textsc{erg} serow \textsc{dem} at.all \textsc{neg}-\textsc{inf}-ask.permission \textsc{erg} like.that \textsc{ipfv}-\textsc{inf}-come \textsc{dem} really \textsc{emph} \textsc{ifr}.\textsc{ipfv}-hate \textsc{sens}-be \\
	\glt `The horse hated that the serow came like that without asking (permission) at all.' (ma he lu-zh, 9)
\end{exe} 

  \subsection{Antipassive and past imperfective} \label{sec:antipassive.pst.ipfv}
Antipassive verbs, unlike most dynamic verbs (§\ref{sec:pst.ifr.ipfv.morphology}), are compatible with Past Imperfective \forme{pɯ-} and Inferential Imperfective \forme{pjɤ-}. This question is however difficult to study for three reasons. 

First, many of the transitive base verbs in Tables \ref{tab:antipassive1} and \ref{tab:antipassive2}, for instance \japhug{βzjoz}{learn} and \japhug{ɕpʰɤt}{patch}, select the \textsc{downwards} preverbs as one of their lexicalized orientation (see for instance the perfective form \forme{pa-βzjoz} `she learned it' in \ref{ex:tWsNaR.paBzjoz} above). As a consequence, there is syncretism for these verbs between Aorist and Inferential on the one hand, and Past Imperfective and Inferential Imperfective on the other hand (§\ref{sec:pst.ifr.ipfv.morphology}). For instance, the form \forme{pɯ-rɤ-βzjoz} `she went to school' in (\ref{ex:Wme.pWrABzjoz}) above is Aorist, but in a different context the same form can be analyzed as a Past Imperfective `she was studying'.

Second, while simple Past and Inferential Imperfective are attested with antipassive verbs, they are also compatible with the Periphrastic Aorist and Inferential Imperfective (§\ref{sec:pst.ifr.ipfv}), combining the Imperfective verb form (§\ref{sec:ipfv.periphrastic.TAME}) with a copula in Inferential Imperfective (\forme{pjɤ-ŋu}) or in Past Imperfective (\forme{pɯ-ŋu}) as illustrated by example (\ref{ex:pjWrABzjoz.pWNu}). It is unclear whether any function difference exists between non-periphrastic and periphrastic tenses for these verbs, apart from the fact that the latter are unambiguously imperfective.
 
\begin{exe}
\ex \label{ex:pjWrABzjoz.pWNu}
\gll tɕe ɯ-me nɯnɯ <xianzhong> nɯtɕu pjɯ-rɤ-βzjoz pɯ-ŋu ri, \\
\textsc{lnk} \textsc{3sg}.\textsc{poss}-daughter \textsc{dem} district.school \textsc{dem}:\textsc{loc} \textsc{ipfv}-\textsc{apass}-learn \textsc{ipfv}-be \textsc{lnk} \\
\glt `(At the time when) her daughter was going to school, (her father was drinking alcohol and neglected his parental duties).' (17-lhazgron, 67)
\end{exe}

Third, since some of the \forme{sɤ-} antipassives can be used as stative verbs (propensity antipassives, §\ref{sec:antipassive.sA}), they are thus compatible with Aorist and Inferential Imperfective anyway.

Despite these difficulties, unambiguous examples of  non-periphrastic Aorist and Inferential Imperfective are not common  in the corpus even with verbs selecting the \textsc{downwards} orientation. In (\ref{ex:pjArABzjoz.ipfv}), it is clear from the context that \forme{pjɤ-rɤ-βzjoz} is Inferential Imperfective rather than Inferential Perfective both due to the context, the presence of the adverb \forme{ntsɯ} and the unambiguous  Inferential Imperfective with Progressive \forme{pjɤ-k-ɤsɯ-ndɯn-ci} `he was reading it' in the following clause.

\begin{exe}
\ex \label{ex:pjArABzjoz.ipfv}
\gll <caichen> nɯnɯ pjɤ-rɤ-βzjoz ntsɯ. jɯɣi ntsɯ pjɤ-k-ɤsɯ-ndɯn-ci. \\
\textsc{anthr} \textsc{dem} \textsc{ifr}.\textsc{ipfv}-\textsc{apass}-learn always book always \textsc{ifr}.\textsc{ipfv}-\textsc{peg}-\textsc{prog}-read-\textsc{peg} \\
\glt `Caichen was always studying, always reading books.' (150907 niexiaoqian-zh, 21)
\end{exe}

In other contexts, there is genuine ambiguity: example (\ref{ex:pjArAɕphAt}) can either mean `the wife was patching clothes' (the interpretation provided by Tshendzin) or `the wife (had) patched clothes'.

\begin{exe}
\ex \label{ex:pjArAɕphAt}
\gll tɤ-rʑaβ nɯ pjɤ-rɤ-ɕpʰɤt, \\
\textsc{indef}.\textsc{poss}-wife \textsc{dem} \textsc{ifr}.\textsc{ipfv}-\textsc{apass}-patch \\
\glt `The wife was patching clothes.' (qajdoskAt 2002, 20)
\end{exe}

  \subsection{The uses of the antipassive derivations } \label{sec:antipassive.function}

  \subsubsection{Antipassive derivations and indefinite patients } \label{sec:antipassive.indefinite.patient}
In Japhug, non-overt objects of (non-labile) transitive verb are always interpreted as transitive, even when a particular verb form happens to lack unambiguous markers of transitivity (§\ref{sec:transitivity.morphology}). The minimal example (\ref{ex:pWBzjoza}) with the \textsc{1sg}\fl{}3 Aorist of \japhug{βzjoz}{learn} could in principle be an intransitive form (due to the fact that the past transitive \forme{-t} suffix cannot surface with a close syllable stem, §\ref{sec:indexation.mixed}). Yet, this example cannot mean `I studied something' or `I went to school', but necessarily implies that the object has been previously mentioned in the discourse. 

\begin{exe}
\ex \label{ex:pWBzjoza}
\gll χsɯ-sla pɯ-βzjoz-a \\
three-months \textsc{aor}-learn-\textsc{1sg} \\
\glt `I have learned it (how to make ploughshares) for three months.' (2010-09, 14)
\end{exe}

The antipassive derivations, which not only demote the object, but suppress it syntactically (except for ditransitive verbs, see §\ref{sec:antipassive.ditransitive}), are a way to express a non-referential indefinite patient, either a non-specific patient (translatable as `something'), or a whole range of possible patients.  For instance in the case of \japhug{rɤβzjoz}{learn things}, the implicit patient can be a school curriculum (`go to school, study')  as in (\ref{ex:pWrABzjoza}) below and (\ref{ex:Wme.pWrABzjoz}) above, or a non-specific topic `study something' as in (\ref{ex:pjArABzjoz.ipfv}) in §\ref{sec:antipassive.pst.ipfv}.

\begin{exe}
\ex \label{ex:pWrABzjoza}
\gll nɯra nɯ-pʰe nɯtɕu, χsɯ-xpa nɯ pɯ-rɤ-βzjoz-a. \\
\textsc{dem}:\textsc{pl} \textsc{3pl}.\textsc{poss}-\textsc{dat} \textsc{dem}:\textsc{loc} three-years \textsc{dem} \textsc{pst}-\textsc{apass}-learn-\textsc{1sg} \\
\glt `I have studied with them for three years.' (160721 XpWN, 30)
\end{exe}

Example (\ref{ex:turAthu.mWnWjAG}) also illustrates the contrast between the transitive verb \japhug{tʰu}{ask}\footnote{Although \japhug{tʰu}{ask} is ditransitive, since it has indirective alignment, it does not behave differently from monotransitive verbs regarding antipassivization (§\ref{sec:antipassive.ditransitive}). } with an overt object \forme{ɯʑɤɣ nɯ} `his own (question)' and the corresponding antipassive verb \japhug{rɤtʰu}{ask questions} with an non-specific patient (the prohibition against asking additional questions anymore is not limited to the one he forgot to ask).

\begin{exe}
\ex \label{ex:turAthu.mWnWjAG}
\gll   ɯʑɤɣ nɯ kɤ-nɯ-tʰu na-nɯ-jmɯt ɲɯ-ŋu, tɕeri nɯ ma tu-rɤ-tʰu mɯ-nɯ-jɤɣ ɲɯ-ŋu, \\
\textsc{3sg}:\textsc{gen} \textsc{dem} \textsc{inf}-\textsc{auto}-ask \textsc{aor}:3\flobv{}-\textsc{auto}-forget \textsc{sens}-be \textsc{lnk} \textsc{dem} apart.from \textsc{ipfv}-\textsc{apass}-ask \textsc{neg}-\textsc{aor}-be.allowed \textsc{sens}-be \\
\glt `He had forgotten to ask his own (question), but he was not allowed to ask questions anymore.' (divination, 2005, 48-9)
\end{exe}


In rarer cases, the demoted patient is referential and semantically recoverable from the context. In (\ref{ex:korAthunW}) for instance, by contrast with (\ref{ex:turAthu.mWnWjAG}), the demoted patient of  \japhug{rɤtʰu}{ask questions} corresponds to the previous sentence (\ref{ex:kWsi.torANgat}): taking the context into consideration the verb form \forme{ko-rɤ-tʰu-nɯ} means `They consulted the lama (about the reason why the man was about to die and what to do about it)'.

\begin{exe}
\ex 
\begin{xlist}
\ex \label{ex:kWsi.torANgat} 
\gll tɤ-tɕɯ nɯ kɯ-si to-rɤŋgat. \\
\textsc{indef}.\textsc{poss}-son \textsc{dem} \textsc{sbj}:\textsc{pcp}-die \textsc{ifr}-be.about \\
\glt `The man (had fallen ill) and was about to die.' (rkongrgyal2.2002, 24)
\ex \label{ex:korAthunW} 
\gll  βlama ɲɤ-ɕar-nɯ tɕe, ko-rɤ-tʰu-nɯ ri \\
lama \textsc{ifr}-search-\textsc{pl} \textsc{lnk} \textsc{ifr}-\textsc{apass}-ask-\textsc{pl} \textsc{lnk} \\
\glt `They looked a lama and consulted him/ask for his advice.' (rkongrgyal2.2002, 25)
\end{xlist}
\end{exe}

Similarly, in (\ref{ex:kWsAndza.toGWGu}), the implicit patient of the verb \forme{sɤ-ndza} `eat (someone)' is not `people' in general, but rather the group of characters present at the moment of the action with the demoness, and it is thus in fact partially referential. 

\begin{exe}
\ex \label{ex:kWsAndza.toGWGu}  
\gll tɕe kɯ-sɤ-ndza tu-oɣɯɣu ri,\\
\textsc{lnk} \textsc{sbj}:\textsc{pcp}-\textsc{apass}-eat \textsc{ipfv}-prepare \textsc{lnk} \\
\glt `(The râkshasî demoness revealed her true nature) and was about to eat someone (one of them), but ...' (28-smAnmi, 397)
\end{exe}

  \subsubsection{The contrast between \forme{rɤ-} and \forme{sɤ-} prefixes  } \label{sec:antipassive.rA.sA}
The semantic difference between the  \forme{rɤ-} and \forme{sɤ-} antipassive prefixes involves a contrast in humanity (following \citealt{jackson06paisheng} on Tshobdun) and animacity.

The  \forme{rɤ-} prefix is required for inanimate/abstract indefinite patients. In the case of verbs of speech, the removed object corresponds to a reported speech complement clause (as in \ref{ex:korAthunW} above, §\ref{sec:reported.speech}). Some \forme{rɤ-} antipassives however have animate non-human patients, in particular  \japhug{rɤftɕɤz}{castrate} and  \japhug{rɤntɕʰa}{butcher} (from  \japhug{ftɕɤz}{castrate}  and  \japhug{ntɕʰa}{butcher}), which can only refer to animals. 

The \forme{sɤ-} prefix by contrast is mainly used to express generic human objects (with a meaning often close to that of the generic object indexation marker \forme{kɯ-} §\ref{sec:non.antipassive.indef.patient}) or indefinite humans (sometimes even semantically recoverable from the context, as in \ref{ex:kWsAndza.toGWGu}). However,  the \forme{sɤ-} prefix when used as a propensity antipassive can demote animal patients, as in (\ref{ex:YWsAsat}) where \forme{sɤ-sat} is a stative verb meaning `have killing power, be lethal' (rather than `kill people').  
 
\begin{exe}
\ex \label{ex:YWsAsat}
\gll tɕe nɯnɯ ju-lɤt-nɯ tɕe tɕe nɯnɯ wuma ʑo ɲɯ-sɤ-sat [...] qro ri ɲɯ-xtɕi ɕti, tɕe nɯnɯ kɯ rcanɯ tɯ-kʰɤl tɕe ʁnɯz χsɯm jamar pjɯ-sat ɲɯ-ŋu \\
\textsc{lnk} \textsc{dem} \textsc{ipfv}-release-\textsc{pl} \textsc{lnk} \textsc{lnk} \textsc{dem} really \textsc{emph} \textsc{sens}-\textsc{apass}-kill {  } pigeon also \textsc{sens}-be.small be.\textsc{aff}:\textsc{fact} \textsc{lnk} \textsc{dem} \textsc{erg} \textsc{unexp}:\textsc{deg} one-place \textsc{loc} two three about \textsc{ipfv}-\textsc{kill} \textsc{sens}-be  \\
\glt `They shoot with (scattering bullets), and it is very lethal, (...) the pigeons are small, (with scattering bullets) they kill two or three (pigeons) in one place.' (28-CAmWGdW, 116)
\end{exe}

In addition, when both agents and patients are (non-human) animals, the prefix \forme{sɤ-} can also be used to demote the object, as in the case of the verb \japhug{sɤmtsʰi}{lead the way} (from \japhug{mtsʰi}{lead}) which can be applied to packs of animals. Note also that the antipassive \forme{sɤndza} from \japhug{ndza}{eat}, can both mean `eat someone' (as in \ref{ex:kWsAndza.toGWGu} above) or `eat other animals' as in (\ref{ex:kWrNi1}) (a sentence which can be glossed by \ref{ex:kWrNi2}).

\begin{exe}
\ex 
\begin{xlist}
\ex \label{ex:kWrNi1}
\gll rɯdaʁ nɯ ɯ-ŋgɯ kɯ-sɤ-ndza nɯ, kɯrŋi tu-kɯ-ti ŋu \\
wild.animals \textsc{dem} \textsc{3sg}.\textsc{poss}-in \textsc{sbj}:\textsc{pcp}-\textsc{apass}-eat \textsc{dem} beast \textsc{ipfv}-\textsc{genr}-say be:\textsc{fact} \\
\glt `Among the animals, the carnivorous ones are called `beasts'.' (elicited, explanation of example \ref{ex:kWsAndza.GW}, §\ref{sec:subject.participle.subject.relative})
\ex \label{ex:kWrNi2}
\gll nɯ-zda rɯdaʁ ɯ-kɯ-ndza tʰamtɕɤt nɯnɯra nɯ-rmi lonba kɯrŋi tu-kɯ-ti ŋu. \\
\textsc{3pl}.\textsc{poss}-companion animal \textsc{3sg}.\textsc{poss}-\textsc{sbj}:\textsc{pcp}-eat all \textsc{dem}:\textsc{pl} \textsc{3pl}.\textsc{poss}-name all beast \textsc{ipfv}-\textsc{genr}-say be:\textsc{fact} \\ 
\glt `All ones that eat the other animals, their name is `beasts'.' (150822 kWrNi, 8)
\end{xlist}
\end{exe}


The human vs. non-human contrast between the two prefixes can be observed on a handful of verbs that are compatible with both \forme{rɤ-} and \forme{sɤ-} antipassives. The indirective verb \japhug{tʰu}{ask} takes the \forme{rɤ-} prefix when the demoted patient is an action/state of affair (as in \japhug{rɤtʰu}{ask questions} in \ref{ex:korAthunW} above), but takes the \forme{sɤ-} prefix when asking for someone in marriage (§\ref{sec:antipassive.ditransitive}), as in (\ref{ex:kWsAthu.Cea}).
 
\begin{exe}
\ex \label{ex:kWsAthu.Cea}
\gll  aʑo kɯ-sɤ-tʰu ɕe-a \\
\textsc{1sg} \textsc{sbj}:\textsc{pcp}-\textsc{apass}-ask go:\textsc{fact}-\textsc{1sg} \\
\glt `I am going to ask for (the girls) in marriage.' (2003 Kunbzang, 5)
\end{exe}

A less lexicalized minimal pair is provided by \japhug{ɕar}{search}, which has the antipassive forms \japhug{rɤɕar}{search for things} and \japhug{sɤɕar}{search for someone}.

\subsubsection{Avoidance} \label{sec:antipassive.avoidance}
The \forme{sɤ-} antipassive can be used as a strategy to avoid overtly referring to a particular entity. In (\ref{ex:znArGAma}) for instance, despite the use of the antipassive \japhug{sɤnɯrtɕa}{tease people} (from \japhug{nɯrtɕa}{tease}), the implicit patient of this verb is not indefinite: it is a type of local deity called  \forme{ʑɯβdaʁ} (from \tibet{གཞི་བདག་}{gʑi.bdag}{local deity}).

\begin{exe}
\ex \label{ex:znArGAma}
\gll tɯ-mɯ wuma ʑo tɤ-me tɕe, tɕendɤre tsʰitsuku ɕ-ku-sɤ-nɯrtɕa-nɯ tɕe, fsaŋ ra ɕ-pjɯ-ta-nɯ, ɕ-pjɯ-rɟaʁ-nɯ nɯra tɕe, tɕe tɯ-mɯ ku-sɯ-lɤt-nɯ pjɤ-ŋgrɤl. tɕe nɯ ɯ-sɤz-nɤma ɣɯ sɤtɕʰa nɯ znɤrɣɤma tu-ti-nɯ ɲɯ-ŋu \\
\textsc{indef}.\textsc{poss}-weather really \textsc{emph} \textsc{aor}-not.exist \textsc{lnk} \textsc{lnk} some.things \textsc{tral}-\textsc{ipfv}-\textsc{apass}-tease-\textsc{pl} \textsc{lnk} fumigation \textsc{pl} \textsc{tral}-\textsc{ipfv}-put-\textsc{pl} \textsc{tral}-\textsc{ipfv}-dance-\textsc{pl} \textsc{dem}:\textsc{pl} \textsc{lnk} \textsc{lnk} \textsc{indef}.\textsc{poss}-weather \textsc{ipfv}-\textsc{caus}-release-\textsc{pl} \textsc{ifr}.\textsc{ipfv}-be.usually.the.case \textsc{lnk} \textsc{dem} \textsc{3sg}.\textsc{poss}-\textsc{obl}:\textsc{pcp}-make \textsc{gen} place \textsc{dem} placename \textsc{ipfv}-say-\textsc{pl} \textsc{sens}-be \\
\glt `When there was no rain at all, people would go there and do teasing, make fumigations and dance, and cause rain to come. They call the place where these activities were performed \forme{znɤrɣɤma}.' (140522 Kamnyu zgo, 246-252)
\end{exe}

Tshendzin explicitly provided the sentence (\ref{ex:ZWBdaR.CkuWwGnWrtCa}) as a gloss to the verb form \forme{ɕ-ku-sɤ-nɯrtɕa-nɯ} in this context.

\begin{exe}
\ex \label{ex:ZWBdaR.CkuWwGnWrtCa}
\gll    ʑɯβdaʁ ɕ-kú-wɣ-nɯrtɕa \\
mountain.god \textsc{tral}-\textsc{ipfv}-\textsc{inv}-tease \\
\glt `One teases the mountain  gods.' (gloss of example \ref{ex:znArGAma}).
\end{exe}
   
The `teasing' in question refers to the belief that rain resulted from the wrath of mountain deities. The aim of this ceremony, rather than appeasing the gods, was to anger them. It has not been practiced for decades, and memory of this practice only survives in this toponym \forme{znɤrɣɤma}, whose etymology is discussed in more detail in §\ref{sec:lexicalized.oblique.participle}.
 
  \subsubsection{Participial forms of antipassive verbs  } \label{sec:antipassive.participle}
The subject participle of antipassive verb is the preferred strategy to build nominals referring to persons with a particular activity of profession, as illustrated by the forms in \tabref{tab:kWrAverb} (see also §\ref{sec:lexicalized.subject.participle}), or having a particular habit (see \ref{ex:qandzxe.thAlwACtsxat}, §\ref{sec:object.verb.compounds}). These nominalized forms are frequently used but not lexicalized, and their meaning is predictable from that of the base verb.

\begin{table}
\caption{Subject participles of antipassive verbs} \label{tab:kWrAverb}
\begin{tabular}{llll}
\lsptoprule 
Base verb & Subject participle of antipassive \\
\midrule
\japhug{βzjoz}{learn} & \forme{kɯ-rɤ-βzjoz} `learner, student, scholar' \\
\japhug{tʂɯβ}{sew} & \forme{kɯ-rɤ-tʂɯβ} `tailor' \\
\japhug{rɤt}{write, draw} & \forme{kɯ-rɤ-rɤt} `writer, painter' \\
\japhug{ntɕʰa}{butcher} & \forme{kɯ-rɤ-ntɕʰa} `butcher' (person who slaughters animals) \\
\midrule
\japhug{sɯxɕɤt}{teach} & \forme{kɯ-sɤ-sɯxɕɤt} `teacher' \\
\japhug{mtsʰi}{lead} & \forme{kɯ-sɤ-mtsʰi} `leader' (in a dance, of a pack of animals) \\
\lspbottomrule
\end{tabular}
\end{table}

Oblique participles (§\ref {sec:lexicalized.oblique.participle}) also need to undergo antipassivization to be used to derive nouns of instruments or of location without overt object. For instance, the antipassive participle \forme{ɯ-z-rɤ-rɤt} (\textsc{3sg}.\textsc{poss}-\textsc{obl}:\textsc{pcp}-\textsc{antip}-write) is used to express the meaning `writing implement, pen' (§\ref {sec:instrumental.participle.relatives}).
 

   \subsubsection{Reflexive use of the antipassive  } \label{sec:antipassive.reflexive}
The indefinite patient interpretation is not the only meaning of the \forme{rɤ-} antipassive derivation. For instance, the verb \japhug{raχtɕi}{wash} (vi) from \japhug{χtɕi}{wash} (vt) does not mean `wash things' as could have been expected, but has a reflexive meaning `wash one's face' or `have a shower/bath' (see example \ref{ex:pjAraXtCinW.torAmpCArnW}, §\ref{sec:rA.non.apass}).\footnote{A typologically similar irregularity is found with the antipassive-durative verb \forme{niza‑lā} `wash oneself' in Bezhta 
\citep[554]{khalilova16bezhta.valency}. } The regular reflexive form \japhug{ʑɣɤχtɕi}{wash oneself} is also attested with this verb (§\ref{sec:refl.other}).

  \subsection{Other strategies used to express indefinite patients } \label{sec:non.antipassive.indef.patient}
As a means of expressing indefinite patients, the antipassive derivations compete with five alternative constructions: indefinite  object pronouns (§\ref{sec:antipassive.vs.indef.pronouns}), generic object nouns (§\ref{sec:antipassive.vs.generic}), light verb constructions with action nominal (§\ref{sec:antipassive.vs.light.verbs}), incorporation (§\ref{sec:antipassive.vs.incorporation}) and pairs of denominal verbs (§\ref{sec:antipassive.vs.denominal.pairs}).  

\subsubsection{Indefinite pronouns } \label{sec:antipassive.vs.indef.pronouns}
Indefinite pronouns (§\ref{sec:indef.pro}), in particular \japhug{tʰɯci}{something} (§\ref{sec:thWci}) or  \japhug{tsʰitsuku}{whatever} (§\ref{sec:tshitsuku}) as object functionally overlap to some extent with the \forme{rɤ-} antipassive derivations. However, indefinite pronouns are preferred if the indefinite patient is referential, as in (\ref{ex:tshitsuku.YAndWn}).\footnote{Examples of antipassive verbs with referential implicit patient are however attested, as in (\ref{ex:kWsAndza.toGWGu}) and (\ref{ex:korAthunW}) above. }

\begin{exe}
\ex \label{ex:tshitsuku.YAndWn}
\gll ɯʑo kɯ tsʰitsuku ɲɤ-ndɯn tɕe kʰɤndɯn ra ɲɤ-βzu \\
\textsc{3sg} \textsc{erg} whatever \textsc{ifr}-read \textsc{lnk} recitation \textsc{pl} \textsc{ifr}-read \\
\glt `He recited something, he recited a formula.' (140510 sanpian yumao-zh, 100)
\end{exe}

\subsubsection{Generic marking  } \label{sec:antipassive.vs.generic}
The generic noun \japhug{tɯrme}{person} (§\ref{sec:tWrme.generic}) and generic person indexation of the object (§\ref{sec:indexation.generic.tr}) compete with the \forme{sɤ-} antipassive. As shown by examples (\ref{ex:mAsAmtsWG}) and (\ref{ex:kW.kWmtsWG})\footnote{On the absence of dual indexation on \forme{kɯ-mtsɯɣ} `they bite people', see §\ref{sec:indexation.generic.tr} and §\ref{sec:agreement.mismatch}. } the meaning of the antipassive \japhug{sɤmtsɯɣ}{bite} (`be a biting/stinging entity') and that of the generic object \forme{kɯ-mtsɯɣ} `$X$ bites people' in the Factual are very close semantically. The antipassive of propensity, being a stative verb, is however more often used in a comparative construction such as that in (\ref{ex:mAsAmtsWG}) (§\ref{sec:sthWci.equative}).  Example \ref{ex:sAndza.kWndza} above (§\ref{sec:antipassive.sA}) presents a minimal pair of the same type.
 
\begin{exe}
\ex \label{ex:mAsAmtsWG}
\gll mtsʰalɤɣrum nɯ ɣɯ ɯ-rme tu ri, [...] mtsʰalɤɲaʁ nɯ stʰɯci mɤ-sɤ-mtsɯɣ. \\
nettle.sp \textsc{dem} \textsc{gen} \textsc{3sg}.\textsc{poss}-hair exist:\textsc{fact} \textsc{lnk} { } nettle.sp \textsc{dem} much \textsc{neg}-\textsc{apass}:\textsc{prop}-bite:\textsc{fact} \\
\glt `Although the white nettle has hairs, it does not sting as much as the black nettle.' (19-mtshalu2, 6)
\end{exe}

\begin{exe}
\ex \label{ex:kW.kWmtsWG}
\gll  mtsʰalɤɲaʁ cʰo mtsʰalɤɣrum nɯ ʁnaʁna kɯ kɯ-mtsɯɣ \\
nettle.sp \textsc{comit} nettle.sp  \textsc{dem} both \textsc{erg} \textsc{genr}:S/O-bite:\textsc{fact} \\
\glt `Both black and white nettle sting.' (11-mtshalu, 26)
\end{exe}

\subsubsection{Nominal+light verbs collocation  } \label{sec:antipassive.vs.light.verbs}
Collocations involving transitive light verbs and their objects have a functional overlap with antipassive verbs, when their objects, either action nominals or nouns describing the product of the action, are generic and not referential. For instance, the collocation \forme{tɤ-ɕpʰɤt+ta} `patch, make patches' (from the inalienable noun \japhug{tɤ-ɕpʰɤt}{patch} and the transitive verb \japhug{ta}{put}) has a meaning close to that of the antipassive \japhug{rɤɕpʰɤt}{patch clothes} (from \japhug{ɕpʰɤt}{patch}, the verb from which the noun \forme{tɤ-ɕpʰɤt} itself is derived, §\ref{sec:bare.action.nominals}) when the possessive prefix on  \forme{tɤ-ɕpʰɤt} is indefinite. As illustrated by example (\ref{ex:tACphAt.kWta}), where the participial form of this collocation \forme{tɤ-ɕpʰɤt ɯ-kɯ-ta} `(person) who makes patches' occurs in opposition to the participle of an antipassive verb \forme{kɯ-rɤ-tʂɯβ} `(person) who sews clothes, taylor'.  

\begin{exe}
\ex \label{ex:tACphAt.kWta}
\gll tɕʰeme [tɤ-ɕpʰɤt ɯ-kɯ-ta] kɯ-xtɕɯ\redp{}xtɕi pjɤ-tu ma nɯ ma kɯ-rɤ-tʂɯβ pjɤ-me. \\
woman \textsc{indef}.\textsc{poss}-patch \textsc{3sg}.\textsc{poss}-\textsc{sbj}:\textsc{pcp}-put  \textsc{sbj}:\textsc{pcp}-\textsc{emph}\redp{}be.small \textsc{ifr}.\textsc{ipfv}-exist \textsc{lnk} \textsc{dem} apart.from \textsc{sbj}:\textsc{pcp}-\textsc{apass}-sew \textsc{ifr}.\textsc{ipfv}-not.exist \\
\glt `There were a few women who made patches, but no (women) tailors.' (12-kAtsxWb, 12)
 \end{exe}
 
 The collocation \forme{tɤ-ɕpʰɤt+ta} can however occur with a \textsc{3sg} possessive prefix on the object \forme{ɯ-ɕpʰɤt+ta} and refer to a definite patient, encoded as possessor of the object. 
 
 Other light verbs occurring in this type of collocation include \japhug{βzu}{make}, \japhug{lɤt}{release} and \japhug{tɕɤt}{take out} (see \ref{ex:tWpGaR.lotCAtndZi2}, §\ref{sec:antipassive.history}).
 
 \subsubsection{Incorporating verbs } \label{sec:antipassive.vs.incorporation}
Object-saturating incorporating verbs (§\ref{sec:incorp.O}) are intransitive verbs derived from transitive bases whose intransitive subject corresponds to the subject of the base verb. The incorporated noun corresponds to the object of the base verb, and is indefinite and non-referential, and thus functionally close to an antipassive, in particular to lexicalized ones (§\ref{sec:antipassive.lexicalized}).\footnote{This closeness in function is also correlated with a closeness in origin, since both incorporating verbs and antipassive derivation come from denominal derivations (§\ref{sec:incorp.denom}). } For instance, the incorporating verb \japhug{ɣɯ<piaozi>fsoʁ}{earn money} from \ch{票子}{piàozi}{ticket, paper money}  and \japhug{fsoʁ}{earn} has a meaning and usage identical to the antipassive \japhug{rɤfsoʁ}{earn money}.

 \subsubsection{Denominal pairs } \label{sec:antipassive.vs.denominal.pairs}
Transitive denominal verbs with the \forme{nɯ-}/\forme{nɤ-} prefix can in some cases have \forme{sɤ-} antipassives (§\ref{sec:antipassive.sA}), but never \forme{rɤ-} antipassives. Instead, intransitive  \forme{rɯ-}/\forme{rɤ-} denominal verbs from the same noun occur as the functional equivalents of the antipassive. For instance, the transitive verb \japhug{nɤma}{do} (work)  derived by the prefix \forme{nɤ-} (§\ref{sec:denom.tr.nW}) from the inalienably possessed noun \japhug{ta-ma}{work} (§\ref{sec:inalienably.possessed.morpho}), does not have an antipassive form such as $\dagger$\forme{rɤ-nɤma}. Rather, the intransitive \japhug{rɤma}{do work} from the same noun (see example §\ref{sec:autoben.permansive}, §\ref{ex:nW.kWnA.pjAnWrAma} and \ref{ex:mAkWmbrAt.YWrAma}, §\ref{sec:inf.converb}) occurs to express the meaning that would have been expected from such an antipassive form (§\ref{sec:denom.rA.pairing}). In addition to the semantic similarity with antipassive verbs, \japhug{rɤma}{do work} is compatible with non-periphrastic Past Imperfective and Inferential Imperfective (§\ref{sec:pst.ifr.ipfv}) like antipassive verbs (§\ref{sec:antipassive.pst.ipfv}).

 
 \subsection{Compatibility with other derivations} \label{sec:antipassive.compatibility}
 The \forme{sɤ-} antipassive derivation can be take as input applicative and tropative verbs. The most commonly attested examples of double derivations are however from verbs whose applicative or tropative prefix is synchronically unanalyzable, such as  \japhug{sɤnɤkʰu}{invite people}  (\ref{ex:YWsAnAkhunW}) and \japhug{sɤnɤkʰe}{bully people}  (\ref{ex:nAtWsAnAkhe}) from  \japhug{nɤkʰu}{invite} (lexicalized applicative of \japhug{akʰu}{call}, §\ref{sec:applicative.lexicalized}) and \japhug{nɤkʰe}{bully} (lexicalized  tropative of \japhug{kʰe}{be stupid}, §\ref{sec:tropative.lexicalized}), respectively.
 
\begin{exe}
\ex \label{ex:YWsAnAkhunW}
\gll  nɯ-ʁjoʁ ra kɯ tɯ-ndza tu-βzu-nɯ nɯra, ɲɯ-sɤ-nɤkʰu-nɯ ra pjɤ-ɕti qʰe,  \\
\textsc{3pl}.\textsc{poss}-servant \textsc{pl} \textsc{erg} \textsc{nmlz}:\textsc{action}-eat \textsc{ipfv}-make-\textsc{pl} \textsc{dem}:\textsc{pl} \textsc{ipfv}-\textsc{apass}-invite-\textsc{pl} \textsc{pl} \textsc{ifr}.\textsc{ipfv}-be.\textsc{aff}:\textsc{fact} \textsc{lnk} \\
\glt `Their servants were making food, they were inviting people.' (28-qAjdoskAt, 136-137)
  \end{exe}
  
 \begin{exe}
\ex \label{ex:nAtWsAnAkhe}
\gll  nɤ-tɯ-sɤ-nɤkʰe nɯ mɤ-ra \\
\textsc{2sg}-\textsc{nmlz}:\textsc{deg}-\textsc{apass}-bully \textsc{dem} \textsc{neg}-be.needed:\textsc{fact} \\
\glt `You should not bully people like that (your bullying of other people should not cross the line).' (28-qAjdoskAt, 14)
 \end{exe}
 
Antipassive forms from non-lexicalized applicative and tropative verbs are not attested in the corpus, but can be elicited, for instance \forme{sɤ-nɯ-rga} \textsc{apass}-\textsc{appl}-like `like people' from \japhug{nɯrga}{like}  (vt) (§\ref{sec:applicative.semi.object}) and \forme{sɤ-nɤ-mpɕɤr} \textsc{apass}-\textsc{trop}-be.beautiful `find people beautiful' from \japhug{nɤmpɕɤr}{find beautiful}  (§\ref{sec:tropative}).

Only one causative verb takes the \forme{rɤ-} antipassive:  \japhug{jtsʰi}{give to drink} (§\ref{sec:antipassive.ditransitive}), an irregular causative from \japhug{tsʰi}{drink} (§\ref{sec:caus.j}), which competes with the regular one \forme{sɯ-tsʰi} `make/let drink, drink with'.

Other causative verbs are only attested with the \forme{sɤ-} antipassive, for instance  \forme{sɤ-sɯ-rtoʁ} \textsc{apass}-\textsc{caus}-look) `show to people'. The lexicalized causative  \japhug{nɯsɯkʰo}{rob, extort}, which historically derives from \japhug{kʰo}{give} by the combination of the sigmatic causative \forme{sɯ-} and the autive \forme{nɯ-} prefixes (see \ref{ex:YWkWnWsWkhoa}, §\ref{sec:sig.caus.lexicalized}), has two \forme{sɤ-} antipassive forms (§\ref{sec:antipassive.ditransitive}): the \forme{sɤ-} can be directly prefixed to the complex verb stem as in \japhug{sɤnɯsɯkʰo}{rob people} (\ref{ex:zYAsAnWsWkhonW}) (historically \forme{sɤ-nɯ-sɯ-kʰo} \textsc{apass}-\textsc{auto}-\textsc{caus}-give), but the alternative form \forme{nɯ-sɤ-sɯ-kʰo} (\textsc{auto}-\textsc{apass}-\textsc{caus}-give) with the autive prefix \forme{nɯ-} switching position with the antipassive is also possible  (§\ref{sec:autoben.lexicalized}).
 
\begin{exe}
\ex  \label{ex:zYAsAnWsWkhonW}
\gll   iɕqʰa tɕaχpa nɯra kɯ li, laχtɕʰa z-ɲɤ-sɤ-nɯsɯkʰo-nɯ  \\
the.aforementioned robber \textsc{dem}:\textsc{pl} \textsc{erg} again thing \textsc{tral}-\textsc{ifr}-\textsc{apass}-rob-\textsc{pl}   \\
\glt `The robbers had gone and robbed some people from their things.' (140512 alibaba-zh, 113)
\end{exe}  

Applicative verbs can serve as input for the sigmatic causative derivation. With the \forme{rɤ-} antipassive, the \forme{z-} allomorph of the causative is selected (§\ref{sec:caus.z}) as in \forme{z-rɤ-rɤt}  (\ref{ex:pjAwGzrArAt}) and \forme{z-rɤ-tʂɯβ} (\ref{ex:chWwGzrArtsxWB})  from \japhug{rɤrɤt}{write/draw things} and \japhug{rɤtʂɯβ}{sew things} (antipassives of \japhug{rɤt}{write, draw} and \japhug{tʂɯβ}{sew}, respectively, §\ref{sec:antipassive.rA}). 

In combination with the antipassive, both the permissive/precative `make $X$, let $X$, ask to $X$' (as in \ref{ex:pjAwGzrArAt}) and the instrumental `$X$ with' (\ref{ex:chWwGzrArtsxWB}) uses of the sigmatic causative are attested.

\begin{exe}
\ex  \label{ex:pjAwGzrArAt}
\gll  Dai.Song ɣɯ, nɤkinɯ, ɯ-ɕaχpu nɯ kɯ ɲɤ́-wɣ-sqɤr tɕe, nɤkinɯ, pjɤ́-wɣ-z-rɤ-rɤt ɲɯ-ŋu \\
\textsc{anthr} \textsc{gen} \textsc{filler} \textsc{3sg}.\textsc{poss}-friend \textsc{dem} \textsc{erg} \textsc{ifr}-\textsc{inv}-ask.to.do \textsc{lnk} \textsc{filler} \textsc{ifr}-\textsc{inv}-\textsc{caus}-\textsc{appl}-draw \textsc{sens}-be \\
\glt `One of Dai Song's friend's asked him to draw something.' (2010-kewen-07, 10)
\end{exe}  
 
 
 \begin{exe}
\ex  \label{ex:chWwGzrArtsxWB}
\gll  tɕe nɯ tɤ-ri ɲɯ́-wɣ-nɯ-βzu tɕe cʰɯ́-wɣ-z-rɤ-tʂɯβ ŋu \\
\textsc{lnk} \textsc{dem} \textsc{indef}.\textsc{poss}-thread \textsc{ipfv}-\textsc{inv}-\textsc{auto}-make \textsc{lnk} \textsc{ipfv}-\textsc{inv}-\textsc{caus}-\textsc{apass}-sew \textsc{be}:\textsc{fact} \\
\glt `One (can) then make the thread and use it to sew things.' (13-tAsAsqAri, 40)
\end{exe}  

\section{Distributed property} \label{sec:distributed.amW}
The \forme{amɯ-} prefix expressing a distributed property can be applied to a handful of transitive and semi-transitive verbs.
 
With the cognition verbs \japhug{sɯz}{know} and \japhug{tso}{know, understand} (semi-transi\-tive, see §\ref{sec:semi.transitive}), the meaning of this derivation is `be $X$ed by everybody, be $X$able', as shown by \japhug{amɯtso}{be clear (of speech)} (\ref{ex:mAZW.amWtso}) and \japhug{amɯsɯz}{be well-known}.

\begin{exe}
\ex \label{ex:mAZW.amWtso}
 \gll nɯ tu-kɯ-ti tɕe mɤʑɯ amɯ-tso. \\
 \textsc{dem} \textsc{ipfv}-\textsc{genr}-say \textsc{lnk} even.more \textsc{distr}-understand:\textsc{fact} \\
 \glt  `If one says this, it is clearer (easier to understand).'  (heard several times during elicitation sessions)
\end{exe}

The verb \forme{amɯtso} also has a reciprocal reading `understand each other', corresponding to the \forme{amɯ-} reciprocal derivation (example \ref{ex:pjWkAmWtso.YWra}, §\ref{sec:amW.reciprocal}). It is possible that \forme{amɯtso} was the pivot form between the two derivations. A reanalysis from reciprocal to `distributed property' may have taken place through the causative of the reciprocal \forme{sɯ-ɤmɯ-tso} `cause people to understand each other', which originally took as objects causee (see \ref{ex:kAsAmWtsotandZi}) and as optional semi-object the speech/words. 

\begin{exe}
\ex \label{ex:kAsAmWtsotandZi}
 \gll a-pi cʰo a-ʁi ni kɤ-sɯ-ɤmɯ-tso-t-a-ndʑi  \\
\textsc{1sg}.\textsc{poss}-elder.sibling \textsc{comit} \textsc{1sg}.\textsc{poss}-younger.sibling \textsc{du} \textsc{aor}-\textsc{caus}-\textsc{recip}-understand-\textsc{pst}:\textsc{tr}-\textsc{1sg}-\textsc{du} \\
\glt `I helped my elder and younger brothers/sisters understanding each other.' (elicited)
\end{exe}

The semi-object was then reinterpreted as the object, and forms such as \forme{ta-sɯ-ɤmɯ-tso} in (\ref{ex:tasAmWtso}) were then reinterpreted as meaning `cause (words) to be understandable' rather than `cause people to understand (words)'.\footnote{Note that the meanings of the causative \forme{sɯ-ɤmɯ-tso} are correlated with different orientation prefixes: \textsc{eastwards} when meaning `cause to understand each other' and `up' when meaning `speak clearly'. } The distributed property meaning of \forme{amɯ-} was then created by back-formation from this causative form.

\begin{exe}
\ex \label{ex:tasAmWtso}
 \gll tɯ-rju ta-sɯ-ɤmɯ-tso \\
 \textsc{indef}.\textsc{poss}-word \textsc{aor}:3\flobv{}-\textsc{caus}-???-understand \\
 \glt `He spoke clearly'. (Literally: `He made the words understandable'; elicited)
 \end{exe}

When occurring on other verbs, however, the \forme{amɯ-} prefix expresses that the action spreads everywhere (within a particular location), without external agent. The verb \forme{amɯ-rmbɯ} (derived from \japhug{rmbɯ}{pile up}) is specifically used to describe food piled up high in a container (bowl or pot) to the point of filling it up completely, as in (\ref{ex:tAmWrmbW}), in particular as the result of cooking.
 
 \begin{exe}
\ex \label{ex:tAmWrmbW}
 \gll fsosoz ndɤre, tɤ-amɯ-rmbɯ ʑo. \\
 next.morning \textsc{lnk} \textsc{aor}-\textsc{distr}-pile.up \textsc{emph} \\
\glt `The next morning, (the bowl) was completely filled (with food).' (2003kandZislama, 145)
 \end{exe}

The verb \forme{amɯ-zwɤr} (from \japhug{zwɤr}{burn}) occurs to refer to the (non-controlled) spread of fire, as in (\ref{ex:YAmWzwAr.mbat}).  

 \begin{exe}
\ex \label{ex:YAmWzwAr.mbat}
 \gll taŋi tɤ-ɣe tɕe ɣndʑɤβ a-mɤ-tɤ-lɯɣ ra ma ɲɯ-ɤmɯ-zwɤr mbat \\
 drought \textsc{aor}-come[II] \textsc{lnk} fire.hazard \textsc{irr}-\textsc{neg}-\textsc{pfv}-come.off be.needed:\textsc{fact} \textsc{lnk} \textsc{ipfv}-\textsc{distr}-burn be.easy:\textsc{fact} \\
 \glt  `When there is a drought, there should not be a fire as it can spread easily (it this case).' (elicited)
  \end{exe}


It is also used metaphorically to describe the spread of information as in (\ref{ex:YAkAmWzwArci.YAkAmWsWzci}). This example illustrates two verbs with distinct sub-functions of the distributed property \forme{amɯ-} prefix occurring with a common subject, suggesting a path of reanalysis from the first sub-function (`be known by everybody'  $\rightarrow$ `be known to everybody') to the second one (`be $X$ed everywhere').

 \begin{exe}
\ex \label{ex:YAkAmWzwArci.YAkAmWsWzci}
 \gll kɯki ɣɯ ɯ-tɕha nɯnɯ, nɯɕimɯma ʑo rɟɤlkʰɤβ nɯtɕu ɲɤ-k-ɤmɯ-zwɤr-ci tɕe ɲɤ-k-ɤmɯ-sɯz-ci. \\
 \textsc{dem}.\textsc{prox} \textsc{gen} \textsc{3sg}.\textsc{poss}-news \textsc{dem} immediately \textsc{emph} kingdom \textsc{dem}:\textsc{loc} \textsc{ifr}-\textsc{peg}-\textsc{distr}-burn-\textsc{peg} \textsc{lnk} \textsc{ifr}-\textsc{peg}-\textsc{distr}-know-\textsc{peg} \\
 \glt `The news about this spread in the whole kingdom (like fire) and became known to everyone.' (150820 meili de meiguihua-zh, 84)
\end{exe}

The verb \japhug{amɯzɣɯt}{be evenly distributed}, derived from the motion verb \japhug{zɣɯt}{reach, arrive}, also illustrates the spatially distributed meaning  of the prefix \forme{amɯ-} (originally `reach everywhere'). Note that despite the fact that the root \forme{zɣɯt} generally causes a \ipa{ɯ} \fl{} \ipa{ɤ} vowel change on derivational and inflectional prefixes directly attached to it (§\ref{sec:intr.person.irregular}), it is not the case with \forme{amɯ-}.

This verb \forme{amɯzɣɯt} mainly occurs in a serial verb construction (§\ref{sec:svc.manner}). In  (\ref{ex:YAmWzGWt.YWlhoR}), it shares the same subject and TAME category as the following verb \forme{ɲɯ-ɬoʁ} `(it) comes out'. It expresses the manner in which the action takes place, and has to be translated as the adverb `evenly'.

\begin{exe}
\ex \label{ex:YAmWzGWt.YWlhoR}
\gll  tɤ-ndɤr nɯnɯ, [...] nɯ-βri rcanɯ, kɯ-so ɲɯ-me ʑo, ɲɯ-ɤmɯ-zɣɯt ʑo ɲɯ-ɬoʁ ɲɯ-ŋu. \\
\textsc{indef}.\textsc{poss}-pustule \textsc{dem} { } \textsc{3pl}.\textsc{poss}-body \textsc{unexp}:\textsc{deg} \textsc{sbj}:\textsc{pcp}-be.empty \textsc{ipfv}-not.exist \textsc{emph} \textsc{ipfv}-\textsc{distr}-reach \textsc{emph} \textsc{ipfv}-come.out \textsc{sens}-be \\
\glt `The pustules (...) come out evenly everywhere on their body, without any empty spot.' (27-kharwut, 63-64)
\end{exe} 

When occurring with a transitive verb in this serial verb construction, the sigmatic causative form \japhug{sɤmɯzɣɯt}{do evenly} is used instead, as in (\ref{ex:YWsAmWzGWt.YWmar}) (§\ref{sec:svc.manner.other}).

\begin{exe}
\ex \label{ex:YWsAmWzGWt.YWmar}
\gll tɕe tamar kɯnɤ, nɤki tɕʰorzi nɯ ɯ-mŋu me,  ɯ-qa me nɯra ɲɯ-sɯ-ɤmɯ-zɣɯt ʑo ɲɯ-mar ɲɯ-ra  \\
\textsc{lnk} butter also \textsc{filler} alcohol.jar \textsc{dem} \textsc{3sg}.\textsc{poss}-opening whether \textsc{3sg}.\textsc{poss}-bottom whether \textsc{dem}:\textsc{pl} \textsc{ipfv}-\textsc{caus}-\textsc{distr}-reach \textsc{emph} \textsc{ipfv}-smear \textsc{sens}-be.needed \\
\glt `(The jar maker) also to apply butter evenly to the opening and the  bottom of the alcohol jar.' (30-kWrAfcAr, 65-66)
\end{exe} 

The distributed property \forme{amɯ-} prefix is restricted to a handful of very lexicalized verbs. Hence, if the hypothesis of reanalysis from reciprocal presented above is valid, this process must have taken place in the remote past. The opposite hypothesis of reanalysis from the distributed property function to the reciprocal function also deserves to be taken into consideration.

\section{Proprietive} \label{sec:proprietive}
The proprietive \forme{sɤ-} prefix\footnote{This category was previously referred to as `deexperiencer' in previous publications, such as \citet{jacques12demotion}. } derives stative verbs from verbs of perception, feeling or some verbs of involuntary action. \tabref{tab:proprietive} presents representative examples of this derivation, which include some loanwords from Tibetan (\japhug{scit}{be happy}, \japhug{rga}{like} and 
\japhug{βzi}{be drunk} from \tibet{སྐྱིད་}{skʲid}{be happy}, \tibet{དགའ་}{dga}{like} and \tibet{བཟི་}{bzi}{be drunk}, respectively), showing that the proprietive prefix is productive.
 
\begin{table}
\caption{Examples of proprietive verbs in Japhug} \label{tab:proprietive}
\begin{tabular}{lllllll}
\lsptoprule
Base verb & Derived verb \\
\midrule
\japhug{mtsɯr}{be hungry} & \japhug{sɤmtsɯr}{be a famine} \\
\midrule
\japhug{ŋgio}{slip} & \japhug{sɤŋgio}{be slippery} \\
\japhug{aʁdɤt}{slip} & \japhug{saʁdɤt}{be slippery} \\
\japhug{ɕke}{get burned} & \japhug{sɤɕke}{be burning} \\
\japhug{scit}{be happy} & \japhug{sɤscit}{be pleasant} \\
\japhug{βzi}{be drunk} & \japhug{sɤβzi}{be very intoxicating} \\
\japhug{ɲat}{be tired} & \japhug{sɤɣɲat}{be exhausting} \\
\japhug{mu}{be afraid} & \japhug{sɤɣmu}{be frightening} \\
\japhug{dɯɣ}{have enough of} & \japhug{sɤɣdɯɣ}{be unpleasant} \\
\midrule
\japhug{rga}{like} & \japhug{sɤrga}{be adorable} \\
\japhug{tso}{know, understand} & \japhug{sɤtso}{be understandable} \\
\japhug{nɤz}{dare} & \japhug{sɤnɤz}{be such that people dare to} \\
\japhug{cʰa}{can} & \japhug{sɤcʰa}{be such that people can} \\
\midrule
\japhug{mto}{see} & \japhug{sɤmto}{be visible} \\
\japhug{sɯz}{know} & \japhug{sɤsɯz}{be known} \\
\japhug{spa}{be able to} & \japhug{sɤspa}{be known} \\
\japhug{mtsʰɤm}{hear} & \japhug{sɤmtsʰɤm}{be audible} \\
\japhug{rndu}{obtain} & \japhug{sɤrndu}{be easy to find} \\
\japhug{nɯzdɯɣ}{worry about} & \japhug{sɤnɯzdɯɣ}{causing people to worry} \\
\lspbottomrule
\end{tabular}
\end{table}


The base verbs are mainly intransitive (proprietive verbs derived from semi-transitive and transitive verbs are treated in §\ref{sec:proprietive.semi.tr} and §\ref{sec:proprietive.tr}), and encode as subject the experiencer (\japhug{mu}{be afraid}, \japhug{scit}{be happy}) or a patientive argument suffering from the action (\japhug{ɕke}{get burned}, \japhug{ŋgio}{slip}). 

The proprietive derivation removes this experiencer or patientive argument from the argument structure and promotes instead the stimulus to subject status. Compare for instance the base verb \japhug{mu}{be afraid} whose subject is the experiencer feeling fear (\ref{ex:pjAmu}) with the proprietive \japhug{sɤɣmu}{be frightening} (\ref{ex:pjAsAGmu.pjAnWGmu}) whose subject is the entity causing fear to people. 


\begin{exe}
\ex \label{ex:pjAmu}
\gll nɯ rgɤtpu nɯ pjɤ-mu tɕe, \\
\textsc{dem} old.man \textsc{dem} \textsc{ifr}.\textsc{ipfv}-be.afraid \textsc{lnk} \\
\glt `The old man was afraid.' (140426 xiaohaizi he hua de shizi-zh, 10)
\end{exe}


Example (\ref{ex:pjAsAGmu.pjAnWGmu}) also shows that the subject of the proprietive verb \forme{sɤɣ-mu} is the same referent as the object of the applicative verb \forme{nɯɣ-mu} `be afraid of' derived from the same verb root (§\ref{sec:applicative.promoted}).

\begin{exe}
\ex \label{ex:pjAsAGmu.pjAnWGmu}
\gll sɯŋgi nɯnɯ pjɤ-sɤɣ-mu tɕe pjɤ-nɯɣ-mu ɲɯ-ŋu.\\
lion \textsc{dem} \textsc{ifr}.\textsc{ipfv}-\textsc{prop}-be.afraid \textsc{lnk} \textsc{ifr}.\textsc{ipfv}-\textsc{appl}-be.afraid \textsc{sens}-be \\
\glt `The lion$_i$ was terrifying and (the old man) was afraid of it$_i$.' (shizi yu nongfu-zh, 7)
\end{exe}


A minority of proprietive verbs allow the demoted experiencer to be encoded with the genitive, for instance \japhug{sɤscit}{be pleasant} (from \japhug{scit}{be happy}) in (\ref{ex:aZWG.mAsAscit}), which takes as oblique experiencer \forme{aʑɯɣ} \textsc{1sg}:\textsc{gen}.

\begin{exe}
\ex \label{ex:aZWG.mAsAscit}
\gll  nɯnɯ ɯ-tɕɯ nɯ fso tʰɯ-wxti tɕe tha aʑɯɣ mɤ-sɤ-scit \\
\textsc{dem} \textsc{3sg}.\textsc{poss}-son \textsc{dem} in.the.future \textsc{aor}-be.big \textsc{lnk} later \textsc{1sg}:\textsc{gen} \textsc{neg}-\textsc{prop}-be.happy:\textsc{fact} \\
\glt `When his son has grown up, it will not be a pleasant situation for me.' (28-smAnmi, 18)
\end{exe}
 

In the case of the intransitive verbs \japhug{mtsɯr}{be hungry} and \japhug{ɕpaʁ}{be thirsty}, the proprietive derivation removes the experiencer from the argument structure of the verb without adding a stimulus and the resulting proprietive verbs \japhug{sɤmtsɯr}{be a famine} and \japhug{sɤɕpaʁ}{be a lack of drink} have dummy subjects, as in (\ref{ex:pjAsAmtsWr}).

\begin{exe}
\ex \label{ex:pjAsAmtsWr}
\gll tɯ-xpa tɕe, wuma pjɤ-sɤ-mtsɯr \\
one-year \textsc{lnk} really \textsc{ipfv}.\textsc{ifr}-\textsc{prop}-be.hungry \\
\glt `One year, there was a famine.' (elicited)
\end{exe}

\subsection{Allomorphy} \label{sec:proprietive.allomorphy}
The main allomorph of the proprietive prefix is \forme{sɤ-}. The variant \forme{sa-} occurs when prefixed on a verb stem containing a cluster with a uvular preinitial (§\ref{sec:A.vs.a.prefixes}). 

The proprietive prefix, like other derivations (§\ref{sec:caus.sWG})), had at an earlier stage the allomorph \forme{sɤɣ-} with intrusive \forme{-ɣ}, but it ceased to be productive and remains on only three verbs: \japhug{sɤɣɲat}{be exhausting}, \japhug{sɤɣmu}{be frightening} and \japhug{sɤɣdɯɣ}{be unpleasant}.

\subsection{Proprietive derivation and generic marking} \label{sec:proprietive.generic}
With verbs that can undergo proprietive derivation (see above), the generic intransitive subject \forme{kɯ-} form and proprietive have overlapping uses. 

In (\ref{ex:kWNGio.YWsaRdAt}), the generic form \forme{kɯ-ŋgio} `one will slip' expresses a potential consequence of the state described by the proprietive verb \forme{ɲɯ-sɤ-aʁdɤt} `it is slippery'. A semantic commonality between generic and proprietive verb forms in this example is that both refer to an action to which all humans (including speaker and addressee) are potentially subjected.

\begin{exe}
\ex \label{ex:kWNGio.YWsaRdAt}
\gll nɯnɯtɕu pjɯ́-wɣ-rɤtɕaʁ a-pɯ-ŋu tɕe, [...] ʑgrɯɣ ʑo kɯ-ŋgio ɕti ma, nɯnɯ ɲɯ-sɤ-aʁdɤt kɯ-fse. \\
\textsc{dem}:\textsc{loc} \textsc{ipfv}-\textsc{inv}-tread \textsc{irr}-\textsc{ipfv}-be \textsc{lnk} { } certainly \textsc{emph} \textsc{genr}:S/O-\textsc{acaus}:glide be.\textsc{aff}:\textsc{fact} \textsc{lnk} \textsc{dem} \textsc{sens}-\textsc{prop}-slip \textsc{sbj}:\textsc{pcp}-be.like \\
\glt `If one walks on it (the moss), one will certainly slip, as it is slippery.' (03-zhenzhuquan-zh, 23)
\end{exe}

With experiencer and modal verbs, the semantic closeness of generic and proprietive is even more obvious, if one compares for instance the generic of \japhug{nɤz}{dare} (\ref{ex:kAndza.mWpWkWnAz}) with the proprietive  \japhug{sɤnɤz}{be such that people dare to}  in (\ref{ex:kAnAjaR.mAsAnAz}): the experiencer argument of the base verb that has been demoted by the proprietive derivation is by default interpreted as generic, unless it can be recovered from the context, or expressed by an oblique case as in (\ref{ex:aZWG.mAsAscit}) above.

\begin{exe}
\ex \label{ex:kAndza.mWpWkWnAz}
\gll [kɤ-ndza] mɯ-pɯ-kɯ-nɤz ma sɤndɤɣ tu-ti-nɯ ɲɯ-ŋu. \\
\textsc{inf}-eat \textsc{neg}-\textsc{pst}.\textsc{ipfv}-\textsc{genr}:S/A-dare \textsc{lnk} be.poisonous:\textsc{fact} \textsc{ipfv}-say-\textsc{pl} \textsc{sens}-be \\
\glt `We/people would not dare to eat it, because they say that it is poisonous.' (19-khWlu, 87)
\end{exe}

In (\ref{ex:kAnAjaR.mAsAnAz}), note that generic reference is expressed by the proprietive derivation on \forme{mɤ-sɤ-nɤz} `one does not dare to...' and by the generic possessor on \forme{tɯ-jaʁ} `one's hand' in the immediately following clause.

\begin{exe}
\ex \label{ex:kAnAjaR.mAsAnAz}
\gll tɕe [kɤ-nɤjaʁ] mɤ-sɤ-nɤz ʑo ma tɯ-jaʁ ku-otsa ɕti tɕe mŋɤm. \\
\textsc{lnk} \textsc{inf}-touch \textsc{neg}-\textsc{prop}-dare:\textsc{fact} \textsc{emph} \textsc{lnk} \textsc{genr}.\textsc{poss}-hand \textsc{ipfv}-prick.into be.\textsc{aff}:\textsc{fact} \textsc{lnk} hurt:\textsc{fact} \\
\glt `It is such that people do not dare to touch it, as it pricks in one's hand and it hurts.' (15-babW, 58)
\end{exe}

 In addition, both the generic and the proprietive can serve as a indirect way to express first person (§\ref{sec:1.genr}), as shown by examples (\ref{ex:kArAZi.YWkWdWG}) and (\ref{ex:aZosti.YWsAGdWG}), where the generic \forme{ɲɯ-kɯ-dɯɣ} and proprietive \forme{ɲɯ-sɤɣ-dɯɣ} have exactly the same meaning `be fed up with $X$, don't feel like $X$'. In both cases, the implicit experiencer is the \textsc{1sg} referent mentioned in the previous clause.

\begin{exe}
\ex \label{ex:kArAZi.YWkWdWG}
\gll nɯfse kɯ-nɤ-ŋkɯ\redp{}ŋke ɕe-a ma tɕe kʰa kɤ-rɤʑi ntsɯ ɲɯ-kɯ-dɯɣ \\
like.that \textsc{sbj}:\textsc{pcp}-\textsc{distr}:walk go:\textsc{fact}-\textsc{1sg} \textsc{lnk} \textsc{lnk} house \textsc{inf}-stay always \textsc{sens}-\textsc{genr}:S/O-have.enough.of \\
\glt `I go to walk (without a special reason), because I don't feel well staying at home all the time.' (conversation, 2013-12-24)
\end{exe}

\begin{exe}
\ex \label{ex:aZosti.YWsAGdWG}
\gll kutɕu aʑo-sti ɲɯ-ɕti-a tɕe ɲɯ-sɤɣ-dɯɣ \\
\textsc{dem}:\textsc{prox}:\textsc{loc} \textsc{1sg}-alone \textsc{sens}-be.\textsc{aff}-\textsc{1sg} \textsc{lnk} \textsc{sens}-\textsc{prop}-have.enough.of \\
\glt `I am here alone, and I am fed up of that (the fact of being alone make one feel fed up).' (07-deluge, 42)
\end{exe}

\subsection{Proprietive derivations from semi-transitive verbs} \label{sec:proprietive.semi.tr}
The proprietive derivation takes some semi-transitive verbs such as \japhug{dɯɣ}{have enough of}, \japhug{tso}{know, understand} or \japhug{rga}{like} (§\ref{sec:semi.transitive}) as input. In such cases, the intransitive subject of the proprietive verb corresponds to the semi-object of the base verb. For instance, in (\ref{ex:kArAZi.YWdWGa}), the infinitival complement clause \forme{nɯ kɯ-fse kɤ-rɤʑi} `staying like that' is semi-object of \japhug{dɯɣ}{have enough of}, while in (\ref{ex:kArAZi.pjAsAGdWG}) the complement clause is the intransitive subject of \forme{sɤɣ-dɯɣ} (the experiencer being unexpressed in this clause).

\begin{exe}
\ex \label{ex:kArAZi.YWdWGa}
\gll [nɯ kɯ-fse kɤ-rɤʑi] ɲɯ-dɯɣ-a. \\
\textsc{dem} \textsc{sbj}:\textsc{pcp}-be.like \textsc{inf}-stay \textsc{sens}-have.enough.of-\textsc{1sg} \\
\glt `I am fed up of living like that.' (140426 jiagou he lang-zh, 36)
\end{exe}

\begin{exe}
\ex \label{ex:kArAZi.pjAsAGdWG}
\gll [kɤ-rɤʑi] wuma ʑo pjɤ-sɤɣ-dɯɣ. \\
\textsc{inf}-stay really \textsc{emph} \textsc{ifr}.\textsc{ipfv}-\textsc{prop}-have.enough.of \\
\glt `Living (there) was very difficult to endure.' (150827 taisui-zh, 13)
\end{exe}

The semi-transitive modal verbs \japhug{cʰa}{can} and \japhug{nɤz}{dare} also have proprietive forms     \japhug{sɤcʰa}{be such that people can} and \japhug{sɤnɤz}{be such that people dare to} which select  the same types of complement clauses as those selected by their base verbs, as illustrated by (\ref{ex:luwGtaR.YWsAcha}) (see also \ref{ex:kAndza.mWpWkWnAz}) and (\ref{ex:kAnAjaR.mAsAnAz}) in §\ref{sec:proprietive.generic} above). These complement clauses are the intransitive subject of the proprietive verbs.
  
\begin{exe}
\ex \label{ex:luwGtaR.YWsAcha}
\gll  tɯ-sŋi [mɯntoʁ ʁnɯz χsɯm jamar lú-wɣ-taʁ] ɲɯ-sɤ-cʰa \\
one-day flower two three about \textsc{ipfv}:\textsc{upstream}-\textsc{inv}-weave  \textsc{sens}-\textsc{prop}-can \\
\glt `In one day, it is possible to weave two or three patterns (on the belt).' (2011-06-thaXtsa, 52)
\end{exe}

However, the complement clauses can be elided, and the subject participle forms of the proprietive verbs can refer to an argument (generally object) of the  (elided or overt) verb in the complement clause. For instance in (\ref{ex:kWsAcha.mAkWsAcha}), the participle \forme{kɯ-sɤ-cʰa} can be translated as `(the mice) that can be (caught)' (an elided infinitive form such as \forme{kɤ-ndo} \textsc{inf}-grab is implicit in this example).

\begin{exe}
\ex \label{ex:kWsAcha.mAkWsAcha}
\gll βʑɯ [...], kɯ-sɤ-cʰa nɯra aʑo pjɯ-sat-a ŋu, mɤ-kɯ-sɤ-cʰa jɤ-kɯ-ɤ<nɯ>ri nɯra nɯʑora kɯ pɯ-sat-nɯ ra nɤ \\
mouse {   } \textsc{sbj}:\textsc{pcp}-\textsc{prop}-can \textsc{dem}:\textsc{pl} \textsc{1sg} \textsc{ipfv}-kill-\textsc{1sg} be:\textsc{fact} \textsc{neg}-\textsc{sbj}:\textsc{pcp}-\textsc{prop}-can \textsc{aor}-<\textsc{auto}>go[II] \textsc{dem}:\textsc{pl} \textsc{2pl} \textsc{erg} \textsc{imp}-kill-\textsc{pl} be.needed:\textsc{fact} \textsc{sfp} \\
\glt `The mice, I will kill the ones that (I) can (get), but the one that (I) cannot (get) and have gone away, you will have to kill them.' (150831 BZW kAnArRaR, 32)
\end{exe}

\subsection{Proprietive derivations from transitive verbs} \label{sec:proprietive.tr}
The proprietive prefix occurs with some transitive verb stems, removing the transitive subject and converting the object  argument to intransitive subject status, like the passive, anticausative and object-oriented facilitative (§\ref{sec:facilitative.nWGW}) derivations. Proprietive verbs derived from transitive verbs are rare: the derivational \forme{sɤ-} prefix on verbs is more often interpreted as an object-suppressing antipassive (§\ref{sec:antipassive.sA}).

In the case of the labile perception verb \japhug{mto}{see}, one could propose that the proprietive form \japhug{sɤmto}{be visible} derives from the stative intransitive use of this verb (meaning `have sharp eyesight', §\ref{sec:labile.tr-intr}), but this is unlikely from a semantic point of view (the meaning of \forme{sɤmto} is not `be such that people have sharp eyesight'). The other transitive verbs deriving proprietive forms in \tabref{tab:proprietive} are not labile, and no such ambiguity exists.

The transitive verbs that can be subjected to the proprietive derivation belong to four related semantic classes: verbs of perception (\japhug{mto}{see}, \japhug{mtsʰɤm}{hear}), of cognition (\japhug{sɯz}{know}, \japhug{spa}{be able to}), of obtaining (\japhug{rndu}{obtain}) and of evaluation (\japhug{nɯzdɯɣ}{worry about}, \japhug{saχpaʁ}{respect}). All of these verbs have experiencer or experiencer-like transitive subjects.

The proprietive of verbs of perception and obtaining such as \japhug{sɤmtsʰɤm}{be audible} (\ref{ex:tumbri.YWsAmtshAm}) and \japhug{sɤrndu}{be easy to find} (\ref{ex:tAjmAG.mWjsArndu}) express a potential state, semantically close to the object-oriented facilitative (§\ref{sec:facilitative.nWGW}).

\begin{exe}
\ex \label{ex:tumbri.YWsAmtshAm}
\gll tɕe sɯŋgɯ ku-rɤʑi ɕti tɤɣa kɤ-mto me. ri [tu-mbri] nɯ sɤ-mtsʰɤm.\\
\textsc{lnk} forest \textsc{ipfv}-stay be.\textsc{aff}:\textsc{fact} in.the.open \textsc{obj}:\textsc{pcp}-see not.exist:\textsc{fact} \textsc{lnk} \textsc{ipfv}-make.noise \textsc{dem} \textsc{prop}-hear:\textsc{fact} \\
\glt `It stays in the forest and is never seen in the open, but it can be heard singing.' (23-scuz, 119-120)
\end{exe}

\begin{exe}
\ex \label{ex:tAjmAG.mWjsArndu}
\gll tɯ-mɯ mɯ́j-lɤt tɕe tɤjmɤɣ mɯ́j-sɤ-rndu \\
\textsc{indef}.\textsc{poss}-sky \textsc{neg}:\textsc{sens}-release \textsc{lnk} mushroom \textsc{neg}:\textsc{sens}-\textsc{prop}-obtain \\
\glt `It is not raining, and so it is not easy to find mushrooms.' (elicited)
\end{exe}

The proprietive of \japhug{sɯz}{know} and \japhug{spa}{be able to} both mean `be (widely) known', and mainly occur with the noun \japhug{tɤ-rmi}{name}, as in (\ref{ex:Wrmi.YWsAsWz}) and (\ref{ex:Wrmi.mAkWsAspa}).

\begin{exe}
\ex \label{ex:Wrmi.YWsAsWz}
\gll ɯ-rmi ɲɯ-sɤ-sɯz \\
\textsc{3sg}.\textsc{poss}-name \textsc{sens}-\textsc{prop}-know \\
\glt `His name is well-known.' (elicited)
\end{exe}

\begin{exe}
\ex \label{ex:Wrmi.mAkWsAspa}
\gll mɤʑɯ ɯ-rmi mɤ-kɯ-sɤ-spa xcat ɕti \\
again \textsc{3sg}.\textsc{poss}-name \textsc{neg}-\textsc{sbj}:\textsc{pcp}-\textsc{prop}-be.able be.many:\textsc{fact} be.\textsc{aff}:\textsc{fact} \\
\glt `There are many other (plants) whose name is not known.' (08-tWrgi, 54)
\end{exe}
 
The case of the verb \japhug{sɤnɯzdɯɣ}{causing people to worry}, which is derived from applicative verb \japhug{nɯzdɯɣ}{worry about} (§\ref{sec:applicative}) is treated in §\ref{sec:proprietive.compatibility}.
  
%\japhug{saχpaʁ}{respect} \japhug{sɤsaχpaʁ}{be respected}

\subsection{Lexicalized proprietive} \label{sec:proprietive.lexicalized}
The meaning of some proprietive verbs is not entirely predictable from their base verbs. 

The verb \japhug{ɕke}{get burned} encodes as subject the entity that suffers from burning, as in (\ref{ex:pWCke}). The expected meaning of its proprietive \forme{sɤɕke} would be `be burning'. While this meaning is indeed attested as in (\ref{ex:pjAsACke}), \forme{sɤɕke} can also simply mean `be hot', concerning for instance the weather as in (\ref{ex:tukWNke.YWsACKe}), without the implication that people subjected to this weather suffer from burns.

\begin{exe}
\ex \label{ex:pWCke}
\gll a-jaʁ pɯ-ɕke \\
\textsc{1sg}.\textsc{poss}-hand \textsc{aor}-burn \\
\glt `My hand was burnt.' (elicited)
\end{exe}

\begin{exe}
\ex \label{ex:pjAsACke}
\gll smi tʰa-βlɯ-nɯ ri, pjɤ-sɤ-ɕke qʰe, ʑakastaka jo-nɯ-rɤɕi-nɯ qʰe, \\
fire \textsc{aor}:3\flobv{}-burn-\textsc{pl} \textsc{lnk} \textsc{ifr}.\textsc{ipfv}-\textsc{prop}-burn \textsc{lnk} each.their.own \textsc{ifr}-\textsc{auto}-pull-\textsc{pl} \textsc{lnk} \\
\glt `They made a fire (and used their legs as tripods to make tea), but since it was burning, and each of them pulled (their legs and were not able to prepare food).' (2014-kWlAG, 333)
\end{exe}

\begin{exe}
\ex \label{ex:tukWNke.YWsACKe}
\gll <diandian> ʑo kɯ-rɤʑi ɲɯ-mɯɕtaʁ, tu-kɯ-ŋke tɕe ɲɯ-sɤ-ɕke \\
shop \textsc{emph} \textsc{genr}:S/O-stay \textsc{sens}-be.cold \textsc{ipfv}-\textsc{genr}:S/O-walk \textsc{lnk} \textsc{sens}-\textsc{prop}-burn \\
\glt `Staying in the shop it is cold, but walking (on the street) it is hot.' (conversation, 14-05-10)
\end{exe}

\subsection{Compatibility with other derivations} \label{sec:proprietive.compatibility}
The proprietive derivation can take as input anticausative verbs (§\ref{sec:anticausative.other.derivations}). For instance,  \japhug{sɤŋgio}{be slippery} comes from  \japhug{ŋgio}{slip}, itself from \japhug{kio}{cause to slip}. The proprietive verb \forme{sɤŋgio} expresses that the action of the base verb \forme{kio} is possible due to the nature of the ground, as shown by (\ref{ex:WthoR.YWsANgio}).


\begin{exe}
\ex \label{ex:WthoR.YWsANgio}
\gll ɯ-tʰoʁ ɲɯ-sɤ-ŋgio tɕe tɕoχtsi ɲɯ́-wɣ-kio ɲɯ-kʰɯ \\
\textsc{3sg}.\textsc{poss}-ground \textsc{sens}-\textsc{prop}-\textsc{acaus}:glide \textsc{lnk} table \textsc{ipfv}-\textsc{inv}-glide \textsc{sens}-be.possible \\
\glt `The ground is slippery, and one can move the table by making it glide on it.' (elicited)
\end{exe}
 
The verb \japhug{nɯzdɯɣ}{worry about} (which could be analyzed as a lexicalized applicative of \japhug{zdɯɣ}{suffer}, §\ref{sec:nWzdWG}) has the \forme{sɤ-} proprietive \japhug{sɤnɯzdɯɣ}{causing people to worry}, as shown by example (\ref{ex:YWsAnWzdWG}), homophonous with an antipassive verb meaning `worrying about people'.

\begin{exe}
\ex \label{ex:YWsAnWzdWG}
\gll nɤ-wa nɯtʰamtɕɤt ʑo tʰɯ-wxti tɕe, nɯ kɯ-fse [...] ɕɯ-kɤ-ɤlɯlɤt  [...] cʰa ɕi kɯma, ɲɯ-sɤ-nɯ-zdɯɣ\\
\textsc{2sg}.\textsc{poss}-father so.much \textsc{emph} \textsc{aor}-be.big \textsc{lnk} \textsc{dem} \textsc{sbj}:\textsc{pcp}-be.like { } \textsc{tral}-\textsc{inf}-fight { } can:\textsc{fact} \textsc{qu} \textsc{sfp} \textsc{sens}-\textsc{prop}-worry.about\\
\glt `Your father has become so old, can he go to war like that? He is cause of worry (for all of us).' (150828 huamulan-zh, 19)
\end{exe}

Proprietive verbs cannot be subjected to (sigmatic \forme{sɯ-} or velar \forme{ɣɤ-}) causative derivations. On the other hand, they can take the tropative \forme{nɤ-} prefix, as shown by the common verbs \japhug{nɤsɤscit}{find pleasant} and \japhug{nɤsɤɣdɯɣ}{find unpleasant} derived from \japhug{sɤscit}{be pleasant} and \japhug{sɤɣdɯɣ}{be unpleasant}, respectively (§\ref{sec:tropative.other.derivations}).

\subsection{Relationship with other derivations} 
The proprietive \forme{sɤ\trt}, like several other voice prefixes (§\ref{sec:voice.denominal}), is related to the denominal \forme{sɤ-} derivation (§\ref{sec:sA.history} ). It has a common origin with the antipassive \forme{sɤ-} (§\ref{sec:antipassive.sA}; see in particular the propensitive function of the antipassive). 

The proprietive and the \forme{sɤ-} antipassive are however synchronically quite distinct. While only the former can have intransitive or semi-transitive verbs as input, both can occur on transitive verbs, in which case the former is object-oriented (§\ref{sec:proprietive.tr}), while the latter is subject-oriented (§\ref{sec:antipassive.sA}).


\section{Facilitative} \label{sec:facilitative}
Japhug has two productive facilitative derivations, the \forme{ɣɤ-} and \forme{nɯɣɯ-} prefixes, deriving stative verbs whose subjects correspond to the subject and the object (or goal) of the base verbs, respectively. Cognates of both prefixes are found in Tshobdun (\forme{wɐ-} and \forme{nwə-}), with an identical functional constrast \citep{jackson14morpho}


These prefixes are only valency-decreasing when prefixed to transitive verbs, and do not change transitivity when the base verb is intransitive.

The meaning of these prefixes is very similar to that of complement clauses headed by the stative verb \japhug{mbat}{be easy} (§\ref{sec:adjective.complement}). For instance, (\ref{ex:kABzjoz.YWmbat}) was provided as a gloss for (\ref{ex:YWnWGWBzjoz}) during an elicitation session.

\begin{exe}
\ex 
\begin{xlist}
\ex \label{ex:YWnWGWBzjoz}
\gll ɲɯ-nɯɣɯ-βzjoz \\
\textsc{sens}-\textsc{facil}-learn \\
\ex \label{ex:kABzjoz.YWmbat}
\gll kɤ-βzjoz ɲɯ-mbat \\
\textsc{inf}-learn \textsc{sens}-be.easy \\
\end{xlist}
\glt `It is easy to learn.' (elicited)
\end{exe}

A third way of expressing facilitative meaning is by verb compounding; the only example of this type is \japhug{apɤmbat}{be easy to do} from the transitive verb \japhug{pa}{do} and the stative verb  \japhug{mbat}{be easy}  (§\ref{sec:denom.compound.verbs}).
 
\subsection{Subject-oriented facilitative} \label{sec:facilitative.GA}
The prefix \forme{ɣɤ\trt}, homophonous with the velar causative \forme{ɣɤ-} (§\ref{sec:velar.causative}), generally derives stative verbs meaning `become/get X easily', `tend to become/get X'. For instance, from \japhug{wxti}{be big} one can derive the stative verb \japhug{ɣɤwxti}{become big easily}, as illustrated by (\ref{ex:wuma.Zo.YWGAwxti}). 

\begin{exe}
\ex \label{ex:wuma.Zo.YWGAwxti}
\gll si kɯ-wxtɯ\redp{}wxti ɲɯ-βze cʰa ma wuma ʑo ɲɯ-ɣɤ-wxti.  \\
tree \textsc{sbj}:\textsc{pcp}-\textsc{emph}\redp{}be.big \textsc{ipfv}-grow[III] can:\textsc{fact} \textsc{lnk} really \textsc{emph} \textsc{sens}-\textsc{facil}-be.big \\
\glt `It can grow into a hug tree, as it easily becomes huge.' (07-Zmbri, 10)
\end{exe}
 
The facilitative prefix \forme{ɣɤ-} is very productive on stative verbs, including on loanwords from Tibetan. For instance, \japhug{rgɤz}{be old} (from \tibet{རྒས་}{rgas}{get old}) has a facilitative form \japhug{ɣɤrgɤz}{age quickly}, attested in (\ref{ex:mAkWGArgAz}).

\begin{exe}
\ex \label{ex:mAkWGArgAz}
\gll tɯrme mɤ-kɯ-ɲɟɯr, nɯnɯ mɤ-kɯ-ɣɤ-rgɤz tɕe tɕe nɯnɯ tɤtʰo tu-sɤrmi-nɯ ŋu. \\
people \textsc{neg}-\textsc{sbj}:\textsc{pcp}-change \textsc{dem} \textsc{neg}-\textsc{sbj}:\textsc{pcp}-\textsc{facil}-be.old \textsc{lnk} \textsc{lnk} \textsc{dem}  \textsc{anthr} \textsc{ipfv}-call-\textsc{pl} be:\textsc{fact} \\
\glt `Persons who don't change, who don't age quickly, people call them `pines'.' (07-tAtho, 26)
\end{exe}

%ɕom nɯ tɕe ɲɯ-ɣɤsa

Some dynamic intransitive verbs are also attested with the \forme{ɣɤ-} derivation, with meanings such as `X quickly/early/easily' or `X often'. For instance, \japhug{mda}{be the time} derives the form \forme{ɣɤ-mda}, which can mean `be the time (ripen) earlier', as in (\ref{ex:YWGAmda}), and \japhug{rɤru}{get up} has the facilitative \japhug{ɣɤrɤru}{getting up early}, `getting up easily' (as soon as one wakes him/her up).

\begin{exe}
\ex \label{ex:YWGAmda}
\gll nɯra iʑo ji-ji pɯ-kɤ-z-mɤku nɯ iʑo ɯ-pʰɯt ku-z-mɤku-j ma ɲɯ-ɣɤ-mda \\
\textsc{dem}:\textsc{pl} \textsc{1pl} \textsc{1pl}.\textsc{poss}-field \textsc{aor}-\textsc{obj}:\textsc{pcp}-\textsc{caus}-be.first \textsc{dem} \textsc{1pl} \textsc{3sg}.\textsc{poss}-\textsc{bare}.\textsc{inf}:cut \textsc{ipfv}-\textsc{caus}-be.first-\textsc{1pl} \textsc{lnk} \textsc{sens}-\textsc{facil}-be.the.time \\
\glt `Our fields, that have been (sowed) first, we harvest they first, because the (crops) ripen earlier (than in other places, higher up in altitude).' (2010-09)
\end{exe}

It also occurs on some anticausative verbs (§\ref{sec:anticausative}); for instance, the intransitive \japhug{ɴɢrɯ}{break} has the facilitative form \forme{ɣɤ-ɴɢrɯ} `easily break', which has the same meaning as the object-oriented facilitative \forme{nɯɣɯ-qrɯ} of the transitive base verb \japhug{qrɯ}{break} (§\ref{sec:anticausative.other.derivations}). It is also attested with denominal verbs such as \japhug{sɤmbrɯ}{get angry} (§\ref{sec:denom.sA.proprietive}), whose facilitative is \japhug{ɣɤsɤmbrɯ}{get angry easily}. 

The \forme{ɣɤ-} prefix is also found on intransitive verbs that only occur in collocation with a particular noun, for instance \japhug{nmu}{shake (of earthquakes)} (on which see §\ref{sec:volitional.mW}) or the collocation  \japhug{ɯ-ʁo+mbi}{be discouraged} (§\ref{sec:anticausative}, §\ref{sec:orphan.verb}), from which \forme{ɣɤ-nmu} or \japhug{ɯ-ʁo+ɣɤmbi}{be easily discouraged}  can be derived (examples \ref{ex:YWGAnmu} and \ref{ex:nARo.WtWGAmbi}).
 
\begin{exe}
\ex \label{ex:YWGAnmu}
\gll iɕqʰa sɤtɕʰa nɯ waɟɯ ɲɯ-ɣɤ-nmu. \\
the.aforementioned place \textsc{dem} earthquake \textsc{sens}-\textsc{facil}-shake \\
\glt `Earthquakes are frequent in this place.' (elicited)
\end{exe}

\begin{exe}
\ex \label{ex:nARo.WtWGAmbi}
\gll nɤ-ʁo ɯ-tɯ-ɣɤ-mbi nɯ! \\
\textsc{2sg}.\textsc{poss}-discourage(1) \textsc{3sg}-\textsc{nmlz}:\textsc{deg}-\textsc{facil}-\textsc{acaus}:discourage(2) \textsc{sfp} \\
\glt `You are so easily discouraged!' (elicited)
\end{exe}

With the dynamic motion verb \japhug{ɕe}{go}, the facilitative \forme{ɣɤ-ɕe} only occurs in collocation with \japhug{tɤ-rʑaʁ}{time} in the meaning `pass quickly (of time)'), as in (\ref{ex:WYWGACe}). In this example, the facilitative applied to the whole collocation \forme{tɤ-rʑaʁ+ɕe}  `spend (one's time)' (§\ref{sec:intr.light.verbs})

\begin{exe}
\ex \label{ex:WYWGACe}
\gll nɤ-tɤ-rʑaʁ ɯ-ɲɯ́-ɣɤ-ɕe \\
\textsc{2sg}.\textsc{poss}-\textsc{indef}.\textsc{poss}-time \textsc{qu}-\textsc{sens}-\textsc{facil}-go \\
\glt `Is the time passing quickly for you?' (elicited)
\end{exe}

 The facilitative \forme{ɣɤ-} is also attested on a handful of transitive experiencer verbs, where it has an antipassive-like valency-decreasing function:  \japhug{ɕɯftaʁ}{remember} and \japhug{jmɯt}{forget} derive the stative verbs \japhug{ɣɤɕɯftaʁ}{to have a good memory}  and \japhug{ɣɤjmɯt}{be forgetful}, respectively (\ref{ex:YWGACWftaR}). 

\begin{exe}
\ex \label{ex:YWGACWftaR}
\gll ɲɯ-ɕqraʁ tɕe ɲɯ-ɣɤ-ɕɯftaʁ    \\
  \textsc{sens}-be.intelligent \textsc{lnk} \textsc{sens}-\textsc{facil}-memorize \\
 \glt `He is intelligent, he has a good memory.' (elicited)
\end{exe}

This derivation demotes the object, and the intransitive subject of the facilitative verbs \forme{ɣɤɕɯftaʁ} and \forme{ɣɤjmɯt} corresponds to the transitive subject of the base verb. In order to build a stative verb expressing a property of the stimulus/object `to be easy to remember/forget', the other facilitative prefix \forme{nɯɣɯ-} is used instead (§\ref{sec:facilitative.nWGW}). Hence, the prefix \forme{ɣɤ-} can be described as a subject-oriented facilitative, following Sun's (\citeyear{jackson14morpho}) description of the cognate prefix \forme{wɐ-} in Tshobdun.
 
\subsection{Object-oriented facilitative} \label{sec:facilitative.nWGW}
The facilitative  \forme{nɯɣɯ-} prefix is one of the rare disyllabic derivational prefix in Japhug. Like \forme{ɣɤ-} (§\ref{sec:facilitative.GA}), it is used to build stative verbs meaning `be easy to X'. It is most commonly prefixed to transitive action verbs: a typical example is for instance \forme{nɯɣɯ-krɤɣ} `be easy to shear' from \japhug{krɤɣ}{mow, shear}, as in (\ref{ex:mWjnWGWkrAG}). Additional examples are found in \tabref{tab:facilitative.nWGW}. 

\begin{exe}
\ex \label{ex:mWjnWGWkrAG}
\gll tɕe qaʑo kɯ-wxti nɯra ɣɯ qʰe, nɯ-rme ɲɯ-rɲɟi qʰe ɲɯ-nɯɣɯ-krɤɣ, kɯ-xtɕi nɯra ɣɯ, nɯ-rme ɲɯ-xtɯt qʰe, mɯ́j-nɯɣɯ-krɤɣ. \\
\textsc{lnk} sheep \textsc{sbj}:\textsc{pcp}-be.big \textsc{dem}:\textsc{pl} \textsc{gen} \textsc{lnk} \textsc{3pl}.\textsc{poss}-hair \textsc{sens}-be.long \textsc{lnk} \textsc{sens}-\textsc{facil}-shear \textsc{sbj}:\textsc{pcp}-be.small \textsc{dem}:\textsc{pl} \textsc{gen}   \textsc{3pl}.\textsc{poss}-hair \textsc{sens}-be.short \textsc{lnk} \textsc{neg}:\textsc{sens}-\textsc{facil}-shear \\
\glt `The big sheep, their wool is long and thus easy to shear, the small ones, their wool is short and difficult to shear.' (160712 smAG, 28-29)
\end{exe}  

When the base verb is transitive, the \forme{nɯɣɯ-} prefix is a valency-decreasing derivation, removing the transitive subject, and turning the object into an intransitive subject. In (\ref{ex:mAnWGWmto}) for instance, \japhug{nɯɣɯmto}{be easy to see} appears in the Factual third singular without stem III alternation, showing that it is an intransitive verb (§\ref{sec:transitivity.morphology}).

\begin{exe}
\ex \label{ex:mAnWGWmto}
\gll ci nɯ xɕaj ɯ-mdoʁ tsa ɯ-kɯ-ndo nɯ mɤ-nɯɣɯ-mto. tɕe wuma ʑo mɯ\redp{}mɤ-pɯ-kɯ-tso nɤ mɤ́-wɣ-mto \\
\textsc{indef} \textsc{dem} grass \textsc{3sg}.\textsc{poss}-colour a.little \textsc{3sg}.\textsc{poss}-\textsc{sbj}:\textsc{pcp}-take \textsc{dem} \textsc{neg}-\textsc{facil}-see:\textsc{fact} \textsc{lnk} really \textsc{emph} \textsc{cond\redp}{}\textsc{neg}-\textsc{pst}.\textsc{ipfv}-\textsc{genr}:S/O-understand \textsc{lnk} \textsc{neg}-\textsc{inv}-see \\
\glt `The other one, which has the colour of grass, is not easy to see; unless you know it very well, you won't see it.' (07-Cku, 58)
\end{exe}  

Among the verbs in \tabref{tab:facilitative.nWGW}, the facilitative \japhug{nɯɣɯjpa}{be convenient} is particularly lexicalized, with both an irregular \forme{-j-} element occurring between he prefix \forme{nɯɣɯ-} and the root \forme{|pa|}, and a non-predictable meaning derivation.
 
In the case of the verb \japhug{ti}{say}, the facilitative \forme{nɯɣɯti} can either mean `be easy to pronounce' or `be easy to express', as in example (\ref{ex:mWjnWGWti}), uttered by Tshendzin during an elicitation session.

\begin{exe}
\ex \label{ex:mWjnWGWti}
\gll  nɯ mɯ́j-nɯɣɯ-ti \\
\textsc{dem} \textsc{neg}:\textsc{sens}-\textsc{facil}-say \\
\glt `This (meaning) is difficult to express (in Japhug).' (heard in context)
\end{exe}  

 \begin{table}[H]
\caption{Examples of the facilitative \forme{nɯɣɯ-} prefix in Japhug}\label{tab:facilitative.nWGW}
\begin{tabular}{lllllllll} 
\lsptoprule
basic verb  & derived  verb &\\
 \midrule
 \japhug{ŋke}{walk} & \japhug{nɯɣɯŋke}{be easy to walk (on)} \\
\midrule
\japhug{ŋga}{wear} & \japhug{nɯɣɯŋga}{be nice to wear} \\
\japhug{ndza}{eat} &  \japhug{nɯɣɯndza}{be easy/nice to eat} \\
\japhug{ntɕʰoz}{use} & \japhug{nɯɣɯntɕʰoz}{be easy to use} \\
\japhug{mto}{see} & \japhug{nɯɣɯmto}{be easy to see} \\
\japhug{ti}{say} & \japhug{nɯɣɯti}{be easy to say} \\
\japhug{ɕɯftaʁ}{remember} & \japhug{nɯɣɯɕɯftaʁ}{be easy to remember} \\
\japhug{jmɯt}{forget} & \japhug{nɯɣɯjmɯt}{be easy to forget} \\
\japhug{pa}{do} & \japhug{nɯɣɯjpa}{be convenient} \\
\lspbottomrule
\end{tabular}
\end{table}
 
Some intransitive verbs are attested with the \forme{nɯɣɯ-} prefix. The only common one is \japhug{nɯɣɯŋke}{be easy to walk (on)} (from \japhug{ŋke}{walk}); the \forme{nɯɣɯ-} derivation appears to be possible on a few other motion verbs and verbs of location (such as \japhug{rɟɯɣ}{run} or \japhug{rɤʑi}{stay}), though their acceptability has to be rechecked. The subject of the facilitative verbs derived from such intransitive verbs corresponds to the locative adjunct (\ref{ex:WmAtWnWGWNke}) or goal (\ref{ex:YWnWGWCe}) of the base verb.

\begin{exe}
\ex \label{ex:WmAtWnWGWNke}
\gll  tɕʰeme nɯnɯ tʰɤlwa ɲɯɣɲɯɣ zɯ to-ɕe qʰe, maka ɯ-mɤ-tɯ-nɯɣɯ-ŋke pjɤ-saχaʁ ʑo, \\
girl \textsc{dem} earth \textsc{idph}(II):soft \textsc{loc} \textsc{ifr}:\textsc{up}-go \textsc{lnk} at.all \textsc{3sg}.\textsc{poss}-\textsc{neg}-\textsc{nmlz}:\textsc{deg}-\textsc{facil}-walk \textsc{pst}.\textsc{ipfv}-be.extremely \textsc{emph} \\
\glt `The girl went up the (path made of) soft earth, and it was extremely difficult to walk on it.' (2014-kWlAG, 148)
\end{exe}

\begin{exe}
\ex \label{ex:YWnWGWCe}
\gll  tʂu ɯ-rkɯ ɲɯ-ɤrmbat tɕe, ɲɯ-nɯɣɯ-ɕe \\
path \textsc{3sg}.\textsc{poss}-side \textsc{sens}-be.close \textsc{lnk} \textsc{sens}-\textsc{facil}-go \\
\glt `(This place) is close to the road, it is easy to go to.' (elicited)
\end{exe}

The semi-transitive \japhug{tso}{know, understand} (§\ref{sec:semi.transitive}) can also undergo the facilitative derivation to \japhug{nɯɣɯtso}{easy to understand}. Semi-transitive verbs in \forme{a\trt}, such as \japhug{aʁe}{have to eat/drink} or \japhug{atɯɣ}{meet}, are not compatible with \forme{nɯɣɯ\trt}, which appears to lack an allomorph with vowel fusion; a form $\dagger$\forme{nɯɣɤʁe} for instance (intended meaning: `easy to get to eat') is utterly inacceptable.

Since semi-objects, goals and some locative adjuncts do have partial objectal properties (§\ref{sec:semi.object}, §\ref{absolutive.goal}, §\ref{sec:theme.ditransitive}, §\ref{sec:semi.object.relativization}, §\ref{sec:locative.relativization}), it is nevertheless appropriate to describe \forme{nɯɣɯ-} as an object-oriented derivation, following \citet{jackson14morpho}. The contrast between \forme{nɯɣɯ-} and subject-oriented \forme{ɣɤ-} is clearest with the transitive verbs \japhug{jmɯt}{forget} and \japhug{ɕɯftaʁ}{remember}, which can be subjected to both derivations: compare object-oriented \japhug{nɯɣɯjmɯt}{be easy to forget} and subject-oriented  \japhug{ɣɤjmɯt}{be forgetful} (§\ref{sec:facilitative.GA}).

It is possible to apply the facilitative derivation to an intransitive verb already derived by the sigmatic causative, for instance \forme{nɯɣɯ-sɯɣ-ɲaʁ} (\textsc{facil}-\textsc{caus}-be.black) `be easily blackened'. It is also possible to causativize a facilitative verb (with the \forme{z-} allomorph of the sigmatic causative, §\ref{sec:caus.z}). For example, the causative of \japhug{nɯɣɯntɕʰoz}{be easy to use} is \forme{z-nɯɣɯ-ntɕʰoz} (\textsc{caus}-\textsc{facil}-use) `make easy to use'. The relative order of the facilitative and of the causative prefixes thus reflects their semantic scope (§\ref{sec:sig.caus.other.derivations}).

%8_nWGWsWGYaR

