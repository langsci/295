\chapter{Complement clauses} \label{chap:complement.clauses}

\section{Introduction} \label{sec:complement.intro}
This chapter, based on Sun's (\citeyear{sun12complementation}) work on Tshobdun and on earlier research on Japhug \citep{jacques08zh, jacques16complementation}, presents an account of complement clauses and complementation strategies\footnote{On the notion of complementation strategy, see \citep[34--40]{dixon06complementation} and §\ref{sec:strategies}. } in Japhug.
 
This chapter comprises six sections. The first section following the introduction §\ref{sec:complement.types} provides a classification of complement clauses based on the form of the main verb in the clause. The second one, §\ref{sec:strategies} presents an overview of complementation strategies (including ambiguous relative clauses). The third one, §\ref{sec:complement.morphosyntax} analyses a certain number of morphosyntactic specificities of complement clauses (aside from verbal morphology) that distinguishes them from the corresponding independent clauses. The fourth section, §\ref{sec:complement.taking.verbs} surveys com\-ple\-ment-taking verbs and describes the complement clause types and complementation strategies that they are compatible with. The fifth one, §\ref{sec:complement.taking.nouns} discusses com\-ple\-ment-taking nouns and noun-verb collocations and how they differ from prenominal (§\ref{sec:prenominal.relative}) and genitival (§\ref{sec:genitival.relatives}) relative clauses. The sixth section, §\ref{sec:syntactic.errors} briefly analyzes syntactic errors related to complementation in the corpus.

In this chapter, all complement clauses are systematically indicated between square brackets (with double embedding in some cases).
 
The sections on participles (§\ref{sec:participles}) and infinitives (§\ref{sec:inf}) in a previous chapter partially overlap with some of the topics covered in this chapter.
 
\section{Complement types} \label{sec:complement.types}
This section illustrates the different categories of complements attested in Japhug. Five main types of complement clauses are distinguished: velar infinitival complements, bare infinitival complements, finite complements, multiclausal complements and reported speech. In addition, Japhug has many different complementation strategies, discussed in §\ref{sec:strategies}.

\subsection{Velar infinitive clauses} \label{sec:velar.infinitives.complement.clauses}
Velar infinite clauses (§\ref{sec:velar.inf}) are one of the most common types of complement clauses in Japhug (§\ref{sec:inf.complementation}). Two velar infinitives are attested, \forme{kɯ-} for stative verbs and impersonal intransitive verbs, and \forme{kɤ-} for dynamic and/or morphologically transitive verbs. 

A recurrent problem in the study of subordinate clauses in Japhug is the ambiguity between velar participles and infinitives (§\ref{sec:infinitives.participles}), making participial clauses and infinitival clauses only distinguishable in specific contexts, and the ambiguity between complement clauses and relative clauses with some com\-ple\-ment-taking verbs (§\ref{sec:relative.complement.ambiguities}, §\ref{sec:relative.core.arg}). As a result, participial (§\ref{sec:participial.relatives}) and finite  relative clauses  (§\ref{sec:finite.relatives}) are in some case difficult to differentiate from velar infinitival complement clauses (§\ref{sec:velar.infinitives.complement.clauses}) and finite complement clauses  (§\ref{sec:finite.complement}), respectively. 


\subsubsection{Case marking} \label{sec:case.infinitive}
While velar infinitives bear no person indexation markers, noun phrases receive the same case markers in infinitive clauses as in independent clauses, showing that infinitives have the same argument structures as finite verb forms.

When an argument is shared between the complement and the matrix clause, it does not necessarily have the same syntactic function in both clauses, as in (\ref{ex:kW.nWra.kAstu}), where \japhug{tɤɕime}{lady} is transitive subject in the complement clause (see §\ref{sec:ditransitive.secundative} on the argument structure of \japhug{stu}{do like}) and intransitive subject in the matrix clause (§\ref{sec:semi.transitive}). 

\begin{exe}
\ex \label{ex:kW.nWra.kAstu}
\gll [tɤɕime nɯ kɯ nɯra kɤ-stu] pjɤ-cʰa \\
princess \textsc{dem} \textsc{erg} \textsc{dem}:\textsc{pl} \textsc{inf}-do.like.this \textsc{ifr}-can \\
\glt `The princess succeeded in doing it.' (140511 alading-zh)
\japhdoi{0003953\#S241}
\end{exe}

In this sentence, the noun takes the ergative marker \forme{kɯ} in accordance with the verb of the complement clause (§\ref{sec:A.kW}), showing that it belongs to the complement clause rather than to the matrix clause directly. This is the most commonly observed pattern in Japhug texts: in infinitival clauses, the shared arguments more often take the case marking selected by the verb of the complement clause than that of the matrix clause.

However, when the com\-ple\-ment-taking verb is a stative verb, there are cases where the ergative on the transitive subject is optional, as shown by (\ref{ex:tWrme.kAndza.sna}) (without ergative on \japhug{tɯrme}{person}; Tshendzin has confirmed that this example is correct) and (\ref{ex:tWrme.kW.kAndza.sna}) (with ergative). The precise conditions for this phenomenon still remain to be investigated.

\begin{exe}
\ex \label{ex:tWrme.kAndza.sna}
\gll  [ɯ-mat nɯ kɤ-ndza] sna, [tɯrme kɤ-ndza] sna \\
\textsc{3sg}.\textsc{poss}-fruit \textsc{dem} \textsc{inf}-eat be.good:\textsc{fact} people \textsc{inf}-eat be.good:\textsc{fact} \\
\glt `Its fruit is nice to eat, it is nice for people to eat.' (09-stoR)
\japhdoi{0003470\#S45}
\end{exe}

\begin{exe}
\ex \label{ex:tWrme.kW.kAndza.sna}
\gll  [nɯnɯ tɯrme kɯ kɤ-ndza] sna \\
\textsc{dem} people \textsc{erg} \textsc{inf}-eat be.good:\textsc{fact} \\
\glt `It is nice for people to eat.' (13-NanWkWmtsWG)
\japhdoi{0003492\#S157}
\end{exe}

\subsubsection{Coreference restrictions} \label{sec:velar.inf.coreference}
Coreference restrictions between the arguments of complement clauses with \forme{kɤ-} infinitives and their matrix clauses differ from verb to verb, and four cases can be distinguished.

First, in the case of impersonal verbs such as \japhug{ra}{be needed}, `be necessary' (§\ref{sec:ra.khW.jAG.verb}), there is no argument coreference between the matrix clause and the complement clause. In this case, the arguments are neither indexed on the matrix verb nor on the verb in the complement clause.

Second, with a few transitive com\-ple\-ment-taking verbs such as the transitive \japhug{spa}{be able} (§\ref{sec:spa.verb}) and the intransitive \japhug{nɤz}{dare} (§\ref{sec:nAz.verb}), coreference between the subject of the matrix clause and that of the complement clause is required. 

Third, a handful of verbs, including \forme{sɯxcʰa} (§\ref{sec:sWxcha}), the causative of \japhug{cʰa}{can} (§\ref{sec:sig.caus.modal}), have coreference between the subject of the complement clause and the \textit{object} of the matrix clause.

Fourth, for most verbs taking infinitives (like the semi-transitive \japhug{rga}{like} or the transitive \japhug{rɲo}{experience}),  the subject of the matrix clauses can be coreferential to either the subject of an intransitive verb (\ref{ex:kAnWrAGo.rganW}), the subject of a transitive verb (\ref{ex:kAnArtoXpjAt.pWrgaa}), the object (\ref{ex:YWrganW}) and also possessors of core arguments (§\ref{sec:rYo.complements}).

\begin{exe}
   \ex   \label{ex:kAnWrAGo.rganW} 
\gll tsuku tɕe [kɤ-nɯrɤɣo] wuma ʑo rga-nɯ tɕe \\
some \textsc{lnk} \textsc{inf}-sing really \textsc{emph} like:\textsc{fact}-\textsc{pl}  \textsc{lnk} \\
\glt `Some people like to sing.' (26-kWrNukWGndZWr)  (S=S)
\japhdoi{0003672\#S102}
\end{exe}  
 
\begin{exe}
\ex   \label{ex:kAnArtoXpjAt.pWrgaa} 
\gll aʑo [qajɯ nɯra kɤ-nɤrtoχpjɤt] pɯ-rga-a tɕe  	\\
  	\textsc{1sg} bugs \textsc{dem}:\textsc{pl} \textsc{inf}-observe \textsc{pst}.\textsc{ipfv}-like-\textsc{1sg} \textsc{lnk}  \\
 \glt `I liked to observe bugs.' (26-quspunmbro) (A=S)
\japhdoi{0003684\#S15}
\end{exe}  
 
\begin{exe}
   \ex   \label{ex:YWrganW} 
\gll maka [tu-kɤ-nɤjoʁjoʁ], [tu-kɤ-fstɤt] nɯ ɲɯ-rga-nɯ  \\
at.all \textsc{ipfv}-\textsc{inf}-flatter \textsc{ipfv}-\textsc{inf}-praise \textsc{dem} \textsc{ipfv}-like-\textsc{pl} \\
\glt `They like to be flattered or praised.' (140427 yuanhou-zh) (P=S)
\japhdoi{0003870\#S53}
\end{exe}  


Other types of complement clauses differ from velar infinitive clauses by their constraints on coreference (see §\ref{sec:bare.inf.coreference}).

 \subsection{Bare infinitives and dental infinitives} \label{sec:bare.dental.inf}
Complement clauses with bare (§\ref{sec:bare.inf}) and dental infinitives (§\ref{sec:dental.inf}) are less widespread than those with velar infinitives. Only a limited number of com\-ple\-ment-taking verbs select them: phasal verbs (including \japhug{ʑa}{begin}, \japhug{sɤʑa}{begin}, \japhug{stʰɯt}{finish}, and \japhug{jɤɣ}{finish}), causative verbs derived from adjectives, the aspectual verb \japhug{rɲo}{experience} and the causative \forme{sɯpa} `cause to do' (§\ref{sec:bare.dental.inf.sWpa}). With the exception of \japhug{jɤɣ}{finish}, these com\-ple\-ment-taking verbs are all morphologically transitive.

In the Tshobdun corpus \citet{jackson19tshobdun}, the cognate verbs \forme{ɟeʔ} `begin'  and \forme{joɣʔ} `finish' take velar infinitives, and there is no infinitival form comparable to the Japhug dental infinitive.

\subsubsection{Complementary distribution} \label{sec:bare.inf.dental.complementary}
Bare and dental infinitives are found in complementary distribution. Bare infinitives occur when the main verb of the complement clause is morphologically transitive (§\ref{sec:transitivity.morphology}) as in (\ref{ex:WCA.tuZanW.Cti}).

\begin{exe} 
\ex \label{ex:WCA.tuZanW.Cti}
\gll pɤjkʰu pjɯ-si ɕɯŋgɯ ʑo [ɯ-ɕa ɯ-ndza] tu-ʑa-nɯ ɕti \\
still \textsc{ipfv}-die before \textsc{emph} \textsc{3sg}.\textsc{poss}-flesh \textsc{3sg}.\textsc{poss}-\textsc{bare}.\textsc{inf}:eat \textsc{ipfv}-start-\textsc{pl} be.\textsc{aff}:\textsc{fact} \\
\glt `[The lions] start eating its flesh before it dies (i.e. while it is still alive).' (20-sWNgi)
\japhdoi{0003562\#S44}
  \end{exe} 

Dental infinitives on the other hand are  found when the verb of the complement clause is intransitive as in (\ref{ex:tWmNAm.taZa}) (including labile verbs §\ref{sec:lability.morphosyntax}) or transitive with dummy subject (\ref{ex:Wmat.tWBzu.naZa}) (§\ref{sec:transitive.dummy}).

\begin{exe} 
\ex \label{ex:tWmNAm.taZa}
\gll [aʑo a-ku (a-mɤtsa), tɯ-mŋɤm] ta-ʑa \\
\textsc{1sg} \textsc{1sg}.\textsc{poss}-head \textsc{1sg}.\textsc{poss}-MZCh \textsc{inf}:II-hurt \textsc{aor}:3\flobv{}-start \\
\glt `My head$_i$, cousin, when it$_i$ starts hurting...' (TaRrdo 2003 conversation)
  \end{exe} 
  
\begin{exe} 
\ex \label{ex:Wmat.tWBzu.naZa}
\gll ɯ-mɯntoʁ nɯ pɯ-ŋgra tɕe [ɯ-ŋgɯ ɯ-mat tɯ-βzu] na-ʑa ri tɕe \\
\textsc{3sg}.\textsc{poss}-flower \textsc{dem} \textsc{aor}-\textsc{acaus}:cause.to.fall \textsc{lnk} \textsc{3sg}.\textsc{poss}-inside \textsc{3sg}.\textsc{poss}-fruit \textsc{inf}:II-make \textsc{aor}:3\flobv{}-start \textsc{loc} \textsc{lnk} \\
\glt `When its flower has fallen, and its fruit has started growing inside...' (12-ndZiNgri)
\japhdoi{0003488\#S109}
 \end{exe} 
 
Since most com\-ple\-ment-taking verbs selecting dental infinitives are transitive, case marking on the common subject can either be in the ergative or in the absolutive (§\ref{sec:complement.clause.case.marking}).

\subsubsection{Coreference restrictions} \label{sec:bare.inf.coreference}
Bare and dental infinitives strongly differ from velar infinitives as to their coreference restrictions. When the verb \japhug{rɲo}{experience} occurs with velar infinitives, the subject of the matrix clause can be coreferential with either the subject, the object or even the possessor of the intransitive subject of the complement clause (§\ref{sec:rYo.complements}).

The ambiguity between transitive subject or object coreference is particularly clear with the verb \japhug{nɤkʰu}{invite} (to one's home as a guest, see examples \ref{ex:kAnAkhu1} and \ref{ex:kAnAkhu2}), as with this verb both arguments are equal in terms of volition and control.

\begin{exe}
\ex  \label{ex:kAnAkhu1}
\gll [ɯʑo kɯ kɤ-nɤkʰu] pɯ-rɲo-t-a  \\
\textsc{3sg} \textsc{erg} \textsc{inf}-invite \textsc{aor}-experience-\textsc{pst}:\textsc{tr}-\textsc{1sg} \\
\glt `I have been to his house as a guest.'  (= `He invited me to come to his house as a guest and I came.') (P=A)
\ex  \label{ex:kAnAkhu2}
\gll [ɯʑo kɤ-nɤkʰu] pɯ-rɲo-t-a  \\
\textsc{3sg}  \textsc{inf}-invite \textsc{aor}-experience-\textsc{pst}:\textsc{tr}-\textsc{1sg} \\
\glt `He has been to my house as a guest.' (= `I have invited him to come to my house as a guest and he came.') (A=A)
\end{exe}

In the case of bare infinitives, on the other hand, the subjects of the matrix and complement clause must be coreferent, but the object of the matrix clause can, however, be neutralized to third person.

In example (\ref{ex:nAkhu1}), the shared subject (referring to the host) is \textsc{3sg}. The verb of the matrix clause takes the complement clause as a \textsc{3sg} object (hence the verb takes the 3\flobv{} form without \textsc{1sg} marking), while the verb of the complement clause takes a \textsc{1sg} object (referring to the guest), marked by the possessive prefix \forme{a-}.\footnote{In the English translation, the \textsc{1sg} is rendered as a subject, because translating  \forme{a-nɤkʰu pa-rɲo} as `He invited me' would be inexact, as this English sentence does not imply that the \textsc{1sg} did attend the invitation. }

\begin{exe}
\ex  \label{ex:nAkhu1}
\gll [a-nɤkʰu] pa-rɲo \\
\textsc{1sg}.\textsc{poss}-\textsc{bare}.\textsc{inf:}invite \textsc{aor}:3\flobv{}-experience \\
\glt `I have been to his house as a guest.' (= `He invited me to come to his house as a guest and I came.')
\ex  \label{ex:nAkhu2}
\gll [ɯʑo ɯ-nɤkʰu] pɯ-rɲo-t-a  \\
\textsc{3sg}  \textsc{3sg}.\textsc{poss}-\textsc{bare}.\textsc{inf}:invite \textsc{aor}-experience-\textsc{pst}:\textsc{tr}-\textsc{1sg} \\
\glt `He has been to my house as a guest.' (= `I invited him to come to my house as a guest and he came.')
\end{exe}

This generalization is observed for all transitive verbs taking bare infinitive complement clauses. However, the intransitive impersonal verb \japhug{jɤɣ}{finish} takes the bare infinitive clause as intransitive subject, and remains in third person singular regardless of the subject and object of the complement clause, as in (\ref{ex:Wti.tojAG}), where although the subject of the complement clause is third person plural, no plural marker can appear on \forme{jɤɣ}.

\begin{exe}
\ex \label{ex:Wti.tojAG}
\gll [nɯra ɯ-ti] to-jɤɣ tɕe \\
\textsc{dem}:\textsc{pl} \textsc{3sg}.\textsc{poss}-\textsc{bare}.\textsc{inf}:say \textsc{ifr}-finish \textsc{lnk}\\
\glt `After saying that, (they went to the park)' (140515 congming de wusui xiaohai-zh)
\japhdoi{0003998\#S15}
\end{exe}

With dental infinitives, subject coreference is the same, except in the case of transitive verb with dummy subject (example \ref{ex:Wmat.tWBzu.naZa}, §\ref{sec:bare.inf.dental.complementary}), where the subject of the matrix verb is coreferent with the sole argument of the complement clauses, whose status is intermediate between that of a subject and an object (§\ref{sec:absolutive.nature}).

 \subsection{Finite complements} \label{sec:finite.complement}
 Complement clauses, like relative clauses (§\ref{sec:finite.relatives}), can have a verb in finite, rather than infinitival form in Japhug and other Gyalrong languages. These constructions are called ``finite complement clauses'' in the present work, corresponding to Sun's (\citeyear[475-477]{sun12complementation})  ``S-like (sentence-like) clauses''.\footnote{I chose ``finite'' rather than ``S-like'' to avoid confusion with ``S'' as abbreviation for ``intransitive subject''.   } This category excludes reported speech complement, which present different characteristics (§\ref{sec:reported.speech}).

 \subsubsection{TAME forms} \label{sec:TAM.finite}
In finite clauses other than reported speech, TAME marking in the complement clause presents some restrictions. Of the 11 primary TAME categories (§\ref{sec:tame.intro}), only the Imperfective (§\ref{sec:ipfv.complement}) and the Factual Non-Past (§\ref{sec:fact.complement}) are compatible with most if not all verbs selecting finite complement clauses, regardless of the TAME category of the matrix verb. In (\ref{ex:junWCea.tAchaa}) for instance, the verb of the complement clause is the Imperfective while that of the matrix clause is in the Aorist.

\begin{exe}
\ex \label{ex:junWCea.tAchaa}
 \gll [aʑo a-ŋga ra tu-nɯ-ŋge-a, jɤɣɤt ju-nɯ-ɕe-a] ra tɤ-cʰa-a \\
 \textsc{1sg} \textsc{1sg}-clothes \textsc{pl} \textsc{ipfv}-\textsc{auto}-wear[III]-\textsc{1sg} toilet \textsc{ipfv}-\textsc{auto}-go-\textsc{1sg} \textsc{pl} \textsc{aor}-can-\textsc{1sg} \\
 \glt  `I am able now to get dressed by myself and use the bathroom by myself (again, after an accident).' (conversation, 17-08-21)
 \end{exe}

The Irrealis (§\ref{sec:irrealis.complement.clauses}) and Imperative (§\ref{sec:imp.compl}) are also found, but only with modal auxiliaries such as \japhug{ra}{be needed} as in (\ref{ex:amApWwGnWClWG}). Unlike in Tshobdun \citep[807]{jackson07irrealis},  verbs of cognition such as \japhug{sɯso}{think} do not select the Irrealis; although complement clauses in the Irrealis are found with these verbs, they are best analyzed in Japhug as reported speech (§\ref{sec:irrealis.complement.clauses}, §\ref{sec:reported.speech}). 

\begin{exe}
\ex \label{ex:amApWwGnWClWG}
\gll  ndɤre [kɯ-xtɕɯ\redp{}xtɕi a-mɤ-pɯ́-wɣ-nɯ-ɕlɯɣ] ɲɯ-ra ma rca nɯ ɲɯ-ndoʁ qʰe ɕlaʁ ʑo pjɯ-ɴɢrɯ ɲɯ-ɕti. \\
\textsc{lnk} \textsc{inf}:\textsc{stat}\redp{}be.small \textsc{irr}-\textsc{neg}-\textsc{pfv}:\textsc{down}-\textsc{inv}-\textsc{auto}-drop \textsc{sens}-be.needed because \textsc{unexp}:\textsc{deg} \textsc{dem} \textsc{sens}-be.brittle \textsc{lnk} at.once \textsc{emph} \textsc{ipfv}-\textsc{acaus}:break \textsc{sens}-be:\textsc{aff} \\
\glt `However, one should not let it drop at all, otherwise, as it is very brittle, it would break at once.' (30-Com)
\japhdoi{0003736\#S26}
\end{exe}

Other TAME categories, such as Aorist (§\ref{sec:aor.complement}) or Inferential (\ref{ex:chAsWjGAt.pjAchaa}), are only attested in the complement clause if the matrix verb is also in the Aorist or in the Inferential, respectively.\footnote{This constraint may not apply to complements of perception and cognition verbs such as \japhug{mto}{see} (§\ref{sec:mto.mtshAm.complement}). } No semantic difference has yet been ascertained between Imperfective complements (\ref{ex:junWCea.tAchaa}) and complements whose TAME category is copied from that of the matrix verb (\ref{ex:tAnANkWNkea.tAchaa}); the latter are considerably rarer.

\begin{exe}
\ex \label{ex:tAnANkWNkea.tAchaa}
 \gll [kɯ-xtɕɯ-xtɕi tɤ-nɤŋkɯŋke-a] tɤ-cʰa-a. \\
 \textsc{sbj}:\textsc{pcp}-\textsc{emph}\redp{}be.small \textsc{aor}-walk.around-\textsc{1sg} \textsc{aor}-can-\textsc{1sg} \\
 \glt `I am now able to walk around a little bit (again, after an accident).' (conversation, 17-08-21)
\end{exe}

\begin{exe}
\ex \label{ex:chAsWjGAt.pjAchaa}
 \gll [zdɯm kɯ-ɲaʁ nɯ cʰɤ-sɯ-jɣɤt] pjɤ-cʰa ɲɯ-ŋu. \\
 cloud \textsc{sbj}:\textsc{pcp}-be.black \textsc{dem} \textsc{ifr}:\textsc{downstream}-\textsc{caus}-turn.back \textsc{ifr}-can \textsc{sens}-be \\
 \glt `He succeeded in making the black cloud retreat.' (25-kAmYW-XpAltCin)
\japhdoi{0003642\#S61}
 \end{exe}
 
The TAME agreement between the matrix verb and the verb of the complement clause is not limited to Aorist and Inferential, and is also attested with Irrealis and Egophoric Present, as in (\ref{ex:akAGWt.apWcha}) and (\ref{ex:kutWndze.Wkura}), respectively.

\begin{exe}
\ex \label{ex:akAGWt.apWcha}
 \gll  [nɯnɯ a-kɤ-ɣɯt] a-pɯ-cʰa tɕe nɯnɯ pʰɤn \\
 \textsc{dem} \textsc{irr}-\textsc{pfv}:\textsc{east}-bring \textsc{irr}-\textsc{pfv}-can \textsc{lnk} \textsc{dem} be.efficient:\textsc{fact} \\
 \glt `If he succeeds in bringing him here, it will be efficient [to cure my disease].' (2011-04-smanmi)
 \end{exe}
 
\begin{exe}
\ex \label{ex:kutWndze.Wkura}
 \gll   [nɤ-ɕqʰe smɤn ku-tɯ-ndze] ɯ-kú-ra? \\
 \textsc{2sg}.\textsc{poss}-cough medicine \textsc{prs}-2-eat[III] \textsc{qu}-\textsc{prs}-be.needed \\
 \glt `Do you have to take medicine for your cough (now)?' (conversation 2019-08-16)
 \end{exe} 
 
 Examples like (\ref{ex:tAnANkWNkea.tAchaa}), (\ref{ex:chAsWjGAt.pjAchaa}) and (\ref{ex:akAGWt.apWcha}) are not analyzable as serial verb constructions, since subject coreference is not always observed, depending on the matrix verb (§\ref{sec:svc.finite.agreement}).
 
 \subsubsection{Coreference restrictions} \label{sec:finite.complement.coref}
There are four different patterns of coreference restriction between the arguments of finite complement clauses and those of the matrix verb.

The first type includes verbs like \japhug{spa}{be able} (§\ref{sec:spa.verb}), which require coreference between their subject and the (transitive or intransitive) subject of the complement clause. In (\ref{ex:tulata.mWjspea}), both the verb of the complement clause and the matrix verb are in \textsc{1sg}\fl{}3 form, with \textsc{1sg} coreference. The same coreference restriction is observed when this verb takes a velar infinitive complement (§\ref{sec:velar.inf.coreference}).
 
\begin{exe}
\ex \label{ex:tulata.mWjspea}
 \gll  [<weixin> tu-lat-a] nɯra mɯ́j-spe-a. \\
 Wechat \textsc{ipfv}-release-\textsc{1sg} \textsc{dem}:\textsc{pl} \textsc{neg}:\textsc{sens}-be.able[III]-\textsc{1sg} \\
 \glt `I am not able to use Wechat.' (conversation, 17-03-27)
  \end{exe}
 
The second type is represented by the semi-transitive com\-ple\-ment-taking verb \japhug{rga}{like}, which allows co-reference between its subject and either the (transitive or intransitive) subject (\ref{ex:kunWtaj.rgaj}) or the object (\ref{ex:tukWnAjoRjoRa.rgaa}) of the complement clause \citep[238]{jacques16complementation}. 
   
\begin{exe}
\ex \label{ex:kunWtaj.rgaj}
 \gll [nɯ ku-nɯ-ta-j] wuma ʑo rga-j   \\
 \textsc{dem} \textsc{ipfv}-\textsc{auto}-put-\textsc{1pl} really \textsc{emph} like:\textsc{fact}-\textsc{1sg} \\
 \glt `We like to put it (in the tea kettle to make tea).' (30-macha)
 \japhdoi{0003746\#S11}
\end{exe}
     
 \begin{exe}
\ex \label{ex:tukWnAjoRjoRa.rgaa}
 \gll [nɤʑo kɯ tu-kɯ-nɤjoʁjoʁ-a] nɯra rga-a \\
 \textsc{2sg} \textsc{erg} \textsc{ipfv}-2\fl{}1-flatter-\textsc{1sg} \textsc{dem}:\textsc{pl}  like:\textsc{fact}-\textsc{1sg}\\
\glt `I like it when you flatter me.' (elicited)
 \end{exe}
 
 The third type includes impersonal modal and aspectual verbs such as \japhug{ra}{be needed}, `be necessary' or \japhug{ntsʰi}{have better}, which take their complement clause as as intransitive subject (§\ref{sec:ra.khW.jAG.verb}. For instance, in (\ref{ex:pjWtWsi.ntshi}), the modal auxiliary \forme{ntsʰi} can only occur in \textsc{3sg} form, and cannot take the \textsc{2sg} indexation of the verb in complement clause.
 
 \begin{exe}
\ex \label{ex:pjWtWsi.ntshi}
 \gll [nɤʑo mɤlɤn ʑo pjɯ-tɯ-si] ntsʰi\\
 \textsc{2sg} absolutely \textsc{emph} \textsc{ipfv}-2-die be.better:\textsc{fact} \\
 \glt `You must die!' (140512 fushang he yaomo1-zh)
\japhdoi{0003969\#S22}
 \end{exe} 
 
 Fourth, verbs of perception and cognition such as \japhug{mto}{see} (§\ref{sec:mto.mtshAm.complement}) have no constraints on person indexation and neither require nor prohibit any coreference between main and complement clause.

\subsubsection{Finite complements vs. serial verb constructions} \label{sec:svc.finite.agreement}
Some com\-ple\-ment-taking verbs such as \japhug{cʰa}{can} occur with clauses sharing the same subject and the same TAME category, as in (\ref{ex:kotshi.pjAcha}). This could appear to be analyzable as a serial verb construction (§\ref{sec:svc}) or pseudocoordination \citep{loedrup14agree}.

\begin{exe}
\ex   \label{ex:kotshi.pjAcha}
\gll qajdo nɯ kɯ [tɯ-ci ko-tsʰi] pjɤ-cʰa. \\
crow \textsc{dem} \textsc{erg} \textsc{indef}.\textsc{poss}-water \textsc{ifr}-drink \textsc{ifr}-can \\
\glt `The crow succeeded in drinking water.' (aesop kouke de wuya-zh)
\end{exe}

However, I prefer to analyze these constructions as a particular type of finite complements with TAME agreement between the matrix verb and the complement verb (§\ref{sec:TAM.finite}). 

Subject coreference is not a specificity of this construction, but rather a property of the com\-ple\-ment-taking verb (§\ref{sec:finite.complement.coref}). With impersonal verbs such as \japhug{ra}{be needed}, which do not require subject coreference, TAME agreement is also attested, as in (\ref{ex:totinW.pjAra}), where the verb \forme{to-ti-nɯ} is in the Inferential (a category that does not normally occur in finite complement clauses, §\ref{sec:finite.complement.coref}) by agreeing with the matrix verb \forme{pjɤ-ra}. 

\begin{exe}
\ex   \label{ex:totinW.pjAra}
\gll [``ɣa" nɤ ``ɣa" nɯ to-ti-nɯ] pjɤ-ra. \\
yes \textsc{add} yes \textsc{dem} \textsc{ifr}-say-\textsc{pl} \textsc{ifr}.\textsc{ipfv}-be.needed \\
\glt `They had no other choice but to say `yes'.' (140518 jinyin chengbao-zh)
\japhdoi{0004028\#S54}
\end{exe}

Phasal verbs such as \japhug{ʑa}{begin} and \japhug{stʰɯt}{finish} (§\ref{sec:aspectual.complement}), which generally select bare/dental infinitives (§\ref{sec:bare.dental.inf}), are marginally used in the construction, as in (§\ref{ex:tAtWnAmat.tAtWZat}).  
 
\begin{exe}
\ex \label{ex:tAtWnAmat.tAtWZat}
 \gll [jɯxɕo tɕʰi tɤ-tɯ-nɤma-t tɤ-tɯ-ʑa-t] ʑo nɯ, ɕɤr mɤɕtʂa ɯʑo tɤ-nɤme ra \\
 this.morning what \textsc{aor}-2-do-\textsc{pst}:\textsc{tr} \textsc{aor}-2-start-\textsc{pst}:\textsc{tr} \textsc{emph} \textsc{dem} night until \textsc{3sg} \textsc{imp}-do[III] be.needed:\textsc{fact} \\
\glt `That which you have started doing this morning, do it until night.' (140515 jiesu de laoren-zh)
\japhdoi{0004004\#S122}
\end{exe}

\subsection{Multiclausal complements} \label{sec:multiclausal.complements}
Complements are not always restricted to one single clause. Example (\ref{ex:pjWsAtse.tuNke}) illustrates a finite biclausal complement: \japhug{cʰa}{can} (§\ref{sec:cha.verb}) has two complement clauses, the first of which \forme{ɯ-mi pjɯ-sɯ-ɤtse} specifies the manner of the second one (§\ref{sec:svc.manner}).

\begin{exe}
\ex \label{ex:pjWsAtse.tuNke}
\gll ma ju-mtsaʁ ʁɟa ʑo ma nɯ ma [[ɯ-mi pjɯ-sɯ-ɤtse] [tu-ŋke]] mɯ́j-cʰa\\
\textsc{lnk} \textsc{ipfv}-jump completely \textsc{emph} \textsc{lnk} \textsc{dem} apart.from \textsc{3sg}.\textsc{poss}-foot \textsc{ipfv}-\textsc{caus}-be.inserted[III] \textsc{ipfv}-walk \textsc{neg}:\textsc{sens}-can \\
\glt `It only jumps, as it is not able to walk with its feet (by taking steps).'  (28-qaCpa2)
\japhdoi{0003716\#S4}
\end{exe} 

Multiclausal infinitive complements are also found, as in (\ref{ex:kAnWrtsW.kANke.ra.tAcha}), though they are ambiguous with infinitive complements embedded within other infinitive complements, as in (\ref{ex:kAndza.kAGAndWB}) (a manner causative complement, §\ref{sec:causative.manner.complement}) or complements containing converbial manner clauses (§\ref{sec:inf.converb}).


\begin{exe}
\ex \label{ex:kAnWrtsW.kANke.ra.tAcha}
\gll [kɤ-nɯrtsɯ] [kɤ-ŋke] ra tɤ-cʰa \\
\textsc{inf}-crawl \textsc{inf}-walk \textsc{pl} \textsc{aor}-can \\
\glt `When [the baby] is able to crawl and walk, ...' (140426 tApAtso kAnWBdaR1)
\end{exe} 

\begin{exe}
\ex \label{ex:kAndza.kAGAndWB}
\gll [[kɤ-ndza] kɤ-ɣɤ-ndɯβ] mɯ́j-kʰɯ tɕe lú-wɣ-sɯ-qioʁ ɲɯ-ŋu tɕe \\
\textsc{inf}-chew \textsc{inf}-\textsc{caus}-be.minute \textsc{neg}:\textsc{sens}-be.possible \textsc{lnk} \textsc{ipfv}-\textsc{inv}-\textsc{caus}-vomit \textsc{sens}-be \textsc{lnk} \\
\glt `[Dogs]$_i$ cannot chew [\forme{pɤŋɤxɕaj}]$_j$ (a type of grass) into fine parts, and it$_j$ makes them vomit.' (140505 panaxCAj)
\japhdoi{0003915\#S4}
\end{exe} 

Another type of biclausal complement, exemplified by (\ref{ex:pe.mApe.arAtCha}) and (\ref{ex:rtaR.mArtaR.zrAtCha}), comprises an affirmative verb form, followed by its negative counterpart, expressing a disjunction `whether $X$ or $\lnot X$'.

\begin{exe} 
\ex \label{ex:pe.mApe.arAtCha}
\gll  [[ɯ-ngra pe] [mɤ-pe]] nɯ, [[kɤ-nɤma pe] [mɤ-pe]] arɤtɕʰa \\
\textsc{3sg}.\textsc{poss}-salary good:\textsc{fact} \textsc{neg}-be.good:\textsc{fact} \textsc{dem} \textsc{inf}-work good:\textsc{fact} \textsc{neg}-be.good:\textsc{fact} be.determined.from:\textsc{fact} \\
\glt `Whether his salary is good or not is determined by/depends on whether his work is good or not.' (elicited)
\end{exe} 

\begin{exe} 
\ex \label{ex:rtaR.mArtaR.zrAtCha}
\gll  [[ɯ-spa rtaʁ] [mɤ-rtaʁ]] tɤ-z-rɤtɕʰe \\
\textsc{3sg}.\textsc{poss}-material be.enough:\textsc{fact} \textsc{neg}-be.enough:\textsc{fact} \textsc{imp}-\textsc{caus}-be.determined.from[III] \\
\glt `Determine [the quantity of clothes that you are going to make] based on whether the [quantity of] cloth is sufficient.' (elicited)
\end{exe}

With bare and dental infinitives, multiclausal complements are only attested in the simultaneity construction (§\ref{sec:bare.dental.inf.sWpa}).

\subsection{Reported speech} \label{sec:reported.speech}
Verbs of speech (such as \japhug{ti}{say} or \japhug{fɕɤt}{tell}) and cognition (in particular \japhug{sɯso}{think}, `want') can take reported speech complements, in which the speaker either (exactly or partially) reproduces a sentence uttered by the person he is quoting, or verbalizes the words he assumes a person is thinking. These clauses have finite verb forms, and are thus a sub-category of finite complement clauses, but present some properties distinguishing them from the complements studied in §\ref{sec:finite.complement}.

\subsubsection{Sentence final particles} \label{sec:reported.speech.sfp}
Reported speech clauses stand out among complement clauses in having no restriction on the verb form, in particular in terms of TAME (§\ref{sec:TAM.finite}), and also in their ability to occur with sentence final particles, which are otherwise never found in subordinate clauses, including relatives, complements and other types of clauses . 

For instance, the complement clause in (\ref{ex:thWnWmblWmnW.je}) contains the imperative/hortative particle \forme{je} (§\ref{sec:fsp.imp}) as well as an interjection, and in (\ref{ex:mAnAtsa.loB.toti}) we find the utterance mitigation particle \forme{loβ} (§{sec:fsp.attitude}).   

 \begin{exe}
\ex \label{ex:thWnWmblWmnW.je}
 \gll ``ja, tʰɯ-nɯmbjɯm-nɯ je" to-ti. \\
 \textsc{interj} \textsc{imp}-warm.by.fire-\textsc{pl} \textsc{sfp} \textsc{ifr}-say \\
 \glt `She said: `Come, get warm by the fire!' (160703 poucet3-v2)
 \japhdoi{0006221\#S27}
 \end{exe} 
 
 \begin{exe}
\ex \label{ex:mAnAtsa.loB.toti}
 \gll tɤ-tɕɯ nɯ kɯ ``nɯ mɤ-nɤtsa loβ" to-ti, kʰro mɯ-to-kʰɯ. \\
 \textsc{indef}.\textsc{poss}-son \textsc{dem} \textsc{erg} \textsc{dem} \textsc{neg}-be.appropriate:\textsc{fact} \textsc{sfp} \textsc{ifr}-say much \textsc{neg}-\textsc{ifr}-agree \\
 \glt `The boy said `This is inappropriate' and did not agree [to marry her]. (150828 donglang)
\japhdoi{0006312\#S56}
\end{exe} 
  
\subsubsection{Hybrid indirect speech} \label{sec:hybrid indirect}
While Japhug allows direct speech quotation, the corpus reveals examples of mismatches between the viewpoint of the original speaker (the person whose speech or thoughts are quoted, subject of the com\-ple\-ment-taking verb of speech/thought which selects the reported speech clause) and the current speaker (the person quoting the words of the original speaker).

In the present work, these phenomena are referred to as Hybrid Indirect Speech (\citealt{jacques16complementation} following \citet{tournadre08conjunct}).\footnote{\citet{aikhenvald08semidirect} also uses the term ``Semi-Indirect Speech''. } In Hybrid Indirect Speech, the verb morphology (in particular person indexation) invariably presents the viewpoint of the original speaker, while pronouns and adverbs follow that of the current speaker.

Since grammatical relations are mainly marked by verb morphology and overt pronouns are not common (§\ref{sec:pers.pronouns}), distinguishing between Direct Speech and Hybrid Indirect Speech is only possible in a minority of cases. Mismatch between pronouns and person indexation only occurs when a pronoun or possessive prefix is overt, \textit{and} that pronoun or prefix indexes a referent that is assigned a different grammatical person by the  original speaker and the current speaker.
 
   
Example (\ref{ex:nWGi.kAsWso}) provides an example of this phenomenon. The verb \forme{nɯɣi} `he comes/will come back (home)' in the complement clause of the verb \forme{kɤ-sɯso} `think' is in the Factual Non-Past \textsc{3sg}. In the same clause we find the \textsc{2sg} pronoun \forme{nɤʑo}; there is no pause between the pronoun and the verb, and no indication from the prosody that \forme{nɤʑo} is left-dislocated. 


\begin{exe}
\ex \label{ex:nWGi.kAsWso}
\gll ma nɤ-wa kɯ [\rouge{nɤʑo} 	\bleu{nɯɣi}]  kɤ-sɯso kɯ kʰa ɯ-rkɯ tɕe ʁmaʁ χsɯ-tɤxɯr pa-sɯ-lɤt ɕti tɕe \\
\textsc{lnk} \textsc{2sg}.\textsc{poss}-father \textsc{erg} \textsc{2sg} {come.back:\textsc{fact}}  \textsc{inf}-think \textsc{erg} house \textsc{3sg}.\textsc{poss}-side \textsc{lnk} soldier three-circle \textsc{aro}:3\flobv{}-\textsc{caus}-release be.\textsc{aff}:\textsc{fact} \textsc{lnk}\\
\glt \textbf{Direct}: `Your father, thinking `\bleu{He is coming back}', put three rings of soldiers around the house.' 
\glt  \textbf{Indirect}: `Your father, thinking that \rouge{you are coming back},'
\glt  \textbf{Hybrid indirect}: `Your father, thinking that `\rouge{you}' \bleu{is coming back},' (qachGa 2003)
\end{exe}

This type of mismatch between pronouns and indexation on the verb is anomalous and never found in independent sentences. Here the verb form corresponds to the point of view of the original speaker (indicated in blue in all following examples), whose original sentence would have been \forme{ɯʑo nɯɣi} (\textsc{3sg} {come.back:\textsc{fact}} `he is coming back)'. The pronoun reflects the point of view of the current speaker (in red), for whom the equivalent sentence would be converted to \forme{nɤʑo tɯ-nɯɣi} (\textsc{2sg} {2-come.back:\textsc{fact}} `you are coming back'), since the addressee of the current situation corresponds to the subject of the original situation.

Examples (\ref{ex:WtCW.anAtWGa}), (\ref{ex:tunAmea}) and (\ref{ex:WkWmtChW.juGWta}) illustrate that possessive prefixes on nouns undergo the same shift towards the point of the view of the current speaker, while the verb remains in the same form that was either thought or uttered by the original speaker.

In (\ref{ex:WtCW.anAtWGa}) and (\ref{ex:tunAmea}), the possessive forms \forme{ɯ-tɕɯ} `his son' and \forme{ɯ-pi}`her brother'  underwent a shift to third person (representing the point of view of the current speaker). The original sentences corresponding to the complement clauses in (\ref{ex:WtCW.anAtWGa}) and (\ref{ex:tunAmea}) are presented in (\ref{ex:WtCW.anAtWGa2}) and (\ref{ex:tunAmea2}): the possessive pronoun is first person and coreferential with the subject of the main verb.
 
\begin{exe}
\ex 
\begin{xlist}
\ex \label{ex:WtCW.anAtWGa}
\gll rgɤtpu nɯ kɯ ``\rouge{ɯ-tɕɯ} nɯ \bleu{a-nɯ-ɤtɯɣ-a}" ɲɤ-sɯso ɕti \\
old.man \textsc{dem} \textsc{erg} \textsc{3sg}.\textsc{poss}-son \textsc{dem} \textsc{irr}-\textsc{pfv}-meet-\textsc{1sg} \textsc{ifr}-think be.\textsc{aff}:\textsc{fact} \\
\glt  \textbf{Direct}: `The old man thought ``\bleu{I wish I could meet} \bleu{my son}".'
\glt  \textbf{Indirect}:  `The old man$_i$ \rouge{wanted to meet his$_i$ son}.'
\glt  \textbf{Hybrid indirect}:  `The old man$_i$ thought \bleu{I$_i$ wish I$_i$ could meet} \rouge{his$_i$son}".' (150908 menglang-zh)
\japhdoi{0006320\#S36}
\ex \label{ex:tunAmea}
\gll tɕendɤre ta-ʁi nɯ kɯ [\rouge{ɯ-pi} ɣɯ ɯ-sci \bleu{tu-nɤme-a} ra] ɲɤ-sɯso tɕe, \\
\textsc{lnk}  \textsc{indef}.\textsc{poss}-younger.sibling \textsc{dem} \textsc{erg}  {\textsc{3sg}.\textsc{poss}-elder.sibling}  \textsc{gen} \textsc{3sg}.\textsc{poss}-revenge {\textsc{ipfv}-make[III]-\textsc{1sg}} be.needed:\textsc{fact} \textsc{ifr}-think \textsc{lnk} \\
\glt  \textbf{Direct}: `The [younger] sister thought ``\bleu{I have to get revenge} on \bleu{my brother}".'
\glt  \textbf{Indirect}:  `The [younger] sister$_i$ \rouge{wanted to get revenge on her$_i$ brother}.'
\glt  \textbf{Hybrid indirect}:  `The [younger] sister$_i$ thought \bleu{I$_i$ have to get revenge} on \rouge{her$_i$ brother}".' (xiong he mei)
\end{xlist}
\end{exe} 

\begin{exe}
\ex
\begin{xlist}
\ex \label{ex:WtCW.anAtWGa2}
\gll \bleu{a-tɕɯ} nɯ \bleu{a-nɯ-ɤtɯɣ-a} (ra) \\
 \textsc{1sg}.\textsc{poss}-son \textsc{dem} \textsc{irr}-\textsc{pfv}-meet-\textsc{1sg} be.needed:\textsc{fact}\\
\glt `I wish I could meet my son.' (elicitation based on \ref{ex:WtCW.anAtWGa})
\ex \label{ex:tunAmea2}
\gll \bleu{a-pi} ɣɯ ɯ-sci \bleu{tu-nɤme-a} ra \\
 {\textsc{1sg}.\textsc{poss}-elder.sibling}  \textsc{gen} \textsc{3sg}.\textsc{poss}-revenge {\textsc{ipfv}-make[III]-\textsc{1sg}} be.needed:\textsc{fact}  \\
\glt `I have to get revenge on my brother.' (elicitation based on \ref{ex:tunAmea})
\end{xlist}
  \end{exe}
    
Example (\ref{ex:WkWmtChW.juGWta}) illustrates the same phenomenon as in (\ref{ex:WtCW.anAtWGa}) and (\ref{ex:tunAmea}), but with the verb of speech \japhug{ti}{say} instead of \japhug{sɯso}{think}. In this example, we know from the context that the girl is the addressee, so that if the sentence were in direct speech, a second person singular prefix form \forme{nɤ-kɯmtɕʰɯ} (\textsc{2sg}.\textsc{poss}-toy) `your toy' would be expected instead.
 
\begin{exe}
\ex \label{ex:WkWmtChW.juGWta}
\gll  tɤɕime nɯ kɯ pjɯ-tɯ-mtsʰɤm tɕe, [nɯnɯ \rouge{ɯ-kɯmtɕʰɯ} nɯ  	\bleu{ju-ɣɯt-a} ŋu] ɯ-kɯ-ti pjɤ-tu ndɤre, \\
girl \textsc{dem} \textsc{erg} \textsc{ipfv}-\textsc{conv}:\textsc{imm}-hear \textsc{lnk} \textsc{dem}  \textsc{3sg}.\textsc{poss}-toy  \textsc{dem}  \textsc{ipfv}-bring-\textsc{1sg}  be:\textsc{fact} \textsc{3sg}.\textsc{poss}-\textsc{sbj}:\textsc{pcp}-say \textsc{ifr}.\textsc{ipfv}-exist \textsc{lnk} \\
\glt   \textbf{Direct}: `As soon as the girl heard that there was someone saying ``\bleu{I will bring your toy}".'
\glt   \textbf{Indirect}:  `As soon as the girl heard that there was someone saying that \rouge{he would bring her toy}.'
\glt   \textbf{Hybrid indirect}: `As soon as the girl$_i$ heard that there was someone saying ``\bleu{I will bring} \rouge{her$_i$ toy}".' (140429 qingwa wangzi-zh)
\japhdoi{0003890\#S49}
\end{exe}


In (\ref{ex:mWpWrAZia}), one could be tempted to analyze the pronoun \japhug{ɯʑo}{he} as exterior to the reported speech clause, as the subject of the matrix verb \japhug{sɯso}{think}. However, since \forme{sɯso} is transitive and requires its subject to be marked with the ergative (§\ref{sec:A.kW}), this analysis is not possible. Instead,  \japhug{ɯʑo}{he} belongs to the complement clause whose verb \japhug{rɤʑi}{stay} is intransitive. The person mismatch, as in (\ref{ex:nWGi.kAsWso}) above, is due to Hybrid Indirect Speech: the verb form \forme{mɯ-pɯ-rɤʑi-a} with first singular marking reflects the viewpoint of the original speaker (the subject of the verb \forme{ɲɯ-nɯ-sɯsɤm}), while the pronoun \japhug{ɯʑo}{he} corresponds to that of the current speaker (the narrator of the story). 
  
\begin{exe}
\ex \label{ex:mWpWrAZia}
\gll   ``\bleu{ɯʑo} χsɯ-sŋi χsɤ-rʑaʁ ma \rouge{mɯ-pɯ-rɤʑi-a}" ɲɯ-nɯ-sɯsɤm pjɤ-ŋu \\
 \textsc{3sg} three-day  three-night apart.from \textsc{neg}-\textsc{pst}.\textsc{ipfv}-stay-\textsc{1sg} \textsc{ipfv}-\textsc{auto}-think[III] \textsc{pst}.\textsc{ipfv}-be \\
\glt    \textbf{Direct}: `He was thinking ``\bleu{I have only stayed} for three days and three nights".'
\glt    \textbf{Indirect}: `He was thinking that \rouge{he had only stayed} for three days and three nights.'
\glt  \textbf{Hybrid Indirect}: `He was thinking that \rouge{he} \bleu{have only stayed} for three days and three nights.' 
\japhdoi{0003376\#S96}
\end{exe}

 

A potentially even more confusing case occurs when the original speaker is the current speaker's addressee, and when both the original and the current speakers are referred to in the original utterance. This is the situation observed in (\ref{ex:YWnWfsWGa}), a sentence pronounced by a fox who helped a prince to succeed in various tasks. Here,  the first singular possessive prefix \forme{a-} on the possessed noun \japhug{ɯ-tʂɯnlɤn}{favour} and the first person singular suffix \forme{-a} on the verb \forme{ɲɯ-nɯ-fsɯɣ-a} do not correspond to the same referent. The verb form \forme{ɲɯ-nɯ-fsɯɣ-a} `I will pay back' is the sentence that the fox attributes to his addressee (the prince), so that the first person here corresponds to the prince, while the possessive prefix on \japhug{ɯ-tʂɯnlɤn}{favour} reflects the point of view of the fox and thus refers to himself.

 \begin{exe}
 \ex \label{ex:YWnWfsWGa}
\gll  \rouge{a-tʂɯnlɤn} \bleu{ɲɯ-nɯ-fsɯɣ-a} ɯ-ɲɯ-tɯ-sɯsɤm nɤ, nɯ tɤ-ste ti ɲɯ-ŋu \\
  {\textsc{1sg}.\textsc{poss}-favour} {\textsc{ipfv}-\textsc{auto}-pay.back-\textsc{1sg}} \textsc{q}-\textsc{ipfv}-2-think[III] \textsc{lnk} \textsc{dem} \textsc{imp}-do.this.way[III] say:\textsc{fact} \textsc{sens}-be \\
\glt    \textbf{Direct}: `If you think ``\bleu{I will return the favour (which I received from you)}", do like that.'
\glt    \textbf{Indirect}: `If you want to \rouge{return the favour (which you received from me}), do like that.'
\glt   \textbf{Hybrid Indirect}: `If you think ``\bleu{I will return} the \rouge{favour (which you received from me}), do like that.'
\japhdoi{0003372\#S192}
\end{exe}

In such a situation, the referents corresponding to first and second person are exactly reversed between the point of view of the current and the original speaker, and therefore between pronouns and possessive prefixes on the one hand and verbal indexation on the other hand.
   
The corresponding sentence in Direct speech would be (\ref{ex:YWnWfsWGa2}), with a second person singular possessive prefix on the noun  \japhug{ɯ-tʂɯnlɤn}{favour} instead.

\begin{exe}
\ex \label{ex:YWnWfsWGa2}
\gll nɤ-tʂɯnlɤn ɲɯ-nɯ-fsɯɣ-a \\
  {\textsc{2sg}.\textsc{poss}-favour} {\textsc{ipfv}-\textsc{auto}-pay.back-\textsc{1sg}} \\ 
 \glt `I will return the favour (which I received from you).'
\end{exe}


Surprisingly, despite this complex shift of perspective between the original speaker and the current speaker, there is no logophoric pronoun in Japhug \citep{hagege74logophoriques, nikitina12logophoric}. A logophoric pronoun is, however, attested in the closely related Stau language, which appears to have a similar system of Hybrid Indirect Speech \citep{jacques17stau}.

Hybrid indirect speech is not rare in Japhug, and this grammar contains additional examples, such as  (\ref{ex:mAwGndzaj.thaN}) in §\ref{sec:fsp.epistemic}).



\section{Morphosyntactic properties of complement clauses}  \label{sec:complement.morphosyntax}
\subsection{Word order and constituency}  \label{sec:complement.word.order}
Complement clauses, are strictly preverbal in Japhug like other core arguments (§\ref{sec:basic.word.order}), except in the case of right dislocated constituents (§\ref{sec:right.dislocation}) such as the infinitive clause \forme{kɤ-sɤ-fstɯn} `to serve' in (\ref{ex:tutWste.kAsAfstWn}) (§\ref{sec:similative.verb.complementation}).

\begin{exe}
\ex \label{ex:tutWste.kAsAfstWn}
\gll tɕʰi tu-tɯ-ste ŋu, [kɤ-sɤ-fstɯn]\\
what \textsc{ipfv}-2-do.like[III] be:\textsc{fact} \textsc{inf}-\textsc{apass}-serve \\
\glt `How do you treat people?' (2002 qaCpa)
\end{exe}

While complement clauses are generally located directly before the verb, in examples such as (\ref{ex:kACWnNo.mWYWchaa}) the subject of the matrix verb inserted after the complement clause.

\begin{exe}
\ex \label{ex:kACWnNo.mWYWchaa}
\gll  [nɤʑo kɤ-ɕɯ-nŋo] aʑo mɯ-ɲɯ-cʰa-a \\
\textsc{2sg} \textsc{inf}-\textsc{caus}-be.defeated \textsc{1sg} \textsc{neg}-\textsc{ipfv}-can-\textsc{1sg} \\
\glt `I cannot defeat you anymore.' (140513 abide he mogui-zh)
\japhdoi{0003975\#S83}
\end{exe}

Discontinuous complement clauses are rare in Japhug. The only clear example in the corpus is (\ref{ex:lWlu.kW.aZo}). In this example, the \textsc{1sg} pronoun \forme{aʑo} is the subject of the matrix clause, and has no syntactic role in the complement clause, but it appears between the transitive subject \forme{lɯlu kɯ} `the cat' and the object \japhug{ʁnɯz}{two} of the complement clause. Despite the rarity of this construction, this sentence was not considered to be unusual by Tshendzin when listening again to the recording.
 
 \begin{exe}
\ex \label{ex:lWlu.kW.aZo}
\gll [lɯlu kɯ \textbf{aʑo} ʁnɯz ʑo ka-ndo] pɯ-mto-t-a \\
cat \textsc{erg} \textsc{1sg} two \textsc{emph} \textsc{aor}:3\flobv{}-take \textsc{aor}-see-\textsc{pst}:\textsc{tr}-\textsc{1sg} \\
\glt `I saw a cat catching two of them.' (22-kumpGatCW)
\japhdoi{0003590\#S53}
\end{exe}

\subsection{Case marking} \label{sec:complement.clause.case.marking}
When the verb of the matrix and the complement clauses sharing the same subject have different transitivity values, there is a conflict in case assignment on their common subject, which can can either take absolutive (§\ref{sec:absolutive.S}) or ergative marking (§\ref{sec:erg.kW}). 

Examples (\ref{ex:no.erg.rga}) and  (\ref{ex:erg.rga})  provide a minimal pair illustrating this optional treatment. In both examples, the matrix verb \japhug{rga}{like} is semi-transitive (and its subject cannot take ergative marking), while \japhug{ndza}{eat} is transitive (and thus requires a subject with the ergative).

In (\ref{ex:no.erg.rga}), the subject \forme{fsapaʁ ra} `farm animals' has no ergative marking, showing that its case marking is assigned by the semi-transitive \japhug{rga}{like} (§\ref{sec:semi.transitive}), and therefore that the infinitival complement clause in this example is restricted to the sole infinitive verb form \forme{kɤ-ndza} `to eat'.

\begin{exe}
\ex \label{ex:no.erg.rga}
\gll fsapaʁ ra [kɤ-ndza] wuma rga-nɯ  \\
animals \textsc{pl}  \textsc{inf}-eat very  like:\textsc{fact}-\textsc{pl} \\
\glt `Farm animals like to eat it.' (19-qachGa mWntoR)
\japhdoi{0003546\#S107}
\end{exe}

By contrast, in example (\ref{ex:erg.rga}), the common subject \forme{paʁ ra}  `pigs' takes the ergative \forme{kɯ} selected by the transitive verb  \japhug{ndza}{eat} in the complement clause, suggesting that it should be analyzed as belonging to the complement clause.

\begin{exe}
\ex \label{ex:erg.rga}
\gll [paʁ ra kɯ kɤ-ndza] wuma ʑo rga-nɯ \\
pig \textsc{pl} \textsc{erg} \textsc{inf}-eat very \textsc{emph}  like:\textsc{fact}-\textsc{pl} \\
 \glt `Pigs like to eat it.' (12 ndZiNgri)
\japhdoi{0003488\#S137}
\end{exe}

In (\ref{ex:erg.rga}) it is not possible to argue that the case marking on the subject is a case of long-distance ergative (§\ref{sec:long.distance.kW}), since  \forme{paʁ ra kɯ} does not serve as subject for a transitive verb located a few clauses afterwards; in fact there is no mention of the pigs in the rest of the text.


Examples (\ref{ex:tWNke}) and (\ref{ex:tWnWrAGo}) with dental infinitive complements (§\ref{sec:bare.dental.inf}) containing an intransitive verb (§\ref{sec:dental.inf}) and the transitive matrix verb \japhug{ʑa}{begin} illustrate the opposite situation: the transitive verb requires the ergative on third person subjects (§\ref{sec:erg.kW}), while the intransitive one precludes it. In (\ref{ex:tWNke}), the common subject is in the absolutive, showing that it owes its case marking to the intransitive verb \japhug{ŋke}{walk}, and therefore that it belongs to the complement clauses.

\begin{exe}
\ex \label{ex:tWNke}
\gll [<xinbada> nɯ tɕe li tɯ-ŋke] to-ʑa \\
Sinbad \textsc{dem} \textsc{lnk} again  \textsc{inf}-walk \textsc{ifr}-begin \\
\glt `Sinbad started to walk again.' (140511 xinbada-zh)
\japhdoi{0003961\#S206}
\end{exe}

In (\ref{ex:tWnWrAGo}), the common subject is in the ergative following the transitive matrix verb \japhug{ʑa}{begin} (§\ref{sec:phasal.complements}), and is thus located outside of the complement clause.

\begin{exe}
\ex \label{ex:tWnWrAGo}
\gll pɣɤtɕɯ nɯ kɯ [nɯɕimɯma ʑo tɯ-nɯrɤɣo] cʰɤ-ʑa \\
bird \textsc{dem} \textsc{erg} immediately \textsc{emph} \textsc{inf}-sing \textsc{ifr}-begin \\
\glt `The bird immediately started to sing.' (140514 huishuohua de niao-zh)
\japhdoi{0003992\#S208}
\end{exe}

In the case of transitive matrix verbs such as \japhug{rɲo}{experience} that allow coreference between the subject of the matrix clause and either the subject or object of the complement clause, the common argument receives ergative marking even when it is the object in the complement clause, as illustrated by example (\ref{ex:kAmtsWG.pjArYo}) (§\ref{sec:rYo.complements}).

\subsection{Determiners} \label{sec:complement.determiner}
Finite (§\ref{sec:finite.complement}), reported speech (§\ref{sec:reported.speech}) and infinitive (§\ref{sec:velar.infinitives.complement.clauses}) complement clauses, like relative clauses (§\ref{sec:relative.determiners.complementizer}), are often followed by \forme{nɯ} and \forme{nɯnɯ} (\ref{ex:mAkACe.nW.mAkhW})  (which occur as  demonstrative pronouns and determiners, §\ref{sec:demonstrative.pronouns}, §\ref{sec:demonstrative.determiners}, §\ref{sec:nW.topic}, §\ref{sec:definiteness}) and/or by the plural marker \forme{ra} (\ref{ex:YWnWqambWmbjom.ra.mWjspe}) (§\ref{sec:plural.determiners}), which adds the nuance that other activities may be implied. For instance, in (\ref{ex:YWnWqambWmbjom.ra.mWjspe}), \forme{ɲɯ-nɯqambɯmbjom ra mɯ́j-spe} can be glossed as `it is not able to fly (and do other related activities)'.

 \begin{exe}
\ex \label{ex:mAkACe.nW.mAkhW}
\gll [mɤ-kɤ-ɕe] nɯ mɤ-kʰɯ ri \\
\textsc{neg}-\textsc{inf}-go \textsc{dem} \textsc{neg}-be.possible:\textsc{fact} \textsc{lnk} \\
\glt `[I] have no choice but to go.' (2005 Norbzang)
\end{exe}

 \begin{exe}
\ex \label{ex:YWnWqambWmbjom.ra.mWjspe}
\gll  sɤtɕʰa nɯ ju-rɤtɣe kɯ-fse qʰe, tu-ŋke ma nɯ ma [ɲɯ-nɯqambɯmbjom] ra mɯ́j-spe \\
ground \textsc{dem} \textsc{ipfv}-measure.by.span \textsc{inf}:\textsc{stat}-be.like \textsc{lnk} \textsc{ipfv}-walk \textsc{lnk} \textsc{dem} apart.from \textsc{ipfv}-fly \textsc{pl} \textsc{neg}:\textsc{sens}-be.able[III] \\
\glt `The [inchworm] moves as if it were measuring the ground span by span, it is not able fly.' (26-qambalWla)
\japhdoi{0003680\#S72}
\end{exe}
 
These markers are analyzed in this grammar as determiners of the entire complement clause rather than as complementizers, the analysis proposed by \citet[481]{sun12complementation} concerning a similar construction in Tshobdun. As evidence for the analysis as determiners, note that the position of the markers \forme{nɯ} and \forme{ra} relative to complement clauses  is exactly the same as that between demonstrative and plural determiners and nouns in a noun phrase (§\ref{sec:demonstrative.determiners}). In particular, cirmcumposed determiners are attested with complement clauses as in (\ref{ex:nWra.jAGWta.ra.mWtoti}), where the reported speech complement \forme{tɕʰeme jɤ-ɣɯt-a} `I brought a girl' is both preceded and followed by plural markers.\footnote{Examples like (\ref{ex:nWra.jAGWta.ra.mWtoti}) directly refute my previous claim \citep[258]{jacques16complementation} that determiners of complement clauses are strictly postclausal. }

\begin{exe}
\ex \label{ex:nWra.jAGWta.ra.mWtoti}
\gll nɯra [tɕʰeme jɤ-ɣɯt-a] ra mɯ-to-ti. \\
\textsc{dem}.\textsc{pl} girl \textsc{aor}-bring-\textsc{1sg} \textsc{pl} \textsc{neg}-\textsc{ifr}-say \\
\glt  `He did not say that he had brought a girl home (and the related events).'  (150909 hua pi-zh)
\japhdoi{0006278\#S50}
\end{exe}
 
Bare and dental infinitival clauses (§\ref{sec:bare.dental.inf}) rarely take the determiners \forme{nɯ} and \forme{ra}, but examples such as (\ref{ex:tWmbri.ra.taZanW.pWtsu}) are attested in the corpus.

\begin{exe}
\ex \label{ex:tWmbri.ra.taZanW.pWtsu}
\gll [pɣɤtɕɯ ra tɯ-mbri] ra ta-ʑa-nɯ pɯ-tsu \\
bird \textsc{pl} \textsc{inf}:II-cry \textsc{pl} \textsc{aor}:3\fl{}3-start-\textsc{pl} \textsc{pst}.\textsc{ipfv}-have.time.to \\
\glt `[The solar eclipse] lasted long enough for the birds to start singing (as if night had come).' (29-RmGWzWn)
\japhdoi{0003730\#S30}
\end{exe}
 
 
\subsection{Restrictive and additive focus} \label{sec:complement.restriction}
Various focus markers are inserted between the complement clause and the matrix verb, including the additive/scalar focus marker \japhug{kɯnɤ}{also, even} (§\ref{sec:kWnA}) in (\ref{ex:kWnA.mAsna2}).

\begin{exe}
\ex \label{ex:kWnA.mAsna2}
\gll [kɤ-nɯ-βlɯ] kɯnɤ mɤ-sna \\
\textsc{inf}-\textsc{auto}-burn also \textsc{neg}-be.good:\textsc{fact} \\
\glt `It is not even good for burning (as firewood).' (11-qarGW)
\japhdoi{0003480\#S100}
\end{exe}

In (\ref{ex:manWma.compl}),  the exceptive construction \forme{ma nɯ ma} `apart from that' (§\ref{sec:exceptive})  located between the infinitival clause \forme{kɤ-mtsʰɤm} and the matrix verb has a different status from \forme{kɯnɤ} in (\ref{ex:kWnA.mAsna}), as the matrix verb \forme{pɯ-rɲo-t-a} (\textsc{aor}-experience-\textsc{pst}:\textsc{tr}-\textsc{1sg}) in affirmative form can be inserted after the complement clause (\forme{kɤ-mtsʰɤm \textbf{pɯ-rɲo-t-a} ma nɯ ma mɯ-pɯ-rɲo-t-a}) and its absence in (\ref{ex:manWma.compl}) is due to elision.

\begin{exe}
\ex \label{ex:manWma.compl}
\gll  [kɤ-mtsʰɤm] ma nɯ ma mɯ-pɯ-rɲo-t-a \\
\textsc{inf}-hear \textsc{lnk} \textsc{dem} apart.from \textsc{neg}-\textsc{aor}-experience-\textsc{pst}:\textsc{tr}-\textsc{1sg} \\
\glt `I only heard about it.' (I did not see it and do not even claim that it exists, of a mythological animal) (20-RmbroN)
\japhdoi{0003560\#S108}
\end{exe}

\subsection{Raising of preverb orientation} \label{sec:orientation.raising}
With the exception of finite complements and some velar infinitives (§\ref{sec:infinitives.other.prefixes}), verbs in complement clauses generally lack orientation preverbs, and only the matrix verb encodes orientation.

While some com\-ple\-ment-taking verbs keep the same orientation regardless of the complement type and the lexical orientation of the verb in the complement clause (for instance, \japhug{rɲo}{experience} always takes the \textsc{downwards} orientation, §\ref{sec:rYo.complements}), other matrix verbs, in particular causative verbs (§\ref{sec:causative.manner.complement}) and phasal verbs such as  \japhug{ʑa}{begin}  (§\ref{sec:phasal.complements}), select the orientation of the complement verb. This phenomenon is illustrated in \tabref{tab:Za.complement.preverb} and the following examples.


\begin{table}
\caption{Examples of raising of preverb orientation from the complement clause of \japhug{ʑa}{begin}, Inferential \textsc{3sg}(\flobv{})) } \label{tab:Za.complement.preverb}
\begin{tabular}{llllll}
\lsptoprule
Orientation & Example & &Dental/bare inf.&\\
&&& + \japhug{ʑa}{begin}&\\
\midrule
upwards & \forme{\textbf{to}-mna}  &(\ref{ex:YWnWqambWmbjom.tocha}),  &\forme{tɯ-mna \textbf{to}-ʑa} &(\ref{ex:tWmna.toZa}) \\
&`s/he got better' &§\ref{sec:ifr.inchoative}& `s/he started getting better' \\
downwards & \forme{\textbf{pjɤ}-fɕɤt} &(\ref{ex:WsAfCAt}), &\forme{ɯ-fɕɤt \textbf{pjɤ}-ʑa} &(\ref{ex:WfCAt.pjAZa}) \\
 &  `s/he told it' &§\ref{sec:other.oblique.participle.relatives}  & `s/he started telling it' \\
  \midrule
upstream & \forme{\textbf{lo}-fsoʁ}  &(\ref{ex:WkWCar.chACe}), & \forme{tɯ-fsoʁ \textbf{lo}-ʑa} &(\ref{ex:tWfsoR.loZa}) \\
 &   `the day broke' &§\ref{sec:purposive.clause.motion.verbs}&  `the day started breaking' \\
downstream & \forme{\textbf{cʰɤ}-lɤt}  &(\ref{ex:Juli.chAlAt}), &\forme{ɯ-lɤt \textbf{cʰɤ}-ʑa} &(\ref{ex:WlAt.chAZa}) \\
 & `s/he played' &§\ref{sec:preverb.speech} & `s/he started playing' \\
 & (the flute) && (the flute)&\\
 \midrule
eastwards & \forme{\textbf{ko}-rɟaʁ-ndʑi}  && \forme{tɯ-rɟaʁ \textbf{ko}-ʑa-ndʑi}   \\
 &   `they danced' &&  `they started dancing' &(\ref{ex:tWrJaR.koZandZi}) \\
westwards & \forme{\textbf{ɲɤ}-mɯnmu}   & &\forme{tɯ-mɯnmu \textbf{ɲɤ}-ʑa}   \\
  &  `s/he moved' & &  `s/he started moving' & (\ref{ex:tWmWnmu.YAZa})\\
\lspbottomrule
\end{tabular}
\end{table}


As shown in \tabref{tab:Za.complement.preverb}, the orientation preverb on \forme{ʑa} in examples (\ref{ex:tWmna.toZa}) to (\ref{ex:tWmWnmu.YAZa}) (in all of these cases a D-type preverb marking the Inferential, §\ref{sec:kamnyu.preverbs}) is the same as that found when the verb in the complement clause is used as a main verb conjugated in the Inferential. 
\newpage
\begin{exe} 
\ex \label{ex:tWmna.toZa}
\gll tɯ-mna to-ʑa tɕe \\
\textsc{inf}:II-be.better \textsc{ifr}:\textsc{up}-start \textsc{lnk} \\
\\
\glt `He started getting better.' (150907 yingning-zh)
\japhdoi{0006264\#S52}
\end{exe} 

\begin{exe} 
\ex \label{ex:WfCAt.pjAZa}
\gll ɯ-χpi ɯ-fɕɤt pjɤ-ʑa. \\
\textsc{3sg}.\textsc{poss}-story \textsc{3sg}.\textsc{poss}-\textsc{bare}.\textsc{inf}:tell \textsc{ifr}:\textsc{down}-start \\
\glt `He started telling her a story.' (140517 buaishuohua-zh)
\japhdoi{0004018\#S65}
\end{exe} 

\begin{exe} 
\ex \label{ex:tWfsoR.loZa}
\gll tɯ-fsoʁ lo-ʑa tɕe, \\
\textsc{inf}:II-be.light \textsc{ifr}:\textsc{upstream}-start \textsc{lnk} \\
\\
\glt `The day broke.' (140511 yinzi-zh)
\japhdoi{0003963\#S38}
\end{exe} 

\begin{exe} 
\ex \label{ex:WlAt.chAZa}
\gll  ɟuli ɯ-lɤt cʰɤ-ʑa  \\
flute  \textsc{3sg}.\textsc{poss}-\textsc{bare}.\textsc{inf}:release \textsc{ifr}:\textsc{downstream}-start \\
\glt `He started playing the flute.' (140513 mutong de disheng-zh)
\japhdoi{0003977\#S138}
\end{exe} 

\begin{exe} 
\ex \label{ex:tWrJaR.koZandZi}
\gll tɯ-rɟaʁ ko-ʑa-ndʑi \\
\textsc{inf}:II-dance \textsc{ifr}:\textsc{east}-start-\textsc{du} \\
\glt `They started dancing.' (140504 huiguniang-zh)
\japhdoi{0003909\#S141}
\end{exe} 

\begin{exe} 
\ex \label{ex:tWmWnmu.YAZa}
\gll tɯ-mɯnmu ɲɤ-ʑa \\
\textsc{inf}:II-move \textsc{ifr}:\textsc{west}-start \textsc{lnk} \\
\\
\glt `It started moving.' (150904 yaoshu-zh)
\japhdoi{0006394\#S77}
\end{exe} 

When an orientable verb (§\ref{sec:orientable.verbs}) is found in the complement clause, \forme{ʑa} can take the indefinite orientation preverbs as in (\ref{ex:tWCe.joZa}).

\begin{exe} 
\ex \label{ex:tWCe.joZa}
\gll li tɯ-ɕe jo-ʑa \\
again \textsc{inf}:II-go \textsc{ifr}:\textsc{indefinite}-start \\
\glt `He started leaving.' (140511 xinbada-zh)
\japhdoi{0003961\#S273}
\end{exe} 

The verb \japhug{ʑa}{begin} can also be used with nominal objects as in (\ref{ex:rAGo.chAZa}). 

\begin{exe}
\ex \label{ex:rAGo.chAZa}
\gll rɤɣo cʰɤ-ʑa \\
song \textsc{ifr}:\textsc{downstream}-start \\
\glt `It started singing (i.e. began a song).' (140519 yeying-zh)
\japhdoi{0004040\#S75}
\end{exe}

If the object in question has a corresponding denominal verb, such as \japhug{nɯrɤɣo}{sing} from \japhug{rɤɣo}{song} (§\ref{sec:denom.intr.nW}), the same orientation preverb that \forme{ʑa} takes when used with the base noun will be the same as that selected by the corresponding denominal verb: for instance, \textsc{downstream} is found in both cases in (\ref{ex:rAGo.chAZa}) and (\ref{ex:tWnWrAGo.chAZa}).


\begin{exe}
\ex \label{ex:tWnWrAGo.chAZa}
\gll pɣɤtɕɯ nɯ kɯ nɯɕimɯma ʑo tɯ-nɯrɤɣo cʰɤ-ʑa  \\
bird \textsc{dem} \textsc{erg} immediately \textsc{emph} \textsc{inf}-sing \textsc{ifr}:\textsc{downstream}-start \\
\glt `The bird immediately started to sing.' (140514 huishuohua de niao)
\japhdoi{0003992\#S208}
\end{exe}

All phasal verbs (§\ref{sec:phasal.complements}) and causative com\-ple\-ment-taking verbs behave like \forme{ʑa}. Additional examples of this phenomenon with \japhug{stʰɯt}{finish} are presented in §\ref{sec:phasal.complements}.


\section{Complementation strategies}  \label{sec:strategies}
\citet{dixon06complementation} introduces the term ``complementation strategy'' to refer to constructions with a meaning corresponding to that expressed by complement clauses in some languages, but which either are not core arguments or the verb of the main clause or are not clauses with a complete argument structure \citep[34--40]{dixon06complementation}. Complementation strategies include nominalizations (when the verb sheds its argument structure as it becomes a noun), relative clauses (which are formally a modifier of a core argument), serial verb constructions and clause linking.

\subsection{Relative clauses in core argument function}   \label{sec:relative.core.arg}
Some com\-ple\-ment-taking verbs can alternatively select (semi-)objects instead of complement clauses. For instance \japhug{cʰa}{can} occurs with the noun \ch{考试}{kǎoshì}{exam} (borrowed from Chinese) (\ref{ex:kaoshi.pWcha}).

\begin{exe}
	\ex \label{ex:kaoshi.pWcha}
	\gll   <kaoshi> pɯ-cʰa  \\
	exam \textsc{aor}-can \\
	\glt `He passed the exam.' (12-BzaNsa)
	\japhdoi{0003484\#S69}
\end{exe}

In (\ref{ex:tWtatWt.pjAcha}), the clause \forme{tɯ\redp{}ta-tɯt} superficially resembles a finite complement clause (§\ref{sec:finite.complement}), but four pieces of evidence indicate that it should rather be analyzed as a headless relative clause in semi-object function like the noun \forme{<kaoshi>} in (\ref{ex:kaoshi.pWcha}).

\begin{exe}
	\ex \label{ex:tWtatWt.pjAcha}
	\gll  tɤ-pɤtso nɯ kɯ nɯra [tɯ\redp{}ta-tɯt] nɯra pjɤ-cʰa \\
	\textsc{indef}.\textsc{poss}-child \textsc{dem} \textsc{erg} \textsc{dem}:\textsc{pl} \textsc{total}\redp{}\textsc{aor}:3\flobv{}-say[II] \textsc{dem}:\textsc{pl} \textsc{ifr}-can \\
	\glt `The child had succeeded in doing everything that [the old king] had said.' (140428 yonggan de xiaocaifeng-zh)
	\japhdoi{0003886\#S240}
\end{exe}

First, the subject of both clauses are not co-referential (this example cannot be interpreted as meaning `the boy succeeded in saying all these things'). If the subordinate clause in (\ref{ex:tWtatWt.pjAcha}) were a complement, subject coreference would be expected (§\ref{sec:finite.complement.coref}). Second, the verb of the relative clause has totalitative reduplication, a morphological device  found in relative clauses (§\ref{sec:totalitative.relatives}), but not in complement clauses. Third, the verb of the relative clause is in the Aorist while that of the the main clause is in the Inferential; in finite complement clauses other than reported speech, the verb could be in the Aorist only if the matrix verb were in Aorist form too (see §\ref{sec:TAM.finite}). Fourth, it is possible to add an overt head noun in (\ref{ex:tWtatWt.pjAcha}).


Correlative relative clauses (§\ref{sec:interrogative.relative}) particularly commonly occur as objects or semi-objects. For instance, in (\ref{ex:tAstuta.tustea}) the clause \forme{ɯ-wa tɕʰi tɤ-stu-t-a} can be analyzed as the semi-object of the secundative verb \japhug{stu}{do like} (§\ref{sec:ditransitive.secundative}, §\ref{sec:similative.verb.complementation}).

\begin{exe}
	\ex \label{ex:tAstuta.tustea}
	\gll [ɯ-wa tɕʰi tɤ-stu-t-a] nɯ tu-ste-a ɲɯ-ɬoʁ \\
	\textsc{3sg}.\textsc{poss}-father what \textsc{aor}-do.like-\textsc{pst}:\textsc{tr}-\textsc{1sg} \textsc{dem} \textsc{ipfv}-do.like[III]-\textsc{1sg} \textsc{sens}-be.needed \\
	\glt `I have to deal with him in the same way as I dealt with his father. (=How I treated his father, I have to treat him like that)' (2012 Norbzang)
\japhdoi{0003768\#S135}
\end{exe}

Ambiguity between finite or participial relative clauses on the one hand, and finite and infinitival complement clauses on the other hand, is pervasive in the case of verbs of cognition and perception such as \japhug{mto}{see} (§\ref{sec:finite.relative.complement.ambiguity}, §\ref{sec:non-finite.relative.complement.ambiguity}) as in (\ref{ex:WftaR.tAkAta}), where \forme{tɤ-kɤ-ta} can be either analyzed as an object participle (§\ref{sec:object.participle}) or a velar infinitive (§\ref{sec:velar.infinitives.complement.clauses}). Disambiguization between the two analyses is possible when the verb is dynamic intransitive (§\ref{sec:non-finite.relative.complement.ambiguity}).

\begin{exe}
	\ex  \label{ex:WftaR.tAkAta}
	\gll nɯɕimɯma ʑo iɕqʰa [kɯm nɯtɕu ɯ-ftaʁ tɤ-kɤ-ta] nɯ pjɤ-mto \\
	immediately \textsc{emph} the.aforementioned door \textsc{dem}:\textsc{loc} \textsc{3sg}.\textsc{poss}-mark \textsc{pdv}-\textsc{inf}/\textsc{obj}:\textsc{pcp}-put \textsc{dem} \textsc{ifr}-see \\
	\glt `She immediately saw the mark that had been put on the door / that someone had put a mark on the door.' (140512 alibaba-zh)
\japhdoi{0003965\#S177}
\end{exe}

\subsection{Participial clauses} \label{sec:participial.clause.complementation strategies}
Putting aside participial relative clauses used with verbs compatible with both nouns and complement clauses (§\ref{sec:relative.core.arg}, §\ref{sec:non-finite.relative.complement.ambiguity}), participial clauses are found in four types of complements and complementation strategies: purposive clauses, constructionalized object relative clauses, essive participial clauses and genuine participial complements.

\subsubsection{Purposive clauses of motion verbs} \label{sec:purposive.clause.motion.verbs}
The motion verbs \japhug{ɕe}{go}, \japhug{ɣi}{come} and  \japhug{ɬoʁ}{come out} (§\ref{sec:motion.verbs}) select purposive clauses whose verb is obligatorily in participle form (§\ref{sec:am.vs.mvc}).\footnote{This construction is reminiscent of the use of the future participle with motion verbs in Ancient Greek \citep[§177.B]{vernhes96hermaion}. } The participles in this construction cannot take orientation, polarity and associated motion prefixes.

Both subject and object purposive clauses are found. Subject purposive clauses are subject participial clauses (§\ref{sec:subject.participle.complementation}), as in (\ref{ex:WkWCar.chACe}) and (\ref{ex:WkWCar.jolhoR}). In this construction, there is obligatory coreference between the subject of the motion verb and that of the purposive clause, unlike other superficially similar constructions (§\ref{sec:constr.participial.clause}).

\begin{exe}
	\ex \label{ex:WkWCar.chACe}
	\gll lo-fsoʁ tɕe tɕe tɤ-mu nɯ [ɯ-tɕɯ ɯ-kɯ-ɕar] cʰɤ-ɕe tɕe,\\
	\textsc{ifr}-be.bright \textsc{lnk} \textsc{lnk} \textsc{indef}.\textsc{poss}-mother \textsc{dem} \textsc{3sg}.\textsc{poss}-son \textsc{3sg}.\textsc{poss}-\textsc{sbj}:\textsc{pcp}-search \textsc{ifr}:\textsc{downstream}-go \textsc{lnk} \\
	\glt `When the sun came up (in the morning), the mother went to look for her son.' (tWJo 2012)
\japhdoi{0004089\#S31}
\end{exe}

\begin{exe}
	\ex \label{ex:WkWCar.jolhoR}
	\gll pri nɯ [ɯ-zda ɯ-kɯ-ɕar] jo-ɬoʁ. \\
	bear \textsc{dem} \textsc{3sg}.\textsc{poss}-companion \textsc{3sg}.\textsc{poss}-\textsc{sbj}:\textsc{pcp}-search \textsc{ifr}-come.out \\
	\glt `The bear came out to look for its companion (another bear).' (elicited)
\end{exe}

To express coreference with the \textit{object} of the purposive clause (in the case of a transitive verb), the object participle is normally used instead (§\ref{sec:object.participles.complement}).  Coreference between the object of the complement clause and the subject of the matrix clause is possible only if the object has control over the action as in (\ref{ex:kAmto.mWjGi}); control of the subject of the purposive clause is not necessary in this case.

\begin{exe}
	\ex \label{ex:kAmto.mWjGi}
	\gll [kʰro kɤ-mto] mɯ́j-ɣi ma ɯ-kɤ-ndza ɣɤʑu. \\
	much \textsc{obj}:\textsc{pcp}-see \textsc{neg}:\textsc{sens}-come \textsc{lnk} \textsc{3sg}.\textsc{poss}-\textsc{obj}:\textsc{pcp}-eat exist:\textsc{sens} \\
	\glt `(In the years when there are a lot of things to eat in the forest), it$_i$ does not come to [places where it$_i$ can] be seen, because it$_i$ has things to eat. (23-pGAYaR)
\japhdoi{0003606\#S77}
\end{exe}

There is however at least one unexplained exception. The verb \japhug{nɯβlɤmtɕʰɤt}{ask to perform a task} (\ref{ex:tAnWBlAmtChata}) consistently occurs in \textit{subject} participle form in the purposive construction in the meaning `go/come to perform a task (at someone's invitation)', where coreference is thus between the subject of the motion verb and the object of the transitive verb. The \forme{kɤ-} participle is also acceptable with motion verbs, but this construction is not attested in texts. This problem deserves additional research.\footnote{Another puzzling feature of this construction is the fact that there is no obligatory possessive prefix on the participle, and that when present, it indexes the subject rather than the object. }

\begin{exe}
	\ex \label{ex:tAnWBlAmtChata}
	\gll aʑo βlama  tɤ-nɯβlɤmtɕʰat-a \\
	\textsc{1sg} lama \textsc{aor}-ask.to.perform.a.task-\textsc{1sg} \\
	\glt `I asked a lama to perform a [ceremony] for me.' (elicited)
\end{exe}

\begin{exe}
	\ex \label{ex:kWnWBlAmtChAt.Cea}
	\gll kɯ-nɯβlɤmtɕʰɤt ɕe-a ra \\
	\textsc{sbj}:\textsc{pcp}-ask.to.perform.a.task go:\textsc{fact}-\textsc{1sg} be.needed:\textsc{fact} \\
	\glt `(Their lama said:) `I have to go to perform [a ceremony at someone's invitation].' (160720 kandZislama)
	\japhdoi{0006147\#S5}
\end{exe}

When the verb in the subject purposive clause is transitive, the participle has a possessive prefix coreferent with the object as in the case of relative clauses (§\ref{sec:subject.participle.possessive}), and the subject can either take absolutive marking following the motion verb (which is morphologically intransitive), as in (\ref{ex:WkWCar.chACe}), or ergative marking following the verb of the purposive clause as in (\ref{ex:WkWCar.loCenW}). This difference in case marking can be analysed as reflecting clausal structure: in (\ref{ex:WkWCar.loCenW}), the subject \forme{tɤ-rɟit ra} `the children' belongs to the purposive clauses, whereas in (\ref{ex:WkWCar.chACe}), the subject \forme{tɤ-mu nɯ} lies outside of it.


\begin{exe}
	\ex \label{ex:WkWCar.loCenW}
	\gll ɯ-fso-soz tɕe, [tɤ-rɟit ra kɯ nɯ ɯ-kɯ-ɕar] jo-ɕe-nɯ ɲɯ-ŋu tɕe \\
	\textsc{3sg}.\textsc{poss}-tomorrow-morning \textsc{lnk} \textsc{indef}.\textsc{poss}-child \textsc{pl} \textsc{erg} \textsc{dem} \textsc{3sg}.\textsc{poss}-\textsc{sbj}:\textsc{pcp}-search \textsc{ifr}:\textsc{upstream}-go \textsc{sens}-be \textsc{lnk} \\
	\glt  `The next morning, the children went (there) to look for him.'  (2012 Norbzang)
	\japhdoi{0003768\#S281}
\end{exe}

The goal of the motion verb can, however, occur within the purposive clause, as in (\ref{ex:khapa.WkWnnAjo}), where the subject in ergative form \forme{ɯ-wa nɯ kɯ} `his father' is stranded from the transitive verb \forme{ɯ-kɯ-n-nɤjo} by the goal \forme{kʰapa tɕe} `downstairs'.

\begin{exe}
	\ex \label{ex:khapa.WkWnnAjo}
	\gll [ɯ-wa nɯ kɯ kʰapa tɕe ɯ-kɯ-n-nɤjo] pjɤ-ɣi.  \\
	\textsc{3sg}.\textsc{poss}-father \textsc{dem} \textsc{erg} downstairs \textsc{loc}    \textsc{3sg}.\textsc{poss}-\textsc{sbj}:\textsc{pcp}-\textsc{auto}-wait \textsc{ifr}:\textsc{down}-come \\
	\glt `His father came downstairs to wait for him.' (140506 loBzi)
\japhdoi{0003923\#S5}
\end{exe}


Motion verbs with purposive clauses have some semantic overlap with the corresponding associated motion prefixes (§\ref{sec:am.prefixes}); the functional difference between the two constructions is discussed in §\ref{sec:am.vs.mvc}.

Apart from the three motion verbs above, a few verbs expressing imminent aspect such as \japhug{aɣɯɣu}{be about to} also selects participial clauses (§\ref{sec:imminent.complements}), as shown by (\ref{ex:kWsi.torANgat2}).

\begin{exe}
	\ex \label{ex:kWsi.torANgat2}
	\gll kɯ-maqʰu tɕe, tɤ-tɕɯ nɯ [kɯ-si] to-rɤŋgat \\
	\textsc{sbj}:\textsc{pcp}-be.after \textsc{lnk} \textsc{indef}.\textsc{poss}-son \textsc{dem} \textsc{sbj}:\textsc{pcp}-die \textsc{ifr}-be.about.to \\
	\glt `The man was about to die.' (2002 rkongrgjal2)
\end{exe} 

Although there is potential ambiguity between purposive clauses and headless participial relative clauses (§\ref{sec:headless.relative}) used with motion verbs, ambiguous sentences are not common in the corpus. The presence of determiners such as the indefinite \forme{ci} in (\ref{ex:WkWmtshi.jAGe}) shows that the participial clause can only be a relative and is not interpretable as a purposive clause. However, \forme{ɯ-kɯ-mtsʰi jɤ-ɣe} without the determiner could indeed be parsed as `s/he came to lead it'.

\begin{exe}
	\ex \label{ex:WkWmtshi.jAGe}
	\gll [tsʰɤt ɯ-kɯ-mtsʰi] ci jɤ-ɣe tɕe \\
	goat \textsc{3sg}.\textsc{poss}-lead \textsc{indef} \textsc{aor}-come[II] \textsc{lnk} \\
	\glt `Someone came leading a goat.' (chen-pear)
\end{exe}

\subsubsection{Purposive clauses of manipulation verbs} \label{sec:participial.clause.essive}
Manipulation verbs such as \japhug{tsɯm}{take away} or \japhug{ɣɯt}{bring} (§\ref{sec:manipulation.verbs}) occur with non-finite purposive clauses in \forme{kɤ\trt}, as in (\ref{ex:kAntsGe.jotsWm}), rather than subject participles like the motion verbs (§\ref{sec:purposive.clause.motion.verbs}) despite obligatory subject coreference.

\begin{exe}
	\ex \label{ex:kAntsGe.jotsWm}
	\gll ɯ-mbro ɯ-ndʐi nɯra [kɤ-ntsɣe] jo-tsɯm \\
	\textsc{3sg}.\textsc{poss}-horse \textsc{3sg}.\textsc{poss}-skin \textsc{dem}:\textsc{pl} \textsc{obj}:\textsc{pcp}-sell \textsc{ifr}-take.away \\
	\glt `He took the horses' skins to [the market] to sell them.' (150814 kelaosi-zh)
\japhdoi{0006276\#S82}
\end{exe}

The non-finite verb form \forme{kɤ-ntsɣe} could in principle be either analyzed as an infinitive or as an object participle (§\ref{sec:infinitives.participles}). Unlike infinitival clauses, the \forme{kɤ-} clauses in this construction can be optionally followed by quantifiers as in (\ref{ex:kAntsGe.WkuxtCWxtCo}) and by the relator noun \japhug{ɯ-spa}{material} as in (\ref{ex:kAsat.Wspa.jowGtsWm}) and (\ref{ex:kAntsGe.Wspa.joGWt}). 

In addition, all of these examples present coreference between the object of the transitive verb of the main clause and that of the participial clause, reminiscent of the use of object participles in the purposive clauses of motion verbs to express coreference between the intransitive subject of the motion verb and the object of the transitive verb of the purposive clause (see example \ref{ex:kAmto.mWjGi} in §\ref{sec:purposive.clause.motion.verbs} above and §\ref{sec:object.participles.complement}).


\begin{exe}
	\ex \label{ex:kAntsGe.WkuxtCWxtCo}
	\gll  kutɕu rca [kɤ-ntsɣe] ɯ-kuxtɕɯ\redp{}xtɕo ju-ɣɯt-nɯ ɕti. \\
	\textsc{dem}.\textsc{prox}:\textsc{loc} \textsc{unexp}:\textsc{deg} \textsc{obj}:\textsc{pcp}-sell \textsc{3sg}.\textsc{poss}-\textsc{emph}\redp{}basket \textsc{ipfv}-bring-\textsc{pl} be.\textsc{aff}:\textsc{fact} \\
	\glt `[People] bring many basketfuls [of mushroom] to sell.' (23-mbrAZim)
	\japhdoi{0003604\#S103}
\end{exe}

\begin{exe}
	\ex \label{ex:kAsat.Wspa.jowGtsWm}
	\gll [kɤ-sat] ɯ-spa jó-wɣ-tsɯm ɲɯ-ŋu. \\
	\textsc{obj}:\textsc{pcp}-kill \textsc{3sg}.\textsc{poss}-material \textsc{ifr}-\textsc{inv}-take.away \textsc{sens}-be  \\
	\glt `He was taken away to be executed.' (tou dongxi de xiaohai-zh)
\end{exe}

\begin{exe}
	\ex \label{ex:kAntsGe.Wspa.joGWt}
	\gll tɯrme nɯ kɯ laχtɕʰa [kɤ-ntsɣe] (ɯ-spa) kɯ-dɯ\redp{}dɤn jo-ɣɯt. \\
	person \textsc{dem} \textsc{erg} thing \textsc{obj}:\textsc{pcp}-sell \textsc{3sg}.\textsc{poss}-material \textsc{sbj}:\textsc{pcp}-\textsc{emph}\redp{}be.many \textsc{ifr}-bring \\
	\glt `The man brought a lot of things to sell.' (elicited)
\end{exe}

These clauses are (at least historically) to be analyzed as participial clauses in essive function (§\ref{sec:essive.abs}): \forme{kɤ-ntsɣe (ɯ-spa)} and \forme{kɤ-sat (ɯ-spa)} in the examples above literally mean `(bring/take away) as something to be sold/as someone to be killed', hence their use in purposive function. This construction is not specific to manipulation verbs: in (\ref{ex:Ca.kAndza.Wspa}), the verb of the main clause \japhug{χsu}{raise}, `feed' is also found with the same type of participial clause.\footnote{In (\ref{ex:Ca.kAndza.Wspa}) however, the object of \forme{kɤ-ndza} is not coreferent with that of the main verb, unlike what is found with the manipulation verbs. }

\begin{exe}
	\ex \label{ex:Ca.kAndza.Wspa}
	\gll paʁ nɯra ʁo lɯski, [ɕa kɤ-ndza] ɯ-spa ku-χsu-nɯ pjɤ-ŋu ri \\
	pig \textsc{dem}:\textsc{pl} \textsc{advers} of.course meat \textsc{obj}:\textsc{pcp}-eat \textsc{3sg}.\textsc{poss}-material \textsc{ipfv}-feed-\textsc{pl} \textsc{ifr}.\textsc{ipfv}-be \textsc{lnk} \\
	\glt `As for the pigs, people of course raise them for their meat.' (150820 kAnWCkat)
\japhdoi{0006256\#S26}
\end{exe}



\subsubsection{Constructionalized participial relative clauses} \label{sec:constr.participial.clause}
The verbs of pretense select subject participial relative clauses as objects or semi-objects.\footnote{Unlike the cases discussed in §\ref{sec:relative.core.arg}, these verbs cannot take genuine complement clauses, and require participial relatives. } 

The status of the clauses with subject participles occurring with these verbs, though superficially similar to the purposive clause (§\ref{sec:purposive.clause.motion.verbs}), is, however, entirely distinct. These clauses are not specific constructions, but simply headless relative clauses in object or semi-object function. This is shown by the fact that the same verbs are also found with participial head-internal clauses as in (\ref{ex:pGatCW.kWGAwu.tunWCpWz}), where the three nouns \japhug{pɣɤtɕɯ}{bird}, \japhug{kʰɯna}{dog} and \japhug{lɯlu}{cat} are intransitive subjects of the participial clauses, and are not coreferent with the subject of their matrix verb \forme{tu-nɯɕpɯz}. This example can be literally translated as `it is able to imitate a singing bird, a barking dog and a meowing cat.' (see also  §\ref{sec:relative.pretence}).

\begin{exe}
	\ex \label{ex:pGatCW.kWGAwu.tunWCpWz}
	\gll ɯ-zda nɯra, [[\textbf{pɣɤtɕɯ} kɯ-ɣɤwu], [\textbf{kʰɯna} kɯ-ɤndzɯt], [\textbf{lɯlu} kɯ-ɣɤwu] kɯ-fse, nɯra tu-nɯɕpɯz] ɲɯ-spe \\
	\textsc{3sg}.\textsc{poss}-companion \textsc{dem}:\textsc{pl} bird \textsc{sbj}:\textsc{pcp}-cry dog \textsc{sbj}:\textsc{pcp}-bark cat \textsc{sbj}:\textsc{pcp}-cry \textsc{inf}:\textsc{stat}-be.like \textsc{dem}:\textsc{pl} \textsc{ipfv}-imitate \textsc{sens}-be.able[III] \\
	\glt `It is able to imitate other animals, sing like a bird, bark like a dog, meow like a cat or call like a fox.' (27-kikakCi)
	\japhdoi{0003700\#S132}
\end{exe}


\subsubsection{Participial complements} \label{sec:participial.complements.negative}
In addition to the constructions studied above, subject participles also occur in a handful of syntactic contexts where an infinitival complement is normally expected; I call these clauses `participial complements.'

The velar infinitive + existential verb construction expressing impossibility (§\ref{sec:inf.exist}) has a variant with Imperfective subject participles, as in (\ref{ex:tukWGi.YAGAme}). 

\begin{exe}
	\ex \label{ex:tukWGi.YAGAme}
	\gll   tu-kɯ-ɣi ɲɤ-ɣɤ-me qʰe,  \\
	\textsc{ipfv}:\textsc{up}-\textsc{sbj}:\textsc{pcp}-come \textsc{ifr}-\textsc{caus}-not.exist \textsc{lnk} \\
	\glt `She made it impossible for her to come out (again).' (2003-kWBRa)
\end{exe}

In addition, when a com\-ple\-ment-taking verb is itself in the subject participle form, it is possible for the complement either to be in the expected form (infinitive or finite), or to be in the subject participle form itself. For instance, in example (\ref{ex:ndZikWsAndu}) the subject participle \forme{kɯ-cʰa} `the one who can' takes a complement whose verb is a subject participle with a possessive prefix coreferent with the object (\forme{ndʑi-kɯ-sɤndu}), instead of the expected \forme{kɤ-} infinitive (or finite clause).

\begin{exe}
	\ex \label{ex:ndZikWsAndu}
	\gll  [[rŋɯl kɯ ndʑi-kɯ-sɤndu] kɯ-cʰa] kɯ-fse pɯ\redp{}pɯ-tu nɤ  \\
	silver \textsc{erg} \textsc{3du}-\textsc{sbj}:\textsc{pcp}-exchange \textsc{sbj}:\textsc{pcp}-can \textsc{sbj}:\textsc{pcp}-be.like 
	\textsc{cond}\redp{}\textsc{pst}.\textsc{ipfv}-exist if \\
	\glt `If there was someone who could redeem [the life of the two brothers] with money, ...' (140507 jinniao-zh)
	\japhdoi{0003931\#S327}
\end{exe}

\subsection{Action nominals}   \label{sec:complementation.strategy.action.nominals}
Some light verbs, in particular  \japhug{βzu}{make} (§\ref{sec:tr.light.verbs}) are combined with action nominals (§\ref{sec:action.nominals}) in highly grammaticalized constructions.

\subsubsection{Action nominals} \label{sec:action.nominal.Bzu}
The \forme{tɯ-} action nominals (§\ref{sec:tW.action.nominal}) can be combined with the verb \japhug{βzu}{make} to express habitual actions, especially actions taking a considerable amount of time. For instance \forme{tɯ-taʁ cʰɯ-βze} `she was weaving' in (\ref{ex:tWtAR.chWBze}) and \forme{tɯ-ɕkʰo pjɯ́-wɣ-nɯ-βzu} `(when) we dry (grains in the field)' refer to actions taking place every day (and taking up most of the day) during a certain time period.  

The main verb \forme{βzu} takes over the orientation selected by the verb in action nominal form (§\ref{sec:orientation.raising}): \textsc{downstream} in (\ref{ex:tWtAR.chWBze}) (§\ref{sec:orientation.loom}) and \textsc{downwards} in (\ref{ex:tWCkho.chWBze}).

\begin{exe}
	\ex \label{ex:tWtAR.chWBze}
	\gll tɤ-tɕɯ nɯ lu-rɤ-ji,  tɕe tɕʰeme nɯ kɯ li tɯ-taʁ cʰɯ-βze tɕe, mɯntoʁ ra tu-tʂɯβ qʰe ku-nɯ-rɤʑi-ndʑi pjɤ-ŋu.  \\
	\textsc{indef}.\textsc{poss}-boy \textsc{dem} \textsc{ipfv}-\textsc{apass}-plant \textsc{lnk} girl \textsc{dem} \textsc{erg} again \textsc{nmlz}:\textsc{action}-weave \textsc{ipfv}:\textsc{downstream}-make[III] \textsc{lnk} flower \textsc{pl} \textsc{ipfv}-sew \textsc{lnk} \textsc{ipfv}-\textsc{auto}-stay-\textsc{du} \textsc{ipfv}.\textsc{ifr}-be \\
	\glt `The boy was working in the fields, the girl was weaving and doing embroidery, they were living like that.' (150828 donglang)
\japhdoi{0006312\#S133}
\end{exe}


\begin{exe}
	\ex \label{ex:tWCkho.chWBze}
	\gll tɕendɤre tɯ-ɕkʰo pjɯ́-wɣ-nɯ-βzu qʰe, nɯ kɤ-ɣndʑɯr ɯ-spa nɯ pjɯ́-wɣ-ɕkʰo tɕe nɯ-rom kóʁmɯz cʰɯ́-wɣ-ndʑɯr ra tɕe, ɯnɯnɯra ɣɯ-tu-mɯrki tu-ndze ŋu. \\
	\textsc{lnk} \textsc{nmlz}:\textsc{action}-dry.in.the.sun \textsc{ipfv}:\textsc{down}-\textsc{inv}-\textsc{auto}-make \textsc{lnk} \textsc{dem} \textsc{obj}:\textsc{pcp}-grind \textsc{3sg}.\textsc{poss}-material \textsc{dem} \textsc{ipfv}:\textsc{down}-\textsc{inv}-dry.in.the.sun \textsc{lnk} \textsc{aor}-be.dry only.after \textsc{ipfv}-\textsc{inv}-grind be.needed:\textsc{fact} \textsc{lnk} \textsc{dem}:\textsc{pl} \textsc{cisl}-\textsc{ipfv}-steal[III] \textsc{ipfv}-eat[III] be:\textsc{fact} \\
	\glt `When we dry things in the sun, when we dry the grains before grinding (one grinds them only after they have dried), it comes, steals them and eats them.' (22-CAGpGa)
\japhdoi{0003586\#S62}
\end{exe}

This construction is found with both transitive (\ref{ex:tWCkho.chWBze}) and intransitive verbs (\ref{ex:tWrJaʁ.pjWBzunW}). In the former case, the object is not overt. Although \forme{βzu} remains transitive, this construction shares with the antipassive derivations (§\ref{sec:antipassive}) the function of demoting the object (§\ref{sec:antipassive.vs.light.verbs}).

\begin{exe}
	\ex \label{ex:tWrJaʁ.pjWBzunW}
	\gll rɤɣo ra cʰɯ-βzu-nɯ, tɯ-rɟaʁ ra pjɯ-βzu-nɯ pjɤ-ŋgrɤl ɲɯ-ŋu \\
	song \textsc{pl} \textsc{ipfv}-make-\textsc{pl} \textsc{nmlz}:\textsc{action}-dance \textsc{pl} \textsc{ipfv}-make-\textsc{pl} \textsc{indef}.\textsc{ipfv}-be.usually.the.case \textsc{sens}-be \\
	\glt `(Every year, when the festival took place), [people] would sing and dance.' (150906 toutao-zh)
	\japhdoi{0006326\#S13}
\end{exe}

Action nominals either refer to the action itself or an object affected by the action (§\ref{sec:tW.action.nominal}) . While the collocation with \japhug{βzu}{make} probably derives from the first meaning of the action nominals, in some examples we have to take into consideration that the second meaning of the action nominal may intervene. For instance, in (\ref{ex:tWCkho.chWBze}), \forme{tɯɕkʰo+βzu} can be understood as `to do the action of drying in the sun', but also as `to do the action related to grains that are dried in the sun', since the grains in question are directly referred to in the next clause as \forme{kɤ-ɣndʑɯr ɯ-spa} `(grains) to be ground'.

Action nominals can also occur with \japhug{lɤt}{release}, though this construction is considerably less productive. This collocation can be used to indicate a sudden semelfactive action as in (\ref{ex:tWmWrRWz.chAlAt}), and the object can be overt.

\begin{exe}
	\ex \label{ex:tWmWrRWz.chAlAt}
	\gll    tɕe aʑo a-mtʰɯm ta-nɯ-tsɯm qʰe, a-jaʁ ra tɯ-mɯrʁɯz cʰɤ-lɤt qʰe, tɤ-se pa-tɕɤt.  \\
	\textsc{lnk} \textsc{1sg} \textsc{1sg}.\textsc{poss}-meat \textsc{aor}:\textsc{up}:3\flobv{}-\textsc{vert}-take.away  \textsc{lnk} \textsc{1sg}.\textsc{poss}-hand \textsc{pl} \textsc{nmlz}:\textsc{action}-scratch \textsc{ifr}:\textsc{downstream}-release \textsc{lnk} \textsc{indef}.\textsc{poss}-blood \textsc{aor}:3\flobv{}-take.out \\
	\glt  `The kite took away the meat (that was in my hand), it scratched my hand, and made it bleed.' (150909 qandZGi)
\japhdoi{0006358\#S9}
\end{exe}

Apart from \forme{βzu} and \forme{lɤt}, the transitive verb \japhug{kʰɤt}{do repeatedly}, `do for a long time' also selects action nominals, but with ergative case marking (§\ref{sec:oblique.kW}). The modal verb \japhug{ra}{be needed} (§\ref{sec:ra.khW.jAG.verb}) can take degree nominals as subjects (§\ref{sec:degree.nominal.complement}) instead of finite or infinitival complements.


\subsubsection{Simultaneous action nominals} \label{sec:simult.action.nominal.Bzu}
The simultaneous action nominals, prefixed in \forme{tɯ-tɯ-} (§\ref{sec:simultaneous.action.nominal}), are also used in collocation with \japhug{βzu}{make}.   As in the previous construction, there is raising of the orientation preverb onto the main verb \forme{βzu} (§\ref{sec:orientation.raising}). For instance,  in (\ref{ex:tWtWnWmdar.pjABzundZi}) the \textsc{downwards} orientation preverb \forme{pjɤ-} on \forme{βzu} from the nominalized verb \japhug{nɯmdar}{jump}. 

With an intransitive verb, the simultaneous constructions occur with a dual or plural subject, and means that several individuals referred do an action together at the same moment, as \forme{tɯ-tɯ-nɯmdar} `jumping together' in (\ref{ex:tWtWnWmdar.pjABzundZi}). The auxiliary \forme{βzu} can even take the indefinite orientation prefixes \forme{jɤ-} \forme{ja-} as in (\ref{ex:tWtWCe.jaBzundZi}) when occurring with a motion verb.

\begin{exe}
	\ex \label{ex:tWtWnWmdar.pjABzundZi}
	\gll  mtsʰu ɯ-ŋgɯ tɯ-tɯ-nɯmdar pjɤ-βzu-ndʑi  \\
	lake \textsc{3sg}.\textsc{poss}-inside \textsc{simult}-\textsc{nmlz}:\textsc{action}-jump \textsc{ifr}:\textsc{down}-make-\textsc{du}  \\
	\glt `They (the two of them) had jumped into the lake together.' (nyima wodzer 2003)
\end{exe}

When used with transitive verbs, this construction is only found with dual or plural objects, and implies that several entities were subjected to the action together at the same time, as \forme{tɯ-tɯ-tʂaβ} `causing to roll down together' in (\ref{ex:tWtWtsxaB.Zo.pjABzu}), or that several object end up being tied together as result of the action as with \forme{tɯ-tɯ-tʂɯβ} `sewing together' in (\ref{ex:tWtWtsxWB.kABzuta}). 


\begin{exe}
	\ex \label{ex:tWtWtsxaB.Zo.pjABzu}
	\gll  pri cʰo jlɤkrɯ nɯra tɯ-tɯ-tʂaβ ʑo pjɤ-βzu  \\
	bear \textsc{comit} \textsc{dem}:\textsc{pl}  basket  \textsc{simult}-\textsc{nmlz}:\textsc{action}-cause.to.roll.down \textsc{emph} \textsc{ifr}:\textsc{down}-make \\
	\glt `He made the basket with the bear [in it] roll down (together).' (2011-13-qala) 
\end{exe}

The comitative (§\ref{sec:comitative}) can be used as in (\ref{ex:tWtWtsxaB.Zo.pjABzu}) and example  (\ref{ex:tWtWrqoR.koBzu}) in  (§\ref{sec:simultaneous.action.nominal}) to link two nouns referring to the patients that undergo the action together.

\begin{exe}
	\ex \label{ex:tWtWtsxWB.kABzuta}
	\gll   tɯ-ŋga tɯ-tɯ-tʂɯβ kɤ-βzu-t-a \\
	\textsc{indef}.\textsc{poss}-clothes \textsc{simult}-\textsc{nmlz}:\textsc{action}-sew \textsc{aor}-make:\textsc{pst}:\textsc{tr}-\textsc{1sg} \\
	\glt `I stiched the clothes together.' (elicited)
\end{exe}

In the case of transitive verbs, the verb \japhug{βzu}{make} takes the indexation of both subject and object, as in (\ref{ex:tWtWtsxWB.tAtaBzunW}) where it takes the portmanteau \forme{ta-} prefix, showing that the simultenous action nominal does \textit{not} have object function.

\begin{exe}
	\ex \label{ex:tWtWtsxWB.tAtaBzunW}
	\gll nɯʑora tɯ-tɯ-qur tɤ-ta-βzu-nɯ \\
	\textsc{2pl} \textsc{simult}-\textsc{nmlz}:\textsc{action}-help \textsc{aor}-1\fl{}2-make-\textsc{pl} \\
	\glt `I helped you all at the same time.' (elicited)
\end{exe}  


\subsubsection{Compound action nominals} \label{sec:compound.action.nominal.Bzu}
The verb \japhug{βzu}{make} is also combined with noun-verb compound action nominals (§\ref{sec:action.nominal.compounds}) as semi-object. Two such compound nominals have been identified. With \japhug{kʰramba}{lie} as first element, this construction means `pretend to do $X$' (where $X$ stands for the second element of the compound) as in (\ref{ex:khrambaqur}) and (\ref{ex:khrambanAre}). 

\begin{exe}
	\ex \label{ex:khrambaqur}
	\gll [kʰramba-qur] ma-tɤ-kɯ-βzu-a \\
	lie-help \textsc{neg}-\textsc{imp}-2\fl{}1-make-\textsc{1sg} \\
	\glt `Do not pretend to help me!' (elicited)
\end{exe}

\begin{exe}
	\ex \label{ex:khrambanAre}
	\gll [kʰramba-nɤre] ɲɤ-βzu \\
	lie-laugh \textsc{ifr}-make \\
	\glt `He pretended to laugh.' (elicited)
\end{exe}

With the lexicalized participle \japhug{kɯzɣa}{a long time} as first element of the compound, it means `do $X$ for a long time' as in (\ref{ex:kWzGArAlaj}) (§\ref{sec:action.nominal.compounds}). 

\begin{exe}
	\ex \label{ex:kWzGArAlaj}
	\gll qajɣi pjɯ́-wɣ-rɤpɣi tɕe tɕe, li [kɯzɣɤ-rɤlaj] ʑo pjɯ́-wɣ-βzu. \\
	bread \textsc{inf}-\textsc{inv}-mix.flour.and.water \textsc{lnk} \textsc{lnk} again long.time-knead \textsc{emph} \textsc{ipfv}-\textsc{inv}-make \\
	\glt `When one mixes flour and water to make bread, one has to knead [the dough] for a long time.' (160706 thotsi)
\japhdoi{0006133\#S34}
\end{exe}

When the verbal component of the compound action nominal is transitive, both its subject and objects are indexed on the verb \japhug{βzu}{make}, as shown by the 2\fl{}1 configuration in (\ref{ex:khrambaqur}). In addition, the orientation preverb on \forme{βzu} reflects the lexicalized orientation of the verb in the compound (§\ref{sec:orientation.raising}), for instance \textsc{upwards} in (\ref{ex:khrambaqur}),  \textsc{westwards} in (\ref{ex:khrambanAre}) and \textsc{downwards} in (\ref{ex:kWzGArAlaj}).


The verb \japhug{lɤt}{release} (§\ref{sec:lAt.lv}) is also productively used with action nominal compounds (§\ref{sec:action.nominal.compounds}) comprising a verb root such as \japhug{rpu}{bump into} or \japhug{tɕʰɯ}{gore, stab} as second element.


\subsection{Coordination} \label{sec:coordination.comp.str}
Some verbs take post-verbal coordinated clauses instead of complement clauses. A clausal linker is inserted before that clause, showing its non-subordinated status. There are several constructions of this type.

First, some attitudinal verbs such as \japhug{ʁnɯ}{suspect}, \japhug{nɯsɯmʁɲiz}{hesitate}, \japhug{nɯʁlɯmbɯɣ}{guess, estimate} or \japhug{nɯʁjɯβtsʰɤt}{guess, estimate} occur in a coordinating construction strikingly similar to that described in Tshobdun  by \citet[487--488]{sun12complementation}: the attitudinal verb is followed by the affirmative copula \japhug{ɕti}{be} and a linker such as \forme{ri} or \forme{ma}, as in (\ref{ex:tunWRlWmbWGa}) and (\ref{ex:tuRnWa}).

\begin{exe}
	\ex \label{ex:tunWRlWmbWGa}
	\gll nɯ tu-nɯʁlɯmbɯɣ-a ɕti ri, ɯʑo kɯ kɤ-nɤma nɯ sɤpe \\
	\textsc{dem} \textsc{ipfv}-guess-\textsc{1sg} be.\textsc{aff}:\textsc{fact} \textsc{lnk} \textsc{3sg} \textsc{erg} \textsc{obj}:\textsc{pcp}-work \textsc{dem} do.well:\textsc{fact} \\
	\glt `I suppose that he will perform this task well.'
\end{exe}

\begin{exe}
	\ex \label{ex:tuRnWa}
	\gll tu-ʁnɯ-a ɕti ma ɯʑo kɯ ta-tɯt ŋu maʁ mɤ-xsi \\
	\textsc{ipfv}-suspect-\textsc{1sg} be.\textsc{aff}:\textsc{fact} \textsc{lnk} \textsc{3sg} \textsc{erg} \textsc{aor}:3\flobv{}-say be:\textsc{fact} not.be:\textsc{fact} \textsc{neg}-\textsc{genr}:know \\
	\glt `I am wondering whether what he said is true or not (suspecting that it is not true).' (elicited)
\end{exe}

Second, the verbs of perception  \japhug{rtoʁ}{look},  \japhug{ru}{look at} and \japhug{sɤŋo}{listen} occur (exclusively in the Imperfective, §\ref{sec:ipfv.perception}) in narratives in a coordinating construction with the linkers \forme{tɕe} or \forme{qʰe} as in (\ref{ex:kurtoR.tCe.pjAmpCAr}) and (\ref{ex:luru.tCe}). 

\begin{exe}
	\ex \label{ex:kurtoR.tCe.pjAmpCAr}
	\gll  ku-rtoʁ tɕe [tɕʰeme nɯ wuma ʑo pjɤ-mpɕɤr] \\
	\textsc{ipfv}-look \textsc{lnk} girl \textsc{dem} really \textsc{emph} \textsc{ifr}.\textsc{ipfv}-be.beautiful \\
	\glt `He saw that that the girl was very beautiful.' (150909 xiaocui-zh)
\japhdoi{0006386\#S42}
\end{exe}

\begin{exe}
	\ex \label{ex:luru.tCe}
	\gll tɕe lu-ru tɕe [ɯ-ŋgɯ nɯtɕu ɯʑo lu-ntɕʰɤr ɲɯ-ŋu] \\
	\textsc{lnk} \textsc{ipfv}:\textsc{upstream}-look.at  \textsc{lnk}  \textsc{3sg}.\textsc{poss}-inside \textsc{dem}:\textsc{loc} himself \textsc{ipfv}:\textsc{upstream}-be.reflected \textsc{sens}-be \\
	\glt `[The rooster] looked at [a window] and [saw] its own reflection in it.' (150819 kumpGa)
\japhdoi{0006388\#S62}
\end{exe}

The verb of speech \japhug{fɕɤt}{tell} is often used in the Imperfective with a zero object cataphorically referring to reported speech clause(s) following it. The reported speech clause(s) can serve as complement of another verb of speech as in (\ref{ex:pjWfCAtnW.tCe}) or be free-standing.

\begin{exe}
	\ex \label{ex:pjWfCAtnW.tCe}
	\gll kɯɕɯŋgɯ pjɯ-fɕɤt-nɯ tɕe, ``nɯtɕu tɕaχpa pjɤ-tu'' tu-ti-nɯ tɕe,  \\
	in.former.times \textsc{ipfv}-tell-\textsc{pl} \textsc{lnk} \textsc{dem}:\textsc{loc} bandit \textsc{ifr}.\textsc{ipfv}-exist \textsc{ipfv}-say-\textsc{pl} \textsc{lnk} \\
	\glt `In former times, people used to say that there were bandits there.' (17-lhazgron)
\end{exe}

These constructions are particularly common in texts translated from Chinese, where they might reflect calque from original language since in Chinese the perception verb precedes the complement, but it is also found in non-translated texts as in (\ref{ex:YAsANo.tCe.nWme.pjAtu}) (see also \forme{ku-rtoʁ} `he saw that...' in  \ref{ex:chain.pjANu}, §\ref{sec:ipfv.periphrastic.TAME}) and there is no doubt that it is a native construction.

\begin{exe}
	\ex \label{ex:YAsANo.tCe.nWme.pjAtu}
	\gll ɲɯ-sɤŋo tɕe, smɤt kɯβʁa nɯra, wuma ʑo nɯ-me, nɯ-me χsɯm pjɤ-tu tɕe \\
	\textsc{ipfv}-listen \textsc{lnk}  \textsc{topo} noble \textsc{dem}:\textsc{pl} really \textsc{emph} \textsc{3pl}.\textsc{poss}-daughter  \textsc{3pl}.\textsc{poss}-daughter  three \textsc{ifr}.\textsc{ipfv}-exist \textsc{lnk} \\
	\glt `He had heard that nobles of Smad had three daughters.' (2002 qaCpa) 
\end{exe}

Third, some verbs optionally use a coordinating construction instead of complementation, without constructionalized semantic difference, in particular similative verbs (§\ref{sec:similative.verb.complementation}).


\section{Complement-taking verbs} \label{sec:complement.taking.verbs}

An exhaustive survey of com\-ple\-ment-taking verbs in Japhug would require a much larger corpus than the one available and a complete monograph of at least the size of the present grammar. For this reason, only a selection of the most frequent com\-ple\-ment-taking verbs in the corpus is treated in this section. For each verb, the available complement types and complementation strategies are listed and exemplified.

Negative existential verbs can also take (subject) complements, but are discussed in §\ref{sec:negation.existential}. 

\subsection{Causative verbs}
Some verbs derived with the causative prefixes (§\ref{sec:sig.causative}, §\ref{sec:velar.causative}) take complements or occur in complementation strategies. Three types of constructions are attested, expressing periphrastic causative, simultaneous events and manner.

\subsubsection{Periphrastic causative constructions} \label{sec:sWpa.sABzu}
In addition to the sigmatic and velar causative derivations (§\ref{sec:sig.causative}, §\ref{sec:velar.causative}), a variety of periphrastic causative constructions are also attested. Four groups of verbs are used as causative auxiliaries, including both causative verbs and non-derived verbs.

First, \japhug{βzu}{make} can be combined with subject participles of adjectival stative verbs to express change of state or increase of degree, as in (\ref{ex:kWndWndWB.chWBzunW}) and (\ref{ex:matAtWGAwxti}).

\begin{exe}
\ex  \label{ex:kWndWndWB.chWBzunW}
\gll cʰɯ-ndɯl-nɯ tɕe [kɯ-ndɯ\redp{}ndɯβ] ʑo cʰɯ-βzu-nɯ \\
\textsc{ipfv}-grind-\textsc{pl} \textsc{lnk} \textsc{sbj}:\textsc{pcp}-\textsc{emph}\redp{}be.fine \textsc{emph} \textsc{ipfv}-make-\textsc{pl} \\
\glt `They grind [tobacco] and make it very smooth.' (30-CnAto)
\japhdoi{0003734\#S8}
\end{exe}

Example (\ref{ex:matAtWGAwxti2}) illustrates the fact that the velar causative \japhug{ɣɤwxti}{make bigger} is semantically similar to the use of \forme{βzu} with the participle \forme{kɯ-wxti}.

\begin{exe}
\ex  \label{ex:matAtWGAwxti2}
\gll  ɯ-pʰɯ ɲɯ-wxti tɕe, nɯra tʰamtɕɤt ma-tɤ-tɯ-ɣɤ-wxti, aʑo nɯ tu-nɯ-χti-a [nɯra stʰɯci kɯ-wxti] ɯ-pʰɯ ma-tɤ-tɯ-βze, [kɯ-tʂaŋ] ci tɤ-βze \\
   \textsc{3sg}.\textsc{poss}-price \textsc{sens}-be.big \textsc{lnk} \textsc{dem}:\textsc{pl} all \textsc{neg}-\textsc{imp}-2-\textsc{caus}-be.big \textsc{1sg} \textsc{dem} \textsc{ipfv}-\textsc{auto}-buy[III]-\textsc{1sg}   \textsc{dem}:\textsc{pl} as.much \textsc{sbj}:\textsc{pcp}-be.big \textsc{3sg}.\textsc{poss}-price \textsc{neg}-\textsc{imp}-2-make[III] \textsc{sbj}:\textsc{pcp}-be.fair a.little \textsc{imp}-make[III] \\
\glt `It is too expensive, don't make it that expensive, I will buy it, don't make its price that expensive, offer it for a fair price.' (Bargaining 12,  12)
\end{exe}

This construction is not restricted to stative verbs. It is attested with transitive verbs as in  (\ref{ex:pjWkWlAt.toBzu}) to express indirect causation.

\begin{exe}
\ex  \label{ex:pjWkWlAt.toBzu}
\gll  cɯβjiz kɯ-fse cʰɤ-ta tɕe, ɯ-kɯr ɯ-ŋgɯ tɕe, [tɯ-ci pjɯ-kɯ-lɤt] to-βzu \\
flat.stone \textsc{sbj}:\textsc{pcp}-be.like \textsc{ifr}:\textsc{downstream}-put \textsc{lnk} \textsc{3sg}.\textsc{poss}-mouth \textsc{3sg}.\textsc{poss}-in \textsc{loc} \textsc{indef}.\textsc{poss}-water \textsc{ipfv}-\textsc{sbj}:\textsc{pcp}-release \textsc{ifr}-make \\
\glt `He positioned [the leaf of a rhododendron] like a flat stone [next to his younger brother's$_i$ mouth] in such a way that water could flow into his$_i$ mouth.' (2011-05-nyima)
\end{exe}

With dynamic verbs however, the preferred construction is to use an impersonal modal verb such as \japhug{kʰɯ}{be possible} or  \japhug{ra}{be needed} (§\ref{sec:ra.khW.jAG.verb}) in participial form taking a complement verb, as in (\ref{ex:mAkWkhW.tuBzunW}).


\begin{exe}
\ex  \label{ex:mAkWkhW.tuBzunW}
\gll la-rɤɕi-nɯ tɕe, [[lu-nɯ-ɬoʁ] mɤ-kɯ-kʰɯ] tu-βzu-nɯ  \\
\textsc{aor}:3\flobv{}-pull-\textsc{pl} \textsc{lnk} \textsc{ipfv}:\textsc{upstream}-\textsc{auto}-come.out \textsc{neg}-\textsc{sbj}:\textsc{pcp}-be.possible \textsc{ipfv}-make-\textsc{pl} \\
\glt `They pull [on the thread] (to close the opening) and prevent it from coming out.' (30-CnAto)
\japhdoi{0003734\#S38}
\end{exe}

Second, the sigmatic causative forms \forme{sɯ-βzu}, \forme{sɯ-pa}, \forme{sɯ-ɤβzu} and \forme{sɯ-ɤpa} derived from the verbs \japhug{βzu}{make}, \japhug{pa}{do}, \japhug{aβzu}{become, grow} and \japhug{apa}{become} (on the latter two verbs see §\ref{sec:passive.lexicalized}) are also commonly used as causative auxiliaries.

Like the base verb \forme{βzu}, these causative verbs are also most often used with a modal impersonal auxiliary in subject participle form as in (\ref{ex:CWkABde.mAkWra}), (\ref{ex:mAkWkhW.YWtWsApe}) and (\ref{ex:kWra.tAwGsWBzuanW}).

\begin{exe}
\ex  \label{ex:CWkABde.mAkWra}
\gll [ɕɯ-kɤ-βde mɤ-kɯ-ra] nɯ ndʑiʑo kɯ nɯ-tɯ-sɯ-ɤβzu-ndʑi ŋu  \\
\textsc{tral}-\textsc{inf}-throw  \textsc{neg}-\textsc{sbj}:\textsc{pcp}-be.needed \textsc{dem} \textsc{2du} \textsc{erg} \textsc{aor}-2-\textsc{caus}-become-\textsc{du} be:\textsc{fact} \\
\glt `Thanks to both of you, there is no need to throw [people] (in the lake) anymore.' (2011-05-nyima)
\end{exe}


 \begin{exe}
\ex  \label{ex:mAkWkhW.YWtWsApe}
\gll a-tɯ-ci ɲɯ-tɯ-s-qarndɯm tɕe [[aʑo tɯ-ci kɯ-ɤmgri kɤ-tsʰi] mɤ-kɯ-kʰɯ] ɲɯ-tɯ-sɯ-ɤpe ɲɯ-ŋu  \\
\textsc{1sg}.\textsc{poss}-\textsc{indef}.\textsc{poss}-water \textsc{ipfv}-\textsc{2}-\textsc{caus}-be.muddy \textsc{lnk} \textsc{1sg} \textsc{indef}.\textsc{poss}-water \textsc{sbj}:\textsc{pcp}-be.clear \textsc{inf}-drink \textsc{neg}-\textsc{sbj}:\textsc{pcp}-be.possible \textsc{ipfv}-2-\textsc{caus}-become[III] \textsc{sens}-be \\
\glt `You have spoiled my water, you have made it so I am unable to drink clear water.' (lang he yang-zh)
\end{exe}

Other com\-ple\-ment-taking stative verbs such as \japhug{sna}{be good, be worthy}, also occur in this causative construction, as in (§\ref{ex:mAkWsna.YWsABze}).

 \begin{exe}
\ex  \label{ex:mAkWsna.YWsABze}
\gll tɕe tɤ-mtʰɯm tʰamtɕɤt [kɤ-ndza mɤ-kɯ-sna] ɲɯ-sɯ-ɤβze ɲɯ-cʰa. \\
\textsc{lnk} \textsc{indef}.\textsc{poss}-meat all \textsc{inf}-eat \textsc{neg}-\textsc{sbj}:\textsc{pcp}-be.good \textsc{ipfv}-\textsc{caus}-become[III] \textsc{sens}-can \\
\glt `[Maggots] can make all the meat improper for consumption (unfit to eat).' (25-akWzgumba)
\japhdoi{0003632\#S100}
\end{exe}

The causee can be indexed as an object on the causative verb, as in (\ref{ex:pjWkWlAt.Zo.YWtasABzu}) or (\ref{ex:kWra.tAwGsWBzuanW}) (which have \textsc{2sg} and \textsc{1sg} causees, respectively), but this indexation is only optional, as shown by examples such as (\ref{ex:mAkWkhW.YWtWsApe}) which rather select a \textsc{3sg} object despite the \textsc{1sg} causee.

 \begin{exe}
\ex  \label{ex:pjWkWlAt.Zo.YWtasABzu}
\gll praʁ ɯ-pa nɯtɕu (...) ku-ta-z-rɤʑi tɕe, [tɯ-ci nɯnɯ sɲikuku ʑo tɕetu (...) nɤ-taʁ nɯtɕu pjɯ-kɯ-lɤt] ʑo tu-ta-sɯ-ɤβzu ŋu tɕe \\
cliff \textsc{3sg}.\textsc{poss}-down \textsc{dem}:\textsc{loc} { } \textsc{ipfv}-1\fl{}2-\textsc{caus}-stay \textsc{lnk} \textsc{indef}.\textsc{poss}-water \textsc{dem} every.day \textsc{emph} up.there { } \textsc{3sg}.\textsc{poss}-on \textsc{dem}:\textsc{loc} \textsc{ipfv}:\textsc{down}-\textsc{sbj}:\textsc{pcp}-release \textsc{emph} \textsc{ipfv}-1\fl{}2-\textsc{caus}-become be:\textsc{fact} \textsc{lnk} \\
\glt `(I will not kill you), but I will keep you under the cliffs, so that the water flows down onto you.' (150901 changfamei-zh)
\japhdoi{0006352\#S153}
\end{exe}

 \begin{exe}
\ex  \label{ex:kWra.tAwGsWBzuanW}
\gll [mɯ-tu-kɤ-nɤtɯti kɯ-ra] tɤ́-wɣ-sɯ-βzu-a-nɯ ndʐa ɕti ma  \\
\textsc{neg}-\textsc{ipfv}-\textsc{inf}-\textsc{distr}:say \textsc{sbj}:\textsc{pcp}-be.needed \textsc{aor}-\textsc{inv}-\textsc{caus}-make-\textsc{1sg}-\textsc{pl} reason be.\textsc{aff}:\textsc{fact} \textsc{lnk} \\ 
\glt `It is because they made me [swear] not to tell anyone about it.'  (tWxtsa 2003)
\end{exe}

Another verb that can be used with the modal verbs \forme{ra} and \forme{kʰɯ} in a periphrastic causative construction is \japhug{tɕɤt}{take out}. It is rarer and only compatible with negative participial forms to express the meaning `prevent from $X$'.

\begin{exe}
\ex  \label{ex:mAkWra.YAtWtCAt}
\gll  [kɤ-ɤlɯlɤt mɤ-kɯ-ra] ɲɤ-tɯ-tɕɤt \\
\textsc{inf}-fight \textsc{neg}-\textsc{sbj}:\textsc{pcp}-be.needed \textsc{ifr}-2-take.out \\
\glt `You stopped them from fighting.' (elicited)
\end{exe}
 
 Fourth, the velar causatives  \japhug{ɣɤra}{cause to have to} and  \japhug{ɣɤkʰɯ}{make it possible to} of the modal auxiliaries involved in the constructions above can take infinitive or finite complement clauses (§\ref{sec:velar.caus.modal}) to express indirect causation as in (\ref{ex:kAsci.mWnWtWGAkhWt}).

\begin{exe}
\ex \label{ex:kAsci.mWnWtWGAkhWt}
\gll   [kɤ-sci] mɯ-nɯ-tɯ-ɣɤ-kʰɯ-t  \\
\textsc{inf}-be.born \textsc{neg}-\textsc{ipfv}-2-\textsc{caus}-be.possible-\textsc{pst}:\textsc{tr} \\
\glt `You made it impossible for me to be born.' (Gesar)
\end{exe}  

\subsubsection{Periphrastic tropative} \label{sec:tropative.sWpa}
The causative verb \japhug{sɯpa}{cause to do} (from \japhug{pa}{do}, on which see
§\ref{sec:pa.complements}, §\ref{sec:pa.lv}), among other functions (for instance §\ref{sec:bare.dental.inf.sWpa}), has the meaning `consider $X$ to be $Y$'.

The parameter $Y$ can be a stative verb in participial form, such as \forme{kɯ-βdi} in (\ref{ex:kWBdi.tusWpanW}).

\begin{exe}
\ex \label{ex:kWBdi.tusWpanW}
\gll tɤ-ɕpʰɤt nɯ a-pɯ-ɤndʑɤmstu tɕe, nɯ [kɯ-βdi] tu-sɯ-pa-nɯ ŋu \\
\textsc{indef}.\textsc{poss}-patch \textsc{dem} \textsc{irr}-\textsc{ipfv}-be.level \textsc{lnk} \textsc{dem} \textsc{sbj}:\textsc{pcp}-be.well \textsc{ipfv}-\textsc{caus}-do-\textsc{pl} be:\textsc{fact} \\
\glt `If the patch is level (if it is neatly sewn), people consider it [to have been sewn] well.' (12-kAtsxWB-29)
\japhdoi{0003486\#S26}
\end{exe}


The object $X$ can receive ergative case (§\ref{sec:comparee.kW}) to express a comparative meaning `consider $X$ to be more $Y$ (§\ref{sec:sAz.kW}), as in (\ref{ex:kWmWm.tusWpanW}). Ergative marking in this construction can be ambiguous, since the subject (experiencer) of \forme{sɯpa} also receives ergative case.

\begin{exe}
\ex \label{ex:kWmWm.tusWpanW}
\gll nɯ ɯ-spa ra ɲɯ-naχtɕɯɣ ɕti, tɕe tʰoŋraʁ nɯ kɯ [kɯ-mɯm] tu-sɯ-pa-nɯ ŋu \\
\textsc{dem} \textsc{3sg}.\textsc{poss}-material \textsc{pl} \textsc{sens}-be.the.same be.\textsc{aff}:\textsc{fact} \textsc{lnk} bucket.alcohol \textsc{dem} \textsc{erg} \textsc{sbj}:\textsc{pcp}-be.tasty \textsc{ipfv}-\textsc{caus}-do-\textsc{pl} be:\textsc{fact} \\
\glt `Although its ingredients are the same (as those used to make other types of alcohol), [some people] consider alcohol in a bucket to be tastier.' (30-thoNraR)
\japhdoi{0003762\#S16}
\end{exe}

The combination of \forme{sɯ-pa} with participles is semantically similar to the tropative \forme{nɤ-} derivation (§\ref{sec:tropative}). For instance, the second clause in (\ref{ex:kWmWm.tusWpanW}) can be glossed as (\ref{ex:YWnAmWmnW}) with the tropative verb \japhug{nɤmɯm}{consider to be tasty} (in this example, the ergative on \japhug{tʰoŋraʁ}{alcohol in bucket} is also a comparee marker, and does not mark transitive subject).

\begin{exe}
\ex \label{ex:YWnAmWmnW}
\gll tʰoŋraʁ nɯ kɯ ɲɯ-nɤ-mɯm-nɯ \\
bucket.alcohol \textsc{dem} \textsc{erg} \textsc{sens}-\textsc{trop}-be.tasty-\textsc{pl} \\
\glt `They consider alcohol in bucket to be tastier.' (elicited)
\end{exe}


Unlike the \forme{nɤ-} tropative derivation,\footnote{The \forme{nɯ-} denominal prefix can be sporadically used with a tropative meaning (§\ref{sec:denom.tr.nW}), for instance  \japhug{nɯʁgra}{treat as an enemy} from \japhug{ʁgra}{enemy}, but it synchronically different from the tropative \forme{nɤ\trt}, though these two prefixes are historically related (§\ref{sec:sW.caus.history}). } the \forme{sɯ-pa} periphrastic tropative can be used with nouns as parameters, instead of stative verbs. As shown in (\ref{ex:NgumdZWG.tutasWpa}), the direct object is the person/entity that the subject considers to have the property described by the parameter, while the parameter $Y$ (here the noun \japhug{ŋgumdʑɯɣ}{leader}) is a semi-object, not indexed on the verb.

\begin{exe}
\ex \label{ex:NgumdZWG.tutasWpa}
\gll nɤʑo ŋgumdʑɯɣ tu-ta-sɯ-pa ŋu \\
\textsc{2sg} leader \textsc{ipfv}-1\fl{}2-\textsc{caus}-do be:\textsc{fact} \\
\glt `I consider you to be [my] leader.' (elicited)
\end{exe}

The participle clauses in the periphrastic tropative construction (such as in \ref{ex:kWBdi.tusWpanW} and \ref{ex:kWmWm.tusWpanW} above) can be analyzed as headless participial relative clauses (§\ref{sec:constr.participial.clause}). The literal meaning of this construction is thus `consider $X$ to be something that is $Y$.' 


\subsubsection{Simultaneity} \label{sec:bare.dental.inf.sWpa}
In addition to its periphrastic causative and tropative functions discussed in the previous section, the causative verb \japhug{sɯpa}{cause to do} also occurs in a construction expressing simultaneity between two actions. 

This construction requires two complements, either two bare infinitives as in (\ref{ex:Wti.cho.WnAma}) or a combination of a bare infinitive with a dental infinitive as in (\ref{ex:Wti.cho.WtWCe}), connected by the comitative \forme{cʰo}.

\begin{exe} 
\ex \label{ex:Wti.cho.WnAma}
\gll tɕe ɯ-ti cʰo ɯ-nɤma ra ci ʑo to-sɯ-pa \\
\textsc{lnk} \textsc{3sg}.\textsc{poss}-\textsc{bare}.\textsc{inf}:say \textsc{comit} \textsc{3sg}.\textsc{poss}-\textsc{bare}.\textsc{inf}:work \textsc{pl} one \textsc{emph} \textsc{ifr}-\textsc{caus}-do \\
\glt `As he said it, he did it at the same time.' (150826 liyu tiao longmen-zh)
\japhdoi{0006266\#S40}
\end{exe} 

\begin{exe} 
\ex \label{ex:Wti.cho.WtWCe}  
\gll  nɯɕimɯma ɯ-ti cʰondɤre ɯ-tɯ-ɕe ci to-sɯ-pa tɕe \\
immediately \textsc{3sg}.\textsc{poss}-\textsc{bare}.\textsc{inf}:say \textsc{comit} \textsc{3sg}.\textsc{poss}-\textsc{inf}:\textsc{II}-go  one \textsc{ifr}-\textsc{caus}-do \textsc{lnk} \\
\glt `As he said this, he immediately went [there].' (150830 baihe jiemei-zh)
\japhdoi{0006368\#S173}
\end{exe} 

Unlike other syntactic contexts (§\ref{sec:bare.dental.inf}) where dental infinitives does not take possessive prefixes, in this construction they obligatorily take a possessive prefix coreferent with the intransitive subject, as shown by the forms \forme{ɯ-tɯ-ɕe} vs. \forme{a-tɯ-ɕe} in (\ref{ex:Wti.cho.WtWCe}) and (\ref{ex:Wti.cho.atWCe}), respectively.

\begin{exe} 
\ex \label{ex:Wti.cho.atWCe}
\gll   ɯ-ti cʰo a-tɯ-ɕe ci tɤ-sɯ-pa-t-a  \\
\textsc{3sg}.\textsc{poss}-\textsc{bare}.\textsc{inf}:say \textsc{comit} \textsc{1sg}.\textsc{poss}-\textsc{inf}:\textsc{II}-go  one  \textsc{aor}-\textsc{caus}-do-\textsc{pst}:\textsc{tr}-\textsc{1sg} \textsc{lnk} \\
\glt `I said it and went there [at the same time].' (elicited based on \ref{ex:Wti.cho.WtWCe})
\end{exe} 

\subsubsection{Manner} \label{sec:causative.manner.complement}
Both sigmatic (§\ref{sec:sig.causative}) and velar causative (§\ref{sec:velar.causative}) causative derivations from adjectival verbs can be used as com\-ple\-ment-taking verbs, expressing the manner in which the action described in the complement clause takes place.


This construction is found with dental/bare infinitival complements (\ref{ex:WpW.Wpa.atAtWGABdi}) or velar infinitival complements (\ref{ex:kArtoR.apWtWsWRzAB}). Additional example sentences are presented in §\ref{sec:sig.caus.complement} and §\ref{sec:velar.caus.complement}. 

\begin{exe} 
\ex \label{ex:WpW.Wpa.atAtWGABdi}
\gll ki tɤ-ndɤm tɕe, koŋla ʑo, [ɯ-pɯ ɯ-pa] a-tɤ-tɯ-ɣɤ-βdi ma  \\
\textsc{dem}.\textsc{prox} \textsc{imp}-take[III] \textsc{lnk} completely \textsc{emph} \textsc{3sg}.\textsc{poss}-keep(1) \textsc{3sg}.\textsc{poss}-\textsc{bare}.\textsc{inf}:keep(2) \textsc{irr}-\textsc{pfv}-2-\textsc{caus}-be.well \textsc{lnk} \\
\glt `Take this and keep it safe (make sure not to lose it).' (140428 mu e guniang-zh)
\japhdoi{0003880\#S23}
\end{exe} 

\begin{exe} 
\ex \label{ex:kArtoR.apWtWsWRzAB}
\gll [nɤ-kɯ-mŋɤm koŋla kɤ-rtoʁ] a-pɯ-tɯ-sɯ-ʁzɤβ \\
\textsc{2sg}.\textsc{poss}-\textsc{sbj}:\textsc{pcp}-hurt completely \textsc{inf}-look \textsc{irr}-\textsc{pfv}-2-\textsc{caus}-be.careful \\
\glt `Let [a doctor] carefully examine your condition.' (elicited)
\end{exe} 


\subsection{\japhug{pa}{do}} \label{sec:pa.complements}
The transitive verb \forme{pa}, which occurs as a light verb (§\ref{sec:pa.lv}) and presents ergative lability (§\ref{sec:lability.pass}), selects imperfective infinitive complements when used in the meaning `discuss and agree/decide to do $X$' (\ref{ex:kukAnAjWjo.topandZi}). All examples of this construction in the corpus occur with B-type orientation preverbs on the infinitive (§\ref{sec:infinitives.other.prefixes}). 

\begin{exe}
\ex \label{ex:kukAnAjWjo.topandZi}
\gll   [nɯtɕu tɕe ɲɯ-kɤ-ɤtɯɣ tɕe ku-kɤ-ɤ-nɤjɯ\redp{}jo] to-pa-ndʑi ri, \\
  \textsc{dem}:\textsc{loc} \textsc{lnk} \textsc{ipfv}-\textsc{inf}-meet \textsc{lnk} \textsc{ipfv}-\textsc{inf}-\textsc{recip}-wait \textsc{ifr}-do-\textsc{du} \textsc{lnk} \\
\glt `The two of them agreed to meet there and wait for each other.' (150820 qaprANar)
\japhdoi{0006246\#S44}
 \end{exe}
 
The verb \forme{pa} can take more than one infinitive complement. There is generally subject coreference between the clauses. However, in example (\ref{ex:tukAqur.topandZi}) the singular subject of the infinitive clauses corresponds to only part of the dual subject of \forme{pa}.\footnote{This interesting sentence differs from the original Chinese text, which has \ch{酋长...决定留下来帮助他}{qiúzhǎng juédìng liúxiàlái bāngzhù tā}{The chieftain ... and decided to stay and help him}. The dual \forme{to-pa-ndʑi} is a translation error, but Tshendzin considered the resulting sentence to be grammatical and meaningful, though semantically different from the original text. }

\begin{exe}
\ex \label{ex:tukAqur.topandZi}
\gll [ɯʑo ku-kɤ-rɤʑi tɕe tu-kɤ-qur] to-pa-ndʑi. \\
\textsc{3sg} \textsc{ipfv}-\textsc{inf}-stay \textsc{lnk} \textsc{ipfv}-\textsc{inf}-help \textsc{ifr}-do-\textsc{du} \\
\glt `They$_{i+j}$ had a discussion and decided that he$_i$ would stay and help him$_j$.' (140512 fushang he yaomo-zh)
\japhdoi{0003967\#S45}
\end{exe}

The verb of speech \japhug{kɤtɯpa}{tell}, which selects reported speech complements, is derived from \forme{pa} (§\ref{sec:irregular.transitive}, §\ref{sec:incorp.denom}).

 \subsection{Modal verbs}
Modality in Japhug is encoded by TAME categories (§\ref{sec:TAME.modal}), derivations such as the abilitivative (§\ref{sec:abilitative}), various sentence final particles (§\ref{sec:fsp.epistemic}), and most importantly by com\-ple\-ment-taking modal auxiliaries and noun-verb collocations (§\ref{sec:nouns.cognition.complement}). This section provides a description of the use of some of the most common auxiliaries. 
  
In addition to the verbs discussed in this section, some verbs of cognition such as  \japhug{sɯso}{think} also have modal functions (§\ref{sec:sWso.complement}).


 \subsubsection{Impersonal modal verbs} \label{sec:ra.khW.jAG.verb}
An important number of modal auxiliary verbs in Japhug are intransitive, and take a complement clause as their subject, and are therefore in \textsc{3sg} invariable form (§\ref{sec:intransitive.invariable}), regardless of the subject or object in the complement clause. The verbs \japhug{ra}{be needed}, `need', \japhug{ɬoʁ}{have to}, `have better',\footnote{This modal verb is probably grammaticalized from the motion verb \japhug{ɬoʁ}{come out}(§\ref{sec:motion.verbs}).} and \japhug{kʰɯ}{be possible} are used with either finite complements (§\ref{sec:finite.complement}) or velar infinitives (§\ref{sec:velar.infinitives.complement.clauses}), as shown by (\ref{ex:YWnWpea.kHW}) and (\ref{ex:kurAZia.YWlhoR}) on the one hand, and (\ref{ex:mAkABzu.mWjkHW}) on the other hand. In addition, \forme{ra} is also attested with degree nominals (§\ref{sec:degree.nominal.complement}).

\begin{exe} 
\ex \label{ex:YWnWpea.kHW}
\gll tʰa [aʑo-sɯso ɲɯ-nɯ-pe-a] kʰɯ \\
later \textsc{1sg}-as.wish \textsc{ipfv}-\textsc{auto}-do[III]-\textsc{1sg} be.possible:\textsc{fact} \\
\glt `(Once he dies), it will become possible for me to do whatever I want.' (28-smAnmi)
\japhdoi{0004063\#S56}
\end{exe} 

\begin{exe} 
\ex \label{ex:kurAZia.YWlhoR}
\gll [kutɕu ku-rɤʑi-a] ɲɯ-ɬoʁ \\
\textsc{dem}.\textsc{prox}:\textsc{loc} \textsc{ipfv}-stay-\textsc{1sg} \textsc{sens}-be.needed \\
\glt `I had better (have no choice but to) stay here.' (28-qAjdoskAt)
\japhdoi{0003718\#S75}
\end{exe} 

When impersonal verbs are combined with velar infinitive complement, the subject is not indexed anywhere in the clause. This construction occurs with generic person, as in (\ref{ex:mAkABzu.mWjkHW}) and (\ref{ex:kAnWXa.lhoR}), but not exclusively, as shown by (\ref{ex:aZosWso.kAnWpa}) with an implicit \textsc{1sg} transitive subject in the complement clause, which has the same meaning as that in (\ref{ex:YWnWpea.kHW}) above.

\begin{exe} 
\ex \label{ex:mAkABzu.mWjkHW}
\gll [ʑara nɯ-skɤt mɤ-kɤ-βzu] mɯ́j-kʰɯ. \\
\textsc{3pl} \textsc{3pl}.\textsc{poss}-speech \textsc{neg}-\textsc{inf}-make \textsc{neg}:\textsc{sens}-be.possible \\
\glt `There was no choice but to speak their language (with them, as they did not understand Japhug or Chinese) (150901 tshuBdWnskAt)
\japhdoi{0006242\#S21}
\end{exe} 

\begin{exe} 
\ex \label{ex:kAnWXa.lhoR}
\gll  nɯ nɯ-mbɯt qʰe, tɕe [tɯ-mɲitsi kɤ-ɤ<nɯ>χa] ɬoʁ \\
\textsc{dem} \textsc{aor}-\textsc{acaus}:take.off \textsc{lnk} \textsc{lnk} \textsc{genr}.\textsc{poss}-life \textsc{inf}-<\textsc{auto}>have.a.gap be.needed:\textsc{fact} \\
\glt `If it (one's tooth) falls out, one is left with a gap [in one's teeth] all one's life.' (27-tWCGArgu)
\japhdoi{0003708\#S58}
\end{exe} 

\begin{exe} 
\ex \label{ex:aZosWso.kAnWpa}
\gll [aʑo-sɯso kɤ-nɯ-pa] tu-kʰɯ ɕti \\
\textsc{1sg}-as.wish \textsc{inf}-\textsc{auto}-do \textsc{ipfv}-be.possible be.\textsc{aff}:\textsc{fact} \\
\glt `Then it will become possible for me to do whatever I want.' (150907 niexiaoqian-zh)
\japhdoi{0006262\#S107}
\end{exe} 


The modal auxiliaries \japhug{ntsʰi}{have better}, \japhug{zgɤt}{have to} and \japhug{jɤɣ}{be allowed} are almost only attested with finite complement clauses, as in (\ref{ex:zYWCara.YWntshi}) and (\ref{ex:pjWsia.YWzgAt}).

\begin{exe} 
\ex \label{ex:zYWCara.YWntshi}
\gll a-ʁi kutɕu kɤ-rɤʑi tɕe, [aʑo tɯ-ci z-ɲɯ-ɕar-a] ɲɯ-ntsʰi \\
\textsc{1sg}.\textsc{poss}-younger.sibling \textsc{dem}.\textsc{prox}:\textsc{loc} \textsc{imp}-stay \textsc{lnk} \textsc{1sg} \textsc{indef}.\textsc{poss}-water \textsc{tral}-\textsc{ipfv}-look.for-\textsc{1sg} \textsc{sens}-have.better \\
\glt `Brother, stay here, let me look for water.' (2011-05-nyima)
\end{exe} 

\begin{exe} 
\ex \label{ex:pjWsia.YWzgAt}
\gll   [aʑo pjɯ-si-a] ɲɯ-zgɤt ma \\
\textsc{1sg} \textsc{ipfv}-die-\textsc{1sg} \textsc{sens}-be.needed \textsc{lnk} \\
\glt `I must die (I deserve to die).' (nongfu yu she-zh)
\end{exe} 


Despite being homophonous to the phasal verb \japhug{jɤɣ}{finish} (§\ref{sec:aspectual.complement}), the modal verb  \japhug{jɤɣ}{be allowed} differs from it both in meaning and in complement type: the modal \forme{jɤɣ} takes finite complements (\ref{ex:YWkhama.jAG2}), while the phasal \forme{jɤɣ} selects dental/bare infinitives (see example \ref{ex:Wti.tojAG}, §\ref{sec:bare.inf.coreference}).

\begin{exe} 
\ex \label{ex:YWkhama.jAG2}
\gll qʰe [ɲɯ-kʰam-a] jɤɣ ri, [nɤ-rca tu-kɯ-tsɯm-a] ra \\
\textsc{lnk} \textsc{ipfv}-give[III]-\textsc{1sg} be.allowed:\textsc{fact} \textsc{lnk} \textsc{2sg}.\textsc{poss}-together.with \textsc{ipfv}:\textsc{up}-2\fl{}-take.away-\textsc{1sg} be.needed:\textsc{fact} \\
\glt `I can give it to you (I agree to give it to you), but then you have to take me (to heaven) with you.' 31-deluge)
\japhdoi{0004077\#S99}
\end{exe} 

Both verbs however are compatible with velar infinite complements, as shown by (\ref{ex:kAtCAt.pWjAG}) and (\ref{ex:kAstu.mWjjAG}), respectively. However, examples of infinitive complements with the modal \forme{jɤɣ} as in (\ref{ex:kAstu.mWjjAG}) are extremely rare.

\begin{exe} 
\ex \label{ex:kAtCAt.pWjAG}
\gll [tɤ-lu kɤ-tɕɤt] pɯ-jɤɣ \\
\textsc{indef}.\textsc{poss}-milk \textsc{inf}-take.out \textsc{aor}-be.finished \\
\glt `After she had finished milking, ...' (2014-kWlAG)
\end{exe} 

\begin{exe} 
\ex \label{ex:kAstu.mWjjAG}
\gll tɕeri tʰam tɕe [nɯra kɤ-stu] mɯ́j-jɤɣ ma,  \\
\textsc{lnk} now \textsc{lnk} \textsc{dem}:\textsc{pl} \textsc{inf}-do.like \textsc{neg}:\textsc{sens}-be.allowed \textsc{lnk} \\
\glt `Now doing these things (i.e. killing whales) is not allowed.' (160703 jingyu)
\japhdoi{0006169\#S41}
\end{exe} 
%nɤ-kɤ-rɤthu pɯ-jɤɣ
%divinitation, 93

Impersonal modal verbs, in addition to finite complements in the Imperfective, are also found with Imperative (§\ref{sec:imp.compl}, §\ref{sec:prohib.function}) and Irrealis (§\ref{sec:irrealis.complement.clauses}) complements. 


Verbs without alternation between stem I and stem III (§\ref{sec:stem3}; in particular intransitive verbs) that select the \textsc{eastwards} orientation take the \forme{kɤ-} orientation preverb in the Imperative (§\ref{sec:imp.morphology}), a form which can be superficially similar to a velar infinitive. For instance, the form \forme{kɤ-nɯ-rɤʑi} in (\ref{ex:kAnWrAZi.jAG}) could in principle either be parsed as an infinitive or an Imperative. However, the is the only possibility here, as demonstrated by the fact that the Imperative dual \forme{kɤ-nɯ-rɤʑi-ndʑi} or plural \forme{kɤ-nɯ-rɤʑi-nɯ} forms can occur in the same context, while the infinitive does not inflect for person and number.


\begin{exe} 
\ex \label{ex:kAnWrAZi.jAG}
\gll [nɤʑo kutɕu kɤ-nɯ-rɤʑi] jɤɣ \\
\textsc{2sg} \textsc{dem}.\textsc{prox}:\textsc{loc} \textsc{imp}-\textsc{auto}-stay be.allowed:\textsc{fact} \\
\glt `You can stay here.' (140504 baixuegongzhu-zh)
\japhdoi{0003907\#S91}
\end{exe}

The impersonal auxiliary \japhug{ŋgrɯ}{succeed}, on the other hand, is only found with velar infinitive complements, as in (\ref{ex:kAnWBzu.mWpjANgrW}).

\begin{exe} 
\ex \label{ex:kAnWBzu.mWpjANgrW}
\gll ɯ-kʰa kɤ-nɯ-βzu mɯ-pjɤ-ŋgrɯ \\
\textsc{3sg}.\textsc{poss}-house \textsc{inf}-\textsc{auto}-make \textsc{neg}-\textsc{ifr}-succeed \\
\glt `(As he spends all his time singing rather than working, in the end) he did not manage to build his house.' (26-NalitCaRmbWm)
\japhdoi{0003676\#S50}
\end{exe}


The verbs \japhug{kʰɯ}{be possible} and \japhug{jɤɣ}{be allowed} both encode epistemic modality, but the former is used in the case of possibility due to external circumstances, while the latter expresses permission by the speaker (\ref{ex:YWkhama.jAG}, \ref{ex:kAnWrAZi.jAG}), the addressee (in interrogative forms such as \forme{ɯ́-jɤɣ}, see \ref{ex:YWtWkhAm.WjAG}, §\ref{sec:interrogative.W.morpho}), or another referent (\ref{ex:YAtWGnW.mWjjAG}). 


\begin{exe} 
\ex \label{ex:YAtWGnW.mWjjAG}
\gll tɕe [tɯ-xpa tɯ-ɣjɤn ma ɲɯ-ɤtɯɣ-nɯ] mɯ́j-jɤɣ \\
\textsc{lnk} one-year one-time apart.from \textsc{ipfv}-meet-\textsc{pl} \textsc{neg}:\textsc{sens}-be.allowed \\
\glt `They can only meet (i.e. they are not allowed by the gods of heaven to meet more than) once a year.' (150828 niulang-zh)
\japhdoi{0006318\#S181}
\end{exe}

Among the verbs discussed in this section, only \forme{ra} and \forme{kʰɯ} are attested in forms other than \textsc{3sg}. The verb \forme{kʰɯ}, in addition to its function as modal auxiliary, can occur with any person when used in the meaning `agree, listen to, obey' (without complement clause) as in (\ref{ex:mWjtWkhWnW}) with a \textsc{2pl} subject.

\begin{exe} 
\ex \label{ex:mWjtWkhWnW}
\gll mɯ́j-tɯ-kʰɯ-nɯ qʰe tɕe, tɕe atu a-mu ci tu tɕe, nɯ ɯ-ɕki zɯ ɣɯ-tɤ-nɯ-tʰu-nɯ ma \\
\textsc{neg}:\textsc{sens}-2-agree-\textsc{pl} \textsc{lnk} \textsc{lnk} \textsc{lnk} up.there \textsc{1sg}.\textsc{poss}-mother \textsc{indef} exist:\textsc{fact} \textsc{lnk} \textsc{dem} \textsc{3sg}.\textsc{poss}-\textsc{dat} \textsc{loc} \textsc{cisl}-\textsc{imp}-\textsc{auto}-ask-\textsc{pl} \textsc{lnk} \\
\glt `(I am telling you that I am not the person you are looking for), you don't want to [listen to me], 
my mother is up there, come (with me) and ask her (about it) as suits you.' (2003 sras)
\end{exe} 

As for \forme{ra}, in the meaning `need' it does not need to have a complement clause as subject, and can select a noun or even a first or second person referent as in (\ref{ex:aZWG.mAtWra}) (see also \ref{ex:mAtWra}, §\ref{sec:intransitive.invariable}).  

\begin{exe} 
\ex \label{ex:aZWG.mAtWra}
\gll  aʑɯɣ mɤ-tɯ-ra \\
 \textsc{1sg}:\textsc{gen} \textsc{neg}-2-need:\textsc{fact} \\
 \glt `I don't need you' (elicited)
\end{exe} 

The experiencer (the person/entity in need) is marked as a genitive oblique argument (§\ref{sec:gen.beneficiary}), either as a genitive pronoun (\ref{ex:aZWG.mAtWra}) or as a possessive prefix (\ref{ex:nAkWqur.tAra}).
 
\begin{exe} 
\ex \label{ex:nAkWqur.tAra}
\gll nɤʑo nɤ-kɯ-qur tɤ-ra tɕe \\
\textsc{2sg} \textsc{2sg}.\textsc{poss}-\textsc{sbj}:\textsc{pcp}-help \textsc{aor}-need \textsc{lnk} \\
\glt `When you need someone to help you...' (140506 shizi he huichang de bailingniao-zh)
\japhdoi{0003927\#S156}
\end{exe}  

%川川总爱刨根问底
%@chuanchuan nɯ kɯ maka ɯ-qa ʑo tu-nɯɬoʁ pjɤ-naʁzi,

The causative verbs \japhug{ɣɤra}{cause to have to} and  \japhug{ɣɤkʰɯ}{make it possible to} (derived by the velar causative prefix from \japhug{ra}{be needed} and \japhug{kʰɯ}{be possible}, respectively), are also com\-ple\-ment-taking verbs, used in one of the periphrastic causative constructions (§\ref{sec:velar.caus.modal}, §\ref{sec:sWpa.sABzu}).

 \subsubsection{\japhug{cʰa}{can}} \label{sec:cha.verb}
The verb \japhug{cʰa}{can} is morphologically intransitive, and can either be used as a plain intransitive (in the meaning `be fine', see for instance \ref{ex:WkutWpe}, §\ref{sec:anticipation.person}) or as a semi-transitive verb (§\ref{sec:semi.transitive}) selecting either a finite complement (§\ref{sec:finite.complement}) or an infinitive one (§\ref{sec:velar.infinitives.complement.clauses}), as shown by (\ref{ex:tukWqura.mAtWcha}).

 \begin{exe}
\ex \label{ex:tukWqura.mAtWcha}
 \gll  [a-sroʁ kɤ-ri] ri mɤ-tɯ-cʰa, [tu-kɯ-qur-a] ri mɤ-tɯ-cʰa  \\
 \textsc{1sg}.\textsc{poss}-life \textsc{inf}-save also \textsc{neg}-2-can:\textsc{fact} \textsc{ipfv}-2\fl{}1-help-\textsc{1sg} also \textsc{neg}-2-can:\textsc{fact}  \\
 \glt `You can neither save my life nor help me.' (shizi he lare neithaoshu-zh)
\end{exe}    

With finite complements, \japhug{cʰa}{can} can either have subject coference as in (\ref{ex:tukWqura.mAtWcha}, \ref{ex:chWtWmACi.tWcha}, \ref{ex:YWkWCWGmua}) or be used as an impersonal verb as in (\ref{ex:chWtWmACi.cha}), with slightly different meanings.
   
\begin{exe}
 \ex
\begin{xlist}
\ex \label{ex:chWtWmACi.tWcha}
 \gll [cʰɯ-tɯ-mɤɕi] tɯ-cʰa \\
 \textsc{ipfv}-2-be.rich 2-can:\textsc{fact} \\
 \glt `You can become rich.' (elicited)
 \ex \label{ex:chWtWmACi.cha}
 \gll [cʰɯ-tɯ-mɤɕi] cʰa \\
 \textsc{ipfv}-2-be.rich can:\textsc{fact} \\
 \glt `It will be possible for you to become rich.' (140515 facaimeng-zh)
 \japhdoi{0004000\#S11}
\end{xlist}
\end{exe}
  
 Since the \textsc{1sg} form \forme{cʰa-a} (can:\textsc{fact}-\textsc{1sg}) is phonetically identical to the \textsc{3sg} \forme{cʰa} (can:\textsc{fact}), distinguishing between these two patterns is not possible when the subject is \textsc{1sg} (or \textsc{3sg}). Clear examples of the \forme{cʰa} in impersonal use (where it can be tested) are rare.
 
All examples in the corpus of \japhug{cʰa}{can} with a verb in \textsc{2sg}\fl{}\textsc{1sg} form in the complement clause have \textsc{2sg} indexation as in (\ref{ex:tukWqura.mAtWcha}) above and (\ref{ex:YWkWCWGmua}) below.

\begin{exe}
\ex \label{ex:YWkWCWGmua}
\gll [aʑo ɲɯ-kɯ-ɕɯɣ-mu-a] mɤ-tɯ-cʰa  \\  
  \textsc{1sg} \textsc{ipfv}-2\fl{}1-\textsc{caus}-be.afraid-\textsc{1sg} \textsc{neg}-2-can:\textsc{fact} \\
\glt `You cannot scare me.' (140516 guowang halifa-zh)
\japhdoi{0004008\#S52}
  \end{exe}


\subsubsection{\japhug{sɯxcʰa}{can}} \label{sec:sWxcha}
The verb \forme{sɯx-cʰa} is the sigmatic causative derivation (§\ref{sec:sig.causative}) of \japhug{cʰa}{can} (§\ref{sec:cha.verb}). It is nearly always in negative form. It can occur with an overt nominal object in the meaning `cause to be (un)able to bear' (example \ref{ex:mWYWsWxche}, §\ref{sec:sig.caus.modal}). As a com\-ple\-ment-taking verb, it requires a velar infinitive complement (§\ref{sec:velar.infinitives.complement.clauses}), and exclusively occurs in inverse form with a non-overt causer, and has the specific meaning of `make (physically) unable to $X$', with a non-overt implicit agent, the piglet or lamb in (\ref{ex:mWYWwGsWxcha}) and the bee in (\ref{ex:kAsat.mAwGsWxcha}).

\begin{exe}
\ex \label{ex:mWYWwGsWxcha}
\gll tʰɯ-wxti-nɯ tsa tɕe tɕe, ta-tsɯm tɕe tu-ɣɤrʁɤβjɤβ ɲɯ-ɕti tɕe ɯʑo [kɤ-tsɯm] mɯ-ɲɯ́-wɣ-sɯx-cʰa. \\
\textsc{aor}-be.big-\textsc{pl} a.little \textsc{lnk}  \textsc{lnk} \textsc{aor}:3\flobv{}-take.away \textsc{lnk} \textsc{ipfv}-struggle \textsc{sens}-be.\textsc{aff}   \textsc{lnk} \textsc{3sg} \textsc{inf}-take.away \textsc{neg}-\textsc{sens}-\textsc{inv}-\textsc{caus}-can \\
\glt `When [piglets and lambs] have gotten big, [if the eagle tries to] carry away [one of them]$_i$, it$_i$ struggles and [the eagle] is not strong enough to take it$_i$ away.' (150819 RarphAB)
\japhdoi{0006356\#S5}
\end{exe}	

\begin{exe}
\ex \label{ex:kAsat.mAwGsWxcha}
\gll  ɣʑo kɯ-fse kɯ-wxti nɯra rcanɯ, [ʑaʑa ʑo kɤ-sat] mɤ́-wɣ-sɯx-cʰa ma ɯʑo ɲɯ-xtɕi, ci nɯ ɲɯ-wxti tɕe \\
bee \textsc{sbj}:\textsc{pcp}-be.like \textsc{sbj}:\textsc{pcp}-be.big \textsc{dem}:\textsc{pl} \textsc{unexp}:\textsc{deg} soon \textsc{emph} \textsc{inf}-kill \textsc{neg}-\textsc{inv}-\textsc{caus}-can:\textsc{fact} \textsc{lnk} \textsc{3sg} \textsc{sens}-be.small \textsc{indef} \textsc{dem} \textsc{sens}-be.big \textsc{lnk} \\
\glt `In the case of a (relatively) big [insect] like a bee,$_i$ [the spider]$_j$ cannot kill it$_i$ quickly, because it$_j$ is small while the other one$_i$ is bigger.' 
\japhdoi{0003674\#S100}
\end{exe}	

The transitive subject of the infinitival clauses \forme{kɤ-tsɯm} and \forme{ʑaʑa ʑo kɤ-sat} in (\ref{ex:mWYWwGsWxcha}) and (\ref{ex:kAsat.mAwGsWxcha}) is here coreferent with the causee of \forme{sɯx-cʰa}. Since \forme{cʰa} is morphologically intransitive, this causee is syntactically equivalent to the direct object. It is one of the rare com\-ple\-ment-taking verbs with obligatory object-subject coreference (§\ref{sec:velar.inf.coreference}).

 \subsubsection{\japhug{spa}{be able}}  \label{sec:spa.verb}
The transitive \japhug{spa}{be able}, `know how to' is historically a lexicalized abilitative of \japhug{pa}{do} (§\ref{sec:abilitative.lexicalized}). It indicates abiliy that was acquired through a learning process.

This verb can either select a noun or a noun phrase as object (\ref{ex:kupaskAt.mWjspea}, \ref{ex:kospa}). 


\begin{exe}
\ex \label{ex:kupaskAt.mWjspea}
\gll aʑo kupa-skɤt mɯ́j-spe-a \\
\textsc{1sg} Chinese-language \textsc{neg}:\textsc{sens}-be.able[III]-\textsc{1sg} \\
\glt `I am not able to [speak] Chinese.' (160721 XpWN)
\japhdoi{0006181\#S84}
\end{exe}

In Inferential or Aorist forms, it can be understood as `learn how to' (`acquire the ability to').

\begin{exe}
\ex \label{ex:kospa}
\gll ɕoŋβzu ko-spa \\
carpentry \textsc{ifr}-be.able \\
\glt `He learned carpentry.' (elicited)
\end{exe}

Alternatively, \japhug{spa}{be able} takes infinitival (§\ref{sec:velar.infinitives.complement.clauses}) or finite complement clauses (§\ref{sec:finite.complement}) as in (\ref{ex:tuti.WYWspe}). Subject coreference between \forme{spa} and the verb in the complement clause is required  (§\ref{sec:velar.inf.coreference},  §\ref{sec:finite.complement.coref}).

\begin{exe}
\ex \label{ex:tuti.WYWspe}
 \gll [``a-mu" tu-ti] ɯ-ɲɯ́-spe? \\
 \textsc{1sg}.\textsc{poss}-mother \textsc{ipfv}-say \textsc{qu}-\textsc{sens}-be.able[III] \\
 \glt `Can he (a baby) say ``mummy''?' (conversation, 15-01-13)
  \end{exe}
  
 \subsubsection{\japhug{nɤz}{dare}, \japhug{pʰot}{dare}}  \label{sec:nAz.verb}
 The verbs \japhug{nɤz}{dare} and \japhug{pʰot}{dare} (the second is barely used in the Kamnyu dialect) can be used with both infinitival and finite complements. They require subject coreference. 
 
 
In example (\ref{ex:CWkAnAkhu.mAnaza})  with the verb \japhug{nɤkʰu}{invite} (to one's home as a guest' (one of the few transitive verbs implying a volitional action of both subject and object, §\ref{sec:bare.inf.coreference} ), the interpretation `I do not dare to go to his house as a guest' (with coreference of the object of the complement clause and the subject of the main clause) is not possible, and a different construction is needed to express this meaning (\ref{ex:kAnAkhu.kACe.mAnaza}, with an object participle as explained in §\ref{sec:purposive.clause.motion.verbs}).

\begin{exe}
\ex \label{ex:CWkAnAkhu.mAnaza}
\gll  [ɕɯ-kɤ-nɤkʰu] mɤ-naz-a \\
\textsc{tral}-\textsc{inf}-invite \textsc{neg}-dare:\textsc{fact}-\textsc{1sg} \\
\glt `I do not dare to go and invite him.' (elicited)
\end{exe}


\begin{exe}
\ex \label{ex:kAnAkhu.kACe.mAnaza}
\gll [[kɤ-nɤkʰu] kɤ-ɕe] mɤ-naz-a \\
\textsc{obj}:\textsc{pcp}-invite \textsc{inf}-go \textsc{neg}-dare:\textsc{fact}-\textsc{1sg} \\
\glt `I do not dare to go [to his house] as a guest.' (elicited)
\end{exe}

The verb \japhug{nɤz}{dare} is attested in a double negation constructions `not dare not to $X$' (example \ref{ex:mAkAGi.mAnAzi}, §\ref{sec:double.negation}).

 \subsection{Verbs of cognition}

 \subsubsection{\japhug{sɯso}{think}, `want'} \label{sec:sWso.complement}
 The verb \forme{sɯso} describes either a cognitive process (`think that $X$') or volitional deontic modality (`want to $X$').

As a modal verb, it either takes infinitive complement clauses (\ref{ex:kAtshi.tasWsonW}, §\ref{sec:velar.infinitives.complement.clauses}), or finite clauses (\ref{ex:kuGWta.nWsWsota}, \ref{ex:YWtWsANo.tAtWsot}, §\ref{sec:finite.complement}), with subject coreference between the complement and the main clause.  
 
\begin{exe}
\ex \label{ex:kAtshi.tasWsonW}
\gll [cʰɤci kɤ-tsʰi] ta-sɯso-nɯ tɕe, tɕe ki ɯ-qa ɯ-tʰɯm nɯ ɲɯ-χɕoʁ-nɯ tɕe \\
alcohol \textsc{inf}-drink \textsc{aor}:3\flobv{}-want-\textsc{pl} \textsc{lnk} \textsc{lnk} \textsc{dem}.\textsc{prox} \textsc{3sg}.\textsc{poss}-bottom {3sg}.\textsc{poss}-cork \textsc{dem} \textsc{ipfv}-remove-\textsc{pl} \textsc{lnk} \\
\glt `When people want to drink the alcohol, they remove the cork at the bottom (of the jar).' (160703 araR)
\japhdoi{0006101\#S63}
\end{exe}

\begin{exe}
\ex \label{ex:kuGWta.nWsWsota}
\gll [ku-ɣɯt-a] nɯ-sɯso-t-a ri ɲɤ-nɯ-jmɯt-a ma\\
\textsc{ipfv}:\textsc{east}-bring-\textsc{1sg} \textsc{aor}-want-\textsc{pst}:\textsc{tr}-\textsc{1sg} \textsc{lnk} \textsc{ifr}-\textsc{auto}-forget-\textsc{1sg} \textsc{lnk}\\
\glt `I wanted to bring it but I forgot it.' (23-tshAYCAnW)
\japhdoi{0003616\#S2}
\end{exe}
 
\begin{exe}
\ex \label{ex:YWtWsANo.tAtWsot}
\gll [a-rɤɣo ɲɯ-tɯ-sɤŋo] tɤ-tɯ-sɯso-t tɕe, a-ɣɯ-jɤ-kɯ-sɯɣe-a qʰe nɯ ɕti \\
\textsc{1sg}.\textsc{poss}-song \textsc{ipfv}-2-listen \textsc{aor}-2-want-\textsc{pst}:\textsc{tr} \textsc{lnk} \textsc{irr}-\textsc{cisl}-invite-\textsc{1sg} \textsc{lnk} \textsc{dem} be.\textsc{aff}:\textsc{fact} \\
\glt   `When you want to listen to my songs, send people to come and invite me and that's it.' (140519 yeying-zh)
\japhdoi{0004040\#S227}
\end{exe}

The verb \forme{sɯso} normally selects the \textsc{westwards} orientation as in (\ref{ex:kuGWta.nWsWsota}) (or \textsc{downwards} when used in the Past or Inferential Imperfective (§\ref{sec:pst.ifr.ipfv}), but examples (\ref{ex:kAtshi.tasWsonW}) and (\ref{ex:YWtWsANo.tAtWsot}) also show that it sometimes occurs with the Aorist \textsc{upwards} orientation preverbs \forme{tɤ-} or \forme{ta-} in temporal clauses meaning ``when $X$ wants to $Y$'' (§\ref{sec:aor.temporal}).
  
% (§\ref{sec:irrealis.complement.clauses}, \citeyear[807]{jackson07irrealis}).
\subsubsection{\japhug{tso}{know, understand}, `know', \japhug{sɯχsɤl}{recognize}, `realize'} \label{sec:tso.sWXsAl}
The semi-transitive verb \japhug{tso}{know, understand}, `realize',  `know' can take nominal semi-objects (§\ref{sec:semi.transitive}), and occur with three types of complement clauses or complementation strategy. First, it takes finite complete clauses (§\ref{sec:finite.complement}) as in (\ref{ex:GAZu.nW.kotsonW}).

\begin{exe}
\ex \label{ex:GAZu.nW.kotsonW}
\gll  nɯnɯ [tɤ-tɕɯ nɯ ɣɯ ɯ-kʰa nɯtɕu tɕʰeme kɯ-mpɕɤr ci ɣɤʑu] nɯ ko-tso-nɯ. \\
\textsc{dem} \textsc{indef}.\textsc{poss}-son \textsc{dem} \textsc{gen} \textsc{3sg}.\textsc{poss}-house \textsc{dem}:\textsc{loc} girl \textsc{sbj}:\textsc{pcp}-be.beautiful \textsc{indef} exist:\textsc{sens} \textsc{dem} \textsc{ifr}-understand-\textsc{pl} \\
\glt `They realized that there was a beautiful girl in the boy's house.' (150828 donglang)
\japhdoi{0006312\#S89}
\end{exe}

Second, subject participial clauses in \forme{kɯ-} occur as semi-objects of this verb (§\ref{sec:relative.core.arg}), as in (\ref{ex:mWpWkWsi.kotso}).

\begin{exe}
\ex \label{ex:mWpWkWsi.kotso}
\gll  li [iɕqʰa tɤɕime nɯ mɯ-pɯ-kɯ-si] nɯ ko-tso \\
again the.aforementioned girl \textsc{dem} \textsc{neg}-\textsc{aor}-\textsc{sbj}:\textsc{pcp}-die \textsc{dem} \textsc{ifr}-understand \\
\glt  `She realized that the girl had not died.' (140504 baixuegongzhu-zh)
\japhdoi{0003907\#S174}
\end{exe}

Third, \forme{tso} also selects a (finite or participial) correlative clauses with an interrogative pronoun as in (\ref{ex:thAstWG.nari.mWpjAtso}).

\begin{exe}
\ex \label{ex:thAstWG.nari.mWpjAtso}
\gll [tɤ-rʑaʁ tʰɤstɯɣ nɯ-ari] mɯ-pjɤ-tso \\
\textsc{indef}.\textsc{poss}-time how.much \textsc{aor}-go[II] \textsc{neg}-\textsc{ifr}-understand \\
\glt `He had not realized how much time had passed.' (28-smAnmi)
\japhdoi{0004063\#S246}
\end{exe}

The transitive  \japhug{sɯχsɤl}{recognize}, `realize' also takes finite complement clauses  (see \ref{ex:tari.CtatWt}, §\ref{sec:redundant.AM}) or participial clauses as in (\ref{ex:tAkWnWCpWz.pjAsWXsAl}).

\begin{exe}
\ex \label{ex:tAkWnWCpWz.pjAsWXsAl}
 \gll tɕe nɯnɯ kɯ [qaʑo tɤ-kɯ-nɯɕpɯz] nɯ pjɤ-sɯχsɤl. \\
 \textsc{lnk} \textsc{dem} \textsc{erg} sheep \textsc{aor}-\textsc{sbj}:\textsc{pcp}-disguised \textsc{dem} \textsc{ifr}-recognize \\
\glt `He (the shepherd boy) had noticed that the [nobleman] was disguised as a sheep.' (40513 mutong de disheng-zh)
\japhdoi{0003977\#S58}
\end{exe}

However, despite being superficially identical to (\ref{ex:mWpWkWsi.kotso}) above, in (\ref{ex:kWNu.amApWwGsWXsala}) \forme{sɯχsɤl} selects as direct object the \textsc{1sg}  intransitive subject  of the participial clause. A possible way to explain this observation is to analyze the participial clause \forme{aʑo tɕʰeme kɯ-ŋu} as a relative clause whose head is \forme{aʑo}  (`me who am a girl').
 
\begin{exe}
\ex \label{ex:kWNu.amApWwGsWXsala}
 \gll <liangshanbo> nɯ kɯ [aʑo tɕʰeme kɯ-ŋu] nɯ a-mɤ-pɯ́-wɣ-sɯχsal-a \\
  \textsc{anthr} \textsc{dem} \textsc{erg} \textsc{1sg} girl \textsc{sbj}:\textsc{pcp}-be \textsc{dem} \textsc{irr}-\textsc{neg}-\textsc{pfv}-\textsc{inv}-realize-\textsc{1sg} \\
 \glt `[I hope that] Liang Shanbo will not realize that I am a girl.' (150826 liangshanbo zhuyingtai-zh)
\japhdoi{0006244\#S73}
 \end{exe}

%ŋotɕu pɯ-ŋke nɯ ɲɯ-saχsɤl.
 %kɯ-ɤrqhɯ-rqhi ju-kɯ-ru tɕe, ŋotɕu mɲɤm kɯ-tu nɯ ɲɯ-saχsɤl 
 
 \subsubsection{\japhug{βzjoz}{learn} and \japhug{sɯxɕɤt}{teach}} \label{sec:Bzjoz.sWxCAt.complement}
 The verbs of learning \japhug{βzjoz}{learn} and \japhug{sɯxɕɤt}{teach} take infinitive complements, as shown by (\ref{ex:kAlAt.kaBzjoz}) and (\ref{ex:kAlAt.pjWtasWxCAt}).
 
\begin{exe}
\ex \label{ex:kAlAt.kaBzjoz}
 \gll [<tuolaji> kɤ-lɤt] ra ka-βzjoz \\
 tractor \textsc{inf}-release \textsc{pl} \textsc{aor}:3\flobv{}-learn \\
 \glt  `He learned to drive a tractor.' (14-siblings)
 \japhdoi{0003508\#S208}
 \end{exe}

In the case of the secundative verb \forme{sɯxɕɤt} (§\ref{sec:ditransitive.secundative}), the direct object is the person being taught, the \textsc{2sg} in (\ref{ex:kAlAt.pjWtasWxCAt}), while the infinitive clause is the theme.

\begin{exe}
\ex \label{ex:kAlAt.pjWtasWxCAt}
 \gll [tɤfsɤri kɤ-βzu] ci pjɯ-ta-sɯxɕɤt \\
 thread \textsc{inf}-make a.little \textsc{ipfv}-1\fl{}2-teach \\
 \glt `Let me teach you how to make a thread.' (vid-20140506043657)
  \end{exe}
  
 \subsection{Verbs of perception}   \label{sec:mto.mtshAm.complement}
The transitive perception verbs \japhug{mto}{see} and \japhug{mtsʰɤm}{hear}, `smell', `feel (non-visually)' express non-volitional perception. The contrast between \japhug{ru}{look at} (§\ref{sec:coordination.comp.str}) and \forme{mto} can be illustrated by (\ref{ex:schAru.mWpjAmto}), where the latter has a meaning close to `find'.

\begin{exe}
\ex \label{ex:schAru.mWpjAmto}
 \gll kʰɯɣɲɟɯ ra s-cʰɤ-ru ri mɯ-pjɤ-mto \\
 window \textsc{pl} \textsc{tral}-\textsc{ifr}:\textsc{downstream}-look \textsc{lnk} \textsc{neg}-\textsc{ifr}-see \\
 \glt `She went and looked from the window but did not see him.' (140505 xiaohaitu-zh)
\japhdoi{0003921\#S61}
\end{exe}

These two verbs are compatible with finite complements (§\ref{sec:finite.complement}), as in (\ref{ex:YWGAwu.pjAmtshAm}), without any coreference restriction. 

\begin{exe}
\ex \label{ex:YWGAwu.pjAmtshAm}
 \gll [sɯŋgi nɯ ɲɯ-ɣɤwu] nɯ pjɤ-mtsʰɤm \\
 lion \textsc{dem} \textsc{ipfv}-cry \textsc{dem} \textsc{ifr}-hear \\
 \glt `[The mouse] heard that the lion was crying.' (shizi he laoshu-zh)
\end{exe}

They are incompatible with infinitival complements, but often take participial or finite relative clauses as objects (§\ref{sec:non-finite.relative.complement.ambiguity}, §\ref{sec:relative.core.arg}). 

Non-finite verb form in \forme{kɤ-} such as \forme{kɤ-ntsɣe} in (\ref{ex:kAntsGe.mWpWmtota}) resemble velar infinitives, but since intransitive dynamic verbs do not occur with \forme{kɤ-} in this context, these forms are analyzed as object participles (§\ref{sec:infinitives.participles}).


\begin{exe}
\ex \label{ex:kAntsGe.mWpWmtota}
 \gll tɕe [ɯ-rdoʁ nɯ kɯ-fse kɤ-ntsɣe] nɯ mɯ-pɯ-mto-t-a  \\
 \textsc{lnk} \textsc{3sg}.\textsc{poss}-grain \textsc{dem} \textsc{sbj}:\textsc{pcp}-be.like \textsc{obj}:\textsc{pcp}-sell \textsc{dem} \textsc{neg}-\textsc{aor}-see-\textsc{pst}:\textsc{tr}-\textsc{1sg} \\
 \glt `I have not seen its grains (of buckthorn) sold as is (it is always sold in processed form).' (09-mi)
 \japhdoi{0003466\#S55}
\end{exe}

The verb \forme{mtsʰɤm} is often used in combination with the form \forme{kɤ-ti} as in (\ref{ex:YWsAmtsWG.kAti}). This form could be analyzed as a complementizer, but since positive evidence of its grammaticalized status is not obvious, I analyze \forme{kɤ-ti} here as an object participle taking an object complement clause (\forme{nɯ ɲɯ-sɤ-mtsɯɣ}). That complement clause itself is also the relativized element of the head-internal object participial clause \forme{[``\textbf{nɯ ɲɯ-sɤ-mtsɯɣ}'' kɤ-ti]} (§\ref{sec:relativized.complement.clause}), a construction that can be glossed as `I have not heard `It bites people' being said'.

\begin{exe}
\ex \label{ex:YWsAmtsWG.kAti}
 \gll [[nɯ ɲɯ-sɤ-mtsɯɣ] kɤ-ti] mɯ-pɯ́-wɣ-mtsʰɤm. \\
 \textsc{dem} \textsc{sens}-\textsc{apass}-bite \textsc{obj}:\textsc{pcp}-say \textsc{neg}-\textsc{aor}-\textsc{inv}-hear \\
 \glt `I have not heard that it bites people.' (28-tshAwAre)
 \japhdoi{0003722\#S69}
\end{exe}
 
With intransitive verbs, the subject participle in \forme{kɯ-} is always found in the clauses occurring as the object of \forme{mto} and \forme{mtsʰɤm}, as in (\ref{ex:pWkWntChAr.pjAmto}). Such clauses are to be analyzed as participial relatives (§\ref{sec:relative.core.arg}).

\begin{exe}
\ex \label{ex:pWkWntChAr.pjAmto}
 \gll pjɯ-ru qʰe [tɕeki tɯ-ci ɯ-ŋgɯ \textbf{ɯʑo} pɯ-kɯ-ntɕʰɤr] nɯ pjɤ-mto. \\
 \textsc{ipfv}:\textsc{down}-look \textsc{lnk} down \textsc{indef}.\textsc{poss}-water \textsc{3sg}.\textsc{poss}-in \textsc{3sg} \textsc{aor}:\textsc{down}-\textsc{sbj}:\textsc{pcp}-appear \textsc{dem} \textsc{ifr}-see \\
 \glt  `He looked down, and saw his own reflection in the water below (himself reflected in the water).' (140519 chou xiaoya-zh)
\japhdoi{0004034\#S164}
 \end{exe}
 
 In   (\ref{ex:WkWti.pjAmtshAm}), \forme{mtsʰɤm} even takes as object a head-internal relative clause in \forme{kɯ-} (rather than \forme{kɤ-} as in \ref{ex:YWsAmtsWG.kAti} above), with the transitive subject as relativized element (§\ref{sec:tr.subject.relativization}).
 
\begin{exe}
\ex \label{ex:WkWti.pjAmtshAm}
 \gll  ɲɯ-sɤŋo tɕe [iɕqʰa \textbf{sŋaʁspa} \textbf{nɯ} kɯ ``tɤtʂu ɣɯ-tɤ-sɤndu-nɯ" ɯ-kɯ-ti] nɯ pjɤ-mtsʰɤm tɕe \\
 \textsc{ipfv}-listen \textsc{lnk} the.aforementioned sorcerer \textsc{dem} \textsc{erg} lamp \textsc{cisl}-\textsc{imp}-exchange-\textsc{pl} \textsc{3sg}.\textsc{poss}-\textsc{sbj}:\textsc{pcp}-say \textsc{dem} \textsc{ifr}-hear \textsc{lnk} \\
 \glt `She heard the sorcerer saying `Come and exchange [your] lamp'.' (140511 alading-zh)
 \japhdoi{0003953\#S212}
 \end{exe}

Verbs of perception and cognition also occur with finite clauses containing interrogative pronouns which look like correlatives (§\ref{sec:interrogative.relative}), such as \forme{ŋotɕu jɤ-nɯ-ɬoʁ} in (\ref{ex:NotCu.jAnWlhoR.mWpWmtoj}).

\begin{exe}
\ex \label{ex:NotCu.jAnWlhoR.mWpWmtoj}
 \gll [ŋotɕu jɤ-nɯ-ɬoʁ] tɕi mɯ-pɯ-mto-j, [ŋotɕu jɤ-ɕqʰlɤt] tɕi kɤ-ɕɯftaʁ mɯ-pjɤ-cʰa-j \\
where \textsc{ifr}-\textsc{auto}-come.out also \textsc{neg}-\textsc{aor}-see-\textsc{1pl} where \textsc{ifr}-disappear also \textsc{inf}-remember \textsc{neg}-\textsc{ifr}-can-\textsc{1pl} \\
\glt `We neither saw where she came from, nor can we remember which way she went.' (2003 sras)
\end{exe}
 
Although the volitional perception verbs  \japhug{sɤŋo}{listen} (labile), \japhug{rtoʁ}{look} (transitive) and \japhug{ru}{look at} (§\ref{sec:orienting.verbs}) can have nominal objects or semi-objects, they do not take complement clauses. However, they do occur in coordinating complementation strategies (§\ref{sec:ipfv.perception}, §\ref{sec:coordination.comp.str}).
 

  \subsection{Phasal verbs and other aspectual auxiliaries} \label{sec:aspectual.complement}
Aspectual and phasal com\-ple\-ment-taking verbs present a  much greater variety of constructions than modal verbs. \tabref{tab:phasal} summarizes the constructions attested with each verb, not all of which are equally common.\footnote{The abbreviations are as follows: 	I. (velar infinitive, §\ref{sec:velar.infinitives.complement.clauses}), BI (bare infinitive and \forme{tɯ-} infinitive, §\ref{sec:bare.dental.inf}), 	F. (finite complement, §\ref{sec:finite.complement}), tr. (transitive), impers. (intransitive impersonal). }  
  

\begin{table}[H]
\caption{Inventory of phasal and aspectual auxiliaries in Japhug} \label{tab:phasal} 
\begin{tabular}{lllllll}
\lsptoprule
Verb  & 	&  	I. & 	 	BI & 	F. & 	Compl. strategy   	\\
\midrule
\japhug{rɲo}{experience} & 	tr. & 	\Y & 	 	\Y & 	\N & 	\N & 	\\
\japhug{sɤʑa}{begin}  & 	tr. & 	\Y & 	 	\Y & 	\N & 	\N & 	\\
\japhug{ʑa}{begin} & 	tr. & 	\Y & 	\Y & 	§\ref{sec:svc.finite.agreement} &	\\
\japhug{stʰɯt}{finish} & 	tr. & 	\Y & 	\Y & 	 §\ref{sec:svc.finite.agreement}	 &  	\\
\japhug{sɤtɕɯtʂi}{continue} & 	tr. & 	\Y & 	 	\N & 	§\ref{sec:svc.finite.agreement}		  & 	\\
\japhug{nɯftɕaka}{prepare} & 	tr. & 	\Y & 	\N & 	\Y & 	\N & 	\\
\japhug{sɯɣjɤɣ}{finish} & 	tr. & 	\Y & 	\N & 	\N & 	\N & 	\\
\japhug{kʰɤt}{do repeatedly} & 	tr. & 	\Y & 	\N & 	\N & 	§\ref{sec:complementation.strategy.action.nominals} & 	\\
\midrule
\japhug{jɤɣ}{finish} & 	impers. & 	\Y & 	\Y & 	\N & 	\N & 	\\
\japhug{ŋgrɤl}{be usually the case} & 	impers. & 		\N & 	\N & 	\Y & 	\N & 	\\
\japhug{rɤŋgat}{be about to} & 	intr. & 	\N &  	\N & 	\N & 	§\ref{sec:purposive.clause.motion.verbs} & 	\\
\japhug{aɣɯɣu}{be about to} & 	intr. & 	\N &  	\N & 	\N & 	§\ref{sec:purposive.clause.motion.verbs}  & 	\\ 
\japhug{mda}{be the time} & 	impers. & 	\Y &   \N & 	\Y & 	\N & 	\\
\lspbottomrule
\end{tabular}
\end{table}

As shown in §\ref{sec:bare.inf.coreference}, verbs in this group have different coreference restrictions depending on the complement type: when transitive verbs select bare/dental infinitive complements, coreference is required between the subject of the complement clause and that of the matrix verb.

\subsubsection{Experiential} \label{sec:rYo.complements}
The transitive verb \forme{rɲo} can be used with a nominal object in the meaning `try, taste' (of food), as shown in (\ref{ex:tukWsWrYoa}) with the sigmatic causative \forme{sɯ-rɲo} `let $X$ taste' (§\ref{sec:sig.causative}).

\begin{exe}
\ex \label{ex:tukWsWrYoa}
 \gll aʑo nɤ-paχɕi ci tu-kɯ-sɯ-rɲo-a \\
 \textsc{1sg} \textsc{2sg}.\textsc{poss}-apple a.little \textsc{ipfv}-2\fl{}1-\textsc{caus}-taste-\textsc{1sg} \\
 \glt `Let me taste your apples.' (150904 zhongli-zh)
 \japhdoi{0006348\#S15}
\end{exe}    
    
As a com\-ple\-ment-taking verb, \forme{rɲo} means `have already done...' (like Chinese \zh{曾经……过} \forme{céngjīng...guò}), and commonly occurs with both bare/dental infinitival complements (§\ref{sec:bare.dental.inf}) and velar infinitival complements (§\ref{sec:velar.infinitives.complement.clauses}). These two complement types differ by their coreference restrictions: the former requires subject coreference (§\ref{sec:bare.inf.coreference}), while the latter does not.


In example (\ref{ex:kAmtsWG.P}), the first sentence \forme{aʑo kɤ-mtsɯɣ mɯ-pɯ-rɲo-t-a} is ambiguous, and could be translated as either `I have never bitten it' (subject coreference) or `I have never been bitten by it' (object coreference), regardless of the fact that the \textsc{1sg} is a transitive subject in both cases as shown by the absence of an inverse prefix and the presence of the \forme{-t-} suffix (§\ref{sec:indexation.mixed}, §\ref{sec:other.TAME}). The verb \japhug{rɲo}{experience} lacks inverse forms other than the generic subject (§\ref{sec:indexation.generic.tr}), and the 3\fl{}\textsc{1sg} inverse form $\dagger$\forme{mɯ-pɯ́-wɣ-rɲo-a} is rejected, and cannot be used to replace  \forme{mɯ-pɯ-rɲo-t-a} in (\ref{ex:kAmtsWG.P}).

\begin{exe}
\ex   \label{ex:kAmtsWG.P} 
\gll aʑo [kɤ-mtsɯɣ] mɯ-pɯ-rɲo-t-a ri, χpɤltɕɯn kɯ pjɤ-rɲo  \\
\textsc{1sg} \textsc{inf}-bite \textsc{neg}-\textsc{aor}-experience-\textsc{pst}:\textsc{tr}-\textsc{1sg} but p.n \textsc{erg}  \textsc{ifr}-experience \\
\glt `I have never been stung [by a wasp], but Dpalcan has.' (26-ndzWrnaR) 
\japhdoi{0003678\#S19}
\end{exe}  
 
In the second sentence, the personal name \forme{χpɤltɕɯn} predictably takes the ergative, being the transitive subject of \japhug{rɲo}{experience}. However, it is at the same time object of (elided) infinitive \forme{kɤ-mtsɯɣ} `to bite' present in the first clause, and the absolutive form would be expected if the verb in the complement clause had precedence over the matrix verb, as happens in some cases (§\ref{sec:complement.clause.case.marking}). The complete clause without elision would \forme{χpɤltɕɯn kɯ \textbf{kɤ-mtsɯɣ} pjɤ-rɲo} `Dpalcan has been stung (by a wasp) before'.


\begin{exe}
\ex   \label{ex:kAmtsWG.pjArYo} 
\gll χpɤltɕɯn kɯ [kɤ-mtsɯɣ]  pjɤ-rɲo  \\
 p.n \textsc{erg} \textsc{inf}-bite \textsc{ifr}-experience \\
\glt `Dpalcan has been stung [by a wasp] before'. (elicitation based on \ref{ex:kAmtsWG.P})
\end{exe} 

The sentence (\ref{ex:kAmtsWG.pjArYo}) is also ambiguous, but can be interpreted as expressing coreference between the transitive subject of \japhug{rɲo}{experience} with the object of the transitive verb \japhug{mtsɯɣ}{bite}, with the ergative flagging of the matrix clause taking over the absolutive marking expected in the complement clause.


The subject of \japhug{rɲo}{experience} can also be coreferential with the possessor of the intransitive subject in the complement clause, as in example (\ref{ex:kAmNAm}), where the non-overt subject should be \forme{a-xtu} `my belly', as in (\ref{ex:kAmNAm.pWrYota}) (§\ref{sec:other.collocation.intr}).
 
 \begin{exe}
\ex \label{ex:kAmNAm}
\gll aʑo pɯ-xtɕɯ\redp{}xtɕi-a ʑo ri tɯxtɤŋɤm nɯ-atɯɣ-a tɕe, [nɯ kɤ-mŋɤm] pɯ-rɲo-t-a \\
\textsc{1sg} \textsc{pst}.\textsc{ipfv}-\textsc{emph}\redp{}be.small-\textsc{1sg} \textsc{emph} \textsc{loc} dysentery \textsc{aor}-meet-\textsc{1sg} \textsc{lnk} \textsc{dem} \textsc{inf}-hurt \textsc{aor}-experience-\textsc{1sg} \\
\glt `When I was very small, I had dysentery, [my belly] ached.'  (24-pGArtsAG)
\japhdoi{0003624\#S116}
\end{exe}

\begin{exe} 
\ex \label{ex:kAmNAm.pWrYota}
\gll [a-xtu kɤ-mŋɤm] pɯ-rɲo-t-a. \\
\textsc{1sg}.\textsc{poss}-belly \textsc{inf}-hurt \textsc{aor}-experience-\textsc{pst}:\textsc{tr}-\textsc{1sg} \\
\glt `I have had stomachache.' (elicited)
\end{exe} 

However, despite the fact that \forme{rɲo} indexes either the subject or the object of its complement clause without difference in the matrix clause as illustrated by (\ref{ex:kAmtsWG.P}) and (\ref{ex:kAmtsWG.pjArYo}) above, their relativization patterns are different. The subject of the infinitive clause is relativized by means of the subject participle \forme{pɯ-kɯ-rɲo}, while the object is relativized by using either a finite relative clause or the object participle \forme{pɯ-kɤ-rɲo} (§\ref{sec:out.complement.relativization.tr}). 

\subsubsection{Phasal verbs} \label{sec:phasal.complements}

Phasal verbs such as  \japhug{ʑa}{begin} and  \japhug{stʰɯt}{finish} are most often attested with dental or bare infinitives (§\ref{sec:bare.inf.dental.complementary}) as in (\ref{ex:tasthWt}).  They can also take  velar infinitives as in (\ref{ex:nasthWt}).

\begin{exe}
\ex \label{ex:tasthWt}
\gll [nɯ ɯ-ti] ta-stʰɯt  \\
\textsc{dem} \textsc{3sg}.\textsc{poss}-\textsc{bare}.\textsc{inf}:say \textsc{aor}:3\flobv{}-finish \\
\glt `When she finished saying that...' (150818 muzhi guniang-zh)
\japhdoi{0006334\#S122}
\end{exe}
 

\begin{exe}
\ex \label{ex:nasthWt}
\gll [tɯ-nɯ kɤ-jtsʰi] na-stʰɯt tɕe tɕe tɤ-pɤtso nɯ li ɯ-sta nɯtɕu ko-ɕɯ-rŋgɯ	\\
\textsc{indef}.\textsc{poss}-breast \textsc{inf}-give.to.drink \textsc{aor}:3\flobv{}-finish \textsc{lnk} \textsc{lnk} \textsc{indef}.\textsc{poss}-child \textsc{dem} again \textsc{3sg}.\textsc{poss}-bed \textsc{dem}:\textsc{loc} \textsc{ifr}-\textsc{caus}-lay \\
\glt `After she had finished breastfeeding, she put back the child on his bed.' (140429 jiedi-zh)
\end{exe}

In addition, some phasal verbs are also attested with finite complements sharing the same TAME and person indexation (§\ref{sec:svc.finite.agreement}), though this construction is considerably rarer.

\begin{exe}
\ex \label{ex:pasthWt}
\gll [nɯra pa-βzjoz] pa-stʰɯt tɕe ɯ-sloχpɯn nɯ kɯ taqaβ tɯ-ldʑa ɲɤ́-wɣ-mbi \\
 \textsc{dem}:\textsc{pl} \textsc{aor}:3\flobv{}-learn \textsc{aor}:3\flobv{}-finish \textsc{lnk} \textsc{3sg}.\textsc{poss}-teacher \textsc{dem} \textsc{erg} needle one-\textsc{cl} \textsc{ifr}-\textsc{inv}-give  \\
\glt `When he finished learning this [craft], his teacher gave him a needle.' (140508 benling gaoqiang de si xiongdi-zh)
\japhdoi{0003935\#S95}
\end{exe} 

Unlike \japhug{rɲo}{experience} (which selects the \textsc{downwards} orientation), phasal verbs take the lexical orientation of the verb in the complement clause (§\ref{sec:orientation.raising}), \textsc{upwards} in (\ref{ex:tasthWt}) (§\ref{sec:preverb.speech}), \textsc{eastwards} in (\ref{ex:nasthWt}) (§\ref{sec:preverb.giving}) and \textsc{downwards} in (\ref{ex:pasthWt}).

\subsubsection{Imminent aspect} \label{sec:imminent.complements}
The intransitive verb \forme{rɤŋgat} means `prepare to depart' when used on its own without a subordinate clause. It also selects participial clauses like motion verbs (§\ref{sec:purposive.clause.motion.verbs}), with the meaning `be about to'. With subject participial clauses in \forme{kɯ-} as in (\ref{ex:kWsi.tArANgat}), there is coreference between the subject of \forme{rɤŋgat} and that of its subordinate clause.


\begin{exe}
\ex \label{ex:kWsi.tArANgat}
 \gll [kɯ-si] tɤ-rɤŋgat, tu-rɤru mɯ-nɯ-cʰa ri, { } nɯ ɯ-ɕɣa ra pɯ-tsʰoz ʑo. \\
 \textsc{sbj}:\textsc{pcp}-die \textsc{aor}-be.about \textsc{ipfv}-get.up \textsc{neg}-\textsc{aor}-can \textsc{lnk} {  } \textsc{dem} \textsc{3sg}.\textsc{poss}-tooth \textsc{pl} \textsc{pst}.\textsc{ipfv}-be.complete \textsc{emph} \\
 \glt `Even when she was about to die (from old age), when she was not able to get up anymore, (...) she still had all her teeth.' (27-tWCGArgu)
\japhdoi{0003708\#S24}
\end{exe}

Like motion verb, \japhug{rɤŋgat}{be about to} also takes object participial clauses in \forme{kɤ-} with coreference with the object of the clause `be about to be $X$' as in (\ref{ex:kAsat.tAkWrANgat}).

\begin{exe}
\ex \label{ex:kAsat.tAkWrANgat}
 \gll jɤ-tɯ-ari tɕe, (...)  tɯrme [kɤ-sat] tɤ-kɯ-rɤŋgat (...) ʁnɯz tɯ-mtɤm ri \\
 \textsc{aor}-2-go[II] \textsc{lnk} {  } person \textsc{sbj}:\textsc{pcp}-kill \textsc{aor}-\textsc{sbj}:\textsc{pcp}-be.about.to {  } two 2-see[III]:\textsc{fact} \textsc{lnk} \\
 \glt `When you go there, you will see two people who are about to be killed.' (140507 jinniao-zh)
\japhdoi{0003931\#S305}
\end{exe} 

The intransitive contracting verb \japhug{aɣɯɣu}{be about to} has the same meaning as \forme{rɤŋgat}, and is used in the same constructions (\ref{ex:akWRndW.tutAGWGu}), though it is slightly less frequent.

\begin{exe}
\ex \label{ex:akWRndW.tutAGWGu}
 \gll [a-kɯ-ʁndɯ] tu-tɯ-ɤɣɯɣu ɲɯ-ŋu tɕe \\
\textsc{1sg}.\textsc{poss}-\textsc{sbj}:\textsc{pcp}-hit \textsc{ipfv}-2-be.about.to \textsc{sens}-be \textsc{lnk} \\
\glt `You are about to hit me.' (2014-kWlAG)
\end{exe}

The meaning of the imminent aspect auxiliaries overlaps with that of the Proximative (§\ref{sec:proximative}, §\ref{sec:proximative.periphrastic}), as shown by examples (\ref{ex:GWCaBa.tANu}) and (\ref{ex:akWCaB.tArANgat}) from the same passage of two versions of the same story, the former (\ref{ex:GWCaBa.tANu}) with Periphrastic Proximative (§\ref{sec:proximative.periphrastic}) and the latter with \japhug{rɤŋgat}{be about to}.


\begin{exe}
\ex \label{ex:GWCaBa.tANu}
 \gll aʑo ɣɯ-ɕaβ-a tɤ-ŋu tɕe, nɤʑo χcʰa nɯ a-nɯ-tɯ-ɕtʰɯz \\
 \textsc{1sg} \textsc{inv}-catch.up:\textsc{fact}-\textsc{1sg} \textsc{aor}-be \textsc{lnk} \textsc{2sg} right \textsc{dem} \textsc{irr}-\textsc{pfv}:\textsc{west}-2-turn.towards \\
 \glt `When they are about to catch up with me, turn the [thing in your] right [hand] in [their direction].' (2011-04-smanmi)
\end{exe}

\begin{exe}
\ex \label{ex:akWCaB.tArANgat}
 \gll [aʑo a-kɯ-ɕaβ] tɤ-rɤŋgat-nɯ tɕe, nɤʑo kɯ sŋaʁ kɯmtɕʰoχsɯm a-nɯ-tɯ-ɕtʰɯz \\
 \textsc{1sg} \textsc{1sg}.\textsc{poss}-\textsc{sbj}:\textsc{pcp}-catch.up \textsc{aor}-be.about.to-\textsc{pl} \textsc{lnk}  \textsc{2sg} \textsc{erg} sorcery triratna \textsc{dem} \textsc{irr}-\textsc{pfv}:\textsc{west}-2-turn.towards \\
 \glt `When they are about to catch up with me, turn the triratna in [their direction].' (2003 smanmi)
\end{exe}



\subsubsection{Habitual aspect} \label{sec:NgrAl}
The habitual aspect auxiliary \japhug{ŋgrɤl}{be usually the case} is not compatible with infinitival complements. It only occurs with finite complements, as in (\ref{ex:tu.NgrAl}).

\begin{exe}
\ex \label{ex:tu.NgrAl}
 \gll [ɕɤr tɕe tu] ŋgrɤl, [tu-mbri] ŋgrɤl \\
night \textsc{loc} exist:\textsc{fact} be.usually.the.case:\textsc{fact} \textsc{ipfv}-call be.usually.the.case:\textsc{fact} \\
\glt `[Owls] appear, howl during the night.' (22-pGAkhW)
\japhdoi{0003594\#S19}
\end{exe}

In the Past Perfective with an Imperfective complement, it expresses a former habit `use to $X$'  (\ref{ex:chWnWXsea.pWNgrAl}).

\begin{exe}
\ex \label{ex:chWnWXsea.pWNgrAl}
 \gll [aʑo kumpɣa cʰɯ-nɯ-χse-a] pɯ-ŋgrɤl  \\
\textsc{1sg}  chicken \textsc{ipfv}-\textsc{auto}-feed[III]-\textsc{1sg} \textsc{pst}.\textsc{ipfv}-be.usually.the.case \\
\glt `I used to raise chickens (for my own sake).' (150819 kumpGa)
\japhdoi{0006388\#S68}
\end{exe}

The auxiliary \forme{ŋgrɤl}, especially in negative form, can have overtones of epistemic (\ref{ex:tWmaR.mANgrAl}) or deontic modality (\ref{ex:tukWti.mANgrAl}).

\begin{exe}
\ex \label{ex:tWmaR.mANgrAl}
 \gll nɤʑo tɯ-maʁ maka mɤ-ŋgrɤl ma, nɤʑo ma ji-zda pɯ-kɯ-rɤʑi me \\
 \textsc{2sg} 2-not.be:\textsc{fact} \textsc{neg}-be.usually.the.case:\textsc{fact} \textsc{lnk} \textsc{2sg} apart.form \textsc{1pl}.\textsc{poss}-companion \textsc{aor}-\textsc{sbj}:\textsc{pcp}-stay not.exist:\textsc{fact} \\
\glt `It is impossible that it is not you [who stole it], since there was nobody else in our vicinity.' (31-deluge)
\japhdoi{0004077\#S97}
\end{exe}

\begin{exe}
\ex \label{ex:tukWti.mANgrAl}
 \gll tɕʰeme ɯ-ɕki ``a-βɣo" tu-kɯ-ti mɤ-ŋgrɤl \\
 girl \textsc{3sg}.\textsc{poss}-\textsc{dat} \textsc{1sg}.\textsc{poss}-FB \textsc{ipfv}-\textsc{genr}-say \textsc{neg}-be.usually.the.case:\textsc{fact} \\
\glt `One cannot address a woman as `my uncle'.' (140425 kWmdza 02)
\end{exe} 

\subsection{Similative verbs}  \label{sec:similative.verb.complementation}
The verbs \japhug{fse}{be like} and  \japhug{stu}{do like} are semi-transitive (§\ref{sec:semi.transitive}) and secundative (§\ref{sec:ditransitive.secundative}), respectively. Both select as semi-object the manner in which the action is performed.

These verbs are found with velar infinitive complements (§\ref{sec:velar.infinitives.complement.clauses}), either with a demonstrative expressing the action (such as \forme{nɯ} in \ref{ex:CAmWGdW.kAlAt.nW.tustunW}) or in questions with the interrogative pronoun \japhug{tɕʰi}{what} (§\ref{sec:tChi}) as in (\ref{ex:tChi.atAfsej}). The demonstrative or the interrogative pronoun are always located closer to the similative verb than the infinitive clause.

\begin{exe}
\ex \label{ex:CAmWGdW.kAlAt.nW.tustunW}
 \gll tɕe [ɕɤmɯɣdɯ kɤ-lɤt] nɯ tu-stu-nɯ ɲɯ-ŋu \\
 \textsc{lnk} gun \textsc{inf}-release \textsc{dem} \textsc{ipfv}-do.like-\textsc{pl} \textsc{sens}-be \\
 \glt `People shoot with guns like that.' (28-CAmWGdW)
\japhdoi{0003712\#S90}
\end{exe}

 \begin{exe}
\ex \label{ex:tChi.atAfsej}
\gll  [kɤ-pʰɣo] tɕʰi a-tɤ-fse-j?    \\
\textsc{inf}-flee what \textsc{irr}-\textsc{pfv}-be.like-\textsc{1pl} \\
\glt  `How will we flee?' (2012 Norbzang)
\japhdoi{0003768\#S60}
\end{exe} 

The subject of the infinitival clause is always coreferent with that of \forme{fse} or \forme{stu}, but its object is never indexed on the matrix verb. For instance, although \japhug{fsraŋ}{protect}, `save' and \japhug{bɯwa}{carry on the back} are transitive verbs that can index first or second person objects (see for instance \ref{ex:aZo.tutafsraN}, §\ref{sec:indexation.local}), in (\ref{ex:kAfsraN.tChi.tustea}) and (\ref{ex:kAbWwa.tChi.atAste})  the \textsc{2sg} object is not indexed on the matrix verb \forme{stu}, otherwise the forms \forme{tu-ta-stu} (\textsc{ipfv}-1\fl{}2-do.like) and \forme{a-tɤ-tɯ́-wɣ-stu} (\textsc{irr}-\textsc{pfv}-2-\textsc{inv}-do.like) would be expected. These examples show that the complement clause saturates the direct object of \forme{stu}.

 \begin{exe}
\ex \label{ex:kAfsraN.tChi.tustea}
\gll [nɤʑo kɤ-fsraŋ] tɕʰi tu-ste-a? \\
\textsc{2sg} \textsc{inf}-protect what \textsc{ipfv}-do.like[III]-\textsc{1sg} \\
\glt `How [can] I save you?' (150901 dongguo xiansheng he lang-zh)
\japhdoi{0006336\#S39}
\end{exe} 

 \begin{exe}
\ex \label{ex:kAbWwa.tChi.atAste}
\gll pɣɤtɕɯ kɯ [nɤʑo kɤ-bɯwa] tɕʰi a-tɤ-ste ɕti? \\
bird \textsc{erg} \textsc{2sg} \textsc{inf}-carry.on.the.back what \textsc{irr}-\textsc{pfv}-do.like[III] be.\textsc{aff}:\textsc{fact} \\
\glt `How will the bird carry you on its back?' (2003 zrAntCWtWrme, 132
\end{exe} 


Alternatively, \forme{stu} can also index as object the object of the verb in the infinitival clause, as shown by (\ref{ex:kAfsraN.tChi.tutastu}), a sentence whose meaning is the same as that of (\ref{ex:kAfsraN.tChi.tustea}).

 \begin{exe}
\ex \label{ex:kAfsraN.tChi.tutastu}
\gll [nɤʑo kɤ-fsraŋ] tɕʰi tu-ta-stu? \\
\textsc{2sg} \textsc{inf}-protect what \textsc{ipfv}-1\fl{}2-do.like[III]  \\
\glt `How [can] I save you?' (elicitation based on \ref{ex:kAfsraN.tChi.tustea})
\end{exe} 

Deixis manner verbs are also often used in a variety of complementation strategies. First, they occur with a coordinated clause (generally in the Factual Non-Past) expressing the purpose of the action, as in( §\ref{sec:coordination.comp.str}). Second, they can take as semi-object a totalitative relative (§\ref{sec:totalitative.relatives}) as object (§\ref{sec:relative.core.arg}) as in (\ref{ex:tWtastu.tostu2}) in the meaning `do everything in the same way as $X$' (where $X$ is the transitive subject of the relative). Third, \forme{fse} and \forme{stu} are found in serial verb constructions (§\ref{sec:svc.similative.verb}).

 \begin{exe}
\ex \label{ex:tChi.tuwGstu}
\gll tɕʰi ci ʑo tú-wɣ-stu tɕe pʰɤn \\
what \textsc{indef} \textsc{emph} \textsc{ipfv}-\textsc{inv}-do.like \textsc{lnk} be.efficient:\textsc{fact} \\
\glt `What should [we] do to successfully [treat your disease]?' (2011-04-smanmi)
\end{exe} 

\begin{exe}
\ex \label{ex:tWtastu.tostu2}
\gll [tɤ-wɯ nɯ kɯ tɯ\redp{}ta-stu] ʑo to-stu \\
\textsc{indef}.\textsc{poss}-grandfather \textsc{dem} \textsc{erg} \textsc{total}\redp{}\textsc{aor}:3\flobv{}-do.like \textsc{emph} \textsc{ifr}-do.like  \\
\glt `He did everything like the old man.' (140511 xinbada-zh)
\japhdoi{0003961\#S243}
\end{exe}

\subsection{Complement-taking adjectival verbs} \label{sec:adjective.complement}
Some adjectival stative verbs like \japhug{mkʰɤz}{be expert}, `be knowledgeable' are semi-transitive (§\ref{sec:semi.transitive}) and optionally take either a noun (\ref{ex:CoNBzu.mkhAz}) or a complement clause (\ref{ex:kAtaR.mkhAztCi}) as semi-object in addition to their subject. The complement clause can be either infinitival (§\ref{sec:velar.infinitives.complement.clauses}) or finite (§\ref{sec:finite.complement}).

\begin{exe}
\ex \label{ex:CoNBzu.mkhAz}
\gll ɯ-nmaʁ jɤ-kɯ-ɣe nɯ ɕoŋβzu mkʰɤz tɕe \\
\textsc{3sg}.\textsc{poss}-husband \textsc{aor}-\textsc{sbj}:\textsc{pcp}-come[II] \textsc{dem} carpentry be.expert:\textsc{fact} \textsc{lnk} \\
\glt `Her husband who came [to live in her family] is very good at carpentry.' (14-siblings)
\japhdoi{0003508\#S250}
\end{exe}

\begin{exe}
\ex \label{ex:kAtaR.mkhAztCi}
\gll tɕiʑo rcanɯ, [kɤ-taʁ] wuma ʑo mkʰɤz-tɕi 	 \\
\textsc{1du}  \textsc{unexpected} \textsc{inf}-weave really \textsc{emph} be.expert:\textsc{fact}-\textsc{1du} \\
\glt `We are very good at weaving.' (140521 huangdi de xinzhuang-zh)
\japhdoi{0004047\#S20}
\end{exe}


Adjectives such as \japhug{ɴqa}{be difficult}, \japhug{mbat}{be easy}, which unlike \japhug{mkʰɤz}{be expert} do not have a semi-object, can take infinitival or finite complement clauses as their subject (\ref{ex:YAjRu.YWmbat}).\footnote{Note that this construction has a meaning close to that of the facilitative \forme{nɯɣɯ-} derivation (§\ref{sec:facilitative}). }

\begin{exe}
\ex \label{ex:YAjRu.YWmbat}
\gll <gang> stʰɯci mɯ́j-rko qʰe, ɲɯ-mpɯ qʰe [tu-ŋgɤɣ, ɲɯ-ɤjʁu] nɯra ɲɯ-mbat \\
steel as.much \textsc{neg}:\textsc{sens}-be.hard \textsc{lnk} \textsc{sens}-be.soft \textsc{lnk} \textsc{ipfv}-\textsc{acaus}:bend \textsc{ipfv}-be.curved  \textsc{dem}:\textsc{pl}] \textsc{sens}-be.easy \\
\glt `[Iron] is not as hard as steel, it is soft and bends easily.' (30-Com)
\japhdoi{0003736\#S41}
\end{exe}

Even in nominalized forms such as the degree nominals (§\ref{sec:degree.nominals}), these verbs retain the ability to take subject complements (§\ref{sec:degree.nominal.arguments}), as in (\ref{ex:chWwxti.WtWmbat2}).

\begin{exe}
\ex \label{ex:chWwxti.WtWmbat2}
\gll [cʰɯ-wxti] ɯ-tɯ-mbat ɲɯ-sɤre ʑo  \\
\textsc{ipfv}-be.big \textsc{3sg}.\textsc{poss}-\textsc{nmlz}:\textsc{deg}-be.easy \textsc{sens}-be.ridiculous \textsc{emph} \\
\glt `It grows extremely easily.' (25-akWzgumba)
\japhdoi{0003632\#S70}
\end{exe}

Other adjectival stative verbs selecting infinitival complements as intransitive subjects include \japhug{aɲaj}{be done quickly} (\ref{ex:kAnWrdoR.maYaj}) or \japhug{pe}{be good} (\ref{ex:kAji.mApe}).

\begin{exe}
\ex \label{ex:kAnWrdoR.maYaj}
\gll [kɤ-nɯrdoʁ] ri mɤ-aɲaj ma kɯ-ndɯ-ndɯβ ɕti \\
\textsc{inf}-pick.up also \textsc{neg}-be.quick:\textsc{fact} \textsc{lnk} \textsc{sbj}:\textsc{pcp}-\textsc{emph}\redp{}be.small be.\textsc{aff}:\textsc{fact} \\
\glt `It takes a lot of time (it is not done quickly) to pick up [the Zanthoxyllum seeds], because they are very small.' (07-tCGom)
\japhdoi{0003434\#S15}
\end{exe}

 \begin{exe}
\ex \label{ex:kAji.mApe}
\gll tɤtʂo ɯ-ŋgɯ nɯtɕu [staχpɯ kɤ-ji] mɤ-pe \\
earth.type \textsc{3sg}.\textsc{poss}-in \textsc{dem}:\textsc{loc} pea \textsc{inf}-plant \textsc{neg}-be.good:\textsc{fact} \\
\glt `Planting peas in \forme{tɤtʂo}-type earth is not good.' (25-cWXCWz)
\japhdoi{0003636\#S70}
\end{exe}


Some adjectival verb such as \japhug{ʁzɤβ}{be careful} select bare infinitive complements (§\ref{sec:bare.inf}) like their causative form (§\ref{sec:causative.manner.complement}).



  \section{Complement-taking nouns} \label{sec:complement.taking.nouns}
 Not all noun-modifying clauses should be analyzed as relative clauses: only subordinate clauses whose head noun (overt or covert) has a syntactic role in the clause (whether argument, adjunct or possessor) can be considered to be a relative.

\subsection{Complement-taking nouns and denominal verbs}  \label{sec:compl.taking.nouns.denominal}
In examples (\ref{ex:kongzhi}) and (\ref{ex:kAtAB.ftCaka}), the head noun \japhug{ftɕaka}{method}, `manner' is neither a core argument nor an adjunct. It is not possible to convert the infinitive subordinate clauses \forme{mɯ-tu-kɤ-mbro} `not become too high' and \forme{qartsɤβ kɤ-kɤ-βzu ra kɤ-tɤβ} `thresh the (grains) that have been harvested' into independent sentences that would include \forme{ftɕaka}. These clauses  should therefore be analyzed as adnominal complement clauses, rather than a prenominal relatives.
  
 \begin{exe}
\ex \label{ex:kongzhi}
\gll a-<xuetang> ɯ-tɯ-mbro <kongzhi> tu-βze-a ŋu. [mɯ-tu-kɤ-mbro] \textbf{ftɕaka} tu-βze-a ŋu. \\
  \textsc{1sg}.\textsc{poss}-blood.sugar \textsc{3sg}.\textsc{poss}-\textsc{nmlz}:\textsc{deg}-be.high control \textsc{ipfv}-do[III]-\textsc{1sg} be:\textsc{fact} \textsc{neg}-\textsc{ipfv}-\textsc{inf}-be.high manner \textsc{ipfv}-do[III]-\textsc{1sg} be:\textsc{fact} \\
\glt `I control my blood sugar, I do what I can to prevent it from being too high'. (conversation, 15-12-05)
\end{exe}
  
 \begin{exe}
\ex \label{ex:kAtAB.ftCaka}
\gll [qartsɤβ kɤ-kɤ-βzu ra kɤ-tɤβ] \textbf{ftɕaka} ɣɯ-βzu ra   \\
 harvest \textsc{aor}-\textsc{obj}:\textsc{pcp}-make \textsc{pl} \textsc{inf}-thresh manner \textsc{inv}-make be.needed:\textsc{fact} \\
 \glt `Then one has to prepare to thresh the [grains] that have been harvested.' (2010.10)
\end{exe}
 
The construction exemplified by (\ref{ex:kongzhi}) and (\ref{ex:kAtAB.ftCaka}) is a collocation combining the com\-ple\-ment-taking noun \forme{ftɕaka} with the verb \japhug{βzu}{make}, meaning either  `do by any means possible' (as in \ref{ex:kongzhi}) or `prepare to $X$' ( \ref{ex:kAtAB.ftCaka}). In the latter meaning, the collocation is  homonymous with the transitive denominal verb \japhug{nɯftɕaka}{prepare} (§\ref{sec:denom.tr.nW}), which also selects velar infinitive complements, as shown by (\ref{ex:kAtAB.kuwGnWftCaka}), uttered just before (\ref{ex:kAtAB.ftCaka}) in the same recording.

 \begin{exe}
\ex \label{ex:kAtAB.kuwGnWftCaka}
\gll [kɤ-tɤβ] kú-wɣ-nɯftɕaka ra \\
\textsc{inf}-thresh \textsc{ipfv}-\textsc{inv}-prepare be.needed:\textsc{fact} \\
\glt `One has to prepare the threshing.'  (2010.10)
\end{exe}

The infinitive complement \forme{kɤ-tɤβ} in (\ref{ex:kAtAB.kuwGnWftCaka}) is exactly parallel to the adnominal complement clause in (\ref{ex:kAtAB.ftCaka}). This parallelism between com\-ple\-ment-taking noun and the com\-ple\-ment-taking denominal verb derived from it is not found in all the cases (§\ref{sec:complement.taking.noun.list}), but confirms the observation that the adnominal clause in (\ref{ex:kAtAB.ftCaka}) is not a prenominal relative.

\subsection{Adnominal complement clause and possessor} \label{sec:complement.taking.noun.possessor}
Among com\-ple\-ment-taking nouns, some like \japhug{ftɕaka}{method}, `manner' (§\ref{sec:compl.taking.nouns.denominal}, §\ref{sec:nouns.manner.complement}) and \japhug{kowa}{manner} (§\ref{sec:nouns.manner.complement}) are alienably possessed nouns, but most (§\ref{sec:complement.taking.noun.list}) are inalienably possessed (§\ref{sec:complement.taking.IPN}).

Two categories must be distinguished among these inalienably possessed nouns. First, some nouns such as \japhug{ɯ-skɤt}{language}, `sound' (§\ref{sec:nouns.speech.complement}) always take a \textsc{3sg} possessive prefix coreferent with the complement clause (treated as possessor of the noun), as in (\ref{ex:amAjAtWGinW.WskAt}).

\begin{exe}
\ex \label{ex:amAjAtWGinW.WskAt}
\gll ji-wɯ kɯ ``a-mɤ-jɤ-tɯ-ɣi-nɯ'' \textbf{ɯ-skɤt} to-βzu ɕti \\
\textsc{1pl}.\textsc{poss}-grandfather \textsc{erg} \textsc{irr}-\textsc{neg}-\textsc{pfv}-2-come-\textsc{pl} \textsc{3sg}.\textsc{poss}-language \textsc{ifr}-make be.\textsc{aff}:\textsc{fact} \\
\glt `(By these words) our father-in-law means that he does not want us to come back' (he means: `don't you come back') (2005 tAwakWcqraR)
\end{exe}

Other nouns like \japhug{ɯ-sɯm}{mind} (§\ref{sec:nouns.cognition.complement}) on the other hand select as possessor the experiencer, which may not be \textsc{3sg}, and the complement clause is not syntactically a possessor. For instance, in (\ref{ex:tutWndAm.nAsWm.WCe}) the possessor of \forme{-sɯm} is \textsc{2sg}, not \textsc{3sg} as would be expected if the finite clause \forme{nɯ-βdaʁmu nɯ tu-tɯ-ndɤm} were the possessor of this noun.

 \begin{exe}
\ex \label{ex:tutWndAm.nAsWm.WCe}
\gll [nɯ-βdaʁmu nɯ tu-tɯ-ndɤm] \textbf{nɤ-sɯm} ɯ́-ɕe? \\
\textsc{3pl}.\textsc{poss}-queen \textsc{dem} \textsc{ipfv}-2-take[III] \textsc{2sg}.\textsc{poss}-mind \textsc{qu}-go:\textsc{fact} \\
\glt `Do you want to become their queen?' (150818 muzhi guniang-zh)
\japhdoi{0006334\#S472}
\end{exe}

\subsection{Overview of com\-ple\-ment-taking nouns} \label{sec:complement.taking.noun.list}
Complement-taking nouns selecting infinitive (§\ref{sec:velar.infinitives.complement.clauses}), finite (§\ref{sec:finite.complement}) or reported speech complements (§\ref{sec:reported.speech}) can be divided into three semantic groups.

\subsubsection{Nouns of manner} \label{sec:nouns.manner.complement}
The alienably possessed nouns  \japhug{ftɕaka}{method}, `manner' (§\ref{sec:compl.taking.nouns.denominal}) and \japhug{kowa}{manner},\footnote{These nouns are both translated into Chinese as \ch{办法}{bànfǎ}{method}, `means', `manner'.} both borrowed from Tibetan (from \tibet{བཅའ་ཀ}{btɕa.ka}{implement} and \tibet{བཀོད་པ}{bkod.pa}{arrangement}, `method', respectively), occur in collocation with the verb \japhug{βzu}{make} (§\ref{sec:Bzu.lv}) in the meanings `try to $X$ by any means' (in Chinese \ch{想尽办法}{xiǎng jìn bànfǎ}{try to do by any means}) or `prepare'. They can select an infinitive clause with raising of the person indexation on the main verb, as shown by the 3\fl{}\textsc{2sg} form \forme{tɯ́-wɣ-βzu} `they will $X$ you' in (\ref{ex:kAndza.kowa.tuwGBzu}). 
   

 \begin{exe}
\ex \label{ex:kAndza.kowa.tuwGBzu}
\gll  a-rɟit ra nɯ-ɣi-nɯ ɕti tɕetʰa, [kɤ-ndza] \textbf{kowa} tɯ́-wɣ-βzu ɕti tɕe ku-ta-sɯ-ɤnbaʁ ŋu \\
\textsc{1sg}.\textsc{poss}-offspring \textsc{pl} \textsc{vert}-come:\textsc{fact}-\textsc{pl} be.\textsc{aff}:\textsc{fact} soon \textsc{inf}-eat manner 2-\textsc{inv}-make:\textsc{fact} be.\textsc{aff}:\textsc{fact} \textsc{lnk} \textsc{ipfv}-1\fl{}2-\textsc{caus}-hide be:\textsc{fact} \\
\glt `My children are coming back home soon, and they will try to eat you, let me hide you.' (2012 Norbzang)
\japhdoi{0003768\#S258}
\end{exe}

The com\-ple\-ment-taking transitive denominal verbs \japhug{nɯftɕaka}{prepare} and \japhug{nɯkowa}{prepare} (§\ref{sec:compl.taking.nouns.denominal}) can be derived from these two nouns.
 
\subsubsection{Nouns of speech and sound} \label{sec:nouns.speech.complement}
The inalienably possessed nouns related to speech and noise \japhug{ɯ-ti}{way of saying}, `wording', `expression', \japhug{ɯ-fɕɤt}{story}, \japhug{ɯ-skɤt}{language}, `sound', \japhug{tɤ-zgra}{sound}, `noise' and \japhug{tɯ-tɕʰa}{information, news} (about someone) (§\ref{sec:biactantial.ipn}), and the alienably possessed \japhug{kʰɤcɤl}{discussion} can occur with finite complement clauses (\ref{ex:WfCAt.tu}, \ref{ex:YWnWqambWmbjom.Wzgra}) (see also \ref{ex:rCWB4.pjWlAt}, §\ref{sec:ideo.X}) or reported speech complements (\ref{ex:ra.WskAt.ra.toBzu}).

\begin{exe}
   \ex  \label{ex:WfCAt.tu}
\gll  tɕeri [zlawiɕɤrɤβ kɯ tɕʰoz pɯ-asɯ-zgrɯβ] \textbf{ɯ-fɕɤt} tu ma [jɯm pɯ-asɯ-ɕar] \textbf{ɯ-fɕɤt} me \\
 but Zlaba.shesrab \textsc{erg} religion \textsc{pst}.\textsc{ipfv}-\textsc{prog}-accomplish  \textsc{3sg}.\textsc{poss}-story
exist:\textsc{fact} but wife:\textsc{hon} \textsc{pst}.\textsc{ipfv}-\textsc{prog}-search  \textsc{3sg}.\textsc{poss}-story not.exist:\textsc{fact} \\
\glt `People say that Zlaba shesrab was studying religion, not looking for a wife.'  (sras 79-80)
\end{exe}

\begin{exe}
   \ex  \label{ex:YWnWqambWmbjom.Wzgra}
\gll [ɲɯ-nɯqambɯmbjom] \textbf{ɯ-zgra} nɯ ``vɯrwɯrwɯr'' tu-ti ŋgrɤl. \\
\textsc{ipfv}-fly \textsc{3sg}.\textsc{poss}-noise \textsc{dem} \textsc{onom} \textsc{ipfv}-say be.usually.the.case:\textsc{fact} \\
\glt `The sound it makes when it flies is `vrvr' (as its flying sound, it says `vrvr').' (26-quspunmbro)
\japhdoi{0003684\#S13}
\end{exe}
 

As shown by (\ref{ex:WfCAt.tu}) and (\ref{ex:YWnWqambWmbjom.Wzgra}), \japhug{ɯ-fɕɤt}{story} and \japhug{tɤ-zgra}{sound} can take complements even without a noun-verb collocation. 

The noun \japhug{ɯ-skɤt}{language}, `sound' mainly takes complements when occurring with \japhug{βzu}{make} (§\ref{sec:Bzu.lv}) or \japhug{stu}{do like}. The collocation with this verbs means `do/say something that means $X$' as in (\ref{ex:ra.WskAt.ra.toBzu})\footnote{The poignant anecdote in (\ref{ex:ra.WskAt.ra.toBzu}) shows that the collocations \forme{ɯ-skɤt+βzu} and \forme{ɯ-skɤt+stu} are used even when the expression is not linguistic but based on gesture and facial expression. The hunter did not have the heart to shoot the monkey mother (see example \ref{ex:mWpjAcha.khi} in §\ref{sec:fsp.hearsay}). } or (\ref{ex:amAjAtWGinW.WskAt}) (in §\ref{sec:complement.taking.noun.possessor} above).

\begin{exe}
   \ex  \label{ex:ra.WskAt.ra.toBzu}
\gll  [wortɕʰi ʑo pɤjkʰu ma-tɤ-tɯ-lɤt tɕe, nɤkinɯ, pɤjkʰu a-pɯ tɯ-nɯ ɲɯ-jtsʰi-a ra] \textbf{ɯ-skɤt} ra to-βzu tɕe [...] tɕe nɯnɯ nɯ-jɤɣ tɕe tɕendɤre ɯ-pɯ nɯ ki kɯ-fse ɲɤ-ɣɤntaβ,
tɕe [tɤ-lɤt jɤɣ] \textbf{ɯ-skɤt} kɯra to-stu \\
please \textsc{emph} yet \textsc{neg}-\textsc{imp}-2-release \textsc{lnk} \textsc{filler} yet \textsc{1sg}.\textsc{poss}-young \textsc{indef}.\textsc{poss}-breast \textsc{ipfv}-give.to.drink-\textsc{1sg} be.needed:\textsc{fact} \textsc{3sg}.\textsc{poss}-speech \textsc{pl} \textsc{ifr}-make \textsc{lnk} {  } \textsc{lnk} \textsc{dem} \textsc{aor}-finish \textsc{lnk} \textsc{lnk} \textsc{3sg}.\textsc{poss}-young \textsc{dem} \textsc{dem}.\textsc{prox} \textsc{sbj}:\textsc{pcp}-be.like \textsc{ifr}-put \textsc{lnk} \textsc{imp}-release be.allowed:\textsc{fact} \textsc{3sg}.\textsc{poss}-speech \textsc{dem}.\textsc{prox}:\textsc{pl} \textsc{ifr}-do.like \\
\glt `[The monkey mother] made [a sign to the hunter] meaning `Please don't shoot yet, I still have to nurse my baby.' (...) and once she had finished, she put her baby aside like that, and made a sign meaning `(now) you can shoot'.' (19-GzW)
\japhdoi{0003536\#S69}
\end{exe}

The noun \japhug{tɯ-tɕʰa}{information, news} takes reported speech clauses when used in collocation with \japhug{ɣɯt}{bring} in the meaning `(go somewhere) and come back to tell about $X$', as in (\ref{ex:WtCha.ajAtWGWt}) (see also \ref{ex:WtCha.pjWGWta}, §\ref{sec:biactantial.ipn}). This construction is also compatible with participial complementation strategies  (§\ref{sec:relative.clause.compl.strategy}).

\begin{exe}
\ex  \label{ex:WtCha.ajAtWGWt}
\gll <donggua> cʰo <qiezi> ni tɕʰi ʑo mɯ́j-naχtɕɯɣ] ɣɯ ɯ-tɕʰa a-jɤ-tɯ-ɣɯt ra \\
gourd \textsc{comit} eggplant \textsc{du} what \textsc{emph} \textsc{neg}:\textsc{sens}-be.the.same \textsc{gen} \textsc{3sg}.\textsc{poss}-information \textsc{irr}-\textsc{pfv}-2-bring need:\textsc{fact} \\
\glt `[Go there and come back to] to tell me how gourds and eggplants are different.' (yici bi yici you jinbu-zh)
\end{exe}

Among the nouns of speech and sound, \japhug{ɯ-ti}{way of saying}  and \japhug{ɯ-fɕɤt}{story} are bare action nominals (§\ref{sec:bare.action.nominals}) deriving from \japhug{ti}{say} and \japhug{fɕɤt}{tell}, respectively. The rest were borrowed from Tibetan as nouns; their sources are \tibet{སྐད}{skad}{language}, \tibet{སྒྲ}{sgra}{sound} and \tibet{ཆ}{tɕʰa}{part}, respectively.

\subsubsection{Nouns of cognition} \label{sec:nouns.cognition.complement}
The noun of cognition \japhug{tɯ-sɯm}{mind} occur in collocation with the motion verb \japhug{ɕe}{go} in the meaning `want to $X$' (§\ref{sec:motion.light.verbs}) with either finite clause as in (\ref{ex:jutWCe.asWm.mWjCe}) and (\ref{ex:CkunWrNgWa.asWm}) (see also \ref{ex:tutWndAm.nAsWm.WCe} in §\ref{sec:complement.taking.noun.possessor}) or an infinitival one as in (\ref{ex:kAti.WsWm.mWpjACe})(\ref{ex:CWkArNgW.asWm}).


\begin{exe}
\ex  \label{ex:jutWCe.asWm.mWjCe}
\gll [[nɤʑo ju-tɯ-ɕe] a-sɯm mɯ́j-ɕe]] ndʐa ŋu \\
\textsc{2sg} \textsc{ipfv}-2-go \textsc{1sg}.\textsc{poss}-mind \textsc{neg}:\textsc{sens}-go reason be:\textsc{fact} \\
\glt `This is because I don't want you to leave.' (140506 shizi he huichang de bailingniao-zh)
\japhdoi{0003927\#S68}
\end{exe}

\begin{exe}
\ex  \label{ex:kAti.WsWm.mWpjACe}
\gll [``ɣa" kɤ-ti] ɯ-sɯm mɯ-pjɤ-ɕe \\
yes \textsc{inf}-say \textsc{3sg}.\textsc{poss}-mind \textsc{neg}-\textsc{ifr}.\textsc{ipfv}-go \\
\glt `He did not want to say `yes'.' (140506 shizi he huichang de bailingniao-zh)
\japhdoi{0003927\#S56}
\end{exe}

The possessive prefix on the com\-ple\-ment-taking noun encodes the experiencer, while the motion verb \japhug{ɕe}{go} remains in \textsc{3sg} form. When the complement clause is an infinitival clause, its subject must be coreferent with the experiencer in the main clause as in (\ref{ex:CWkArNgW.asWm}) and (\ref{ex:kAti.WsWm.mWpjACe}). However, when the complement clause is finite, subject corefence is possible (\ref{ex:CkunWrNgWa.asWm}, \ref{ex:tutWndAm.nAsWm.WCe}) but not required (\ref{ex:jutWCe.asWm.mWjCe}).

\begin{exe}
\ex 
\begin{xlist}
\ex  \label{ex:CkunWrNgWa.asWm}
\gll [ɕ-ku-nɯ-rŋgɯ-a] a-sɯm mɯ́j-ɕe \\
\textsc{tral}-\textsc{ipfv}-\textsc{auto}-lie.down-\textsc{1sg} \textsc{1sg}.\textsc{poss}-mind \textsc{neg}:\textsc{sens}-go \\
\ex  \label{ex:CWkArNgW.asWm}
\gll [ɕɯ-kɤ-rŋgɯ] a-sɯm mɯ́j-ɕe \\
\textsc{tral}-\textsc{inf}-lie.down \textsc{1sg}.\textsc{poss}-mind \textsc{neg}:\textsc{sens}-go \\
\end{xlist}
\glt `I don't want to go to sleep.' (elicited)
\end{exe}

The verb \japhug{ɕe}{go} in this construction is never found in the Aorist or the Inferential. To express an inchoative meaning, the motion verb \japhug{ɣi}{come} occurs instead with \forme{tɯ-sɯm}, as in (\ref{ex:kAnWCe.WsWm.toGi}).

\begin{exe}
\ex  \label{ex:kAnWCe.WsWm.toGi}
\gll daltsɯtsa tɕe tɕe kɤ-nɯ-ɕe tsa ɯ-sɯm to-ɣi.  \\
slowly \textsc{lnk} \textsc{lnk} \textsc{inf}-\textsc{vert}-go a.little \textsc{3sg}.\textsc{poss}-mind \textsc{ifr}-come \\
\glt `He slowly started thinking of going home.' (150907 laoshandaoshi-zh)
\japhdoi{0006398\#S53}
\end{exe}

Like \forme{tɯ-sɯm}, the noun \japhug{ɯ-ʁjiz}{wish} (a fossil \forme{-z} nominalization, §\ref{sec:z.nmlz}) occurs in collocation with \japhug{ɣi}{come} (§\ref{sec:motion.light.verbs}). This collocation, which means `feel like/want to $X$', has the same morphosyntactic properties as those described above: it is compatible with both infinitive complement clauses (\ref{ex:kAnAma.tWRjiz}) and finite ones (\ref{ex:Cture.WRjiz}).

\begin{exe}
\ex  \label{ex:kAnAma.tWRjiz}
\gll [kɤ-nɤma] tɯ-ʁjiz maka mɤ-ɣi. \\
\textsc{inf}-work \textsc{genr}.\textsc{poss}-wish at.all \textsc{neg}-come[III]:\textsc{fact} \\
\glt `(When one is sick), one does not feel like working.' (27-tWfCAl)
\japhdoi{0003710\#S17}
\end{exe}

\begin{exe}
\ex  \label{ex:Cture.WRjiz}
\gll  iɕqʰa [χpɯn kɯ-tsʰu nɯ kɯ tɯ-ci ɕ-tu-re] nɯ ɯ-ʁjiz mɯ-pjɤ-ɣi  \\
the.aforementioned monk \textsc{sbj}:\textsc{pcp}-be.fat \textsc{dem} \textsc{erg} \textsc{indef}.\textsc{poss}-water \textsc{tral}-\textsc{ipfv}:\textsc{up}-bring[III] \textsc{dem} \textsc{3sg}.\textsc{poss}-wish \textsc{neg}-\textsc{ifr}.\textsc{ipfv}-come  \\
\glt `The fat monk did not feel like (going down the river and) bringing the water [up to the monastery]. (150830 san ge heshang-zh)
\japhdoi{0006416\#S100}
\end{exe} 
 
 
\subsubsection{Time} \label{sec:free.time.complement}
The noun \japhug{ta-ʁa}{free time} can take infinitival complements, in particular when occurring in collocation with existential verbs (§\ref{sec:existential.light.verbs}) as in (\ref{ex:kACe.aRa.maNe}). 
 
\begin{exe}
\ex  \label{ex:kACe.aRa.maNe}
\gll [ʁdɯrɟɤt aʑo kɤ-ɕe] a-ʁa maŋe \\
\textsc{topo} \textsc{1sg} \textsc{inf}-go \textsc{1sg}.\textsc{poss}-free.time not.exist:\textsc{sens} \\
\glt `I don't have time to go to Gdongbrgyad.' (conversation, 2016-03-20)
\end{exe} 

 
\subsubsection{Cause} \label{sec:nouns.cause.complement}
The noun \japhug{ɯ-ndʐa}{cause}, which is used as a relator noun to indicate the cause or the beneficiary (§\ref{sec:IPN.cause}) with noun phrases, is also attested with finite and infinitive complement clauses, as in (\ref{ex:mWYWkABzu.Wndzxa}). With ergative marking, they can either serve as causal (§\ref{sec:causal.clauses}) or purposive (§\ref{sec:purposive.clauses}) subordinate clauses.


\begin{exe}
\ex  \label{ex:mWYWkABzu.Wndzxa}
\gll tɕe nɯ [rɟɤdɯm mɯ-ɲɯ-kɤ-βzu] ɯ-ndʐa kɯ tɯ-ci kɯ-mɯɕtaʁ nɯtɕu pjɯ́-wɣ-fkri tɕe tɕe rɟɤdɯm mɤ-βze \\
\textsc{lnk} \textsc{dem} lumps  \textsc{neg}-\textsc{ipfv}-\textsc{inf}-make \textsc{3sg}.\textsc{poss}-reason \textsc{erg} \textsc{indef}.\textsc{poss}-water \textsc{sbj}:\textsc{pcp}-be.cold \textsc{dem}:\textsc{loc} \textsc{ipfv}-\textsc{inv}-melt \textsc{lnk} \textsc{lnk} lumps \textsc{neg}-make[III] \\
\glt `In order to prevent lumps from forming, one melts the flour in cold water [by mixing it] and lumps do not form.' (140428 rJAdWm)
\japhdoi{0003882\#S8}
\end{exe} 

\subsection{Relative clauses as a complementation strategy} \label{sec:relative.clause.compl.strategy}
Some inalienably possessed nouns can take subordinate clauses that look like complement clauses but are in fact relatives.\footnote{A similar situation occurs with complement-taking verbs (§\ref{sec:relative.core.arg}, §\ref{sec:relative.pretence}). }

For instance, in (\ref{ex:tundze.GW.Wdi}), the clause \forme{tu-ndze} `(the thing that) he eats' is formally a headless finite object relative clause (§\ref{sec:object.relativization}). The head noun \japhug{tɤ-di}{smell} has no syntactic role in that clause and is not the relativized element. Otherwise, the construction would be nonsensical, its expected meaning being something like `the smell that he eats'. The relation between the prenominal clause and its head in (\ref{ex:tundze.GW.Wdi}) is simply the same as that between a possessor and its possessee (§\ref{sec:gen.possession}), and it is preferable not to analyze here \forme{tu-ndze} as a complement clause, since its meaning is not `the smell of him eating' (on the ambiguity between finite relative and complement clauses, see §\ref{sec:finite.relative.complement.ambiguity}).\footnote{The precise meaning of this sentence has been ascertained with Tshendzin. The corresponding passage in the Chinese text from which it has been translated is \ch{果酱味招来了一群苍蝇}{guǒjiàngwèi zhāolái le yīqún cāngyíng}{The smell of the jam attracted a swarm of flies}: the adnominal construction in (\ref{ex:tundze.GW.Wdi}) is thus not due to calquing from the original. }

\begin{exe}
\ex  \label{ex:tundze.GW.Wdi}
\gll [tu-ndze] nɯ ɣɯ ɯ-di nɯ pjɤ-mtsʰɤm-nɯ tɕe tɕe jo-ɣi-nɯ. \\
\textsc{ipfv}-eat \textsc{dem} \textsc{gen} \textsc{3sg}.\textsc{poss}-smell \textsc{dem} \textsc{ifr}-perceive-\textsc{pl} \textsc{lnk} \textsc{lnk} \textsc{ifr}-come-\textsc{pl} \\
\glt `[The flies] smelled [the jam] that he was eating and came.' (140428 yonggan de xiaocaifeng-zh)
\japhdoi{0003886\#S11}
\end{exe}

Example (\ref{ex:tumtshia.GW.WXpi}) provides a similar case with a head-internal finite relative. It is obvious here that \forme{ɯ-χpi} cannot be the relativized element (since the head noun \japhug{ɕkom}{muntjac} is overt). At the same time, if \forme{aʑo ɕkom tu-mtsʰi-a} were a complement clause, the expected meaning would be `a story about (why) I am leading this muntjac'. 

\begin{exe}
\ex  \label{ex:tumtshia.GW.WXpi}
\gll kɯki [aʑo ɕkom tu-mtsʰi-a] ki ɣɯ ɯ-χpi ci pjɯ-fɕat-a \\
\textsc{dem}.\textsc{prox} \textsc{1sg} muntjac \textsc{ipfv}-lead-\textsc{1sg} \textsc{dem}.\textsc{prox} \textsc{gen} \textsc{3sg}.\textsc{poss}-story \textsc{indef} \textsc{ipfv}-tell-\textsc{1sg} \\
\glt `I will tell (you) a story about this muntjac that I am leading.' (140512 fushang he yaomo-zh)
\japhdoi{0003967\#S75}
\end{exe}

The adnominal clauses in (\ref{ex:tundze.GW.Wdi}) and (\ref{ex:tumtshia.GW.WXpi}) are thus relative clauses used as possessors of inalienable nouns.

With non-finite clauses, given the similarity between participles and infinitives (§\ref{sec:infinitives.participles}), there are cases where deciding whether a clause is a participial relative serving as possessor or an infinitival complement clause is not trivial (on ambiguity between participial clauses and infinitival complements, see §\ref{sec:non-finite.relative.complement.ambiguity}). 

For instance, in (\ref{ex:biaozhun}), the Chinese borrowing  \ch{标准}{biāozhǔn}{criterion} takes as possessor the clause(s) \forme{tɕʰeme kɯ-pe mɤ-kɯ-pe}, whose status is ambiguous. If one analyses the form \forme{kɯ-pe} as a subject participle (§\ref{sec:subject.participles}), \forme{tɕʰeme [kɯ-pe] [mɤ-kɯ-pe]} can be seen as two post-nominal participial relative clauses in apposition sharing the same head noun, implying a translation `the criterion (distinguishing between) good and bad women'. If on the other hand \forme{kɯ-pe} is analyzed as a stative infinitive (§\ref{sec:velar.inf}), the subordinative clauses are infinitival complements, and the translation would rather be `criterion (by which one judges whether) a woman is good or bad'. In this particular case, since both syntactic analyses are meaningful, the fact that the postclausal noun is a Chinese borrowing makes it difficult to reliably use elicitation to distinguish between the two possibilities (for instance, by testing whether a dynamic intransitive verb would rather have a \forme{kɤ-} prefixed form in this context).

\begin{exe}
\ex  \label{ex:biaozhun}
\gll   	[tɕʰeme kɯ-pe mɤ-kɯ-pe] nɯ ɣɯ koŋla ʑo ɯ-<biaozhun>   ɲɯ-ŋu   \\
woman ?-be.good \textsc{neg}-?-be.good \textsc{dem} \textsc{gen} really \textsc{emph} \textsc{3sg}.\textsc{poss}-criterion \textsc{sens}-be \\
\glt `It is [one of the] criteria [by which one judges whether] a woman is good or bad.' (thaXtsa 2002)
\end{exe}

Tests can be applied to non-finite adnominal clauses occurring with \japhug{tɯ-tɕʰa}{information, news}. This noun, in addition to finite complement clauses (§\ref{sec:nouns.speech.complement}), is compatible with non-finite clauses in \forme{kɯ-} or \forme{kɤ-} as in (\ref{ex:jAkWGe.WtCha}). The fact that non-stative intransitive verbs in this construction have \forme{kɯ-} rather than \forme{kɤ-} however (\forme{jɤ-kɯ-ɣe} instead of ?\forme{jɤ-kɤ-ɣe}) shows that these clauses are participial clauses, and therefore that they should be analyzed as relative clauses like (\ref{ex:tundze.GW.Wdi}) and (\ref{ex:tumtshia.GW.WXpi}) above, despite what the meaning of this construction suggests. The constituent \forme{[a-tɕɯ tɤ-kɯ-ɣe] ɯ-tɕʰa} in (\ref{ex:jAkWGe.WtCha}) thus literally means `news about my son who came'.

\begin{exe}
\ex  \label{ex:jAkWGe.WtCha}
\gll  [a-tɕɯ jɤ-kɯ-ɣe] ɣɯ ɯ-tɕʰa ja-ɣɯt\\
\textsc{1sg}.\textsc{poss}-son \textsc{aor}-\textsc{sbj}:\textsc{pcp}-come[II]  \textsc{gen} \textsc{3sg}.\textsc{poss}-news \textsc{aor}:3\flobv{}-bring\\
 \glt `S/he came to tell [me/you] that my son came.' (elicited)
 \end{exe}



\section{Syntactic errors} \label{sec:syntactic.errors}
Example (\ref{ex:turAt.kAsthWt.CWNgW}) presents an interesting case of incorrect exchange of prefix between the verb in the complement clause and the com\-ple\-ment-taking verb. 

Since the com\-ple\-ment-taking verb \japhug{stʰɯt}{finish} is embedded in a temporal clause in \japhug{ɕɯŋgɯ}{before}, it would be expected to be in the Imperfective (§\ref{sec:precedence.CWNgW}), while its complement verb should either be in the bare infinitive or the velar infinitive as in (\ref{ex:kArAt.tusthWt.CWNgW}) (§\ref{sec:phasal.complements}).

\begin{exe}
\ex \label{ex:turAt.kAsthWt.CWNgW}
\gll $\dagger$[ɯʑo kɯ [nɯra tu-rɤt] kɤ-stʰɯt] ɕɯŋgɯ tɕe \\
\textsc{3sg} \textsc{erg} \textsc{dem}:\textsc{pl} \textsc{ipfv}-draw \textsc{inf}-finish before \textsc{lnk} \\
\glt `Before he had finished drawing it.' (160718 huashetianzu-zh)
\japhdoi{0006123\#S27}
\end{exe} 

\begin{exe}
\ex \label{ex:kArAt.tusthWt.CWNgW}
\gll [ɯʑo kɯ [nɯra kɤ-rɤt] tu-stʰɯt] ɕɯŋgɯ tɕe \\
\textsc{3sg} \textsc{erg} \textsc{dem}:\textsc{pl} \textsc{inf}-draw \textsc{ipfv}-finish before \textsc{lnk} \\
\glt Correction of (\ref{ex:turAt.kAsthWt.CWNgW})
\end{exe}   

In (\ref{ex:turAt.kAsthWt.CWNgW}) however, the com\-ple\-ment-taking verb is in Velar Infinitive form and the complement verb in the Imperfective, the opposite of the correct form (\ref{ex:kArAt.tusthWt.CWNgW}).

Tshendzin has no hesitation to recognize (\ref{ex:turAt.kAsthWt.CWNgW}) as a speech error, which is nevertheless unusual enough to deserve mention.

