\chapter{Postpositions and relator nouns} \label{chap:postpositions.relators}
Japhug is particularly poor in case-marking morphology, and grammatical relations are indicated by verbal morphology (treated in Chapter \ref{chap:indexation}) combined with postpositions and relator nouns.\footnote{Word order also plays a minor role (§\ref{sec:word.order}). } In this chapter, I first present the various possible functions of bare (absolutive) noun phrases, and then describe the uses of all postpositions and relator nouns in combination with noun phrases. Some postpositions and relator nouns also occur in various types of subordinate clauses (§\ref{sec:marked.subordinate}).
 

\section{Absolutive} \label{sec:absolutive}
In Japhug \textit{absolutive} refers to the bare form of a noun phrase, without postposition, relator noun or locative suffix. This form is used for intransitive subject, object and various other grammatical functions described in this section, in particular goals and some locative phrases, for which locative postpositions (§\ref{sec:locative}) and relator nouns (§\ref{sec:relator.location}) are optional.

\subsection{Intransitive subject} \label{sec:absolutive.S}
The only argument of morphologically intransitive and syntactically monoactantial verbs, the intransitive subject (S), is in absolutive form and is indexed on the verb, as \forme{tɕʰeme nɯra} `the women' with plural indexation in (\ref{ex:tCheme.nWra.thWstanW}). 

\begin{exe}
\ex \label{ex:tCheme.nWra.thWstanW}
\gll tɕʰeme nɯra tʰɯ-sta-nɯ  \\
woman \textsc{dem}:\textsc{pl} \textsc{aor}-wake.up-\textsc{pl} \\
\glt  `The women woke up.'  (2005 Norbzang, 123)
\end{exe}


Apparent examples of ergative \forme{kɯ} with third person intransitive subjects are due to the effect of long distance ergative marking (§\ref{sec:long.distance.kW}), due to bracketing issues between main clause and complement clause (§\ref{sec:complement.clause.case.marking}) or to errors, as in example (\ref{ex:kW.chAsta}): the intransitive \japhug{sta}{wake up} only takes an absolutive argument (as in \ref{ex:tCheme.nWra.thWstanW} above), and the ergative \forme{kɯ} is preceded and followed by pauses, reflecting the hesitation of the narrator. As an isolated sentence, the use of ergative is considered to be agrammatical in this example.

\begin{exe}
\ex \label{ex:kW.chAsta}
 \gll nɯ jamar kóʁmɯz nɯ lɯlu nɯ, /kɯ/, cʰɤ-sta. \\
 \textsc{dem} about only.then \textsc{dem} cat \textsc{dem} \textsc{erg} \textsc{ifr}-wake.up \\
 \glt (150826 shier shengxiao, 147)
\end{exe}

Some morphologically intransitive verbs have a second argument, which can be either absolutive (semi-objects §\ref{sec:semi.object} or goals §\ref{absolutive.goal}) or oblique (dative with \japhug{ru}{look at}, genitive with \japhug{ra}{be needed} and existential verbs §\ref{sec:gen.beneficiary},  comitative with \japhug{naχtɕɯɣ}{be like} §\ref{sec:comitative}), but in all these cases the argument indexed on the verb is always in absolutive form. 

The intransitive subject can only be relativized using subject participial relatives (§\ref{sec:intr.subject.relativization} ).

\subsection{Transitive subject} \label{sec:absolutive.A}
The transitive subject (A argument) is usually marked by the ergative postposition \forme{kɯ} (§\ref{sec:erg.kW}). The ergative is optional with first and second person pronouns in transitive subject function, as shown by examples such as (\ref{aZo.absolutive.A}), where \japhug{mto}{see} is a transitive verb . The use of ergative on these pronouns is however possible, especially in the case of contrastive focalization (§\ref{sec:A.kW}).

\begin{exe}
\ex \label{aZo.absolutive.A}
 \gll aʑo nɯ mɯ-pɯ-mto-t-a ri nɯ ɯ-sŋi wuma ʑo pɯ-ɣɤndʐo \\
\textsc{1sg} \textsc{dem} \textsc{neg}-\textsc{aor}-see-\textsc{tr}:\textsc{pst}-\textsc{1sg} \textsc{lnk} \textsc{dem} \textsc{3sg}.\textsc{poss}-day really \textsc{emph} \textsc{pst}.\textsc{ipfv}-be.cold  \\
 \glt `I did not see it (the eclipse), but on that day is was very cold.' (29-RmGWzWn2, 26)
\end{exe}

Third person transitive subjects are obligatorily marked with the ergative (§\ref{sec:A.kW}). Apparent counterexamples in the corpus are due to the emphatic use of third person pronouns (§\ref{sec:pronouns.emph}), or to errors involving hesitations.

\subsection{Object} \label{sec:absolutive.P}
Morphologically transitive verbs (§\ref{sec:transitivity.morphology}) take an object (O argument) in absolutive form, indexed by the verb morphology (§\ref{sec:polypersonal}). For instance \forme{tsuku tɯrme ra} `some people' in (\ref{ex:tsuku.tWrme.ra.kuwGmtsWGnW}) has no case marking, and is coreferent with the plural indexation on \forme{kú-wɣ-mtsɯɣ-nɯ} `they are stung' (with inverse marking, §\ref{sec:direct-inverse}). Only absolutive phrases can be indexed as objects; there are no verbs in Japhug indexing a postpositional phrase other than an ergatively marked transitive subjects.

\begin{exe}
\ex \label{ex:tsuku.tWrme.ra.kuwGmtsWGnW}
\gll tsuku tɯrme ra kú-wɣ-mtsɯɣ-nɯ tɕe mɯ́j-ʁdɯɣ, \\
some people \textsc{pl} \textsc{ipfv}-\textsc{inv}-bite-\textsc{pl} \textsc{lnk} \textsc{neg}.\textsc{sens}-be.serious \\
\glt `Some people, when they are stung (by bees) are fine.' (26-ndzWrnaR, 65-67)
\end{exe}

Objects can be relativized by object participial relatives (like semi-objects, §\ref{sec:semi.object}) or by finite relatives (like semi-objects, goals §\ref{absolutive.goal}  and locative phrases §\ref{absolutive.locative}).

Most transitive verbs have  a strict requirement on the semantic role of the object. However, speakers can in limited cases select an object with a different role with stylistic effect. For instance, the verb \japhug{ʑmbri}{play} (an instrument), `make noise with' normally takes a object a musical instrument or any other noise-making implement  (§\ref{sec:caus.Z}), as in (\ref{ex:taZmbri}). However, in the next sentence of the same story (\ref{ex:taZmbri2}), this verb appears with the phrase \forme{ɯ-mu cʰo ndʑi-kɤ-ndza kɤ-tsʰi kɯ-dɯ\redp{}dɤn} `a lot of food and drink for (him) and his mother' as object, a resultative construction: the direct object is not the instrument used to make the noise, but rather the boon obtained by making noise with the wish-granting \japhug{pʰantsɯt}{plate}.

\begin{exe}
\ex 
\begin{xlist}
\ex \label{ex:taZmbri}
 \gll [iɕqʰa pʰantsɯt nɯ] ta-ʑmbri ɲɯ-ŋu \\
 the.aforementioned plate \textsc{dem} \textsc{aor}:3\fl{}3'-make.noise \textsc{sens}-be \\
\glt `He made noise with the (magical wish-granting) plate (by hitting on it). (2003 tWxtsa, 134)
\ex \label{ex:taZmbri2}
 \gll [ɯ-mu cʰo ndʑi-kɤ-ndza kɤ-tsʰi kɯ-dɯ\redp{}dɤn ʑo nɯ] ta-ʑmbri nɤ, \\
 \textsc{3sg}.\textsc{poss}-mother \textsc{comit} \textsc{3du}.\textsc{poss}-\textsc{obj}:\textsc{pcp}-eat \textsc{obj}:\textsc{pcp}-drink \textsc{sbj}:\textsc{pcp}-\textsc{emph}\redp{}be.many \textsc{emph} \textsc{dem} \textsc{aor}:3\fl{}3'-make.noise \textsc{add} \\
\glt `He (made appear) a lot of food and drink for him and his mother by making noise (with the wish-granting plate).' (2003 tWxtsa, 135)
\end{xlist}
\end{exe}

\subsection{Sole argument of predicates of natural forces} \label{sec:absolutive.nature}
While some verbs referring to natural forces are intransitive (\japhug{mnu}{shake} (as an earthquake), §\ref{sec:volitional.mW}), most are morphologically transitive auxiliary verbs (mainly \japhug{lɤt}{throw}, \japhug{βzu}{make}, \japhug{ta}{put} and \japhug{rku}{put in}), used in collocation with a noun in absolutive form, as in (\ref{ex:tWmW.YAsWlAt}) and (\ref{ex:qale.taBzu}).

The presence of the progressive \forme{asɯ-} prefix in (\ref{ex:tWmW.YAsWlAt}) and of the C-type Aorist \forme{ta-} preverb (§\ref{sec:kamnyu.preverbs}) in (\ref{ex:qale.taBzu}) show that these verbs are morphologically transitive (§\ref{sec:transitivity.morphology}). The noun in these collocations is also relativized like a direct object (§\ref{sec:semi.object.relativization}).
 
\begin{exe}
\ex \label{ex:tWmW.YAsWlAt}
 \gll tɯ-mɯ ɲɯ-ɤsɯ-lɤt \\
 \textsc{indef}.\textsc{poss}-sky \textsc{sens}-\textsc{prog}-release \\
 \glt `It is raining.' (heard in context)
\end{exe}

\begin{exe}
\ex \label{ex:qale.taBzu}
 \gll ɕlaʁ ʑo cʰɯ-mda ɕɯŋgɯ tɕe tɕe qale ta-βzu tɕe tɕe cʰɯ-tʂaβ ŋu tɕe, \\
 \textsc{idph}(I):suddenly \textsc{emph} \textsc{ipfv}-be.the.time before  \textsc{lnk} \textsc{lnk} wind \textsc{aor}:3/fl{}3'-make    \textsc{lnk} \textsc{lnk} \textsc{ipfv}-cause.to.fall be:\textsc{fact} \textsc{lnk} \\
 \glt `When there is wind suddenly before (the barley or ) is fully ripe, it presses them (on the ground, and they cannot grow any more).' (25-cWXCWz, 42)
\end{exe}

However, these constructions have intransitive-like properties: no ergatively marked argument can occur, they are in dental infinitive rather than bare infinitive form when used with phasal complement-taking verbs (see \ref{ex:tWlAt.pjAZa}, §\ref{sec:dental.inf} and §\ref{sec:bare.inf.dental.complementary}), and they are relativized like intransitive subjects (§\ref{sec:dummy.subj.object.relativization}).

\subsection{Semi-object} \label{sec:semi.object}
Some morphologically intransitive verbs in Japhug can take a second absolutive argument (§\ref{sec:semi.transitive}, \citealt[4--5]{jacques16relatives},  \citealt[224]{jacques16complementation}). The verb \japhug{rga}{like} is such an example; in (\ref{ex:aZo.Ro.nW.mArgaa}) the intransitive subject \forme{paʁ ra} `pigs' and \japhug{nɯŋa}{cow} are in absolutive form, with  \textsc{3pl} indexation on the verb for the first. The second argument  \japhug{ɕirɴɢo}{Anisodus tanguticus} is topicalized. In (\ref{ex:aZo.Ro.nW.mArgaa}), the subject is topicalized, and the second argument  \japhug{nɯ}{that one} appears just before the verb, in absolutive form (in this particular case, the \textsc{1sg} indexation suffix merges with the verb stem in the Kamnyu dialect, §\ref{sec:synizesis}).\footnote{The semi-transitive \japhug{rga}{like}, though it can take an overt object as in  (\ref{ex:aZo.Ro.nW.mArgaa}), tends to be used more often with complement clauses or left-dislocated objects, the applicative \japhug{nɯrga}{like} being preferred with definite objects, §\ref{sec:applicative.promoted}.} This second argument is called \textit{semi-object}.

\begin{exe}
\ex \label{ex:paR.ra.mArganW}
\gll ɕirɴɢo nɯnɯ, paʁ ra mɤ-rga-nɯ ri, nɯŋa wuma ʑo rga.\\
Anisodus.tanguticus \textsc{dem} pig \textsc{pl} \textsc{neg}-like:\textsc{fact}-\textsc{pl} \textsc{lnk} cow really \textsc{emph} like:\textsc{fact} \\
\glt `The Anisodus tanguticus, pigs don't like it, but cows do.' (16-CWrNgo, 9)
\end{exe}

\begin{exe}
\ex \label{ex:aZo.Ro.nW.mArgaa}
\gll aʑo ʁo nɯ mɤ-rga-a \\
\textsc{1sg}  \textsc{top}.\textsc{advers} \textsc{dem} \textsc{neg}-like-\textsc{1sg} \\
\glt `But as for me, I don't like it.' 
\end{exe}

Other verbs taking semi-objects include \japhug{tso}{understand} (\ref{WYWtWtso}), \japhug{sɤŋo}{listen}, \japhug{rmi}{be called}.

\begin{exe}
\ex \label{WYWtWtso}
 \gll naŋrzoŋ ɯ-ɲɯ-tɯ-tso? \\
interior.decoration \textsc{qu}-\textsc{sens}-2-understand \\
\glt `Do you understand (the word) `interior decoration'?' (12-BzaNsa, 91)
\end{exe}

Semi-objects are relativized (§\ref{sec:semi.tr.relativization}) either with object participles in \forme{kɤ-} (§\ref{sec:object.participle.relatives}) or with finite relative clauses (§\ref{sec:finite.relatives}). Semi-objects are also found is some triactantial verbs. In particular, the theme of secundative verbs is a type of semi-object (§\ref{sec:theme.ditransitive}).

\subsection{Theme}  \label{sec:theme.ditransitive}
There are both secundative and indirective ditransitive verbs in Japhug (§\ref{sec:ditransitive}). In the case of indirective verbs, the theme is the object, while the recipient being marked with an oblique case, as in example (\ref{ex:tAtsxu.YAkho}) with the verb  \japhug{kʰo}{give, pass over}. With this type of verbs, the theme is relativized in exactly the same way as the object of a monotransitive verb (§\ref{sec:ditransitive.indirective}) and is indexed on the verb.

\begin{exe}
\ex \label{ex:tAtsxu.YAkho} 
\gll ɯ-mu ɯ-ɕki tɤtʂu nɯ ɲɤ-kʰo. \\
\textsc{3sg}.\textsc{poss}-mother \textsc{3sg}.\textsc{poss}-\textsc{dat} lamp \textsc{dem}  \textsc{ifr}-give \\
 \glt `He gave the lamp to his mother.' (140511 alading-zh, 167)
\end{exe}

With secundative verbs, the recipient is the object and appears in absolutive form, with object indexation. The theme of these verb is also in the absolutive, as \forme{a-me} `my daughter' in (\ref{ex:ame.tambi}).

\begin{exe}
\ex \label{ex:ame.tambi}
 \gll a-me ta-mbi ra \\
\textsc{1sg}.\textsc{poss}-daughter 1\fl{}2-give:\textsc{fact} be.needed:\textsc{fact} \\
\glt  `I will give you my daughter.'  (28-smAnmi, 267)
\end{exe}

The theme of secundative verbs cannot be indexed by the verb morphology. In (\ref{ex:RnWz.nWtambi}), the theme is the numeral \japhug{ʁnɯz}{two}, but dual indexation on \forme{nɯ-ta-mbi} `I gave $X$ to you' is not possible, as the form \forme{nɯ-ta-mbi-ndʑi} (\textsc{aor}-1\fl{}2-give-\textsc{du}) can only mean `I gave $X$ to both of you', the number referring to the recipient and not the theme (§\ref{sec:ditransitive.secundative}). 

\begin{exe}
\ex \label{ex:RnWz.nWtambi}
 \gll mɤ-ta-mbi ma ʁnɯz nɯ-ta-mbi ri, nɤʑo cʰɤ-tɯ-ɕɣɤz ɕti tɕe \\
 \textsc{neg}-1\fl{}2-give:\textsc{fact} \textsc{lnk} two  \textsc{aor}-1\fl{}2-give \textsc{lnk} \textsc{2sg} \textsc{ifr}-2-give.back be.\textsc{aff}:\textsc{fact} \textsc{lnk} \\
 \glt `I won't give her (my youngest daughter) to you, as I gave you two (of my daughters), and you sent them back.' (2002 qaCpa, 56)
\end{exe}

The theme of secundative verb is however relativized like an object (see §\ref{sec:object.participle.relatives}, §\ref{sec:secundative.theme.relativization}), and it can be considered as a sub-type of semi-object (§\ref{sec:semi.object}).

\subsection{Essive} \label{sec:essive.abs} 
Essive noun phrases are not arguments of the sentence, but are used to indicate `the property of fulfilling the role of an N' (\citealt[606]{creissels14functive}; \citealt[225]{jacques16complementation}).

In Japhug, bare noun phrases without any case marker can be interpreted as essive adjuncts, such as \japhug{nɤ-rʑaβ}{your wife} in (\ref{ex:nArZaB.YWkhotCi}) and \japhug{nɤ-kɯmtɕʰɯ}{your toy} (\ref{ex:nAkWmtChW.YWtambi}). These adjuncts are neither recipients (in both examples the recipient is \textsc{2sg}, encoded as object of the ditransitive verb \japhug{mbi}{give}, §\ref{sec:ditransitive.secundative}) nor themes (in \ref{ex:nArZaB.YWkhotCi} the theme is \forme{ji-me} `our daughter', and it serves as object of the indirective \japhug{kʰo}{give}, §\ref{sec:ditransitive.indirective}; in \ref{ex:nAkWmtChW.YWtambi} the theme is non-overt).

\begin{exe}
\ex \label{ex:nArZaB.YWkhotCi}
\gll  ji-me nɯ nɤ-rʑaβ ɲɯ-kʰo-tɕi je, ɲɯ-ta-mbi je to-ti\\
 \textsc{1pl}.\textsc{poss}-daughter \textsc{dem} \textsc{2sg}.\textsc{poss}-wife \textsc{ipfv}-give-\textsc{1du} \textsc{sfp} \textsc{ipfv}-1\fl{}2-give \textsc{sfp} \textsc{ifr}-say \\
 \glt `He said: `We will give you our daughter in marriage.'' (150831 laoshu jianv-zh, 58)
\end{exe} 

\begin{exe}
\ex \label{ex:nAkWmtChW.YWtambi}
\gll  nɤ-kɯmtɕʰɯ ɲɯ-ta-mbi \\
 \textsc{2sg}.\textsc{poss}-toy \textsc{ipfv}-1$\rightarrow$2-give   \\
 \glt `I give it to you as a toy.' (28-kWpAz, 180)
\end{exe}

Since essive adjuncts are formally indistinguishable from absolutive arguments such as objects and intransitive subjects, some sentences may appear to be ambiguous. In (\ref{ex:tWrme.YWpe}), the noun \japhug{tɯrme}{person} could be interpreted as the subject, and the phrase \forme{kɯki stu kɯ-xtɕi ki} as a topic.

\begin{exe}
\ex \label{ex:tWrme.YWpe}
\gll kɯki stu kɯ-xtɕi ki tɯrme ɲɯ-pe tɕe \\
\textsc{dem}.\textsc{prox} most \textsc{sbj}:\textsc{pcp}-be.small \textsc{dem}.\textsc{prox} person \textsc{sens}-be.good \textsc{lnk} \\
\glt `The smallest one is very nice as a person.' (31-deluge, 124)
\end{exe}

That \forme{tɯrme} here is not the subject can be seen if one chooses a first or second person form: in (\ref{ex:tWrme.YWtWpe}), the noun \forme{tɯrme} is still present despite \textsc{2sg} indexation on the verb. This piece of evidence demonstrates that the a²lysis as essive adjunct is the only possible one.

\begin{exe}
\ex \label{ex:tWrme.YWtWpe}
\gll  nɤʑo tɯrme wuma ɲɯ-tɯ-pe \\
\textsc{2sg} person really \textsc{sens}-2-be.good \\
\glt `You are very nice as a person.'  (elicited)
\end{exe}

%nɤʑo wuma ʑo tɯrme ɲɯ-tɯ-pe
%wuma ʑo tɕheme ɲɯ-tɯ-mpɕɤr???

 
It is common to have a related word as essive and as the subject or object which it refers to. In (\ref{ex:tWGli.Wuma.Zo.pe}), for instance, the inalienably possessed noun \japhug{tɯ-ɣli}{dung} occurs as both intransitive subject of the verb \japhug{pe}{be good} (with \textsc{3sg} possessive prefix) and as essive adjunct, in the latter in the specialized meaning `fertilizer'. Note that the essive adjunct is closer to the verb than the oblique phrase in \forme{ɯ-taʁ} (§\ref{sec:word.order.intr}).

\begin{exe}
\ex \label{ex:tWGli.Wuma.Zo.pe}
\gll tsʰɤt nɯ ɣɯ ɯ-ɣli nɯnɯ tɤ-rɤku ɯ-taʁ tɯ-ɣli wuma ʑo pe. \\
goat \textsc{dem} \textsc{gen} \textsc{3sg}.\textsc{poss}-manure \textsc{dem} \textsc{indef}.\textsc{poss}-crops \textsc{3sg}.\textsc{poss}-on \textsc{indef}.\textsc{poss}-manure really \textsc{emph} be.good:\textsc{fact} \\
\glt `Goat manure is very good as a fertilizer for the crops.' (05-qaZo, 32)
\end{exe}

 In (\ref{ex:ambro.tArkunW.ra}), the noun \japhug{mbro}{horse} occurs both in the object of \japhug{rku}{put in} (meaning here `include the dowry') and in the essive adjunct \forme{a-mbro}  `as a horse (for me)'.

\begin{exe}
\ex \label{ex:ambro.tArkunW.ra}
\gll mbro tɯ-skɤt kɯ-tso ci, a-mbro tɤ-rku-nɯ ra \\
horse c.\textsc{poss}-speech \textsc{sbj}:\textsc{pcp}-understand \textsc{indef} \textsc{1sg}.\textsc{poss}-horse \textsc{imp}-put.in-\textsc{pl} be.needed:\textsc{fact} \\
\glt `Give me a horse who understands human speech as my horse.'  (2003 kandzWsqhaj, 49)
\end{exe}

The essive adjunct can be either closer to the verb than the core argument it refers to, as in (\ref{ex:nArZaB.YWtambi}), (\ref{ex:tWGli.Wuma.Zo.pe}) and (\ref{ex:ambro.tArkunW.ra}) above, or further away as in (\ref{ex:WphW.rNWl}), where the intransitive  subject of \forme{mɤ-ra} `is not needed', \japhug{rŋɯl}{silver}, occurs between the verb and the essive \forme{ɯ-pʰɯ} `as its price'.

\begin{exe}
\ex \label{ex:WphW.rNWl}
\gll  aʑɯɣ ɯ-pʰɯ rŋɯl mɤ-ra, \\
\textsc{1sg}:\textsc{gen} \textsc{3sg}.\textsc{poss}-price silver \textsc{neg}-be.needed:\textsc{fact} \\
\glt `I don't want money as the price (for these clothes).' (140506 shizi he huichang de bailingniao-zh, 241)
\end{exe}

With the monotransitive verb \japhug{pʰɯt}{cut, pluck}, an essive adjunct can be used to express the purpose of the action as in (\ref{ex:paʁndza.YWnWphWtnW}).  

\begin{exe}
\ex \label{ex:paʁndza.YWnWphWtnW} 
\gll tsuku kɯ paʁndza ɲɯ-nɯ-pʰɯt-nɯ ɲɯ-ŋu ri,  \\
some \textsc{erg} hogwash \textsc{ipfv}-\textsc{auto}-pluck-\textsc{pl} \textsc{sens}-be \textsc{lnk}   \\
\glt `Some people cut it (the Sambucus) as hogwash.' (12-ndZiNgri, 30-31)
\end{exe}

This essive phrase sometimes occurs with the noun \japhug{ɯ-spa}{material} with this verb as in (\ref{ex:zGAmbu.Wspa}), a construction that is in the process of grammaticalizing into a purposive construction when used with a participial clause (§\ref{sec:participial.clause.essive}).  
 
\begin{exe}
\ex \label{ex:zGAmbu.Wspa}
\gll  zɣɤmbu ɯ-spa ɲɯ-nɯ-pʰɯt-nɯ ŋgrɤl  \\
broom \textsc{3sg}.\textsc{poss}-material \textsc{ipfv}-\textsc{auto}-cut-\textsc{pl} be.usually.the.case:\textsc{fact} \\
\glt `They cut it to make brooms. (=as a material for brooms.)' (140505 sWjno, 22)
\end{exe}

Some verbs, like \japhug{sɤrtsi}{consider as} (§\ref{sec:sig.caus.other.derivations}), have an essive argument (rather than adjunct), which is not indexed on the verb and receive no case marking, as shown by  (\ref{ex:tutanWsArtsi}), where the essive argument is \japhug{a-tɕɯ}{my son}.

\begin{exe}
\ex \label{ex:tutanWsArtsi}
 \gll aʑo a-tɕɯ tu-ta-nɯ-sɤrtsi ŋu \\
 \textsc{1sg} \textsc{1sg}.\textsc{poss}-son \textsc{ipfv}-1\fl{}2-\textsc{auto}-consider.as be:\textsc{fact} \\
 \glt `I consider you as my son.' (elicited)
\end{exe}


The transitive verb \japhug{wum}{gather} can also select an essive argument different from the object when used in the meaning `take as a student', as in (\ref{ex:naslama.kukWwuma}).\footnote{This expression is similar to Chinese \ch{收……为徒}{shōu ... wéitú}{take as a disciple}, where \zh{收} \forme{shōu} has the same meaning as \japhug{wum}{gather}, and where the essive is overtly marked by \zh{为} \forme{wéi}. } In this example the object is \textsc{1sg}, and the essive is the noun \forme{nɤ-slama}, with a possessive prefix coreferent with the subject. The verb \forme{wum} in this meaning requires the orientation \textsc{eastwards} in its centripetal function (see example \ref{ex:YWqAt.nA.kuwum}, §\ref{sec:centripetal.centrifugal}).

\begin{exe}
\ex \label{ex:naslama.kukWwuma}
 \gll wortɕhi ʑo aʑo nɤ-slama ku-kɯ-wum-a \\
please \textsc{emph} \textsc{1sg} \textsc{2sg}.\textsc{poss}-student \textsc{ipfv}-2\fl{}1-gather-\textsc{1sg} \\
\glt `Please take me as your student.' (150907 laoshandaoshi-zh, 40)
\end{exe}

Verb ellipsis may explain the presence of essive noun phrases with verbs that are not usually used with adjuncts of this type. For instance, in (\ref{ex:pGAtCW.tAmdzu}), the nouns \japhug{pɣɤtɕɯ}{bird} and \japhug{tɤ-mdzu}{thorn} are not objects (there \textsc{1sg} object indexation on both \forme{pɯ́-wɣ-sɯ-sat-a} and \forme{tʰɯ́-wɣ-sɯ-sat-a}) and are analyzable as essive adjuncts `me, as a bird/thorn', `while I was a bird/thorn'. \footnote{This example comes from a story where the main character was repeatedly killed by her sister, and reborn several times, first as a bird, and then as a thorn, as describe in (\ref{ex:pGAtCW.tAmdzu}), from another version of the same story. }  

\begin{exe}
\ex \label{ex:pGAtCW.tAmdzu}
\gll a-pi kɯ nɯra pɯ-fse tɕe, pɣɤtɕɯ ri pɯ́-wɣ-sɯ-sat-a,  tɤ-mdzu ri tʰɯ́-wɣ-sɯ-sat-a  \\
\textsc{1sg}.\textsc{poss}-elder.sibling \textsc{erg} \textsc{dem}:\textsc{pl} \textsc{pst}.\textsc{ipfv}-be.like \textsc{lnk} bird also \textsc{aor}-\textsc{inv}-\textsc{caus}-kill-\textsc{1sg} \textsc{indef}.\textsc{poss}-thorn also \textsc{aor}-\textsc{inv}-\textsc{caus}-kill-\textsc{1sg} \\
\glt `Your elder sister was like that, she had me killed (while I was a) bird, and had also me killed (while I was a) thorn.' (2005 Kunbzang, 403-404)
\end{exe}
 
The presence of these adjuncts in this particular context is probably due to the ellipsis of \forme{tɤ-sci-a} `I was born', an intransitive verb which takes essive noun phrases with the meaning `be reborn as, be reincarnated as' as in (\ref{ex:pGAtCW.tAscia}).
 
 \begin{exe}
\ex \label{ex:pGAtCW.tAscia}
\gll  nɯ ɯ-qʰu tɕe pɣɤtɕɯ tɤ-sci-a, nɯ ɯ-qʰu tɕe tɤ-mdzu tɤ-sci-a qʰe \\
\textsc{dem} \textsc{3sg}.\textsc{poss}-after \textsc{lnk} bird \textsc{aor}-be.born-\textsc{1sg} \textsc{dem} \textsc{3sg}.\textsc{poss}-after \textsc{lnk} \textsc{indef}.\textsc{poss}-thorn \textsc{aor}-be.born-\textsc{1sg}  \textsc{lnk} \\
 \glt `After that I was reborn as a bird, and after that I was reborn as a thorn.' (2003 Kunbzang, 454-455)
 \end{exe}

\subsection{Goal} \label{absolutive.goal}  
Motion verbs (\japhug{ɕe}{go}, \japhug{ɣi}{come}, \japhug{ɬoʁ}{come out} etc), manipulation verbs (\japhug{ɣɯt}{bring}, \japhug{tsɯm}{take away} etc) and some perception verbs (like \japhug{ru}{look at}) have arguments referring to the location towards which the action is directed. These arguments are not indexed in the verb morphology (in particular, motion verbs are intransitive, see §\ref{sec:intr.goal}). They can be marked with locative postpositions (§\ref{sec:core.locative}) or dative (§\ref{sec:dative}), but also occur in absolutive form, as \japhug{tɯrme-kʰa}{other people's house} in (\ref{ex:kha.jari}).

\begin{exe}
\ex \label{ex:kha.jari}
 \gll tɕe nɯ-mɤrʑaβ qʰe tɕe tɯrme-kʰa jɤ-ari ɕti qʰe  \\
 \textsc{lnk} \textsc{aor}-marry \textsc{lnk}  \textsc{lnk} people-house \textsc{aor}-go[II] be.\textsc{aff}:\textsc{fact} \textsc{lnk} \\
\glt `She married and went to (live at) other people's (her in laws') house.' (14-siblings, 10)
\end{exe}

Goals can be relativized (§\ref{sec:locative.relativization}) using oblique participial relatives (§\ref{sec:locative.participle.relatives}) or finite relatives (§\ref{sec:locative.relativization.finite}), but not object participial relatives except in a very restricted context (§\ref{sec:locative.relativization.object}), unlike most absolutive arguments.

\subsection{Location} \label{absolutive.locative}
In addition to goals, absolutive noun phrases expressing a static location are found in Japhug, in particular with the verb \japhug{rɤʑi}{stay} as in (\ref{ex:rqaco.pWrAzi}).  Locative postpositions can also occur, but are optional (§\ref{sec:core.locative}).

\begin{exe}
\ex \label{ex:rqaco.pWrAzi}
 \gll  tɕe pjɯ-sɤ-sɯxɕat-a pɯ-ŋu tɕe, rqaco pɯ-rɤʑi-a, tsʰuβdɯn pɯ-rɤʑi-a. \\
 \textsc{lnk} \textsc{ipfv}-\textsc{antipass}-teach-ɕ \textsc{pst}.\textsc{ipfv}-be \textsc{lnk}  \textsc{topo} \textsc{pst}.\textsc{ipfv}-stay-\textsc{1sg}  \textsc{topo} \textsc{pst}.\textsc{ipfv}-stay-\textsc{1sg}  \\
 \glt `When I was teaching, I lived in Rqakyo and in Tshobdun.' (150819 kumpGa, 2)
\end{exe}

Absolutive locative phrases are also found with some transitive verbs, as \forme{nɯ-mtʰɤɣ} `their waist' with \japhug{rtɤβ}{attach} in (\ref{ex:nɯmthAG.YWrtABnW}). 

\begin{exe}
\ex \label{ex:nɯmthAG.YWrtABnW}
 \gll tɯrme ra kɯ nɯ-mtʰɤɣ ɲɯ-rtɤβ-nɯ. \\
 people \textsc{pl} \textsc{erg} \textsc{3pl}.\textsc{poss}-waist \textsc{ipfv}-attach-\textsc{pl} \\
 \glt `People attach it (badger skin) around their waist (as a remedy for rheumatism).' (27-spjaNkW, 135)
\end{exe}

Oblique participial relatives (§\ref{sec:locative.participle.relatives}, §\ref{sec:locative.relativization.finite}) or finite relatives (§\ref{sec:locative.relativization.finite}) are used to relativize these absolutive locative phrases.

\section{Postpositions} \label{ex:postpositions}
Postposition are invariable words which necessarily follow a noun phrase, and specify the syntactic function of that noun phrase, be it core argument or adjunct. They cannot be used on their own, and at the very least require a demonstrative pronoun such as \forme{nɯ} (§\ref{sec:anaphoric.demonstrative.pro}). 

Although their uses have commonalities with relator nouns (§\ref{sec:relator.nouns}), they differ from those in lacking a possessive prefix. Postposition stacking is rare; only two cases exist: the sequence of locative postpositions \forme{tɕu zɯ} and \forme{tɕe} (§\ref{sec:core.locative}) and also the case of genitive postpositional phrases used in the meaning `the one from/of ...' (with an elided head noun), as in (\ref{ex:GW.cho}), where the genitive \forme{ɣɯ} is followed by the comitative (§\ref{sec:comitative}). 

\begin{exe}
\ex \label{ex:GW.cho}
 \gll tɕe pɤjka ɣɯ cʰo wuma ʑo naχtɕɯɣ. \\
 \textsc{lnk} pumpkin \textsc{gen} \textsc{comit} really \textsc{emph} be.similar:\textsc{fact} \\
 \glt `Its (little thorn-like things) are like those of the pumpkin.' 
 \end{exe}
 
\subsection{Independent words vs. clitics}  \label{ex:word.vs.clitic.postp}  
Since other Gyalrongologists, in particular \citet{jackson98morphology, jackson14morpho}, treat the postpositions in related languages as clitics rather than as independent words as is done in the present work, a justification of the present analysis is necessary.

In Japhug, the postpositions \japhug{kɯ}{ergative} (§\ref{sec:erg.kW}) and  \japhug{ɣɯ}{genitive} (§\ref{sec:genitive}) do have some clitic-like characteristics: they cannot be used without a preceding noun phrase (or a subordinate clause, §\ref{sec:marked.subordinate}), are unstressed, and in the case of the genitive have special irregular forms with pronouns (§\ref{sec:pronouns.gen}).

However, a pause can occur between these postpositions (\ref{ex:kW.nAmWmnW}) and the noun phrase they follow. For instance, in example (\ref{ex:kW.nAmWmnW}), a two second pause (with an inspiration) is found between the phrase \forme{nɯŋa ra} and the following ergative \forme{kɯ}. 

\begin{exe}
\ex \label{ex:kW.nAmWmnW}
\gll tɕe tɯrtsi nɯ pjɯ́-wɣ-βzu tɕe, nɯŋa ra, kɯ nɤ-mɯm-nɯ cʰo wuma ʑo ɣɯ-ɕɯ-fka-nɯ \\
\textsc{lnk} cow.food \textsc{dem} \textsc{ipfv}-\textsc{inv}-make \textsc{lnk} cow \textsc{pl} \textsc{erg} \textsc{trop}-be.tasty:\textsc{fact}-\textsc{pl} \textsc{comit} really \textsc{emph} \textsc{inv}-\textsc{caus}-be.satiated:\textsc{fact}-\textsc{pl} \\
\glt `They make cow food with flour, the cows find it tasty, and it satisfies their hunger.' (140513 tWrtsi, 15)
\end{exe}

A filler (§\ref{sec:fillers}) can even be inserted between the noun phrase and the following ergative, as in (\ref{ex:nW.nAkinW.kW}).

\begin{exe}
\ex \label{ex:nW.nAkinW.kW}
\gll  tɕendɤre iɕqʰa <xifangping> nɯ, nɤkinɯ, kɯ `ɯ-wa ɣɯ nɯnɯ tʰɯci ɯ-tɯtʂaŋ ci a-pɯ-tu ra' ntsɯ ɲɯ-sɯsɤm pjɤ-ŋu. \\
\textsc{lnk} the.aforementioned  \textsc{anthr} \textsc{dem} \textsc{filler} \textsc{erg} \textsc{3sg}.\textsc{poss}-father \textsc{gen} \textsc{dem} something \textsc{3sg}.\textsc{poss}-justice \textsc{indef} \textsc{irr}-\textsc{ipfv}-exist be.needed:\textsc{fact} always \textsc{sens}-think[III] \textsc{ifr}-be \\
\glt `Xi Fangping wanted to obtain justice for his father.' (150909 xifangping-zh, 30)
\end{exe}

Such cases are by no means exceptional; at least 54+35 examples of ergative and genitive preceded by a pause are attested in the corpus (they can be found by searching \forme{kɯ} or \forme{ɣɯ}  preceded by a comma). Most of these cases are found in sentences where the speaker hesitates, and are especially common in texts translated from Chinese.

The same is true of all postpositions studied in this section. Examples of pause between the noun phrase and the following postposition can be found for most of them, for instance (\ref{ex:nWnWtCu.zW}) for the locative \forme{zɯ} (§\ref{sec:core.locative}).

\begin{exe}
\ex \label{ex:nWnWtCu.zW}
\gll  <bageda> kɤ-ti nɯnɯtɕu, zɯ, nɤkinɯ, \\
\textsc{topo} \textsc{obj}:\textsc{pcp}-say \textsc{dem}:\textsc{loc} \textsc{loc} \textsc{filler} \\
\glt `In the (place) called Bagdad...' (140515 facaimeng-zh, 2)
\end{exe}

\subsection{Ergative} \label{sec:erg.kW}
The ergative \forme{kɯ}, like genitive \forme{ɣɯ}, is borrowed from Tibetan \citep{jacques16comparative} and shares with the Tibetan ergative the functions of marking transitive subject, instrument and cause. It has however a series of specific functions not found in Tibetic languages, such as that of comparee (§\ref{sec:comparee.kW}), distributive (§\ref{sec:distributive.kW}),  and oblique argument (§\ref{sec:oblique.kW}) marker. It is homophonous with the orientation adverb \forme{kɯ} \textsc{eastwards} (§\ref{sec:locative.adv}).

Although postpositional phrases in \forme{kɯ} have a different syntactic status depending on the various sub-functions of this marker (as can be shown with tests like relativization), \forme{kɯ} is glossed as \textsc{erg} in all cases.

The postposition \forme{kɯ} does not appear in combination with the additive focus marker \japhug{kɯnɤ}{also, even} (§\ref{sec:kWnA}), regardless of its function.

\subsubsection{Transitive subject} \label{sec:A.kW}
The core function of \forme{kɯ} is marking the subject of morphologically transitive verbs. Ergative is obligatory on third person transitive subjects as in (\ref{ex:WtCW.nW.kW}), and is agrammatical on objects and intransitive subjects, except in the case of long distance ergative (§\ref{sec:long.distance.kW}) and some semi-transitive verbs (§\ref{sec:S.kW}). Apparent counterexamples are speech errors (§\ref{sec:absolutive.S}).

\begin{exe}
\ex \label{ex:WtCW.nW.kW}
\gll tɕeri ɯ-tɕɯ nɯ kɯ nɯ pjɤ-sɯχsɤl \\
\textsc{lnk} \textsc{3sg}.\textsc{poss}-son \textsc{dem} \textsc{erg} \textsc{dem} \textsc{ifr}-recognized \\
\glt `But the son realized it (that she was a râkshasî).' (28-smAnmi, 21)
\end{exe}

Transitive subjects in the ergative most often precede the object as in (\ref{ex:WtCW.nW.kW}), but can also follow it as in (\ref{ex:nAZo.tWsqar.kW}), an example illustrating a third person inanimate (\forme{tɯsqar kɯ} `tsampa') acting on first/second person (§\ref{sec:indexation.mixed}).

\begin{exe}
\ex \label{ex:nAZo.tWsqar.kW}
\gll nɤʑo tɯsqar kɯ nɤki nɯ tʰɯ-tɯ́-wɣ-stu ɕti tɕe,  \\
\textsc{2sg} tsampa \textsc{erg} \textsc{dem}.\textsc{medial} \textsc{dem} \textsc{aor}-2-\textsc{inv}-do.like be.\textsc{aff}:\textsc{fact} \textsc{lnk} \\
\glt `Tsampa made you the (way you are now) = You grew that big by eating tsampa.' (2011-07-tWsqar, 4)
\end{exe}

While ergative is obligatory on third persons, it is optional on first and second person pronouns, as shown by example (\ref{ex:aZo.tAmYota}) where \japhug{aʑo}{\textsc{1sg}} is in absolutive form with the transitive verb \japhug{mɲo}{prepare}. 

\begin{exe}
\ex \label{ex:aZo.tAmYota}
\gll maʁ nɤ, aʑo tɤ-mɲo-t-a, kɯki kɯ tɕi mɤ-βze rca!  \\
not.be:\textsc{fact} \textsc{sfp} \textsc{1sg} \textsc{aor}-prepare-\textsc{pst}:\textsc{tr}-\textsc{1sg} \textsc{dem}.\textsc{prox} \textsc{erg} also \textsc{neg}-make[III]:\textsc{fact} \textsc{sfp} \\
\glt `No, it is I who prepared (our lunch), she does not do it.' (Answer to the question 'Did she made (your lunch?), conversation 140510)
\end{exe}

Using the ergative on a first  or second person pronoun in transitive subject function is however never impossible, as in (\ref{ex:WZo.kW.aZo.kW}), and more common in the case of contrastive focus (§\ref{sec:focalization.overt}).

\begin{exe}
\ex \label{ex:WZo.kW.aZo.kW}
\gll nɯ kɯ-fse tɕe tɤscoz ɯʑo kɯ ɲɯ-sɯ-ɣɯt, aʑo kɯ ku-sɯ-tsɯm-a tɕe, \\
\textsc{dem} \textsc{sbj}:\textsc{pcp}-be.like \textsc{lnk} letter \textsc{3sg} \textsc{erg} \textsc{ipfv}:\textsc{west}-\textsc{caus}-bring \textsc{1sg} \textsc{erg} \textsc{ipfv}:\textsc{east}-\textsc{caus}-take.away-\textsc{1sg} \textsc{lnk} \\
\glt `And like that, she sent me letters (by mail), and I sent her letters.' (12-BzaNsa, 26)
\end{exe}

Transitive subjects are relativized using subject \forme{kɯ-} participial relative clauses, the participle taking a possessive prefix coreferent with the object (§\ref{sec:subject.participle.possessive}). 

\subsubsection{Long distance ergative} \label{sec:long.distance.kW}
Postpositional phrases in \forme{kɯ} referring to the subject of a transitive verb can be stranded from their verb by another clause with an intransitive verb. 

In (\ref{ex:long.erg2}), for instance, the clause \forme{nɯ ma ɯ-kɤpa pjɤ-me qʰe}  `she had no other way' separates the subject \forme{tɤɕime nɯ kɯ} `the princess' from the main verb \forme{to-ti} `she said'; note the presence of a pause and of the filler \forme{nɤkinɯ} after the transitive subject. The transitive subject here also happens to be coreferent with the possessor of \forme{ɯ-kɤpa} `her method, her way' in the standing clause, resulting in a surface case mismatch.

\begin{exe}
\ex \label{ex:long.erg2}
\gll  tɕendɤre tɤɕime nɯ kɯ, \textbf{nɤkinɯ}, \textbf{nɯ} \textbf{ma} \textbf{ɯ-kɤpa} \textbf{pjɤ-me} \textbf{qʰe} `jɤɣ jɤɣ jɤɣ' 	to-ti ɲɯ-ŋu. \\
\textsc{lnk} young.lady \textsc{dem} \textsc{erg} \textsc{filler} \textsc{dem}  apart.from \textsc{3sg}.\textsc{poss}-method \textsc{ipfv}.\textsc{ifr}-not.exist \textsc{lnk} be.possible:\textsc{fact} be.possible:\textsc{fact} be.possible:\textsc{fact} \textsc{ifr}-say \textsc{sens}-be \\
\glt `The young lady had no other way but to say ``yes, yes, yes".' (140428 mu e guniang-zh, 92)
\end{exe}

Similarly in (\ref{ex:long.erg3}), the minimal clause \forme{jo-ɣi} `he came' consisting of a single verb occurs between the subject \forme{iɕqʰa rgɤtpu nɯ kɯ} `the old man' and the rest of the main clause \forme{ɬɤndʐi nɯnɯ jo-tsʰi} `he stopped the demon'.

\begin{exe}
\ex \label{ex:long.erg3}
\gll   iɕqʰa rgɤtpu nɯ kɯ, \textbf{jo-ɣi} \textbf{tɕe}, \textbf{nɤki}, ɬɤndʐi nɯnɯ jo-tsʰi  \\
the.aforementioned old.man \textsc{dem} \textsc{erg} \textsc{ifr}-come \textsc{lnk} \textsc{filler} demon \textsc{dem} \textsc{ifr}-block \\
\glt `The old man came and stopped the demon.' (140512 fushang he yaomo1-zh, 62)
\end{exe}

Here the intransitive subject of \forme{jo-ɣi}  `he came' and the transitive subject of  \forme{jo-tsʰi}  `he blocked him' happen to be coreferent. If analyzed superficially, (\ref{ex:long.erg3}) could seem to be an example of ergative appearing on an intransitive subject. In isolation, however, without context, a clause such as $\dagger$\forme{rgɤtpu nɯ kɯ jo-ɣi} is not considered to be correct by native speakers, showing that it is preferable to analyze \forme{jo-ɣi}  as an incision in this context rather than forming a constituent with the preceding postpositional phrase in \forme{kɯ}.

\subsubsection{Intransitive subject} \label{sec:S.kW}
Genuine examples of intransitive subjects with ergative appear to be nevertheless attested at least with some semi-transitive verbs like \japhug{tso}{know, understand} as in (\ref{ex:kW.tso}). It is optional and much less common than the absolutive form.

\begin{exe}
\ex \label{ex:kW.tso}
\gll ɕɯ kɯ tso ma \\
who \textsc{erg} understand:\textsc{fact} \textsc{lnk} \\
\glt `Who would know.' (150909 xiaocui-zh, 65)
\end{exe}

Ergative marking on the subject of reflexive verbs (which are morphologically intransitive, §\ref{sec:reflexive}) is also attested for marking emphasis, with the same pronoun preceding and following the ergative (for instance \forme{tɯʑo kɯ tɯʑo} in example \ref{ex:tWZo.kW.tWZo}, §\ref{sec:genr.pro}).

\subsubsection{Instrumental} \label{sec:instr.kW}
In addition to marking the transitive subject, the postposition \forme{kɯ} occurs on instruments, as in (\ref{ex:Wlu.kW.chWsWXse}). It is possible to find examples with  postpositional phrases in \forme{kɯ}, one corresponding to the subject and the other one to the instrument, as in (\ref{ex:rJAlpu.kW.nWnW.kW}). No good examples of instruments in \forme{kɯ} are found with an intransitive main verb in the corpus (only manner or causal adjuncts are found, §\ref{sec:manner.nominal.kW}).

\begin{exe}
\ex \label{ex:Wlu.kW.chWsWXse}
 \gll ɯ-pɯ nɯ ɯ-lu kɯ cʰɯ-sɯ-χse ɲɯ-ŋu. \\
 \textsc{3sg}.\textsc{poss}-young \textsc{dem} \textsc{3sg}.\textsc{poss}-milk \textsc{erg} \textsc{ipfv}-\textsc{caus}-feed[III] \textsc{sens}-be \\
\glt `(The whale) feeds its young with milk.' (160703 jingyu, 14)
\end{exe}

\begin{exe}
\ex \label{ex:rJAlpu.kW.nWnW.kW}
 \gll  rɟɤlpu kɯ nɯnɯ kɯ ɯ-βri a-pɯ-sɯ-χtɕi ndɤre, mɤʑɯ nɤ-sɤ-scit tʰaŋ nɤ!\\
 king \textsc{erg} \textsc{dem} \textsc{erg} \textsc{3sg}.\textsc{poss}-body \textsc{irr}-\textsc{pfv}-\textsc{caus}-wash \textsc{lnk} even.more \textsc{trop}-\textsc{prop}-be.happy:\textsc{fact} \textsc{hypoth} \textsc{sfp} \\
 \glt `If the king washes his body with this, he will find it even nicer.' (140514 xizajiang he lifashi-zh, 88-89)
\end{exe}

When an instrument in \forme{kɯ} occurs in a clause, the main verb generally takes the causative prefix as in (\ref{ex:Wlu.kW.chWsWXse}) and (\ref{ex:rJAlpu.kW.nWnW.kW}), as if the instrument were a type of causee -- it differs from a causee however in that in the case of the latter the postposition \forme{kɯ} is optional (§\ref{sec:causee.kW}). Causative marking in clauses with  instruments is optional, and one can find the two constructions with or without the causative marker side by side in the same narrative, as shown by examples (\ref{ex:instr3}) and (\ref{ex:instr4}).

\begin{exe} 
\ex \label{ex:instr3}
\gll   qartsʰaz  ɯ-ndʐi kɯ cʰɯ-βzu-nɯ tɕe, nɯ stu kɯ-ʑru.   \\
 deer \textsc{3sg}.\textsc{poss}-hide \textsc{erg} \textsc{ipfv}-do-\textsc{pl} \textsc{lnk} \textsc{dem} most \textsc{sbj}:\textsc{pcp}-precious \\
 \glt   `They make (shoes) with deer hide, it is the most precious (type of skin).' (30 mboR, 48)
\end{exe} 

 \begin{exe} 
\ex \label{ex:instr4}
\gll   qartsʰaz ɯ-ndʐi ʁɟa kɯ ʑo tʰɯ-kɤ-sɯ-βzu  \\
 deer \textsc{3sg}.\textsc{poss}-hide entirely \textsc{erg} \textsc{emph} \textsc{aor}-\textsc{obj}:\textsc{pcp}-\textsc{caus}-do\\
 \glt   `(It is) entirely made of deer hide.' (30 mboR, 53)
\end{exe} 

The instrument is mainly a concrete object, but can also refer to an entire action, as in (\ref{ex:nW.kW.ndZixtu}) where the anaphoric \japhug{nɯ}{that} refers to the actions described in the previous clauses.

\begin{exe}
\ex \label{ex:nW.kW.ndZixtu}
 \gll tɤ-tɕɯ nɯ kɯ sɲikuku ʑo si ʑ-lu-pʰɯt tɕe, nɯ ɲɯ-ntsɣe-ndʑi tɕe,  nɯ kɯ ndʑi-xtu cʰɯ-sɯ-χsu-ndʑi pɯ-ŋu ɲɯ-ŋu,  \\
 \textsc{indef}.\textsc{poss}-son \textsc{dem} \textsc{erg} every.day \textsc{emph} tree \textsc{tral}-\textsc{ipfv}-fell \textsc{lnk} \textsc{dem} \textsc{ipfv}-sell-\textsc{du} \textsc{lnk} \textsc{dem} \textsc{erg} \textsc{3du}.\textsc{poss}-belly \textsc{ipfv}-\textsc{caus}-feed-\textsc{du} \textsc{pst}.\textsc{ipfv}-be \textsc{sens}-be \\
 \glt `The son went every day to fell trees, they sold (the wood), and they fed their bellies this way.' (2003tWxtsa, 2-3)
\end{exe}

Instruments differ from transitive subjects in that they are relativized using oblique participles in \forme{sɤ(z)-} (§\ref{sec:oblique.participle}) rather than subject participles.
 

\subsubsection{Manner and cause} \label{sec:manner.nominal.kW}
Postpositional phrases in \forme{kɯ} can also describe the manner in which an action takes place, or its cause. Manner adjuncts in \forme{kɯ} are generally formed with abstract nouns (§\ref{sec:tA.abstract.nouns}) as \japhug{tɤ-mqe}{verbal fight} and \japhug{tɤ-ndɯt}{dispute}  in (\ref{ex:tAmqe.tAndWt.kW}) and \japhug{tɤŋɤm}{pain} in (\ref{ex:tANAm.kW.pjWsi}). Unlike instruments, manner adjuncts do not trigger the addition of a causative prefix on the main verb, and are fully compatible with intransitive verbs.

\begin{exe}
\ex \label{ex:tAmqe.tAndWt.kW}
 \gll kumpɣa cʰo kʰɯna ni li tɤ-mqe tɤ-ndɯt kɯ jo-ɣi-ndʑi tɕe, \\
 chicken \textsc{comit} dog du again \textsc{indef}.\textsc{poss}-verbal.fight \textsc{indef}.\textsc{poss}-dispute \textsc{erg} \textsc{ifr}-come-\textsc{du} \textsc{lnk} \\
 \glt `The chicken and the dog came fighting and arguing (with each other).' (150826 shier shengxiao-zh, 120)
\end{exe}

\begin{exe}
\ex \label{ex:tANAm.kW.pjWsi}
 \gll tɤŋɤm kɯ pjɯ-si ɲɯ-ra. \\
 pain \textsc{erg} \textsc{ipfv}-die \textsc{sens}-be.needed \\
 \glt `(The animal that is devoured alive by the lions) dies in pain.' (20-sWNgi, 48)
\end{exe}

Causal adjuncts, like manner adjuncts, also take the postposition \forme{kɯ} without causative form on the verb, as in (\ref{ex:WRrWm.kW}). The inalienably possessed noun \japhug{ɯ-ndʐa}{its cause} in particular is often used with the ergative to specify a cause  (\ref{ex:nW.Wndzxa.kW}). 

\begin{exe}
\ex \label{ex:WRrWm.kW}
 \gll   kʰa ɯ-ʁrɯm nɯ kɯ tɕe tɕe ɲɯ-ɣɤɕu \\
house \textsc{3sg}.\textsc{poss}-shade \textsc{dem} \textsc{erg} \textsc{lnk} \textsc{lnk}  \textsc{sens}-be.cool \\
\glt `Due to the shade of the buildings, (this road) is not exposed to the heat of the sun.' (conversation 2014-05-10)
\end{exe}

 \begin{exe}
\ex \label{ex:nW.Wndzxa.kW}
 \gll   qambalɯla nɯ, nɯ ɯ-ndʐa kɯ tʰa ɯβrɤ-si ma \\
 butterfly \textsc{dem} \textsc{dem} \textsc{3sg}.\textsc{poss}-cause \textsc{erg}  later \textsc{rh}.\textsc{q}-die:\textsc{fact} \textsc{sfp} \\
 \glt `The butterfly might die because of that.' (150818 muzhi guniang, 199)
\end{exe}
 

Various subordinate clauses with finite or non-finite verbs are also made with the postposition \forme{kɯ} (see §\ref{sec:causality} and §\ref{sec:rectification.clauses} for instance).

\subsubsection{Causee} \label{sec:causee.kW}
Causative verbs in \forme{sɯ(ɣ)-/z-} derived from transitive verbs have three arguments: causer, causee and object (§\ref{sec:ditransitive.causative}, §\ref{sec:sig.caus.tr}). The causer is treated as the transitive subject, and is marked with the ergative. The causee can also receive  ergative marking as \forme{qapri kɯ-ɲaʁ nɯ kɯ} `the black snake' in (\ref{ex:kW.losWqioR1}) and \forme{kɯ-wɣrum nɯ kɯ} `the white one' (\ref{ex:kW.losWqioR2}). The most common word order is to put the causee before the object as in (\ref{ex:kW.losWqioR1}), but the opposite order is also attested as in (\ref{ex:kW.losWqioR2}).

\begin{exe}
\ex \label{ex:kW.losWqioR1}
 \gll   li mdaʁʑɯɣ ci to-lɤt tɕe, tɕendɤre, nɤki, qapri kɯ-ɲaʁ nɯ kɯ kɯ-wɣrum nɯnɯ lo-sɯ-qioʁ tɕe tɕe ɲɤ-sɯ-ɤkɤlɤt. \\
 again bow one \textsc{ifr}-release \textsc{lnk} \textsc{lnk} filler snake \textsc{sbj}:\textsc{pcp}-be.black \textsc{dem} \textsc{erg} \textsc{sbj}:\textsc{pcp}-be.white \textsc{dem} \textsc{ifr}-\textsc{caus}-vomit \textsc{lnk} \textsc{lnk} \textsc{ifr}-\textsc{caus}-detach \\
 \glt `He shot an arrow and caused the black snake to vomit the white one, and separated it (from the other one).' (28-smAnmi, 106)
\end{exe}

\begin{exe}
\ex \label{ex:kW.losWqioR2}
 \gll  to-lɤt tɕe tɕendɤre nɯnɯ qapri kɯ-ɲaʁ nɯ, kɯ-wɣrum nɯ kɯ lo-sɯ-qioʁ tɕe ɲɤ-sɯ-ta. \\
  \textsc{ifr}-release \textsc{lnk} \textsc{lnk} \textsc{dem} snake \textsc{sbj}:\textsc{pcp}-be.black \textsc{dem}  \textsc{sbj}:\textsc{pcp}-be.white \textsc{dem} \textsc{erg} \textsc{ifr}-\textsc{caus}-vomit \textsc{lnk} \textsc{ifr}-\textsc{caus}-put \\
  \glt `He shot (an arrow) and caused the white one to vomit the black snake and to release it.' (28-smAnmi, 99)
\end{exe}

However, the presence of ergative on the causee is optional, as shown by example (\ref{ex:CkozRAru}) where the causee \forme{ɯ-tɕɯ stu kɯ-xtɕi nɯ} `his youngest son' is in absolutive form.

\begin{exe}
\ex \label{ex:CkozRAru}
 \gll rɟɤlpu kɯ ɯ-tɕɯ stu kɯ-xtɕi nɯ ɕ-ko-z-rɯru. \\
 king \textsc{erg} \textsc{3sg}.\textsc{poss}-son most \textsc{sbj}:\textsc{pcp}-be.small \textsc{dem} \textsc{tral}-\textsc{ifr}-\textsc{caus}-guard \\
 \glt  `The king had his youngest son (go and) guard (the tree).' (140507 jinniao-zh, 39)
\end{exe}

When the object is third person, and the causee first or second, the causee is obligatorily indexed, as in (\ref{ex:pWwGsWmtoa}) (see §\ref{sec:sig.caus.tr} for more examples), resulting in a verb form that is identical with that when both causer and causee are third person, and the object is first or second person as in (\ref{ex:kW.pWwGsWmtoa}). With such ambiguous verb forms, the presence of the ergative postposition on the causee as \forme{a-tɕɯ kɯ} `my son' in (\ref{ex:kW.pWwGsWmtoa}) is a way to disambiguate from the interpretation of the noun as object as in (\ref{ex:pWwGsWmtoa}). 

\begin{exe}
\ex
\begin{xlist}
\ex \label{ex:pWwGsWmtoa}
 \gll ɯʑo kɯ a-tɕɯ pɯ́-wɣ-sɯ-mto-a \\
\textsc{3sg} \textsc{erg} \textsc{1sg}.\textsc{poss}-son  \textsc{aor}-\textsc{inv}-\textsc{caus}-see-\textsc{1sg} \\
\glt `He let me see my son.' (elicitation)
\ex \label{ex:kW.pWwGsWmtoa}
 \gll ɯʑo kɯ a-tɕɯ kɯ pɯ́-wɣ-sɯ-mto-a \\
\textsc{3sg} \textsc{erg} \textsc{1sg}.\textsc{poss}-son \textsc{erg} \textsc{aor}-\textsc{inv}-\textsc{caus}-see-\textsc{1sg} \\
\glt `He let my son see me.' (elicitation)
\end{xlist}
\end{exe}

No minimal pair similar to (\ref{ex:pWwGsWmtoa}) and \ref{ex:kW.pWwGsWmtoa}), where the ergative on the causee has a disambiguating function, is attested in the corpus.


\subsubsection{Comparee marker} \label{sec:comparee.kW}
In the comparative construction (§\ref{sec:comparison}), in addition to the standard markers (such as \forme{sɤz} and its variants, see §\ref{sec:comparative}), the postposition \forme{kɯ} can appear on the comparee, although the comparee is syntactically an intransitive subject  (indexed on the stative adjectival predicate). 

The comparee with  \forme{kɯ} can either precede (\ref{ex:nWnW.kW.aZo.sAz}) or follow the standard, but most often this marker appears when no overt standard is present as in (\ref{ex:nW.kW.xtCi}). In all of these examples, including the last one, the marker  \forme{kɯ}  is optional.

\begin{exe}
\ex \label{ex:nWnW.kW.aZo.sAz}
 \gll mahi nɯnɯ kɯ aʑo sɤz cʰa \\ 
water.buffalo \textsc{dem} \textsc{erg} \textsc{1sg} \textsc{comp} can:\textsc{fact} \\ 
\glt `The water buffalo is stronger than me.'  (150831 laoshu jianv-zh, 59)
\end{exe}

\begin{exe}
\ex \label{ex:nW.kW.xtCi}
\gll ndʑi-tsʰɯɣa nɯra wuma naχtɕɯɣ. tɕeri tɯrgilaŋlaŋ nɯ kɯ xtɕi. \\
\textsc{2du}.\textsc{poss}-form \textsc{dem}:\textsc{pl} really be.similar:\textsc{fact} but fir.cone \textsc{dem} \textsc{erg} be.small:\textsc{fact} \\
\glt  `Their shape is similar, but the fir cone is smaller.' (08-tWrgi, 80)
\end{exe}

See \citet{jacques16comparative} for a historical hypothesis explaining how the ergative marker came to be used to mark the comparee. 

\subsubsection{Distributive} \label{sec:distributive.kW}
The postposition \forme{kɯ}, when occurring with a counted noun designating a quantity, can be used to focus on the distributive meaning (`for one X', `per'). It occurs in constructions with intransitive verbs where no agent or instrument is present, but exclusively to express the price of the quantity designated, as in  (\ref{ex:tWtWrpa.kW1}) (see additional examples in \citealt[5--6]{jacques16comparative}). It cannot be used with time counted nouns.

 \begin{exe} 
\ex \label{ex:tWtWrpa.kW1}
\gll  tɯ-tɯrpa kɯ sqi jamar ɲɯ-ra. \\
one-pound \textsc{erg} ten about \textsc{sens}-be.needed \\
\glt `You need ten (yuans) per pound (of Angelica).' (17 ndZWnW, 22)
\end{exe}  

As argued in \citet[23]{jacques16comparative}, this construction results from the elision of a verb such as \japhug{sɤndu}{exchange}, which can take as instrument (§\ref{sec:instr.kW}) the counted noun expressing a quantity, as in (\ref{ex:kW.YWwGsAndu}).


\begin{exe}
\ex \label{ex:kW.YWwGsAndu}
\gll tɯ-tɯrpa kɯ ɣurʑa jamar ɲɯ́-wɣ-sɤndu ɲɯ-kʰɯ \\
one-pound \textsc{erg} hundred about \textsc{ipfv}-\textsc{inv}-exchange \textsc{sens}-be.possible \\
\glt `One can exchange (sell) one pound for a hundred (yuans).' (elicited)
\end{exe}

\subsubsection{Partitive} \label{sec:kW.mtshAt}
The intransitive verb \japhug{mtsʰɤt}{be full} generally has a dummy subject, and selects a locative argument and a partitive argument, indicating the material / elements that the location is full of. This partitive argument is generally in a absolutive form, as in (\ref{ex:YAmtshAt}), but we also find examples with the ergative, as in (\ref{ex:kW.pjAmtshAt}).\footnote{Although (\ref{ex:kW.pjAmtshAt}) is translated from Chinese, the original has \ch{下面坐满了观众}{xiàmiàn zuòmǎn le guānzhòng}{Below, the spectator seats were filled}, and there is nothing in the structure of the original sentence that could allow to interpret the ergative as a calque. In addition, Tshendzin confirmed that the ergative is correct here. }

\begin{exe}
\ex \label{ex:YAmtshAt}
\gll tʂapa tʰamtɕɤt, nɯ-mbro nɯ-jla, nɯŋa paʁ nɯra ɲɤ-mtsʰɤt. \\
pen all \textsc{3pl}.\textsc{poss}-horse \textsc{3pl}.\textsc{poss}-hybrid.yak cow pig \textsc{dem}:\textsc{pl} \textsc{ifr}-be.full \\
\glt `All the pens had become full of horses, hybrid yak, cows and pigs.' (28-qajdoskAt, 131)
\end{exe}


\begin{exe}
\ex \label{ex:kW.pjAmtshAt}
\gll  ɯ-pa nɯtɕu rca, kɯ-nɤmɲo kɯ pjɤ-mtsʰɤt ʑo \\
\textsc{3sg}.\textsc{poss}-down \textsc{dem}:\textsc{loc} \textsc{unexp}:\textsc{deg} \textsc{sbj}:\textsc{pcp}-watch \textsc{erg} \textsc{ipfv}.\textsc{ifr}-be.full \textsc{emph} \\
\glt `Down (the stage), (the seats) were full of spectators.' (150822 yan muouxi de ren-zh, 55)
\end{exe}

Another verb with an optionally ergative partitive argument is the passive \japhug{amar}{be smeared with} (example \ref{ex:tAse.kW.pjAkAmarci}, §\ref{sec:passive.agent}).

\subsubsection{Oblique argument} \label{sec:oblique.kW}
 The transitive verb \japhug{kʰɤt}{do repeatedly}, `do for a long time' and its causative form \japhug{sɯ-kʰɤt}{cause to do repeatedly}, `cause to do for a long time' occur in a construction with instrumental-like noun phrases marked with the ergative \forme{kɯ}, indicating the action which is performed repeatedly or done over a long time. These noun phrases can include either an action nominal derived from a verb with the prefix \forme{tɯ-} (§\ref{sec:action.nominals}) as in (\ref{ex:tWqioR.kW}), or an underived action noun, as in (\ref{ex:tama.kW.takhAt}) and (\ref{ex:khAcAl.kW.takhata}).  
 
  \begin{exe}
\ex \label{ex:tWqioR.kW}
\gll tɯ-qioʁ kɯ tó-wɣ-sɯ-kʰɤt ʑo tɕe, tɕe nóʁmɯz nɤ tɯɣ nɯnɯ ló-wɣ-sɯ-tɕɤt  \\
\textsc{nmlz}:\textsc{action}-vomit \textsc{erg} \textsc{ifr}-\textsc{inv}-\textsc{caus}-do.a.long.time \textsc{emph} \textsc{lnk} \textsc{lnk} only.then \textsc{lnk} poison \textsc{dem} \textsc{ifr}-\textsc{inv}-\textsc{caus}-take.out \\
\glt `(The medicine) caused (Gesar) to vomit a long time until he expelled the poison.' (Gesar, 266)
\end{exe}

  \begin{exe}
\ex \label{ex:tama.kW.takhAt}
\gll ta-ma kɯ ta-kʰɤt ʑo  \\
\textsc{indef}.\textsc{poss}-work \textsc{erg} \textsc{aor}:3$\rightarrow$3'-do.a.long.time \textsc{emph} \\
\glt `He did a lot of work.' (elicited)
\end{exe}

Example (\ref{ex:khAcAl.kW.takhata}), with the verb \japhug{kʰɤt}{do repeatedly, do a long time}  taking \textsc{1sg}\fl{}3 indexation (§\ref{sec:indexation.mixed}), shows that the ergative phrase cannot be analyzed as a transitive subject; moreover, the fact that adding the causative in this case would imply a real causative interpretation (`cause X to repeatedly') also indicates that this phrase is not an instrumental adjunct (see §\ref{sec:instr.kW}).

  \begin{exe}
\ex \label{ex:khAcAl.kW.takhata}
\gll kʰɤcɤl kɯ tɤ-kʰat-a ʑo \\
conversation \textsc{erg} \textsc{aor}-do.a.long.time-\textsc{1sg} \textsc{emph} \\
\glt `I have a long conversation.' (elicited)
\end{exe}

No other verb takes this type of oblique ergative phrase.

\subsection{Genitive} \label{sec:genitive}
With the exception of particular forms for some pronouns (§\ref{sec:pronouns.gen}), the genitive postposition has the invariant form \forme{ɣɯ} in Kamnyu Japhug. Like the ergative \forme{kɯ}, it is likely borrowed from the Amdo clitic \forme{-ɣə/-kə} (\citealt[62]{haller04themchen}). It is used in possessive contructions, but also expresses beneficiary and recipient.

\subsubsection{Possession} \label{sec:gen.possession}
The genitive \forme{ɣɯ} occurs in various type of possessive constructions, including genitival noun complements and possessive existential predicates (§\ref{sec:possessive.mihi.est}).

Inside the noun phrase, the genitive occurs between possessor and possessum, and a possessive prefix is found on the possessum (§\ref{ex:prefix.expression.of.possession}), as in (\ref{ex:GZAndza.GW.WjwaR}).  

\begin{exe}
\ex \label{ex:GZAndza.GW.WjwaR}
\gll ri ɣʑɤndza ɣɯ ɯ-jwaʁ nɯra mɤ-wxti ri, ɲaʁ ʑo qhe, \\
\textsc{lnk} Agastache.rugosa \textsc{gen} \textsc{3sg}.\textsc{poss}-leaf \textsc{dem}:\textsc{pl} \textsc{neg}-be.big:\textsc{fact} \textsc{lnk} be.black:\textsc{fact} \textsc{emph} \textsc{lnk} \\
\glt `The leaves of the \textit{Agastache rugosa} are not large and quite dark in colour.' (11-qarGW, 137)
\end{exe}

Genitival phrases without possessive prefix on the possessum are rare but do exist, in particular when the possessum is a noun borrowed from Chinese and non-fully nativized like \ch{国语}{guóyǔ}{national language} in (\ref{ex:iZo.GW.guoyu}).  

\begin{exe}
\ex \label{ex:iZo.GW.guoyu}
\gll iʑo ɣɯ <guoyu> ɲɯ-ŋu tɕe, nɯnɯ kɤsɯfse ɣɯ ji-rju ɲɯ-ŋu tɕe, \\
\textsc{1pl} \textsc{gen} national.language \textsc{sens}-be \textsc{lnk} \textsc{dem} all \textsc{gen} \textsc{1pl}.\textsc{poss}-speech \textsc{sens}-be \textsc{lnk} \\
\glt `(Chinese) is our national language, it is the language of all of us.' (150901 tshuBdWnskAt, 15-16)
\end{exe}

For singular noun possessors, the presence or not of a third person possessive prefix \forme{ɯ-} is not always easy to tell from recordings, as due to the external sandhi (§\ref{sec:sandhi.word}), \forme{ɣɯ ɯ-} merges as \ipa{ɣɯ} when no pause occurs between the two. In careful speech, the third person prefix is clearly audible.

Nominal modifiers can sometimes be marked like possessors, with the genitive and/or with a possessive prefix on the following head noun, see §\ref{sec:gen.other}. 

The genitive can also appear between a noun phrase and a relator noun (§\ref{sec:relator.nouns}), and even be followed by focus markers in this position, as in (\ref{ex:GW.kWnA.WrkW.ri}).

\begin{exe}
\ex \label{ex:GW.kWnA.WrkW.ri}
\gll   tɯ-ci kɯ-wxti ɣɯ kɯnɤ ɯ-rkɯ ri nɯra tu ŋgrɤl.  \\
\textsc{indef}.\textsc{poss}-water \textsc{sbj}:\textsc{pcp}-be.big \textsc{gen} also \textsc{3sg}.\textsc{poss}-side \textsc{loc} \textsc{dem}:\textsc{pl} exist:\textsc{fact} be.usually.the.case:\textsc{fact} \\
\glt `(Dragonflies) are also found near (large) rivers.' (26-quspunmbro, 7)
\end{exe}

In these constructions, the genitive is always optional, and the prefix on the possessum suffices to express possession, as in (\ref{ex:paXCi.WjwaR}) (see §\ref{ex:prefix.expression.of.possession}).

\begin{exe}
\ex \label{ex:paXCi.WjwaR}
\gll paχɕi ɯ-jwaʁ tsa fse ri, nɯ sɤznɤ artɯm,\\
apple \textsc{3sg}.\textsc{poss}-leaf a.little be.like:\textsc{fact} \textsc{lnk} \textsc{dem} \textsc{comp} be.round:\textsc{fact} \\
\glt `(Its leaves) are a little like the leaves of an apple tree, but more round.' (09-mi, 15)
\end{exe}

When the possessum is elided however, the genitive postposition becomes obligatory, as in (\ref{ex:baigua.GW.sAz}).

\begin{exe}
\ex \label{ex:baigua.GW.sAz}
\gll ɯ-rɣi nɯnɯ, nɤki, <beigua> ɣɯ sɤz ɲɯ-jaʁjɯ. \\
\textsc{3sg}.\textsc{poss}-seed \textsc{dem} \textsc{filler}  pumpkin \textsc{gen} \textsc{comp} \textsc{sens}-be.thick.and.strong \\
\glt `Its seeds are thicker than those of the pumpkin.' (16-CWrNgo, 130)
\end{exe}

While there are transitive and semi-transitive verbs expressing possession (§\ref{sec:possessive.constructions}), the most common possessive construction involves an existential verb taking the possessum as subject, with the possessor marked by a possessive prefix on the possessum, and optionally with the genitive, as in (\ref{ex:phu.nW.GW.WRrW.GAZu}). 

\begin{exe}
\ex \label{ex:phu.nW.GW.WRrW.GAZu}
\gll qartsʰaz pʰu nɯ ɣɯ ɯ-ʁrɯ ɣɤʑu \\
deer male \textsc{dem} \textsc{gen} \textsc{3sg}.\textsc{poss}-horn exist:\textsc{sens} \\
\glt `The male deer has horns.' (27-qartshAz, 32)
\end{exe}

This construction is also used for abstract possession, as in (\ref{ex:aZWG.aBlu.tu}).

\begin{exe}
\ex \label{ex:aZWG.aBlu.tu}
\gll aʑɯɣ a-βlu tu \\
\textsc{1sg}:\textsc{gen} \textsc{1sg}.\textsc{poss}-trick exist:\textsc{fact} \\
\glt `I have an idea.' (140507 tangguowu-zh, 29)
\end{exe}

The causative verbs \japhug{ɣɤtu}{cause to have} and \japhug{ɣɤme}{cause not to have, destroy} derived from \japhug{tu}{exist} and \japhug{me}{not exist} (§\ref{sec:velar.caus.modal}) select an oblique argument with the genitive, as in (\ref{ex:WZo.GW.tuGAtea}). Although this argument could be considered to be a type of beneficiary (§\ref{sec:other.uses.poss.prefixes}), we observe here stability in case marking of the possessor between the base construction and the derived causative one.

\begin{exe} 
\ex \label{ex:WZo.GW.tuGAtea} 
\gll ɯʑo kɯ maka kɤ-ntɕʰoz mɤ-kɯ-ɤrɕo kɯ-fse ʑo tɯrɟɯ laχtɕʰa ɯʑo ɣɯ tu-ɣɤ-te-a jɤɣ \\ 
\textsc{3sg}.\textsc{poss} \textsc{erg} at.all \textsc{inf}-use \textsc{neg}-\textsc{inf}:\textsc{stat}-be.finished \textsc{inf}:\textsc{stat}-be.like \textsc{emph} wealth thing \textsc{3sg}.\textsc{poss} \textsc{gen} \textsc{ipfv}-\textsc{caus}-exist[III]-\textsc{1sg} be.possible:\textsc{fact} \\ 
\glt `(If someone saves me), I will make him have more wealth and riches than he can ever use.' (140512 yufu yu mogui-zh, 84) 
\end{exe} 
%ma aʑo a-kɤ-cha,  a-kɤ-cha kɯ-tu nɯra, a-kɤ-spa, tu-βze-a kɤ-cha nɯra lonba ʑo nɤʑɯɣ tɤ-ɣɤtu-t-a ɕti tɕe,

Not all combinations of existential verbs and genitival phrases are existential possessive constructions. For instance, in (\ref{ex:BZW.GW.WqiW}), the second clause could appear to contain a possessive construction meaning `the mouse only has half of it', but the context makes it clear that a different interpretation is necessary (§\ref{sec:existential.comparative}).

\begin{exe}
\ex \label{ex:BZW.GW.WqiW}
\gll qamtɕɯr nɯ ɯ-mtɕʰi nɯnɯ βʑɯ sɤznɤ mɤʑɯ ʑo amtɕoʁ tɕe nɯ βʑɯ ɣɯ ɯ-qiɯ kɯnɤ me \\
shrew \textsc{dem} \textsc{3sg}.\textsc{poss}-mouth \textsc{dem} mouse \textsc{comp} yet \textsc{emph} be.pointy \textsc{lnk} \textsc{dem} mouse \textsc{gen} \textsc{3sg}.\textsc{poss}-half even not.exist:\textsc{fact} \\
\glt `The shrew's mouth is even sharper than that of the mouse, and (its size) is not even half that of the mouse.' (27-spjaNkW, 204-205)
\end{exe}




\subsubsection{Recipient and beneficiary} \label{sec:gen.beneficiary}
 
The genitive can be used to mark the recipient by the indirective verb \japhug{kʰo}{give, pass over}, as in (\ref{ex:aZWG.nWkhAm}) and (\ref{ex:GW.anWtWkhAm}).   

\begin{exe}
\ex \label{ex:aZWG.nWkhAm}
 \gll ɕɯ ʑo stu kɯ-mɤku pɯ-tɯ-mto-t nɯnɯ, laχtɕha pɯ-nnɯ-ŋu, tɯrme pɯ-nnɯ-ŋu nɯ, aʑɯɣ nɯ-kʰɤm tɕe tɕendɤre, aʑo ɲɯ-ta-lɤt jɤɣ \\
 who \textsc{emph} most \textsc{sbj}:\textsc{pcp}-be.first \textsc{aor}-2-see-\textsc{pst}:\textsc{tr} \textsc{dem}  thing \textsc{pst}.\textsc{ipfv}-\textsc{auto}-be   person \textsc{pst}.\textsc{ipfv}-\textsc{auto}-be \textsc{dem} \textsc{1sg}:\textsc{gen} \textsc{imp}-give[III] \textsc{lnk} \textsc{lnk} \textsc{1sg} \textsc{ipfv}-1\fl{}2-release be.possible:\textsc{fact} \\
 \glt `Give me the first thing you see (when you go back home), be it a person or an object, and I will release you.' (140506 shizi he huichang de bailingniao-zh, 50-52)
\end{exe}

\begin{exe}
\ex \label{ex:GW.anWtWkhAm}
 \gll jɤ-tsɯm tɕe iɕqʰa nɯ kɯβʁa nɯ ɣɯ a-nɯ-tɯ-kʰɤm \\
 \textsc{imp}-take.away \textsc{lnk} the.aforementioned \textsc{dem} noble \textsc{dem} \textsc{gen} \textsc{irr}-\textsc{pfv}-2-give[III] \\
 \glt  Take it and give it to the nobleman.' (150831 renshen wawa-zh, 43)
\end{exe}
 
The recipient of the verb  \japhug{kʰo}{give, pass over} can alternatively also be marked by a possessive prefix on the inalienably possessed noun \japhug{tɯ-jaʁ}{hand} (with the meaning `hand over', §\ref{sec:semi.grammaticalized.relator}) or, with the dative relator nouns \forme{ɯ-ɕki} or \forme{ɯ-pʰe} (§\ref{sec:dative}).

The genitive is selected by a few intransitive modal verbs to indicate the experiencer/beneficiary, in particular  \japhug{ra}{be needed},`need', \japhug{ʁzi}{be necessary}, as in (\ref{ex:aZWG.WCArW}) and (\ref{ex:aZWG.Rzi}).

\begin{exe}
\ex \label{ex:aZWG.WCArW}
 \gll aʑɯɣ ɯ-ɕɤrɯ ra \\
 \textsc{1sg}:\textsc{gen} \textsc{3sg}.\textsc{poss}-bone be.needed:\textsc{fact} \\
\glt `I want its bones.' (07-deluge, 9)
\end{exe}

\begin{exe}
\ex \label{ex:aZWG.Rzi}
 \gll aʑɯɣ wuma ʑo ʁzi ɲɯ-ŋu, a-kɤ-ntɕʰoz sna ɲɯ-ŋu \\
  \textsc{1sg}:\textsc{gen} really \textsc{emph} be.necessary:\textsc{fact} \textsc{sens}-be \textsc{1sg}.\textsc{poss}-\textsc{obj}:\textsc{pcp}-use be.good:\textsc{fact}  \textsc{sens}-be \\
  \glt `It will be useful for me, it will have good use of it.'  (150902 hailibu-zh, 44-45)
\end{exe}

The experiencer/beneficiary can also be marked by possessive prefixes on the subject, without genitive, as in (\ref{ex:arNWl.mAra}) (see also §\ref{sec:other.uses.poss.prefixes} for additional examples).

\begin{exe}
\ex \label{ex:arNWl.mAra}
 \gll aʑo a-rŋɯl a-χsɤr ra mɤ-ra \\
 \textsc{1sg} \textsc{1sg}.\textsc{poss}-silver \textsc{1sg}.\textsc{poss}-gold \textsc{pl} \textsc{neg}-be.needed:\textsc{fact} \\
 \glt `I don't  need silver or gold.' (2014-kWlAG, 367)
\end{exe}

Other intransitive verbs selecting genitive arguments include \japhug{ŋgrɯ}{succeed}. In (\ref{ex:nAZWG.apWNgrW}), in addition to the oblique \textsc{2sg} argument \forme{nɤʑɯɣ}, the verb takes the infinitival complement clause \forme{βdaʁmu kɤ-ndo} as intransitive subject.

\begin{exe}
\ex \label{ex:nAZWG.apWNgrW}
 \gll [βdaʁmu kɤ-ndo] nɤʑɯɣ a-pɯ-ŋgrɯ qʰe, tɕendɤre tu-tɯ-ŋke maka mɤ-ra \\
queen \textsc{inf}-take \textsc{2sg}:\textsc{gen} \textsc{irr}-\textsc{pfv}-succeed \textsc{lnk} \textsc{lnk} \textsc{ipfv}-2-walk at.all \textsc{neg}-need:\textsc{fact} \\
\glt `If you succeed in becoming the queen, you will not need to walk anymore.' (140504 huiguniang-zh, 197-198)
\end{exe}

The genitive also occurs with beneficiaries/maleficiaries as adjuncts, not selected by the main verb, with transitive verbs such as \japhug{nɤma}{do} (\ref{ex:tChi.tunAmea}) and \japhug{wum}{collect} (\ref{ex:WZAG.pjAmaR}) or stative intransitive verbs such as \japhug{pe}{be good} as in (\ref{ex:aZWG.mApe}) with the meaning `be favourable, advantageous to'.  It is not the only possible way of expressing beneficiary; the relator noun \forme{ɯ-taʁ} also has this function in collocation with the stative verb \japhug{pe}{be good} (§\ref{sec:WtaR}), with the slightly different meaning `be nice to'.

\begin{exe}
\ex \label{ex:tChi.tunAmea}
\gll nɤʑɯɣ tɕʰi tu-nɤme-a ra, tɤ-ti  \\
\textsc{2sg}:\textsc{gen} what \textsc{ipfv}-do[III]-\textsc{1sg} be.needed:\textsc{fact} \textsc{imp}-say \\
\glt `Tell me what I shall do for you.' (140511 alading-zh, 175)
\end{exe}

\begin{exe}
\ex \label{ex:aZWG.mApe}
\gll  ɯ-fso tʰɯ-wxti tɕe aʑɯɣ mɤ-pe \\ 
\textsc{3sg}.\textsc{poss}-tomorrow \textsc{aor}-be.big \textsc{lnk} \textsc{1sg}:\textsc{gen} \textsc{neg}-be.good:\textsc{fact} \\
\glt `In the future, when he will have grown up, he will cause me trouble.' (`he will not be good to me', 2011-05-nyima, 22)
\end{exe}

The beneficiary adjunct is not necessarily contiguous with the verb on which it depends, as in (\ref{ex:iZora.GW.tChi.tufsej}) where the genitive phrase \forme{iʑora ɣɯ} `for us, on our behalf'  is separated from the verb \japhug{tʰu}{ask} by a lengthy complement comprising two clauses.

\begin{exe}
\ex \label{ex:iZora.GW.tChi.tufsej}
\gll  iʑora ɣɯ [tɕʰi tu-fse-j tɕe ji-tɯ-ci ɣɤʑu] tu-tɯ-tʰe ɯ-tɯ́-cʰa? \\
\textsc{1pl} \textsc{gen} what \textsc{ipfv}-be.like-\textsc{1pl} \textsc{lnk} \textsc{1pl}.\textsc{poss}-\textsc{indef}.\textsc{poss}-water exist:\textsc{sens} \textsc{ipfv}-2-ask[III] \textsc{qu}-2-can:\textsc{fact} \\
\glt `Can you ask on our behalf how we should do to have water?' (2005tamukatsa, 14)
\end{exe}

Beneficiary genitive phrases can occur as predicates with a copula as \japhug{ɯʑɤɣ}{\textsc{3sg}:\textsc{gen}} in (\ref{ex:WZAG.pjAmaR}).

 \begin{exe}
\ex \label{ex:WZAG.pjAmaR}
\gll   tʰoʁtɤm ka-wum tɕe, ɯʑɤɣ pjɤ-maʁ kɯ, tɕoχtsi rɟɤlpu ɣɯ ku-wum,  \\
taxes \textsc{aor}:3\fl{}3'-collect \textsc{lnk} \textsc{3sg}:\textsc{gen} \textsc{ifr}.\textsc{ipfv}-not.be \textsc{erg}  \textsc{anthr} king \textsc{gen} \textsc{ipfv}-collect \\
\glt `The taxes that he had collected were not for himself, he was collecting them for the king of Cogtse.' (150901 NAjstsa, 28)
\end{exe}

In this use too, it is alternatively possible to indicate the beneficiary as a possessive prefix on the object, without genitive postposition, as in (\ref{ex:atWci.tArke}).

 \begin{exe}
\ex \label{ex:atWci.tArke}
\gll   χsɤr kʰɯtsa ɯ-ŋgɯ nɯtɕu a-tɯ-ci ci tɤ-rke ma wuma ɲɯ-ɕpaʁ-a \\
gold bowl \textsc{3sg}.\textsc{poss}-inside \textsc{dem}:\textsc{loc} \textsc{1sg}.\textsc{poss}-\textsc{indef}.\textsc{poss}-water a.little \textsc{imp}-put.in[III] \textsc{lnk} really \textsc{sens}-be.thirsty-\textsc{1sg} \\
\glt  `Please pour some water in the golden bowl for me, I am thirsty.' (140428 mu e guniang-zh, 47)
\end{exe}

The genitive is also attested with a noun-verb collocations (§\ref{sec:light.verb}), like \japhug{ɯ-kɤrnoʁ+mtɕɯr}{feel dizzy}, in which the possessor  of the noun is an experiencer as in (\ref{ex:fsapaR.GW.kWnA}). This example also illustrates the use of the genitive followed by a focus marker, as (\ref{ex:GW.kWnA.WrkW.ri}) above.

\begin{exe}
\ex \label{ex:fsapaR.GW.kWnA}
\gll tɕeri fsapaʁ ɣɯ kɯnɤ ɯ-kɤrnoʁ ɲɯ-mtɕɯr ɲɯ-ŋu \\
\textsc{lnk} animal \textsc{gen} also \textsc{3sg}.\textsc{poss}-head \textsc{sens}-turn \textsc{sens}-be \\
\glt `But animals too can feel dizzy.' (29-tAmtshAzkAkWndo, 71)
\end{exe}

Finally, the experiencer subject argument demoted by the proprietive derivation can also in some cases be optionally encoded with the genitive case (see example \ref{ex:aZWG.mAsAscit}, §\ref{sec:proprietive}).

\subsubsection{Other uses} \label{sec:gen.other}
The genitive \forme{ɣɯ} occurs with various types of noun complements which are semantically neither possessive or beneficiaries/recipients. 

Nouns used as prenominal modifiers are in rare cases followed by a genitive postposition before the head noun. If the head noun is an alienably possessed noun, the presence of a third singular possessive prefix \forme{ɯ-} is optional, as shown by examples such as (\ref{ex:χsAr.GW.khWtsa}) and (\ref{ex:ftsoR.kWngWt.WphW}). 

\begin{exe}
\ex \label{ex:χsAr.GW.khWtsa}
\gll  χsɤr ɣɯ, nɤkinɯ, kʰɯtsa ci to-nɯ-ndo. \\
gold \textsc{gen} \textsc{filler} bowl \textsc{indef} \textsc{ifr}-\textsc{auto}-take \\
\glt `He took a golden bowl' (140508 shier ge tiaowu de gongzhu-zh, 158)
\end{exe}

This type of construction is most common in texts translated from Chinese, but does also occur in more spontaneous material as in (\ref{ex:ftsoR.kWngWt.WphW}), with a complex modifier \forme{ftsoʁ kɯngɯt ɯ-pʰɯ} `the price of nine female hybrid yaks'.

\begin{exe}
\ex \label{ex:ftsoR.kWngWt.WphW}
\gll tɕendɤre ɯ-jaʁ nɯtɕu [ftsoʁ kɯngɯt ɯ-pʰɯ] ɣɯ srɯnloʁ pjɤ-k-ɤ-rku-ci \\
\textsc{lnk} \textsc{3sg}.\textsc{poss} \textsc{dem}:\textsc{loc} female.hybrid.yak nine \textsc{3sg}.\textsc{poss}-price \textsc{gen} ring \textsc{ifr}.\textsc{ipfv}-\textsc{peg}-pass-put.in-\textsc{peg} \\
\glt `She had a ring worth nine female hybrid yak in her hand.' (2003gesar, 239)
\end{exe}

In a construction with a prenominal modifier marker in the genitive, even when a possessive prefix is present on the head noun (in particular when it is an inalienably possessed noun), that prefix does not necessarily refer to the modifier. For instance, in (\ref{ex:tWpAlAskAr.GW.nWmgozmArAB}), the third plural possessive prefix \forme{nɯ-} on \forme{nɯ-mgozmɤrɤβ} `their vegetables' refers to the people eating the vegetable, not  the modifier \japhug{tɯxpalɤskɤr}{the whole year} (on whose formation see §\ref{sec:dvandva.coll}) which would require a third singular prefix instead (an option which is also attested with this noun). Alternatively, it is also possible to have an indefinite possessor prefix on the head noun if inalienably possessed, as in (\ref{ex:tWxpa.GW.tWGli}). Note that both options are attested in the construction with a prenominal modifier without the genitive, as seen in §\ref{sec:possessive.prefixes.prenominal}.
 
 \begin{exe}
\ex \label{ex:tWpAlAskAr.GW.nWmgozmArAB}
\gll  tɕe nɯnɯ tɯxpalɤskɤr ɣɯ nɯ-mgozmɤrɤβ nɯ nɯ ma pjɤ-me.  \\
\textsc{lnk} \textsc{dem} whole.year \textsc{gen} \textsc{3pl}.\textsc{poss}-vegetable \textsc{dem} \textsc{dem} apart.from \textsc{ifr}.\textsc{ipfv}-not.exist \\
\glt `It was the only vegetable that they had the whole year.'  (140522 kAmYW tWji, 23)
\end{exe}

\begin{exe}
\ex \label{ex:tWxpa.GW.tWGli}
\gll  tɯ-xpa ɣɯ tɯ-ɣli nɯ cʰɯ́-wɣ-tɕɤt tú-wɣ-rmbɯ  \\
one-year \textsc{gen} \textsc{indef}.\textsc{poss}-manure \textsc{dem} \textsc{ipfv}:\textsc{downstream}-\textsc{inv}-take.out \textsc{ipfv}-\textsc{inv}-heap \\
\glt `People take out (from the stable) the whole year's manure and heap it up.' (2010-tArAku)
\end{exe}

Some apparently unclassifiable uses of the genitive can be accounted for to some extent by assuming the elision of a head noun.  For instance, in (\ref{ex:iZora.GW}), the phrase \forme{iʑora ɣɯ}, meaning `in our language', can be explained as coming from \forme{iʑora ɣɯ ji-skɤt} `our language' used as a absolutive locative phrase (§\ref{absolutive.locative}) `in our language', with elision of the head noun. This example does not illustrate a separate function of the genitive: it is simply a particular case of possessive.

\begin{exe}
\ex \label{ex:iZora.GW}
\gll  <longtoutan> nɯ kupa-skɤt ɕti. tɕe iʑora ɣɯ tɕʰi tu-kɯ-ti ŋu mɤ-xsi. \\
\textsc{topo} \textsc{dem} Chinese-language be.\textsc{aff}:\textsc{fact} \textsc{lnk} \textsc{1pl} \textsc{gen} what \textsc{ipfv}-\textsc{genr}-say be:\textsc{fact} \textsc{neg}-\textsc{genr}:know \\
\glt `Longtoutan is a Chinese word; I don't know how it is said in our (language).'  (150820 qaprANar, 32)
\end{exe}

The same is true of the use of the genitive with the verb \japhug{mŋɤm}{be painful} and its causative \japhug{ɕɯmŋɤm}{cause to be painful}, which take a body part (not the person or animal feeling pain) as their subject and object, respectively. In (\ref{ex:aZWG.taCWmNAm}), the genitive first person \japhug{aʑɯɣ}{\textsc{1sg}:\textsc{gen}} is not an oblique argument or even a malefactive adjunct. Rather, its presence implies an elided noun \japhug{a-βri}{my body} (`he caused pain to my body'). It is however likely that sentences like this are the pivot constructions which made possible the reanalysis of possessive genitive phrases as benefactive/malefactive adjuncts.

\begin{exe}
\ex \label{ex:aZWG.taCWmNAm}
\gll aʑɯɣ ta-ɕɯ-mŋɤm, aʑɯɣ a-laχtɕʰa ra ja-nɯ-tsɯm-nɯ \\
\textsc{1sg}:\textsc{gen} \textsc{aor}:3\fl{}3'-\textsc{caus}-be.painful \textsc{1sg}:\textsc{gen} \textsc{1sg}.\textsc{poss}-thing \textsc{pl} \textsc{aor}:3\fl{}3'-\textsc{vert}-take.away-\textsc{pl} \\
\glt `He hurt me and took away my things.' (140426 luozi he qiangdao, 35)
\end{exe}

The genitive can also occur between prenominal relatives (§\ref{sec:genitival.relatives}) and their head noun. In this construction the head noun generally does not take a possessive prefix. This type of relative is particularly common in story translated from Chinese, where it calques the prenominal relatives in \zh{的} <de>, as in (\ref{ex:makWra.GW.sAtCha}). The same situation has been observed in Khroskyabs (\citealt[640--643]{lai17khroskyabs}).

\begin{exe}
\ex \label{ex:makWra.GW.sAtCha}
\gll [kɯ-ɣɤndʐo ri kɯ-me], [kɯ-sɤ-mtsɯr ri kɯ-me], [kɤ-nɯsɯmɯzdɯɣ ri mɤ-kɯ-ra] ɣɯ sɤtɕʰa nɯtɕu jo-ɕe-ndʑi ɲɯ-ŋu. \\
\textsc{sbj}:\textsc{pcp}-be.cold also \textsc{sbj}:\textsc{pcp}-not.exist \textsc{sbj}:\textsc{pcp}-\textsc{prop}-be.hungry also \textsc{sbj}:\textsc{pcp}-not.exist \textsc{inf}-worry also \textsc{neg}-\textsc{sbj}:\textsc{pcp}-be.needed \textsc{gen} place \textsc{dem}:\textsc{loc} \textsc{ifr}-go-\textsc{du} \textsc{sens}-be \\
\glt  `The two of them went to a place where they was cold cold and hunger, and where one did not need to worry.' (140519 mai huochai de xiao nvhai-zh, 182-183)
\end{exe}

However, this type of relative is also attested, though rarer, in non-translated texts, for instance in (\ref{ex:tWxpa.tukWlhoR.GW.sWjno}) with intransitive subject relativization.
 

\begin{exe}
\ex \label{ex:tWxpa.tukWlhoR.GW.sWjno}
\gll  tɕe [tɯ-xpa tu-kɯ-ɬoʁ] ɣɯ sɯjno nɯ ŋu tɕe, \\
\textsc{lnk} one-year \textsc{ipfv}-\textsc{sbj}:\textsc{pcp}-come.out \textsc{gen}  \textsc{topo} \textsc{dem} be:\textsc{fact} \textsc{lnk} \\
\glt  `It is an annual plant.' (18-NGolo, 105)
\end{exe}

Genitival prenominal relative clauses are to be distinguished from relatives as possessors, as in (\ref{ex:tWCGA.kWmNAm.GW.WrJAnNgo}), where the possessum  \japhug{ɯ-rɟɤŋgo}{its radiating pain} is not an argument of the relative \forme{tɯ-ɕɣa kɯ-mŋɤm} `a tooth that hurts'. 

\begin{exe}
\ex \label{ex:tWCGA.kWmNAm.GW.WrJAnNgo}
\gll tɯ-ɕɣa a-tɤ-mŋɤm tɕe tɕe tɤ-rca tɯ-ɣmba, tɯ-ku nɯra tu-mŋɤm ɲɯ-ŋu tɕe,  nɯnɯ ``[tɯ-ɕɣa kɯ-mŋɤm] ɣɯ ɯ-rɟɤŋgo ɣɤʑu" tu-kɯ-ti ŋu. \\
\textsc{genr}.\textsc{poss}-tooth \textsc{irr}-\textsc{pfv}-be.painful \textsc{lnk} \textsc{lnk} \textsc{indef}.\textsc{poss}-following \textsc{genr}.\textsc{poss}-cheek \textsc{genr}.\textsc{poss}-head \textsc{dem}:\textsc{pl} \textsc{ipfv}-be.painful \textsc{sens}-be \textsc{lnk} \textsc{dem} \textsc{indef}.\textsc{poss}-tooth \textsc{sbj}:\textsc{pcp}-be.painful \textsc{gen} \textsc{3sg}.\textsc{poss}-radiating.pain exist:\textsc{sens} \textsc{ipfv}-\textsc{genr}-say be:\textsc{fact} \\
\glt `When one has a toothache, and that one feels pain in one's cheek or a headache, one says `the toothache has a radiating pain.'' (140516 WrJANgo, 3)
\end{exe}

Adnominal complement clauses (§\ref{sec:complement.taking.nouns}) can also take a genitive marker, as in (\ref{ex:mWjnaXtChWG.GW.WtCha}).

\begin{exe}
\ex \label{ex:mWjnaXtChWG.GW.WtCha}
\gll [<donggua> cʰo <qiezi> ni tɕʰi ʑo mɯ́j-naχtɕɯɣ] ɣɯ ɯ-tɕʰa a-jɤ-tɯ-ɣɯt ra \\
gourd \textsc{comit} eggplant \textsc{du} what \textsc{emph} \textsc{neg}:\textsc{sens}-be.the.same \textsc{gen} \textsc{3sg}.\textsc{poss}-information \textsc{irr}-\textsc{pfv}-2-bring be.needed:\textsc{fact} \\
\glt `(Go there and come back to) tell me in what way gourd and eggplant differ from each other.' (2010-02-yitian bi yitian-zh, 7)
\end{exe}

Some relative clauses can take possessors marked in the genitive, as in  (\ref{ex:stu.WkAnWmga}) and (\ref{ex:slama.ra.GW}). It is debatable whether the genitival phrase belongs to the relative in this type of construction.

\begin{exe}
\ex \label{ex:stu.WkAnWmga}
 \gll tɕe paʁ ɣɯ [stu ɯ-kɤ-nɯmga], iʑora ji-kɤ-nɯmga nɯ ɯ-ɕa ŋu tɕe \\
 \textsc{lnk} pig \textsc{gen} most \textsc{3sg}.\textsc{poss}-\textsc{obj}:\textsc{pcp}-want.from \textsc{1pl} \textsc{1pl}.\textsc{poss}-\textsc{obj}:\textsc{pcp}-want.from \textsc{dem} \textsc{3sg}.\textsc{poss}-meat be:\textsc{fact} \textsc{lnk} \\
\glt  `What is most wanted from pigs, what we want from them is their meat.' (05-paR, 13)
\end{exe}

\begin{exe}
\ex \label{ex:slama.ra.GW}
\gll  slama ra ɣɯ [tʰɯtʰɤci kɯ-fse], nɯ kɤ-rɤ-βzjoz ra ɲɯ-stu mɯ́j-stu nɯ, nɯ-stu ɲɯ-nɤma-nɯ mɯ́j-nɤma-nɯ,  nɯnɯra nɯ-pʰama ra nɯ-ɕki kɯ-rɤ-fɕɤt ɲɯ-ra. \\
student \textsc{pl} \textsc{gen} something \textsc{sbj}:\textsc{pcp}-be.like \textsc{dem}  \textsc{inf}-\textsc{antipass}-learn \textsc{pl} \textsc{sens}-be.assiduous \textsc{sens}-be.assiduous \textsc{dem} \textsc{3pl}.\textsc{poss}-truth \textsc{sens}-work-\textsc{pl} \textsc{neg}:\textsc{sens}-work-\textsc{pl} \textsc{dem}:\textsc{pl} \textsc{3pl}.\textsc{poss}-parent \textsc{pl} \textsc{3pl}.\textsc{poss}-\textsc{dat} \textsc{genr}:S/O-\textsc{antipass}-tell:\textsc{fact} \textsc{sens}-be.needed \\
\glt `One had to tell the parents all kinds of things concerning the students, whether they try hard or not, whether they work seriously or not.' (150901 tshuBdWnskAt, 18-20)
\end{exe}

The genitive \forme{ɣɯ} can optionally be used after the object in purposive complements (§\ref{sec:purposive.clause.motion.verbs}) containing a transitive verb as in (\ref{ex:GW.WkWrtoR}); in this type of clauses, the verb is in subject participial form and transitive verbs take a possessive prefix coreferent with the object (§\ref{sec:subject.participle.possessive}).

\begin{exe}
\ex \label{ex:GW.WkWrtoR}
\gll rgɤtpu nɯ ɣɯ ɯ-kɯ-rtoʁ jo-ɣi. \\
old.man \textsc{dem} \textsc{gen} \textsc{3sg}.\textsc{poss}-\textsc{sbj}:\textsc{pcp}-see \textsc{ifr}-come \\
\glt `He came to see the old man.' (150908 menglang-zh, 12)
\end{exe}

\subsection{Locative} \label{sec:locative}
 

\subsubsection{Core locative postpositions} \label{sec:core.locative}
There are three locative postpositions in Japhug, \forme{zɯ}, \forme{tɕu} and \forme{ri}, the latter being homophonous with the correlative additive focus \forme{ri} (§\ref{sec:ri.additive}). The exact conditions of their uses is still an unsolved problem of Japhug grammar. They appear to be always optional (goals and locative adjuncts can always be in absolutive form, see §\ref{absolutive.goal}  and \ref{absolutive.locative}) and seem to be interchangeable, as is illustrated in this section.

All three postpositions can be used to express static location, motion into, motion or from a place. Location or motion (into/from/on) a surface, (into/from/in) a container or with/without contact does not seem to be relevant factors for the selection of the locative postpositions.

The locative \forme{tɕu}  is most often used in combination with a demonstrative \forme{nɯ} as in (\ref{ex:co.nWtCu}), a form identical to the locative of the demonstrative pronoun (\japhug{nɯtɕu}{there}, see §\ref{sec:locative.pronoun}). Without demonstrative, \forme{tɕu} is also found as in (\ref{ex:tsxu.tCu}) and (\ref{ex:khAxtAndo.tCu}).

\begin{exe}
\ex \label{ex:co.nWtCu}
\gll japa tɕe alo <ercha> nɯtɕu, nɤkinɯ, iɕqʰa tsʰapa co nɯtɕu tɯɲɤt cʰɤ-ɣi. \\
last.year \textsc{lnk} upstream  \textsc{topo} \textsc{dem}:\textsc{loc} \textsc{filler}   \textsc{filler}   \textsc{topo} valley \textsc{dem}:\textsc{loc} rock.slide \textsc{ifr}:\textsc{downstream}-come \\
\glt `Last year, at Ercha, at the valley of Tshapa, there was a rock slide.' (160715 nWNa, 1)
\end{exe}

\begin{exe}
\ex \label{ex:tsxu.tCu}
\gll jo-nɯ-ɕe tɕe, tʂu tɕu ɲɤ-mtsɯr, \\
\textsc{ifr}-\textsc{auto}-go \textsc{lnk} road \textsc{loc} \textsc{ifr}-be.hungry \\
\glt `He went away, and on the road he felt hungry.' (2002qajdoskAt, 109)
\end{exe}

\begin{exe}
\ex \label{ex:khAxtAndo.tCu}
\gll  kʰɤxtɤndo tɕu ko-zo \\
side.of.the.top.terrace \textsc{loc} \textsc{ifr}-land \\
\glt `(The raven) landed on the side of the top terrace.' (2002qajdoskAt, 24)
\end{exe}

The above examples show \forme{tɕu} used for location without motion (\ref{ex:co.nWtCu}), motion via a place (\ref{ex:tsxu.tCu}), motion onto a place resulting in contact with the surface (\ref{ex:khAxtAndo.tCu}), and (\ref{ex:WkWm.nWtCu}) illustrates \forme{tɕu} expressing motion from the inside.

\begin{exe}
\ex \label{ex:WkWm.nWtCu}
\gll nɯnɯ kɯβʁa ra ɣɯ nɯ-kʰa ɯ-kɯm nɯtɕu cʰɤ-nɯ-ɬoʁ tɕe, \\
\textsc{dem} nobleman \textsc{pl} \textsc{gen} \textsc{3pl}.\textsc{poss}-house \textsc{3sg}.\textsc{poss}-door \textsc{dem}:\textsc{loc} \textsc{ifr}:\textsc{downstream}-\textsc{auto}-come.out \textsc{lnk} \\
\glt  `He went out from the door of the nobleman's house.' (140513 mutong de disheng-zh, 166)
\end{exe}

The same diversity of uses is found with the locative \forme{zɯ}; (\ref{ex:nAmkha.zW.pjAnWlhoRnW}) shows \forme{zɯ} expressing static location (`in their hands') and motion from a place (`from the sky').  Example (\ref{ex:Wjme.zW.kAzo}) illustrates \forme{zɯ} used with the verb \japhug{zo}{land}, which is attested with \forme{tɕu} in (\ref{ex:khAxtAndo.tCu}). Note that in another version of the same story by the same speaker, the relator noun \japhug{ɯ-taʁ}{on} (§\ref{sec:relator.nouns.3d}) is found instead of a locative postposition (see \ref{ex:ambro} in §\ref{ex:prefix.expression.of.possession}).

\begin{exe}
\ex \label{ex:nAmkha.zW.pjAnWlhoRnW}
\gll tɯmɯkɤrŋi ɯ-me kɯɕnɯz nɯ nɯ-jaʁ zɯ, nɤkinɯ, <shanzi> kɯ-mpɕɯ\redp{}mpɕɤr ʑo, qale ɯ-sɤ-lɤt nɯ pjɤ-k-ɤsɯ-ndo-nɯ-ci tɕe,  tɯmɯnɤmkʰa zɯ pjɤ-nɯ-ɬoʁ-nɯ. \\
heaven \textsc{3sg}.\textsc{poss}-daughter seven \textsc{dem} \textsc{3pl}.\textsc{poss}-hand \textsc{loc} \textsc{filler} fan \textsc{sbj}:\textsc{pcp}-\textsc{emph}\redp{}be.beautiful \textsc{emph} wind \textsc{3sg}.\textsc{poss}-\textsc{obl}:\textsc{pcp}-release \textsc{dem} \textsc{ifr}.\textsc{ipfv}-\textsc{peg}-\textsc{prog}-hold-\textsc{pl}-\textsc{peg} \textsc{lnk} sky \textsc{loc} \textsc{ifr}:\textsc{down}-\textsc{auto}-come.out-\textsc{pl} \\
\glt `The seven daughters of heaven, holding beautiful fans in their hands, came down from the sky.' (150828 niulang-zh, 48)
\end{exe}

\begin{exe}
\ex \label{ex:Wjme.zW.kAzo}
\gll   a-mbro ɯ-jme zɯ kɤ-zo, \\
\textsc{1sg}.\textsc{poss}-horse \textsc{3sg}.\textsc{poss}-tail \textsc{loc} \textsc{imp}-land \\ 
\glt `Land on my horse's tail.' (2014-kWlAG, 562)
\end{exe}

The postposition \forme{zɯ} is related to the suffix \forme{-s} in Situ, which expresses motion from an origin or towards a goal (\citealt[330--331]{linxr93jiarong}). Situ is certainly most archaic in this regard (as it also preserves a locative \forme{-j} suffix of which only lexicalized traces remain in Japhug, §\ref{sec:locative.j}), and the Japhug form has to be explained as debonding  from suffix to clitic to independent word (see \ref{ex:nWnWtCu.zW} in §\ref{ex:word.vs.clitic.postp} for evidence that \forme{zɯ} is not a clitic). Japhug-internal evidence for the degrammaticalization is the otherwise unexplainable voicing to \forme{z}, a process that applied to all fricative codas (§\ref{sec:codas.inventory}), and the fact that some frozen forms preserve a \forme{-z} suffix, in particular the approximate locative \forme{cʰiz} (§\ref{sec:approximate.locative}), the related indefinite pronoun \japhug{ciscʰiz}{somewhere} (§\ref{sec:cischiz}) and the relator \japhug{ɯ-ŋgɯz}{among} (see \ref{ex:kAndZWRi.nWNgWz}  and \ref{ex:arNi.WNgWz}  in §\ref{sec:other.locative.relator}) and the postposition  \japhug{ʁaz}{while ... still}  (§\ref{sec:denominal.postposition.s}).  


 The locative \forme{ri} also occurs in all the meanings attested above for \forme{tɕu} and \forme{zɯ}, though due to homophony with the additive correlative \forme{ri} (§\ref{sec:ri.additive}), some examples are ambiguous. Examples (\ref{ex:khAXtu.ri.pWrAZia}) and (\ref{ex:kha.ri.kari}) show the locative \forme{ri} marking static location and motion towards a place, respectively.

\begin{exe}
\ex \label{ex:khAXtu.ri.pWrAZia}
\gll   kʰɤxtu ri pɯ-rɤʑi-a tɕe tɤ-mtʰɯm tu-ndze-a pɯ-ŋu ri, \\
terrace \textsc{loc} \textsc{pst}.\textsc{ipfv}-stay-\textsc{1sg} \textsc{lnk} \textsc{indef}.\textsc{poss}-meat \textsc{ipfv}-eat[III]-\textsc{1sg} \textsc{pst}.\textsc{ipfv}-be \textsc{lnk} \\
\glt `I was on the terrace eating meat.' (150909 qandZGi, 5)
 \end{exe}
 
 \begin{exe}
\ex \label{ex:kha.ri.kari}
\gll   tɕe nɯ nɯ-kʰa ri ɯʑo kɤ-ari ɯ-qʰu tɕe, \\
\textsc{lnk} \textsc{dem} \textsc{3pl}.\textsc{poss}-house \textsc{loc} \textsc{3sg} \textsc{aor}:\textsc{east}-go[II] \textsc{3sg}.\textsc{poss}-after \textsc{lnk} \\
\glt `After he went to their (his wife's family's) house...' (14-siblings, 210)
  \end{exe}
  
The same interchangeability and optionality of the locative postpositions is observed when these are combined with relator nouns (§\ref{sec:relator.postposition.location}). Although the three postpositions are almost identical in their range of uses, there are nevertheless differences in their compatibilities.  With the locative demonstrative pronouns \japhug{kɯre}{here}, and \japhug{nɯre}{there}  (§\ref{sec:locative.pronoun}) as well as \japhug{aʁɤndɯndɤt}{everywhere}, only \forme{ri} can be added, as in (\ref{ex:nWre.ri.li.akumpGa}). The postpositions \forme{tɕu} and \forme{zɯ} can be combined as \forme{nɯtɕu zɯ} as in (\ref{ex:tCAkW.nWtCu.zW}); no other combination of locative postpositions are possible.

 \begin{exe}
\ex \label{ex:nWre.ri.li.akumpGa}
\gll tɕendɤre tsʰuβdɯn kɯ-sɤ-sɯxɕɤt lɤ-ari-a. tɕe nɯre ri li a-kumpɣa pɯ-tu tɕe \\
\textsc{lnk}  \textsc{topo} \textsc{nmlz}-\textsc{antipass}-teach \textsc{aor}:\textsc{upstream}-go[II]-\textsc{1sg} \textsc{lnk} \textsc{dem}:\textsc{loc} \textsc{loc} again \textsc{1sg}.\textsc{poss}-hen \textsc{pst}.\textsc{ipfv}-exist \textsc{lnk} \\
\glt `I went to Tshobdun to teach. There, I had a hen again.' (150819 kumpGa, 78-79)
  \end{exe}

 \begin{exe}
\ex \label{ex:tCAkW.nWtCu.zW}
\gll   ɯʑo tɕekɯ nɯtɕu zɯ pjɤ-rɤʑi qʰe \\
\textsc{3sg} \textsc{east} \textsc{dem}:\textsc{loc} \textsc{loc} \textsc{ifr}.\textsc{ipfv}-stay \textsc{lnk}  \\
\glt `He was on the east side.' (28-qAjdoskAt, 152)
  \end{exe}
   
All three postpositions are also used with time adjuncts, but do present some noticeable differences in usage. They can be interchangeably used with temporal relator nouns such as \japhug{ɯ-raŋ}{during, the time when} (§\ref{sec:relator.temporal}), but in other contexts 

With counted nouns expressing time (§\ref{sec:CN.time}), there is not a single example with \forme{zɯ} in the corpus. The postposition \forme{ri} is used to express a point in time  (as in \ref{ex:tWxsoz.ri}), while \forme{tɕu} mostly means `within (the time period)' as in (\ref{ex:tWxpa.nWtCu}).  

\begin{exe}
\ex \label{ex:tWxsoz.ri}
\gll tɯ-xsoz ri tɕe jo-ɣi ri, ɯ-βri tɤʑri ɣɤʑu, ɲɯ-ɤci. \\
one-morning \textsc{loc} \textsc{lnk} \textsc{ifr}-come \textsc{lnk} \textsc{3sg}.\textsc{poss}-body dew exist:\textsc{sens} \textsc{sens}-be.wet \\
\glt `One morning, (our hen) came (back from the forest, where it was laying eggs) and its body was wet from the dew.' (150819 kumpGa, 21-22)
\end{exe}

\begin{exe}
\ex \label{ex:tWxpa.nWtCu}
\gll  tɯ-xpa nɯtɕu [...] tɯ-tɯpʰu, ɣʑo nɯra kɯ, ɣʑɤzga nɯnɯ, nɯnɯ sqɯ-tɯrpa ɯ-ro nɯ ku-sɯ-ɤjtɯ-nɯ ɲɯ-cʰa-nɯ. \\
one-year \textsc{dem}:\textsc{loc} { } one-hive bee \textsc{dem}:\textsc{pl} \textsc{erg} honey \textsc{dem} \textsc{dem} ten-pound \textsc{3sg}.\textsc{poss}-excess \textsc{dem} \textsc{ipfv}-\textsc{caus}-accumulate-\textsc{pl} \textsc{sens}-can-\textsc{pl} \\
\glt  `In one year, one hive, the bees, the honey, they can gather more than ten pounds of it.' (26-GZo, 34-36)
\end{exe}

The form \forme{tɕe}, which  is mainly analyzable as a linker (§\ref{sec:coordination}) and in some cases a topic marker (§\ref{sec:tCe.topic}) occurs with locative and temporal adjuncts and could be analyzed as a postposition in these usages (see also §\ref{sec:locative.j}). With the temporal counted nouns, like \forme{ri} it expresses a specific point in time as in (\ref{ex:tWsNi.tCe}) rather than a duration.

 \begin{exe}
\ex \label{ex:tWsNi.tCe}
\gll   tɕe tɯ-sŋi tɕe ɲɤ-k-ɤtɯɣ-ci tɕe, \\
\textsc{lnk} one-day \textsc{lnk} \textsc{ifr}-\textsc{peg}-meet-\textsc{peg} \textsc{lnk} \\
\glt  `One day, (the bear finally) met (the rabbit).' (2011-13-qala, 22)
\end{exe}


\subsubsection{Approximate locative} \label{sec:approximate.locative}
The suffix \forme{-cʰu} is used to indicate approximate location, referring to a broad area rather than a specific place, like the plural \forme{ra} (§\ref{sec:plural.determiners}). In (\ref{ex:kupa.chu}) it means `area/region', while in (\ref{ex:Wqhu.chu}) it can be translated as `side' and is opposed to \japhug{ɯ-stu}{straight, side}.
 
 \begin{exe}
\ex \label{ex:kupa.chu}
 \gll  tɕe kupa-cʰu nɯra, atʰi pɕoʁ nɯra, ɯ-pɕi nɯra kɯ kɯre ri ɣɯ-cʰɯ-sɯ-χtɯ-nɯ ŋu.  \\
 \textsc{lnk} Chinese-\textsc{approx}.\textsc{loc} \textsc{dem}:\textsc{pl}  downstream side \textsc{dem}:\textsc{pl} \textsc{3sg}.\textsc{poss}-outside \textsc{dem}:\textsc{loc} \textsc{erg} \textsc{dem}.\textsc{prox}:\textsc{loc} \textsc{loc} \textsc{cisl}-\textsc{ipfv}:\textsc{downstream}-\textsc{caus}-but-\textsc{pl} be:\textsc{fact} \\
 \glt `People from Chinese areas, from downstream, from outside, send (people) here to buy (these mushrooms). (20-grWBgrWB, 59)
 \end{exe}
 
 It also occurs with various relator nouns, in particular \japhug{ɯ-qʰu}{after, behind} and \japhug{ɯ-ŋgɯ}{inside} as in (\ref{ex:Wqhu.chu}) and (\ref{ex:WNgW.chu}), respectively. The form \forme{ɯ-qʰu-cʰu} exclusively has a locative meaning, unlike \forme{ɯ-qʰu} which can be used with the temporal meaning `after' Note that \forme{ɯ-ŋgɯ} undergoes vowel assimilation to [\forme{ɯŋgucʰu}], showing that it must be analyzed as a suffix rather than as a postposition. Most relator nouns cannot be used with \forme{cʰu}, for instance one cannot say $\dagger$\forme{ɯ-ʁɤri-cʰu}.

  \begin{exe}
\ex \label{ex:Wqhu.chu}
 \gll  ɯ-jwaʁ ɯ-ʁɤri ɯ-stu nɯ ɲɯ-ɤrŋi, ɯ-jwaʁ ɯ-qʰu-cʰu nɯ ɲɯ-pɣi, \\
 \textsc{3sg}.\textsc{poss}-leaf  \textsc{3sg}.\textsc{poss}-front \textsc{3sg}.\textsc{poss}-direction \textsc{dem} \textsc{sens}-be.green   \textsc{3sg}.\textsc{poss}-leaf  \textsc{3sg}.\textsc{poss}-behind-\textsc{approx}.\textsc{loc} \textsc{dem} \textsc{sens}-be.grey \\
 \glt  `The upper side of its leaves is green, and the lower side is grey.' (13-NanWkWmtsWG, 8)
  \end{exe}
  
   \begin{exe}
\ex \label{ex:WNgW.chu}
 \gll tɕe ɯ-xtɤpa cʰo ɯ-mi, ɯ-jaʁ, ɯ-ŋgɯ-cʰu nɯra ɲɯ-wɣrum. \\
\textsc{lnk} \textsc{3sg}.\textsc{poss}-lower.belly \textsc{comit}  \textsc{3sg}.\textsc{poss}-foot  \textsc{3sg}.\textsc{poss}-hand  \textsc{3sg}.\textsc{poss}-\textsc{inside}-\textsc{approx}.\textsc{loc} \textsc{dem}:\textsc{pl}  \textsc{sens}-be.white \\
\glt `The lower part of its body, its feet, its paws, the inside part are white.' (20-xsar, 23)
    \end{exe}

The approximate locative \forme{-cʰu} is commonly used in particular with the \japhug{lo}{upstream} / \japhug{tʰi}{downstream} and \japhug{kɯ}{east} / \japhug{ndi}{west} locative adverbs (§\ref{sec:locative.adv}), as in (\ref{ex:lo.chu}).\footnote{For the use of \japhug{mɤɕtʂa}{until} to express restrictive focalization in (\ref{ex:lo.chu}), see §\ref{sec:terminative}. }

\begin{exe}
\ex \label{ex:lo.chu}
 \gll  pa jɯl pɕoʁ nɤki, tɯ-ji ɯ-rkɯ nɯra maka me. rɯŋgu kɯnɤ lo-cʰu koŋla ʑo kɯ-ɣɤndʐo mɤɕtʂa tʰi nɯra me \\
 down village side \textsc{filler} \textsc{indef}.\textsc{poss}-field \textsc{3sg}.\textsc{poss}-side \textsc{dem}:\textsc{pl} at.all not.exist:\textsc{fact} pasture also upstream-\textsc{approx}.\textsc{loc} completely \textsc{emph} \textsc{sbj}:\textsc{pcp}-be.cold until downstream \textsc{dem}:\textsc{pl} not.exist:\textsc{fact} \\
\glt  `It is not found down in the villages, near the fields. Even on the pastures, it only exists at high altitudes where it is very cold, not lower.' (15-babW, 116-117)
\end{exe}
%(6:20)
 
The suffix \forme{-cʰu} is itself related to the indefinite locative postpositions \forme{cʰiz} and \forme{scʰiz} `at/in/towards a X, somewhere where X' as in (\ref{ex:schiz.kazGWtnW}), which come from the combination of \forme{-cʰu} with the locative suffix \forme{-z} (degrammaticalized as the locative postposition \forme{zɯ}, see §\ref{sec:core.locative}). For the notation \forme{-iz} (instead of the more etymological \forme{-ɯz}) here, see §\ref{sec:W.i.contrast}. In the case of the variant \forme{scʰiz}, the root of \forme{-cʰu} was both prefixed and suffixed by the locative \forme{-z}.

\begin{exe}
\ex \label{ex:schiz.kazGWtnW}
 \gll  tɕe tɕekɯ\redp{}kɯ ʑo scʰiz sɤstoŋ ʑo scʰiz kɤ-azɣɯt-nɯ ɲɯ-ŋu, nɤki, si ri kɯ-me, rdɤstaʁ ri kɯ-me scʰiz ʑo kɤ-azɣɯt-nɯ ɲɯ-ŋu. \\
 \textsc{lnk} east\redp{}\textsc{emph} \textsc{emph} \textsc{indef}.\textsc{loc} desert \textsc{emph} \textsc{indef}.\textsc{loc} \textsc{aor}:\textsc{east}-reach-\textsc{pl} \textsc{sens}-be \textsc{filler} tree also \textsc{sbj}:\textsc{pcp}-not.exist stone also \textsc{sbj}:\textsc{pcp}-not.exist \textsc{indef}.\textsc{loc}  \textsc{emph}  \textsc{aor}:\textsc{east}-reach-\textsc{pl} \\
 \glt `Further east, they arrived at  a desert, somewhere where there were neither trees nor stones.' (2005 Kunbzang, 169-170)
\end{exe}

The postpositions \forme{cʰiz} and \forme{scʰiz} have reduced forms \forme{cʰɯ} and \forme{scʰɯ} with loss of final \forme{-z} (see \ref{ex:nWrJara.chW} below), probably originally sandhi variants.

\begin{exe}
\ex \label{ex:nWrJara.chW}
\gll nɯ-rɟara cʰɯ pjɤ-k-ɤmdzɯ-ci tɕe, \\
\textsc{3pl}.\textsc{poss}-yard \textsc{indef}.\textsc{loc} \textsc{pst}.\textsc{ipfv}-\textsc{peg}-sit-\textsc{peg} \textsc{lnk} \\
\glt `She was sitting in their yard.'(150907 yingning-zh, 61)
\end{exe}

There is also a rarer disyllabic variant of the approximate locative \forme{cʰizɯ} as in (\ref{ex:chizW}), with the fully syllabic form \forme{zɯ} of the locative postposition.

\begin{exe}
\ex \label{ex:chizW}
\gll tɯ-βzɯr cʰizɯ, nɤki, kɯ-spoʁ ci pjɤ-tu tɕe,  \\
\textsc{indef}.\textsc{poss}-corner \textsc{indef}.\textsc{loc} \textsc{filler} \textsc{sbj}:\textsc{pcp}-have.a.hole \textsc{indef} \textsc{ifr}.\textsc{ipfv}-exist \textsc{lnk} \\
\glt `In one of the corners, there was a hole.' (140510 sanpian sheye-zh, 64)
\end{exe}

Some nouns use a suffix \forme{-cʰɯ} (homophonous with the reduced form of \forme{cʰiz} in \ref{ex:nWrJara.chW}) to indicate direction, and are not compatible with \forme{cʰiz} and other forms: \japhug{mɤpɕoʁ-cʰɯ}{towards the opposite side} and \japhug{tɯ-mɯ-cʰɯ}{towards the sky}, the latter mainly used with the verb \japhug{ru}{look at} to mean `lying on one's back face up (towards the sky)' as in (\ref{ex:tWmWchW}).

\begin{exe}
\ex \label{ex:tWmWchW} 
\gll tɕe tɯ-mɯ-cʰɯ nɯ ɯ-kɯ-ru nɯ ɯ-ɣmɤr ɯ-ŋgɯ ʑo ɕ-pjɤ-lɤt, \\
\textsc{lnk} \textsc{indef}.\textsc{poss}-sky-towards \textsc{dem} \textsc{3sg}.\textsc{poss}-\textsc{sbj}:\textsc{pcp}-look.at \textsc{dem} \textsc{3sg}.\textsc{poss}-mouth \textsc{3sg}.\textsc{poss}-inside \textsc{emph} \textsc{tral}-\textsc{ifr}-release \\
\glt `It dropped (the medicine) inside the mouth of (Gesar), who was lying on his back) face up towards the sky.' (Gesar, 265)
\end{exe}

\subsubsection{\forme{tɕe}: linker or postposition} \label{sec:tCe.postposition}
The word \forme{tɕe} is one of the most common words in Japhug, and it has several different morphosyntactic functions, including that of linker (§\ref{sec:coordination}) and topic marker (§\ref{sec:tCe.topic}). In addition, it is also used as a postposition, expressing both motion and location; it especially commonly occurs with an ablative meaning, as in (\ref{ex:kutɕu.zgo.tCe}) and (\ref{ex:pa.tCe.taR}). Its etymology is discussed in the following section (§\ref{sec:locative.j}).

\begin{exe}
\ex \label{ex:kutɕu.zgo.tCe}
\gll kutɕu zgo tɕe tɕekɯ zgo ku-nɯ-tsɯm ɲɯ-ŋgrɤl ma \\
\textsc{dem}.\textsc{prox}:\textsc{loc} mountain \textsc{loc} east mountain \textsc{ipfv}:\textsc{east}-\textsc{auto}-take.away \textsc{sens}-be.usually.the.case \textsc{lnk} \\
\glt `(The crossoptilon) would take (the weasel) from the mountain here to the mountain over there (on the other side of the river).' (23-qapGAmtWmtW, 88)
\end{exe}

\begin{exe}
\ex \label{ex:pa.tCe.taR}
\gll tɯ-ci nɯra pa tɕe taʁ ʁɟa ʑo ɣɯ-tsɯm pjɤ-ra. \\
\textsc{indef}.\textsc{poss}-water \textsc{dem}:\textsc{pl} down \textsc{loc} up completely \textsc{emph} \textsc{inv}-take.away:\textsc{fact} \textsc{ifr}.\textsc{ipfv}-be.needed \\
\glt `One had to bring water from the lower part (of the valley) upwards.' (140522 RdWrJAt, 9)
\end{exe}

The postposition \forme{tɕe} is also found with temporal adjuncts, as in (\ref{ex:qartsW.tCe.YWjpum}).

\begin{exe}
\ex \label{ex:qartsW.tCe.YWjpum}
\gll qartsɯ tɕe ɲɯ-jpum ftɕar tɕe tu-mbro ŋu. \\
winter \textsc{loc} \textsc{ipfv}-be.thick summer \textsc{loc} \textsc{ipfv}-be.high be:\textsc{fact} \\
\glt `It grows thicker in winter, and taller in summer. (07-tAtho, 22)
\end{exe}

It appears in the expression \forme{ɯ-sɯm tɕe} `in his opinion, in his mind' as in (\ref{ex:asWm.tCe}) and (\ref{ex:WsWm.tCe}); the core locative postpositions are not used in this meaning.

\begin{exe}
\ex \label{ex:asWm.tCe}
\gll aʑo a-sɯm tɕe, nɯ-ʁrɯ ʑo ɣɤʑu ɕti tɕe \\
\textsc{1sg} \textsc{1sg}.\textsc{poss}-mind \textsc{loc} \textsc{3pl}.\textsc{poss}-horn \textsc{emph} exist:\textsc{sens} be.\textsc{aff}:\textsc{fact} \textsc{lnk} \\
\glt `In my opinion, (since) they have horns, (they should be able to fight the predators off).' (20-RmbroN, 64)
\end{exe}


\begin{exe}
\ex \label{ex:WsWm.tCe}
\gll tɕe ɯʑo ɯ-sɯm tɕe tɕe tu-tsɯm tɕe tɕendɤre iʑora ji-sɤtɕʰa ra lonba ʑo tɯ-ci ɲɯ-sɯ-ɤβze to-ʁmɯɣ. \\
\textsc{lnk} \textsc{3sg} \textsc{3sg}.\textsc{poss}-mind \textsc{loc} \textsc{lnk} \textsc{ipfv}:\textsc{up}-take \textsc{lnk} \textsc{lnk} \textsc{1pl} \textsc{1pl}.\textsc{poss}-place \textsc{pl} all \textsc{emph} \textsc{indef}.\textsc{poss}-water \textsc{ipfv}-\textsc{caus}-become[III] \textsc{ifr}-have.the.intention \\
\glt `In his mind, (the snake) wanted to take the water upwards and transform our whole area into water.' (150820 qaprANar, 20)
\end{exe} 

Distinguishing between the uses of \forme{tɕe} as a postposition and as a topic marker (§\ref{sec:tCe.topic}) is not always trivial; it is analyzed as a topic marker when it can be replaced by \forme{nɯ} §\ref{sec:nW.topic}, or when two \forme{tɕe} appear in a row as in (\ref{ex:WsWm.tCe}): in this case, the first one is a postposition and the second one a topic marker.


\subsubsection{Traces of the locative suffix \forme{*-j}} \label{sec:locative.j}
Situ has a locative suffix \forme{-j}, also used in the possessive construction (\citealt[325--330]{linxr93jiarong}), which has disappeared in Japhug, though a few traces remain.

The form \forme{tɕe}, which is mainly used as a linker (§\ref{sec:coordination}) and also occurs as postposition (§\ref{sec:tCe.postposition}) and as a topic marker (§\ref{sec:tCe.topic}), probably originates from the combination of the locative postposition \forme{tɕu} and the locative suffix \forme{*-j}, with vowel merger at a stage preceding the sound change \forme{*o} $\rightarrow$ \forme{u}  (\forme{*tɕo-j} $\rightarrow$ \forme{tɕe}; see §\ref{sec:historical.phono} for a discussion of these sound changes).

Another trace of the locative suffix \forme{*-j} is found in the linker \japhug{qʰe}{then} and the time ordinal \japhug{qʰuj}{this afternoon} (§\ref{sec:time.ordinals}), combining the relator noun \japhug{ɯ-qʰu}{after} (§\ref{sec:relator.temporal}) with the coda \forme{*-j}, in the former with vowel fusion (an earlier lexicalization), and the latter without fusion. The form  \japhug{qʰuj}{this afternoon}  shows that the suffix \forme{*-j} was still productive in Japhug after the sound change \forme{*o} $\rightarrow$ \forme{u} took place.

The interrogative pronoun \japhug{ŋoj}{where}, variant of \japhug{ŋotɕu}{where} (§\ref{sec:NotCu}), also has a trace of the \forme{*-j} suffix without vowel fusion.

Finally, the suffix \forme{-re} in the locative pronouns \japhug{kɯre}{here}, \japhug{nɯre}{there} and related forms is most probably the plural marker \forme{ra} (§\ref{sec:plural.determiners}) to which the locative \forme{*-j} has been added, with the same vowel fusion as in the forms above (§\ref{sec:locative.pronoun}).

\subsection{Comitative} \label{sec:comitative} 
Postpositional phrases with the comitative postposition \japhug{cʰo}{and, with} and its variants \forme{cʰondɤre} and \forme{cʰonɤ} (comprising the additive \forme{nɤ} and the linker \forme{ndɤre}) are selected by verbs with non-singular subjects (§\ref{sec:intrinsically.n.sg.subject}), including \japhug{naχtɕɯɣ}{be the same} (§\ref{sec:identity.modifier}), \japhug{amɯmi}{be in good terms with}  (\ref{ex:cho.kWnaXtCWG}), \japhug{rɤkrɤz}{discuss} and reciprocal verbs (§\ref{sec:redp.reciprocal}).

\begin{exe}
\ex \label{ex:cho.kWnaXtCWG}
\gll [ɯʑo cʰo] kɯ-naχtɕɯɣ [sɯjno, xɕaj ma mɤ-kɯ-ndza nɯra cʰonɤ] amɯmi-nɯ tɕe, \\
\textsc{3sg} \textsc{comit} \textsc{sbj}:\textsc{pcp}-be.the.same vegetables grass apart.from \textsc{neg}-\textsc{sbj}:\textsc{pcp}-eat \textsc{dem}:\textsc{pl} \textsc{comit} be.in.good.terms:\textsc{fact}-\textsc{pl} \textsc{lnk} \\
\glt `(The rabbit) is in good terms with (the animals) which eat only grass and vegetables like him.' (04-qala2, 8)
\end{exe}

\begin{exe}
\ex \label{ex:cho.torAkrAzndZi}
\gll [kʰu chondɤre] mbro ni to-rɤkrɤz-ndʑi. \\
tiger \textsc{comit} horse \textsc{du} \textsc{ifr}-discuss-\textsc{du} \\
\glt `The tiger and the horse had a discussion.' (20-tArka, 32)
\end{exe}

Postpositional phrases in \forme{cʰo} can be considered to be oblique arguments in the sense that they are relativized using the oblique participle (§\ref{sec:other.oblique.participle.relatives}, §\ref{sec:comitative.relativization}). However, verbs that select \forme{cʰo} phrases index not only the intransitive subject proper, but the sum of the subject and the \forme{cʰo} phrase, which can be in the dual as in (\ref{ex:cho.YWnaXtCWGndZi}) (the white birch and the red birch) or in the plural (\ref{ex:cho.kWnaXtCWG}) (the rabbit and the other animals). 

\begin{exe}
\ex \label{ex:cho.YWnaXtCWGndZi}
\gll tɕe ɯ-rqʰu nɯ ɣɯrni laʁma ɯ-ŋgɯ nɯ [sɤjku cʰo] ɲɯ-naχtɕɯɣ-ndʑi ri\\
\textsc{lnk} \textsc{3sg}.\textsc{poss}-bark \textsc{dem} be.red:\textsc{fact} apart.from.the.fact \textsc{3sg}.\textsc{poss}-inside \textsc{dem} birch \textsc{comit} \textsc{sens}-be.the.same-\textsc{du} \textsc{lnk} \\
\glt `Apart from the fact that its bark is red, it is identical in the inside with the birch.' (06-mbrAj, 13)
\end{exe}

The verb \japhug{naχtɕɯɣ}{be the same} with a \forme{cʰo} phrase can be used in an equative construction (§\ref{sec:nmlz.equative}).

 Apart from the function presented above, \japhug{cʰo}{and, with} is commonly used to link together two nouns inside a single noun phrase, as in (\ref{ex:awW.cho.aRi}). In this case too, the main verb of the clause indexes the whole noun phrase, comprising the sum of referents designated by the nouns linked by \forme{cʰo}.

\begin{exe}
\ex \label{ex:awW.cho.aRi}
\gll a-wɯ cʰo a-ʁi ni cʰɯ-ɣi-ndʑi ra ma ʑɤni-sti kɤ-rɤʑi mɤ-cʰa-ndʑi tɕe, \\
\textsc{1sg}.\textsc{poss}-grand.father \textsc{comit} \textsc{1sg}.\textsc{poss}-younger.sibling \textsc{du} \textsc{ipfv}:\textsc{downstream}-come-\textsc{du} be.needed:\textsc{fact} \textsc{lnk} \textsc{3du}-alone \textsc{inf}-stay \textsc{neg}-can:\textsc{fact}-\textsc{du} \textsc{lnk} \\ 
\glt `My grandfather and my younger brother have to come, they cannot stay by themselves.' (2011-05-nyima, 209)
\end{exe}

The marker \forme{cʰo} can also link verb phrases and even entire clauses (see §\ref{sec:neutral.addition} and \citealt[313]{jacques14linking}).

Given the apparently equal status of the two linked nouns in (\ref{ex:awW.cho.aRi}), in particular with regard to indexation, it is legitimate to wonder whether analyzing it as a postposition makes more sense than considering it to be a coordinator; this question is explored in §\ref{sec:coordinator.cho}). 

A \forme{cʰo} phrase can be followed by the associative plural marker \forme{ra} (§\ref{sec:number.determiners}) as in (\ref{ex:cho.ra.kW}) to mean `et cetera', and the whole phrase can take case marking such as ergative.

\begin{exe}
\ex \label{ex:cho.ra.kW}
\gll tɕeri ɯʑo ndɤre, qajdo cʰo ra kɯ ndɤ tú-wɣ-ndza ɕti \\
but \textsc{3sg} \textsc{advers} crow \textsc{comit} \textsc{pl} \textsc{erg} \textsc{advers} \textsc{ipfv}-\textsc{inv}-eat be.\textsc{aff}:\textsc{fact} \\
\glt `But it is eaten by crows and other (animals).' (26-NalitCaRmbWm, 140)
\end{exe}

The ergative \forme{kɯ} is however optional on comitative phrases, as shown by (\ref{ex:WpW.ra.cho}), where ergative marking would be expected on the phrase \forme{ɯ-pɯ ra cʰo}.

\begin{exe}
\ex \label{ex:WpW.ra.cho}
\gll [ɯ-pɯ ra cʰo] tɯtɯrca to-ndza-nɯ tɕe to-nɯ-ʑɣɤ-ɕɯ-fka-nɯ ʑo ɲɯ-ŋu. \\
\textsc{3sg}.\textsc{poss}-young \textsc{pl} \textsc{comit} together \textsc{ifr}-eat-\textsc{pl} \textsc{lnk} \textsc{ifr}-\textsc{auto}-\textsc{refl}-\textsc{caus}-be.full-\textsc{pl} \textsc{emph} \textsc{sens}-be \\
\glt `(The eagle) ate them together with its fledglings and they ate to their full.' (huli yu shanying-zh, 28)
\end{exe}

A postpositional comitative phrase can also serve as a dual or plural possessor, as if from a complex noun phrase `$X$ \forme{cʰo} $Y$' with elided Y element, as in (\ref{ex:cho.ndZime}).\footnote{Note that (\ref{ex:cho.ndZime}) does not mean `There is his wife and their daughter' (dual indexation would be expected on the verb).} See §\ref{sec:coordinator.cho} for additional discussion.

\begin{exe}
\ex \label{ex:cho.ndZime}
\gll tɕe ɯ-rʑaβ cʰo ndʑi-me ci tu tɕe, \\
\textsc{lnk} \textsc{3sg}.\textsc{poss}-wife \textsc{comit} \textsc{3du}.\textsc{poss}-daughter one exist:\textsc{fact} \textsc{lnk} \\
\glt `He and his wife have a daughter.' (14-siblings, 313)
\end{exe}
 
The comitative is also used in the simultaneous action construction (§\ref{sec:simultaneous.action.nominal}, §\ref{sec:simult.action.nominal.Bzu})
 
\subsection{Additive} \label{sec:additive.nA} 
The additive adposition \forme{nɤ}, possibly from Tibetan \tibet{ན་}{na}{locative}, appears after the protasis of some conditional clauses (§\ref{sec:real.conditional}), but it also occurs in direct adjacency between two nouns (generally identical ones), most commonly to express  repeated action as in (\ref{ex:tArpi.nA.tArpi}) and (\ref{ex:cha.nA.cha}).
 
\begin{exe}
\ex  \label{ex:tArpi.nA.tArpi}
\gll tɤ-rpi nɤ tɤ-rpi ʑo ɲɯ-sɯ-βzu-nɯ ɲɯ-ŋu tɕe, \\
\textsc{indef}.\textsc{poss}-sutra \textsc{add} \textsc{indef}.\textsc{poss}-sutra \textsc{emph} \textsc{ipfv}-\textsc{caus}-make-\textsc{pl} \textsc{sens}-be \textsc{lnk} \\
\glt `They ask (lamas) to chant sutras after sutras.' (2003kandZislama, 112)
\end{exe} 

 \begin{exe}
\ex  \label{ex:cha.nA.cha}
\gll  tɕe ɯ-χti nɯ kɯ cʰa ntsɯ ku-tsʰi pɯ-ŋu tɕe cʰa nɤ cʰa ku-tsʰi pjɯ-ɕti tɕe \\
\textsc{lnk} \textsc{3sg}.\textsc{poss}-companion \textsc{dem} \textsc{erg} alcohol always \textsc{ipfv}-drink \textsc{pst}.\textsc{ipfv}-be \textsc{lnk} alcohol add alcohol \textsc{ipfv}-drink \textsc{ipfv}-be.\textsc{aff} \textsc{lnk} \\
\glt  `Her husband used to drink all the time, drank alcohol again and again.' (17-lhazgron, 64)
\end{exe} 

It can also be interpreted as gradual increase (\ref{ex:taR.nA.taR}) and/or the meaning `all the way' with locative nouns as in (\ref{ex:tsxu.nA.tsxu}).

\begin{exe}
\ex  \label{ex:taR.nA.taR}
\gll tɯ-ci nɯ taʁ nɤ taʁ, taʁ nɤ taʁ tu-ɣi pjɤ-ɕti \\
\textsc{indef}.\textsc{poss}-water  \textsc{dem} up \textsc{add}  up up \textsc{add} up \textsc{ipfv}:\textsc{up}-come \textsc{ifr}.\textsc{ipfv}-be.\textsc{aff} \\
\glt `The water was raising up and up.' (31-deluge, 19)
\end{exe} 

\begin{exe}
\ex  \label{ex:tsxu.nA.tsxu}
\gll  tʂu nɤ tʂu pjɤ-ɕe qʰe, \\
path \textsc{add} path \textsc{ifr}:\textsc{down}-go \textsc{lnk} \\
\glt `He went all the way down.' (140511 alading-zhn 105)
\end{exe} 

The additive can also occur between finite verbs with a similar range of meanings (§\ref{sec:distributed.action}), with numerals and counted nouns (§\ref{sec:CN.repetition}) to express distributivity, and with ideophones to describe a rhythmically occurring action (§\ref{sec:ideo.III}).

\subsection{Standard marker} \label{sec:comparative} 
Japhug has several postpositions that are mainly used to mark the standard in the comparative construction. The most common one is \japhug{sɤz}{compared with}, but the variants \forme{staʁ}, \forme{sɤznɤ}, \forme{staʁnɤ}, \forme{sɯstaʁ} (\ref{ex:WCGa.anWxtCi} in §\ref{sec:preverb.adjectives.size})  \forme{χtanɤ} (\ref{ex:nWBde.wo}, §\ref{sec:fsp.imp}) and \forme{sɯχta} are also attested (see §\ref{sec:denominal.adverb.s.prefix} concerning their etymology). Their relative frequency appears to be speaker-dependent, and no meaningful difference could be detected between them. 

In the comparative construction (§\ref{sec:sAz.kW}), the comparee is the intransitive subject of the main verb (the parameter, generally an adjectival stative verb) and is indexed on the verb. The comparee is either in the absolutive or in the ergative (§\ref{sec:comparee.kW}). The standard is necessarily marked by one of the postpositions listed above, and cannot be indexed on the main verb. Neither the standard not the comparee are required to be overt. An adjectival stative verb with a standard postpositional phrase as in  (\ref{ex:sAznA.YWwxti})  is a well-formed comparative construction. Examples like (\ref{ex:aZo.YWwxti}) with overt comparee and standard are rarer.

\begin{exe}
\ex \label{ex:sAznA.YWwxti}
\gll  qandʑɣi sɤznɤ ɲɯ-wxti, qaliaʁ sɤznɤ ɲɯ-xtɕi \\
falcon \textsc{comp} \textsc{sens}-be.big eagle \textsc{comp} \textsc{sens}-be.small \\
\glt `It is bigger than a falcon, and smaller than an eagle.' (2011-08-kuwu, 40-41)
\end{exe}

\begin{exe}
\ex \label{ex:aZo.YWwxti}
\gll ɯʑo nɯ aʑo sɤz tɯ-xpa wxti  \\
\textsc{3sg} \textsc{dem} \textsc{1sg} \textsc{comp} one-year be.big:\textsc{fact} \\
\glt `She is one year older than me.' (12-BzaNsa, 94)
\end{exe}

The standard marker \forme{sɤz} (and its variants) also occurs in a  construction expressing progressive increase throughout the time, where a time counted noun like \japhug{tɯ-sŋi}{one day} or \japhug{tɯ-xpa}{one year} is followed by the standard marker and then repeated, as \forme{tɯ-xpa sɤz tɯ-xpa} `more year after year' in (\ref{ex:tWxpa.sAz.tWxpa}). This construction, although attested in non-translated texts, is more common in texts from Chinese, where it calques the construction \ch{一年比一年}{yīnián bǐ yīnián}{more year after year}. The more idiomatic Japhug construction to express the same meaning is through partial reduplication of the first syllable of the main verb (§\ref{sec:redp.gradual.increase}).
 
 \begin{exe}
 \ex \label{ex:tWxpa.sAz.tWxpa}
 \gll nɯ-jwaʁ nɯ, [...] tɯ-xpa sɤz tɯ-xpa lu-dɤn ŋu ma \\
 \textsc{3pl}.\textsc{poss}-leaf \textsc{dem} { } one-year \textsc{comp} one-year \textsc{ipfv}-be.many be:\textsc{fact} \textsc{lnk} \\
\glt  `There are more needles (leaves) each year.' (08-saCW, 17)
\end{exe}
 
 The standard markers can also be used with subordinate clauses (§\ref{sec:incremental.addition}). The standard marker with the distal demonstrative \forme{nɯ sɤznɤ} has the meaning `rather than that, could ... as well' as in (\ref{ex:nW.sAznA.arca}).  
 
 \begin{exe}
 \ex \label{ex:nW.sAznA.arca}
 \gll  nɤ-mu kɤ-fsraŋ mɤ-tɯ-cʰa tɕe, nɯ sɤznɤ, a-rca jɤ-ɣi tɕe, a-rca, nɤki, laχɕi pɯ-βzjoz \\
 \textsc{2sg}.\textsc{poss}-mother \textsc{inf}-protect \textsc{neg}-2-can:\textsc{fact} \textsc{lnk} \textsc{dem} \textsc{comp} \textsc{1sg}.\textsc{poss}-following \textsc{imp}-come \textsc{lnk} \textsc{1sg}.\textsc{poss}-following \textsc{filler} trade \textsc{imp}-learn \\
\glt `You cannot save your mother, rather than that, come with me to learn  some abilities.' (150826 baoliandeng-zh, 142-143)
\end{exe}

This phrase can also be used as a scalar marker `even' with scope over the following clause, as in (\ref{ex:nW.sAznA.chaa}), and occurs in incremental additive constructions `not only $X$, but also $Y$' (§\ref{sec:incremental.addition}).
 

 \begin{exe}
 \ex \label{ex:nW.sAznA.chaa}
 \gll ki kɤ-rtsi kɯ-tu me nɤ, aʑo nɯ sɤznɤ, nɤkinɯ, kʰa kɯ-qanɯ\redp{}nɯ ɯ-ŋgɯ zɯ, nɤkinɯ, tɯ-ɕpɤβ kɯβde-rzɯɣ tɤ-kɤ-lɤt nɯnɯ ku-sɤlɤɣi-a cʰa-a ɕti nɤ! \\
 \textsc{dem}.\textsc{prox} \textsc{inf}-count \textsc{sbj}:\textsc{pcp}-exist not.exist:\textsc{fact} \textsc{sfp} \textsc{1sg} \textsc{dem} \textsc{comp} \textsc{filler} house   \textsc{sbj}:\textsc{pcp}-\textsc{emph}\redp{}be.dark \textsc{3sg}.\textsc{poss}-inside \textsc{loc}  \textsc{filler} \textsc{indef}.\textsc{poss}-corpse four-section \textsc{aor}-\textsc{obj}:\textsc{pcp}-release \textsc{dem} \textsc{ipfv}-combine-\textsc{1sg} can:\textsc{fact}-\textsc{1sg} be.\textsc{aff}:\textsc{fact} \textsc{sfp}  \\
\glt  `(What you ask) is nothing, I am even able to put together a corpse that had been cut into four pieces in a dark house.'  (140512 alibaba-zh, 170)
\end{exe}

The phrase \forme{nɯ sɤznɤ} is also used as a marker of adversative topic as in (\ref{ex:nW.sAznA.YWwxti}), where it can be replaced by the marker \forme{ʁo} (§\ref{sec:adversative.topic}).

\begin{exe}
\ex \label{ex:nW.sAznA.YWwxti}
\gll kɯki sɤlaŋpʰɤn ki nɯ sɤznɤ ɲɯ-wxti wo tɕe kʰa ju-nɯ-tsɯm-a tɕe, \\
\textsc{dem}.\textsc{prox} basin \textsc{dem}.\textsc{prox}  \textsc{dem} \textsc{comp}  \textsc{sens}-be.big \textsc{sfp} \textsc{lnk} house \textsc{ipfv}-\textsc{vert}-take-\textsc{1sg} \textsc{lnk} \\
\glt `This basin is really big, I will take it home.' (150831 jubaopen, 22)
\end{exe}

\subsection{Exceptive} \label{sec:exceptive} %\japhug{ma}{apart from} laʁma mɯma
The exceptive postposition \japhug{ma}{apart from} and its reduplicated variant \forme{mɯma} are not selected by any verb, and only used in adjunct postpositional phrases as in (\ref{ex:kWm.ci.mWma}).

 \begin{exe}
 \ex \label{ex:kWm.ci.mWma}
 \gll kɯm ci mɯma nɯnɯ tɕe znde ʁɟa ʑo ɕti \\
 door one apart.from \textsc{dem} \textsc{lnk} wall completely \textsc{emph} be.\textsc{aff}:\textsc{fact} \\
 \glt `Apart from one door, there are walls everywhere.' (2011-11-kha, 40)
\end{exe}

The exceptive \japhug{ma}{apart from} is used in particular in restrictive focus constructions (§\ref{sec:restrictive.focus}).

When the scope of the restrictive construction is on an entire clause rather than a single noun phrase, the clause is followed by the linker \forme{ma} (homophonous with the exceptive) and an exceptive phrase limited to the demonstrative pronoun \forme{nɯ} (here in resumptive use, coreferent with the entire preceding clause) and the postposition \forme{ma}, as in (\ref{ex:ra.ma.nW.ma}). The first \forme{ma} in this construction is not to be analyzed as the postposition: while it is possible to reduplicated the second one as in \forme{ma nɯ mɯma} (example \ref{ex:mAspea.ma.nW.ma}), reduplication of the first \forme{ma} is not attested.

 \begin{exe}
 \ex \label{ex:ra.ma.nW.ma}
 \gll [aʑɯɣ ɯ-ɕa ra] ma nɯ ma kɯ-ra me \\
 \textsc{1sg}.\textsc{gen} \textsc{3sg}.\textsc{poss}-meat be.needed:\textsc{fact} \textsc{lnk} \textsc{dem} apart.from \textsc{sbj}:\textsc{pcp}-be.needed not.exist:\textsc{fact} \\
 \glt `I want its meat, and nothing else.' (02-deluge2012, 14)
\end{exe}

 \begin{exe}
 \ex \label{ex:mAspea.ma.nW.ma}
 \gll tɤ-pɤtso kɯ-ɣɤwu ʑo kɤ-nɯɕpɯz mɤ-spe-a ma nɯ mɯma spe-a \\
 \textsc{indef}.\textsc{poss}-child \textsc{sbj}:\textsc{pcp}-cry \textsc{emph} \textsc{inf}-imitate \textsc{neg}-be.able[III]:\textsc{fact}-\textsc{1sg} \textsc{lnk} \textsc{dem} apart.from be.able[III]:\textsc{fact}-\textsc{1sg}  \\
\glt  `I cannot imitate a baby crying, but apart form that I can imitate (all animal sounds).' (27-kikakCi, 143)
\end{exe}

\subsection{Terminative} \label{sec:terminative}  
The postposition \japhug{mɤɕtʂa}{until} is used after noun phrases to indicate temporal (\ref{ex:RnWpArme.mACtsxa}) or locative (\ref{ex:akW.mACtsxa}) limit. It can be used in opposition with the egressive postpositions (see \ref{ex:tWmAmke.mACtsxa} in §\ref{sec:egressive}) or with \japhug{kóʁmɯz}{only after} (example \ref{ex:sqamNuxpa.koRmWz} in §\ref{sec:temporal.postpositions}).

\begin{exe}
\ex \label{ex:RnWpArme.mACtsxa}
 \gll tɤ-pɤtso kɯ-dɤn nɯra tɕe, tɯ-pɤrme, ʁnɯ-pɤrme jamar mɤɕtʂa tɯ-nɯ ku-tsʰi-nɯ. \\
 \textsc{indef}.\textsc{poss}-child \textsc{sbj}:\textsc{pcp}-be.many \textsc{dem}:\textsc{pl} \textsc{lnk} one-year.old two-year.old about until \textsc{indef}.\textsc{poss}-breast \textsc{ipfv}-drink-\textsc{pl} \\
 \glt `In (families where) children are many, (mothers) breastfeed (the children) until (they are) one or two years old.' (140426 tApAtso kAnWBdaR, 13)
\end{exe}

\begin{exe}
\ex \label{ex:akW.mACtsxa}
 \gll akɯ mɤɕtʂa ɣɯ-ku-ta-lɤt \\ 
east until \textsc{cisl}-\textsc{ipfv}:\textsc{east}-1\fl{}2-release \\
\glt  `I come with you (see you off) until the (land of the) east.' (28-smAnmi, 220)
\end{exe}

In combination with the demonstrative \japhug{nɯ}{that}, \japhug{mɤɕtʂa}{until} means `otherwise', as in (\ref{ex:nW.mACtsxa}).

\begin{exe}
\ex \label{ex:nW.mACtsxa}
 \gll   kɤ-sɤŋo ʁɟa qʰe,  nɯ-mtɕʰi kɤ-χpjɤt ʁɟa kɯ kú-wɣ-spa ɕti.  nɯ mɤɕtʂa mɤ-kʰɯ. \\
 \textsc{inf}-hear completely \textsc{lnk} \textsc{3pl}.\textsc{poss}-mouth \textsc{inf}-observe completely \textsc{erg} \textsc{ipfv}-\textsc{inv}-be.able be.\textsc{aff}:\textsc{fact} \textsc{dem} until \textsc{neg}-be.possible:\textsc{fact} \\
\glt `(In order to learn the Tshobdun language, since it has no writing system), one has no choice but to listen and observe people's mouth to learn it, otherwise it is not possible.' (150901 tshuBdWnskAt, 41-44)
\end{exe}

With a verb in negative form, the terminative can express restrictive focalization of locative and temporal adjuncts (§\ref{sec:restrictive.focus}), as in (\ref{ex:01.20hao.mACtsxa}).

\begin{exe}
\ex \label{ex:01.20hao.mACtsxa}
 \gll  iʑora pɤjkʰu <yiyue> <ershiduohao> mɤɕtʂa mɯ́j-lɤt-nɯ kʰi. \\
 \textsc{1pl} still January twentieth.plus until \textsc{neg}:\textsc{sens}-release-\textsc{pl} hearsay \\
 \glt `(At) our (place), they will only have (vacations) on the twentieth something of January, they say.' = `They won't have (vacations) until the twentieth of January.' (conversation, 14-12-24)
 \end{exe}
 
The terminative postposition \forme{mɤɕtʂa} can also be used to build temporal subordinate clauses (§\ref{sec:terminative.clause}).

\subsection{Egressive} \label{sec:egressive}  
There are six egressive postpositions in Japhug, which are built by combining the root \forme{ɕaŋ-/ɕoŋ-} (among the words where \forme{-aŋ} and \forme{-oŋ} are in free variation; the variant \forme{ɕaŋ-} is generalized in the orthography,  see §\ref{sec:aN.oN.free}) with either the root of locative relator nouns (§\ref{sec:relator.location}) or orientation adverbs (§\ref{sec:preverbs.adverbs}) as shown in \tabref{tab:egressive}. There is a one-to-one relationship between the orientations of these postpositions and the six definite orientations found in verb morphology (§\ref{sec:preverbs.adverbs}).

The egressive postpositions are mainly used with noun phrases of location expressing length or height (\ref{ex:tWfsu.CaNtaR}) or a reference point marking a limit (\ref{ex:praRwW.CaNdi}). However, \japhug{ɕaŋtaʁ}{up from} and \japhug{ɕaŋpa}{down from} can also follow noun phrases referring to time reference or durations, as in (\ref{ex:kWmNArZaR}, \ref{ex:tWGjAn.CaNtaR.pWme}) or more generally any quantity (\ref{ex:XsWm.CaNtaR}). No examples of these postpositions following finite subordinate clauses have been found.


\begin{exe}
\ex \label{ex:tWfsu.CaNtaR}
 \gll tɯrme tɯ-fsu ɕaŋtaʁ tu-mbro mɤ-cʰa. \\
people \textsc{genr}.\textsc{poss}-same.size up.from \textsc{ipfv}:\textsc{up}-be.high \textsc{neg}-can:\textsc{fact} \\
\glt `It cannot grow higher than a person.'(11-qarGW, 29)
\end{exe}
 
\begin{exe}
\ex \label{ex:praRwW.CaNdi}
 \gll ma kɯtɕimke nɯnɯtɕu, akɯ ku-ru tɕe, praʁwɯ ɕaŋdi sɤ-mto, 
andi tɕe tɕe, ɕɯfco ɕaŋkɯ nɯ sɤ-mto tɕe, \\
\textsc{lnk}  \textsc{topo} \textsc{dem}:\textsc{loc} east \textsc{ipfv}:\textsc{east}-look.at \textsc{lnk}  \textsc{topo} west.from \textsc{prop}-see:\textsc{fact} west \textsc{lnk} \textsc{lnk}  \textsc{topo} east.from \textsc{dem} \textsc{prop}-see:\textsc{fact} \textsc{lnk} \\
\glt `In Kuchimke, looking towards the east, (the areas) to the west of Praqwu are visible, and in the west, (the areas) to the east of Shyufkyo are visible.' (150904 tshAcim, 30)
\end{exe}

\begin{exe}
\ex \label{ex:kWmNArZaR}
 \gll kɯmŋɤsqɤ-rʑaʁ ɕaŋtaʁ cʰɯ-mdɯ-nɯ mɤ-ŋgrɤl tu-ti-nɯ ɲɯ-ŋu.  \\
 fifty-day up.from \textsc{ipfv}-live.up.to-\textsc{pl} \textsc{neg}-be.usually.the.case:\textsc{fact} \textsc{ipfv}-say-\textsc{pl} \textsc{sens}-be \\
\glt `They cannot live more than fifty days, it is said.' (26-GZo, 41)
\end{exe}

\begin{exe}
\ex \label{ex:tWGjAn.CaNtaR.pWme}
 \gll tɕe stu kɯ-dɤn nɯnɯ tɯ-xpa [tɯ-ɣjɤn ɕaŋtaʁ] kɯ-ɤmɯtɯɣ pɯ-me \\
 \textsc{lnk} most \textsc{sbj}:\textsc{pcp}-be.many \textsc{dem} one-year one-time up.from \textsc{genr}-\textsc{recip}:meet \textsc{pst}.\textsc{ipfv}-not.exist \\
 \glt `At most, we would only meet once per year (we had no opportunity to meet more than once a year).' (12-BzaNsa, 42)
 \end{exe}

\begin{exe}
\ex \label{ex:XsWm.CaNtaR}
 \gll ɯ-pɯ nɯnɯ χsɯm ɕaŋtaʁ tu mɯ́j-ŋgrɤl\\
\textsc{3sg}.\textsc{poss}-young \textsc{dem} three up.from exist:\textsc{fact} \textsc{neg}:\textsc{sens}-be.usually.the.case\\
\glt `It does not usually have more than three offsprings.' (2011-08-kuwu, 14)
\end{exe}

The postposition \japhug{ɕaŋtaʁ}{up from} is by far more common than all the other ones, and is often combined with a negative predicate in a comparative construction (meaning `at most, no more than ...', §\ref{sec:egressive.comparative}, §\ref{sec:negative.existential.superlative}), as shown by (\ref{ex:tWfsu.CaNtaR}), (\ref{ex:kWmNArZaR}), (\ref{ex:tWGjAn.CaNtaR.pWme}) and (\ref{ex:XsWm.CaNtaR}) above. 


In addition to the locational, temporal and quantitative meanings presented above, \japhug{ɕaŋtaʁ}{up from} and \japhug{ɕaŋpa}{down from}  can be used to refer to relative age (down from the upper generation, up from the lower generation, as shown in \ref{ex:CaNpa.CaNtaR}) or social status (up from the lowliest person, as in \ref{ex:BGAru.ci.CaNtaR}).


 \begin{exe}
	\ex \label{ex:CaNpa.CaNtaR}
	\gll nɤ-mu nɤ-wa ni ɕaŋpa, a-ɣe ɕaŋtaʁ tɤ-rɯndzaŋspa-nɯ je! \\
	\textsc{2sg}.\textsc{poss}-mother 	\textsc{2sg}.\textsc{poss}-father \textsc{du} down.from \textsc{1sg}.\textsc{poss}-grandchild up.from \textsc{imp}-be.careful \textsc{sfp} \\
	\glt `Be careful, (all of you) from your parents (in the upper generation) to my grandson (in the lower one).' (conversation, 29-09-2020)
\end{exe}

 \begin{exe}
	\ex \label{ex:BGAru.ci.CaNtaR}
	\gll  wortɕʰi ʑo βɣɤru ci ɕaŋtaʁ ʑo tɤ-sɯ-ɤwɯwum-nɯ tɕe, \\
	please \textsc{emph} miller \textsc{indef} up.from \textsc{emph} \textsc{ifr}-\textsc{caus}-\textsc{recip}:gather-\textsc{pl} \textsc{lnk} \\
	\glt `Please gather (everybody), from the miller (the lowliest of servants) up (to the highest ranking person).' (2003 kandZislama, 174)
\end{exe}

\begin{table}
\caption{Egressive postpositions} \label{tab:egressive}  
\begin{tabular}{llllll}
\lsptoprule
Postposition & Relator noun & Orientation adverb\\
\midrule
\japhug{ɕaŋtaʁ}{up from} & \japhug{ɯ-taʁ}{up, top}& \\
\japhug{ɕaŋpa}{down from} & \japhug{ɯ-pa}{down, bottom}& \\
\japhug{ɕaŋlo}{upstream from} & & \japhug{alo}{upstream} \\
\japhug{ɕaŋtʰi}{downstream from} & & \japhug{atʰi}{upstream} \\
\japhug{ɕaŋkɯ}{east from} & & \japhug{akɯ}{east} \\
\japhug{ɕaŋdi}{west from} & & \japhug{andi}{west} \\
\lspbottomrule
\end{tabular}
\end{table}

The egressive postpositions can be used in contrast with the terminative \japhug{mɤɕtʂa}{until}, as in (\ref{ex:tWmAmke.mACtsxa}).

\begin{exe}
\ex \label{ex:tWmAmke.mACtsxa}
 \gll tɯ-mke ɕaŋpa tɕe tɕe ki tɯ-mɤmke mɤɕtʂa kɯ-zɣɯt kɯ-rɲɟi pjɯ-ŋu ra.  \\
\textsc{indef}.\textsc{poss}-neck down.from \textsc{lnk} \textsc{lnk} \textsc{dem}.\textsc{prox} \textsc{indef}.\textsc{poss}-ankle until \textsc{sbj}:\textsc{pcp}-reach  \textsc{sbj}:\textsc{pcp}-be.long \textsc{ipfv}-be be.needed:\textsc{fact} \\  
\glt  `(Tibetan clothes) have to be long (enough) so as to reach the ankle down from the neck.' (30-tWNga, 3)
\end{exe}

As other postpositional phrases, egressive phrases followed by demonstratives (§\ref{sec:demonstrative.determiners}) mean `the person(s)/thing(s) from X', with a locative (\ref{ex:TshuBdWn.CaNlo}) or temporal (\ref{ex:sqamnW.pArme.CaNpa}) interpretation.

 \begin{exe}
\ex \label{ex:TshuBdWn.CaNlo}
 \gll iʑora kɯ, nɤki, tsʰuβdɯn ɕaŋlo nɯra `stɤtpa-pɯ' tu-ti-j ŋu. \\
 \textsc{1pl} \textsc{erg} \textsc{filler}  \textsc{topo} upstream.from \textsc{dem}:\textsc{pl} pl.n.-person \textsc{ipfv}-say-\textsc{1pl} be:\textsc{fact} \\
\glt `We call the people (who live) in Tshobdun and further upstream `Stotpa'.' (23-tCAphW, 14)
\end{exe}

 \begin{exe}
\ex \label{ex:sqamnW.pArme.CaNpa}
 \gll sqamnɯ-pɤrme ɕaŋpa nɯnɯ tɯ-ɕɣa tu-nɤsci kʰɯ \\
 twelve-years.old down.from \textsc{dem} \textsc{genr}.\textsc{poss}-tooth \textsc{ipfv}-exchange be.possible:\textsc{fact} \\
 \glt `Those under twelve years old, their teeth can be replaced.' (27-tWCGArgu, 58)
\end{exe}

The postposition \japhug{ɕaŋlo}{upstream from} is homophonous with, and historically related to, the noun \japhug{ɕaŋlo}{seating place} (for old people and ladies) (§\ref{sec:orientation.kitchen}).

\subsection{Other temporal postpositions} \label{sec:temporal.postpositions}
Apart from the locative, terminative and egressive postpositions, a certain number of specifically temporal postpositions are found in Japhug, including \japhug{ɕɯŋgɯ}{before},  \japhug{ɕimɯma}{immediately after}, \japhug{kóʁmɯz}{only after}, \japhug{pɕintɕɤt}{since}, \japhug{jɤz}{when} and \japhug{ɕaŋpɕi}{since}, `from ... on'. All can be used with noun phrases and subordinate clauses; the latter use is studied in the section on temporal clauses (§\ref{sec:temporal.clauses}).

The postposition  \japhug{ɕɯŋgɯ}{before} is an ancient compound containing as first element the \textit{status constructus} of a root cognate to Tangut \tangut{𗪘}{2104}{śji}{1.10} `formerly, before' and the relator noun \japhug{ɯ-ŋgɯ}{inside} (§\ref{sec:relator.location}) as second element. It contrasts with  \japhug{ɯ-qʰu}{after} (§\ref{sec:relator.temporal}), and can follow a noun phrase referring to a point in time, as in the common expression \japhug{saχsɯ ɕɯŋgɯ}{before lunch} (with the noun  \japhug{saχsɯ}{lunch}), or a duration, as in (\ref{ex:sqamNusNi.CWNgW}) and (\ref{ex:XsArZaR.CWNgW}). The latter example shows that \japhug{ɕɯŋgɯ}{before} can be used to refer to events occurring \textit{after} the current temporal point of reference (in \ref{ex:XsArZaR.CWNgW}, before three days from the present in the story).

\begin{exe}
\ex \label{ex:sqamNusNi.CWNgW}
 \gll tɤte sqamŋu-sŋi ɕɯŋgɯ nɯtɕu tɕe, nɤki, nɯ-rlaʁ tɕe nɯ-me pɯ-ŋu ɲɯ-ŋu, tɕʰeme nɯ, \\
 that.is fifteen-day before \textsc{dem}.\textsc{loc} \textsc{lnk} \textsc{filler} \textsc{aor}-disappear \textsc{lnk} \textsc{aor}-not.exist \textsc{pst}.\textsc{ipfv}-be \textsc{sens}-be girl \textsc{dem} \\
 \glt `That is, fifteen days before, she had disappeared, that girl.' (tWxtsa, 15)
\end{exe}


\begin{exe}
\ex \label{ex:XsArZaR.CWNgW}
 \gll χsɯ-sŋi χsɤ-rʑaʁ mɤɕtʂa a-mɤ-tɤ-tɯ-rɤru ra ma tɕe mɤ-pʰɤn nɯra to-ti. matɕi tɕetʰa χsɯ-sŋi χsɤ-rʑaʁ ɕɯŋgɯ ɯʑo ɲɯ-pʰɣo pjɤ-ra lo \\
 three-day  three-night until \textsc{irr}-\textsc{neg}-\textsc{pfv}-2-get.up be.needed:\textsc{fact} \textsc{lnk} \textsc{lnk}  \textsc{neg}-be.efficient:\textsc{fact} \textsc{dem}:\textsc{pl} \textsc{ifr}-say because later   three-day  three-night before \textsc{3sg} \textsc{ipfv}-flee \textsc{ifr}.\textsc{ipfv}-be.needed \textsc{sfp} \\
 \glt `(The rabbit) said `Don't get up until three days and three nights (have passed)', because (the rabbit was buying time) and had to flee before (the end of) these three days and nights.' (140427 qala cho kWrtsag, 30-31)
\end{exe}

The postposition   \japhug{ɕɯŋgɯ}{before} however most commonly occurs with subordinate clauses, and requires a finite verb in the Imperfective (§\ref{sec:ipfv.temporal}, §\ref{sec:precedence.CWNgW}). It is attested following personal pronouns, as in (\ref{ex:aʑo.CWNgW}),  \japhug{ɕɯŋgɯ}{before} refers to an action concerning the referent of the pronoun, whose nature can be determined from the context, as if the main verb of a subordinate clause had been elided.

\begin{exe}
\ex \label{ex:aʑo.CWNgW}
 \gll  aʑo ɕɯŋgɯ a-pi ra atu rɤʑi-nɯ tɕe, nɯnɯra ɣɯ nɯ-rmi tɤ-z-mɤke qʰe, \\
 \textsc{1sg} before \textsc{1sg}.\textsc{poss}-elder.sibling \textsc{pl} up.there stay:\textsc{fact}-\textsc{pl} \textsc{lnk} \textsc{dem}:\textsc{pl} \textsc{gen} \textsc{3pl}.\textsc{poss}-name \textsc{imp}-\textsc{caus}-be.first[III] \textsc{lnk} \\
 \glt `Before (you choose a name for) me, my elder brothers up there, (choose) their names first.' (Gesar, 124)
\end{exe}


The common adverb \japhug{kɯɕɯŋgɯ}{in former times} comes from the combination of the \textit{status constructus} of the proximal demonstrative \japhug{ki}{this} (§\ref{sec:demonstrative.pronouns}) with the postposition \japhug{ɕɯŋgɯ}{before}. The phrases \forme{ki ɕɯŋgɯ} `before this' \forme{nɯ ɕɯŋgɯ} `before that' with the demonstratives \japhug{ki}{this} and  \japhug{nɯ}{that} are also attested.

While the neutral antonym of \japhug{ɕɯŋgɯ}{before} is the relator noun \japhug{ɯ-qʰu}{after} (§\ref{sec:relator.temporal}), subsequent temporality can also be expressed by \japhug{ɕimɯma}{immediately after} and \japhug{kóʁmɯz}{only after} (one of the rare uninflected words with a non-final stress, §\ref{sec:stress}). These postpositions are mainly attested following the demonstrative pronoun \forme{nɯ}, and often used adverbially as \japhug{nɯ ɕimɯma}{immediately} and \japhug{nɯ kóʁmɯz nɤ}{only then}, but are also attested with temporal and conditional clauses (§\ref{sec:immediate.subsequence}, §\ref{sec:only.if}) and with temporal counted nouns or adverbs as in (\ref{ex:sqamNuxpa.koRmWz}), which also illustrates the opposition between \japhug{kóʁmɯz}{only after} and the terminate postposition \japhug{mɤɕtʂa}{until} (§\ref{sec:terminative}. 

\begin{exe}
\ex \label{ex:sqamNuxpa.koRmWz}
\gll ɯ-xso ʑɴɢɯloʁ nɯnɯ, tɕe sqamŋu-xpa mɤɕtʂa ɯ-mat ku-tsʰoʁ mɯ́j-cʰa. pɯ́-wɣ-ji ɕimɯma, ɯ-rɣi pjɯ́-wɣ-ji ɯ-qʰu,  sqamŋu-xpa kóʁmɯz nɤ ɯ-mat ku-tsʰoʁ ŋu tu-ti-nɯ ŋgrɤl tɕe,  \\
\textsc{3sg}.\textsc{poss}-normal walnut  \textsc{dem} \textsc{lnk} fifteen-year until \textsc{3sg}.\textsc{poss}-fruit \textsc{ipfv}-attach \textsc{neg}:\textsc{sens}-can \textsc{aor}-\textsc{inv}-plant just.after \textsc{3sg}.\textsc{poss}-seed \textsc{ipfv}-\textsc{inv}-plant \textsc{3sg}.\textsc{poss}-after fifteen-year only.after \textsc{lnk} \textsc{3sg}.\textsc{poss}-fruit \textsc{ipfv}-attach be:\textsc{fact} \textsc{ipfv}-say-\textsc{pl} be.usually.the.case:\textsc{fact} \textsc{lnk} \\
\glt `Usually, the walnut tree cannot have walnuts until fifteen years (have passed). Just after one has planted it, after one plants its seeds, it is only fifteen years later that it bears nuts, they say.' (12-ndZiNgri, 166-167)
\end{exe}

There is an adverb \japhug{nóʁmɯz}{only then} (found for instance in \ref{ex:tWqioR.kW}, §\ref{sec:oblique.kW}) whose meaning is identical to \japhug{nɯ kóʁmɯz nɤ}{only then}, and apparently results from the fusion of the demonstrative \forme{nɯ} with the root \forme{-oʁmɯz}. The origin of the \forme{k-} element in \japhug{kóʁmɯz}{only after} is unclear; it could be the fused form of the proximal demonstrative \forme{ki} (§\ref{sec:anaphoric.demonstrative.pro}).

The postposition \japhug{pɕintɕɤt}{since} (from Tibetan \tibet{ཕྱིན་ཆད་}{pʰʲin.tɕʰad}{thereafter}) can follow a date (\ref{ex:69nian.pCintCAt}) or a subordinate clause, and is most often used with the relator noun \japhug{ɯ-qʰu}{after} to mean `from that time on' as in (\ref{ex:nW.Wqhu.pCintCAt}). Another postposition,  \japhug{ɕaŋpɕi}{since}, `from ... on' (combining the \forme{ɕaŋ-} element found in egressive postpositions §\ref{sec:egressive}  with \forme{-pɕi} from Tibetan \tibet{ཕྱི་}{pʰʲi}{later}) can be used like \japhug{pɕintɕɤt}{since} following  \japhug{ɯ-qʰu}{after} as in (\ref{ex:Wqhu.CaNpCi}), or after temporal clauses (§\ref{sec:since.clause}). The two postpositions \japhug{pɕintɕɤt}{since} and \japhug{ɕaŋpɕi}{since}, `from ... on' have the same meaning, but the latter is considered by Tshendzin to be an influence from the Xtokavian dialects (§\ref{sec:tChi}, §\ref{sec:xtokavian.preverbs}).

 \begin{exe}
\ex \label{ex:69nian.pCintCAt}
 \gll tɕe tʰam kɯβdesqi ɯ-ro to-pa ma <liu.jiu.nian> pɕintɕɤt \\
 \textsc{lnk} now fourty \textsc{3sg}.\textsc{poss}-excess \textsc{ifr}-pass.X.years \textsc{lnk}  1969 since \\
 \glt `Now it has been forty years (we have known each other), since 1969.' (12-BzaNsa, 13)
 \end{exe}
 
  \begin{exe}
\ex \label{ex:nW.Wqhu.pCintCAt}
 \gll  tɕiʑo pɯ-ari-tɕi ŋu, mtsʰukʰa pɯ-ftɕɤt-tɕi ŋu, nɯ ɯ-qʰu pɕintɕɤt tɕe,  nɯ-ʁgra nɯ nɯ-me ŋu \\
 \textsc{1du} \textsc{aor}:\textsc{down}-go[II]-\textsc{1du} be:\textsc{fact} lake \textsc{aor}-subdue-\textsc{1du} be:\textsc{fact} \textsc{dem} \textsc{3sg}.\textsc{poss}-after since \textsc{lnk} \textsc{2pl}.\textsc{poss}-enemy \textsc{dem} \textsc{aor}-not.exist be:\textsc{fact} \\
 \glt `We went down (into the lake), subdued the (demons in) the lake, and from that time on, your enemy is no more.' (Nyima.'Odzer2003.2, 109-110)
 \end{exe}
 
 \begin{exe}
\ex \label{ex:Wqhu.CaNpCi}
\gll  tɕe tɤ-mu nɯ, nɯ ɯ-qʰu ɕaŋpɕi ʑo kɤ-rɯndzɤqʰɤjɯ ta-znɯna \\
\textsc{lnk} \textsc{indef}.\textsc{poss}-mother \textsc{dem} \textsc{dem} \textsc{3sg}.\textsc{poss}-after from.that.time.on \textsc{emph} \textsc{inf}-eat.without.sharing \textsc{aor}:3\fl{}3'-stop \\
\glt `From that (time) on, the mother stopped to eat on her own without sharing.' (tWJo 2005, 53) 
\end{exe}

The postposition \japhug{jɤz}{when} and its variant \japhug{jɤznɤ}{when} follows either subordinate clauses (§\ref{sec:temporal.reference}), temporal adverbs (\ref{ex:jWfCWr.jAznA}) or the noun \japhug{ɯ-ŋgu}{beginning} (from \tibet{འགོ་}{go}{head, beginning}) as in (\ref{ex:jWfCWr.jAznA}); it is not attested with other noun phrases.

\begin{exe}
\ex \label{ex:jWfCWr.jAznA}
\gll  jɯfɕɯr nɯtɕu iɕqʰa tɤtɕɯpɯ nɯnɯra, jɯfɕɯr jɤznɤ tu-ndze-a tɕe pɯ-apa wo ri \\ 
yesterday \textsc{dem}:\textsc{loc} the.aforementioned boy \textsc{dem}:\textsc{pl} yesterday when \textsc{ipfv}-eat[III]-\textsc{1sg}  \textsc{lnk} \textsc{pst}.\textsc{ipfv}-be.correct \textsc{sfp} \textsc{lnk} \\
\glt  `I should have eaten these boys yesterday.' (160705 poucet5-v2, 36)
\end{exe}

 \begin{exe}
\ex \label{ex:WNgu.jAznA}
\gll  tɕe ɯ-ŋgu jɤznɤ tɕe, ɯ-mɤlɤjaʁ nɯra mɯ-cʰɯ-pʰaʁ-nɯ tɕe,  tɕe nɯ ɣɯ ɯ-ndʐi kɯ-fsɯ\redp{}fse nɯ pjɯ-qaʁ-nɯ tɕe tɕe \\
\textsc{lnk} \textsc{3sg}.\textsc{poss}-beginning when \textsc{lnk} \textsc{3sg}.\textsc{poss}-limb \textsc{dem}:\textsc{pl} \textsc{neg}-\textsc{ipfv}-cut-\textsc{pl} \textsc{lnk} \textsc{lnk} \textsc{dem} \textsc{gen} \textsc{3sg}.\textsc{poss}-skin \textsc{sbj}:\textsc{pcp}-\textsc{emph}\redp{}be.like \textsc{dem} \textsc{ipfv}-remove.skin-\textsc{pl} \textsc{lnk} \textsc{lnk} \\
\glt `In the beginning, they don't cut off the limbs (from the cattle's body), and take out the skin (in such a way as to preserve its shape) exactly like (that of the living animal).' (06-BGa, 94)
\end{exe}


 
\section{Relator nouns}  \label{sec:relator.nouns}  
Relator nouns are inalienably possessed nouns (§\ref{sec:inalienably.possessed}) used to mark the grammatical relations of oblique arguments or adjuncts. They differ from postpositions by at least three properties. 

First, they have an obligatory possessive prefix, which is coreferent with the preceding noun phrase or clause when one is present (generally the third person singular \forme{ɯ-} prefix). 

Second, unlike postpositions, which require at the very least a demonstrative (§\ref{ex:postpositions}), they can occur without a preceding noun phrase or clause if the referent indicated by the possessive prefix is definite (including first or second person, as in \ref{ex:aCki.jAGe}) or generic (see example \ref{ex:tWCki} in §\ref{sec:indef.genr.poss}). 

\begin{exe}
\ex \label{ex:aCki.jAGe}
\gll jɯfɕɯr <gongxun> a-ɕki jɤ-ɣe. \\
yesterday  \textsc{anthr} \textsc{1sg}.\textsc{poss}-\textsc{dat} \textsc{aor}-come[II] \\
\glt `Yesterday Gong Xun came to (see) me.' (160320, conversation)
\end{exe}

Third, the genitive \forme{ɣɯ} (§\ref{sec:genitive}) can optionally occur between the preceding noun phrase and the relator (as in \ref{ex:GW.WCki.nWtCu} below and \ref{ex:GW.kWnA.WrkW.ri}), since relator noun phrases are a subtype of the possessive construction (§\ref{sec:gen.possession}).

\begin{exe}
\ex \label{ex:GW.WCki.nWtCu}
\gll tɕe smɤt tɯmda rɟɤlpu ɣɯ ɯ-ɕki nɯtɕu, nɤkinɯ, ɯ-rʑaβ ɯ-kɯ-tʰu cʰɤ-ɕe tɕe, \\
\textsc{lnk}  \textsc{topo} \textsc{topo} king \textsc{gen} \textsc{3sg}.\textsc{poss}-\textsc{dat} \textsc{dem}:\textsc{loc} \textsc{filler} \textsc{3sg}.\textsc{poss}-wife \textsc{3sg}.\textsc{poss}-\textsc{sbj}:\textsc{pcp}-ask \textsc{ifr}:\textsc{downstream}-go \textsc{lnk} \\
\glt `He went to ask the king of the lower valley for (one his daughter to take as) a wife.' (2014-kWlAG, 16)
\end{exe}

In addition, most relator nouns still preserve non-grammaticalized uses revealing their diachronic source, and some of them can be followed by the locative postpositions (§\ref{sec:core.locative}) or by other relator nouns.

\subsection{Dative} \label{sec:dative} 
Two dative markers are attested in Japhug, \forme{ɯ-ɕki} and \forme{ɯ-pʰe}; some speakers like Tshendzin prefer the former (as in \ref{ex:aCki.jAGe}, \ref{ex:WCki.zW}, \ref{ex:WCki.toti}), but most speakers I have recorded favor the latter (for instance, Kunbzang Mtsho who tells the story from which \ref{ex:Wphe.toti} is taken).

The dative can be followed by the locative postpositions \forme{zɯ} and \forme{tɕu}, as in (\ref{ex:WCki.zW}), (\ref{ex:nWCki.zYArNo}) and (\ref{ex:slANe.ZNgri.ra.nWphe}).


\begin{exe}
\ex \label{ex:WCki.zW}
\gll nɯ mbro tɯ-skɤt kɯ-tso nɯnɯ ɯ-ɕki zɯ .... to-ti \\
\textsc{dem} horse \textsc{indef}.\textsc{poss}-speech \textsc{sbj}:\textsc{pcp}-understand \textsc{dem} \textsc{3sg}.\textsc{poss}-\textsc{dat} \textsc{loc} { } \textsc{ifr}-say \\
\glt `She said ... to the horse who could understand speech' (2003kAndzWsqhaj, 25)
\end{exe}

The dative is used to mark the recipient or addressee. It occurs with indirective verbs of speech such as \japhug{ti}{say} (\ref{ex:WCki.zW}, \ref{ex:WCki.toti} and \ref{ex:Wphe.toti}), \japhug{fɕɤt}{tell} and \japhug{tʰu}{ask} (\ref{ex:nWCki.tAthe}), and also with some intransitive verbs of speech such as \japhug{rɯɕmi}{speak} (\ref{ex:WCki.torWCmi}).

\begin{exe}
\ex \label{ex:WCki.toti}
\gll iɕqʰa srɯnmɯ nɯ kɯ, [...] smɤnmimitoʁ kuɕana ɯ-ɕki `nɤʑo tɕʰi ɯ-rɯɣ tɯ-ŋu' to-ti ri, \\
the.aforementioned râkshasî \textsc{dem} \textsc{erg} { }  \textsc{anthr} \textsc{anthr} \textsc{3sg}.\textsc{poss}-\textsc{dat} \textsc{2sg} what \textsc{3sg}.\textsc{poss}-race 2-be:\textsc{fact} \textsc{ifr}-say \textsc{lnk} \\
\glt `The râkshasî asked Smanmi Metog Koshana, `What type of being are you?' (28-smAnmi, 378)
\end{exe}

\begin{exe}
\ex \label{ex:Wphe.toti}
\gll tɕe tɤ-tɕɯ nɯ kɯ ɯ-wa ɯ-pʰe nɯra pɯ-kɯ-fse nɯra to-ti ɲɯ-ŋu \\
\textsc{lnk} \textsc{indef}.\textsc{poss}-son \textsc{dem} \textsc{erg} \textsc{3sg}.\textsc{poss}-father \textsc{3sg}.\textsc{poss}-\textsc{dat} \textsc{dem}.\textsc{pl} \textsc{aor}-\textsc{sbj}:\textsc{pcp}-be.like \textsc{dem}.\textsc{pl}  \textsc{ifr}-say \textsc{sens}-be \\
\glt `The boy told his father the things that had happened.' (qachGa2012, 175)
\end{exe}

\begin{exe}
\ex \label{ex:nWCki.tAthe}
\gll nɤʑo ɯ-mɤ-ɲɯ-tɯ-stu nɤ, ʑara nɯ-ɕki tɤ-tʰe jɤɣ \\
\textsc{2sg} \textsc{qu}-\textsc{neg}-\textsc{sens}-2-believe \textsc{lnk} \textsc{3pl} \textsc{3pl}.\textsc{poss}-\textsc{dat} \textsc{imp}-ask[III] be.possible:\textsc{fact} \\
\glt `If you don't believe it, ask them!' (140508 shier ge tiaowu de gongzhu-zh, 190)
\end{exe}

\begin{exe}
\ex \label{ex:WCki.torWCmi}
\gll nɯ tɤ-pɤtso nɯ ɯ-ɕki to-rɯɕmi. \\
\textsc{dem} \textsc{indef}.\textsc{poss}-child \textsc{dem} \textsc{3sg}.\textsc{poss}-\textsc{dat} \textsc{ifr}-speak \\
\glt  `It spoke to the child.' (150831 renshen wawa-zh, 36)
\end{exe}

It also occurs with verbs of giving to mark the recipient as in (\ref{ex:WCki.YAkho}) with the verb \japhug{kʰo}{give, pass over}, but also the source as in (\ref{ex:nWCki.zYArNo}) with verbs such as \japhug{rŋo}{borrow}, \japhug{sɤmbi}{ask for} and \japhug{χtɯ}{buy} (§\ref{sec:ditransitive.indirective}, §\ref{sec:sig.caus.inversive}).

 

\begin{exe}
\ex \label{ex:WCki.YAkho}
\gll  ɯ-nmaʁ ɯ-ɕki ɲɤ-kʰo tɕe,  \\
\textsc{3sg}.\textsc{poss}-husband \textsc{3sg}.\textsc{poss}-\textsc{dat} \textsc{ifr}-give \textsc{lnk} \\
\glt `She gave it to her husband.' (qajdoskAt, 71)
\end{exe}

\begin{exe}
\ex \label{ex:nWCki.zYArNo}
\gll 
kɯ-rɤrma ra nɯ-ɕki nɯtɕu, kuxtɕo ci z-ɲɤ-rŋo, \\
\textsc{sbj}:\textsc{pcp}-work \textsc{pl} \textsc{3pl}.\textsc{poss}-\textsc{dat} \textsc{dem}:\textsc{loc} basket \textsc{indef} \textsc{tral}-\textsc{ifr}-borrow \\
\glt `(The snow leopard) borrowed a basket from the workers.' (qala2002, 43)
\end{exe}

With the verb \japhug{kʰo}{give, pass over} the recipient is more often encoded with the genitive or a possessive prefix on the theme (§\ref{sec:gen.beneficiary}) or with the semi-grammaticalized noun \japhug{tɯ-jaʁ}{hand} (§\ref{sec:semi.grammaticalized.relator}).

 
The semi-transitive verb \japhug{ru}{look at} can mark its goal with the dative, as in (\ref{ex:WCki.Cturu}); this is however optional, as this verbs also takes goals in the absolutive (§\ref{absolutive.goal}) or locative (§\ref{sec:locative}).

\begin{exe}
\ex \label{ex:WCki.Cturu}
\gll tɕe tɤŋe nɯ nɯɣ-me tɕe, tɕe tɤŋe ɯ-ɕki ʁɟa ʑo ɕ-tu-ru tɕe, tɯʑo tɯ-ɕki maka ʑo mɤ-ru \\
\textsc{lnk} sun \textsc{dem} \textsc{appl}-be.afraid[III]:\textsc{fact}-\textsc{1sg} \textsc{lnk} \textsc{lnk} sun \textsc{3sg}.\textsc{poss}-\textsc{dat} completely \textsc{emph} \textsc{tral}-\textsc{ipfv}:\textsc{up}-look.at \textsc{lnk} \textsc{genr} \textsc{genr}.\textsc{poss}-\textsc{dat} at.all \textsc{emph} \textsc{neg}-look.at:\textsc{fact} \\
\glt `(If the yeti catches you), it is afraid of the sun, it looks at the sun the whole time, and does not look at you.' (140510 mYWrgAt, 13)
\end{exe}

The dative \forme{ɯ-ɕki} derives from a relator noun meaning `side', `near' or `at X's place' (with or without motion). These locative meanings are still marginally present in Japhug in examples like (\ref{ex:aCki.jAGe}) above and (\ref{ex:WCki.loc}), (\ref{ex:slANe.ZNgri.ra.nWphe}) and (\ref{ex:WCki.kunWrAZi}) below.

\begin{exe}
\ex \label{ex:WCki.loc}
\gll  ɯ-rte nɯ ɯ-rna ɯ-ɕki pɯ-kɯ-ɴqoʁ nɯnɯ pjɤ-mɟa tɕe ɯ-ku ɯ-taʁ to-ta. \\
\textsc{3sg}.\textsc{poss}-hat \textsc{dem} \textsc{3sg}.\textsc{poss}-ear \textsc{3sg}-\textsc{dat} \textsc{aor}:\textsc{down}-\textsc{nmlz}:S/A-hang \textsc{dem} \textsc{ifr}:\textsc{down}-take \textsc{lnk} \textsc{3sg}.\textsc{poss}-head \textsc{3sg}-on \textsc{ifr}-put \\
\glt `He took the hat that was hanging on his ear and put it on his head.' (140505 liuhaohan zoubian tianxia-zh, 164)
\end{exe}

\begin{exe}
\ex \label{ex:slANe.ZNgri.ra.nWphe}
\gll   tɤŋe cʰo slɤŋe ʑŋgri ra nɯ-pʰe nɯtɕu kɤ-nɤɕqa a-pɯ-tɯ-cʰa ra ma, \\
sun \textsc{comit} moon star \textsc{pl} \textsc{3pl}.\textsc{poss}-\textsc{dat} \textsc{dem}:\textsc{loc} \textsc{inf}-bear \textsc{irr}-\textsc{pfv}-2-can be.needed:\textsc{fact} \textsc{lnk} \\
\glt `(When you are) by the sun, the moon and the stars, you will have to bear (the heat and the cold), otherwise...' (2003kandZislama, 53)
\end{exe}
 
\begin{exe}
\ex \label{ex:WCki.kunWrAZi}
\gll  li tɕetu tɤ-ɣe qʰe, ɯ-wa ɯ-ɕki ku-nɯ-rɤʑi, tɕeki pɯ-ari qʰe, ɯ-wɯ ɯ-wi ni ndʑi-ɕki ju-nɯ-ɕe qʰe, \\
again up.there \textsc{aor}:\textsc{up}-go[II] \textsc{lnk} \textsc{3sg}.\textsc{poss}-father \textsc{3sg}.\textsc{poss}-\textsc{dat} \textsc{ipfv}-\textsc{auto}-stay, down.there \textsc{aor}-go[II] \textsc{lnk} \textsc{3sg}.\textsc{poss}-grand.father \textsc{3sg}.\textsc{poss}-grand.mother \textsc{du} \textsc{3du}.\textsc{poss}-\textsc{dat} \textsc{ipfv}-\textsc{auto}-go \textsc{lnk}  \\
\glt `When she comes up there, she stays at her father's house, and when she goes down there, she goes to her grandparent's place.' (14-siblings, 305)
\end{exe}

\subsection{Secutive} \label{sec:secutive} 
The secutive relator noun \japhug{ɯ-rca}{following} is used with verbs of motion such as \japhug{gi}{come} to express the meaning `follow', `come/go with' as in (\ref{ex:nArca.Gia}).

\begin{exe}
\ex \label{ex:nArca.Gia}
 \gll  aʑo kɯnɤ nɤ-rca ɣi-a ɕti \\
 \textsc{1sg} also \textsc{2sg}.\textsc{poss}-following come:\textsc{fact}-\textsc{1sg} be.\textsc{aff}:\textsc{fact} \\
\glt `I am coming/going with you.' (2011-05-nyima, 171)
\end{exe}

The secutive can have a meaning similar to that of the comitative adverb (§\ref{sec:comitative.adverb}) `together with X', as in (\ref{ex:WBGi.Wrca}).

\begin{exe}
\ex \label{ex:WBGi.Wrca}
 \gll  pɤnmawombɤr ɣɯ ɯ-ɕɤrɯ ɯ-βɣi ɯ-rca tsʰɯntsʰɯn ʑo ta-wum-nɯ ɲɯ-ŋu \\ 
\textsc{anthr} \textsc{gen} \textsc{3sg}.\textsc{poss}-bone \textsc{3sg}.\textsc{poss}-ash \textsc{3sg}.\textsc{poss}-following \textsc{idph}(II):neat \textsc{emph} \textsc{aor}:3\fl{}3'-collect-\textsc{pl} \textsc{sens}-be  \\
\glt `They collected all of Padma 'Od-'bar's bones together with his ashes.' (2005 Norbzang, 410)
\end{exe}

The secutive phrase can follow  (\ref{ex:WBGi.Wrca}), or precede (\ref{ex:tWjAGAt.Wrca}) the noun phrase it accompanies.

\begin{exe}
\ex \label{ex:tWjAGAt.Wrca}
 \gll   tɯ-jɤɣɤt ɯ-rca tɤ-se cʰɯ-nɯ-ɬoʁ \\
 \textsc{indef}.\textsc{poss}-feces \textsc{3sg}.\textsc{poss}-following \textsc{indef}.\textsc{poss}-blood \textsc{ipfv}:\textsc{downstream}-\textsc{auto}-come.out \\
 \glt `(In the case of this disease), blood comes out together with the feces.' (24-pGArtsAG, 117)
 \end{exe}

%The secutive with third person singular possessive \forme{ɯ-rca} is also used as a linker meaning `in addition'. 
With the indefinite possessor prefix \forme{tɤ-rca} and  \forme{tɯ-tɯ-rca}, the secutive appears in adverbial function with the meaning `together' (example \ref{ex:iZo.kWBde}, §\ref{sec:uses.numerals}), though in examples such as  (\ref{ex:tArAku.tArca}) the form  \forme{tɤ-rca} retains its nominal status.


\begin{exe}
\ex \label{ex:tArAku.tArca}
 \gll ɕoʁ nɯnɯ tɤ-rɤku tɤ-rca ŋu, sɯjno maʁ. \\
 buckwheat \textsc{dem} \textsc{indef}.\textsc{poss}-crops \textsc{indef}.\textsc{poss}-following be:\textsc{fact} grass not.be:\textsc{fact} \\
 \glt `Buckwheat (belongs) with the crops, it is not a (type of) grass.' (13-NanWkWmtsWG, 68)
\end{exe}

In addition, the unexpected focus marker \forme{rcanɯ} (§\ref{sec:unexpected}) and the epistemic sentence final particle \forme{rca} (§\ref{sec:fsp.epistemic}) are historically related to the secutive.
 
\subsection{Deputative} \label{sec:deputative} 
The relator noun \forme{ɯ-tsʰɤt} has two meanings. First, it can serve as a deputative relator noun `instead of, on behalf of' as in (\ref{ex:nWtAsno.WtshAt}) and (\ref{ex:nWsi.WtshAt}). No verb selects this relator noun. 

The deputative adjunct can correspond to the intransitive subject (as in \ref{ex:nWtAsno.WtshAt}, with the verb \japhug{tu}{exist}), the transitive subject (as in \ref{ex:aZo.nAtshAt}, with \japhug{ɣɯjtsi}{support}) or the object.

\begin{exe}
\ex \label{ex:nWtAsno.WtshAt}
\gll nɯʑora ɣɯ nɯ-tɤ-sno kɯ-fse ɯ-tsʰɤt nɯ, tɕiʑo ɣɯ, tɕi-xɕɤndʑu χsɯ-ldʑa pɯ-tu tɕe, nɯnɯ lɤ-nɯ-βlɯ-tɕi ɕti wo \\
\textsc{2pl} \textsc{gen} \textsc{2pl}.\textsc{poss}-\textsc{indef}.\textsc{poss}-saddle \textsc{sbj}:\textsc{pcp}-be.like \textsc{3sg}.\textsc{poss}-instead.of \textsc{dem} \textsc{1du} \textsc{gen} \textsc{1du}.\textsc{poss}-twig three-long.object \textsc{pst}.\textsc{ipfv}-exist \textsc{lnk} \textsc{dem} \textsc{aor}-\textsc{auto}-burn-\textsc{1du} be.\textsc{aff}:\textsc{fact} \textsc{sfp} \\
\glt `Instead of a saddle like yours, we had three twigs, this is what we burned.' (Kunbzang 2003, 203)
\end{exe}

\begin{exe}
\ex \label{ex:aZo.nAtshAt}
\gll aʑo nɤ-tsʰɤt, nɤki, si nɯ tu-ɣɯjtsi-a jɤɣ \\
\textsc{1sg} \textsc{2sg}.\textsc{poss}-instead.of \textsc{filler} tree \textsc{dem} \textsc{ipfv}-support-\textsc{1sg} be.possible:\textsc{fact} \\
\glt `I can support the tree for you/instead of you (while you fetch it).' (150830 afanti-zh, 136)
\end{exe}

 The noun phrase headed by \forme{ɯ-tsʰɤt} can be either an adjunct as in (\ref{ex:nWtAsno.WtshAt}) and (\ref{ex:aZo.nAtshAt}), the object of the verb \japhug{βzu}{make}, or a nominal predicate with a copula as in (\ref{ex:nWsi.WtshAt}) and (\ref{ex:aZo.atshAt}).  In the latter case, to express the meaning `do to $X$ instead of to $Y$', a biclausal construction `do to $X$, ($X$) is instead of $Y$' is used as in (\ref{ex:aZo.atshAt}).

\begin{exe}
\ex \label{ex:nWsi.WtshAt}
\gll si maŋe tɕe tɕe nɯnɯtɕu tɕe, nɯ-si ɯ-tsʰɤt ɲɯ-ŋu  \\
tree not.exist:\textsc{sens} \textsc{lnk} \textsc{lnk} \textsc{dem}:\textsc{loc} \textsc{lnk} \textsc{3pl}.\textsc{poss}-wood \textsc{3sg}.\textsc{poss}-instead.of \textsc{sens}-be \\
\glt `(since) there are no trees, (dung) is used there to replace the firewood.' (05-tamar, 10-11)
\end{exe}


\begin{exe}
\ex \label{ex:aZo.atshAt}
\gll nɯ tɤ-nɯ-ndɤm tɕe aʑo a-tsʰɤt ŋu tɕe \\
\textsc{dem} \textsc{imp}-\textsc{auto}-take[III] \textsc{lnk} \textsc{1sg} \textsc{1sg}.\textsc{poss}-instead.of be:\textsc{fact} \textsc{lnk} \\
\glt `Take these instead of me (as a compensation).' (2003kAndzwsqhaj2, 141)
\end{exe}

The examples (\ref{ex:aZo.nAtshAt}) and (\ref{ex:aZo.atshAt}) also show that the relator noun \japhug{ɯ-tsʰɤt}{instead of} can occur with a first or second person possessive prefix.

Second, \forme{ɯ-tsʰɤt} also means `with proper measure', mainly occurring in adverbial function as in (\ref{ex:WtshAt.tsa}) or in collocation with the verb \japhug{βzu}{make} in the sense `do with proper measure' as in (\ref{ex:WtshAt.tusWBzunW}). 

\begin{exe}
\ex \label{ex:WtshAt.tsa}
\gll rkaŋraŋ ɯ-tsʰɤt tsa ɲɯ-kɯ-nɤɕtʂaʁli-a-nɯ raʁmaʁ ma  \\
\textsc{anthr} \textsc{3sg}.\textsc{poss}-proper.measure a.little \textsc{ipfv}-2\fl{}1-torture-\textsc{1sg}-\textsc{pl} \textsc{sfp} \textsc{sfp}  \\
\glt `Rkangrang, don't go over the top in your torturing of me (torture me with proper measure), please? (Norbzang 2005, 245)
\end{exe}

\begin{exe}
\ex \label{ex:WtshAt.tusWBzunW}
\gll ɯ-tsʰɤt tu-sɯ-βzu-nɯ mɯ́j-kʰɯ ma, nɯ-kɤ-kʰo nɯ mɯ-tʰa-ɕkɯt mɤɕtʂa tu-ndze ɲɯ-ɕti. \\
\textsc{3sg}.\textsc{poss}-proper.measure \textsc{ipfv}-\textsc{caus}-make-\textsc{pl} \textsc{neg}:\textsc{sens}-be.possible \textsc{lnk}  \textsc{aor}-\textsc{obj}:\textsc{pcp}-give \textsc{dem} \textsc{neg}-\textsc{aor}:3\fl{}3'-eat.completely until \textsc{ipfv}-eat[III] \textsc{sens}-be.\textsc{aff} \\
\glt `They cannot make (the monkey eat) with measure, as it continues eating the (food) that is given to it until there is none.' (19-GzW, 60)
\end{exe}

In some contexts as in (\ref{ex:nWnW.WtshAt}), \japhug{ɯ-tsʰɤt}{proper measure} in adverbial used is better translated as `depending on the circumstances'.\footnote{This example is taken from a text describing goats and sheep; goats are called \forme{tsʰɤt} in Japhug, but it is clear from the context that \forme{ɯ-tsʰɤt} cannot be the \textsc{3sg} possessed form of this noun, though the two forms are indeed homophonous. }

\begin{exe}
\ex \label{ex:nWnW.WtshAt}
\gll tɕe nɯnɯ ɯ-tsʰɤt nɯnɯ ɯ-pɯ ci ci ʁnɯz tu, ci ci tɯ-rdoʁ ma me tɕe núndʐa ɲɯ-ŋu. \\
\textsc{lnk} \textsc{dem} \textsc{3sg}.\textsc{poss}-proper.measure \textsc{dem}  \textsc{3sg}.\textsc{poss}-young once once two exist:\textsc{fact} once once one-piece apart.from not.exist:\textsc{fact} \textsc{lnk} for.this.reason \textsc{sens}-be \\
\glt `This is why, depending on the circumstances, sometimes (the goat) has two kids, sometimes only one.' (05-qaZo, 28)
\end{exe}

The inalienably possessed noun  \forme{ɯ-tsʰɤt} (at least in the meaning `proper measure') is borrowed from \tibet{ཚད་}{tsʰad}{measure, limit}. It occurs as second element in the compound \japhug{xtɤtsʰɤt}{restraint of one's appetite}(with the \textit{status constructus} \forme{xtɤ-} of \japhug{tɯ-xtu}{belly}).

\subsection{Locative relator nouns} \label{sec:relator.location}
The dearth of specific locative postpositions (§\ref{sec:locative}) other than the egressive ones (§\ref{sec:egressive}) in Japhug is compensated by the existence of many relator nouns expressing various types of location, as shown in \tabref{tab:relator.location}. 

As other relator nouns, they can follow noun phrases (including headless relative clauses), with the possessive prefix agreeing in person and number with the preceding constituant, but can also occur on their own if the referent is definite, as \japhug{ɯ-taʁ}{on} in (\ref{ex:WtaR.zW.kAmdzW}) and (\ref{ex:WtaR.nWtCu.YAXtAr}) below.

\begin{exe}
\ex \label{ex:WtaR.zW.kAmdzW}
\gll  ɯ-taʁ zɯ kɤ-amdzɯ nɤ tɤ-lu pa-tɕɤt ɲɯ-ŋu\\
\textsc{3sg}.\textsc{poss}-on \textsc{loc} \textsc{aor}-sit \textsc{lnk} \textsc{indef}.\textsc{poss}-milk \textsc{aor}:3\fl{}3'-take.out \textsc{sens}-be\\
\glt `She sat on him and milked.' (2005 Kunbzang, 81)
\end{exe}

 

\subsubsection{The tridimensional system} \label{sec:relator.nouns.3d}
The first six nouns in \tabref{tab:relator.location} are derived from orientation adverbs (§\ref{sec:locative.adv}). The vertical dimension relator nouns \japhug{ɯ-taʁ}{on} and \japhug{ɯ-pa}{below} are simply built by adding a possessive prefix to the root of the adverb, while the other ones combine the \textit{status constructus} of the adverbial root (\forme{lɤ\trt}, \forme{tʰɤ\trt}, \forme{kɤ\trt}, \forme{ndɤ-} from \forme{lo}, \forme{tʰi}, \forme{kɯ} and \forme{ndi}, respectively) with a suffix \forme{-cu}. This hexameric system comprising three pairs of elements along three dimensions (vertical, fluvial, solar) is the same as found in verbal morphological (§\ref{sec:preverbs.adverbs}) and also egressive postpositions (§\ref{sec:egressive}).

\begin{table}
\caption{Locative relator nouns in Japhug} \label{tab:relator.location}
\begin{tabular}{lllll}
\lsptoprule
& Lexical origin \\
\midrule
\japhug{ɯ-taʁ}{on, above} & \japhug{taʁ}{upwards}\\
\japhug{ɯ-pa}{below, under} & \japhug{pa}{downwards}\\
\japhug{ɯ-lɤcu}{upstream of} & \japhug{lo}{upstream}\\
\japhug{ɯ-tʰɤcu}{downstream of} & \japhug{thi}{downstream}\\
\japhug{ɯ-kɤcu}{east of} & \japhug{kɯ}{eastwards}\\
\japhug{ɯ-ndɤcu}{west of} & \japhug{ndi}{westwards}\\
\midrule
\japhug{ɯ-ku}{top of} & \japhug{tɯ-ku}{head}\\
\japhug{ɯ-qa}{bottom of} & \japhug{tɤ-qa}{paw,  root} \\
\japhug{ɯ-ʁɤri}{before, in front of} \\
\japhug{ɯ-qʰu}{after}, `behind' \\
\japhug{ɯ-ŋgɯ}{inside} \\
\japhug{ɯ-pɕi}{outside} &  \tibet{ཕྱི་}{pʰʲi}{outside}\\
\japhug{ɯ-rkɯ}{side} \\
\japhug{ɯ-χcɤl}{middle, center} & \tibet{དཀྱིལ་}{dkʲil}{middle}\\
\japhug{ɯ-pɤrtʰɤβ}{between} & \tibet{བར་}{bar}{middle, between}\\
\japhug{ɯ-tʰɤβ}{between} \\
\japhug{ɯ-mŋu}{opening, edge, border} \\
\japhug{ɯ-ndo}{edge, border} \\
\lspbottomrule
\end{tabular}
\end{table}


\subsubsection{Other locative relator nouns} \label{sec:other.locative.relator}
Outside of the tridimensional system, other locative relator nouns also occur in antithetic pairs, in particular \japhug{ɯ-ʁɤri}{before, in front of}  vs. \japhug{ɯ-qʰu}{after, behind}  (the latter also occurs as a temporal relator noun, cf §\ref{sec:relator.temporal}), \japhug{ɯ-ŋgɯ}{inside} vs. \japhug{ɯ-pɕi}{outside}  and  \japhug{ɯ-mŋu}{opening, edge, border}  vs. \japhug{ɯ-ndo}{edge, border} or \japhug{ɯ-qa}{bottom}.

A few of these relator nouns are borrowed from Tibetan, as indicated in \tabref{tab:relator.location}), but most of them are native Gyalrong words. In particular \japhug{ɯ-ʁɤri}{before, in front of} is one of the very rare cases of a disyllabic word that has a cognate in Tangut sharing both syllables (\tangut{𘁞𗙷}{5416-567}{ɣwə-rjir}{2.25-2.74} `before').\footnote{The syllable \forme{ʁɤ-} = \tangut{𘁞}{5416}{ɣwə}{2.25} may be a fossilized allomorph of \japhug{tɯ-ku}{head} with a uvular as in the Stau cognate \stau{ʁə}{head}.} The relator noun \japhug{ɯ-ku}{top of} (as in \ref{ex:Wku.zW}) is transparently grammaticalized from the body part \japhug{tɯ-ku}{head}.

\begin{exe}
\ex \label{ex:Wku.zW}
 \gll tɕetu si ɯ-ku zɯ qaliaʁ ɣɤʑu tɕe, \\
 up.there tree \textsc{3sg}.\textsc{poss}-top.of \textsc{loc} eagle \textsc{sens}:exist \textsc{lnk} \\
 \glt `Up there on the top of the tree there is an eagle.' (140427 laoying mao he yezhu-zh, 31)
\end{exe}

Some relator nouns other than \japhug{ɯ-ku}{top of} have non-grammaticalized uses; for instance \japhug{ɯ-ŋgɯ}{inside} still occurs as a noun meaning  `internal part, inside' in (\ref{ex:WNgW.nW.so}).

\begin{exe}
\ex \label{ex:WNgW.nW.so}
 \gll tɕe nɯnɯ kɯ-spoʁ ŋu, ɯ-ŋgɯ nɯ so tɕe  \\
 \textsc{lnk} \textsc{dem} \textsc{sbj}:\textsc{pcp}-have.a.hole be:\textsc{fact} \textsc{3sg}.\textsc{poss}-inside \textsc{dem} be.hollow:\textsc{fact} \textsc{lnk} \\
 \glt `Its (inside) has a hole, its inside is hollow.' (12-Zmbroko, 23)
\end{exe}

Some of the relator nouns above can derive verbs of location such as \japhug{mɤku}{be first} and \japhug{mɤpɕi}{be outside} with the denominal prefix \forme{mɤ-} (§\ref{sec:denom.mA}).

The meaning of the relator nouns \forme{ɯ-mŋu} and \forme{ɯ-ndo} requires a specific description. These nouns are not antithetic to another in all cases. The basic (non-grammaticalized) meaning of  \forme{ɯ-mŋu}  is the border of the opening or mouth of a container / bag (\ref{ex:khWtsa.WmNu}), or the shoreline (of a lake), as in (\ref{ex:mtshu.WmNu}). In this use, it is opposed to \japhug{ɯ-qa}{bottom of}, as in (\ref{ex:mtshu.Wqa}).

\begin{exe}
\ex \label{ex:khWtsa.WmNu}
\gll  tɕendɤre nɯnɯ kʰɯtsa ɯ-mŋu jamar kɯ-wxti tu,\\
\textsc{lnk} \textsc{dem} bowl \textsc{3sg}.\textsc{poss}-border about \textsc{sbj}:\textsc{pcp}-be.big exist:\textsc{fact}\\
\glt  `Some are about as big as the mouth of a bowl.' (22-BlamajmAG, 130)
\end{exe}

\begin{exe}
\ex \label{ex:mtshu.WmNu}
\gll   tɕendre pɣɤtɕɯ nɯ mtsʰu ɯ-mŋu nɯtɕu `ʂɯt' to-ti to-nɯ-ɬoʁ.  \\
\textsc{lnk} bird \textsc{dem} lake \textsc{3sg}.\textsc{poss}-border \textsc{dem}:\textsc{loc} \textsc{idph}.I:sound \textsc{ifr}-say \textsc{ifr}:\textsc{up}-\textsc{auto}-come.out \\
\glt  `The bird came out of the shore of the lake with a noise.' (2014-kWlAG, 556)
\end{exe}

\begin{exe}
\ex \label{ex:mtshu.Wqa}
\gll  mtsʰu ɯ-qa zɯ nɤrwɯ mɤ-kɯ-naχtɕɯɣ ci ɣɤʑu tɕe, \\
lake \textsc{3sg}.\textsc{poss}-bottom \textsc{loc} jewel \textsc{neg}-\textsc{sbj}:\textsc{pcp}-be.similar \textsc{indef} exist:\textsc{sens} \textsc{lnk} \\
\glt `At the bottom of the sea, there is a jewel unlike any other.' (2012 Kunbzang, 12)
\end{exe}

The basic meaning of \forme{ɯ-ndo} includes `extremity' (for instance, of a limb as in \ref{ex:WRar.Wndo}) and also `end' in both the locative and temporal sense (\ref{ex:Wndo.tCe}).

\begin{exe}
\ex \label{ex:WRar.Wndo}
\gll ɯ-ʁar ɯ-ndo nɯra hanɯni ɲɯ-ɲaʁ. \\ 
\textsc{3sg}.\textsc{poss}-wing \textsc{3sg}.\textsc{poss}-border \textsc{dem}:\textsc{pl} a.little \textsc{sens}-be.black \\
\glt  `The extremities of its wings are a bit black.' (23-scuz, 133)
\end{exe}

\begin{exe}
\ex \label{ex:Wndo.tCe}
\gll  ɯ-ndo tɕe maka kɯ-tu ɲɯ-me ɲɯ-ŋu tɕe,  \\
\textsc{3sg}.\textsc{poss}-border \textsc{lnk} at.all \textsc{sbj}:\textsc{pcp}-exist \textsc{ipfv}-not.exist \textsc{sens}-be \textsc{lnk} \\
\glt `In the end, nothing is left.' (04-xiaocunzhuang-zh, 63)
\end{exe}

In the case of clothes, \forme{ɯ-ndo} refers to the lower opening (towards the feet), as opposed to the collar, as in (\ref{ex:Wndo.ri.kulAtnW}).

\begin{exe}
\ex \label{ex:Wndo.ri.kulAtnW}
\gll tɕe nɯ ɯ-ndʐi nɯnɯ pjɯ-χtsɤβ-nɯ tɕe tɕe tɯ-ŋga ɯ-ndo ri ku-lɤt-nɯ, tɯ-ŋga ɯ-kuŋa, ɯ-pɤloʁ ɯ-ku, ɯ-ndo nɯra ku-lɤt-nɯ,  tu-sɯ-fskɤr-nɯ ŋu.   \\
\textsc{lnk} \textsc{dem} \textsc{3sg}.\textsc{poss}-skin \textsc{dem} \textsc{ipfv}-tan-\textsc{pl} \textsc{lnk} \textsc{lnk} \textsc{indef}.\textsc{poss}-clothes \textsc{3sg}.\textsc{poss}-border \textsc{loc} \textsc{ipfv}-release-\textsc{pl} \textsc{indef}.\textsc{poss}-clothes \textsc{3sg}.\textsc{poss}-collar \textsc{3sg}.\textsc{poss}-sleeves \textsc{3sg}.\textsc{poss}-top \textsc{3sg}.\textsc{poss}-border  \textsc{dem}:\textsc{pl}  \textsc{ipfv}-release-\textsc{pl} \textsc{ipfv}-\textsc{caus}-surround-\textsc{pl} be:\textsc{fact} \\
\glt `They tan its hide (of the otter) and put it on the lower opening of the clothes, on the collar of clothes, the cuffs of the sleeves and the lower opening, and make it around (these openings).' (28-qapar, 96)
\end{exe} 


A contrast between \forme{ɯ-mŋu} and \forme{ɯ-ndo} occurs in their uses as relator nouns, indicating opposite extremities or sides. In the case of fields (as in \ref{ex:tWji.WmNu.Wndo}), \forme{ɯ-mŋu} designates the higher side of the field (towards the mountain), while \forme{ɯ-ndo} refers to the side closer to the river (all arable lands in Gyalrong area lie in narrow valleys).

\begin{exe}
\ex \label{ex:tWji.WmNu.Wndo}
\gll  qaʑmbri nɯ, tɯ-ji ɯ-ndo, tɯ-ji ɯ-mŋu nɯra aʁɤndɯndɤt ʑo tu-ɬoʁ ɕti \\
vine \textsc{dem} \textsc{indef}.\textsc{poss}-field \textsc{3sg}.\textsc{poss}-border \textsc{indef}.\textsc{poss}-field \textsc{3sg}.\textsc{poss}-border \textsc{dem}:\textsc{pl} everywhere \textsc{emph} \textsc{ipfv}:\textsc{up}-come.out be.\textsc{aff}:\textsc{fact} \\
\glt `The vine, it grows everywhere, on both sides of the fields.' (06-qaZmbri, 12)
\end{exe}


The relator noun \forme{ɯ-mŋu}  can also designates the top extremity of stairs, as in (\ref{ex:rJAskAt.WmNu}); for the lower side, either \japhug{ɯ-qa}{bottom} or \forme{ɯ-ndo} can be used (the former more commonly).

\begin{exe}
\ex \label{ex:rJAskAt.WmNu}
\gll  cɯŋglɯɣ nɯ rɟɤskɤt ɯ-mŋu zɯ na-ta ɲɯ-ŋu \\
 pestle \textsc{dem} stairs \textsc{3sg}.\textsc{poss}-border \textsc{loc} \textsc{aor}:3\fl{}3'-put \textsc{sens}-be \\
 \glt `He put the pestle on the top of the stairs.' (tWJo 2005, 49)
\end{exe} 

The superlative derivation (§\ref{sec:superlative.XCWX}) can be applied to most locative relator nouns, as \forme{ɯ-mŋuɕɯmŋu} from  \forme{ɯ-mŋu} in (\ref{ex:WmNuCWmNu}), where it means that the liquid completely fills the bowl to the point of touching the border of its mouth, flowing out at the slightest motion.

\begin{exe}
\ex \label{ex:WmNuCWmNu}
\gll  tʂʰa tɤ́-wɣ-rku tɕe kʰɯtsa ɯ-mŋuɕɯmŋu stʰɯci tu-zɣɯt mɤ-ra ma kɤ-ndo tɕe sɤ-ɕke \\
tea \textsc{aor}-\textsc{inv}-put.in \textsc{lnk} bowl \textsc{3sg}.\textsc{poss}-border:\textsc{superlative} so.much \textsc{ipfv}:\textsc{up}-reach \textsc{neg}-be.needed:\textsc{fact} \textsc{lnk} \textsc{inf}-take \textsc{lnk} \textsc{prop}-burn:\textsc{fact} \\
\glt `When one pours tea, it should not reach the limit of the mouth of the bowl, otherwise it will be burning when one holds it.' (elicited)
\end{exe} 

In addition to the relator nouns described above, the inalienably possessed noun \japhug{ɯ-stu}{straight ahead} is mainly used as an adverb, but can also serve as a relator noun to indicate the goal of a motion verb as in (\ref{ex:Wstu.Zo.YAGi}).

\begin{exe}
\ex \label{ex:Wstu.Zo.YAGi}
\gll  tɕe pʰaʁrgot ri li ɯʑo ɯ-stu ʑo ɲɤ-ɣi qʰe,  ɕɤmɯɣdɯ kɤ-lɤt mɯ-pjɤ-nɤz qʰe pʰaʁrgot jo-nɯ-ɕe. \\
\textsc{lnk} boar also again \textsc{3sg} \textsc{3sg}.\textsc{poss}-straight \textsc{emph} \textsc{ifr}:\textsc{west}-come \textsc{lnk} gun \textsc{inf}-release \textsc{neg}-\textsc{ifr}.\textsc{ipfv}-dare \textsc{lnk} boar \textsc{ifr}-\textsc{auto}-go \\
\glt `The boar came directly at him, but he did not dare to shoot and the boar went away.' (150829 phaRrgot, 8)
\end{exe} 

\subsubsection{The relator noun \japhug{ɯ-taʁ}{on, above}} \label{sec:WtaR} 
The relator noun \japhug{ɯ-taʁ}{on, above} is mainly used to build locative adjuncts or goals (see examples in §\ref{sec:relator.postposition.location}), but it is also selected by a certain number of verbs to mark an oblique argument.\footnote{The relator noun \forme{ɯ-taʁ} is homophonous with the bare infinitive \forme{ɯ-taʁ} of the verb \japhug{taʁ}{weave} (§\ref{sec:bare.inf}), as in example (\ref{ex:WtaR.WtWBdi}), §\ref{sec:degree.nominal.arguments}, but the resemblance between the two forms is fortuitous. }

The intransitive verb \japhug{atsa}{prick} and its causative form \japhug{sɤtsa}{prick, pierce} encode the piercing object as intransitive subject and object, respectively. The patient (the person or object being pricked) is encoded as an oblique argument in \forme{ɯ-taʁ}; example (\ref{ex:tWtaR.kotsa}) illustrates the intransitive verb \forme{atsa} with a generic human argument (§\ref{sec:indef.genr.poss}).

\begin{exe}
\ex \label{ex:tWtaR.kotsa}
\gll tɕe ɯ-ku kɯ-ɤmtɕoʁ ɲɯ-ŋu tɕe, tɯ-taʁ ku-otsa tɕe ɲɯ-ɕɯ-mŋɤm. \\
\textsc{lnk} \textsc{3sg}.\textsc{poss}-head \textsc{sbj}:\textsc{pcp}-be.pointy \textsc{sens}-be \textsc{lnk} \textsc{genr}.\textsc{poss}-on \textsc{ipfv}-prick \textsc{lnk} \textsc{sens}-\textsc{caus}-hurt \\
\glt `(The fir needles) are pointy-headed, they prick people, it hurts.'  (08-tWrgi, 27)
\end{exe} 

The intransitive verb \japhug{ɴqoʁ}{hang} has an additional volitional meaning of `grab, lean on', taking an object or person with both hands with body contact. It is attested for instance to describe a person who has fallen in the water, reaching for floating pieces of wood in order not to sink as in (\ref{ex:WtaR.kANqoRnW}) or jumping on the back of someone and grabbing her to prevent her from leaving (\ref{ex:WtaR.koNqoR}). The object or person being grabbed is marked with \forme{ɯ-taʁ}.

\begin{exe}
\ex \label{ex:WtaR.kANqoRnW}
 \gll nɯnɯ ʑmbrɯ nɯ-kɤ-χtɤr nɯ ɯ-taʁ kɤ-ɴqoʁ-nɯ ra   \\
 \textsc{dem} ship \textsc{aor}-\textsc{obj}:\textsc{pcp}-scatter \textsc{dem} \textsc{3sg}.\textsc{poss}-on \textsc{imp}-hang-\textsc{pl} be.needed:\textsc{fact} \\
 \glt `Grab the (pieces of the) ship that have been scattered.' (2012 Norbzang, 32)
\end{exe}

\begin{exe}
\ex \label{ex:WtaR.koNqoR}
\gll tɯrmɯkʰa tɕe tɯlɤt nɯ ɯ-taʁ ko-ɴqoʁ tɕe \\
evening \textsc{loc} second.sibling \textsc{dem} \textsc{3sg}.\textsc{poss}-on \textsc{ifr}-hang \textsc{lnk} \\
\glt `In the evening, he grabbed the second sister.' (07-deluge, 47)
\end{exe} 

Some stative verb also optionally select an oblique argument in \forme{ɯ-taʁ}, in particular to express a beneficiary/maleficiary with the verbs \japhug{pe}{be good} and \japhug{ŋɤn}{be bad}, in the meaning `be good to X, be nice with, treat X well' and `treat X badly', as in (\ref{ex:WtaR.mWpjApendZi}). The genitive can alternatively be used to mark the beneficiary with these verbs (§\ref{sec:gen.beneficiary}), but the meaning is rather `be advantageous' or `be harmful' depending on polarity.


\begin{exe}
\ex \label{ex:WtaR.mWpjApendZi}
\gll  maka ʑo ɯ-taʁ mɯ-pjɤ-pe-ndʑi qhe, pjɤ-ŋɤn-ndʑi \\
at.all \textsc{emph} \textsc{3sg}.\textsc{poss}-on \textsc{neg}-\textsc{ipfv}.\textsc{ifr}-be.good-\textsc{du} \textsc{lnk} \textsc{neg}-\textsc{ipfv}.\textsc{ifr}-be.bad-\textsc{du} \\
\glt `They did not treat him well, they treated him badly. (2014-kWlAG, 114)
\end{exe}

\begin{exe}
\ex \label{ex:ataR.nWtWpe}
\gll  a-pi ra a-taʁ nɯ-tɯ-pe pɯ-sɤre ʑo tɕe  \\
\textsc{1sg}.\textsc{poss}-elder.sibling \textsc{pl} \textsc{1sg}.\textsc{poss}-on \textsc{3pl}.\textsc{poss}-\textsc{nmlz}:\textsc{deg}-be.good \textsc{pst}.\textsc{ipfv}-be.ridiculous \textsc{emph} \textsc{lnk} \\
\glt `My elder brothers treated me very well.' (140501 tshering skyid, 46)
\end{exe}

The bipartite verb \japhug{stu,mbat}{try hard} (§\ref{sec:bipartite}) can also take a beneficiary in \forme{ɯ-taʁ}, with the meaning `care about $X$ and treat $X$ well' as in (\ref{ex:nAtaR.stua}).

\begin{exe}
\ex \label{ex:nAtaR.stua}
\gll nɤ-taʁ stu-a mbat-a ʑo ɕti\\
\textsc{2sg}.\textsc{poss}-on try.hard(1):\textsc{fact}-\textsc{1sg} try.hard(2):\textsc{fact}-\textsc{1sg} \textsc{emph} be.\textsc{aff}:\textsc{fact}\\
\glt `I will treat you well.' (2005lobzang, 33)
\end{exe}


Other intransitive verbs selecting oblique arguments in \forme{ɯ-taʁ} include \japhug{nɯrɕɤt}{slightly touch in passing} (designating the object touched), \japhug{rpu}{bump on} (§\ref{sec:goal.labile}) and \japhug{ndzoʁ}{be attached} (\ref{ex:WtaR.kondzoR}, §\ref{sec:anticausative.volitionality}).

The transitive verb \japhug{lɤt}{throw, release}, when used as a light verb (§\ref{sec:light.verb}) in collocation with nouns expressing either objects, weapons or substances that are thrown or that the subject hits people with, selects a phrase in \forme{ɯ-taʁ} to indicate the recipient, as in (\ref{ex:tWmci.pjWwGlAt}) and (\ref{ex:scoRqhu.tulAt}). This recipient, although syntactically quite different from an object or a semi-object, can however be relativized using the object participle (§\ref{sec:object.participle.other.relative}).

\begin{exe}
\ex \label{ex:tWmci.pjWwGlAt}
\gll ɯ-taʁ tɯ-mci pjɯ́-wɣ-lɤt \\
\textsc{3sg}.\textsc{poss}-on \textsc{genr}.\textsc{poss}-spit \textsc{ipfv}-\textsc{inv}-release \\
\glt `If one spits on it,' (08-qaJAGi, 27)
\end{exe}

\begin{exe}
\ex \label{ex:scoRqhu.tulAt}
\gll ʁjoʁ ra nɯ-taʁ kɯnɤ kɤ-ari tɕe scoʁ-qʰu ci tu-lɤt, nɯ-ɣe tɕe tɤŋkʰɯt ci tu-lɤt \\
servant \textsc{pl} \textsc{3pl}.\textsc{poss}-on also \textsc{aor}:\textsc{east}-go[II] \textsc{lnk} ladle-back once \textsc{ipfv}-release \textsc{aor}:\textsc{west}-come[II] \textsc{lnk} fist once \textsc{ipfv}-release \\
\glt `Every time she went right she would hit the servants with the back of the ladle, every time she went left she hit them with the fist.' (2002 qaCpa, 143)
\end{exe} 

In addition, the relator noun \forme{ɯ-taʁ} is required with some categories of nouns in specific contexts.

First, \forme{ɯ-taʁ} in combination with a noun designating a type of transport (boat, car etc) can express the transportation used in a particular trip. With the verb \japhug{ɕe}{go} taking the orientation \textsc{upwards}, the meaning is `to take (a boat, a car etc)' as in (\ref{ex:ZmbrW.WtaR.toCe}). 

\begin{exe}
\ex \label{ex:ZmbrW.WtaR.toCe}
\gll ʑmbrɯ ɯ-taʁ to-ɕe tɕe jo-nɯ-ɕe \\
boat \textsc{3sg}.\textsc{poss}-on \textsc{ifr}:\textsc{up}-go \textsc{lnk} \textsc{ifr}-\textsc{vert}-go \\
\glt `He took a boat and went back home.' (150907 niexiaoqian-zh, 140)
\end{exe} 

The verb \japhug{ɬoʁ}{come out} with the orientation \textsc{downwards} in combination with \forme{ɯ-taʁ} expresses the meaning `take off (a boat, a car etc)' as in (\ref{ex:ZmbrW.WtaR.pjAlhoRnW}); the relator noun itself is neutral as to the type of location (static, motion from, motion towards)), a topic discussed in more detail in §\ref{sec:relator.postposition.location}.

\begin{exe}
\ex \label{ex:ZmbrW.WtaR.pjAlhoRnW}
\gll ʑmbrɯ ɯ-taʁ pjɤ-ɬoʁ-nɯ  \\
boat \textsc{3sg}.\textsc{poss}-on \textsc{ifr}:\textsc{down}-come.out-\textsc{pl} \\
\glt `They took off the boats.' (140508 shier ge tiaowu de gongzhu-zh, 149)
\end{exe} 

%krɤɕi <qiche> ɯ-taʁ pɯ-ɣe-a.
To designate the means of transportation used during the whole  travel (as opposed to the begin and end points of the travel as in \ref{ex:ZmbrW.WtaR.toCe} and \ref{ex:ZmbrW.WtaR.pjAlhoRnW}), the relator noun \japhug{ɯ-ŋgɯ}{inside} can also be used (§\ref{sec:WNgW}).

The relator noun \forme{ɯ-taʁ} also occurs to refer to a written medium, as in (\ref{ex:jWGi.WtaR.tufCAt}).

\begin{exe}
\ex \label{ex:jWGi.WtaR.tufCAt}
\gll  tɕe jɯɣi ɯ-taʁ tu-fɕɤt tɕe, \\
\textsc{lnk} book \textsc{3sg}.\textsc{poss}-on \textsc{ipfv}-tell \textsc{lnk} \\
\glt `It is told in a book, that...' (27-qaCpa, 5)
\end{exe} 

\subsubsection{The relator noun \japhug{ɯ-ŋgɯ}{inside}} \label{sec:WNgW}
The relator noun \japhug{ɯ-ŋgɯ}{inside} is only partially grammaticalized (§\ref{sec:other.locative.relator}). However, it is selected by a few verbs to mark an oblique argument, in particular the goal of the transitive verbs \japhug{rku}{put in} (§\ref{sec:ditransitive.indirective}) and \forme{lɤt} in the meaning `sow, spread' as in (\ref{ex:khWtsa.WNgW.turkunW}).

\begin{exe}
\ex \label{ex:khWtsa.WNgW.turkunW}
\gll khɯtsa ɯ-ŋgɯ tɯ-ci tu-rku-nɯ tɕe, \\
bowl \textsc{3sg}.\textsc{poss}-inside  \textsc{indef}.\textsc{poss}-water \textsc{ipfv}-put.in-\textsc{pl} \textsc{lnk} \\
\glt `People used to put water in a bowl, and...' (29-mWBZi, 129)
\end{exe} 

\begin{exe}
\ex \label{ex:tWji.WNgW.chWwGlAt}
\gll tɯ-ji ɯ-ŋgɯ zɯ tsʰɤt ɯ-ɣli nɯ cʰɯ́-wɣ-lɤt tɕe, tɤ-rɤku wuma ʑo tu-sɤpe cʰa \\
\textsc{indef}.\textsc{poss}-field  \textsc{3sg}.\textsc{poss}-inside \textsc{loc} goat \textsc{3sg}.\textsc{poss}-manure \textsc{dem} \textsc{ipfv}:\textsc{downstream}-\textsc{inv}-release \textsc{lnk} \textsc{indef}.\textsc{poss}-crops really \textsc{emph} \textsc{ipfv}-do.well can:\textsc{fact} \\
\glt `If one spreads goat manure on the fields, it can make the crops grow really well.' (05-qaZo, 33)
\end{exe} 

It is also found in combination with motion verb to express the means of transportation used during the whole travel, as in (\ref{ex:chuzu.WNgW}).

\begin{exe}
\ex \label{ex:chuzu.WNgW}
\gll  <chuzu> ɯ-ŋgɯ tʰɯ-ɣe-a tɕe \\
taxi \textsc{3sg}.\textsc{poss}-inside \textsc{aor}:\textsc{downstream}-come[II]-\textsc{1sg} \textsc{lnk} \\
\glt `I came on a taxi.' (2010-01-Dpalcan, 28)
\end{exe} 



\subsubsection{Locative relator nouns and locative postpositions} \label{sec:relator.postposition.location}
 All locative relator nouns can be also used with the locative postpositions \forme{zɯ}, \forme{ri}, \forme{tɕu} and the fused forms of the latter two with the demonstratives \forme{nɯre} and \forme{nɯtɕu} (§\ref{sec:locative}). Example (\ref{ex:WtaR.ri.Wpa.ri}) illustrates the use of \japhug{ɯ-taʁ}{on, above} and \japhug{ɯ-pa}{below, under} with the locative \forme{ri} expressing both static position (with the existential verb \japhug{tu}{exist}) and motion towards (with the verb \japhug{lɤt}{throw}, here specifically meaning `direct water').

\begin{exe}
\ex \label{ex:WtaR.ri.Wpa.ri}
\gll ɯ-tʰɤcu maŋtʰi qʰajŋgɯ nɯ kɯ, nɤki, βɣa ɯ-pa ri tɕʰɯŋkʰɤr tu tɕe, tɕʰɯŋkʰɤr ɯ-taʁ ri cʰɯ-lɤt tɕe, tɕʰɯŋkʰɤr ɯ-taʁ nɯre ri βɣɤrnɤjwaʁ kɤ-ti tu tɕe \\
\textsc{3sg}.\textsc{poss}-downstream downstream water.trough \textsc{dem} \textsc{erg} \textsc{filler} mill \textsc{3sg}.\textsc{poss}-under \textsc{loc} water.wheel exist:\textsc{fact} \textsc{lnk} water.wheel \textsc{3sg}.\textsc{poss}-on \textsc{loc} \textsc{ipfv}:\textsc{downstream}-release \textsc{lnk} \textsc{lnk} water.wheel \textsc{dem}:\textsc{loc} \textsc{loc} blades \textsc{obj}:\textsc{pcp}-say exist:\textsc{fact} \textsc{lnk} \\
\glt `The inferior water trough -- under the mill there is a water wheel -- (the water trough) directs (the water) onto that water wheel -- on the water wheel there are things called `blades'.' (06-BGa, 27-31)
\end{exe}

Example (\ref{ex:Wpa.ri.tulhoR}) illustrates \japhug{ɯ-pa}{below, under} with the locative \forme{ri} expressing motion from a place.

\begin{exe}
\ex \label{ex:Wpa.ri.tulhoR}
\gll ɯ-tʰoʁ ɯ-pa ri tu-ɬoʁ nɯra mɯ́j-cʰa \\
\textsc{3sg}.\textsc{poss}-earth \textsc{3sg}.\textsc{poss}-below \textsc{loc} \textsc{ipfv}:\textsc{up}-come.out \textsc{dem}:\textsc{pl} \textsc{neg}:\textsc{sens}-can \\
\glt `(Its shoots) cannot come out from under the ground.'  (15-babW, 45)
\end{exe}

Examples (\ref{ex:khri.WtaR.zW}) and (\ref{ex:WtaR.zW.kAmdzW}) above show the combination of \japhug{ɯ-taʁ}{on, above} with the locative \forme{zɯ}, also for static position and motion.

\begin{exe}
\ex \label{ex:khri.WtaR.zW}
\gll χsɤr kʰri ɯ-taʁ zɯ pjɤ-rɤʑi tɕe ɯ-tɯ-ɣɤχsrɯ pjɤ-saχaʁ ʑo. \\
gold bed \textsc{3sg}.\textsc{poss}-on \textsc{loc} \textsc{ifr}.\textsc{ipfv}-stay \textsc{lnk} \textsc{3sg}.\textsc{poss}-\textsc{nmlz}:\textsc{deg}-be.handsome \textsc{ifr}.\textsc{ipfv}-be.extremely \textsc{emph} \\
\glt `He was sitting on the golden bed, very handsome.' (2014-kWlAG, 409)
\end{exe}

The uses are attested with the locative \forme{tɕu}, as shown by (\ref{ex:WtaR.nWtCu.pjArAZi}) and (\ref{ex:WtaR.nWtCu.YAXtAr}). No clear criterion accounting for the presence or absence of these locative postpositions in combination with the relator nouns has been found.

\begin{exe}
\ex \label{ex:WtaR.nWtCu.pjArAZi}
\gll  rɟɤmtsʰu ɣɯ ɯ-rkɯ qambɯt ɯ-taʁ nɯtɕu pjɤ-rɤʑi. \\
ocean \textsc{gen} \textsc{3sg}.\textsc{poss}-side sand \textsc{3sg}.\textsc{poss}-on \textsc{dem}:\textsc{loc} \textsc{ifr}.\textsc{ipfv}-stay \\
\glt `He stayed on the beach.' (140511 xinbada-zh, 167)
\end{exe}

\begin{exe}
\ex \label{ex:WtaR.nWtCu.YAXtAr}
\gll tɕe ɯ-taʁ nɯtɕu pɣɤmuj tɯ-spra nɯ ɲɤ-χtɤr. \\
\textsc{lnk} \textsc{3sg}.\textsc{poss}-on \textsc{dem}:\textsc{loc} feather one-handful \textsc{ifr}-scatter \\
\glt `He scattered a handful of feathers on it.' (28-smAnmi, 327)
\end{exe}

There is one case of a fossilized \forme{zɯ} locative (§\ref{sec:core.locative}) with the relator noun \japhug{ɯ-ŋgɯ}{inside}, the form \japhug{ɯ-ŋgɯz}{inside, among}, which is used in particular to single out an element for a group, in particular in a superlative construction (\ref{ex:kAndZWRi.nWNgWz}) (§\ref{sec:stu.superlative}) or to describe an intermediate colour (with stative verbs, as in example \ref{ex:arNi.WNgWz}).

\begin{exe}
\ex \label{ex:kAndZWRi.nWNgWz} 
\gll kɤndʑiʁi nɯ-ŋgɯz stu kɯ-xtɕi nɯnɯ kɯ ... nɯra ntsɯ tu-ti pjɤ-ŋu. \\
\textsc{coll}:siblings \textsc{3pl}.\textsc{poss}-among:\textsc{loc} most \textsc{sbj}:\textsc{pcp}-be.small \textsc{dem} \textsc{erg} { } \textsc{dem}:\textsc{pl} always \textsc{ipfv}-say \textsc{ifr}.\textsc{ipfv}-be \\
\glt `The youngest among the sisters was always saying ...' (150828 donglang, 26)
  \end{exe}
  
  \begin{exe}
\ex \label{ex:arNi.WNgWz} 
\gll nɯ ɯ-mdoʁ nɯ aj kɤ-ti mɯ́j-spe-a ma arŋi ɯ-ŋgɯz kɯnɤ pɣi kɯ-fse   \\
\textsc{dem} \textsc{3sg}.\textsc{poss}-colour \textsc{dem} \textsc{1sg} \textsc{inf}-say \textsc{neg}:\textsc{sens}-be.able[III]-\textsc{1sg} \textsc{lnk} be.green:\textsc{fact} \textsc{3sg}.\textsc{poss}-inside:\textsc{loc} also be.grey:\textsc{fact} \textsc{sbj}:\textsc{pcp}-be.like \\
\glt `I cannot say its colour, it is somewhere between green and grey.' (06-qaZmbri, 56)
    \end{exe}
    
 
 \subsection{Temporal relator nouns} \label{sec:relator.temporal}
Many temporal postpositions are found in Japhug (§\ref{sec:egressive}, §\ref{sec:terminative}, §\ref{sec:temporal.postpositions}), and temporal relator nouns are relatively fewer. The relator  \japhug{ɯ-raŋ}{during, the time when} (from Tibetan \tibet{རིང་}{riŋ}{long, during, when}) is very commonly used with subordinate clauses (§\ref{sec:temporal.reference}), but can also follow temporal adverbs and nouns, as in (§\ref{ex:XCitka.WraN}).
 
   \begin{exe}
\ex \label{ex:XCitka.WraN} 
\gll χɕitka ɯ-raŋ tɕe, tɕendɤre pɣa ra, nɯ-kɤ-ndza maka cʰɯ-me ɲɯ-ŋu tɕe, tɕendɤre ɯ-tɯ-mtsɯr pjɤ-saχaʁ ʑo ɲɯ-ŋu,  \\
spring \textsc{3sg}.\textsc{poss}-during \textsc{lnk} \textsc{lnk} bird \textsc{pl} \textsc{3sg}.\textsc{poss}-\textsc{obj}:\textsc{pcp}-eat at.all \textsc{ipfv}-not.exist \textsc{sens}-be \textsc{lnk} \textsc{lnk} \textsc{3sg}.\textsc{poss}-\textsc{nmlz}:\textsc{deg}-be.hungry \textsc{ifr}.\textsc{ipfv}-be.extremely \textsc{emph} \textsc{sens}-be \\
\glt `In spring, the birds' food has gone out and (that crow) was extremely hungry.' (kWjujmAlu, 6)
\end{exe}

The inalienably possessed noun \forme{ɯ-rɤɣ}, whose basic meaning is `specific (and predictable) time' as (\ref{ex:WrAG.GAZu}) (see also §\ref{sec:denom.arA}), can be used as relator noun as in (\ref{ex:nW.WrAG.tCe.li}) to mean `the exact time when'.

\begin{exe}
\ex \label{ex:WrAG.GAZu}
\gll ɯ-rɤɣ ɣɤʑu ma nɤkinɯ sɲikuku ʑo ɲɯ-maʁ \\
\textsc{3sg}.\textsc{poss}-specific.time exist:\textsc{sens} \textsc{lnk} filler every.day \textsc{emph} \textsc{sens}-not.be \\
\glt `(The rut) occurs at a specific time, not every day.' (27-qartshAz, 155)
\end{exe}

\begin{exe}
\ex \label{ex:nW.WrAG.tCe.li}
\gll tɕendɤre ɯ-fsaqʰe tɕe nɯ ɯ-rɤɣ tɕe li lo-ɣi \\
\textsc{lnk} \textsc{3sg}.\textsc{poss}-next.year \textsc{lnk} \textsc{dem}  \textsc{3sg}.\textsc{poss}-specific.time \textsc{lnk} again \textsc{ifr}:\textsc{upstream}-come \\
\glt `It came back at exactly the same time the next year.' (22-qomndroN, 46)
\end{exe}


There are two relator nouns that can be used as antonyms of the preposition \japhug{ɕɯŋgɯ}{before} (§\ref{sec:temporal.postpositions}),  \japhug{ɯ-mpʰru}{after, following} (from \tibet{འཕྲོ་}{ⁿpʰro}{remnant}, in particular in expressions such as \tibet{དེའི་འཕྲོར་}{deɦi.ⁿpʰror}{next, after that}) and \japhug{ɯ-qʰu}{after}, which is also used for spatial relations (§\ref{sec:other.locative.relator}). These two nouns are not synonymous. The former must be used in the expression \japhug{ci ɯ-mpʰru ci}{one after the other} (\ref{ex:ci.Wmphru.ci}), and means `next' as in (\ref{ex:nW.Wmphru.tCe}), and can follow a subordinate clause, expressing temporal subsequence (§\ref{sec:subsequence.neutral}), though most commonly a demonstrative pronoun such as \forme{nɯ} or \forme{nɯnɯ} referring to the previous clause is used instead, as in (\ref{ex:nWnW.Wmphru.tWsla}).

\begin{exe}
\ex \label{ex:ci.Wmphru.ci}
 \gll ci ɯ-mpʰru ci ʑo ko-nɯpoʁ qʰe lo-sɯ-ɣe. \\
 one \textsc{3sg}.\textsc{poss}-after one \textsc{emph} \textsc{ifr}-kiss \textsc{lnk} \textsc{ifr}:\textsc{upstream}-\textsc{caus}-come  \\
\glt `(The mother) kissed (her children) one after the one and had them come (inside the house).' (160701 poucet2, 38)
\end{exe}

\begin{exe}
\ex \label{ex:nW.Wmphru.tCe}
 \gll  tɕe nɯ ɯ-mpʰru tɕe tɕʰi to-ti? \\
 \textsc{lnk} \textsc{dem} \textsc{3sg}.\textsc{poss}-after \textsc{lnk} what \textsc{ifr}-say \\
\glt `What does it say next?' (140522 Kamnyu zgo)
\end{exe}

\begin{exe}
\ex \label{ex:nWnW.Wmphru.tWsla}
 \gll tɕe tɯ-rdoʁ pjɤ-sat tɕe tɕeri nɯnɯ ɯ-mpʰru tɯ-sla jamar nɯnɯ tɯ-ci ɯ-taʁ nɯtɕu ɯ-zda nɯ lo-ɕe nɤ cʰɤ-ɣi, lo-ɕe nɤ cʰɤ-ɣi ɲɤ-ɕar \\
\textsc{lnk} one-piece \textsc{ifr}-kill \textsc{lnk} \textsc{lnk} \textsc{dem} \textsc{3sg}.\textsc{poss}-after one-month about \textsc{dem} \textsc{indef}.\textsc{poss}-water \textsc{3sg}.\textsc{poss}-on \textsc{dem}:\textsc{loc} \textsc{3sg}.\textsc{poss}-mater \textsc{dem} \textsc{ifr}:\textsc{upstream}-go \textsc{lnk} \textsc{ifr}:\textsc{downstream}-come  \textsc{ifr}:\textsc{upstream}-go \textsc{lnk} \textsc{ifr}:\textsc{downstream}-come \textsc{ifr}-search \\
\glt `(Someone) killed one of them, and after that, (the other one) flew along the river searching for its mate for about one month.' (22-qomndroN, 44)
\end{exe}

The locative relator \japhug{ɯ-qʰu}{after} also has temporal uses, in particular to build temporal subordinate clauses (§\ref{sec:subsequence.neutral}), and it can also be used after noun phrases as in the expression \japhug{saχsɯ ɯ-qʰu}{after lunch, afternoon}, or simply following a demonstrative as in \japhug{nɯ ɯ-qʰu}{after that}.
%tɕendɤre χsɯ-xpa ɯ-qhu tɕe tɕendɤre,
%nɯnɯ nɤki, ɯ-sloχpɯn ɣɯ ɯ-laχɕi nɯ lonba ko-spa.
%150902 luban, 149

Another originally locative relator noun has temporal functions: \japhug{ɯ-pɤrtʰɤβ}{the middle of}, which is used after the demonstrative \forme{nɯ} to mean `in the meantime'  (\ref{ex:nW.WpArthAB.tCe}) and also occurs with temporal clauses with the meaning `while'.

\begin{exe}
\ex \label{ex:nW.WpArthAB.tCe}
\gll aʑo ku-ɕe-a tɕe z-ɲɯ-saʁjar-a ŋu tɕe nɯ ɯ-pɤrtʰɤβ tɕe nɯʑo ra kɯ ɕ-pɯ-rla-nɯ je tɕe lɤ-tsɯm-nɯ je \\
\textsc{1sg} \textsc{ipfv}:\textsc{east}-go-\textsc{1sg} \textsc{lnk} \textsc{tral}-\textsc{sens}-delay-\textsc{1sg} be:\textsc{fact} \textsc{lnk} \textsc{dem} \textsc{3sg}.\textsc{poss}-in.the.middle \textsc{loc} \textsc{2pl} \textsc{pl} \textsc{erg} \textsc{tral}-\textsc{imp}-untie-\textsc{pl} \textsc{sfp} \textsc{lnk} \textsc{imp}:\textsc{upstream}-take.away-\textsc{pl} \textsc{sfp} \\
\glt `I will go there and delay him$_i$, and in the meantime, you go there and untie him$_j$, and take him$_j$ away.' (tWJo 2005, 37)
\end{exe}

\subsection{Semi-grammaticalized relator nouns} \label{sec:semi.grammaticalized.relator}

\subsubsection{\japhug{tɯ-jaʁ}{hand}} \label{sec:tWjaR}
The noun \japhug{tɯ-jaʁ}{hand} occurs with several verbs in fixed collocation, the recipient of the action being indexed by the possessive prefix on this noun.

It is found with \japhug{kʰo}{give} to express the meaning `hand over to' as in (\ref{ex:ajaR.tAkhAm}) and (\ref{ex:nAjaR.YWkhoj}).

\begin{exe}
\ex \label{ex:ajaR.tAkhAm}
\gll nɤki tɤtʂu nɯ a-jaʁ tɤ-kʰɤm! \\
\textsc{dem}:\textsc{medial} lamp \textsc{dem}, \textsc{1sg}.\textsc{poss}-hand \textsc{imp}:\textsc{up}-give[III] \\
\glt `Hand over to me (up here) that lamp.' (140511 alading-zh, 122)
\end{exe}

\begin{exe}
\ex \label{ex:nAjaR.YWkhoj}
\gll kɯki mbro ki nɤ-jaʁ ɲɯ-kʰo-j ŋu \\
\textsc{dem}.\textsc{prox} horse \textsc{dem}.\textsc{prox} \textsc{2sg}.\textsc{poss}-hand \textsc{ipfv}-give-\textsc{1pl} be:\textsc{fact} \\
\glt  `(If you succeed), we will give you this horse.'  (X1-qachGa, 62)
\end{exe}

The collocation of \forme{tɯ-jaʁ} with the intransitive verb \japhug{zɣɯt}{reach, arrive} means `receive' or `obtain', as in (\ref{ex:ajaR.anWzGWt}). With the causative form \forme{sɤzɣɯt}, the collocation means `get' (with volition and controlability) as in (\ref{ex:WjaR.junWsAzGWt}) -- the recipient marked by the possessive prefix on  \forme{tɯ-jaʁ} is the same referent as the transitive subject of the main verb.

\begin{exe}
\ex \label{ex:ajaR.anWzGWt}
\gll iɕqʰa tɤtʂu nɯ a-jaʁ a-nɯ-zɣɯt ra \\
the.aforementioned lamp \textsc{dem} \textsc{1sg}.\textsc{poss}-hand \textsc{irr}-\textsc{pfv}-reach be.needed:\textsc{fact} \\
\glt   `I have to obtain this lamp.' (140511 alading-zh, 212)
\end{exe}

\begin{exe}
\ex \label{ex:WjaR.junWsAzGWt}
\gll tɕʰi ra na-sɯso ʑo nɯ, ɯ-jaʁ ju-nɯ-sɯ-ɤzɣɯt pjɤ-cʰa.  \\
what \textsc{pl} \textsc{aor}:3\fl{}3'-think \textsc{emph} \textsc{dem} \textsc{3sg}.\textsc{poss}-hand \textsc{ipfv}-\textsc{auto}-\textsc{caus}-reach \textsc{ifr}.\textsc{ipfv}-can \\
\glt  `He was able to get whatever he wanted.' (140508 benling gaoqiang de si xiongdi-zh, 47)
\end{exe}

The collocation of the noun \japhug{tɯ-jaʁ}{hand} with the verb \japhug{ɣi}{come} also means `obtain' or `find', as in (\ref{ex:tWjaR.mWjGi}).

\begin{exe}
\ex \label{ex:tWjaR.mWjGi}
\gll jinde tɕe ɯ-kɯ-sat koŋla maŋe tɕe, nɯ qarma ɯ-muj kɯnɤ tɯ-jaʁ mɯ́j-ɣi wo\\
nowadays \textsc{lnk} \textsc{3sg}.\textsc{poss}-\textsc{sbj}:\textsc{pcp}-kill completely not.exist:\textsc{sens} \textsc{lnk} \textsc{dem} crossoptilon \textsc{3sg}.\textsc{poss}-feather also \textsc{genr}.\textsc{poss}-hand \textsc{neg}:\textsc{sens}-come \textsc{sfp}\\
\glt `Nowadays nobody kills crossoptilons, one cannot even get their feathers (to use as ornaments).' (23-qapGAmtWmtW, 170)
\end{exe}

The noun \japhug{tɯ-jaʁ}{hand} does not occur in metaphoric use outside of these collocations; it cannot be used in particular to mark any adjunct. Other noun-verb collocations where an experiencer or a recipient is marked as possessor of the noun are described in §\ref{sec:light.verb}.


\subsubsection{Cause and beneficiary} \label{sec:IPN.cause}
Two semi-grammaticalized inalienably possessed nouns can be used to express beneficiaries: \japhug{ɯ-ndʐa}{reason} as in (\ref{ex:nAndzxa.pWGea}), and \japhug{ɯ-skɤt}{speech, sound} as in (\ref{ex:kWpe.GW.WskAt}), though the latter is very rare. These two nouns also take complement clauses (§\ref{sec:nouns.cause.complement}, §\ref{sec:nouns.speech.complement}) instead of noun phrases.

\begin{exe}
\ex \label{ex:nAndzxa.pWGea}
\gll nɤʑo nɤ-ndʐa pɯ-ɣe-a ɕti \\
\textsc{2sg} \textsc{2sg}.\textsc{poss}-reason \textsc{pst}:\textsc{down}-come[II]-\textsc{1sg} \\
\glt `I came down (here) for you.' (2003 tWxtsa, 32)
\end{exe}

\begin{exe}
\ex \label{ex:kWpe.GW.WskAt}
\gll pʰa kɤmɲɯ ɣɯ kɯ-pe ɣɯ ɯ-skɤt nɯtɕu to-βzu-nɯ ɲɯ-ŋu. \\
all  \textsc{topo} \textsc{gen} \textsc{sbj}:\textsc{pcp}-be.good \textsc{gen} \textsc{3sg}.\textsc{poss}-speech \textsc{dem}:\textsc{loc} \textsc{ifr}-make-\textsc{pl} \textsc{sens}-be \\
\glt `People built it there for the benefit of all of Kamyu.' (150904 tshAcim, 32)
\end{exe}

The noun \japhug{ɯ-tʰɯrʑi}{mercy} (borrowed from Tibetan \tibet{ཐུགས་རྗེ་}{tʰugs.rdʑe}{compassion}, also found in a collocation §\ref{sec:borrowed.NV}), occurs in a semi-grammaticalized construction meaning `thanks to', in combination with the ergative \forme{kɯ} as an intrumental adjunct (\ref{ex:nAthWrZi.kW}), or in absolutive form as a nominal predicate (\ref{ex:smAn.WthWrZi}).

  \begin{exe}
\ex \label{ex:nAthWrZi.kW}
\gll pɯ-cʰa-a nɯra nɤj nɤ-tʰɯrʑi kɯ pɯ-cʰa-a \\
\textsc{aor}-can-\textsc{1sg} \textsc{dem}:\textsc{pl} \textsc{2sg} \textsc{2sg}.\textsc{poss}-mercy \textsc{erg} \textsc{aor}-can-\textsc{1sg} \\
\glt `I succeeded all thanks to you.' (2011 04-smanmi, 112)
  \end{exe}

  \begin{exe}
\ex \label{ex:smAn.WthWrZi}
\gll   ma jinde nɯra kɯ-fse kɤ-mtsʰɤm maŋe tɕe, smɤn ɯ-tʰɯrʑi ɯmɤ-ŋu ma \\
\textsc{lnk} nowadays \textsc{dem}:\textsc{pl} \textsc{sbj}:\textsc{pcp}-be.like \textsc{inf}-hear not.exist:\textsc{sens} \textsc{lnk} medicine   \textsc{3sg}.\textsc{poss}-mercy \textsc{prob}-be:\textsc{fact} \textsc{lnk} \\
\glt `Nowadays we do not hear about (this disease), probably thanks to the medicine.' (27-tWfCAl, 52)
  \end{exe}  

The noun \japhug{ɯ-xɕɤt}{strength} is also often used in instrumental adjuncts with the ergative to express cause, as in (\ref{ex:WxCAt.kW2}).

  \begin{exe}
\ex \label{ex:WxCAt.kW2}
\gll   tɤ-zdɯɣ ɯ-xɕɤt kɯ pjɯ-si ɕti, \\
  \textsc{indef}.\textsc{poss}-toil  \textsc{3sg}.\textsc{poss}-strength \textsc{erg/instr} \textsc{ipfv}-die be:\textsc{aff}:\textsc{fact} \\
 \glt `(The bee) dies of exhaustion.' (26 GZo, 40)
  \end{exe}
