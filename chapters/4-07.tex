 \chapter{Tense, aspect, modality and evidentiality} \label{chap:tame}

\section{Introduction} \label{sec:tame.intro}
All finite verb forms in Japhug have one primary TAME (tense, aspect, modality, evidentiality) category, encoded by  orientation preverbs (§\ref{sec:preverb.TAME}), stem alternations (§\ref{sec:stem.TAME})  and additional affixes in some cases (§\ref{sec:other.TAME}). Primary TAME categories can be combined with secondary aspectual and modal prefixes (§\ref{sec:second.aspect}, §\ref{sec:second.modal}).

\tabref{tab:primary.TAME}  summarizes the primary TAME categories in Japhug. Labels such as `Imperfective' or `Egophoric Present' are 
conventional names for Japhug-specific morphological verbal categories, and therefore capitalized in this chapter and in the whole 
grammar following the convention proposed by \citet[674]{haspelmath10categories} and others. The use of these terms makes no implication 
that they have any crosslinguistic validity or that the Japhug categories are amenable to direct comparison with TAME categories in 
other languages.

Apart from the Factual Non-Past, all primary TAME categories take one orientation preverb. Some categories, such as the Sensory, are marked by the same preverb on all verbs. Other categories select a preverb belonging to one of the four series (A, B, C or D, see \tabref{tab:TAME.preverb} below and \tabref{sec:preverb.TAME}, §\ref{sec:kamnyu.preverbs}), whose orientation is lexically specified by the verb (§\ref{sec:lexicalized.orientation}).

\begin{table}
\caption{Primary TAME categories} \label{tab:primary.TAME} 
\begin{tabular}{lllllll}
\toprule
&	&	Stem&	Preverb & Reference\\
\midrule
Factual Non-Past&	\textsc{fact} &	1 or 3&	no preverb &§\ref{sec:factual} \\
Sensory&	\textsc{sens} &	1 or 3&	\forme{ɲɯ-} &§\ref{sec:sensory}\\
Egophoric Present&	\textsc{prs} &	1 or 3&	\forme{ku-}&§\ref{sec:egophoric} \\
\midrule
Imperfective&	\textsc{ipfv} &	1 or 3&	B & §\ref{sec:imperfective} \\
\midrule
Aorist&	\textsc{aor} &	2&	A or C & §\ref{sec:aor} \\
Past Imperfective&	\textsc{pst}.\textsc{ipfv} &	2&	\forme{pɯ-} & §\ref{sec:pst.ifr.ipfv}\\
Inferential&	\textsc{ifr} &	1&	D &§\ref{sec:ifr} \\
Inferential Imperfective&	\textsc{ifr}.\textsc{ipfv} &	1&	\forme{pjɤ-}& §\ref{sec:pst.ifr.ipfv}\\
\midrule
Irrealis&	\textsc{irr} &	1 or 3&	\forme{a-} + A&§\ref{sec:irrealis} \\
Imperative&	\textsc{imp} &	1 or 3&	A&§\ref{sec:imperative}, §\ref{sec:prohibitive} \\
Dubitative &	\textsc{dub} &	1 or 3&	\forme{ku-} + autive&§\ref{sec:dubitative} \\
\bottomrule
\end{tabular}
\end{table}

Primary TAME categories in Japhug are divided into four groups: Non-Past (Factual, Egophoric Present, Sensory, §\ref{sec:TAME.npst}), Imperfective (which is mainly used in subordinate clauses or in periphrastic constructions, §\ref{sec:imperfective}), Past (Aorist, Inferential, §\ref{sec:TAME.pst}) and Modal (Imperative, Irrealis, Dubitative, §\ref{sec:TAME.modal}). This classification is based on both semantic and formal features, and is justified in each of the sections of this chapter.

In addition to the primary categories, periphrastic TAME categories are built by combining finite verbs with the copula \japhug{ŋu}{be}. The main verb can be in the Imperfective (§\ref{sec:ipfv.periphrastic.TAME}), §\ref{sec:pst.ifr.ipfv.periphrastic}), in the Factual Non-Past to express periphrastic Proximative (§\ref{sec:proximative.periphrastic}) or in the Inferential  (§\ref{sec:ifr}). The use of the copula as postverbal TAME auxiliary should not be confused with its focalizing function (§\ref{sec:focalization.final.copula}).


\subsection{Preverbs} \label{sec:preverb.TAME}
Some of the Primary TAME categories require the same preverb with all verbs (Egophoric Present, Sensory, Dubitative). The other categories (Imperfective, Aorist, Inferential, Irrealis, Imperative) only specify preverb type (§\ref{sec:preverb.TAME.morphology}), while the orientation of the preverb is either lexically determined, or expresses spatial orientation (in the case of orientable verbs §\ref{sec:orientable.verbs}).


\subsubsection{Morphology} \label{sec:preverb.TAME.morphology}
Orientation preverbs in Kamnyu Japhug come in four series, which contribute (in combination with other formatives and morphological devices) to mark TAME, transitivity and person. Each series comprises seven orientations, divided into Upper orientations (up, upstream, eastwards), Lower orientation (down, downstream, westwards) and unspecified orientation (§\ref{sec:kamnyu.preverbs}). Table  \ref{tab:TAME.preverb} lists the rules of preverb formation. The \forme{C-} is the initial consonant of the preverb, which encodes the orientation: \forme{t-} \textsc{upwards}, \forme{p-} \textsc{downwards}, \forme{l-} \textsc{upstream}, \forme{tʰ-/cʰ-} \textsc{downstream}, \forme{k-} \textsc{eastwards}, \forme{n-/ɲ-} \textsc{westwards} and \forme{j-} `unspecified'.

Upper vs. Lower orientation preverbs have a different vocalism (except for type C preverbs, whose vowel is neutralized to \forme{Ca-}), and type B and D lower orientation preverb have a palatalized onset, realized as \forme{cʰ-} for \textsc{downstream} preverbs (rather than $\dagger$\forme{tʰj-}) and as \forme{ɲ-} for \textsc{westwards} preverbs. A complete list of all preverbs in presented in \tabref{tab:orientation.preverbs} (§\ref{sec:kamnyu.preverbs}). Only types A and B are attested on non-finite forms (§\ref{sec:subject.participle.other.prefixes}).

\begin{table}
\caption{Orientation preverbs and TAME categories} \label{tab:TAME.preverb}
\begin{tabular}{llll}
\lsptoprule
& Upper & Lower & TAME categories \\
\midrule
A & \forme{Cɤ-} &\forme{Cɯ-} & \textsc{aor}, \textsc{pst}.\textsc{ipfv}, \textsc{imp}, \textsc{irr} \\
B &\forme{Cu-} &\forme{Cjɯ-} & \textsc{ipfv}, \textsc{sens}, \textsc{prs}, \textsc{dub} \\
C &\forme{Ca-}&\forme{Ca-} &\textsc{aor} (3\flobv{}) \\
D & \forme{Co-} &\forme{Cjɤ-}  & \textsc{ifr}, \textsc{ifr}.\textsc{ipfv} \\
\lspbottomrule
\end{tabular}
\end{table}


\subsubsection{Ambiguity with type A preverbs} \label{sec:ambiguity.preverb}
Type A preverbs occur with a wide range of TAME categories. In particular, Aorist and Imperative both select the type A preverb, without any additional prefix. A few ambiguous forms do exist in the case of intransitive verbs in particular: for instance, \forme{kɤ-rɤʑi} with the type A \textsc{eastwards} preverb \forme{kɤ-} and the verb \japhug{rɤʑi}{stay} can either be analyzed as an \textsc{2sg} imperative `stay!' or as a \textsc{3sg} Aorist `he stayed'. However, four additional morphological marks can help disambiguating between the two categories in some contexts. 

First, Aorist selects stem II (§\ref{sec:stem2}), and Imperative stem III (§\ref{sec:stem3}, §\ref{sec:stem.TAME}). Second,  transitive verbs in Aorist take type C preverbs instead of type A. Third, the \forme{-t} past tense suffix is found in the Aorist, but not in the Imperative (§\ref{sec:other.TAME}). Fourth, the rules of vowel contraction (§\ref{sec:contraction}) are different across TAME categories: the contracting vowel of the stem merges with the preverb as \forme{Ca-} in the Aorist and Past Imperfective, and as \forme{Cɤ-} in the Imperative and the Irrealis. For instance, a contrast can be observed between the \textsc{3sg} Aorist \ipa{kamdzɯ} (\forme{kɤ-amdzɯ}, \textsc{aor}-sit) `s/he sat down' and the \textsc{2sg} Imperative  \ipa{kɤmdzɯ} (\forme{kɤ-ɤmdzɯ}, \textsc{imp}-sit) `sit!'.

Syncretism occurs in the cases of verbs whose lexical orientation is  \textsc{westwards} and \textsc{eastwards}, the Imperfective of such verbs is identical to the Egophoric Present or Dubitative on the one hand, and the Sensory on the other hand (except in specific contexts, see §\ref{sec:neg.allomorphs}, §\ref{sec:existential.basic}). For instance, the form \forme{ku-omdzɯ-a} of the verb \japhug{amdzɯ}{sit}, which takes the \textsc{eastwards} preverbs, can either be analyzed as Imperfective (\textsc{ipfv}-sit-\textsc{1sg}) or Egophoric Present (\textsc{prs}-sit-\textsc{1sg}) depending on the context.

Likewise, verbs selecting the \textsc{downwards} orientation have Aorist and Inferential forms that are identical to their Past Imperfective and Inferential Imperfective forms, respectively. Given the constraints on the occurrence of the non-periphrastic  Past Imperfective (§\ref{sec:pst.ifr.ipfv}), however, such cases are much rarer.

\subsubsection{Orientation preverbs and aspect} \label{sec:orientation.preverb.aspect}
The Past Imperfective and Inferential Imperfective preverbs \forme{pɯ-} and \forme{pjɤ-} correspond to the \textsc{downwards} preverbs of series A and D, respectively. The grammaticalization process from `downward' orientation to imperfective has occurred (probably independently) in all core Gyalrong languages, and is discussed in detail in \citet{lin11direction}.

There are two additional less grammaticalized cases of orientations used with a specific aspectual meaning, and overriding the intrinsic orientation of the verb.

First, the \textsc{downstream} orientation occurs with stative verb to express progressive change of state occurring naturally, in particular to describe the growth of plants and animals (§\ref{sec:preverb.adjectives.size}). This usage is common with the adverb \japhug{ʑɯrɯʑɤri}{progressively}, and is observed across various TAME categories, including for instance the Inferential (\ref{ex:chAmtshAt}), the Imperfective (\ref{ex:chWwxti.YWCti}, \ref{ex:chWGWrni}) and the Aorist (\ref{ex:chWGWrni}). The verbs \japhug{mtsʰɤt}{be full}, and \japhug{wxti}{be big} normally select the \textsc{upwards} orientation (see example \ref{ex:Wkha.towxti}, §\ref{sec:preverb.adjectives.size}), and \japhug{ɣɯrni}{be red} the \textsc{westwards} orientation.

\begin{exe}
\ex \label{ex:chAmtshAt}
\gll nɯ-muj nɯra ʑɯrɯʑɤri cʰɤ-mtsʰɤt   \\
 \textsc{3pl}.\textsc{poss}-feather \textsc{dem}:\textsc{pl} progressively \textsc{ifr}:\textsc{downstream}-be.full \\
\glt `Progressively, the (little birds) became full of feathers.' (bailingniao he xiaoniao, 9)
\end{exe}

\begin{exe}
\ex \label{ex:chWwxti.YWCti}
\gll qandʐe nɯ xtɕi qʰe kɯ-xtɕɯ\redp{}xtɕi ma maŋe, taqaβ jamar ma maŋe tɕe ʑɯrɯʑɤri cʰɯ-wxti ɲɯ-ɕti. \\
earthworm \textsc{dem} be.small:\textsc{fact} \textsc{lnk} \textsc{sbj}:\textsc{pcp}-\textsc{emph}\redp{}be.small apart.from not.exist:\textsc{sens} needle about apart.from not.exist:\textsc{sens} \textsc{lnk} progressively \textsc{ipfv}:\textsc{downstream}-be.big \textsc{sens}-be.\textsc{aff} \\
\glt `The earthworm is small, only about as small as a needle, and then it grows progressively.' (25-akWzgumba, 123)
\end{exe}

\begin{exe}
\ex \label{ex:chWGWrni}
\gll ʑɯrɯʑɤri tɕe cʰɯ-ɣɯrni tɕe tʰɯ-ɣɯrni tɕe tɕe cʰɯ-tɯt ŋu tɕe,  \\
progressively \textsc{lnk} \textsc{ipfv}:\textsc{downstream}-be.red \textsc{lnk} \textsc{aor}:\textsc{downstream}-be.red \textsc{lnk} \textsc{lnk} \textsc{ipfv}-be.ripe be:\textsc{fact} \textsc{lnk}  \\
\glt `It progressively becomes red, and after it has become red it ripens.' (16-RlWmsWsi, 18)
\end{exe}

Second, the \textsc{upwards} orientation in the Aorist overrides the lexical orientation of stative verbs to indicate a point of temporal reference in temporal clauses. For instance, the negative existential verb \japhug{me}{not exist} (§\ref{sec:preverb.loss}), whose intrinsic orientation is \textsc{westwards}, occurs with the \textsc{upwards} orientation as in (\ref{ex:nWma.tAme}) in the meaning `when there is no...' without inchoative aspect `when ... disappears' as is usually found in the Aorist (see also §\ref{sec:aor.temporal}, \citealt[283, fn 10]{jacques14linking}).

\begin{exe}
\ex \label{ex:nWma.tAme}
\gll  nɯ ma tɤ-me tɕe ɲɯ-pʰɯt-nɯ  \\
\textsc{dem} apart.from \textsc{aor}:\textsc{up}-not.exist \textsc{lnk}  \textsc{ipfv}-take.out-\textsc{pl}   \\
\glt `When there is nothing else than than this, people cut it.' (140427 qamtsWrmdzu, 12)
\end{exe} 

\subsubsection{Orientation preverbs and evidentiality} \label{sec:orientation.preverb.evd}
The Sensory \forme{ɲɯ-} and Egophoric Present \forme{ku-} markers correspond to the \textsc{westwards} and \textsc{eastwards} preverbs of series B, respectively. It is unlikely that a pair of preverbs encoding the solar dimension (§\ref{sec:solar.dimension}) was directly grammaticalized to express an evidential contrast.

However, an extended function of the \textsc{eastwards}/\textsc{westwards} contrast is the expression of centripetal vs. centrifugal orientation (§\ref{sec:centripetal.centrifugal}). The centripetal meaning of the \textsc{eastwards} orientation goes back at least to proto-Gyalrong, since it is attested in Situ, and the metaphorical use of centripetal/cislocative marker to express speaker affectedness could have been further grammaticalized as an egophoric marker. 

\subsection{Stem alternation} \label{sec:stem.TAME}
Stem alternation (§\ref{sec:stem.alternation}) has a lower functional load than preverbs to encode TAME, since stem II is only attested on a handful of irregular verbs (§\ref{sec:stem2}) and stem III, although found on an important number of transitive verbs (§\ref{sec:stem3.form}), is restricted to only a small subset of the paradigms, as it marks both TAME and person/number (§\ref{sec:stem3.distribution}).

Three cases of systematic redundancy between stem alternation and orientation preverbs are observed. First, stem III alternation is redundant as a TAME marker (but not as a transitivity marker, §\ref{sec:transitivity.morphology}) with B type preverbs in the Imperfective, the Sensory and the Egophoric. Second, in 3\flobv{} configurations, stem II and C type preverbs redundantly mark Aorist (§\ref{sec:stem2}). Third, stem III is redundant with the contrast between A and C type preverbs to distinguish between Imperfective and Aorist forms in the case of transitive verbs; taking \japhug{ndza}{eat} as an example of an alternating transitive verb, its \textsc{2sg}\fl{}3 Imperative \forme{tɤ-ndze} `eat it!' (stem III, A preverb) differs from its \textsc{3sg}\flobv{} Aorist \forme{ta-ndza} `s/he ate it' (base stem, C preverb) by both the preverbs and the stems.

Since intransitive verbs lack the contrast between A and C type preverbs (and also stem III), stem II is  the only clue to disambiguate between the third person Aorist and Imperative, which are both marked by A type preverbs: for instance, while non-alternating verbs such as \japhug{rɤʑi}{stay} have ambiguity between these two forms (§\ref{sec:ambiguity.preverb}), irregular alternating verbs such as \japhug{ɕe}{go} make a clear distinction between Aorist \textsc{3sg} (\forme{jɤ-ari} \textsc{aor}-go[II] `s/he went') and Imperative \textsc{2sg} (\forme{jɤ-ɕe} \textsc{imp}-go `go!') and the corresponding dual and plural forms. It is the only case when stem alternation is critical to distinguish between two TAME categories.


\subsection{Other affixes} \label{sec:other.TAME}
Aside from preverbs and stem alternation, formatives used to mark primary TAME categories are few. 

The negative prefixes (slot -5, §\ref{sec:outer.prefixal.chain}) have different forms depending on TAME categories, and one of them, the Sensory Negative \forme{mɯ́j-} (§\ref{sec:neg.allomorphs}) is a portmanteau encoding both polarity and TAME.

The Past Transitive \forme{-t} (slot +1, \forme{-z} in some dialects of Japhug, §\ref{sec:suffixes})  is a secondary exponent of some TAME categories, only found in the 1/2sg\fl{}3 person configurations (§\ref{sec:indexation.mixed}) of Aorist, Perfective and Imperfective Inferential, Past Imperfective (§\ref{sec:TAME.pst}) and Apprehensive (§\ref{sec:apprehensive.morphology}), unless the Progressive \forme{asɯ-} is also present (§\ref{sec:progressive.morphology}). 

The prefix \forme{a-} in slot -6 (§\ref{sec:outer.prefixal.chain}) is the main marker of the Irrealis (§\ref{sec:irrealis.morphology}), together with type A preverbs and (when applicable) stem III.

Secondary TAME categories (§\ref{sec:second.aspect}, §\ref{sec:second.modal}) are encoded by several prefixes in slots -6 and one in -1 (the Progressive §\ref{sec:progressive}).

\subsection{Evidentiality and person} \label{sec:anticipation.person}
As in many languages with evidential systems \citep{sun18evidentials}, person and evidentiality present some degree of interaction in Japhug.

The Sensory (§\ref{sec:sensory}) and Inferential (§\ref{sec:ifr}) have restrictions on the use of first person in declarative clauses. First person subjects are not impossible, but have very specific meanings (see §\ref{sec:sensory.person} and §\ref{sec:inf.1person}).

The Egophoric Present, on the other hand, is not compatible with second person in declarative clauses, and with first person in interrogative ones.

Like most languages of the Tibetosphere, interrogatives sentences generally adopt the perspective of the addressee rather than that of the speaker, causing a phenomenon referred to as `anticipation rule' (\citealt[244]{tournadre14evidentiality}) or `flipping' \citep{sanroque17interrogativity}:   the speaker anticipates the answer of the addressee and uses the form that he expect the addressee will choose to respond to the question. For instance, in example (\ref{ex:WtWsWz}), the speaker uses the Factual because she expects and answer with the Factual such as \forme{sɯz-a} (know:\textsc{fact}-\textsc{1sg}) `I know'.

\begin{exe}
\ex \label{ex:WtWsWz}
\gll nɤj ɯ-tɯ́-sɯz? \\
\textsc{2sg} \textsc{qu}-2-know:\textsc{fact} \\
\glt `Do you know it?' (19 GzW, 8)
\end{exe}

As a result of this change of perspective, compatibilities between evidential markers and first vs. second person are always reversed between declarative and interrogative sentences  (§\ref{sec:egophoric.interrogative}).

The addressee perspective however is not a syntactic rule. The addressee is free to adopt the evidential form suggested by the speaker who asked the question, or to choose another form if he sees fit: see \citet{garrett07symbiosis} for an account of this phenomenon in Tibetan. It is also possible to have in the same question two verbs referring to the addressee with the Egophoric Present in one case and the Sensory in the other, as in (\ref{ex:WkutWpe}). 

\begin{exe}
\ex \label{ex:WkutWpe}
\gll wo, ɯ-kú-tɯ-pe, ɯ-ɲɯ́-tɯ-cʰa? \\
\textsc{interj} \textsc{qu}-\textsc{egoph}-2-be.good \textsc{qu}-\textsc{sens}-2-be.fine \\
\glt `Are you feeling well, are you fine?' (140425 shizi huli he lu-zh, 16)
\end{exe}

  \section{Imperfective} \label{sec:imperfective}
 The Imperfective is one of the most common finite verb form in Japhug, but rarely appears on its own without an auxiliary verb. It mainly occurs in periphrastic TAME constructions (§\ref{sec:ipfv.periphrastic.TAME}) and subordinate clauses (§\ref{sec:ipfv.temporal}, §\ref{sec:ipfv.complement}). A specific hortative function (§\ref{sec:ipfv.hortative}) has developed from its use in complement clauses with a modal auxiliary. In addition, verbs of perceptions are used in the Imperfective in specific contexts (§\ref{sec:ipfv.perception}).
  
  
  \subsection{Morphology} \label{sec:ipfv.morphology}
The Imperfective selects type B preverbs (§\ref{sec:preverb.TAME.morphology}, §\ref{sec:kamnyu.preverbs}), and stem III (§\ref{sec:stem3.form}, §\ref{sec:stem.TAME}) when appropriate. The preverb orientation is not neutralized. \tabref{tab:ipfv:example.33tr} shows the Imperfective \textsc{3sg}\flobv{} and \textsc{3pl}\flobv{} forms of some transitive verbs, illustrating all six orientation preverbs and alternation between stem III (in \textsc{3sg}\flobv{}) and stem I (in \textsc{3pl}\flobv{}).
 
\begin{table}
\caption{Examples of Imperfective verb forms (3\flobv{} transitive configurations)} \label{tab:ipfv:example.33tr}
\begin{tabular}{llll}
\lsptoprule
Verb&   Orientation & Imperfective \\
\midrule
\japhug{ndza}{eat}  & \textsc{upwards} & \forme{tu-ndze}, \forme{tu-ndza-nɯ}   \\
\japhug{ko}{prevail over} &   \textsc{downwards} & \forme{pjɯ-kɤm}, \forme{pjɯ-ko-nɯ}  \\
\midrule
\japhug{lɣa}{dig} &   \textsc{upstream} & \forme{lu-lɣe}, \forme{lu-lɣa-nɯ}  \\
\japhug{βlɯ}{burn} &  \textsc{downstream} & \forme{cʰɯ-βli}, \forme{cʰɯ-βlɯ-nɯ}  \\
\midrule
\japhug{ndo}{take} &  \textsc{eastwards} & \forme{ku-ndɤm}, \forme{ku-ndo-nɯ}  \\
\japhug{sɯso}{think} &  \textsc{westwards} & \forme{ɲɯ-sɯsɤm}, \forme{ɲɯ-sɯso-nɯ}  \\
\lspbottomrule
\end{tabular}
\end{table} 

\tabref{tab:ipfv.paradigms} presents the Imperfective paradigms of the transitive verb \japhug{ndza}{eat} (\textsc{upwards}) and of the intransitive contracting verb \japhug{amdzɯ}{sit} (\textsc{eastwards}), with vowel contracting of the preverb and the vowel of the stem in the first and third person subject forms.

\begin{table}
\caption{Examples of Imperfective paradigms} \label{tab:ipfv.paradigms}
\begin{tabular}{lllll}
\lsptoprule
\textsc{1sg}(\flobv{}) & \forme{tu-\rouge{ndze}-a} & \forme{ku-omdzɯ-a} \\
\textsc{1du}(\flobv{}) & \forme{tu-ndza-tɕi} &\forme{ku-omdzɯ-tɕi}  \\
\textsc{1pl}(\flobv{}) & \forme{tu-ndza-j} &\forme{ku-omdzɯ-j}  \\
\midrule
\textsc{2sg}(\flobv{}) & \forme{tu-tɯ-\rouge{ndze}} &\forme{ku-tɯ-ɤmdzɯ}  \\
\textsc{2du}(\flobv{}) & \forme{tu-tɯ-ndza-ndʑi} &\forme{ku-tɯ-ɤmdzɯ-ndʑi}  \\
\textsc{2pl}(\flobv{}) & \forme{tu-tɯ-ndza-nɯ} &\forme{ku-tɯ-ɤmdzɯ-nɯ}  \\
\midrule
\textsc{3sg}(\flobv{}) & \forme{tu-\rouge{ndze}} &\forme{ku-omdzɯ}  \\
\textsc{3du}(\flobv{}) & \forme{tu-ndza-ndʑi} &\forme{ku-omdzɯ-ndʑi}  \\
\textsc{3pl}(\flobv{}) & \forme{tu-ndza-nɯ} &\forme{ku-omdzɯ-nɯ}  \\
\midrule
1\fl{}\textsc{2sg}& \forme{tu-ta-ndza} \\
\textsc{2sg}\fl{}\textsc{1sg}& \forme{tu-kɯ-ndza-a} \\
\midrule
3$'$\fl{}\textsc{3sg} &\forme{tú-wɣ-ndza} \\
\lspbottomrule
\end{tabular}
\end{table}
 
 Verbs selecting the \textsc{eastwards} or \textsc{westwards} orientation (§\ref{sec:solar.dimension}, §\ref{sec:centripetal.centrifugal}, §\ref{sec:lexicalized.orientation}) present syncretism between Imperfective on the one hand, and Egophoric Present or Sensory on the other hand. For instance, the \textsc{3sg}\flobv{} form \forme{ɲɯ-sɯsɤm} of the verb \japhug{sɯso}{think} (which requires the \textsc{westwards} preverbs), can either be Imperfective or Sensory (§\ref{sec:sensory.morphology}, §\ref{sec:ipfv.inchoative}). Similarly, the \textsc{1sg} forms \forme{ku-rɤʑi-a} and \forme{ku-omdzɯ-a} of the verbs \japhug{rɤʑi}{stay} and \japhug{amdzɯ}{sit} (whose lexical orientation is \textsc{eastwards}, see \tabref{tab:ipfv.paradigms}), can be Imperfective or Egophoric Present (§\ref{sec:egophoric.morphology}).
 
 Verbs selecting other orientations do not have such ambiguity; for instance, the verb \japhug{rɤma}{work} (selecting \textsc{upwards} orientation) has different Imperfective, Egophoric Present and Sensory forms: \forme{tu-rɤma}, \forme{ku-rɤma} and \forme{ɲɯ-rɤma}, respectively.

\subsection{Use in periphrastic TAME categories} \label{sec:ipfv.periphrastic.TAME}
Periphrastic TAME categories with a main verb in the Imperfective are used to express habitual or ongoing actions. In these constructions, the main verb expresses person/number and aspect, while the copula encodes tense, modality and evidentiality.

The examples in (\ref{ex:tundze.periphrastic}) (with the verb \japhug{ndza}{eat} in the \textsc{3sg}\flobv{} configuration) illustrate attested possibilities. With a copula in the Factual Non-Past \forme{ŋu} (\ref{ex:tundze.Nu}) or in the Sensory \forme{ɲɯ-ŋu} (\ref{ex:tundze.YWNu}), the periphrastic construction has a non-past meaning with an evidential contrast between non-Sensory and Sensory (there is no Periphrastic Egophoric). When the copula is in the Past Imperfective \forme{pɯ-ŋu} or in the Inferential Imperfective \forme{pjɤ-ŋu}, the interpretation is that of a past habitual `used to $X$' or a Past progressive (§\ref{sec:pst.ifr.ipfv.periphrastic}). Additional periphrastic constructions are also attested with Secondary Modal prefixes on the copula (§\ref{sec:second.modal}).

\begin{exe}
\ex \label{ex:tundze.periphrastic}
\begin{xlist}
\ex \label{ex:tundze.Nu}
\gll tu-ndze ŋu \\
\textsc{ipfv}-eat[III] be:\textsc{fact} \\
\glt `S/he/it eats it/is eating it.' (Periphrastic Imperfective)
\ex \label{ex:tundze.YWNu}
\gll tu-ndze ɲɯ-ŋu \\
\textsc{ipfv}-eat[III] \textsc{sens}-be  \\
\glt `S/he/it eats it/is eating it.' (Periphrastic Sensory)
\ex \label{ex:tundze.pWNu}
\gll tu-ndze pɯ-ŋu \\
\textsc{ipfv}-eat[III] \textsc{pst}.\textsc{ipfv}-be  \\
\glt `S/he/it used to eat it/was eating it.' (Periphrastic Past Imperfective)
\ex \label{ex:tundze.pjANu}
\gll tu-ndze pjɤ-ŋu \\
\textsc{ipfv}-eat[III] \textsc{ifr}.\textsc{ipfv}-be  \\
\glt `S/he/it used to eat it/was eating it.' (Periphrastic Inferential Imperfective)
\ex \label{ex:tundze.apWNu}
\gll tu-ndze a-pɯ-ŋu \\
\textsc{ipfv}-eat[III] \textsc{irr}-\textsc{ipfv}-be  \\
\glt `If s/he eats it...' (Periphrastic Irrealis)
\end{xlist}
\end{exe}

In these constructions, the copula never takes any person/number indexation markers or associated motion, which can only be found on the main verb. For instance, in (\ref{ex:tundzea.pWNu}), the \textsc{1sg} indexation suffix \forme{-a} can only occur on \forme{tu-ndze-a}, and putting it on the copula \forme{ŋu} here is categorically rejected (although the copula \japhug{ŋu}{be} is compatible with indexation affixes, §\ref{sec:semi.transitive})

\begin{exe}
\ex \label{ex:tundzea.pWNu}
\gll  tɤ-mtʰɯm tu-ndze-a pɯ-ŋu ri, \\
\textsc{indef}.\textsc{poss}-meat \textsc{ipfv}-eat[III]-\textsc{1sg} \textsc{pst}.\textsc{ipfv}-be \textsc{lnk} \\
\glt `I was eating the meat.' (150909 qandZGi, 5)
\end{exe}

Negation can be expressed in these constructions by using the suppletive negative copula \japhug{maʁ}{not be} (§\ref{sec:suppletive.negative}, §\ref{sec:periphrastic.negation}, §\ref{sec:affirmative.copula.function}), as in (\ref{ex:tumNAm.pWmaR}).

\begin{exe}
\ex \label{ex:tumNAm.pWmaR}
\gll  ma lonba tɯtɯrca tu-mŋɤm pɯ-maʁ. \\
\textsc{lnk} all together \textsc{ipfv}-hurt \textsc{pst}.\textsc{ipfv}-not.be \\
\glt `(My whole head) was not hurting all at the same time.' (24-pGArtsAG, 77)
\end{exe}

Chains of verbs in the Imperfective can share the same copula, which appears at the end, following the last verb. In (\ref{ex:chain.pjANu}), the Inferential Imperfective copula \forme{pjɤ-ŋu} has scope over no less than ten verbs in the Imperfective (marked in red), sharing the same subjects and expressing a list of actions repeatedly occurring in a particular order every day (the first verb \forme{ku-rtoʁ} `he saw that..' does not belong to this chain, see §\ref{sec:ipfv.perception}). One verb in the Aorist (marked in blue) appears in the middle of the Imperfective chain to set a point of temporal reference `when they reach (the ground)'.

\begin{exe}
\ex \label{ex:chain.pjANu}
\gll ku-rtoʁ tɕendɤre sɲikuku ʑo qro χsɯm \rouge{pjɯ-ɣi-nɯ} tɕe tɯmɯnɤmkʰa zɯ \rouge{pjɯ-nɯ-ɬoʁ-nɯ} tɕe  tɕe ɯ-tʰoʁ \bleu{pɯ-azɣɯt-nɯ} tɕe, ci \rouge{ɲɯ-ʑɣɤ-sɤphɤr-nɯ} tɕe, tɕe qro ɯ-ndʐi nɯ \rouge{pjɯ-qaʁ-nɯ} tɕe, ɯʑoz \rouge{ɲɯ-ta-nɯ} tɕe, ɯ-ŋgɯ tɕʰeme kɯ-mpɕɯ\redp{}mpɕɤr ʑo χsɯm ntsɯ \rouge{ɲɯ-nɯ-ɬoʁ} tɕe \rouge{lu-nɤtsoʁ-nɯ} tɕe, tɕe tɯrmɯkʰa tɕe li nɤkinɯ tɤtsoʁ nɯ \rouge{tu-nɯ-ndo-nɯ} ʑara qro ɯ-ndʐi ɯ-ŋgɯ nɯ \rouge{tu-nɯ-ŋga-nɯ} tɕe tɕe \rouge{ɲɯ-ʑɣɤ-sɤpʰɤr-nɯ} qʰe tɕe ci ɯ-qʰu ci ʑo tɯmɯnɤmkʰa nɯtɕu \rouge{tu-ɕqʰlɤt-nɯ} ntsɯ \textbf{pjɤ-ŋu}. \\
\textsc{ipfv}-look \textsc{lnk} everyday \textsc{emph} pigeon three \textsc{ipfv}:\textsc{down}-come-\textsc{pl} \textsc{lnk} heaven \textsc{loc} \textsc{ipfv}:\textsc{down}-\textsc{auto}-come.out-\textsc{pl} \textsc{lnk} \textsc{lnk}  \textsc{3sg}.\textsc{poss}-ground \textsc{aor}:\textsc{down}-reach-\textsc{pl} \textsc{lnk} a.little \textsc{ipfv}-\textsc{refl}-shake-\textsc{pl} \textsc{lnk} \textsc{lnk} pigeon \textsc{3sg}.\textsc{poss}-skin \textsc{dem} \textsc{ipfv}-peel-\textsc{pl} \textsc{lnk} apart \textsc{ipfv}:\textsc{west}-put \textsc{lnk} \textsc{3sg}.\textsc{poss}-inside girl \textsc{sbj}:\textsc{pcp}-\textsc{emph}\redp{}be.beautiful \textsc{emph} three always \textsc{ipfv}:\textsc{west}-\textsc{auto}-come.out \textsc{lnk} \textsc{ipfv}-collect.Potentilla.anserina \textsc{lnk} \textsc{lnk} evening \textsc{loc} again \textsc{filler} Potentilla.anserina \textsc{dem} \textsc{ipfv}:\textsc{up}-\textsc{auto}-take-\textsc{pl} \textsc{3pl} pigeon \textsc{3sg}.\textsc{poss}-skin \textsc{3sg}.\textsc{poss}-in \textsc{dem} \textsc{ipfv}-\textsc{auto}-wear-\textsc{pl} \textsc{lnk} \textsc{lnk}  \textsc{ipfv}-\textsc{refl}-shake-\textsc{pl} \textsc{lnk} \textsc{lnk} one \textsc{3sg}.\textsc{poss}-after one \textsc{emph} heaven \textsc{dem}:\textsc{loc} \textsc{ipfv}:\textsc{up}-disappear-\textsc{pl} always \textbf{\textsc{ifr}.\textsc{ipfv}-be}  \\
\glt `He saw that everyday, three pigeons would come down from heavens, and as they reached the ground, they would shake themselves, shed the pigeon skins and put it a aside, and three beautiful girls would come out from (the skins), they would collect \textit{Potentilla anserina}, and in the evening, they would take the \textit{Potentilla}, wrap themselves in their pigeon skin, shake themselves and disappear in heavens one after the other.' (07-deluge, 31-38)
\end{exe}

Aside from the habitual and progressive meaning in (\ref{ex:tundze.Nu}), the  combination of a main verb in Imperfective with a copula in the Factual Non-Past can also express imminent future, as in (\ref{ex:Cturua.Nu}) and (\ref{ex:pjWnWGia.Nu}). In (\ref{ex:Cturua.Nu}), it also appears in the apodosis of a conditional construction to express the result if the condition in the protasis is verified.

\begin{exe}
\ex \label{ex:Cturua.Nu}
\gll  ɕ-tu-ru-a ŋu tɕe, tɯrme ɲɯ\redp{}ɲɯ-ŋu nɤ, a-ku, nɤki, tu-sɤŋoʁŋoʁ-a ŋu tɕe tɤ-ɣi, \\
\textsc{tral}-\textsc{ipfv}:\textsc{up}-look-\textsc{1sg} be:\textsc{fact} \textsc{lnk} human \textsc{cond}\redp{}\textsc{sens}-be \textsc{add} \textsc{1sg}.\textsc{poss}-head \textsc{filler} \textsc{ipfv}-nod-\textsc{1sg} be:\textsc{fact} \textsc{lnk} \textsc{imp}:\textsc{up}-come \\
\glt `I am going to have a look, and if it is a human, I will nod and you can come (to eat him).' (2012 khu, 48-49)
\end{exe}

\begin{exe}
\ex \label{ex:pjWnWGia.Nu}
\gll wo a-mu ma-pɯ-tɯ-ʑɣɤ-sat tɕe aʑo pjɯ-nɯ-ɣi-a ŋu \\
\textsc{interj} \textsc{1sg}.\textsc{poss}-mother \textsc{neg}-\textsc{imp}-2-\textsc{refl}-kill \textsc{lnk} \textsc{1sg} \textsc{ipfv}:\textsc{down}-\textsc{vert}-come-\textsc{1sg} be:\textsc{fact} \\
\glt `Mother, don't commit suicide, I am coming back.' (2003 kAndzwsqhaj2, 26)
\end{exe}

In (\ref{ex:GWjurea.Nu}), the Imperfective verb \forme{ɣɯ-ju-re-a} refers to an event expected to occur several years in the future. However, this is still analyzable as an imminent future, as this action is to take place immediately after the point of future temporal reference expressed by the verb \forme{jɤ-nɯɣe-a} in the Aorist (§\ref{sec:aor.temporal}).

\begin{exe}
\ex \label{ex:GWjurea.Nu}
\gll  kɯki tɯ-tɕʰɤɣdɯ ki nɤʑo ɯ-pɯ tɤ-pe tɕe,  aʑo jɤ-nɯ-ɣe-a tɕe tɕe ɣɯ-ju-re-a ŋu\\
\textsc{dem}.\textsc{prox} one-jar \textsc{dem}.\textsc{prox} \textsc{2sg} \textsc{3sg}.\textsc{poss}-safekeeping \textsc{imp}-do[II] \textsc{lnk} \textsc{1sg} \textsc{aor}-\textsc{vert}-come[II]-\textsc{1sg} \textsc{lnk} \textsc{lnk} \textsc{cisl}-\textsc{ipfv}-fetch[III]-\textsc{1sg} be:\textsc{fact}\\
\glt `Keep this jarful (of olives for me while I am gone), when I come back I will come and take it back.' (140516 yiguan ganlan-zh, 24-25)
\end{exe}

The existence of Periphrastic TAME categories, in addition to the primary and secondary categories, makes the Japhug TAME system extremely complex. At the present stage of my knowledge of the language, the semantic differences between some categories still eludes me.

For instance, the (Primary) Sensory and the Periphrastic Sensory can both express habitual or generic actions, as in (\ref{ex:YWndze.tundze}) where both forms appear redundantly (with tail-head linkage, §\ref{sec:tail.head.linkeage}).

\begin{exe}
\ex \label{ex:YWndze.tundze}
\gll tɕe ma nɯnɯ dɯdɯt nɯ kɯ tɕɣom kɯnɤ ɲɯ-ndze. tɕɣom kɯnɤ ɲɯ-ndze tɕe tu-ndze ɲɯ-ŋu, pjɯ-kre ɲɯ-ŋu. \\
\textsc{lnk} \textsc{lnk} \textsc{dem} dove \textsc{dem} \textsc{erg} xanthoxylum also \textsc{sens}-eat[III]  xanthoxylum also \textsc{sens}-eat[III] \textsc{lnk} \textsc{ipfv}-eat \textsc{sens}-be \textsc{ipfv}-cause.to.fall[III] \textsc{sens}-be \\
\glt `The dove also eats xanthoxylum. It also eats xanthoxylum, and makes it fall (from the tree).' (22-CAGpGa, 32-33)
\end{exe}

There is overlap between the use of postverbal copulas in the periphrastic TAME constructions and their function as focus marker (§\ref{sec:periphrastic.negation}, §\ref{sec:focalization.final.copula}). In (\ref{ex:tutia.maR.Cti}), the negative copula \forme{maʁ} and the emphatic affirmative \forme{ɕti} are used both to build the Periphrastic Imperfective, and to express contrastive focus on the dative recipient.


\begin{exe}
\ex \label{ex:tutia.maR.Cti}
\gll nɤʑo nɤ-pʰe tu-ti-a maʁ, pɤnmawombɤr ɯ-pʰe tu-ti-a ɕti \\
\textsc{2sg} \textsc{2sg}.\textsc{poss}-\textsc{dat} \textsc{ipfv}-say-\textsc{1sg} not.be:\textsc{fact}  \textsc{anthr} \textsc{3sg}.\textsc{poss}-\textsc{dat}  \textsc{ipfv}-say-\textsc{1sg} be.\textsc{aff}:\textsc{fact} \\
\glt `I am not saying it to you, I am saying it to Padma 'Od'bar.' (Norbzang 2005, 190)
\end{exe}

\subsection{Use in temporal clauses} \label{sec:ipfv.temporal}
Chains of verbs in the Imperfective without subordinating relation, but possibly sharing a tense-marking copula, can indicate a succession of events, as in (\ref{ex:chain.pjANu}) above. The Imperfective also occurs in subordinate temporal clauses expressing precedence or simultaneous action.

Temporal clauses of temporal precedence headed by the postposition \japhug{ɕɯŋgɯ}{before} (§\ref{sec:temporal.postpositions}, §\ref{sec:precedence.CWNgW}, \citealt[286--287]{jacques14linking}) require the Imperfective, to the exclusion of all other finite and non-finite verb forms. In (\ref{ex:lufsoR.CWNgW}) and (\ref{ex:kulAt.CWNgW}) for instance, only \forme{lu-fsoʁ} and \forme{ku-lɤt} can occur with \forme{ɕɯŋgɯ}, and neither infinitive (§\ref{sec:inf}) or Aorist forms (§\ref{sec:aor}) are possible, regardless of the TAME form of the verb of the main clause (Inferential in \ref{ex:lufsoR.CWNgW}, Past Imperfective in \ref{ex:kulAt.CWNgW}).

\begin{exe}
\ex \label{ex:lufsoR.CWNgW}
\gll tɕe nɯ ɯ-fso lu-fsoʁ ɕɯŋgɯ qʰe li nɯra ɕ-to-stu. \\
\textsc{lnk} \textsc{dem} \textsc{3sg}.\textsc{poss}-tomorrow \textsc{ipfv}-be.bright before \textsc{lnk} again \textsc{dem}:\textsc{pl} \textsc{tral}-\textsc{ifr}-do.like \\
\glt `The next day, before the day broke, he went (there) and did like (she had said).' (28-smAnmi, 356)
\end{exe}

\begin{exe}
\ex \label{ex:kulAt.CWNgW}
\gll tɯ-mɯ ku-lɤt ɕɯŋgɯ nɯ ɯ-tɯ-sɤ-ɕke pɯ-saχaʁ ʑo \\
\textsc{indef}.\textsc{poss}-sky \textsc{ipfv}-release before \textsc{dem} \textsc{3sg}.\textsc{poss}-\textsc{nmlz}:\textsc{deg}-\textsc{prop}-burn \textsc{pst}.\textsc{ipfv}-be.extremely \textsc{emph} \\
\glt `Before it rained, it was very hot.' (conversation, 17-09-2018)
\end{exe}

In temporal clauses with \japhug{ɯ-kʰɯkʰa}{while}, Imperfective express an action occurring simultaneously with that of the main clause, as in (\ref{ex:pjWtasWxCAt.WkhWkha}) and (\ref{ex:YWjtshi.WkhWkha}). However, unlike \forme{ɕɯŋgɯ}, \forme{ɯ-kʰɯkʰa} does not select the Imperfective and other finite TAME forms are possible (§\ref{sec:simultaneity}).

\begin{exe}
\ex \label{ex:pjWtasWxCAt.WkhWkha}
\gll aʑo pjɯ-ta-sɯxɕɤt ɯ-kʰɯkʰa lu-taʁ-a ŋu \\
\textsc{1sg} \textsc{ipfv}-1\fl{}-teach \textsc{3sg}.\textsc{poss}-while \textsc{ipfv}-weave be:\textsc{fact} \\
\glt `I am teach you (how to weave) and weaving at the same time.' (elicited)
\end{exe}

\begin{exe}
\ex \label{ex:YWjtshi.WkhWkha}
\gll qʰe ɯ-pɯ tɯ-nɯ ɲɯ-jtsʰi ɯ-kʰɯkʰa,  ɯ-ku kɯra tu-ste tɕe zrɯɣ ra pjɯ-re ɲɯ-ŋu.  tɕe zrɯɣ nɯra tu-ndze ɲɯ-ŋu. \\
\textsc{lnk} \textsc{3sg}.\textsc{poss}-young \textsc{indef}.\textsc{poss}-breast \textsc{ipfv}-give.to.drink \textsc{3sg}.\textsc{poss}-while \textsc{3sg}.\textsc{poss}-head \textsc{dem}.\textsc{prox}:\textsc{pl} \textsc{ipfv}-do.like[III] \textsc{lnk} louse \textsc{pl} \textsc{ipfv}-pick.off[III] \textsc{sens}-be \textsc{lnk} louse \textsc{dem}:\textsc{pl} \textsc{ipfv}-eat[III] \textsc{sens}-be \\
\glt `While (the monkey mother) breastfeeds her young, she does like this on its head at the same time and picks lice off. Then she eats the lice.' (19-GzW, 36-38)
\end{exe}

\subsection{Use in complement clauses} \label{sec:ipfv.complement}
The Imperfective is common in complement clauses, in particular with modal verbs. In such clauses, no auxiliary copula is required.

Imperfective occurs in S-complement clauses with modal verbs such as \japhug{ra}{be needed}, `be necessary' (\ref{ex:pjWwGsat.YWra}, \ref{ex:pjWtWGi.mAra}), \japhug{jɤɣ}{be allowed} (\ref{ex:tukWqura.WjAG}), \japhug{ntsʰi}{be better} and \japhug{kʰɯ}{be possible}. 

\begin{exe}
\ex \label{ex:pjWwGsat.YWra}
\gll  atu pɣɤtɕɯ nɯ pjɯ́-wɣ-sat ɲɯ-ra \\
up.there bird \textsc{dem} \textsc{ipfv}-\textsc{inv}-kill \textsc{sens}-be.needed \\
\glt `One has to kill the bird upstairs.' (2003kongzong, 364)
\end{exe}

\begin{exe}
\ex \label{ex:pjWtWGi.mAra}
\gll pjɯ-tɯ-ɣi mɤ-ra \\
\textsc{ipfv}:\textsc{down}-2-come \textsc{neg}-be.needed:\textsc{fact} \\
\glt `You don't have to come down (with us).' (heard in context)
\end{exe}

\begin{exe}
\ex \label{ex:tukWqura.WjAG}
\gll tu-kɯ-qur-a ɯ́-jɤɣ \\
\textsc{ipfv}-2\fl{}1-help-\textsc{1sg} \textsc{qu}-be.possible:\textsc{fact} \\
\glt `Could you help me?' (150901 dongguo xiansheng he lang-zh, 34)
\end{exe}

The aspectual auxiliary \japhug{ŋgrɤl}{be usually the case} (§\ref{sec:NgrAl}) most often selects a complement in the Imperfective, expressing recurring actions or situations, as in (\ref{ex:tuBze.pjANgrAl}).

\begin{exe}
\ex \label{ex:tuBze.pjANgrAl}
\gll  qala kɯ nɯra kɯ-fse βlaβlu tu-βze pjɤ-ŋgrɤl. \\
hare \textsc{erg} \textsc{dem}:\textsc{pl} \textsc{sbj}:\textsc{pcp}-be.like trick \textsc{ipfv}-do[ILI] \textsc{ifr}.\textsc{ipfv}-be.usually.the.case \\
\glt `The hare used to do tricks like that.' (31-qala, 82)
\end{exe}

The Imperfective is also found in object or semi-object complement clauses with modal verbs such as \japhug{spa}{be able to} (\ref{ex:YWkrAm.mWjspe}) or \japhug{cʰa}{can} (example \ref{ex:YWnWqambWmbjom.tocha}, §\ref{sec:ifr.inchoative}).

\begin{exe}
\ex \label{ex:YWkrAm.mWjspe}
\gll ɲɯ-krɤm mɯ́j-spe qhe, tɕendɤre nɯnɯ ɯ-pɯ tɯ-rdoʁ nɯ kɯ [...] tu-nɯ-ndɤm qhe, ɯʑo stɯsti ʑo tu-nɯ-ndze ɲɯ-ŋu \\
\textsc{ipfv}-share[III] \textsc{neg}:\textsc{sens}-be.able[III] \textsc{lnk} \textsc{lnk} \textsc{dem} \textsc{3sg}.\textsc{poss}-young one-piece \textsc{dem} \textsc{erg} { } \textsc{ipfv}-\textsc{auto}-take[III] \textsc{lnk} \textsc{3sg} alone \textsc{emph} \textsc{ipfv}-\textsc{auto}-eat[III] \textsc{sens}-be \\
\glt `(The mother cat) does not know how to share (the food she has brought for her kitten equally), one of the kitten takes (the whole) and eats it alone (without giving anything to its mother or the other kitten).' (21-lWLU, 81)
\end{exe}

Imperfective complement clauses are found with main verbs in all primary TAME categories, including perfective ones like Inferential (example \ref{ex:YWnWqambWmbjom.tocha}, §\ref{sec:ifr.inchoative}) or the Aorist.


\subsection{Hortative} \label{sec:ipfv.hortative}
The Imperfective used without any copula or auxiliary can have a hortative meaning similar to the one it has when combined with a modal auxiliary such as \japhug{ra}{be needed} or \japhug{ntsʰi}{be better} (§\ref{sec:ipfv.complement}). 

The hortative function occurs in first person subject forms, with either transitive or intransitive verbs (\ref{ex:kuZGACthWza}).

\begin{exe}
\ex \label{ex:kuZGACthWza}
\gll ku-ʑɣɤɕtʰɯz-a ma zdɯxpa \\
\textsc{ipfv}-reveal.one's.true.nature-\textsc{1sg} \textsc{lnk} poor.of \\
\glt `Let me show him who I am, poor of him.' (2003kandzwsqhaj, 146)
\end{exe}

It is also found instead of the Imperative in the 2\fl{}1 configurations (\ref{ex:kukWnAjoa}, \ref{ex:jukWtsWma}, \ref{ex:YWkWmbia.ma}, \ref{ex:YWkWsWBzJWra}), since the Imperative only allows 2\fl{}3 forms (§\ref{sec:imp.morphology}).

\begin{exe}
	\ex \label{ex:kukWnAjoa}
	\gll ku-kɯ-nɤjo-a je \\
	\textsc{ipfv}-2\fl{}1-wait-\textsc{1sg} \textsc{sfp}  \\
	\glt `Wait for me!' (heard in context)
\end{exe}

\begin{exe}
\ex \label{ex:jukWtsWma}
\gll ju-kɯ-tsɯm-a wo, a-wi \\
\textsc{ipfv}-2\fl{}1-take.away-\textsc{1sg} \textsc{sfp} \textsc{1sg}.\textsc{poss}-grandmother \\
\glt `Take me away with you, grandmother!' (140519 mai huochai de xiao nvhai-zh, 160)
\end{exe}

\begin{exe}
\ex \label{ex:YWkWmbia.ma}
\gll  nɯ-me ɲɯ-kɯ-mbi-a-nɯ ma aʑo-sti kɤ-nɯ-ɕe mɤ-cʰa-a \\
\textsc{3pl}.\textsc{poss}-daughter \textsc{ipfv}-2\fl{}1-give-\textsc{1sg}-\textsc{pl} \textsc{lnk} \textsc{1sg}-alone \textsc{inf}-\textsc{vert}-go \textsc{neg}-can:\textsc{fact}-\textsc{1sg} \\
\glt `Give me you daughter, I cannot go back there alone.' (02-deluge2012, 108)
\end{exe}

\begin{exe}
\ex \label{ex:YWkWsWBzJWra}
\gll ɯ-ɲɯ-nɯkɯmaʁ-a nɤ ɲɯ-kɯ-sɯ-βzɟɯr-a \\
\textsc{qu}-\textsc{ipfv}-make.a.mistake-\textsc{1sg} \textsc{add} \textsc{ipfv}-2\fl{}1-\textsc{caus}-correct-\textsc{1sg} \\
\glt `If I make a mistake (when speaking), correct me.' (elicited)
\end{exe}

The hortative meaning of the Imperfective occurs in three main contexts: (i) when the clause containing the verb in the Imperfective followed by a causal clause, with the linker \forme{ma} in between, as in (\ref{ex:kuZGACthWza}) and (\ref{ex:YWkWmbia.ma}); (ii) when the verb is followed by the sentence final particle \forme{wo} (§\ref{sec:fsp.imp}), as in (\ref{ex:jukWtsWma}); (iii) in the apodosis of conditional constructions (\ref{ex:YWkWsWBzJWra}).

\subsection{Inchoative} \label{sec:ipfv.inchoative}
With stative verbs, like the Aorist (§\ref{sec:aor}) and the Inferential (§\ref{sec:ifr}) the Imperfective always expresses ongoing change, whether it appears in main clauses in a Periphrastic tense (\ref{ex:chWwGrum.YWNu}), or in a complement clause (\ref{ex:chWmACWndZi}).

\begin{exe}
\ex \label{ex:chWwGrum.YWNu}
\gll tʰɯ-tɯt ri tɕe cʰɯ-wɣrum ɲɯ-ŋu. \\
\textsc{aor}-be.ripe \textsc{lnk} \textsc{lnk} \textsc{ipfv}-be.white \textsc{sens}-be \\
\glt `When it ripens, it becomes white.' (16-CWrNgo, 165)
\end{exe}

\begin{exe}
\ex \label{ex:chWmACWndZi}
\gll  nɯ kɯnɤ cʰɯ-mɤɕi-ndʑi mɯ-pjɤ-cʰa-ndʑi \\
\textsc{dem} also \textsc{ipfv}-be.rich-\textsc{du} \textsc{neg}-\textsc{ifr}.\textsc{ipfv}-can-\textsc{du} \\
\glt `Despite (their hard work), they could not become rich. (divination 2003, 7)
\end{exe}

This inchoative meaning often appears with initial reduplication of gradual increase (§\ref{sec:redp.gradual.increase}), as in (\ref{ex:tWtumpCAr.Zo.Nu}).

\begin{exe}
\ex \label{ex:tWtumpCAr.Zo.Nu}
\gll ʑɯrɯʑɤri tɕe tɕendɤre ɯ-skɤt nɯ tɯ\redp{}tu-mpɕɤr ʑo ŋu  \\
progressively \textsc{lnk} \textsc{lnk} \textsc{3sg}.\textsc{poss}-voice \textsc{dem} \textsc{incr}\redp{}\textsc{ipfv}-be.beautiful \textsc{emph} be:\textsc{fact} \\
\glt `(As it grows bigger, the rooster's) voice progressively becomes more and more beautiful.' (22-kumpGa, 74)
\end{exe}

Verbs selecting the \textsc{westwards} or \textsc{eastwards} preverbs as intrinsic orientations have syncretism between Imperfective on the one hand, and Sensory or Egophoric Present on the other hand (§\ref{sec:ipfv.morphology}). For instance, the form \forme{ɲɯ-ɲaʁ} is (inchoative) Imperfective in (\ref{ex:YWYaR.Zo.Nu}), and (stative) Sensory in (\ref{ex:kW.YWYaR}) (illustrating the comparative use of the Sensory, §\ref{sec:sensory.other}). The only formal difference between them is the presence of the auxiliary \forme{ŋu} in (\ref{ex:YWYaR.Zo.Nu}).

\begin{exe}
\ex \label{ex:YWYaR.Zo.Nu}
\gll  pɯ-rom tɕe tɕe ɲɯ-ɲaʁ ʑo ŋu \\
\textsc{aor}-be.dry \textsc{lnk} \textsc{lnk} \textsc{ipfv}-be.black \textsc{emph} be:\textsc{fact} \\
\glt `When (the puffball mushroom) dries, it becomes black.' (22-BlamajmAG, 75)
\end{exe}

\begin{exe}
\ex \label{ex:kW.YWYaR}
\gll  tɕe ɯ-jme nɯnɯ kɯ ɲɯ-ɲaʁ \\
\textsc{lnk} \textsc{3sg}.\textsc{poss}-tail \textsc{dem} \textsc{erg} \textsc{sens}-be.black \\
\glt `It tail is more black.' (23-qapGAmtWmtW, 60)
\end{exe}


\subsection{Perception verbs} \label{sec:ipfv.perception}
The verbs \japhug{rtoʁ}{look} and \japhug{sɤŋo}{listen}, which normally express volitional perception, occur in the Imperfective in an unusual construction which has three main characteristics. 

First, the object or semi-object (§\ref{sec:semi.tr.labile}) of the verb is non-overt, and cataphorically refers to the immediately following clause(s), which describe(s) the perceived event, as in (\ref{ex:kurtoRnW.ipfv}). These clauses are in coordinating relationship with the clause of the perception verb, and are not subordinate complement clauses (§\ref{sec:coordination.comp.str}).

\begin{exe}
\ex \label{ex:kurtoRnW.ipfv}
\gll ɯ-zda ra kɯ ku-rtoʁ-nɯ tɕe, [nɯnɯ rdɤstaʁ ɲɤ-k-ɤβzu rcanɯ], wuma ʑo pjɤ-ɲɟɤt-nɯ. \\
\textsc{3sg}.\textsc{poss}-companion \textsc{pl} \textsc{erg} \textsc{ipfv}-look-\textsc{pl} \textsc{lnk} \textsc{dem} stone \textsc{ifr}-\textsc{peg}-become-\textsc{peg} \textsc{unexp}:\textsc{foc} really \textsc{emph} \textsc{ifr}-regret-\textsc{pl} \\
\glt `His companion saw that he had been turned to stone, and regretted very much (not having trusted him).' (150902 hailibu-zh, 154-155)
\end{exe}

Second, despite having an Imperfective form, the verbs can express a semelfactive perception (note in particular the presence of the adverb \japhug{ʁlɤwɯr}{suddenly} in \ref{ex:RlAwWr.kurtoR} below), rather than an ongoing or recurrent perception when the following clause is in the Aorist or in the Inferential (as in \ref{ex:kurtoRnW.ipfv}, \ref{ex:RlAwWr.kurtoR}, \ref{ex:YWsANo.ipfv} and \ref{ex:kurtoR.ipfv} below)  

\begin{exe}
\ex \label{ex:RlAwWr.kurtoR}
\gll spjaŋkɯ kɯ ʁlɤwɯr ʑo ku-rtoʁ tɕe, kʰɯna ɣɯ ɯ-mke nɯtɕu ʁmazgrɯβ ci pjɤ-mto ɲɯ-ŋu.  \\
wolf \textsc{erg} suddenly \textsc{emph} \textsc{ipfv}-look \textsc{lnk} \textsc{dog} \textsc{gen} \textsc{3sg}.\textsc{poss}-neck \textsc{dem}:\textsc{loc} scar \textsc{indef} \textsc{ifr}-see \textsc{sens}-be \\
\glt `The wolf suddenly saw (that there was) a scar on the dog's neck.' (140426 jiagou he lang-zh, 41-42)
\end{exe}

Third, the perception is non-volitional: the verbs in this construction can be translated as `notice' (of something unexpected), as shown by the fact that the verb of non-volitional perception \japhug{mto}{see} occurs in the following clause in (\ref{ex:RlAwWr.kurtoR}) to redundantly express the same perception event as \forme{ku-rtoʁ}. 

 The clauses referring to the perceived event (shown in square brackets below) in the Inferential either express actions that had taken place before the perception event, and whose results only are perceptible (§\ref{ex:kurtoRnW.ipfv}, §\ref{ex:kurtoR.ipfv}), or actions that are immediately perceived as they occur (§\ref{ex:YWsANo.ipfv}).


\begin{exe}
\ex \label{ex:YWsANo.ipfv}
\gll a-wa kɯ-rɯjɤɣɤt ɲɤ-ɕe ri, ɲɯ-sɤŋo tɕe, [ɯ-taʁ nɯtɕu tɯrme ci ɲɤ-ɣi]  \\
\textsc{1sg}.\textsc{poss}-father \textsc{sbj}:\textsc{pcp}-go.to.toilets \textsc{ifr}:\textsc{west}-go \textsc{lnk} \textsc{ipfv}-listen \textsc{lnk} \textsc{3sg}.\textsc{poss}-up \textsc{dem}:\textsc{loc} man \textsc{indef} \textsc{ifr}:\textsc{west}-come \\
\glt `My father had gone to toilets, and heard (there) that someone came upstairs.' (08-kWqhi, 11-12)
\end{exe}

The cataphoric object of the perception verb can comprise more than one clause, as shown by (\ref{ex:kurtoR.ipfv}).

\begin{exe}
\ex \label{ex:kurtoR.ipfv}
\gll  iɕqʰa kɯ-nɯɕɤlɤmbɯmbjom nɯ cʰɤ-sta, [...] tɕendɤre ku-rtoʁ tɕe, [ɯ-tɯ-ci ri pjɤ-lwoʁ tɕe, tɕendɤre iɕqʰa nɯ, tɤɕime nɯ ri kɯ-ɤrqʰɯ\redp{}rqʰi ʑo jo-nɯ-ɕe ɕti] tɕe, \\
the.aforementioned \textsc{sbj}:\textsc{pcp}-racing \textsc{dem} \textsc{ifr}-wake.up { } \textsc{lnk} \textsc{ipfv}-look \textsc{lnk} \textsc{3sg}.\textsc{poss}-\textsc{indef}.\textsc{poss}-water also \textsc{ifr}-spill  \textsc{lnk} \textsc{lnk} \textsc{filler} \textsc{dem} lady \textsc{dem} also \textsc{sbj}:\textsc{pcp}-\textsc{emph\redp}{}be.far \textsc{emph} \textsc{ifr}-\textsc{auto}-go be.\textsc{aff}:\textsc{fact} \textsc{lnk} \\
\glt `The racer woke up, and saw that his water had been spilled, and that the princess had already gone far away.' (140505 liuhaohan zoubian tianxia-zh, 130)
\end{exe}

In this construction, the Imperfective forms \forme{ku-rtoʁ} and \forme{ɲɯ-sɤŋo} can also refer to generic or recurrent events if the following clause is in the Imperfective too, as in example (\ref{ex:chain.pjANu}) above (§\ref{sec:ipfv.periphrastic.TAME}).

The Imperfective of perception should be distinguished from the regular uses of the perception verbs \japhug{rtoʁ}{look} and \japhug{sɤŋo}{listen} in the Imperfective. In (\ref{ex:kuwGrtoR.tCe}) and (\ref{ex:kurtoRa.YWnaXCtWGndZi}) for instance, the direct objects of the verb \forme{rtoʁ} do not cataphorically refer to the following clause, but rather to concrete entities (the leopard's head in \ref{ex:kuwGrtoR.tCe}, two birds in \ref{ex:kurtoRa.YWnaXCtWGndZi}), and there is no non-volitional perception interpretation.


\begin{exe}
\ex \label{ex:kuwGrtoR.tCe}
\gll  tɕe kɯrtsɤɣ nɯnɯ ɯ-ku nɯnɯ kú-wɣ-rtoʁ tɕe, lɯlu tsa ɯ-tsʰɯɣa fse, \\
\textsc{lnk} leopard \textsc{dem} \textsc{3sg}.\textsc{poss}-head \textsc{dem} \textsc{ipfv}-\textsc{inv}-look \textsc{lnk} cat a.little \textsc{3sg}.\textsc{poss}-shape be.like:\textsc{fact} \\
\glt `Looking at the leopard's head, it seems a bit like that of the cat.' (27-qartshAz, 167)
\end{exe}

\begin{exe}
\ex \label{ex:kurtoRa.YWnaXCtWGndZi}
\gll  aʑo ndɤre ku-rtoʁ-a ɲɯ-naχtɕɯɣ-ndʑi ɕti. \\
\textsc{1sg} \textsc{lnk} \textsc{ipfv}-look-\textsc{1sg} \textsc{sens}-be.the.same-\textsc{du} be.\textsc{aff}:\textsc{fact} \\
\glt `(When) I look at them (of two species of birds), they look the same.' (24-ZmbrWpGa, 5)
\end{exe}

The verb \japhug{ru}{look at} occurs in construction similar to that described above for \japhug{rtoʁ}{look} and \japhug{sɤŋo}{listen} (§\ref{sec:coordination.comp.str}) as in (\ref{ex:kArqhi.jukWru}), but it keeps its volitional meaning and is not exclusively found in the Imperfective.

\begin{exe}
\ex \label{ex:kArqhi.jukWru}
\gll ŋɤqa kɤ-ti ci tu tɕe, kɯ-ɤrqʰi ju-kɯ-ru tɕe, [salaboŋboŋ tsa fse].  \\
mushroom.sp \textsc{obj}:\textsc{pcp}-say \textsc{indef} exist:\textsc{fact} \textsc{lnk} \textsc{sbj}:\textsc{pcp}-be.far \textsc{ipfv}-\textsc{genr}:S/O-look.at \textsc{lnk} puffball a.little be.like:\textsc{fact} \\
\glt `There is (a mushroom) caller \forme{ŋɤqa} (cow's foot), when one looks at it from far away, it is a bit like a puffball.' (22-BlamajmAG, 84)
\end{exe}

\section{Non-past categories} \label{sec:TAME.npst}
This section discusses Factual Non-Past (§\ref{sec:factual}), Sensory (§\ref{sec:sensory}) and Egophoric Present (§\ref{sec:egophoric}), three TAME categories which have two commonalities. First, they almost always occur in sentences referring to Non-Past events (one exception is discussed in §\ref{sec:sensory.functions}). Second, they do not have an inchoative meaning when used with stative verbs, unlike the Imperfective (§\ref{sec:ipfv.inchoative}). Like the Imperfective and the Modal categories, they select stem III in transitive direct configurations with singular subject and third person object (§\ref{sec:stem3.distribution}).

Minimal pairs between these three categories can be found in specific contexts \citep{jacques19egophoric}, and the tripartite evidential contrast between them is discussed in §\ref{sec:egophoric.tripartite}.

\subsection{Factual Non-Past} \label{sec:factual}
\subsubsection{Morphology and glossing} \label{sec:fact.morphology}
The Factual is the only finite TAM category in Japhug without an orientation preverb and any other prefix in slot -3.\footnote{The term Factual', taken from Oisel's (\citeyear{oisel13aux}) study of modern Lhasa Tibetan, corresponds to the category referred to as `assertive' in older publications on Tibetan languages (Lhasa Tibetan \forme{yod.pa.red}).} In the case of verbs without stem III alternation (i.e. transitive verbs with a non-alternating rhyme and intransitive verbs), it is realized as the bare stem. It selects the negative prefix \forme{mɤ-} (§\ref{sec:neg.allomorphs}). 

Complete paradigms of transitive and intransitive verbs in the Factual Non-Past are presented in §\ref{sec:polypersonal}, and need not be repeated here.

In spite of the absence of overt marking for most verbs, the gloss \textsc{fact} is nevertheless always specified on verbs in the Factual in this grammar, whether or not stem alternation occurs. This gloss is marked as a suffix (verb:\textsc{fact}), rather than as a prefix, to avoid confusion with derivational prefixes, since suffixes (§\ref{sec:suffixes} are fewer than prefixes (§\ref{sec:prefixal.chain}), and also because stem III, which occurs in Factual singular subject forms, is suffixal in origin (§\ref{sec:stem3.form}).

The formation of the Factual is regular. However, the copula \japhug{ŋu}{be} lacks an Egophoric Present form ($\dagger$\forme{ku-ŋu-a} is not accepted, see §\ref{sec:egophoric}), and it thus appears that the preverbless forms of this verb are syncretic, analyzable either as Factual or Egophoric Present (however, no attempt will be made at distinguishing those two categories in the glosses).

A few verbs, such as \japhug{kɤtɯpa}{tell} and \japhug{mɤ-xsi}{it is not known}, cannot take orientation preverbs and are only attested in the Factual (§\ref{sec:irregular.transitive}).


Due to the absence of orientation preverbs, vowel contraction with prefixes located before slot -3 (§\ref{sec:outer.prefixal.chain}) only occur in first or third person forms of the Factual (§\ref{sec:contraction}). Otherwise, contracting verbs surface with initial \forme{a-} in Factual Non-Past non-negative form.
 
\subsubsection{Main clauses} \label{sec:fact.main.clauses}
The Factual has two main functions when used in an independent clause without an auxiliary verb. 

First,  in the case of stative verbs, whether adjectival stative verbs or existential verbs/copulas, the Factual is used to describe facts considered to belong to everybody's common knowledge. Example (\ref{ex:kumpGAtCW}) illustrates five examples of the use of the Factual in this way, including copulas and adjectives. The Imperfective cannot occur in this function with stative verbs, since it has an inchoative meaning (§\ref{sec:ipfv.inchoative}).


\begin{exe}
\ex \label{ex:kumpGAtCW}
\gll tɕe kumpɣɤtɕɯ nɯnɯ pɣɤtɕɯ nɯ-rca, kɯ-xtɕi ci zdoʁzdoʁ \textbf{ŋu} tɕe, ɯʑo \textbf{xtɕi} ri wuma ʑo \textbf{ɕqraʁ} tɕe ɯ-mɲaʁ ɯ-rkɯ nɯnɯ ra kɯ-ɲaʁ kɯ tú-wɣ-fskɤr, nɯ ɯ-taʁ ri, hanɯni, kɯ-xtɕɯ\redp{}xtɕi kɯ-ɣɯrni kɯ-fse \textbf{tu}, ɯ-xtɤpa nɯ ra, ɯ-rqopa pjɯ-ʑe tɕe, nɯ ra, ɯ-jme mɯ-tʰɯ-nɯ-ɬoʁ mɤɕtʂa nɯ \textbf{wɣrum}  \\
\textsc{lnk} sparrow \textsc{dem} bird \textsc{3pl}-among \textsc{sbj}:\textsc{pcp}-be.small \textsc{indef} \textsc{ideo}:\textsc{stat}:small.and.cute \textbf{be}:\textsc{fact} \textsc{lnk}  \textsc{3sg} \textbf{be.small}:\textsc{fact} but really \textsc{emph} \textbf{be.smart}:\textsc{fact}  \textsc{lnk}  \textsc{3sg}:\textsc{poss}-eye \textsc{3sg}:\textsc{poss}-border \textsc{dem}  \textsc{pl}  \textsc{sbj}:\textsc{pcp}-be.black \textsc{erg} \textsc{ipfv}-\textsc{inv}-surround \textsc{dem} \textsc{3sg}-on \textsc{loc} a.little \textsc{sbj}:\textsc{pcp}-\textsc{emph}\redp{}be.small \textsc{sbj}:\textsc{pcp}-be.red \textsc{sbj}:\textsc{pcp}-be.like \textbf{exist}:\textsc{fact} \textsc{3sg}:\textsc{poss}-belly \textsc{dem}  \textsc{pl}  \textsc{3sg}:\textsc{poss}-throat \textsc{ipfv}-begin[III] \textsc{lnk}   \textsc{dem}  \textsc{pl}  \textsc{3sg}:\textsc{poss}-tail \textsc{neg}-\textsc{aor}-\textsc{auto}-come.out until \textsc{dem} \textbf{be.white}:\textsc{fact} \\
\glt `Among the birds, the sparrow is tiny and cute. Although it is small it is very intelligent. Its eyes are surrounded by black (feathers), and above that there are some red (dots). Its belly is white from the throat until the tail.' (22 kumpGatCW, 2-7)
\end{exe}

With dynamic verbs however, the Imperfective does occur to express general knowledge, especially in generic forms (as in \forme{tú-wɣ-fskɤr}  \textsc{ipfv}-\textsc{inv}-surround  `it surrounds it' above)  or with a dummy subject (S/A) (\forme{pjɯ-ʑe} \textsc{ipfv}-begin[III] `it begins' above). 
 
The Factual also is also found in this function with dynamic verbs, in particular when the subjects are overt and/or definite, as in (\ref{ex:mAndze}).\footnote{The Imperfective would also be possible here, however. }

\begin{exe}
\ex \label{ex:mAndze}
\gll ɯ-ku kɯ-mpɯ nɯ ɲɯ́-wɣ-pʰɯt tɕe, nɯŋa ra kɯ ndza-nɯ, paʁ kɯ mɤ-ndze \\
\textsc{3sg}.\textsc{poss}-head \textsc{sbj}:\textsc{pcp}-be.soft \textsc{dem} \textsc{ipfv}-\textsc{inv}-pluck \textsc{lnk} cow \textsc{pl} \textsc{erg} eat:\textsc{fact}-\textsc{pl} pig \textsc{erg} \textsc{neg}-eat:\textsc{fact} \\
\glt ` One plucks the (leaves) on the extremities, the soft ones, the cows eat it, the pigs don't.' (06 qaZmbri, 20)
\end{exe}
 
Second, with dynamic verbs, the Factual can express immediate future. In  assertive sentences with first person subject (\ref{ex:WqhWqhu.Gia}), or in interrogative sentences with second person subjects (\ref{ex:tWGi}), the Factual can be interpreted as indicating the intention to perform an action.

\begin{exe}
\ex \label{ex:WqhWqhu.Gia}
\gll ŋotɕu tɯ-ɕe nɤ-qʰɯ\redp{}qʰu ɣi-a, nɤ-ŋga nɤ-xtsa fkur-a \\
where 2-go:\textsc{fact} \textsc{2sg}.\textsc{poss}-\textsc{emph}\redp{}after come:\textsc{fact}-\textsc{1sg} \textsc{2sg}.\textsc{poss}-clothes  \textsc{2sg}.\textsc{poss}-shoes carry:\textsc{fact}-\textsc{1sg} \\
\glt `Wherever you go, I will go after you, I will carry your clothes and your shoes.' (26-kWlAGpopo, 37)
\end{exe}

\begin{exe}
\ex \label{ex:tWGi}
\gll mbarkʰom tʰɤjtɕu tɯ-ɣi? \\
Mbarkham when 2-come::\textsc{fact} \\
\glt `When are you coming to Mbarkham?' (Conversation, 2014)
\end{exe}

The Factual can also be used to express non-intentional future events when the speaker has reasonable reasons for assuming that they will take place as in (\ref{ex:GWsata}). 

\begin{exe}
\ex \label{ex:GWsata}
\gll si-a ɲɤ-sɯso, tʰa ɣɯ-sat-a ɲɤ-sɯso \\
die:\textsc{fact}-\textsc{1sg} \textsc{ifr}-think in.a.moment \textsc{inv}-kill:\textsc{fact}-\textsc{1sg} \textsc{ifr}-think \\
\glt `He thought ``I will die", he thought ``It will kill me".' (Buxiejiang gaizuo yisheng-zh, 30)
\end{exe}

Verbs in the Factual in this function can be combined with the affirmative copula \japhug{ɕti}{be} (\ref{ex:Gi.Cti}) to emphasize the certainty that the action will take place. 

\begin{exe}
\ex \label{ex:Gi.Cti}
\gll tɕetʰa ɣi ɕti ma, sɲikuku ʑo ju-ɣi ɲɯ-ɕti tɕe nɯ ntsɯ tu-ti ɲɯ-ɕti \\
 in.a.moment come:\textsc{fact} be:\textsc{aff}:\textsc{fact} \textsc{lnk} every.day \textsc{emph} \textsc{ipfv}-come \textsc{sens}-be:\textsc{aff} \textsc{lnk} \textsc{dem} always \textsc{ipfv}-say \textsc{sens}-be:\textsc{aff} \\
\glt `It will come  soon: it comes everyday and each times says this.' (qaCpa 2003, 185-186)
\end{exe}

They are also compatible with sentence final particles of epistemic modality such as \forme{tʰaŋ} `maybe, probably' (\ref{ex:Gi.thaN}), which can be used to temper the degree of assertion.

\begin{exe}
\ex \label{ex:Gi.thaN}
\gll a-mu jɯɣmɯr tɕe tɕi-scawa ɣe ma kʰu ɣi tʰaŋ nɤ \\
 \textsc{1sg}.\textsc{poss}-mother today.evening \textsc{lnk}  \textsc{1du}.\textsc{poss}-poor.of \textsc{sfp} \textsc{lnk} tiger  come:\textsc{fact} \textsc{sfp} \textsc{sfp} \\
\glt `Mother, poor of us, today evening the tiger is probably coming (for us).' (The tiger, 4)
 \end{exe}

\subsubsection{Use in complement clauses} \label{sec:fact.complement}
The Factual is found in the complement clause of modal auxiliaries such as \japhug{ra}{be needed} and \japhug{jɤɣ}{be allowed}, but occurrences are considerably fewer than those of the Imperfective (§\ref{sec:ipfv.complement}). It is used to refer to actions just about happen at the time of utterance (as in \ref{ex:nWnaj.WjAG} and \ref{ex:Cea.ra}), rather than generic statements.


\begin{exe}
\ex \label{ex:nWnaj.WjAG}
\gll nɯna-j ɯ́-jɤɣ \\
rest:\textsc{fact}-\textsc{1sg} \textsc{qu}-be.allowed:\textsc{fact} \\
\glt `Could we rest?' (2003 qachGa, 270)
\end{exe}

\begin{exe}
\ex \label{ex:Cea.ra}
\gll aʑo kɯ-kaβ ɕe-a ra \\
\textsc{1sg} \textsc{sbj}:\textsc{pcp}-scoop go:\textsc{fact}-\textsc{1sg} be.needed:\textsc{fact} \\
\glt `I have to go to scoop water.' (2014-kWlAG, 69)
\end{exe} 

\subsubsection{Use in Periphrastic TAME categories} \label{sec:fact.periphrastic}
The Factual occurs in fewer Periphrastic TAME constructions than the Imperfective (§\ref{sec:ipfv.periphrastic.TAME}).

With the Past Imperfective \forme{pɯ-ŋu} and Inferential Imperfective \forme{pjɤ-ŋu} forms of the copula \japhug{ŋu}{be}, it is used to build the Periphrastic Proximative (§\ref{sec:proximative.periphrastic}).

Some speakers (in particular Kunbzang Mtsho) combine the Factual with the Sensory \forme{ɲɯ-ŋu} of the copula to express the same meaning as the Inferential in narratives, notably with the verb \japhug{ti}{say}, where the constructions in (\ref{ex:ti.YWNu2}) occurs instead of \forme{to-ti} (\textsc{ifr}-say) `s/he said' (§\ref{sec:aor.narrative}).

\begin{exe}
\ex \label{ex:ti.YWNu2}
\gll `....' ti ɲɯ-ŋu \\
{ } say:\textsc{fact} \textsc{sens}-be \\
\glt `S/he said: `...'' (many examples)
\end{exe}

This form may however not be Factual in the proper sense, but rather the trace that \japhug{ti}{say} (a highly irregular verb, §\ref{sec:stem2.form}) used to have irregular preverbless forms (§\ref{sec:verbs.no.preverbs}). The archaic form \forme{kʰɯ-ti} `s/he said' provides additional support for this idea (§\ref{sec:khWti}).


%Future state
%tɕe pɣɤmbri a-tɤ-βze tɕe, nɯ wuma pe ɲɯ-ŋu
%2011-04-smanmi 196
\subsection{Sensory} \label{sec:sensory}

\subsubsection{Morphology} \label{sec:sensory.morphology}
The assertive Sensory form of regular verbs is build by combining the B type \textsc{westwards} \forme{ɲɯ-} preverb with stem I or stem III depending on person and transitivity (§\ref{sec:stem3.distribution}; stem II also occurs in one case, see \ref{ex:YAstWt} below). It is thus potentially ambiguous with the Imperfective of verbs selecting \textsc{westwards} as their intrinsic orientation. Examples (\ref{ex:YWYaR.Zo.Nu}) and (\ref{ex:kW.YWYaR}) in §\ref{sec:ipfv.inchoative} illustrate this syncretism between Sensory and Imperfective.

The negative Sensory is marked by the portmanteau prefix \forme{mɯ́j-} (§\ref{sec:neg.allomorphs}), differing from the negative Imperfective form \forme{mɯ-ɲɯ-} of verbs selecting the \textsc{westwards} preverbs. The \forme{mɯ́j-} prefix is not compatible with contracting verbs (§\ref{sec:contraction}) however: their negative Sensory form is \forme{mɯ-ɲɯ\trt}, as in (\ref{ex:mWYArtsi}).

\begin{exe}
\ex \label{ex:mWYArtsi}
\gll ɣzɯtʰɯz ndɤre, sɯjno tɕi mɯ-ɲɯ-ɤ-rtsi, si tɕi mɯ-ɲɯ-ɤ-rtsi. \\
Selaginella \textsc{lnk} grass also \textsc{neg}-\textsc{sens}-\textsc{pass}-count tree also \textsc{neg}-\textsc{sens}-\textsc{pass}-count \\
\glt `The selaginella can neither be counted as a species of grass, nor as a tree.' (16-RlWmsWsi, 98)
\end{exe}

The Sensory commonly appears in combination with the Progressive \forme{ɲɯ-ɤsɯ-}/\forme{ɲɯ-ɤz-} with transitive verbs. The negative form of the Sensory progressive is \forme{mɯ-ɲɯ-} (\ref{ex:mWYAsWlAt}), like that of a contracting verbs.

\begin{exe}
\ex \label{ex:mWYAsWlAt}
\gll   tɯ-mɯ pɤjkʰu mɯ-ɲɯ-ɤsɯ-lɤt ri, qale ɲɯ-ɤsɯ-βzu.\\
\textsc{indef}.\textsc{poss}-sky yet \textsc{neg}-\textsc{sens}-\textsc{prog}-release \textsc{lnk} wind \textsc{sens}-\textsc{prog}-make\\
\glt `It is not yet raining, but there is wind.' (conversation, 15-06-05)
\end{exe}

The Sensory form of the verb \japhug{ti}{say} has stem I \forme{ɲɯ-ti} when occurring without the Progressive (\ref{ex:YWtindZi}), but unexpectedly selects stem II \forme{ɲɯ-ɤsɯ-tɯt}, \forme{ɲɯ-ɤs-tɯt} with Progressive (\ref{ex:YAstWt}), which also presents the irregular allomorph \forme{-ɤs-} (§\ref{sec:progressive.morphology}).
 

\begin{exe}
\ex \label{ex:YWtindZi}
\gll nɯ kɯ-fse a-pa a-ma ni kɯ ɲɯ-ti-ndʑi tɕe \\
\textsc{dem} \textsc{sbj}:\textsc{pcp}-be.like \textsc{1sg}.\textsc{poss}-father \textsc{1sg}.\textsc{poss}-mother \textsc{du} \textsc{erg} \textsc{sens}-say-\textsc{du} \textsc{lnk} \\
\glt `My parents are saying these things.' (2003nyima2, 94)
\end{exe}

\begin{exe}
\ex \label{ex:YAstWt}
\gll a-ɬaʁ kɯ nɯ ɲɯ-ɤs-tɯt \\
\textsc{1sg}.\textsc{poss}-FZ \textsc{erg} \textsc{dem} \textsc{sens}-\textsc{prog}-say[II] \\
\glt `My aunt is saying (those things).' (2003 smanmi2, 147)
\end{exe}

The irregular existential verbs \japhug{ɣɤʑu}{exist} and \japhug{maŋe}{not exist} are the suppletive Sensory forms of \japhug{tu}{exist} and \japhug{me}{not exist} (§\ref{sec:suppletive.negative}, §\ref{sec:intr.person.irregular}). They are the only verbs whose Sensory forms is not marked by a prefix. 



\subsubsection{Direct perception} \label{sec:sensory.evd}
The Sensory is used to express access to information \citep{tournadre14evidentiality} through any of the senses, most commonly vision, but also hearing (\ref{ex:tumbri}), touch (\ref{ex:YWmpW}), smell (\ref{ex:tunAmnAmnW}) and taste (\ref{ex:YWmWm}). It implies the discovery of a previously unknown fact or confirmation of an uncertain fact.

\begin{exe}
\ex \label{ex:tumbri}
\gll tu-mbri tɕe ɯ-skɤt wuma ʑo ɲɯ-mpɕɤr \\
\textsc{ipfv}-cry \textsc{lnk} \textsc{3sg}.\textsc{poss}-voice really \textsc{emph} \textsc{sens}-be.beautiful \\
\glt `When it cries, its voice is very beautiful.' (04-cuiniao-zh, 26)
\end{exe}

\begin{exe}
\ex \label{ex:YWmpW}
\gll ɲɯ́-wɣ-nɤmɤle tɕe ɲɯ-mpɯ. \\
\textsc{ipfv}-\textsc{inv}-touch \textsc{lnk} \textsc{sens}-be.soft \\
\glt `It is soft to the touch.' (19 khWlu, 25)
\end{exe}

\begin{exe}
\ex \label{ex:tunAmnAmnW}
\gll tɕe nɯ tu-nɤmnɤm-nɯ tɕe, cɤmtsho ɯ-di, pɯ\redp{}pɯ-ŋu nɤ, ɯ-di ɲɯ-mnɤm, tɕe nɯnɯ tɕu ɯ-fsa tu-ta-nɯ ɲɯ-ŋgrɤl. \\
\textsc{lnk} \textsc{dem} \textsc{ipfv}-smell-\textsc{pl} \textsc{lnk} musk \textsc{3sg}.\textsc{poss}-smell \textsc{cond\redp{}pst.ipfv}-be \textsc{lnk} \textsc{3sg}.\textsc{poss}-smell \textsc{sens}-be.smell \textsc{lnk} \textsc{dem} \textsc{loc} \textsc{3sg}.\textsc{poss}-snare \textsc{ipfv}-put-\textsc{pl} \textsc{sens}-be.usually.the.case \\
\glt `(The hunters) smell (the places where they find deer hair); if it is smell of musk, it is very strong. And they put the snare there.' (27-kikakCi, 68)
\end{exe}

\begin{exe}
\ex \label{ex:YWmWm}
\gll tú-wɣ-ndza tɕe wuma ʑo ɲɯ-mɯm ɲɯ-ti \\
\textsc{ipfv}-\textsc{inv}-eat \textsc{lnk} really \textsc{emph} \textsc{sens}-be.tasty \textsc{sens}-say \\
\glt `She said: `(These ferns, prepared this way) are very nice to eat'.' (said just after eating them; conversation 14.05.10)
\end{exe}

Although in the above examples there is no implication that the person producing the sound or the objects mentioned in the sentences are not visible to the speaker, in these contexts vision is largely irrelevant to determine the property in question, and there is not ambiguity as to which sensory channel was responsible for obtaining the information.  


\subsubsection{Endopathic and extra-sensorial perception} \label{sec:sensory.endopathic}
As in other languages of the area, but unlike Lhasa Tibetan \citep{tournadre14evidentiality}, the Sensory form is used for endopathic sensations (pain, itch, cold etc)  relating to the speaker, as in example (\ref{ex:YWmNAm}).

\begin{exe}
\ex \label{ex:YWmNAm}
\gll tʰam tɕe mɯ́j-cʰa-a, a-mi ɲɯ-mŋɤm. \\
now \textsc{lnk} \textsc{neg}:\textsc{sens}-can-\textsc{1sg} \textsc{1sg}.\textsc{poss}-foot \textsc{sens}-hurt \\
\glt `Now I can't, my foot hurts.' (21-kuGrummAG, 24)
\end{exe}


In (\ref{ex:tWCGa.YWmNAm}), the Sensory is used in a generic sentence, when the speaker has experienced himself the feeling and recounts his experience while presenting it as a generic fact (§\ref{sec:1.genr}), and thus do not count as a real example of endopathic Sensory with an experiencer other than the speaker.

\begin{exe}
\ex \label{ex:tWCGa.YWmNAm}
\gll kɯ-maqʰu qʰe tɯ-ɕɣa ɲɯ-mŋɤm \\
\textsc{sbj}:\textsc{pcp}-be.after \textsc{lnk} \textsc{genr}.\textsc{poss}-tooth \textsc{sens}-hurt \\
\glt `Afterwards tooths hurt.' (27 tApGi, 66)
\end{exe}

In (\ref{ex:nWrqo.YWmNAm}), which describes the effects of foot and mouth disease on cattle, the speaker uses the Sensory to describe an inference about the endopathic feelings of the cattle suffering from the disease (they are in pain), based on information from vision and hearing (their whining and behaviour).

\begin{exe}
\ex \label{ex:nWrqo.YWmNAm}
\gll nɯ-mci kɤ-rɤwum maka mɯ́j-cʰa-nɯ tɕe nɯ-mci tu-ɣɤrɯβrɯβ ʑo ɲɯ-ŋu. tɕe nɯ-rqo ɲɯ-mŋɤm rca, \\
\textsc{3pl}.\textsc{poss}-saliva \textsc{inf}-collect at.all \textsc{neg}:\textsc{sens}-can-\textsc{pl} \textsc{lnk} \textsc{3pl}.\textsc{poss}-saliva \textsc{ipfv}-flow.continuously \textsc{emph} \textsc{sens}-be \textsc{lnk} \textsc{3pl}.\textsc{poss}-throat \textsc{sens}-hurt \textsc{sfp} \\
\glt `They cannot keep the saliva in their mouths, and it flows continuously. Their throats hurt.' (27-kharwut, 6)
\end{exe}

In (\ref{ex:Wmi.YWmNAm}) likewise we have the Sensory used with \japhug{mŋɤm}{hurt} to describe an event visually witnessed by the speaker.

\begin{exe}
\ex \label{ex:Wmi.YWmNAm}
\gll kɯɕnɤsqi tʰɯ-azɣɯt ri, tɕe pɤjkʰu ɯ-mi ɲɯ-mŋɤm tɕe ri, nɯ kɯnɤ kʰa tsʰitsuku ɲɯ-nɤme ɕti. \\
seventy \textsc{aor}-reach but \textsc{lnk} already \textsc{3sg}.\textsc{poss}-foot \textsc{sens}-hurt \textsc{lnk} but \textsc{dem} also house some.things \textsc{sens}-work[III] be.\textsc{aff}:\textsc{fact} \\
\glt `He is seventy, his foot hurts already, but even like that he does all sorts of work at home.' (14-siblings, 49-50)
\end{exe}
 
The Sensory is also possible in the case of extra-sensory perception obtained by divination. In (\ref{ex:XsAr.tWtaNGoR.YArku}), it occurs on the verb \forme{ɲɯ-ɤ-rku} which refers to a present situation (unknown to other people) and on \forme{ɲɯ-pʰɤn} about a future event.
 
\begin{exe}
\ex \label{ex:XsAr.tWtaNGoR.YArku}
\gll ``ɯ-qa nɯtɕu χsɤr tɯ-taɴɢoʁ ɲɯ-ɤ-rku tɕe, nɯ a-tɤ-sɯ-tɕɤt tɕe ɲɯ-pʰɤn'' ɲɯ-ti \\
\textsc{3sg}.\textsc{poss}-root \textsc{dem}:\textsc{loc} gold one-basket \textsc{sens}-\textsc{pass}-be.put.in \textsc{lnk} \textsc{dem} \textsc{irr}-\textsc{pfv}-\textsc{caus}-take.out \textsc{lnk} \textsc{sens}-be.efficient \textsc{sens}-say \\
\glt `(After performing the divination, the lama) said `There is one basketful of gold (buried) at the root (of the tree)$_i$, if it$_i$ has someone take it out, it will solve (its$_i$ problems). ' (2003 divination, 104)
\end{exe}

\subsubsection{The expression of surprise} \label{sec:mirative}
Japhug has several interjections specifically used to express surprise (§\ref{sec:interjections}), \forme{amaŋ} and \japhug{mtsʰɤri}{how strange}. 

The Sensory often occurs in such contexts, as in (\ref{ex:amang.YWmbro}), as expected for a direct visual perception (§\ref{sec:sensory.evd}). This use of the Sensory evidential is what motivated its analysis as a mirative marker in some languages (\citealt{hill12mirativity}, \citealt{delancey12still}, \citealt{aikhenvald12mirativity}).

\begin{exe}
\ex \label{ex:amang.YWmbro}
 \gll amaŋ, nɯstʰɯci ɲɯ-mbro \\
 \textsc{interjection}:\textsc{surprise} so.much \textsc{sens}-be.high \\
 \glt `It is so high!' (150826 liyu tiao longmen-zh, 75)
\end{exe}

When the predicate is a stative verb, the degree nominal construction (§\ref{sec:degree.nominal.construction}, §\ref{sec:non.verbal.predicates}), can alternatively be used to express the unexpected high degree of the observed property.

\begin{exe}
\ex \label{ex:amang.WtWsAre}
 \gll amaŋ, nɯ ɯ-tɯ-sɤre, mtsʰɤri, ɯ-tɯ-sɤmtsʰɤr nɯ \\
 \textsc{interjection}:\textsc{surprise} \textsc{dem} \textsc{3sg}-\textsc{nmlz}:\textsc{deg}-be.funny  how.strange \textsc{3sg}-\textsc{nmlz}:\textsc{deg}-be.surprising \textsc{sfp} \\
 \glt `It is so funny, so surprising!' (150830 baihe jiemei-zh, 112)
\end{exe}



\subsubsection{Other functions} \label{sec:sensory.other}
The Sensory is commonly used instead of the Factual for describing facts about animals that do not live in Tibetan areas. Compare for instance the forms of the stative verbs \japhug{sɤɣmu}{be terrifying} and \japhug{mpɕɤr}{be beautiful}: they appear in the Factual when referring to spiders or flowers found in the area (\ref{ex:sAGmu} and \ref{ex:mpCAr}) and in the Sensory when referring to lions and gnus, which the speaker has only seen in zoos or in the television  (\ref{ex:YWsAGmu} and \ref{ex:YWmpCAr}). The choice of the Sensory instead of the Factual in such context might be a way for the speaker to highlight the fact that this information comes from his/her own personal experience, because s/he may not take for granted that everybody shares this knowledge.

 \begin{exe}
\ex \label{ex:sAGmu}
\gll  ŋgoŋpu ɴɢoɕna kɤ-ti ci tu tɕe, nɯnɯ wxti nɯ stoʁ jamar tu. kú-wɣ-rtoʁ tɕe sɤɣ-mu. \\
disaster spider \textsc{obj}:\textsc{pcp}-say \textsc{indef} exist:\textsc{fact} \textsc{lnk} \textsc{dem} be.big:\textsc{fact}  \textsc{dem} bean about exist:\textsc{fact} \textsc{ipfv}-\textsc{inv}-look.at \textsc{lnk} \textsc{prop}-be.afraid:\textsc{fact} \\
\glt `There is one that is  called `disaster spider', it is big, like the size of a bean. It is terrifying to look at it.' (26 mYaRmtsaR, 151)
\end{exe}


\begin{exe}
\ex \label{ex:mpCAr}
\gll nɯnɯ ɯ-mɯntoʁ nɯ mpɕɤr. \\
\textsc{dem} \textsc{3sg}.\textsc{poss}-flower \textsc{dem} be.beautiful:\textsc{fact} \\
\glt `Its flower is beautiful.' (15-babW, 105)
\end{exe}

\begin{exe}
	\ex \label{ex:YWsAGmu}
	\gll  sɯŋgi nɯ ɲɯ-sɤɣ-mu. \\
	lion \textsc{dem} \textsc{sens}-\textsc{prop}-be.afraid \\
	\glt `The lion is terrifying.' (20 sWNgi, 64)
\end{exe}

\begin{exe}
\ex \label{ex:YWmpCAr}
\gll <jiaoma> nɯ ɲɯ-mpɕɤr \\
gnu \textsc{dem} \textsc{sens}-be.beautiful \\
\glt `The Gnu is beautiful.' (20-RmbroN, 128)
\end{exe}

The Sensory is also common in all finite comparative constructions (§\ref{sec:comparison}), whether the comparee is marked with the ergative (\ref{ex:slAzWn.kW.YWdAn}) (§\ref{sec:comparee.kW}) or the standard takes the comparative postposition (\ref{ex:sAz.YWwxti}) (§\ref{sec:comparative}). The presence of the Sensory in such constructions might be due to the fact that comparisons require an evaluation of the respective positions of the standard and the comparee on a scale, and that the result of this comparison is thus `freshly' obtained information.

\begin{exe}
\ex \label{ex:slAzWn.kW.YWdAn}
\gll slɤzɯn kɯ ɲɯ-dɤn, tɕe tɯ-xpa tɕe ʁnɯ-ɣjɤn jamar ɣɤʑu, ʁmbɣɯzɯn ɲɯ-rkɯn ɕti. \\
lunar.eclipse \textsc{erg} \textsc{sens}-be.many \textsc{lnk} one-year \textsc{loc} two-times about exist:\textsc{sens} solar.eclipse \textsc{sens}-be.few be.\textsc{aff}:\textsc{fact} \\
\glt `Lunar eclipse are more numerous, they occur about twice a year, while solar eclipse are rarer.' (29-mWBZi, 169-170)
\end{exe}

\begin{exe}
\ex \label{ex:sAz.YWwxti}
\gll ɯ-rna nɯ mbro ɣɯ sɤz ɲɯ-wxti. \\
\textsc{3sg}.\textsc{poss}-ear \textsc{dem} horse \textsc{gen} \textsc{comp} \textsc{sens}-be.big \\
\glt `Its ears are bigger than those of the horse.'  (20-tArka, 5)
\end{exe}

The use of the Sensory is not obligatory in such constructions: the Factual is also possible. 

\subsubsection{Sensory evidential and person} \label{sec:sensory.person}
With second person subjects, the Sensory is very commonly used to state a fact about the addressee that the speaker noticed (not something he knew previously). For instance, in contrast to (\ref{ex:tWmkhAz}) in the Factual in which the addressee's (recent) actions are irrelevant, a sentence such as (\ref{ex:YWtWmkhAz}) can be used if the speaker witnessed something revealing the proficiency of the addressee.

\begin{exe}
\ex \label{ex:YWtWmkhAz}
\gll  ɲɯ-tɯ-mkʰɤz \\
\textsc{sens}-2-be.expert \\
\glt `You are good at it.' (heard in several conversations)
\end{exe}

\begin{exe}
\ex \label{ex:tWmkhAz}
\gll nɤʑo stu ʑo tɯ-mkʰɤz tɕe, tɕe nɤʑo ɕ-tɤ-nɤme \\
\textsc{2sg} most \textsc{emph} 2-be.expert:\textsc{fact}   \textsc{lnk} \textsc{lnk} \textsc{2sg} \textsc{tral}-\textsc{imp}-do[III] \\
\glt `You are the best at it, do it!' (150822 laoye zuoshi zongshi duide-zh, 37)
\end{exe}


With first person subjects, the Sensory is not rare. It is common with verbs such as \japhug{rga}{be happy} whose intransitive subject is the experiencer, as in (\ref{ex:YWrgaa}) (see §\ref{sec:egophoric.tripartite} on the contrast between Sensory, Egophoric Present and Factual in such contexts).

\begin{exe}
\ex \label{ex:YWrgaa}
\gll nɤ-tɕɯ tɤ-sci tɕe ɲɯ-pe tɕe papa, aʑo ɲɯ-rga-a\\
 \textsc{2sg}.\textsc{poss}-child \textsc{aor}-born \textsc{lnk} \textsc{sens}-good \textsc{lnk} good  \textsc{1sg} \textsc{sens}-be.happy-\textsc{1sg}\\
\glt `It is nice that your son is born, I am happy.' (conversation, 2013)
\end{exe}

With non-experiencer adjectival stative verbs, it can occur if the speaker discovers something about oneself, for instance from the behaviour of others as in (\ref{ex:YWsAjloRa}).\footnote{This example is taken from the translation of Andersen's story `The Ugly Duckling', when a hunting dog appears before the eponymous character but does not bite him. }

\begin{exe}
\ex \label{ex:YWsAjloRa}
\gll  aʑo ndɤre ɲɯ-sɤjloʁ-a tɕe, tɤrʁaʁkɕi kɯnɤ ʑo kú-wɣ-mtsɯɣ-a mɯ́j-sɯsɤm \\
\textsc{1sg} on.the.other.hand \textsc{sens}-be.ugly-\textsc{1sg} \textsc{lnk} hunting.dog also \textsc{emph} \textsc{ipfv}-\textsc{inv}-bite-\textsc{1sg} \textsc{neg}:\textsc{sens}-think[III] \\
\glt `I am (so) ugly that even a hunting dog does not want to bite me.'  (140519 chou xiaoya-zh, 86)
\end{exe}

 
\subsubsection{Tense and aspect} \label{sec:sensory.functions}
The Sensory never has an inchoative meaning with stative verbs, unlike the Imperfective (§\ref{sec:ipfv.inchoative}). It can be used to describe both ongoing events or habitual/generic situations (§\ref{sec:sensory.other}).
With transitive verbs, it often occurs with the progressive (§\ref{sec:sensory.morphology}). 

The Sensory is mainly found in non-past contexts. However, unlike the Factual and the Egophoric, the Sensory can refer to past events. In (\ref{ex:aphe.YWtinW}), the verb \forme{ɲɯ-ti-nɯ} `they say/said' concerns an event that had occurred decades before.

\begin{exe}
\ex \label{ex:aphe.YWtinW}
\gll  aʑo a-pʰe ``nɤʑo ju-tɯ-ɕe ra" ɲɯ-ti-nɯ. \\
\textsc{1sg} \textsc{1sg}.\textsc{poss}-\textsc{dat} \textsc{2sg} \textsc{ipfv}-2-go be.needed:\textsc{fact} \textsc{sens}-say-\textsc{pl} \\
\glt  `They said to me: `You have to go.'' (2010-09, 104)
\end{exe}

The Sensory \forme{ɲɯ-ti} of the verb \japhug{ti}{say} is the normal way to report the words uttered by a third person, when the speaker has heard them directly (as is obviously the case in \ref{ex:aphe.YWtinW}). The Aorist \forme{ta-tɯt}, which is used to describe past events directly witnessed by the speaker (§\ref{sec:aor.main}), is never found in conversations to quote someone else's words (it occurs in temporal §\ref{sec:aor.temporal} and relative §\ref{sec:aor.relative} clauses).

The Sensory also occurs in future contexts in the apodosis of conditional clause to express the prediction of a likely outcome, as in (\ref{ex:atAwxti.YWphAn}) (see also \ref{ex:XsAr.tWtaNGoR.YArku}, §\ref{sec:sensory.endopathic}).

\begin{exe}
\ex \label{ex:atAwxti.YWphAn}
\gll nɯ tɤ-ŋu tɕe tɕe, si lú-wɣ-ɣɤjɯ, smi a-tɤ-wxti tɕe ɲɯ-pʰɤn \\
\textsc{dem} \textsc{aor}-be \textsc{lnk} \textsc{lnk} wood \textsc{ipfv}-\textsc{inv}-add fire \textsc{irr}-\textsc{pfv}-be.big \textsc{lnk} \textsc{sens}-be.efficient \\
\glt `(He thought:) `In this case, if (I) add more firewood, and if the fire is bigger, it should work.' (150827 taisui-zh, 63)
\end{exe}

The Sensory is also found to express events that one has not yet perceived, but which one expects to be perceptible, as illustrated by the use of the verbs \forme{ɣɤʑu} `exist' and \forme{ɯ-ɲɯ́-ŋu} `isn't it' in (\ref{ex:future.GAZu}).

\begin{exe}
\ex \label{ex:future.GAZu}
\gll  aki ɕ-pɯ-sɤŋo ma [...] ``ndzaʁlaŋ tɯrme jo-ɣi tɕe, tɯ-ci kɤ-kɯ-nɯχtɕɤn ɯ-ŋgɯ ɕ-pjɯ́-wɣ-βde ɲɯ-ra" ɯ-kɯ-ti ci ɣɤʑu tɕe, ɯ-ɲɯ́-ŋu kɯ? \\
down.there \textsc{tral}-\textsc{imp}:\textsc{down}-listen \textsc{lnk} { } Jambudvîpa man \textsc{ifr}-come \textsc{lnk} \textsc{indef}.\textsc{poss}-water \textsc{aor}-\textsc{sbj}:\textsc{pcp}-be.fierce \textsc{3sg}.\textsc{poss}-inside \textsc{tral}-\textsc{ipfv}:\textsc{down}-\textsc{inv}-throw \textsc{sens}-be.needed \textsc{3sg}.\textsc{poss}-\textsc{sbj}:\textsc{pcp}-say \textsc{indef} exist:\textsc{sens} \textsc{lnk} \textsc{qu}-\textsc{sens}-be \textsc{qu} \\
\glt `Go and listen down there to (see whether) there is someone saying `A man from Jambudvîpa has come, let us throw him into the fierce water.' (28-smAnmi, 159)
\end{exe}

\subsection{Egophoric Present} \label{sec:egophoric}


\subsubsection{Morphology} \label{sec:egophoric.morphology}
The Egophoric Present is build combining the B type \textsc{eastwards} \forme{ku-} preverb with stem I or stem III depending on person and transitivity (§\ref{sec:stem3.distribution}). Verbs selecting \textsc{eastwards} as their intrinsic orientation therefore have syncretism between Imperfective and Egophoric. For instance, the form \forme{ku-rɤʑi-a} is in the Egophoric Present in (\ref{ex:kurAZia.egoph}) and in the Imperfective in (\ref{ex:kurAZia.ipfv}) (see §\ref{sec:ipfv.complement} on this use of the Imperfective).

\begin{exe}
\ex \label{ex:kurAZia.egoph}
\gll kɯre ku-rɤʑi-a \\
\textsc{dem}.\textsc{prox}:\textsc{loc} \textsc{prs}-stay-\textsc{1sg} \\
\glt `I am here.' (heard in context)
\end{exe}

\begin{exe}
\ex \label{ex:kurAZia.ipfv}
\gll kutɕu ku-rɤʑi-a ɲɯ-ɬoʁ \\
\textsc{dem}.\textsc{prox}:\textsc{loc} \textsc{ipfv}-stay-\textsc{1sg} \textsc{sens}-be.needed \\
\glt `I have to stay here.' (28-qAjdoskAt, 78)
\end{exe}

Unlike the Sensory (§\ref{sec:sensory.morphology}), the Egophoric Present does not have a special negative form, and selects the \forme{mɯ-} negative prefix, as in (\ref{ex:mWkuchaa}).

\begin{exe}
\ex \label{ex:mWkuchaa}
\gll kɯ-ɤrqʰi kɤ-ɕe mɯ-ku-cʰa-a \\
\textsc{sbj}:\textsc{pcp}-be.far \textsc{inf}-go \textsc{neg}-\textsc{prs}-can-\textsc{1sg} \\
\glt `I cannot go very far.' (for now, due to an accident; conversation 17-09-21)
\end{exe}

The Egophoric Present form of the existential verbs \japhug{tu}{exist} and \japhug{me}{not exist} is regular: \forme{ku-tu} and \forme{ku-me} (\ref{ex:kume}), respectively. The copulas \japhug{ŋu}{be} and \japhug{maʁ}{not be} lack an Egophoric Present form.

\begin{exe}
	\ex \label{ex:kume}
	\gll aʑo kɯre a-ʁa ku-me tɯ-mgo ku-osɯ-βzu-a ɕti \\
	\textsc{1sg} here \textsc{1sg}.\textsc{poss}-free.time \textsc{prs}-not.exist \textsc{indef}.\textsc{poss}-food \textsc{prs}-\textsc{prog}-make-\textsc{1sg} be.\textsc{aff}:\textsc{fact}\\
	\glt `I don't have time, I am making food.'  (Rkangrgyal, 47)
\end{exe}

Transitive verbs often combine the Egophoric Present with the Progressive \forme{asɯ-} (§\ref{sec:progressive.morphology}). The vowel of the preverb merges with that of the Progressive prefix as \forme{ku-osɯ-} / \forme{ku-oz\trt}, as in (\ref{ex:konWsWndzaj2}).

\begin{exe}
\ex \label{ex:konWsWndzaj2}
\gll ɕa ʁɟa ʑo ku-o<nɯ>sɯ-ndza-j ma fsapaʁ ɯ-tɯ-si ɯ-grɤl maŋe. \\
meat completely \textsc{emph} \textsc{prs}-\textsc{prog}<\textsc{auto}>-eat-\textsc{1pl} \textsc{lnk} animal \textsc{3sg}.\textsc{poss}-\textsc{nmlz}:\textsc{deg}-die \textsc{3sg}.\textsc{poss}-order not.exist:\textsc{sens} \\
\glt `(These days) we are eating only meat, as (domestic) animals have been dying in great numbers (due to a disease).' (2003 kandZislama, 132)
\end{exe}

\subsubsection{Egophoric Present and first/second person indexation} \label{sec:egophoric.evd} \label{sec:egophoric.interrogative}
The Egophoric Present, while common in conversations, is nearly non-existent in narrative and procedural texts (outside of quotations) unlike the Factual and the Sensory. 

In declarative sentences, Egophoric Present can occur with first person subject, whether intransitive subject as in \forme{ku-nɯna-j} `we are resting' (\ref{ex:kunWnaj}) or transitive subjects as in \forme{ku-taʁ-a} `I am weaving it' (\ref{ex:kutaRa}).

\begin{exe}
	\ex \label{ex:kunWnaj}
	\gll kɯre ku-nɯna-j \\
	here \textsc{egop}-rest-\textsc{1pl} \\
	\glt `(Today, on the National Holiday), we are resting here.' (conversation, 16-10-01)
\end{exe}
	
\begin{exe}
\ex \label{ex:kutaRa}
\gll <kuabao> ɯ-spa ci ku-taʁ-a \\
satchel \textsc{3sg}.\textsc{poss}-material \textsc{indef} \textsc{prs}-weave-\textsc{1sg} \\
\glt `I am weaving a satchel.' (conversation, 14-11-25)
\end{exe}

No example of Egophoric Present with second person subject in declarative sentences has been found in the corpus, nor could such example be elicited. However, second person objects are possible, as in (\ref{ex:kutanAjo}).

\begin{exe}
	\ex \label{ex:kutanAjo}
	\gll kɯre ku-ta-nɤjo \\
	\textsc{dem}.\textsc{prox}:\textsc{loc} \textsc{prs}-1\fl{}2-wait \\
	\glt `I am right here waiting for you.' (heard in context)
\end{exe}

As a result of the anticipation rule 
(§\ref{sec:anticipation.person}), the person constraint on the Egophoric Present is reversed in interrogative sentences. As shown by (\ref{ex:WkutWscitnW}) and (\ref{ex:kutWrAZi}), the Egophoric Present appears with second person subjects, and is not attested with first person. These two questions expect answers such as (\ref{ex:kusciti}) and (\ref{ex:kurAZia.egoph}) (in §\ref{sec:egophoric.morphology}) with first person and Egophoric.

\begin{exe}
	\ex \label{ex:WkutWscitnW}
	\gll ɯ-kú-tɯ-scit-nɯ? \\
	\textsc{qu}-\textsc{prs}-2-be.happy-\textsc{pl} \\
	\glt `Are you (and your family) happy?' (2002 qaCpa, 121)
\end{exe}

\begin{exe}
	\ex \label{ex:kutWrAZi}
	\gll ŋotɕu ku-tɯ-rɤʑi? \\
	where \textsc{prs}-2-stay \\
	\glt `Where are you?' (heard in context)
\end{exe}

The person constraints described above are the reason for calling the TAME category discussed in this section ``egophoric'', designating a type of evidentiality specifically marking `information as known through conscious personal involvement' \citep{hill20egophoricity}.\footnote{Hill argues in favour of replacing `egophoric' with the term `personal evidential'. } 
 
\subsubsection{Egophoric Present and third person} 
The Egophoric Present can occur with third person subjects. This use is particularly common in declarative sentences when the subject is a noun with a first person possessive prefix, whether an abstract inalienable noun as in the case of \forme{a-ʁa} `my free time' in (\ref{ex:kume}), or kinship terms as in (\ref{ex:kunWsWmWzdWGndZi}).

\begin{exe}
\ex \label{ex:kunWsWmWzdWGndZi}
\gll a-wi cʰo a-wa ni nɤʑo nɤ-ndʐa kɯ wuma ʑo ku-nɯsɯmɯzdɯɣ-ndʑi tɕe \\
\textsc{1sg}.\textsc{poss}-grandmother \textsc{comit} \textsc{1sg}.\textsc{poss}-father \textsc{du} \textsc{2sg} \textsc{2sg}.\textsc{poss}-reason \textsc{erg} really \textsc{emph} \textsc{prs}-worry-\textsc{du} \textsc{lnk} \\
\glt `Grandmother and Father are very worried because of you.' (150819 haidenver-zh, 494)
\end{exe}

Conversely, Egophoric is also frequent with second person possessors in interrogative sentences, such as \forme{nɤ-ma} `your work' as in (\ref{ex:nAma.WkudAn}).

\begin{exe} 
	\ex 
	\begin{xlist}
		\ex \label{ex:nAma.WkudAn}
		\gll nɤ-ma ɯ-kú-dɤn? \\
		\textsc{2sg}.\textsc{poss}-work \textsc{qu}-\textsc{prs}-be.many \\
		\glt `Do you have a lot of work?' (heard in context)
		\ex \label{ex:ama.kudAn}
		\gll a-ma ku-dɤn \\
		\textsc{1sg}.\textsc{poss}-work \textsc{prs}-be.many \\
		\glt `I have a lot of work.' 
	\end{xlist}
\end{exe} 

With non-possessed subjects nouns, it also occurs when the first person is a beneficiary, marked either as a possessor (\ref{ex:kW.koznAma}) or with oblique flagging as in the clause \forme{a-taʁ wuma ku-sna} `he is nice to me' in (\ref{ex:kusciti}) above.

\begin{exe}
	\ex \label{ex:kW.koznAma}
	\gll χpɤltɕin kɯ a-ma ra ku-oz-nɤma \\
	\textsc{anthr} \textsc{erg} \textsc{1sg}.\textsc{poss}-work \textsc{pl} \textsc{prs}-\textsc{prog}-do \\
	\glt `Dpalcan is doing my housework (in my stead).' (conversation, 16-04-12)
\end{exe}

\begin{exe}
	\ex \label{ex:kusciti}
	\gll tɕʰeme nɯ kɯ `wuma ʑo ku-scit-i, rɟɤlpu ri a-taʁ wuma ku-sna, ʁjoʁ ra ri wuma ʑo ku-pe-nɯ' to-ti\\
	girl \textsc{dem} \textsc{erg} really \textsc{emph} \textsc{prs}-be.happy-\textsc{1pl}  king also \textsc{1sg}-on really \textsc{prs}-be.kind servant \textsc{pl} also really \textsc{emph}   \textsc{prs}-be.good \textsc{ifr}-say\\
	\glt `The girl said: `We are very happy, the king is very kind to me, the servants are very nice.'' (2002 qaCpa, 122-4)
\end{exe}

The use of Egophoric with third person subjects is an instance of what has been termed `broad egophoric' by \citep[89]{gawne17bodish}. Japhug appears to allow a wider range of third person referents to occur with egophoric than most languages. There are examples of Egophoric Present with third person subjects in which the personal involvement of the speaker is not immediately obvious.



In (\ref{ex:shangban.kosWBzu}), we find Egophoric Present on the main verb, even though the transitive subject and the speaker were not together at time of utterance.\footnote{There is however a relationship between them, since the subject of (\ref{ex:shangban.kosWBzu}) is the daughter of the speaker. }


\begin{exe}
	\ex \label{ex:shangban.kosWBzu}
	\gll alan nɤki, ʁdɯrɟɤt ri <shangban> ku-osɯ-βzu \\
	\textsc{anthr} \textsc{filler}  \textsc{topo} \textsc{loc} office.work \textsc{prs}-\textsc{prog}-do \\
	\glt `Alan, she is doing office work in Gdongbrgyad (these days).' (conversation, 2014-12-24)
\end{exe}

In (\ref{ex:pAjkhu.kuWnme}), the use of the Egophoric Present is not straightforward.\footnote{It is possible that the speaker selects this form because she describes a situation that directly concerns herself, since she is among the inhabitants of the district, who are not sick from the disease. }

\begin{exe}
	\ex \label{ex:pAjkhu.kuWnme}
	\gll kɯ-ngo wuma ʑo rkɯn, <abazhou> [...] pɯ-nɯ-mɯ\redp{}me ɕti qʰe, pɤjkʰu ku-nɯ-me. \\
	\textsc{sbj}:\textsc{pcp}-be.sick really \textsc{emph} be.few:\textsc{fact} Rngaba {  } \textsc{pst}.\textsc{ipfv}-\textsc{auto}-\textsc{emph}\redp{}not.exist be.\textsc{aff}:\textsc{fact} \textsc{lnk} still \textsc{prs}-\textsc{auto}-not.exist \\
	\\
	\glt `There are few people sick (of the Covid), (here) in Rngaba district there never was anyone, and even now there is not (a single) one.' 	(conversation 2020-07-31)
\end{exe}

Tshendzin explains the form \forme{ku-nɯ-me} `there is not' here as (\ref{ex:tham.kWnA.me}) with the Factual Non-Past.
 
\begin{exe}
	\ex \label{ex:tham.kWnA.me}
	\gll `tʰam kɯnɤ me' kɤ-ti ɲɯ-ŋu. \\
	now also not.exist:\textsc{fact} \textsc{inf}-say \textsc{sens}-be \\
	\glt `It means `not even now''.'
\end{exe}


\subsubsection{Tense and aspect} \label{sec:egophoric.tense}
The Egophoric Present expresses ongoing events or repeated actions occurring during a short time range around the present time. Thus, the sentence (\ref{ex:tChi.kutWnAme}) can either mean `What are you doing \textit{right now?}' or `What are you doing \textit{these days}?'.

\begin{exe}
\ex \label{ex:tChi.kutWnAme}
\gll tɕʰi ku-tɯ-nɤme? \\
what \textsc{prs}-2-do[III] \\
\glt `What are you doing (now, these days)?' (heard in context)
\end{exe}

These two meanings are also possible when the Egophoric Present is combined with the Progressive, as shown by (\ref{ex:jipAri.kosWBzu}) and (\ref{ex:shangban.kosWBzu}).

\begin{exe}
\ex \label{ex:jipAri.kosWBzu}
\gll alan kɯ ji-pɤri ku-osɯ-βzu. \\
\textsc{anthr} \textsc{erg} \textsc{1pl}.\textsc{poss}-dinner \textsc{prs}-\textsc{prog}-make \\
\glt `Alan is preparing the dinner for us (right now).' (conversation, 15-01-02)
\end{exe}

The Egophoric Present marks actions or states that are temporary. For instance, \forme{mɯ-ku-cʰa-a} `I cannot do it' in (\ref{ex:pAjkhu.mWkuchaa}) (and \ref{ex:mWkuchaa} in §\ref{sec:egophoric.morphology} above) means that the speaker describes a non-permanent situation: she previously was able to do it, and will presumably be soon able to do it again when she has fully recovered.

\begin{exe}
\ex \label{ex:pAjkhu.mWkuchaa}
\gll pɤjkʰu kɯnɤ ɯʑo kɯ ku-oz-nɤma ɕti ma pɤjkʰu mɯ-ku-cʰa-a wo. \\
still also \textsc{3sg} \textsc{erg} \textsc{prs}-\textsc{prog}-do be.\textsc{aff}:\textsc{fact} \textsc{lnk} still \textsc{neg}-\textsc{prs}-can-\textsc{1sg} \textsc{sfp} \\
\glt `Even now, he is still doing (the housework), as I am still unable (to do it, due to an accident).' (conversation, 17-09-01)
\end{exe}

The Factual (\ref{ex:tundzea.maChaa}) or the Sensory are used instead when describing situations that have become permanent.

\begin{exe}
\ex \label{ex:tundzea.maChaa}
\gll  tham a-pɯ-ŋu, tu-ndze-a maka mɤ-cha-a \\
now \textsc{irr}-\textsc{ipfv}-be \textsc{ipfv}-eat[III]-\textsc{1sg} at.all \textsc{neg}-can:\textsc{fact}-\textsc{1sg} \\
\glt `Now, I cannot eat (the fruit of the \textit{Ribes stenocarpum}) at all (anymore, because it is too sour, unlike when the speaker was a child and did not mind about the sourness).' (18-NGolo, 22)
\end{exe}

The Egophoric Present is restricted to present tense. In (\ref{ex:kutanAjo.future}), the form \forme{ku-ta-nɤjo} `I (will) be waiting for you' referring to a future event is rather analyzable as an Imperfective, given the syncretism between these two in the case of this verb.

\begin{exe}
\ex \label{ex:kutanAjo.future}
\gll a-jɤ-tɯ-ɣɯt tɕe, aʑo cʰo [...] tɤɕime nɯ kɯ nɯnɯ kɤntɕʰaʁ ɣɯ ɯ-pɕi nɯtɕu ku-ta-nɤjo. \\
\textsc{irr}-\textsc{pfv}-2-bring \textsc{lnk} \textsc{1sg} \textsc{comit} { } princess \textsc{dem} \textsc{erg} \textsc{dem} town \textsc{gen} \textsc{3sg}.\textsc{poss}-outside \textsc{dem}:\textsc{loc} \textsc{ipfv}-1\fl{}2-wait \\
\glt `When you bring (the bird), the princess and I will be waiting for you outside of the town.' (140507 jinniao-zh, 294)
\end{exe}

The Egophoric Present can also occur in the case of periods of time including the present and the past. For instance, when asked whether she had seen a particular person, who was present at the time in Mbarkham, Tshendzin said the sentence (\ref{ex:mWkotWGa.past.present}) with the Egophoric Present.

\begin{exe}
\ex \label{ex:mWkotWGa.past.present}
\gll mɯ-ku-otɯɣ-a \\
\textsc{neg}-\textsc{prs}-meet-\textsc{1sg} \\
\glt `I have not meeting him (these days).' (conversation 16-03-10)
\end{exe} 

 \subsection{Tripartite contrast} \label{sec:egophoric.tripartite}
The tripartite contrast between Egophoric Present (\ref{ex:kusciti2}), Sensory (\ref{ex:YWscita}, \ref{ex:YWscita2}) and Factual (\ref{ex:sciti}) with the experiencer stative verbs such as \japhug{scit}{be happy} in  declarative sentences with a first person subject can help understanding the semantics of these TAME categories in this context.


\begin{exe}
\ex \label{ex:kusciti2}
\gll ku-scit-i \\
\textsc{prs}-be.happy-\textsc{1pl} \\
\glt `We are happy.' (conversation, in a response to a new years' greeting)
\end{exe}

\begin{exe}
\ex \label{ex:YWscita}
\gll nɯtɕu ɲɯ-scit-a ɕti li tɕe tɕe a-zda ri ɲɯ-pe-nɯ, \\
\textsc{dem}:\textsc{loc} \textsc{sens}-be.happy-\textsc{1sg} be.\textsc{aff}:\textsc{fact} again \textsc{lnk} \textsc{lnk} \textsc{1sg}-companion also \textsc{sens}-be.good-\textsc{pl} \\
\glt `I am very happy there, the people with me are very nice.' (140501 jingli, 149)
\end{exe}

\begin{exe}
\ex \label{ex:YWscita2}
\gll nɯ tɤ-ŋu tɕe, aʑo ndɤre, ʁloŋbutɕhi sɤz ndɤre ɲɯ-scit-a tɕe a-kʰi ɲɯ-ŋgɯ \\
\textsc{dem} \textsc{aor}-be \textsc{lnk} \textsc{1sg} on.the.other.hand elephant \textsc{comp} on.the.other.hand \textsc{sens}-be.happy-\textsc{1sg} \textsc{lnk} \textsc{1sg}.\textsc{poss}-luck \textsc{sens}-be.lucky \\
\glt `Since it is like that, I am happier than the elephant, I am luckier than him.' (140425 shizi puluomixiusi he daxiang, 41)
\end{exe}


\begin{exe}
\ex \label{ex:sciti}
\gll χsɯ-xpa jɤ-tsu-j, nɯsthɯci ʑo scit-i, amɯmi-j  \\
three-year \textsc{aor}-pass-\textsc{1sg} so.much \textsc{emph} be.happy:\textsc{fact}-\textsc{1pl} be.in.good.terms:\textsc{fact}-\textsc{1pl} \\
\glt `We have been together for three years now, we are so happy together.' (Norbzang 2005, 95)
\end{exe}

In (\ref{ex:sciti}), the speakers (humans stranded on an island) include the addressees (rākshasīs in human shape) in the first plural, and state their happiness together as an commonly agreed fact (the first step in a plan to cheat the rākshasīs), hence the use of the Factual. 

In (\ref{ex:YWscita2}), the selection of the Sensory here may be due to the presence of a comparative construction (§\ref{sec:sensory.other}). In (\ref{ex:YWscita}) the choice of the Sensory rather than the Egophoric Present expresses that when thinking about it, the speaker feels that she is happy. The Egophoric Present in (\ref{ex:kusciti2}) entails the continuous conscience of being in a state of happiness, and that this state is temporary.

\section{Modal categories} \label{sec:TAME.modal}
\subsection{Irrealis} \label{sec:irrealis}
 
\subsubsection{Morphology} \label{sec:irrealis.morphology}
The Irrealis has three exponents: a dedicated prefix \forme{a-} in slot -6 (§\ref{sec:outer.prefixal.chain}), a type A preverb in slot -3, and stem III in appropriate forms (§\ref{sec:stem3.form}, §\ref{sec:stem.TAME}), as illustrated by (\ref{ex:atAndze}). A cognate verb form with identical triple exponence in found in Tshobdun \citep{jackson07irrealis}.

With dynamic verbs, the preverb follows the lexically selected orientation of the verb (for instance \forme{tɤ-} \textsc{upwards} in \ref{ex:atAndze}). Stative verbs take the \textsc{downwards} \forme{pɯ-} preverb like the Past Imperfective (§\ref{sec:pst.ifr.ipfv.morphology}) when used with a stative meaning, as in (\ref{ex:apWme}), and the intrinsic orientation (\textsc{westwards} in \ref{ex:anWme}) when occurring with an inchoative meaning.

\begin{exe}
\ex 
\begin{xlist}
\ex \label{ex:atAndze}
\gll a$^{-6}$-tɤ$^{-3}$-ndze \\
\textsc{irr}-\textsc{pfv}-eat[III] \\
\glt `Let it/him/her eat/ if s/he/it eats'
\ex \label{ex:apWme}
\gll a$^{-6}$-pɯ$^{-3}$-me \\
 \textsc{irr}-\textsc{ipfv}-not.exist \\
 \glt `If it does not exist'
\ex \label{ex:anWme}
\gll a$^{-6}$-nɯ$^{-3}$-me \\
 \textsc{irr}-\textsc{pfv}-not.exist \\
 \glt `If it disappears'
\end{xlist}
\end{exe}

Unlike the Imperative (§\ref{sec:imp.morphology}), the Irrealis is compatible with all person configurations.

\subsubsection{Main clauses} \label{sec:irrealis.main}
In main clauses, the Irrealis has four main functions. The first three (wish, jussive and uncertainty) are treated in this section, and the fourth one (delayed imperative) in §\ref{sec:irrealis.delayed.imp}.

First, it can express a wish, often in combination with the predicative noun \japhug{smɯlɤm}{prayer} (§\ref{sec:smWlAm.TAME}; see also in Tshobdun \citealt[804]{jackson07irrealis}), as in (\ref{ex:apWtWtso.smWlAm2}), or with the sentence final particle \forme{kɯ} (§\ref{sec:fsp.interrog}) as in (\ref{ex:anApaa.kW}).

\begin{exe}
\ex \label{ex:apWtWtso.smWlAm2}
\gll pja mɯndʐamɯχtɕɯɣ nɯ-skɤt a-pɯ-tɯ-tso smɯlɤm\\
bird all.kinds \textsc{3pl}.\textsc{poss}-speech \textsc{irr}-\textsc{ipfv}-2-understand prayer\\
\glt `May you understand the speech of all species of birds!' (2003kandZislama, 85)
\end{exe}

\begin{exe}
\ex \label{ex:anApaa.kW}
\gll tɯrme ci a-nɯ-ɤpa-a kɯ \\
human \textsc{indef} \textsc{irr}-\textsc{pfv}-become-\textsc{1sg} \textsc{sfp} \\
\glt `If only I could become a human!' (150819 haidenver-zh, 242)
\end{exe}

While in (\ref{ex:anApaa.kW}) the wish is virtual, example (\ref{ex:apWtWtso.smWlAm}) is a magical formula pronounced by a lama, and the Irrealis has a performative function, conveying to the addressee the ability it describes.


Second, the Irrealis can occur with third person referents with a jussive meaning (see also \citealt[811]{jackson07irrealis} on Tshobdun), expressing either that the speaker allows the subject to perform the action (\ref{ex:athWGi}), a request or an order. Jussive Irrealis clauses can be used in a purposive complementation strategy (§\ref{sec:irrealis.purposive}).

\begin{exe}
\ex \label{ex:athWGi}
\gll a-tʰɯ-ɣi, a-tʰɯ-ɣi \\
\textsc{irr}-\textsc{pfv}:\textsc{downstream}-come \textsc{irr}-\textsc{pfv}:\textsc{downstream}-come \\
\glt `Let him come, let him come (as he wishes).' (2003 Kunbzang, 41)
\end{exe}

In combination with the Autive (§\ref{sec:autobenefactive}), the jussive Irrealis expresses that the speaker does not care about the actions of the third person subject, as in (\ref{ex:apWGnWsata}) (compare with the Autive Imperative, §\ref{sec:imp.autive}).

\begin{exe}
\ex \label{ex:apWGnWsata}
\gll ɯ-ɲɯ́-wɣ-sat-a nɤ a-pɯ́-wɣ-nɯ-sat-a ma mɤ-pʰɣo-tɕi \\
\textsc{qu}-\textsc{sens}-\textsc{inv}-kill-\textsc{1sg} add \textsc{irr}-\textsc{pfv}-\textsc{inv}-\textsc{auto}-kill-\textsc{1sg} \textsc{lnk} \textsc{neg}-flee-\textsc{1du} \\
\glt `If she is to kill me let her kill me, we will not flee.' (2002nyimavodzer, 36)
\end{exe}

With second persons, the Irrealis is found instead of the Imperative to express non-controllable actions. In (\ref{ex:atAtWmna.kArAZi}), compare for instance the non-controllable verb \japhug{mna}{be better} in the Irrealis with the controllable one \japhug{rɤʑi}{stay} in the Imperative.

\begin{exe}
\ex \label{ex:atAtWmna.kArAZi}
\gll pɤjkʰu a-tɤ-tɯ-mna, mɤʑɯ kɤ-rɤʑi \\
still \textsc{irr}-\textsc{pfv}-2-be.better yet \textsc{imp}-stay \\
\glt `(Wait till) you get better, stay a little more.' (said to a convalescent person who wants to leave the place where she is taken care of.'  (12-BzaNsa, 110)
\end{exe}

The negative Irrealis is also more felicitous than the Prohibitive (§\ref{sec:prohibitive}) with non-controllable verbs, such as \japhug{nɯtɕʰomba}{have a cold} in (\ref{ex:amAtAtWnWtChomba}).

\begin{exe}
\ex \label{ex:amAtAtWnWtChomba}
\gll a-mɤ-tɤ-tɯ-nɯtɕʰomba ra ma tɕe nɤ-ɕqʰe ɲɯ-tʰɯ ɲɯ-ŋu wo \\
\textsc{irr}-\textsc{neg}-\textsc{pfv}-2-have.a.cold be.needed:\textsc{fact} \textsc{lnk} \textsc{lnk} \textsc{2sg}.\textsc{poss}-cough \textsc{sens}-be.serious \textsc{sens}-be \textsc{sfp} \\
\glt `Don't catch a cold, otherwise you cough will be become even more serious.' (conversation 17-09-01)
\end{exe}

Third, in interrogative clauses, the Irrealis can express the uncertainty of the speaker on his/her ability to realize the action,  as in (\ref{ex:tChi.atAstuj}).

\begin{exe}
\ex \label{ex:tChi.atAstuj}
\gll andi ki stʰɯci smar kɯ-wxti iʑo kɤ-sɯ-βzɯr tɕʰi a-tɤ-stu-j? \\
west \textsc{dem}.\textsc{prox} so.much river \textsc{sbj}:\textsc{pcp}-be.big \textsc{1pl} \textsc{inf}-\textsc{caus}-move what \textsc{irr}-\textsc{pfv}-do.like-\textsc{1pl} \\
\glt `How can we move such a huge river? (2005 tAwakWcqraR, 210)
\end{exe}

 
\subsubsection{Delayed imperative} \label{sec:irrealis.delayed.imp}
The Irrealis can be used as a delayed or postponed imperative. This function indicates, according to Sun's (\citeyear[809]{jackson07irrealis}) apt description of the same phenomenon in Tshobdun, `the  speaker’s  physical  inaccessibility  as  an  eyewitness,  rather  than  simply  delayed  compliance.'

In (\ref{ex:qhihihi.atAtWti}), the verb in the Irrealis \forme{a-tɤ-tɯ-ti} refers to an action to be realized at a future moment expressed by the temporal clause in the Aorist \forme{jɤ-tɯ-azɣɯt} `when you arrive...' (§\ref{sec:aor.temporal}), when the speaker will not be present to remind the addressee to perform the action.

\begin{exe}
\ex \label{ex:qhihihi.atAtWti}
\gll [kʰa kɤ-mto jɤ-tɯ-azɣɯt] tɕe qʰihihi χsɯ-ŋka a-tɤ-tɯ-ti ra \\
house \textsc{inf}-see \textsc{aor}-2-arrive \textsc{lnk} \textsc{interj} three-word \textsc{irr}-\textsc{pfv}-2-say be.needed:\textsc{fact} \\
\glt `When you arrive at visible distance of the house, say \forme{qʰihihi} three times.' (qachGa 2012, 164-165)
\end{exe}

The contrast between the Irrealis and the Imperative to express delayed command can be illustrated by the following pair of examples from two version of the same story, referring to the same action (the method to pass a dangerous place where a pair of magical boulders crush all people coming between them) but from a different perspective. 

In (\ref{ex:atAtWZGAGAZo.delayed}), a nâga explains the method to the main character (Nyima 'Odzer). The nâga is not coming with him; only Nyima 'Odzer and his horse intend to cross the boulders. Therefore, the Irrealis verb forms \forme{a-kɤ-tɯ-βraʁ} and \forme{a-tɤ-tɯ-ʑɣɤ-ɣɤ-ʑo} are selected, as the nâga will not be present when Nyima 'Odzer will have to realize these actions. 

\begin{exe}
\ex \label{ex:atAtWZGAGAZo.delayed}
\gll nɯ-tɯ-armbat-ndʑi tɕe, nɤ-mbro ɯ-jme zɯ pʰuɲi a-kɤ-tɯ-βraʁ, nɤʑo tɤ-muj stʰɯci a-tɤ-tɯ-ʑɣɤ-ɣɤ-ʑo, nɤ-mbro qale stʰɯci a-nɯ-ʑɣɤ-ɣɤ-mbjom \\
\textsc{aor}:\textsc{west}-2-be.near-\textsc{du} \textsc{lnk} \textsc{2sg}.\textsc{poss}-horse \textsc{3sg}.\textsc{poss}-tail \textsc{emph} potentilla.fruticosa \textsc{irr}-\textsc{pfv}-2-attach \textsc{2sg} \textsc{indef}.\textsc{poss}-feather so.much \textsc{irr}-\textsc{pfv}-2-\textsc{refl}-\textsc{caus}-be.light \textsc{2sg}.\textsc{poss}-horse wind so.much \textsc{irr}-\textsc{pfv}-\textsc{refl}-\textsc{caus}-be.quick \\
\glt `When you approach (the boulders), attach a branch of \textit{Potentilla fruticosa} on the tail of your horse, make yourself as light as a feather, and may your horse be as quick as the wind.' (Smanmi2003.2, 60-61)
\end{exe}

By contrast, in (\ref{ex:tAZGAGAZo.delayed}), it is the horse Rtamchog Rinpoche who explains to Nyima 'Odzer how to cross the boulders. Since these two characters will do the crossing together, the horse uses the Imperative \forme{tɤ-ʑɣɤ-ɣɤ-ʑo} instead of the Irrealis (§\ref{sec:imp.function}).

\begin{exe}
\ex \label{ex:tAZGAGAZo.delayed}
\gll aʑo nɯ, qale jamar ʑo tu-ʑɣɤ-ɣɤ-mbjom-a nɤʑo nɯ tɤ-muj jamar tɤ-ʑɣɤ-ɣɤ-ʑo \\
\textsc{1sg} \textsc{dem} wind about \textsc{emph} \textsc{ipfv}-\textsc{refl}-\textsc{caus}-be.quick-\textsc{1sg} \textsc{2sg} \textsc{dem} \textsc{indef}.\textsc{poss}-feather about \textsc{imp}-\textsc{refl}-\textsc{caus}-be.light \\
\glt `I will make myself as quick as the wind, make yourself as light as a feather.' (28-smAnmi, 117-119)
\end{exe}

In negative forms likewise, the Irrealis is better than the Prohibitive (§\ref{sec:prohib.function}) if the addressee is to refrain from doing an action in the absence of the speaker who issued the order/request/suggestion not to do it, as in (\ref{ex:WskAt.amAkAtWsANAm}), where both assertive and negative Irrealis verbs in delayed imperative function are found.


\begin{exe}
\ex \label{ex:WskAt.amAkAtWsANAm}
\gll  [...] ntsɯ kɤtɯpe ri, maka ɯ-jaʁ a-kɤ-tɯ-ndɤm, a-kɤ-tɯ-sɤtɕitʂi ʑo a-lɤ-tɯ-ɣɯt ma ɯ-skɤt a-mɤ-kɤ-tɯ-sɤŋɤm \\
{ } always tell:\textsc{fact} \textsc{lnk} completely \textsc{3sg}.\textsc{poss}-hand \textsc{irr}-\textsc{pfv}-2-take[III] \textsc{irr}-\textsc{pfv}-2-continue \textsc{emph} \textsc{irr}-\textsc{pfv}:\textsc{upstream}-2-bring \textsc{lnk} \textsc{3sg}.\textsc{poss}-word \textsc{irr}-\textsc{neg}-2-listen[III] \\
\glt `She will not stop saying `...', but don't release your grasp on her hand, and directly bring her (here), and don't listen to her words.' (qachGa 2003, 66)
\end{exe}

\subsubsection{Complement clauses} \label{sec:irrealis.complement.clauses}
According to Sun's (\citeyear[807]{jackson07irrealis}) description of Tshobdun, the Irrealis is required when occurring with matrix verbs expressing desire or intention. In particular, he points out that the verb \forme{səsiʔ} means `think' when the complement clause is in realis mode, and `desire, want' with a complement clause in the Irrealis.

In Japhug, the cognate verb \japhug{sɯso}{think} has the meaning `want' rather with Infinitive or Imperfective complement clauses (§\ref{sec:sWso.complement}), such as (\ref{ex:tundzea.YWsWsama}).

\begin{exe}
\ex \label{ex:tundzea.YWsWsama}
\gll [nɯ wuma ʑo tu-ndze-a] ɲɯ-sɯsam-a \\
\textsc{dem} really \textsc{emph} \textsc{ipfv}-eat[III]-\textsc{1sg} \textsc{sens}-think[III]-\textsc{1sg} \\
\glt `I want to eat it a lot.' (140506 woju guniang-zh, 35)
\end{exe}

Complement clauses in the Irrealis with \japhug{sɯso}{think} as matrix verb are in all cases reported speech, reflecting functions such  hypothetical protasis and apodosis (§\ref{sec:irrealis.conditional}) in (\ref{ex:apWsatnW.YWsWsama}) or jussive (§\ref{sec:irrealis.main}) as in (\ref{ex:atAndze.YAsWso}) and (\ref{ex:atAmna.YWsWsama}). 

\begin{exe}
\ex \label{ex:apWsatnW.YWsWsama}
\gll  χawo ʑo kɯ-dɯ\redp{}dɤn kɯ a-kɤ-nɯtsʰɤβ-nɯ tɕe a-tɤ-tɕʰɯ-nɯ tɕe, a-pɯ-sat-nɯ kɯ ɲɯ-sɯsam-a ri \\
\textsc{interj} \textsc{emph} \textsc{sbj}:\textsc{pcp}-\textsc{emph}\redp{} \textsc{erg} \textsc{irr}-\textsc{pfv}-attack.in.pack-\textsc{pl} \textsc{lnk} \textsc{irr}-\textsc{pfv}-gore-\textsc{pl} \textsc{lnk} \textsc{irr}-\textsc{pfv}-kill-\textsc{pl} \textsc{sfp} \textsc{sens}-think[III]-\textsc{1sg} \textsc{lnk} \\
\glt `I am thinking that if only they attacked in pack and gored (the lion), they would probably kill it.' (20-RmbroN, 65-67)
\end{exe}


\begin{exe}
\ex \label{ex:atAndze.YAsWso}
\gll ``nɯnɯ tɕe kɯ-xtɕɯ\redp{}xtɕi ci tu-tɕat-a tɕe nɯ a-tɤ-ndze" ɲɤ-sɯso \\
\textsc{dem} \textsc{lnk} \textsc{sbj}:\textsc{pcp}-\textsc{emph}\redp{}be.small \textsc{indef} \textsc{ipfv}-take.out-\textsc{1sg} \textsc{lnk} \textsc{dem} \textsc{irr}-\textsc{pfv}-eat[III] \textsc{ifr}-think \\
\glt `He thought: `I will take a few (olives from the jar) so that she (can) eat some.' (140516 yiguan ganlan-zh, 42)
\end{exe}


The used of reported speech with \japhug{sɯso}{think} in (\ref{ex:atAmna.YWsWsama}) is a semi-grammaticalized purposive construction (§\ref{sec:irrealis.purposive}), where the Irrealis is not obligatory: similar purposive clause with other TAME categories are also attested.

\begin{exe}
\ex \label{ex:atAmna.YWsWsama}
\gll  ``a-mi nɯnɯ a-tɤ-mna'' ɲɯ-sɯsam-a tɕe, nɯra ku-z-nɯsman-a ŋu. \\
\textsc{1sg}.\textsc{poss}-leg \textsc{dem} \textsc{irr}-\textsc{pfv}-be.better \textsc{sens}-think[III]-\textsc{1sg} \textsc{lnk} \textsc{dem} \textsc{prs}-\textsc{caus}-treat-\textsc{1sg} be:\textsc{fact} \\
\glt `In order for my leg to get better, I am treating it (with footbaths).' (conversation 2013-11-12)
\end{exe}

The Irrealis in jussive function in a reported speech clause (`I am thinking `may my leg get better'') is semantically close to `want' (`I want my leg to get better'). This may explain how the Irrealis became required with the Tshobdun matrix verb \forme{səsiʔ} in the meaning `want'. 


Like the Imperative (§\ref{sec:imp.compl}), the Irrealis occurs in subject complement clauses with modal auxiliary verbs such as \japhug{ra}{be needed}, \japhug{ntsʰi}{be better} and \japhug{jɤɣ}{be possible}, with a jussive meaning as in (\ref{ex:athWGindZi.ra}) and (\ref{ex:athWGindZi.YWntshi}), or a delayed imperative, as in (\ref{ex:qhihihi.atAtWti}) above (§\ref{sec:irrealis.delayed.imp}).

\begin{exe}
\ex \label{ex:athWGindZi.ra}
\gll a-wɯ cʰo a-ʁi ni cʰɯ-ɣi-ndʑi ra ma  ʑɤni-sti kɤ-rɤʑi mɤ-cʰa-ndʑi tɕe, [a-tʰɯ-ɣi-ndʑi] ra \\
\textsc{1sg}.\textsc{poss}-grandfather \textsc{comit} \textsc{1sg}.\textsc{poss}-younger.sibling \textsc{du} \textsc{ipfv}:\textsc{downstream}-come-\textsc{du} be.needed:\textsc{fact} \textsc{lnk} \textsc{3du}-alone \textsc{inf}-stay \textsc{neg}-can:\textsc{fact}-\textsc{du} \textsc{lnk} \textsc{irr}-\textsc{pfv}:\textsc{downstream}-come-\textsc{du} be.needed:\textsc{fact} \\
\glt `My grandfather and younger brother have to come (with me), as they cannot stay on their own, let them come.' (2011-05-nyima 208-209)
\end{exe}

\begin{exe}
\ex \label{ex:athWGindZi.YWntshi}
\gll ki ɲɯ-sɤ-ɕke tɕe, [a-nɯ-ɤfɕu] ɲɯ-ntsʰi \\
\textsc{dem}.\textsc{prox} \textsc{sens}-\textsc{prop}-burn \textsc{lnk} \textsc{irr}-\textsc{pfv}-cool.down \textsc{sens}-be.better \\
\glt `This (tea) is too hot, let it cool down.' (elicited)
\end{exe}

\subsubsection{Conditional clauses} \label{sec:irrealis.conditional}
In the protasis, the Irrealis competes with initial reduplication (§\ref{sec:redp.protasis}), Prohibitive (§\ref{sec:prohib.function}) and Interrogative (§\ref{sec:interrogative.W.function}). When the apodosis is in the Factual Non-Past, the Irrealis protasis expresses a condition whose probability of being realized may not be high, but which, if it is verified, almost certainly brings the outcome expressed in the apodosis (at least in the speaker's opinion), as in (\ref{ex:ajACe.tCetha}).

\begin{exe}
\ex \label{ex:ajACe.tCetha}
\gll nɯ a-jɤ-ɕe tɕe tɕetʰa kɤ-zɣɯt mɤ-cʰa tɕe si ɕti tɕe \\
\textsc{dem} \textsc{irr}-\textsc{pfv}-go \textsc{lnk} soon \textsc{inf}-arrive \textsc{neg}-can:\textsc{fact} \textsc{lnk} die:\textsc{fact} be.\textsc{aff}:\textsc{fact} \textsc{lnk} \\
\glt `Would he go there, he would not be able to reach (his goal), and would die.' (28-smAnmi, 60)
\end{exe}

The combination of a protasis in the Irrealis and an apodosis in the Past Imperfective is used to express counterfactual meaning (§\ref{sec:pst.ifr.ipfv.apodosis}, §\ref{sec:counterfactual}), as in (\ref{ex:apWtu.mWpWnANkWNetCi}).

\begin{exe}
\ex \label{ex:apWtu.mWpWnANkWNetCi}
\gll tɕiʑɤɣ nɯ kɯ-fse ci a-pɯ-tu ndɤre, kɯra kɯ-fse mɯ-pɯ-nɤŋkɯŋke-tɕi wo \\
\textsc{1du}:\textsc{gen} \textsc{dem} \textsc{sbj}:\textsc{pcp}-be.like \textsc{indef} \textsc{irr}-\textsc{ipfv}-exist \textsc{lnk} \textsc{dem}.\textsc{prox}:\textsc{pl} \textsc{sbj}:\textsc{pcp}-be.like \textsc{neg}-\textsc{pst}.\textsc{ipfv}-\textsc{distr}:walk-\textsc{1du} \textsc{sfp} \\
\glt `If we had (so many cattle and fields) like that, we would not be wandering around like that.' (2005 Kunbzang, 188)
\end{exe}


\subsubsection{Purposive} \label{sec:irrealis.purposive}
The Irrealis occurs in purposive complementation strategies. Two constructions are attested. First, as in (\ref{ex:nAkArme.atAzbaR}), the Irrealis clauses expresses the purpose of a action referred to in another clause.

\begin{exe}
\ex \label{ex:nAkArme.atAzbaR}
\gll kʰɤxtu ɕɯ-nɤʁaʁ-tɕi tɕe, [nɤ-kɤrme a-tɤ-zbaʁ] \\
rooftop \textsc{tral}-have.fun:\textsc{fact}-\textsc{1du} \textsc{lnk} \textsc{2sg}.\textsc{poss}-hair \textsc{irr}-\textsc{pfv}-be.dry \\
\glt `Let us go to the rooftop platform to rest, so that your hair can dry (after bathing).' (2002 qaCpa, 287)
\end{exe}

Second, with a similative verb such as \japhug{fse}{be like} or \japhug{stu}{do like} and interrogative pronoun as in (\ref{ex:atAfse.YWphAn}), the Irrealis clause rather corresponds to the action needed to realize the purpose, which is indicated by a coordinated clause \forme{ɲɯ-pʰɤn} `(so that) it is efficient/it works/it solves it'. 

\begin{exe}
\ex \label{ex:atAfse.YWphAn}
\gll [nɯ tɕʰi a-tɤ-fse] tɕe ɲɯ-pʰɤn \\
\textsc{dem} what \textsc{irr}-\textsc{pfv}-be.like \textsc{lnk} \textsc{sens}-be.efficient \\
\glt `How (should he do) to solve (this problem)?' (Divination 2005, 45)
\end{exe}
 
\subsubsection{Periphrastic Irrealis} \label{sec:irrealis.periphrastic}

The Periphrastic Irrealis combines the Irrealis copula \forme{a-pɯ-ŋu} with one or a chain of several verbs in the Imperfective, as in (\ref{ex:pjWwGnWXtCi.apWNu}) (see also §\ref{sec:ipfv.periphrastic.TAME}). 

\begin{exe}
\ex \label{ex:pjWwGnWXtCi.apWNu}
\gll  [ɯ-ŋgɯ pjɯ-kɯ-ɕe tɕe tɯ-ku ci pjɯ́-wɣ-nɯ-sɤɕɤt, pjɯ́-wɣ-nɯ-χtɕi] a-pɯ-ŋu ndɤre, wuma ʑo sɤ-scit tʰaŋ nɤ! \\
\textsc{3sg}.\textsc{poss}-inside \textsc{ipfv}:\textsc{down}-\textsc{genr}:S/O-go \textsc{lnk} \textsc{genr}.\textsc{poss}-head a.little \textsc{ipfv}-\textsc{inv}-\textsc{auto}-comb \textsc{ipfv}-\textsc{inv}-\textsc{auto}-wash \textsc{irr}-\textsc{ipfv}-be \textsc{lnk} really \textsc{emph} \textsc{prop}-be.happy:\textsc{fact} \textsc{sfp} \textsc{sfp} \\
\glt `If one were to dive (in the water), comb and wash one's hair, it would be very nice!' (140515 congming de wusui xiaohai-zh, 25-27)
\end{exe}

It replaces the Imperfective Irrealis (§\ref{sec:irrealis.morphology}) in the case of telic verbs (see §\ref{sec:pst.ifr.ipfv.periphrastic}).

\subsection{Imperative} \label{sec:imperative} 
\subsubsection{Morphology}  \label{sec:imp.morphology}
The Imperative is built by combining type A preverbs with stem I or stem III depending on transitivity and number (§\ref{sec:stem.TAME}). It only has second person subject forms, which are however never marked with the second person \forme{tɯ-} prefix, unlike in other TAME categories (§\ref{sec:intr.23}). In addition, transitive verbs can only take a third person object; 2\fl{}1 configurations cannot be expressed with the Imperative, and the Imperfective in hortative function is used instead (§\ref{sec:ipfv.hortative}).

The Imperative can occur with an overt second person pronoun referring to the subject (see examples 
\ref{ex:tAnWndAm.je}, §\ref{sec:imp.autive} and \ref{ex:tAnWndAm.jAG}, §\ref{sec:imp.compl}).

\tabref{tab:imp.paradigms} illustrates the Imperative paradigms of \japhug{ndza}{eat}, a transitive verb with stem III alternation (§\ref{sec:stem3}), \japhug{amdzɯ}{sit}, an intransitive contracting verb (§\ref{sec:contraction}), and \japhug{ɕe}{go}, a verb with stem II alternation (§\ref{sec:stem2})

\begin{table}
\caption{Examples of Imperative paradigms} \label{tab:imp.paradigms}
\begin{tabular}{llll}
\lsptoprule
Person & \japhug{ndza}{eat} & \japhug{amdzɯ}{sit} & \japhug{ɕe}{go}     \\
\midrule
\textsc{2sg}(\fl{}3) & \forme{tɤ-\rouge{ndze}} & \forme{kɤ-ɤmdzɯ}& \forme{jɤ-ɕe} \\
\textsc{2du}(\fl{}3) & \forme{tɤ-ndza-ndʑi}& \forme{kɤ-ɤmdzɯ-ndʑi} & \forme{jɤ-ɕe-ndʑi} \\
\textsc{2pl}(\fl{}3) & \forme{tɤ-ndza-nɯ} & \forme{kɤ-ɤmdzɯ-nɯ}& \forme{jɤ-ɕe-nɯ} \\
\lspbottomrule
\end{tabular}
\end{table}

For intransitive verbs which are neither contracting nor have stem II alternation, the Imperative forms are identical to the third person Aorist forms (§\ref{sec:ambiguity.preverb}).

The Imperative lacks negative forms: the Prohibitive (§\ref{sec:prohibitive}) or negative Irrealis (§\ref{sec:irrealis.main}) are used instead.
 
\subsubsection{Main clauses} \label{sec:imp.function}
The Imperative expresses actions that the speaker wishes the addressee to realize. It is appropriate for blunt orders (§\ref{ex:jACe}), requests (\ref{ex:jAlAt.je}), and also polite invitations (§\ref{ex:tAndzandZi}).

\begin{exe}
\ex \label{ex:jACe}
\gll jɤ-ɕe! \\
\textsc{imp}-go \\
\glt `Go away!' (many examples)
\end{exe}

\begin{exe}
\ex \label{ex:jAlAt.je}
\gll toʁde tɕe jɤ-lɤt je ma a-<dianhua> ɯ-kɯ-lɤt ɣɤʑu \\
 a.moment \textsc{lnk} \textsc{imp}-release \textsc{sfp} \textsc{lnk} \textsc{1sg}.\textsc{poss}-phone \textsc{3sg}.\textsc{poss}-\textsc{sbj}:\textsc{pcp}-release exist:\textsc{sens} \\
\glt `Call me in a moment, there is someone calling me on the phone!' (conversation, 22-08-2018)
\end{exe}

\begin{exe}
\ex \label{ex:tAndzandZi}
\gll <guazi> tɤ-ndza-ndʑi \\
melon.seed \textsc{imp}-eat-\textsc{du} \\
\glt `Eat some melon seeds!' (conversation 14-05-10)
\end{exe}

The Imperative is not restricted to immediate commands/requests. In (\ref{ex:thWsAtCAt.lArACi}), the actions referred to by the Imperative verbs \forme{lɤ-rɤɕi} `pull it' and \forme{tʰɯ-sɤtɕɤt} `add firewood' are to be realized (in the presence of the speaker) at two points of reference in the future indicated by temporal clauses in the Aorist (§\ref{sec:aor.temporal}).

\begin{exe}
\ex \label{ex:thWsAtCAt.lArACi}
\gll ``a-wi smi tʰɯ-sɤtɕɤt" tɤ-tɯt-a tɕe lɤ-rɤɕi, ``a-wi smi lɤ-rɤɕi" tɤ-tɯt-a tɕe tʰɯ-sɤtɕɤt ra \\
\textsc{1sg}.\textsc{poss}-grandmother fire \textsc{imp}-add.firewood \textsc{aor}-say[II]-\textsc{1sg} \textsc{lnk} \textsc{imp}:\textsc{upstream}-pull \textsc{1sg}.\textsc{poss}-grandmother fire \textsc{imp}:\textsc{upstream}-pull  \textsc{aor}-say[II]-\textsc{1sg} \textsc{lnk} \textsc{imp}-add.firewood be.needed:\textsc{fact} \\
\glt `When I say `grandmother, add firewood', remove the firewood, and when I say `grandmother, remove the firewood', add firewood.' (2005 Kunbzang, 370-371) 
\end{exe}

As in Tshobdun (\citealt[809]{jackson07irrealis}), the Irrealis is found instead of the Imperative to express actions to be performed at a point in the future in the absence of the speaker (§\ref{sec:irrealis.delayed.imp}). 

The Imperative does not commonly occurs for non-controllable verbs, especially stative verbs. However, this constraint is more pragmatic than morphosyntactic, and in some contexts, even a verb like \japhug{mbro}{be high} can be used in the Imperative, as in (\ref{ex:tWrtsAG.tAmbro}). In this type of example, the Imperative \forme{tɤ-mbro} is identical to the \textsc{3sg} Aorist.\footnote{In principle, the quotation in (\ref{ex:tWrtsAG.tAmbro}) could also mean `It grew by one node per day', but Tshendzin is positive that an Imperative was meant here. }

\begin{exe}
\ex \label{ex:tWrtsAG.tAmbro}
\gll tɕe nɯnɯ ɕɤr tɕe tu-mbri tɕe ``coʁcoʁcóʁcoʁcoʁ coʁcoʁcóʁcoʁcoʁ" tu-ti ŋu tɕe, ``tɯ-sŋi tɯ-rtsɤɣ tɤ-mbro, tɯ-sŋi tɯ-rtsɤɣ tɤ-mbro" tu-ti ŋu tu-ti-nɯ ŋgrɤl. \\
\textsc{lnk} \textsc{dem} night \textsc{loc} \textsc{ipfv}-sing \textsc{lnk} \textsc{idph}(X):cry  \textsc{idph}(X):cry \textsc{ipfv}-say be:\textsc{fact} \textsc{lnk} one-day \textsc{one}-node \textsc{imp}-be.high one-day \textsc{one}-node \textsc{imp}-be.high \textsc{ipfv}-say be:\textsc{fact} \textsc{ipfv}-say-\textsc{pl} be.usually.the.case:\textsc{fact}  \\
\glt `(In june, when crops are about to grow), (the \forme{tacoʁcoʁ} bird)$_i$ sings in the night, making the sound \forme{coʁcoʁcóʁcoʁcoʁ}, people say that it$_i$  tells (the crops) `Grow by one node everyday!'.' (23-scuz, 113-114)
\end{exe}

\subsubsection{Imperative and autive} \label{sec:imp.autive}
With the Autive prefix (§\ref{sec:autobenefactive}), the Imperative has two distinct and nearly opposite meanings. 

First, it can indicate a mild suggestion or a request for a favour (§\ref{sec:autoben.proper}), as in (\ref{ex:tAnWndAm.je}).

\begin{exe}
\ex \label{ex:tAnWndAm.je}
\gll laʁjɯɣ nɤʑo tɤ-nɯ-ndɤm je tɕe, aʑo jɤɣɤt ci lu-ɕe-a nɤ \\
staff \textsc{2sg} \textsc{imp}-\textsc{auto}-take[III] \textsc{sfp} \textsc{lnk} \textsc{1sg} toilet \textsc{indef} \textsc{ipfv}:\textsc{upstream}-go-\textsc{1sg} \textsc{sfp} \\
\glt `Take the staff, I am going to the toilets.' (2005 khu, 13)
\end{exe}

Second, it is also used in a mocking way to express defiance (§\ref{sec:autoben.spontaneous}), as in (\ref{ex:nWnWnAre}) and (\ref{ex:aqe.tAnWndze}).

\begin{exe}
\ex  \label{ex:nWnWnAre}
\gll  nɯ-nɯ-nɤre ma nɤʑo qaɕpa nɤ-rʑaβ nɤ-kɯ-mbi kɯ-tu me   \\
\textsc{imp}-\textsc{auto}-laugh \textsc{lnk} \textsc{2sg} frog \textsc{2sg}.\textsc{poss}-wife \textsc{2sg}.\textsc{poss}-\textsc{sbj}:\textsc{pcp}-give \textsc{sbj}:\textsc{pcp}-exist not.exist:\textsc{fact} \\
\glt `Laugh as you wish, nobody will give you a wife, you frog.'   (2002 qaCpa, 176)
\end{exe} 

\begin{exe}
\ex  \label{ex:aqe.tAnWndze}
\gll kɤnɤβdi je a-wɯ tɯɟo, a-qe kɯ-sɤ-ɕkɯ\redp{}ɕke ci tɤ-nɯ-ndze \\
farewell \textsc{sfp} \textsc{1sg}.\textsc{poss}-grandfather demon \textsc{1sg}.\textsc{poss}-shit \textsc{sbj}:\textsc{pcp}-\textsc{prop}-\textsc{emph}\redp{}burn \textsc{indef} \textsc{imp}-\textsc{auto}-eat[III] \\
\glt `Farewell, old demon, eat my hot shit!' (2005 tWJo, 45)
\end{exe}

\subsubsection{Complement clauses} \label{sec:imp.compl}
The Imperative is commonly used in complement clauses of modal verbs. The combination the modal auxiliary \japhug{jɤɣ}{be possible} with an Imperative complement clause expresses that the speaker politely allows the addressee to undertake an action that the addressee himself intends to do, as in (\ref{ex:tAnWndAm.jAG}) and (\ref{ex:tAthe.jAG}).

\begin{exe}
\ex  \label{ex:tAnWndAm.jAG}
\gll [nɤʑo tɤ-nɯ-ndɤm] jɤɣ \\
\textsc{2sg} \textsc{imp}-\textsc{auto}-take[III] be.possible:\textsc{fact} \\
\glt `Please take it.' (divination, 84)
\end{exe}

\begin{exe}
\ex  \label{ex:tAthe.jAG}
\gll nɤ-kɤ-tʰu ɯ-ɣɤʑu nɤ, [tɤ-tʰe] jɤɣ \\
\textsc{2sg}.\textsc{poss}-\textsc{obj}:\textsc{pcp}-ask \textsc{qu}-exist:\textsc{sens} \textsc{add} \textsc{imp}-ask[III] be.possible:\textsc{fact} \\
\glt `If you have a question, please ask it.' (conversation, 14-11-08)
\end{exe}

The modal verb \japhug{ra}{be needed} with Imperative is used in requests (\ref{ex:CtAthe}) and also blunt orders with death treats(\ref{ex:CtAre.ra.CWkAru}).

\begin{exe}
\ex \label{ex:CtAthe} 
\gll ɕ-tɤ-tʰe ra \\
\textsc{tral}-\textsc{imp}-ask[III] be.needed:\textsc{fact} \\
\glt `Go and ask him about it.' (divination, 8)
\end{exe}

\begin{exe}
\ex \label{ex:CtAre.ra.CWkAru}
\gll  ɕ-tɤ-re ra ma ɕɯ-kɤ-ru mɯ\redp{}mɤ-pɯ-tɯ-cʰa ŋu nɤ nɤ-srɤm nɤ-sroʁ lɤt-i \\
\textsc{tral}-\textsc{imp}-bring[III] be.needed:\textsc{fact} \textsc{lnk} \textsc{tral}-\textsc{inf}-bring \textsc{cond}\redp{}\textsc{neg}-\textsc{aor}-2-can be:\textsc{fact} \textsc{lnk} \textsc{1sg}.\textsc{poss}-root \textsc{1sg}.\textsc{poss}-life throw:\textsc{fact}-\textsc{1pl} \\ 
\glt `Go and bring it here; if you do not succeed in going and bringing it here, we will destroy your root and your life.' (Norbzang, 10)
\end{exe}

%\japhug{ntsʰi}{have better}

\subsubsection{Serial verb constructions} \label{sec:imp.SVC}
Two verbs in the Imperative can be used in a Serial Verb Construction (§\ref{sec:svc.manner}), the first verb conveying the manner in which the action is performed, as in (\ref{ex:tAmbGom.tACe}).

\begin{exe}
\ex  \label{ex:tAmbGom.tACe}
\gll tɤ-mbɣom ʑo tɤ-ɕe ra\\
\textsc{imp}-be.in.a.hurry \textsc{emph} \textsc{imp}:\textsc{up}-go be.needed:\textsc{fact} \\
\glt `Hurry up and go upstairs!' (160706 poucet6, 6)
\end{exe}

The first verb can be in the prohibitive (§\ref{sec:prohibitive}), a construction meaning `do $V_2$ without doing $V_1$' (\ref{ex:manWtWBde.tACe}).

\begin{exe}
\ex  \label{ex:manWtWBde.tACe}
\gll kɯki tʂu ki ma-nɯ-tɯ-βde ʑo jɤ-ɕe \\
\textsc{dem}.\textsc{prox} path \textsc{dem}.\textsc{prox} \textsc{neg}-\textsc{imp}-2-throw \textsc{emph} \textsc{imp}-go \\
\glt `Go along this way without leaving it!' (140507 jinniao-zh, 212)
\end{exe}

\subsection{Prohibitive} \label{sec:prohibitive}
 
\subsubsection{Morphology} \label{sec:prohibitive.morpho}
The Prohibitive is built by combining the dedicated negative prefix \forme{ma-} (§\ref{sec:neg.allomorphs}), a type A preverb and the stem III of the verb when appropriate (§\ref{sec:stem3.form}, §\ref{sec:stem.TAME}). In second person subject forms, unlike the Imperative (§\ref{sec:imp.morphology}), the second person prefix obligatorily occurs, as in \forme{ma-nɯ-tɯ-te} `don't put him/her/it' (\ref{ex:manWtWte}).

\begin{exe}
\ex  \label{ex:manWtWte}
\gll nɯtɕu ma-nɯ-tɯ-te ma tɤ-pɤtso ra nɯ-sta ɕti \\
\textsc{dem}:\textsc{loc} \textsc{neg}-\textsc{imp}-2-put[III] \textsc{lnk} \textsc{indef}.\textsc{poss}-child \textsc{pl} \textsc{3pl}.\textsc{poss}-place be.\textsc{aff}:\textsc{fact} \\
\glt `Don't put her there, it is the place of the children.' 
\end{exe}


Unlike the Imperative (§\ref{sec:imp.morphology}), the prohibitive has no constraints on person. It can occurs in 2\fl{}1 configurations (\ref{ex:matAkWndzaa}), but also with a first person intransitive (\ref{ex:Za.matAnWnatCi}) or transitive subject (\ref{ex:khramba.matABzea}) and also in very rare cases in the third person (\ref{ex:manAtCaR.ra}).

\begin{exe}
\ex  \label{ex:matAkWndzaa}
\gll ma-tɤ-kɯ-ndza-a tɕetʰa nɤ-χpi pjɯ-fɕat-a \\
\textsc{neg}-\textsc{imp}-2\fl{}1-eat-\textsc{1sg} later \textsc{2sg}.\textsc{poss}-story \textsc{ipfv}-tell-\textsc{1sg} \\
\glt `Don't eat me, and I will tell you a story.' (tWJo 2012, 62)
\end{exe}

\begin{exe}
\ex  \label{ex:Za.matAnWnatCi}
\gll ʑa ma-tɤ-nɯna-tɕi qʰe \\
soon \textsc{neg}-\textsc{imp}-rest-\textsc{1du} \textsc{lnk} \\
\glt `(In order to catch up with the wasted time), we will not stop (working) early (today).' (conversation 14-05-10)
\end{exe}

\begin{exe}
\ex  \label{ex:khramba.matABzea}
\gll nɯ kʰramba ma-tɤ-βze-a ra ma \\
\textsc{dem} lie  \textsc{neg}-\textsc{imp}-make[III]-\textsc{1sg} be.needed:\textsc{fact} \textsc{sfp} \\
\glt `I should not tell lies.' (27-kikakCi, 222)
\end{exe}

\begin{exe}
\ex  \label{ex:manAtCaR.ra}
\gll tɕe tɯ-mdʑu ɯ-taʁ ma-nɯ-ɤtɕaʁ ra ma tɯ-mdʑu tu-sɤzoŋzoŋ ʑo qʰe cʰɯ-nɯɣmbɤβ ɕti. \\
\textsc{lnk} \textsc{genr}.\textsc{poss}-tongue \textsc{3sg}.\textsc{poss}-on \textsc{neg}-\textsc{imp}-stain be.needed:\textsc{fact} \textsc{lnk} \textsc{genr}.\textsc{poss}-tongue \textsc{ipfv}-make.numb \textsc{emph} \textsc{lnk} \textsc{ipfv}-swell be.\textsc{aff}:\textsc{fact} \\
\glt `(The \textit{Arisaema consanguineum}) should not get on one's tongue$_i$, as it makes the tongue$_i$ numb, and it$_i$ swells.' (14-sWNgWJu, 145)
\end{exe}

The preverb is optional in second person prohibitive forms. For instance with \japhug{ti}{say}, \forme{ma-tɯ-ti} (\textsc{neg}:\textsc{imp}-2-say) and \forme{ma-tɤ-tɯ-ti} (\textsc{neg}-\textsc{imp}-2-say) both occur in free variation.

Despite the clear morphological differences between the Prohibitive and the Imperative, the type A preverb is glossed as \textsc{imp} and the negation as \textsc{neg} (to avoid a redundant gloss \textsc{prohib} on both prefixes). When the preverb is elided, the negative \forme{ma-} is glossed as \textsc{neg}:\textsc{imp}.
 
\subsubsection{Functions} \label{sec:prohib.function}
The Prohibitive with a second person subject is essentially the negative counterpart of the Imperative, with exactly the same range of functions (§\ref{sec:imp.function}), including orders,  requests and polite suggestions, such as the polite expression \forme{ma-tɤ-tɯ-raʁle} (\ref{ex:matAtWrARle}), which corresponds to Chinese \ch{不用客气}{búyòngkèqì}{you'r welcome}.

\begin{exe}
\ex  \label{ex:matAtWrARle}
\gll ma-tɤ-tɯ-raʁle, tɤ-ʑɣɤ-ɕɯ-fka je\\
\textsc{neg}-\textsc{imp}-2-be.polite \textsc{imp}-\textsc{refl}-\textsc{caus}-be.full \textsc{sfp} \\
\glt `Please eat to your full!' (heard in context)
\end{exe}

The prohibition can refer to an action to happen in the future, as in (\ref{ex:taritCi.matWti}), where the moment when the action is to be avoided is indicated by the temporal clause in the Aorist (§\ref{sec:aor.temporal}).

\begin{exe}
\ex  \label{ex:taritCi.matWti}
\gll tɕetu tɤ-ari-tɕi tɕe, ɯ-tɯ-ɣɤndʐo saχaʁ ʑo ri, ``ɯtɕʰɯtɕʰɯ" ma-tɯ-ti \\
up.there \textsc{aor}:\textsc{up}-go[II]-\textsc{1du} \textsc{lnk} \textsc{3sg}.\textsc{poss}-\textsc{nmlz}:\textsc{deg}-be.cold be.extremely:\textsc{fact} \textsc{emph} \textsc{lnk} \textsc{interj} \textsc{neg}:\textsc{imp}-2-say \\
\glt `When we go up there (in the sky, near the moon), it will be extremely cold, but then don't say `brbr'.' (07-deluge-64)
\end{exe}

The negative Irrealis can also be used as a delayed prohibitive as in (\ref{ex:tatWt.amAtAtWti}) (§\ref{sec:irrealis.delayed.imp}), contrasting with the Prohibitive in the same way as the Irrealis in delayed imperative function contrasts with the Imperative (§\ref{sec:imp.function}): the Prohibitive and Imperative imply that the speaker will be present when the action is to be realized or avoided (as in \ref{ex:taritCi.matWti}), while the Irrealis occurs when the speaker will not be present. 

\begin{exe}
\ex  \label{ex:tatWt.amAtAtWti}
\gll iɕqʰa rɯdaʁ ra kɯ ta-tɯt nɯra, tɯrme ɯ-ɕki a-mɤ-tɤ-tɯ-ti ma \\
the.aforementioned animal \textsc{pl} \textsc{erg} \textsc{aor}:3\flobv{}-say[II] \textsc{dem}:\textsc{pl} people \textsc{3sg}.\textsc{poss}-\textsc{dat} \textsc{irr}-\textsc{neg}-\textsc{pfv}-2-say \textsc{lnk} \\
\glt `You will have to avoid telling human what the animal say.' (150902 hailibu-zh, 83)
\end{exe}

Like the Imperative (§\ref{sec:imp.compl}), the prohibitive also occurs in subject complement clauses with auxiliaries such as \japhug{ra}{be needed} (\ref{ex:khramba.matABzea}, §\ref{sec:prohibitive.morpho}).


A construction with the same verb occurring in the Imperfective followed by its Prohibitive form in the first person, with the alternative interrogative particle \forme{ɕi} (§\ref{sec:fsp.interrog}) in between, is used to express hesitation between two possibilities (\ref{ex:kuCea.makACea}).

\begin{exe}
\ex  \label{ex:kuCea.makACea}
\gll ku-ɕe-a ɕi ma-kɤ-ɕe-a kɯ \\
\textsc{ipfv}:\textsc{east}-go-\textsc{1sg} \textsc{sfp} \textsc{neg}-\textsc{imp}-go-\textsc{1sg} \textsc{sfp} \\
\glt `(I wonder) whether to go or not.' (elicited)
\end{exe}

The Prohibitive is also used in manner clauses meaning `without doing $X$', sharing their subject (and also possibly object) with another verb in the Imperative or the Imperfective, as in (\ref{ex:mapWwGsat}) (see also \ref{ex:manWtWBde.tACe}, §\ref{sec:imp.SVC}).

\begin{exe}
\ex  \label{ex:mapWwGsat}
\gll  sɤtɕʰa kɯ-ɤrqʰi zɯ pjɯ́-wɣ-lɤt ma-pɯ́-wɣ-sat ra \\
place \textsc{sbj}:\textsc{pcp}-be.far \textsc{loc} \textsc{ipfv}:\textsc{down}-\textsc{inv}-release \textsc{neg}-\textsc{imp}-\textsc{inv}-kill be.needed:\textsc{fact} \\
\glt `One has to take it far away (from the house) without killing it.' (2010-11, 11)
\end{exe}

The Prohibitive is also found in the protasis of conditionals (§\ref{sec:conditionals}), as in (\ref{ex:matAwGnWBdaR}). This type of `warning' conditional construction expresses a possible undesirable result occurring if the subject fails to perform the action designated by the verb in the Prohibitive.

\begin{exe}
\ex  \label{ex:matAwGnWBdaR}
\gll  tɤ-mtʰɯm kɯnɤ kɤ-kɤ-sqa nɯra ʑatsa ma-tɤ́-wɣ-nɯβdaʁ qʰe, ɯ-taʁ ri kɯ-wɣrum ku-te \\
\textsc{indef}.\textsc{poss}-meat also \textsc{aor}-\textsc{obj}:\textsc{pcp}-cook \textsc{dem}:\textsc{pl} soon \textsc{neg}-\textsc{imp}-\textsc{inv}-take.care \textsc{lnk} \textsc{3sg}.\textsc{poss}-top \textsc{loc} \textsc{sbj}:\textsc{pcp}-be.white \textsc{ipfv}-put[III] \\
\glt `In the case of meat also, if one fails to take care of cooked (meat) in time, white stuff will appear on it.' (20-sWrna,60)
\end{exe}
 
Finally, it can also be used in a counterfactual construction as in (\ref{ex:matAkWsWCqraRa}).

\begin{exe}
\ex  \label{ex:matAkWsWCqraRa}
\gll nɯ sɤznɤ, nɤʑo kɯ ma-tɤ-kɯ-sɯ-ɕqraʁ-a kɯ pjɤ-mna! \\
\textsc{dem} \textsc{comp} \textsc{2sg} \textsc{erg} \textsc{neg}-\textsc{imp}-2\fl{}1-\textsc{caus}-be.intelligent-\textsc{1sg} \textsc{erg} \textsc{pst}.\textsc{ifr}-be.better \\
\glt `It would have been better if you had not made me smart!' (160711 riquet8-v2, 18)
\end{exe}

\subsection{Dubitative} \label{sec:dubitative}
 The Dubitative \forme{ku-} is formally identical to the B type \textsc{eastwards} preverbs, which also markes the Imperfective and the Egophoric Present (§\ref{sec:egophoric}). It always occurs with the Autive \forme{-nɯ-} prefix (§\ref{sec:autobenefactive}  and with the polar question \forme{ɕi} particule (§\ref{sec:fsp.interrog}, see example \ref{ex:kunWZru.Ci.kWma}), the interrogative \forme{kɯ} (§\ref{sec:fsp.interrog}, \ref{ex:CW.ci.kunWNu.kW}) or the alternative polar question construction (combining a positive followed by the equivalent negative verb form as in \ref{ex:kunWphAn.mWkunWphAn}).
 
 \begin{exe}
\ex \label{ex:kunWZru.Ci.kWma}
 \gll  tɕe lu-kɤ-nɯ-ji nɯ kɯ ʑru tu-ti-nɯ ɲɯ-ŋu tɕe mɤ-xsi. ku-nnɯ-ʑru ɕi kɯma. \\
 \textsc{lnk} \textsc{ipfv}-\textsc{obj}:\textsc{pcp}-\textsc{auto}-plant \textsc{dem} \textsc{erg} be.strong:\textsc{fact} \textsc{ipfv}-say-\textsc{pl} \textsc{sens}-be \textsc{lnk} \textsc{neg}-\textsc{genr}:know:\textsc{fact}  \textsc{dub}-\textsc{auto}-be.strong \textsc{qu} \textsc{sfp} \\
 \glt `The cultivated (variety of Angelica) is better (than the wild one), they say, I don't know, maybe it is better.' (17-ndZWnW, 34)
 \end{exe}
 

\begin{exe}
\ex \label{ex:kunWphAn.mWkunWphAn}
 \gll  ku-nɯ-pʰɤn mɯ-ku-nɯ-pʰɤn mɤ-xsi ma \\
 \textsc{dub}-\textsc{auto}-be.efficient \textsc{neg}-\textsc{dub}-\textsc{auto}-be.efficient \textsc{neg}-\textsc{genr}:know:\textsc{fact} \textsc{sfp} \\
\glt `I don't know whether it efficient or not (as medicine).' (19-GzW, 108)
  \end{exe}

In addition, dubitative verb forms are followed either by the sentence final particles \forme{kɯma} or \forme{kɯɣe} (§\ref{sec:fsp.interrog}) as in (\ref{ex:kunWZru.Ci.kWma}) or a verb form such as \japhug{mɤ-xsi}{one does not know} (§ \ref{ex:kunWphAn.mWkunWphAn}).

The dubitative is mainly used to express doubts while reporting opinions from other people (as in \ref{ex:kunWZru.Ci.kWma} and \ref{ex:kunWphAn.mWkunWphAn}), but with the interrogative \forme{kɯ} as in (\ref{ex:CW.ci.kunWNu.kW}), its meaning is rather that of emphasis on the fact that the speaker has no clue about the answer to the question (as in French \textit{donc...bien} in `\textit{Qui donc cela peut-il bien être?}').

\begin{exe}
\ex \label{ex:CW.ci.kunWNu.kW}
 \gll wo, nɯ ɕɯ ci ku-nɯ-ŋu kɯ?  \\
 \textsc{interj} \textsc{dem} who \textsc{indef} \textsc{dub}-\textsc{auto}-be \textsc{qu} \\
\glt `Who on earth is it (who does all) that?' (2014-kWLAG, 619)
 \end{exe}
 
\section{Past categories}   \label{sec:TAME.pst}
The Aorist, Past Imperfective, Inferential Perfective and Inferential Imperfective all strictly express past tense events or states when occurring in main clauses. In addition, they all take the past transitive \forme{-t} suffix (§\ref{sec:other.TAME}) in \textsc{1sg}\fl{}3 and \textsc{2sg}\fl{}3 forms of open stem verbs (see §\ref{sec:aor.morphology}, §\ref{sec:ifr.morphology} and §\ref{sec:pst.ifr.ipfv.morphology}).\footnote{The only non-past TAME category that is compatible with the \forme{-t} suffix is the Apprehensive (§\ref{sec:apprehensive}). }

This section describes the morphology of these TAME categories, their uses in main clauses and subordinate clauses, and also the semantic contrast between them in various contexts.


\subsection{Aorist}    \label{sec:aor}

\subsubsection{Morphology} \label{sec:aor.morphology}
The Aorist is built by combining stem II (§\ref{sec:stem2}, or stem I for non-alternating verbs) with either A-type (\ref{ex:tAGej}) or C-type preverbs (\ref{ex:tatWt}) depending on person configuration and transitivity (§\ref{sec:kamnyu.preverbs}, §\ref{sec:preverb.TAME.morphology}): the latter are restricted to transitive direct 3\flobv{} configurations (§\ref{sec:indexation.non.local}).

\begin{exe}
\ex \label{ex:tAGej}
 \gll tɤ-ɣe-j \\
\textsc{aor}:\textsc{up}-come[II]-\textsc{1pl} \\
\glt `We came (up).' (many attestations)
\end{exe}

\begin{exe}
\ex \label{ex:tatWt}
 \gll ta-tɯt \\
\textsc{aor}:3\flobv{}-say[II]  \\
\glt `S/he said it.' (many attestations)
\end{exe}

In addition, like the Inferential, the Aorist takes the 1/2\textsc{sg}\fl{}3 Past transitive suffix \forme{-t} (§\ref{sec:suffixes}, §\ref{sec:other.TAME}). Complete paradigms of transitive and intransitive verbs in the Aorist are presented in §\ref{sec:polypersonal}, and need not be repeated here.

Some Aorist verb forms are ambiguous and could be interpreted as belonging to other TAME categories. The ambiguity with the Imperative (§\ref{sec:imp.morphology}) is discussed in detail in §\ref{sec:ambiguity.preverb}. A more difficult case is that between Past Imperfective \forme{pɯ-} (§\ref{sec:pst.ifr.ipfv}) and the Aorist of stative verbs selecting \textsc{downwards} as their intrinsic orientation (§\ref{sec:lexicalized.orientation}). The only way of differentiating between the two is the clear inchoative meaning of stative verbs in the Aorist (§\ref{sec:aor.inchoative}), showing that \forme{pɯ-rom} in (\ref{ex:pWrom.kWnA}) can only be analyzed as an Aorist `(when) it has dried' rather than as a Past Imperfective `it was dry' (see also example §\ref{sec:ipfv.inchoative},  §\ref{ex:YWYaR.Zo.Nu}).

\begin{exe}
\ex \label{ex:pWrom.kWnA}
 \gll  ɯ-jwaʁ rcanɯ pɯ-rom kɯnɤ ɯ-mdzu nɯ mɤʑɯ ʑo mtɕoʁ \\
\textsc{3sg}.\textsc{poss}-leaf \textsc{unexp}:\textsc{foc} \textsc{aor}-be.dry also \textsc{3sg}.\textsc{poss}-thorn \textsc{dem} even.more \textsc{emph} be.sharp:\textsc{fact} \\
\glt `When its leaves have dried, the thorns (on the leaves) are even sharper.' (18-NGolo, 70)
\end{exe}

\subsubsection{Main clauses}   \label{sec:aor.main}
The Aorist occurs in main clauses to express past perfective events that the speaker has witnessed him/herself. It is used to report actions that the speaker has performed himself, as in (\ref{ex:tAXtWta.pWBzuta}) and (\ref{ex:nACki.tAtWta}), unlike the Inferential, which is only compatible with first person in very specific contexts (§\ref{sec:ifr}).

\begin{exe}
\ex \label{ex:tAXtWta.pWBzuta}
 \gll nɯ kɯ-fse rcanɯ, <qibajin> ʑo tɤ-χtɯ-t-a. tɕe <dong> pɯ-βzu-t-a  \\
\textsc{dem} \textsc{sbj}:\textsc{pcp}-be.like \textsc{unexp}:\textsc{foc} seven.or.eight.pounds \textsc{emph} \textsc{aor}-buy-\textsc{pst}:\textsc{tr}-\textsc{1sg} \textsc{lnk} freeze \textsc{aor}-make-\textsc{pst}:\textsc{tr}-\textsc{1sg} \\
\glt `(Nettles) like that, I bought seven or eight pounds. Then I put them in the refrigerator.' (conversation, 14-05-10)
\end{exe}

In the absence of any adverb with a function comparable to English `already' in Japhug, the Aorist is used to express this meaning, in combination with a tense adverb as in (\ref{ex:nACki.tAtWta}).\footnote{Tshendzin said (\ref{ex:nACki.tAtWta}) answering a question I had already asked a few days before. }

\begin{exe}
\ex \label{ex:nACki.tAtWta}
 \gll jɯfɕɯndʐi nɤ-ɕki tɤ-tɯt-a ma, wuma ʑo a-tsa ʑo ɲɯ-βze  \\
 a.few.days.ago \textsc{2sg}.\textsc{poss}-\textsc{dat} \textsc{aor}-say[II]-\textsc{1sg} \textsc{lnk} really \textsc{emph} \textsc{1sg}.\textsc{poss}-adapted \textsc{emph} \textsc{sens}-make[III] \\
 \glt `I already told you a few days ago, (the shoes you have sent me) fit me really well.' (conversation, 2019-05-26)
\end{exe}

Actions that the speaker has witnessed as a passive participant are also expressed with the Aorist rather than the Inferential.

\begin{exe}
\ex \label{ex:jAGE.GWnWwGmbia}
 \gll a-kɯ-rtoʁ jɤ-ɣe tɕe ɣɯ-nɯ́-wɣ-mbi-a \\
 \textsc{1sg}.\textsc{poss}-\textsc{sbj}:\textsc{pcp}-look \textsc{aor}-come[II] \textsc{lnk} \textsc{cisl}-\textsc{aor}-\textsc{inv}-give-\textsc{1sg} \\
 \glt `He came to see me and gave it to me.' (conversation, 17-09-21)
\end{exe}

In (\ref{ex:jWfCWCAr.tAjpa.kalAt}), the choice of the Aorist \forme{ka-lɤt} and the Past Imperfective \forme{pɯ-wxti} reflects the fact that the speaker has directly seen the snowfall (rather than deducing its occurrence from the presence of snow on the ground). By contrast, the Inferential \forme{to-ndʐi} `it melted' (rather than the Aorist \forme{tɤ-ndʐi}) indicates that the speaker has not witnessed the melting, and only deduced that it has occurred due to the absence of snow, despite the snowfall in the previous night (§\ref{sec:ifr.evd}).

\begin{exe}
\ex \label{ex:jWfCWCAr.tAjpa.kalAt}
 \gll kutɕu hanɯni ɲɯ-ɣɤndʐo. jɯfɕɯɕɤr tɤjpa ka-lɤt. ka-lɤt ri mɯ-pɯ-wxti. jisŋi tɕe lonba to-ndʐi. \\
 \textsc{dem}.\textsc{prox}:\textsc{loc} a.little \textsc{sens}-be.cold yesterday.night snow \textsc{aor}:3\flobv{}-release \textsc{aor}:3\flobv{}-release  \textsc{lnk} \textsc{neg}-\textsc{pst}.\textsc{ipfv}-be.big today \textsc{lnk} all \textsc{ifr}-\textsc{acaus}:melt \\
\glt `Here it is a bit cold. Yesterday evening there was a snowfall. There was snow but not much, and now it has melted completely.' (conversation, 17-11-23)
\end{exe}

The Aorist is also used to describe the events that the speaker has seen on a film, for instance the pear stories, as in (\ref{ex:aor.pear}). 

\begin{exe}
\ex \label{ex:aor.pear}
 \gll tɤ-pɤtso kɯ-ɤ<nɯ>ɣro tsuku ɣɤʑu-nɯ jɤ-ɣe-nɯ tɕe, nɯra kɯ ɯ-paχɕi ra kɤ-wum ta-qur-nɯ tɕe, ta-sɯɣ-ndzur-nɯ \\
 \textsc{indef}.\textsc{poss}-child \textsc{sbj}:\textsc{pcp}-<\textsc{auto}>play several exist:\textsc{sens}-\textsc{pl} \textsc{aor}-come[II]-\textsc{pl} \textsc{lnk} \textsc{dem}:\textsc{pl} \textsc{erg} \textsc{inf}-collect \textsc{aor}:3\flobv{}-help-\textsc{pl} \textsc{lnk} \textsc{aor}:3\flobv{}-\textsc{caus}-stand-\textsc{pl} \\
 \glt `There were some children playing (there), they came, helped him to collect the apples (that had been spilled) and helped (him) up.' (chen-pear, 10-11)
\end{exe}

For events that have occurred in a more remote past, the requirement on direct (visual) perception may be less strict. For instance, in (\ref{ex:tAngo.thWmdW}), the verbs in the Aorist express a series of events that have happened to a member of the speaker's extended family. They did not live in the same household, and did not meet very frequently, and the speaker did not witness all of the events, but is familiar enough with the situation to feel entitled to use the Aorist rather than the Inferential.

\begin{exe}
\ex \label{ex:tAngo.thWmdW}
 \gll nɯ-mu nɯ tɤ-ngo qʰe ci ci nɯ-si kɯ-fse ci ci tɤ-mna kɯ-fse qʰe kɯɕnɯ-xpa ʑo tʰɯ-mdɯ. \\
 \textsc{3pl}.\textsc{poss}-mother \textsc{dem} \textsc{aor}-be.ill \textsc{lnk} one one \textsc{aor}-die \textsc{sbj}:\textsc{pcp}-be.like  one one \textsc{aor}-be.better  \textsc{sbj}:\textsc{pcp}-be.like \textsc{lnk} seven-year \textsc{emph} \textsc{aor}-live.up.to \\
\glt `Their mother got ill, and survived seven years, sometimes looking like she had died, sometimes looking like she was getting better.' (14-siblings, 32-33)
\end{exe}

\subsubsection{Change of state}   \label{sec:aor.inchoative}
In main clauses, stative verbs (other than copulas, §\ref{sec:copula.basic}) in the Aorist always express a change of state, whether adjectival verbs such as \japhug{dɤn}{be many} (\ref{ex:tAdAnnW3}) or existential verbs (\ref{ex:nWme}).

\begin{exe}
\ex \label{ex:tAdAnnW3}
 \gll nɯre kɯmaʁ pɕoʁ <banqian> jɤ-kɯ-ɣe nɯra tu-nɯ tɕe, tɤ-dɤn-nɯ. \\
 \textsc{dem}:\textsc{loc} other side move \textsc{aor}-\textsc{sbj}:\textsc{pcp}-come[II] \textsc{dem}:\textsc{pl} exist:\textsc{fact}-\textsc{pl} \textsc{lnk} \textsc{aor}-be.many-\textsc{pl} \\
 \glt `(Now) there are people who have come from other places (to settle in that village), (and the number of inhabitants) has increased.' (140522 tshupa, 87)
\end{exe}


\begin{exe}
\ex \label{ex:nWme}
 \gll nɤ-<dian> nɯ-me \\
 \textsc{2sg}.\textsc{poss}-electricity \textsc{aor}-not.exist \\
 \glt `Your (cellphone) is out of battery.' (you don't have any electricity anymore) (heard in context)
\end{exe}

The Aorist is however found on stative verbs in some temporal subordinate clauses (§\ref{sec:aor.temporal}) without change of state meaning. 

In addition to stative verbs, some modal verbs such as \japhug{cʰa}{can} select the \textsc{upwards} orientation with an inchoative meaning, as illustrated by example (\ref{ex:tAnANkWNkea.tAchaa}) (§\ref{sec:TAM.finite}) (see also §\ref{sec:ifr.inchoative}).
 

%tɯrgi wuma ʑo sɯŋgɯ kɯ-wxti tɤ-rkɯn ma kɤ-phɯt ntsɯ kɯ koŋla nɯ-me. 

\subsubsection{Temporal subordinate clauses }   \label{sec:aor.temporal}
The Aorist is used in subordinate clauses to mark a point of temporal reference. It occurs in generic statements to indicate the period when an event takes place. For instance in  (\ref{ex:ftCar.kAndzoR.tulhoR}), the temporal clause \forme{ftɕar kɤ-ndzoʁ} `when summer arrives' must select the Aorist; no other TAME category would be possible here.

\begin{exe}
\ex \label{ex:ftCar.kAndzoR.tulhoR}
 \gll ɯ-fsaqʰe, [ftɕar kɤ-ndzoʁ] qʰe li tu-ɬoʁ. \\
 \textsc{3sg}.\textsc{poss}-next.year summer \textsc{aor}-\textsc{acaus}:attach \textsc{lnk} again \textsc{ipfv}-come.out \\
 \glt `The next year, when summer arrives, it comes out again.' (of a perennial plant, 19-qachGa mWntoR, 29)
\end{exe}

Direct visual perception is irrelevant in temporal clauses. In (\ref{ex:tCaGi.tuti}), the Aorist \forme{jɤ-ɣe} occurs even though the speaker is only reporting a story about a parrot that she has not witnessed personally, and only heard from someone else. Here the meaning of the Aorist is simply to state the temporal condition when the following actions (utterance of human speech) take place.

\begin{exe}
\ex \label{ex:tCaGi.tuti}
 \gll tɯrme jɤ-ɣe tɕe <laikerenle> tu-ti, tɕe <nihao>  tu-ti. \\
 person \textsc{aor}-come[II] \textsc{lnk} a.guest.has.arrived \textsc{ipfv}-say \textsc{lnk} hello \textsc{ipfv}-say  \\
 \glt `(Of a parrot which is able to say a few words) When someone comes (the parrot) says `A guest has arrived', and says `hello'.' (24-qro, 112-113)
\end{exe}

With stative verbs, the Aorist can exceptionally used without inchoative meaning (§\ref{sec:aor.inchoative}) in temporal clauses, but always with the \textsc{upwards} \forme{tɤ-} preverb. For instance in (\ref{ex:tAjpum.tAxtshWm}), \forme{tɤ-jpum} and \forme{tɤ-xtsʰɯm} mean `when it is thick' and `when it is thin', not `when it becomes thick/thin', which would be expressed with the intrinsic \textsc{westwards} preverbs (see also \ref{ex:si.tAme}, §\ref{sec:preverb.loss} and \ref{ex:nWma.tAme}, §\ref{sec:orientation.preverb.aspect}). 

\begin{exe}
\ex \label{ex:tAjpum.tAxtshWm}
 \gll tɤ-jpum tɕe tɕendɤre tɯ-sŋi kʰro lú-wɣ-taʁ ɲɯ-kɯ-cʰa ma tɤ-xtsʰɯm tɕe tɕe, koŋla lú-wɣ-taʁ mɯ́j-sɤ-cʰa   \\
 \textsc{aor}-be.thick \textsc{lnk} \textsc{lnk} one-day much \textsc{ipfv}-\textsc{inv}-weave \textsc{sens}-\textsc{genr}:S/O-can \textsc{lnk} \textsc{aor}-be.thin \textsc{lnk} \textsc{lnk} completely \textsc{ipfv}-\textsc{inv}-weave \textsc{neg}:\textsc{sens}-\textsc{prop}-can \\
\glt `When (the threads) are thick, one can weave a lot in one day, when they are thin, it is not possible to weave.' (2011-06-thaXtsa, 54-55)
\end{exe}

Apart from stative verbs, a similar use of the \textsc{upwards} orientation is found with the transitive verb \japhug{sɯso}{think}, `want' in some contexts (§\ref{sec:sWso.complement}).

While the Aorist is restricted to past events in main clauses, it is found in temporal clauses to express points of time reference in the future. In (\ref{ex:pWtWnWGenW.nWtCu}) and (\ref{ex:pWtari.nWtCu}), the Aorist forms  \forme{pɯ-tɯ-nɯ-ɣe-nɯ} `when you come back' and \forme{pɯ-tɯ-ari} `when you go down' refer to events that have not yet taken place at the time of utterance. The use of the Aorist in future contexts is also observed in conditional clauses (§\ref{sec:aor.cond}).

\begin{exe}
\ex \label{ex:pWtWnWGenW.nWtCu}
 \gll nɯʑo pɯ-tɯ-nɯ-ɣe-nɯ nɯtɕu cuparkʰɤrkʰɤt a-pɯ-tɯ-ɬoʁ-nɯ tɕe aʑo tʰɤlwa ɲɯɣɲɯɣ pjɯ-ɬoʁ-a ŋu \\
 \textsc{2pl} \textsc{aor}:\textsc{down}-2-\textsc{vert}-come[II] \textsc{dem}:\textsc{loc} stone.step \textsc{irr}-\textsc{pfv}:\textsc{down}-2-come.out-\textsc{pl} \textsc{lnk} \textsc{1sg} earth \textsc{idph}(II):soft \textsc{ipfv}:\textsc{down}-come.out-\textsc{1sg} be:\textsc{fact} \\
\glt `When you come back, take the stone steps, I will come down on the soft earth.' (2014-kWlAG, 438)
\end{exe}

\begin{exe}
\ex \label{ex:pWtari.nWtCu}
 \gll tɕetʰa pɯ-tɯ-ari tɕe, ki a-kɤ-tɯ-ɕtʰɯz tɕe, tɕetʰa ju-nɯ-ɕe-nɯ ɕti \\
 soon \textsc{aor}:\textsc{down}-2-go[II] \textsc{lnk} \textsc{dem}.\textsc{prox} \textsc{irr}-\textsc{pfv}:\textsc{east}-2-turn.towards \textsc{lnk} soon \textsc{ipfv}-\textsc{vert}-go-\textsc{1sg} be.\textsc{aff}:\textsc{fact} \\
 \glt `When you go down there, turn (this magical object) in the direction (of the râkshasas), and they will go back (from where they are from).' (2011-04-smanmi, 121)
\end{exe}


In (\ref{ex:tCetha.nWmbrAt}), both the temporal clause and the main clause contain a verb in the Aorist, but in the latter, that verb \forme{nɯ-me} `it disappeared' is embedded in a complement clause headed by the noun \japhug{ɯ-ndʐa}{cause} (§\ref{sec:causal.clauses}, §\ref{sec:nouns.cause.complement}, with elided possessive prefix), while the main verb \japhug{ŋu}{be} is in the Factual.

\begin{exe}
\ex \label{ex:tCetha.nWmbrAt}
 \gll a-ʁa tu ri, tɕetʰa nɯ-mbrɤt tɕe tɕe [a-<dian> nɯ-me] ndʐa ŋu \\
 \textsc{1sg}.\textsc{poss}-free.time exist:\textsc{fact} \textsc{lnk} soon \textsc{aor}-\textsc{acaus}:break \textsc{lnk} \textsc{lnk} \textsc{1sg}.\textsc{poss}-electricity \textsc{aor}-not.exist reason be:\textsc{fact} \\
\glt `I have time (to talk with you), but in a moment when (the phone line) disconnects, it will be because my (cellphone) is out of battery.' (conversation)
\end{exe}

The Aorist in subordinate clauses is not always used to fix a point of temporal reference, however. It can also refer to an event preceding those of the following clauses, and which the speaker has witnessed (as in main clauses, §\ref{sec:aor.main}). In (\ref{ex:nwWGmbia.plusquampft}), the clause \forme{tɤ-mtʰɯm nɯ́-wɣ-mbi-a} is not to be translated as `when (s/he/someone) gave/gives meat' (a translation that is possible in other contexts), but rather as `(someone) had given me' with as pluperfect, as a background event that took place before the whole story begins.

\begin{exe}
\ex \label{ex:nwWGmbia.plusquampft}
 \gll tɤ-mtʰɯm nɯ́-wɣ-mbi-a tɕe, tu-ndze-a pɯ-ŋu tɕe, kʰɤxtu ri pɯ-rɤʑi-a. \\
 \textsc{indef}.\textsc{poss}-meat \textsc{aor}-\textsc{inv}-give-\textsc{1sg} \textsc{lnk} \textsc{ipfv}-eat[III]-\textsc{1sg} \textsc{pst}.\textsc{ipfv}-be \textsc{lnk} roof \textsc{loc} \textsc{pst}.\textsc{ipfv}-stay-\textsc{1sg} \\
 \glt `(Someone) had given me (a piece of) meat, and I was eating it, I was staying on the roof platform (and then a kite flew down and robbed it). (150909 qandZGi, 2)
\end{exe} 
 
\subsubsection{Conditional clauses }   \label{sec:aor.cond}
In the protasis of reduplicated conditional (§\ref{sec:redp.protasis}), the Aorist has a purely aspectual function, and does not express absolute past tense, but past tense relative to the apodosis.

The Aorist occurs in the protasis in generic contexts, as in (\ref{ex:mWmAthWwGndZWr}).

\begin{exe}
\ex \label{ex:mWmAthWwGndZWr}
 \gll tɤɕi, qaj, stoʁ staʁpɯ nɯra mɯ\redp{}mɤ-tʰɯ́-wɣ-ɣndʑɯr nɤ kɤ-ndza mɤ-kʰɯ \\
 barley wheat broad.bean peas \textsc{dem}:\textsc{pl} \textsc{cond}\redp{}\textsc{neg}-\textsc{aor}-\textsc{inv}-grind \textsc{add} \textsc{inf}-eat \textsc{neg}-be.possible:\textsc{fact} \\
 \glt `If we don't grind barley, wheat, broad beans and peas, they cannot be eaten.' 06-BGa, 5)
\end{exe}

It is also found in conditional constructions referring to future events, as in (\ref{ex:tWtAtWtWt}) and (\ref{ex:pWpWnNo.nA}). This usage reminds of the use of the Aorist in future temporal clauses (such as \ref{ex:pWtWnWGenW.nWtCu}, §\ref{sec:aor.temporal}).

\begin{exe}
\ex \label{ex:tWtAtWtWt}
 \gll  tɯ\redp{}tɤ-tɯ-tɯt nɤ tɕe pjɯ-ta-sat ŋu \\
 \textsc{cond}\redp{}\textsc{aor}-2-say[II] add \textsc{lnk} \textsc{ipfv}-1\fl{}2-kill be:\textsc{fact} \\
 \glt `If you tell (anyone) about it, I will kill you.' (150901 changfamei-zh, 54)
\end{exe}

\begin{exe}
\ex \label{ex:pWpWnNo.nA}
\gll pɯ\redp{}pɯ-nŋo nɤ, ndʑiʑo ʁnɯz ɣɯ ndʑi-ku cʰɯ́-wɣ-pʰɯt ra \\
 \textsc{cond}\redp{}\textsc{aor}-be.defeated \textsc{add} \textsc{2du} two \textsc{gen} \textsc{2du}.\textsc{poss}-head \textsc{ipfv}-\textsc{inv}-take.off be.needed:\textsc{fact} \\
 \glt `If he fails, we will decapitate both of you.' (140505 liuhaohan zoubian tianxia-zh, 104-5)
\end{exe}

The Aorist is less felicitous in the protasis of counterfactuals (§\ref{sec:counterfactual}), where the Irrealis is used instead (§\ref{sec:irrealis.conditional}).

\subsubsection{Relative clauses }   \label{sec:aor.relative}
The Aorist commonly occurs in finite relative clauses (§\ref{sec:finite.relatives}), as in the object head-internal relative in (\ref{ex:nakho.nWnW.koCthWz}).


\begin{exe}
\ex \label{ex:nakho.nWnW.koCthWz}
\gll iɕqʰa [srɯnloʁ-pɯ kɯ-fse na-kʰo] nɯnɯ ko-ɕtʰɯz \\
the.aforementioned ring-\textsc{dim} \textsc{sbj}:\textsc{pcp}-be.like \textsc{aor}:3\fl{}3-give \textsc{dem} \textsc{ifr}:\textsc{east}-turn.towards \\
\glt `He$_i$ turned the little ring that (Smanmi Metog Koshana) had given him$_i$ in the direction (of the râkshasas). (28-smAnmi, 173)
\end{exe}

The verb in the relative can also undergo totalitative reduplication of the first syllable (§\ref{sec:totalitative.redp}), as \forme{tɯ\redp{}ta-stu} `all (the ways) in which she had done it'.\footnote{The object of the verb \japhug{stu}{do like} refers to the manner in which the action is performed, not its patient (§\ref{sec:ditransitive.secundative}). }

\begin{exe}
\ex \label{ex:tWtastu.tostu}
\gll [ɯ-pi kɯ tɯ\redp{}ta-stu] nɯ to-stu qʰe \\
\textsc{3sg}.\textsc{poss}-elder.sibling \textsc{erg} \textsc{total}\redp{}\textsc{aor}:3\flobv{}-do.like \textsc{dem} \textsc{ifr}-do.like \textsc{lnk} \\
\glt `She did everything like her elder sister.' (2014-kWlAG, 167)
\end{exe}

In relative clauses, the contrast between Aorist and Inferential is neutralized, as only the Aorist can appear. Thus, in both (\ref{ex:nakho.nWnW.koCthWz}) and (\ref{ex:tWtastu.tostu}) above, the verb of the relative clause in the Aorist, while that of the main clause is in the Inferential. The events referred to in the Aorist in these relative clauses occur in the Inferential earlier in the stories: compare for instance the Inferential \forme{ɲɤ-kʰo} in the main clause in (\ref{ex:kWfse.ci.YAkho}) with the Aorist \forme{na-kʰo} in the relative in (\ref{ex:nakho.nWnW.koCthWz}).

\begin{exe}
\ex \label{ex:kWfse.ci.YAkho}
\gll srɯnloʁ-pɯ kɯ-fse ci ɲɤ-kʰo. \\
 ring-\textsc{dim} \textsc{sbj}:\textsc{pcp}-be.like \textsc{indef} \textsc{ifr}-give \\
 \glt `(Smanmi Metog Koshana) gave him something like a little ring.' (28-smAnmi, 160)
\end{exe}

\subsubsection{Complement clauses }   \label{sec:aor.complement} 
Some finite complement clauses take the Aorist when the main verb is also in the Aorist, as in (\ref{ex:nWnAzreta.pWchaa}). 

 \begin{exe}
\ex \label{ex:nWnAzreta.pWchaa}
\gll aʑo [nɯ-z-nɤre-t-a] pɯ-cʰa-a \\
\textsc{1sg} \textsc{aor}-\textsc{caus}-laugh-\textsc{pst}:\textsc{tr}-\textsc{1sg} \textsc{aor}-can-\textsc{1sg} \\
\glt `I succeeded in making her laugh.' (140430 jin e-zh, 179)
\end{exe}

\subsubsection{Periphrastic Narrative }   \label{sec:aor.narrative}
Some speakers (in particular Kunbzang mtshu) use the Periphrastic Narrative instead of the Inferential as the main TAME category of narration when telling traditional stories.

The Periphrastic Narrative combines a verb in the Aorist with the copula in the Sensory \forme{ɲɯ-ŋu}. In (\ref{ex:thWstanW.YWNu}), the periphrastic construction \forme{tʰɯ-sta-nɯ ɲɯ-ŋu} `they woke up' corresponds to an Inferential \forme{cʰɤ-sta-nɯ} in a similar story told by another speaker (\ref{ex:chAstanW}).

\begin{exe}
\ex \label{ex:thWstanW.YWNu}
\gll  nɯ ʁmaʁ kɯ\redp{}kɯ-tu ʑo tʰɯ-sta-nɯ ɲɯ-ŋu. \\
\textsc{dem} soldier \textsc{total}\redp{}\textsc{sbj}:\textsc{pcp}-exist \textsc{emph} \textsc{aor}-wake-\textsc{pl} \textsc{sens}-be \\
\glt `All the soldiers woke up.' (2003qachGa, 77)
\end{exe}

\begin{exe}
\ex \label{ex:chAstanW}
\gll iɕqʰa kɯ-rɯru ʁmaʁmi nɯra cʰɤ-sta-nɯ. \\
the.aforementioned \textsc{sbj}:\textsc{pcp}-guard soldiers \textsc{dem}:\textsc{pl} \textsc{ifr}-wake-\textsc{pl} \\
\glt `The guards woke up.' (140507 jinniao-zh, 148)
\end{exe}

As in the case of other periphrastic TAME constructions (see in particular \ref{ex:chain.pjANu}, §\ref{sec:ipfv.periphrastic.TAME} concerning the Periphrastic Imperfective), a chain of several verbs in the Aorist can share one copula. In (\ref{ex:mWtAtWtndZi.YWNu}) for instance, the copula \forme{ɲɯ-ŋu} has scope over two clauses, each containing a verb in the Aorist (\forme{ta-tɯt} and \forme{mɯ-ta-tɯt-ndʑi}, respectively).

\begin{exe}
\ex \label{ex:mWtAtWtndZi.YWNu}
\gll tɕe [kɯ-wxti ni ndʑi-pʰe ta-tɯt] ri, [kɯ-wxti ni kɯ mɯ-ta-tɯt-ndʑi] ɲɯ-ŋu. \\
\textsc{lnk} \textsc{sbj}:\textsc{pcp}-big \textsc{du} \textsc{3du}.\textsc{poss}-\textsc{dat} \textsc{aor}:3\flobv{}-say[II] \textsc{lnk} \textsc{sbj}:\textsc{pcp}-big \textsc{du} \textsc{erg} \textsc{neg}-\textsc{aor}:3\flobv{}-say[II]-\textsc{du} \textsc{sens}-be \\
\glt `He told it$_i$ to the two elder (sisters)$_j$, but they$_j$ did not tell it$_i$ (to their$_j$ parents).' (2005-stod-kunbzang, 35)
\end{exe}

In the case of the verb \japhug{ti}{say}, while the regular Periphrastic Narrative is attested (\ref{ex:mWtAtWtndZi.YWNu}), we also find a periphrastic form \forme{ti ɲɯ-ŋu} (\ref{ex:ti.YWNu}) with the Factual \forme{ti} instead of the Aorist \forme{ta-tɯt}.

\begin{exe}
\ex \label{ex:ti.YWNu}
\gll ``a-pi ɲɯ-ɕpaʁ-a" ti ɲɯ-ŋu \\
\textsc{1sg}.\textsc{poss}-elder.sibling \textsc{sens}-be.thirsty-\textsc{1sg} say:\textsc{fact} \textsc{sens}-be \\
\glt `She said: `Sister, I am thirsty.'' (2003 Kunbzang, 306)
\end{exe}

The verb \japhug{ti}{say} in the Factual can share a copula with verbs in the Aorist in Periphrastic Narrative chains. In (\ref{ex:ti.nA.tatWt}), the Factual form \forme{ti} belongs to the same chain as the following Aorist form \forme{ta-tɯt}, and \forme{ɲɯ-ŋu} has scope over both of them. This periphrastic construction is similar to that of (\ref{ex:mWtAtWtndZi.YWNu}) above, but with a verb in the Factual instead of the Aorist in the first clause.

\begin{exe}
\ex \label{ex:ti.nA.tatWt}
\gll [stu kɯ-xtɕi nɯ ɯ-pʰe ti] nɤ, [nɯ kɯ li ta-tɯt] ɲɯ-ŋu \\
most \textsc{sbj}:\textsc{pcp}-be.small \textsc{dem} \textsc{3sg}.\textsc{poss}-\textsc{dat} say:\textsc{fact} \textsc{add} \textsc{dem} \textsc{erg} again \textsc{aor}:3\flobv{}-say[II] \textsc{sens}-be \\
\glt `He told it$_i$ to the youngest (sister)$_j$, and she$_j$ told it$_i$ (to her parents).' (2003 Kunbzang, 56)
\end{exe}
 
\subsection{Inferential} \label{sec:ifr}
In this section, the term `Inferential' is used as abbreviation for `Inferential Perfective', as opposed to the Inferential Imperfective discussed in §\ref{sec:pst.ifr.ipfv}.

\subsubsection{Morphology}   \label{sec:ifr.morphology}
In the Kamnyu dialect of Japhug, the Inferential is built from the stem I of the verb with type D preverbs (§\ref{sec:kamnyu.preverbs}, §\ref{sec:preverb.TAME.morphology}). For instance, the verb \japhug{ɕe}{go} (whose stem II is \forme{-ari}, §\ref{sec:stem2}), has the Inferential \textsc{3sg} form \forme{jo-ɕe} (\textsc{ifr}-say) `he said' with the indefinite orientation type D preverb \forme{jo-} and the stem I \forme{-ɕe}.

In the Xtokavian dialects of Japhug, there are only two series of preverbs (A and B), and the Inferential is marked by combining the B type preverbs with the Inferential prefix \forme{a-}. The second person prefix \forme{tɯ-} is inserted between the preverb and the Inferential \forme{a-} (compare \ref{ex:YEtanWYAmkhe} and \ref{ex:YAtWnWYAmkhe}, §\ref{sec:xtokavian.preverbs}).

Transitive verbs with open syllable stem with first  and second person singular subjects and third person object in addition select the \forme{-t} past tense suffix (§\ref{sec:suffixes}, §\ref{sec:other.TAME}, \forme{-z} in some dialects of Japhug), as illustrated by (\ref{ex:cha.kotshita}), (\ref{ex:mWtotata}) and (\ref{ex:WsroR.kotWrit}) (see also \ref{ex:mWtosAlata}, §\ref{sec:inf.1person}).

\begin{exe}
\ex \label{ex:cha.kotshita}
\gll aʑo cʰa kʰro ko-tsʰi-t-a \\
\textsc{1sg} alcohol much \textsc{ifr}-drink-\textsc{pst}:\textsc{tr}-\textsc{1sg} \\
\glt `I drank a lot of alcohol.' (aesop zuoke de gou-zh, 36)
\end{exe}

\begin{exe}
\ex \label{ex:WsroR.kotWrit}
\gll maka nɤ-βzaŋlɤn βze-a ra ma a-tɕɯ ɯ-sroʁ ko-tɯ-ri-t tɕe \\
completely \textsc{2sg}.\textsc{poss}-payback make[III]:\textsc{fact}-\textsc{1sg} be.needed:\textsc{fact} \textsc{lnk} \textsc{1sg}.\textsc{poss}-son \textsc{3sg}.\textsc{poss}-life \textsc{ifr}-2-save-\textsc{pst}:\textsc{tr} \textsc{lnk} \\
\glt `I have to return the favour, as you have saved my son's life.' (2011-04-smanmi, 49)
\end{exe}

The verb \japhug{ti}{say} is irregular in lacking this suffix in the Inferential; the \textsc{1sg} and \textsc{2sg} Inferential of this verb are thus \forme{to-ti-a} and \forme{to-tɯ-ti} as in (\ref{ex:tChi.totia}), not $\dagger$\forme{to-ti-t-a} and $\dagger$\forme{to-tɯ-ti-t}.

\begin{exe}
\ex \label{ex:tChi.totia}
\gll tɕe nɯ ɯ-qʰu tɕe tɕʰi to-ti-a? \\
\textsc{lnk} \textsc{dem} \textsc{3sg}.\textsc{poss}-after \textsc{lnk} what \textsc{ifr}-say-\textsc{1sg} \\
\glt `What did I say after that?' (140522 Kamnyu zgo, 221)
\end{exe}

The Inferential occurs with the negative prefix \forme{mɯ-} (§\ref{sec:neg.allomorphs}), as in (\ref{ex:mWtotata}).

\begin{exe}
\ex \label{ex:mWtotata}
\gll mɯ-to-ta-t-a \\
\textsc{neg}-\textsc{ifr}-put-\textsc{pst}:\textsc{tr}-\textsc{1sg} \\
\glt `I did not put (the tea on the oven).' (Conversation, 28-04-2018, Dpalcan)
\end{exe}

In the Kamnyu dialect, the preverbs are prevented from merging with the initial \forme{a-} of contracting verbs (§\ref{sec:contraction}) by insertion of the peg circumfix (§\ref{sec:peg.circumfix}), to avoid confusion with the Imperfective (since the result of the vowel merger of B type and D type preverbs with \forme{a-} is identical). This insertion occurs in third (\ref{ex:tWrme.YAkABzunWci}) and first person (see \ref{ex:WRjoR.YAkABzuaci}, §\ref{sec:inf.1person}) forms.


\begin{exe}
\ex \label{ex:tWrme.YAkABzunWci}
\gll koŋla tɯrme ɲɤ-k-ɤβzu-nɯ-ci \\
really person \textsc{ifr}-\textsc{peg}-become-\textsc{pl}-\textsc{peg} \\
\glt `(The puppets) became real people.' (150822 yan muouxi de ren-zh, 46)
\end{exe}

In the second person, the prefix \forme{tɯ-} occurs in slot -2 (§\ref{sec:outer.prefixal.chain}, like the prefixal element \forme{k(ɯ)-} of the peg circumfix.) between the preverbs and the verb stem, 

\begin{exe}
\ex \label{ex:pGa.YAtABzu}
\gll nɤʑo pɣa ɲɤ-tɯ-ɤβzu ɕti tɕe \\
\textsc{2sg} bird \textsc{ifr}-2-become be.\textsc{aff}:\textsc{fact} \textsc{lnk} \\
\glt `You have become (transformed into) a bird.' (160630 abao-zh, 151)
\end{exe}

\tabref{tab:ifr.paradigms} presents the paradigms of a regular transitive verb (\japhug{tsʰi}{drink}), the irregular \japhug{ti}{say} (both with a \textsc{3sg} object) and an intransitive contracting verb (\japhug{aβzu}{become}) in the Kamnyu dialect.

\begin{table}
\caption{Inferential paradigms} \label{tab:ifr.paradigms}
\begin{tabular}{lllll}
\lsptoprule
Subject & \japhug{tsʰi}{drink} & \japhug{ti}{say} & \japhug{aβzu}{become} \\
\midrule
\textsc{1sg}(\flobv{}) & \forme{ko-tsʰi-\rouge{t}-a} & \forme{to-ti-a} & \forme{ɲɤ-\rouge{k}-ɤβzu-a-\rouge{ci}} \\
\textsc{1du}(\flobv{}) & \forme{ko-tsʰi-tɕi} & \forme{to-ti-tɕi} & \forme{ɲɤ-\rouge{k}-ɤβzu-tɕi-\rouge{ci}} \\
\textsc{1pl}(\flobv{}) & \forme{ko-tsʰi-j} & \forme{to-ti-j} & \forme{ɲɤ-\rouge{k}-ɤβzu-j-\rouge{ci}} \\
\midrule
\textsc{2sg}(\flobv{}) & \forme{ko-tɯ-tsʰi-\rouge{t}} & \forme{to-tɯ-ti} & \forme{ɲɤ-tɯ-ɤβzu} \\
\textsc{2du}(\flobv{}) & \forme{ko-tɯ-tsʰi-ndʑi} & \forme{to-tɯ-ti-ndʑi} & \forme{ɲɤ-tɯ-ɤβzu-ndʑi} \\
\textsc{2pl}(\flobv{}) & \forme{ko-tɯ-tsʰi-nɯ} & \forme{to-tɯ-ti-nɯ} & \forme{ɲɤ-tɯ-ɤβzu-nɯ} \\
\midrule
\textsc{3sg}(\flobv{}) & \forme{ko-tsʰi} & \forme{to-ti} &\forme{ɲɤ-\rouge{k}-ɤβzu-\rouge{ci}} \\ 
\textsc{3du}(\flobv{}) & \forme{ko-tsʰi-ndʑi} & \forme{to-ti-ndʑi} &\forme{ɲɤ-\rouge{k}-ɤβzu-ndʑi-\rouge{ci}} \\ 
\textsc{3pl}(\flobv{}) & \forme{ko-tsʰi-nɯ} & \forme{to-ti-nɯ} &\forme{ɲɤ-\rouge{k}-ɤβzu-nɯ-\rouge{ci}} \\ 
\lspbottomrule
\end{tabular}
\end{table}

The contrast between upper A-type (\forme{tɤ\trt}, \forme{lɤ\trt}, \forme{kɤ\trt}, \forme{jɤ-}) and D-type preverbs (\forme{to\trt}, \forme{lo\trt}, \forme{ko\trt}, \forme{jo-}) is neutralized when followed by the inverse prefix (see \tabref{tab:A.D.inv}, §\ref{sec:kamnyu.preverbs}), so that transitive verbs without stem II alternation selecting the upper orientations have syncretism between Aorist and Inferential. For instance, the phonetic form \phonet{kóɣndo} is ambiguous between the Aorist 3$'$\fl{}3 \forme{kɤ́-wɣ-ndo} (\textsc{aor}-\textsc{inv}-take) and the Inferential 3$'$\fl{}3 \forme{kó-wɣ-ndo} (\textsc{aor}-\textsc{inv}-take), both translatable as `someone/it/s/he grabbed him/her'.

\subsubsection{Evidentiality}   \label{sec:ifr.evd}
In main clauses, the Inferential is used to express past perfective event, like the Aorist (§\ref{sec:aor.main}). The contrast between these two categories is of an evidential nature: the Aorist is selected if the speaker had directly witnessed the event, while the Inferential occurs when only indirect clues allow him/her to deduce that the action has taken place.

For instance, (\ref{ex:tAjpa.kalAt2}) in the Aorist can be uttered if the speaker has seen the snowfall, while (\ref{ex:tAjpa.kolAt}) in the Inferential is chosen if the speaker infers that a snowfall has taken place from the presence of snow on the ground.

\begin{exe}
\ex 
\begin{xlist}
\ex \label{ex:tAjpa.kalAt2}
\gll tɤjpa ka-lɤt \\
snow \textsc{aor}:3\fl{}3-release \\
\ex \label{ex:tAjpa.kolAt}
\gll tɤjpa ko-lɤt \\
snow \textsc{ifr}-release \\
\end{xlist}
\glt `It snowed.' (elicited, see \ref{ex:jWfCWCAr.tAjpa.kalAt} above)
\end{exe}

In the case of predicates involving a change of state, selecting the Aorist is only possible if the speaker has witnessed the whole process. For instance, to express the meaning `the water boiled', the Aorist in (\ref{ex:tWci.tala}) is possible only if the speaker has observed the change of phase of water to ebullition, while the Inferential form (\ref{ex:tWci.tokAlaci}) is used when the s/he notices that the water has already started boiling.

\begin{exe}
\ex 
\begin{xlist}
\ex \label{ex:tWci.tala}
\gll tɯ-ci tɤ-ala \\
\textsc{indef}.\textsc{poss}-water \textsc{aor}-boil  \\
\ex \label{ex:tWci.tokAlaci}
\gll tɯ-ci to-k-ɤla-ci  \\
\textsc{indef}.\textsc{poss}-water \textsc{ifr}-\textsc{peg}-boil-\textsc{peg} \\
\end{xlist}
\glt `The water boiled.' (elicited)
\end{exe}

In narratives concerning the speaker, the Aorist is used in the case of actions that s/he has directly seen, while the Inferential is chosen for events that s/he has not directly perceived. For instance, in (\ref{ex:qandZGi.pjAGi}), the speaker selects the Inferential \forme{pjɤ-ɣi} to describe the coming of the falcon, as she had not noticed the presence of that bird until the piece of meat in her hand was snatched away. Selecting the Aorist \forme{pɯ-ɣe} (\textsc{aor}:\textsc{down}-come[II]) `it came down' instead would mean that the speaker had seen the falcon approaching.  The Aorist \forme{ta-nɯ-mɟa} `it took it away' expresses that the speaker felt and saw the meat being taken away; choosing  Inferential \forme{to-nɯ-mɟa} instead would have implied that the speaker had not even noticed the snatching event, and had only realized the disappearance of the meat after it had been taken away.


\begin{exe}
\ex \label{ex:qandZGi.pjAGi}
\gll   kʰɤxtu ri pɯ-rɤʑi-a tɕe tɤ-mtʰɯm tu-ndze-a pɯ-ŋu ri, toʁde tɕendɤre qandʑɣi pjɤ-ɣi tɕe, nɯra mɯ-pɯ-tso-a tɕe, ndɤre a-jaʁ tɤ-mtʰɯm nɯnɯ ta-nɯ-mɟa tɕe \\
terrace \textsc{loc} \textsc{pst}.\textsc{ipfv}-stay-\textsc{1sg} \textsc{lnk} \textsc{indef}.\textsc{poss}-meat \textsc{ipfv}-eat[III]-\textsc{1sg} \textsc{pst}.\textsc{ipfv}-be \textsc{lnk} suddenly \textsc{lnk} falcon \textsc{ifr}:\textsc{down}-come \textsc{lnk} \textsc{dem}:\textsc{pl} \textsc{neg}-\textsc{aor}-understand-\textsc{1sg} \textsc{lnk} \textsc{lnk} \textsc{1sg}.\textsc{poss}-hand \textsc{indef}-meat \textsc{dem} \textsc{aor}:3\flobv{}:\textsc{up}-\textsc{auto}-take \textsc{lnk} \\
\glt `I was on the terrace eating meat, and suddenly a falcon came down without me noticing, and took away the (piece of) meat in my hand.' (150909 qandZGi, 5-7)
 \end{exe}



In retellings of narratives observed on film, such as the \textit{Pear stories}, the Aorist is used for most events that have appeared in the video (§\ref{sec:aor.main}). The Inferential is used when only the result of action is visible. For instance in (\ref{ex:RnWkuxtCo.tosWmtshAt}), the Inferential \forme{to-sɯ-mtsʰɤt} instead of the Aorist \forme{ta-sɯ-mtsʰɤt} occurs because the filling process is already completed in the beginning of the \textit{pear story} video.

\begin{exe}
\ex \label{ex:RnWkuxtCo.tosWmtshAt}
\gll tɕendɤre ʁnɯ-kuxtɕo to-sɯ-mtsʰɤt tɕe \\
\textsc{lnk} two-basket \textsc{ifr}-\textsc{caus}-be.full \textsc{lnk} \\
\glt `(The man) had filled two baskets (with the pears).' (chen-pear, 2)
 \end{exe}

The Inferential can also be used in a more subtle way: to express that one of the characters in the film has not witnessed an event, even though the narrative may have seen it. For instance, the stealing of the pears is described using the Aorist \forme{ja-nɯ-tsɯm} `he took them away' in (\ref{ex:tWkuxtCo.janWtsWm}) since it is visible on the video, but when describing the point of view of the old man discovering that the pears have disappeared when climbing down his ladder, the Inferential \forme{jo-nɯ-tsɯm} `he took them away' occurs instead.

\begin{exe}
\ex \label{ex:tWkuxtCo.janWtsWm}
\gll nɯnɯ kɯ ɯ-paχɕi tɯ-kuxtɕo nɯ ja-nɯ-tsɯm \\
\textsc{dem} \textsc{erg} \textsc{3sg}.\textsc{poss}-apples one-basket \textsc{dem} \textsc{aor}:3\flobv{}-\textsc{auto}-take.away \\
\glt `(The boy) took away one basketful of pears.' (chen-pear, 5)
 \end{exe}
 
\begin{exe}
\ex \label{ex:jonWtsWm.Cti}
\gll rgɤtpu nɯ pɯ-ɬoʁ ri, tɕe pjɤ-sɯχsɤl ri jo-nɯ-tsɯm ɕti tɕe ɯ-kɤpa maŋe,  \\
old.man \textsc{dem} \textsc{aor}:\textsc{down}-come.out \textsc{lnk} \textsc{lnk} \textsc{ifr}-realize \textsc{lnk} \textsc{ifr}-\textsc{auto}-take.away be.\textsc{aff}:\textsc{fact} \textsc{lnk} \textsc{3sg}.\textsc{poss}-method not.exist:\textsc{sens} \\
\glt `When the old man came down from the tree, he realized that (the pears) had been taken away, but could not do anything about it.' (chen-pear, 14)
 \end{exe}

The Inferential can occur to express events seen in dreams, as in (\ref{ex:pWGAjmNota2}).

\begin{exe}
\ex \label{ex:pWGAjmNota2}
 \gll [aʑo [...] qartsʰi ɲɤ-k-ɤpa-a-ci] pɯ-ɣɤjmŋo-t-a \\
\textsc{1sg} {  } cricket \textsc{ifr}-\textsc{peg}-become-\textsc{1sg}-\textsc{peg} \textsc{aor}-dream-\textsc{pst}:\textsc{tr}-\textsc{1sg} \\
\glt `I dreamed that I had become a cricket.' (150904 cuzhi-zh, 193)
\end{exe}

The Inferential is the main TAME category to describe actions occurring in traditional stories, though some speakers rather prefer the Periphrastic Narrative construction (§\ref{sec:aor.narrative}). Example (\ref{ex:chArAGdWt.chAtsxWB}) illustrates this narrative function, where the succession of the verbs in the Inferential reflects the relative temporal order of the actions. The choice of the Inferential rather than the Aorist here is motivated by the fact that the fictional events described in these stories have not been witnessed by the speaker. 


\begin{exe}
\ex \label{ex:chArAGdWt.chAtsxWB}
 \gll ta-ʁi nɯ kɯ nɯŋa ɯ-ndʐi nɯnɯ \rouge{cʰɤ-rɤɣdɯt} qʰendɤre \rouge{cʰɤ-tʂɯβ}. tɕendɤre nɤki, ta-mar tɯ-tɯɣɟaβ nɯ \rouge{to-ndo} qhe, ɯ-ŋgɯ nɯtɕu \rouge{ko-ʑɣɤ-mpʰɯr}. \\
 \textsc{indef}.\textsc{poss}-younger.sibling \textsc{dem} \textsc{erg} cow \textsc{3sg}.\textsc{poss}-skin \textsc{dem} \textsc{ifr}-peel.skin \textsc{lnk} \textsc{ifr}-sew \textsc{lnk} \textsc{filler} \textsc{indef}.\textsc{poss}-butter one-\textsc{nmlz}:\textsc{action}-churn \textsc{dem} \textsc{ifr}-take \textsc{lnk} \textsc{3sg}.\textsc{poss}-inside \textsc{dem}.\textsc{loc} \textsc{ifr}-\textsc{refl}-wrap \\
 \glt `The younger brother skinned the hide of the cow and sewed it. He took one churnfull of butter, and wrapped himself inside (the hide).' (07-deluge, 14-15)
\end{exe}

The Inferential Imperfective occurs for imperfective events and states in narratives (§\ref{sec:pst.ifr.ipfv.main}).

In narratives told in the Inferential, the Aorist is restricted to temporal (§\ref{sec:aor.temporal}) and relative (§\ref{sec:aor.relative}) subordinate clauses.

\subsubsection{Inferential with first person} \label{sec:inf.1person}
The Inferential is not rare with first person subjects in assertive clauses, but has specific meanings.  With volitional verbs, this combination can be uttered when a speaker notices that s/he forgot to do or did not properly do an action. For instance, in (\ref{ex:mWtosAlata}), the speaker (Tshendzin) selects the Inferential when realizing that she forgot to put the water to boil (see \ref{ex:mWtotata}, §\ref{sec:ifr.morphology} for a similar example).

\begin{exe}
\ex \label{ex:mWtosAlata}
\gll tɯ-ci mɯ-to-sɯ-ɤla-t-a \\
\textsc{indef}.\textsc{poss}-water \textsc{neg}-\textsc{ifr}-\textsc{caus}-be.boiling-\textsc{pst}:\textsc{tr}-\textsc{1sg} \\
\glt `I did not put the water to boil.' (Conversation, 01-05-2018, Tshendzin)
\end{exe}

The Inferential with first person is also found when the speaker realizes a fact that s/he had failed to notice or not fully understood before.

For instance, in (\ref{ex:koNla.YAkAtWGAci}), Inferential \textsc{1sg} \forme{ɲɤ-k-ɤtɯɣ-a-ci} `I have met' occurs in a sentence uttered when the speaker has ascertained that the person he has met is a Daoist master, after a long conversation.  

\begin{exe}
\ex \label{ex:koNla.YAkAtWGAci}
\gll  a-kʰi ma kɯki koŋla nɯ ɲɤ-k-ɤtɯɣ-a-ci \\
\textsc{1sg}.\textsc{poss}-luck \textsc{lnk} \textsc{dem}.\textsc{prox} real \textsc{dem} \textsc{ifr}-\textsc{peg}-meet-\textsc{1sg}-\textsc{peg} \\
\glt `I am lucky, I (finally) met a real (Daoist master).' (150907 laoshandaoshi-zh, 37)
\end{exe}

Similarly, in (\ref{ex:WRjoR.YAkABzuaci}), the speaker (a horse) uses the Inferential \forme{ɲɤ-k-ɤβzu-a-ci} `I have become' (as opposed to the Aorist \forme{nɯ-aβzu-a}) to express his sudden realization that it has been tricked into becoming a domestic animal.

\begin{exe}
\ex \label{ex:WRjoR.YAkABzuaci}
\gll tɯrme ɣɯ ɯ-ʁjoʁ ɲɤ-k-ɤβzu-a-ci \\
man \textsc{gen} \textsc{3sg}.\textsc{poss}-servant \textsc{ifr}-\textsc{peg}-become-\textsc{1sg}-\textsc{peg} \\
\glt `I have become a slave of the man.' (aesop ma he lu-zh, 29)
\end{exe}

Example (\ref{ex:mWtozwara}) illustrates the contrast between Aorist and Inferential with first person subjects: the speaker did put the water on the oven, but forgot to open the oven, hence the use of the Inferential for the second verb.

\begin{exe}
\ex \label{ex:mWtozwara}
\gll tɯ-ci kɤ-ta-t-a ri, <dian> mɯ-to-zwar-a \\
\textsc{indef}.\textsc{poss}-water \textsc{aor}-put-\textsc{pst}:\textsc{tr}-\textsc{1sg} \textsc{lnk} electricity \textsc{neg}-\textsc{ifr}-burn-\textsc{1sg} \\
\glt `I put the water (on the oven), but did not open the electricity.' (Conversation, 04-05-2018, Tshendzin)
\end{exe}

The Inferential with first person can also be used when the speaker did the action he intended but on the wrong object, as in (\ref{ex:koGAdZama.YWmaR}), a sentence said after Tshendzin realized (by looking into the pot) that she mistakenly warmed the wrong pot (not the one containing nettles). Here \japhug{mtsʰalu}{nettle} is focalized using the copula \forme{ɲɯ-maʁ} (§\ref{sec:focalization.final.copula}).

\begin{exe}
\ex \label{ex:koGAdZama.YWmaR}
\gll  mtsʰalu ko-ɣɤ-ndʑam-a ɲɯ-maʁ \\
nettle \textsc{ifr}-\textsc{caus}-be.warm-\textsc{1sg} \textsc{sens}-not.be \\
\glt `It is not the nettles that I warmed.' (Conversation, 07-05-2018, Tshendzin)
\end{exe}


The Inferential with first person is particularly common with verbs expressing uncontrollable and non-volitional actions, such as (\ref{ex:kAnWZWB.kordala}). 

\begin{exe}
\ex \label{ex:kAnWZWB.kordala}
\gll kɤ-nɯʑɯβ ko-rdal-a \\
\textsc{inf}-sleep \textsc{ifr}-overshoot-\textsc{1sg} \\
\glt `I overslept.' (elicitation)
\end{exe}

The inferential does not however express by itself non-volitionality; the autive prefix \forme{-nɯ-} (§\ref{sec:autoben.spontaneous}) is used in conjunction with the inferential to insist on the non-volitional character of a particular action, as in (\ref{ex:YAnWjmWta}).

\begin{exe}
\ex \label{ex:YAnWjmWta}
\gll ɲɤ-nɯ-jmɯt-a \\
 \textsc{ifr}-\textsc{auto}-forget-\textsc{1sg} \\
\glt `I forgot.' (many attestations)
\end{exe}


With non-volitional perception verbs such as \japhug{mto}{see} and \japhug{mtsʰɤm}{hear}, Inferential first person negative can be employed to express failure to perceive (\ref{ex:jAtWlAt.mWpjAmtshama}) (see also example \ref{ex:khamu.pasWBzua}, §\ref{sec:pst.ifr.ipfv.main}), or alternatively to state that the speaker has not witnessed a fact of doubtful truthfulness (\ref{ex:aNkhoR.mWpjAmtota}).

\begin{exe}
\ex \label{ex:jAtWlAt.mWpjAmtshama}
\gll jɯfɕɯr a-<dianhua> jɤ-tɯ-lɤt ri mɯ-pjɤ-mtsʰam-a, kʰa pɯ-a-ta tɕe  \\
yesterday \textsc{1sg}.\textsc{poss}-telephone  \textsc{ifr}-2-release \textsc{lnk} \textsc{neg}-\textsc{ifr}-hear-\textsc{1sg} house \textsc{pst}.\textsc{ipfv}-\textsc{pass}-put \textsc{lnk} \\
\glt `Yesterday when you called (me) on the phone, I did not hear it, as (I was away and) had left the phone at home.' (conversation, 2015-06-18)
\end{exe}


\begin{exe}
\ex \label{ex:aNkhoR.mWpjAmtota}
\gll nɤʑo kɯ-fse a-ŋkʰor nɯ mɯ-pjɤ-mto-t-a \\
\textsc{2sg} \textsc{sbj}:\textsc{pcp}-be.like \textsc{1sg}.\textsc{poss}-subject \textsc{dem} \textsc{neg}-\textsc{ifr}-see-\textsc{pst}:\textsc{tr}-\textsc{1sg} \\
\glt `I have never seen anyone like you among my subjects.' (Smanmi 2003-2, 347)
\end{exe}


\subsubsection{Change of state}   \label{sec:ifr.inchoative}
When used with stative verbs, the Inferential Perfective expresses change of state, like the Aorist (§\ref{sec:aor.inchoative}) and the Imperfective (§\ref{sec:ipfv.inchoative}). In (\ref{ex:tompCAr}), \japhug{mpɕɤr}{be beautiful} thus means `become beautiful' in the Inferential.

\begin{exe}
\ex \label{ex:tompCAr}
\gll mɤʑɯ ʑo to-mpɕɤr \\
even.more \textsc{emph} \textsc{ifr}-be.beautiful \\
\glt `(The Phoenix) became even more beautiful than before.' (150901 bainiaochaofeng-zh, 76)
\end{exe}

The verb \japhug{cʰa}{can}, which normally selects the \textsc{downwards} orientation (§\ref{sec:pst.ifr.ipfv.morphology}) to express both Imperfective Inferential/Past Imperfective (`was able to do $X$') and Inferential/Aorist (`succeeded in doing $X$'), has an inchoative meaning `became able to do $X$' when occurring with the \textsc{upwards} orientation, as shown by (\ref{ex:YWnWqambWmbjom.tocha}) (see also \ref{ex:tAnANkWNkea.tAchaa}, §\ref{sec:TAM.finite}).

\begin{exe}
\ex \label{ex:YWnWqambWmbjom.tocha}
\gll mɯrmɯmbju ɯ-pɯ nɯnɯ [...] ɯ-tɯɣmaz ra to-mna tɕe, ɲɯ-nɯqambɯmbjom to-cʰa. \\
swallow \textsc{3sg}.\textsc{poss}-little.one \textsc{dem} { } \textsc{3sg}.\textsc{poss}-wound \textsc{pl} \textsc{ifr}-be.better \textsc{lnk} \textsc{ipfv}-fly \textsc{ifr}-can \\
\glt `The swallow's wounds got better, and it became able to fly.' (150825 huluwa-zh, 44-45)
\end{exe}

\subsubsection{Subordinate clauses}  \label{sec:pst.ifr.subordinate}
The Inferential rarely appears in subordinate clauses, as in most contexts the contrast between Aorist and Inferential is neutralized, and only the former is attested. In particular, in finite relative clauses, the Inferential is not found (§\ref{sec:aor.relative}). 

In complement clauses, the Inferential is only found when the verb of the main clause is also in the Inferential, as in (\ref{ex:CkonARdAn.pjAcha}). 

%qajdo nɯ kɯ tɯ-ci ko-tshi pjɤ-cha tɕe 

\begin{exe}
\ex \label{ex:CkonARdAn.pjAcha}
 \gll  ɲimawozɤr nɯ kɯ, [srɯnmɯ nɯ pjɤ-ftɯl], [...] [rŋgɯ kɤ-kɯ-nɤχtɕɤn ra pjɤ-ftɯl],
tɕe iɕqʰa srɯnmɯ nɯ ɣɯ ɯ-kɯ-ra nɯra [smɤnmimitoʁkuɕana ri ɕ-ko-nɤʁdɤn] pjɤ-cʰa \\
\textsc{anthr} \textsc{dem} \textsc{erg} râkshasî \textsc{dem} \textsc{ifr}-subdue { } boulder \textsc{aor}-\textsc{sbj}:\textsc{pcp}-be.fierce \textsc{pl} \textsc{ifr}-subdue \textsc{lnk} \textsc{filler} râkshasî \textsc{dem} \textsc{gen} \textsc{3sg}.\textsc{poss}-\textsc{sbj}:\textsc{pcp}-be.needed \textsc{dem}:\textsc{pl}  \textsc{anthr} also \textsc{tral}-\textsc{ifr}:\textsc{east}-invite \textsc{ifr}-can \\
\glt `Nyima 'Odzer had subdued the râkshasî, subdued the magical boulders, and also succeeded in inviting Smanmi Meto Koshana to the east, what the râkshasî had requested. (2011-04-smanmi, 261-263)
\end{exe}

In this sentence, the scope of the verb \forme{pjɤ-cʰa} is ambiguous: it could be restricted to the last verb \forme{ɕ-ko-nɤʁdɤn}, but could also be understood as encompassing the first two clauses (whose main verb is \forme{pjɤ-ftɯl}).

Examples of Inferential in the protasis of conditional constructions are presented in §\ref{sec:real.conditional} (example \ref{ex:Wtope.nA}).

\subsection{Past Imperfective and Inferential Imperfective} \label{sec:pst.ifr.ipfv}

\subsubsection{Morphology} \label{sec:pst.ifr.ipfv.morphology}
The Past Imperfective and Inferential Imperfective are built exactly in the same way as Aorist (§\ref{sec:aor.morphology}) and Inferential Perfective (§\ref{sec:ifr.morphology}), but with the \textsc{downwards} preverbs \forme{pɯ-} and \forme{pjɤ-} instead of the preverb corresponding to the intrinsic lexicalized orientation, a peculiarity observed in most Gyalrong varieties \citep{lin11direction}. Thus, verbs selecting \textsc{downwards} as their intrinsic orientation present syncretism between Aorist and Past Imperfective, and between Inferential and Imperfective Inferential.

For instance, the form \forme{pɯ-rom} of the verb \japhug{rom}{be dry}, which selects the \textsc{downwards} orientation, can either be interpreted as an Aorist `it became dry/when it becomes dry' (§\ref{sec:aor.morphology}) or as a Past Imperfective `it was dry'.

Another case of syncretism is provided by the modal verb \japhug{cʰa}{can} which takes the \textsc{downwards} orientation to express the meaning `succeed in doing $X$' (where $X$ refer to the content of the complement clause), as the Inferential Perfective \forme{pjɤ-cʰa} in (\ref{ex:chAsWGjGAt.pjAcha}).

\begin{exe}
\ex \label{ex:chAsWGjGAt.pjAcha}
\gll [zdɯm kɯ-ɲaʁ nɯ cʰɤ-sɯ-jɣɤt] pjɤ-cʰa ɲɯ-ŋu. \\
cloud \textsc{sbj}:\textsc{pcp}-be.black \textsc{dem} \textsc{ifr}:\textsc{downstream}-\textsc{cause}-turn.back \textsc{ifr}-can \textsc{sens}-be \\
\glt `He succeeded in making the black cloud turn back.' (25-kAmYW, 70)
\end{exe}

When the verb of the complement clause is in the Imperfective, the same form \forme{pjɤ-cʰa} is rather an Inferential Imperfective, and means `s/he was able to do $X$' instead, as in (\ref{ex:mWpjAchandZi}).

\begin{exe}
\ex \label{ex:mWpjAchandZi}
\gll cʰɯ-mɤɕi-ndʑi mɯ-pjɤ-cʰa-ndʑi \\
\textsc{ipfv}-be.rich-\textsc{du} \textsc{neg}-\textsc{ifr}.\textsc{ipfv}-can-\textsc{du} \\
\glt `They were unable to become rich.' (Divination, 7)
\end{exe}

In Japhug, not all verbs have Past and Inferential Imperfective forms. Only stative verbs (including adjectives, existential verbs, copulas and passive verbs, §\ref{sec:passive.stative}), some stative transitive verbs with (such as tropative verbs, §\ref{sec:tropative.pst.ipfv}) and some atelic intransitive dynamic verbs (such as \japhug{rɤʑi}{stay}) are compatible with these two TAME categories in main clauses. \tabref{tab:pfv.ipfv.ifr} provides examples of these three categories of verbs, with minimal pairs taken from the text corpus.

\begin{table}
\caption{Examples of contrast between Inferential Perfective and Imperfective}\label{tab:pfv.ipfv.ifr}
\begin{tabular}{llll}
\lsptoprule
Type&Inferential & Inferential \\
 &Perfective & Imperfective \\
\midrule
Stative &\forme{to-mpɕɤr}&\forme{pjɤ-mpɕɤr} \\
& `s/he became beautiful'& `s/he was beautiful' \\
\tablevspace
Tropative&\forme{ɲɤ-nɤ-mpɕɤr} `s/he found &\forme{pjɤ-nɤ-mpɕɤr} `s/he was finding \\
& him/her/it beautiful'&  him/her/it beautiful' \\
\tablevspace
Atelic & \forme{ko-rɤʑi} & \forme{pjɤ-rɤʑi} \\
dynamic &`s/he stayed (there)' &`s/he was staying (there)' \\
\lspbottomrule
\end{tabular}
\end{table}

Atelic dynamic intransitive verbs include the following: some verbs of location such as \japhug{rɤʑi}{stay}, some modal verbs such \japhug{rga}{like} and \japhug{cʰa}{can}, antipassive verbs (§\ref{sec:antipassive.pst.ipfv}) and also some verbs expressing activities requiring a certain amount of time such as \japhug{taʁ}{weave} or \japhug{rɤma}{work}.


Like the Aorist and the Perfective Inferential, the Past and Inferential Imperfective require the past suffix \forme{-t} in the 1/2\fl{}3 forms of open syllable stem transitive verbs (§\ref{sec:suffixes}, §\ref{sec:other.TAME}). However, since only very few transitive verbs are compatible with these two categories, relevant examples such as (\ref{ex:pWnApeta}) are very rare (see also \ref{ex:apWNu.mWpWnWrputa}, §\ref{sec:pst.ifr.ipfv.apodosis}).

\begin{exe}
\ex \label{ex:pWnApeta}
\gll pɯ-nɤ-pe-t-a \\
\textsc{pst}.\textsc{ipfv}-\textsc{trop}-be.good-\textsc{pst}:\textsc{tr}-\textsc{1sg} \\
\glt `I used to like it.' (elicited)
\end{exe}


Most transitive verbs need to take the progressive \forme{asɯ-} prefix (§\ref{sec:progressive}) build Past and Inferential Imperfective forms. The peg circumfix \forme{k-...-ci} is inserted between the \forme{pjɤ-} preverb and the progressive prefix (§\ref{sec:peg.circumfix}) in first and third person forms (§\ref{sec:ifr.morphology}), as shown by (\ref{ex:mWpjAkAsWtsxWBci}). The past transitive suffix \forme{-t} does not occur with the progressive (§\ref{sec:progressive}).

\begin{exe}
\ex \label{ex:mWpjAkAsWtsxWBci}
\gll  rgɤnmɯ nɯ kɯ li iɕqʰa <yuwang> nɯ pjɤ-k-ɤsɯ-tʂɯβ-ci \\
old.woman \textsc{dem} \textsc{erg} again the.aforementioned net \textsc{dem} \textsc{ifr}.\textsc{ipfv}-\textsc{peg}-\textsc{prog}-sew-\textsc{peg} \\
\glt `The old woman was sewing nets (like before).' (140430 yufu he tade qizi-zh, 297)
\end{exe}

One context however where all verbs appear to be found without restriction in the Past Imperfective is in the apodosis of counterfactual conditionals (§\ref{sec:pst.ifr.ipfv.apodosis}).

In other cases, the Periphrastic Past and Inferential Imperfective (§\ref{sec:pst.ifr.ipfv.periphrastic}) are used instead.
 
\subsubsection{Main clauses} \label{sec:pst.ifr.ipfv.main}
In main clauses, the Past and Inferential Imperfective indicate a previous state or habitual situation. For instance, in (\ref{ex:pWtu.pWdAn}), the use of the Past Imperfective to refer to the presence of warts (\forme{pɯ-tu} `there used to be' and \forme{pɯ-dɤn} `there were many') is necessary because of their subsequent disappearance, explicitly mentioned in the following clauses.

\begin{exe}
\ex \label{ex:pWtu.pWdAn}
\gll a-jaʁ ri li kɯkɯra ntsɯ pɯ-tu tɕe pɯ-dɤn. tɕeri nɯ ɯ-qʰu li iɕqʰa stu kɯ-mɤku nɯ-kɯ-ɬoʁ nɯ-nɯ-me qʰe tɕe ɲɤ-nɯ-me \\
\textsc{1sg}.\textsc{poss}-hand \textsc{loc} again \textsc{dem}.\textsc{prox}:\textsc{pl} always \textsc{pst}.\textsc{ipfv}-exist \textsc{lnk} \textsc{pst}.\textsc{ipfv}-be.many \textsc{lnk} \textsc{dem} \textsc{3sg}.\textsc{poss}-after again \textsc{filler} \textsc{most} \textsc{sbj}:\textsc{pcp}-be.first \textsc{aor}-\textsc{sbj}:\textsc{pcp}-come.out \textsc{aor}-\textsc{auto}-not.exist \textsc{lnk} \textsc{lnk} \textsc{ifr}-\textsc{auto}-not.exist \\
\glt `I had (warts) on my hand there, and there were a lot. But later, the (wart) that had first come out disappeared by itself, and (the rest) disappeared.' (24-pGArtsAG, 49-50)
\end{exe}

The Past and Inferential Imperfective also express an ongoing process in the past, during which additional events have occurred, as in (\ref{ex:khamu.pasWBzua}).

\begin{exe}
\ex \label{ex:khamu.pasWBzua}
\gll jɯfɕɯr kʰɯ\redp{}kʰro ʑo jo-tɯ-lɤt ri mɯ-pjɤ-mtsʰam-a ma kʰamu pɯ-asɯ-βzu-a   \\
yesterday \textsc{emph}\redp{}much \textsc{emph} \textsc{ifr}-2-release \textsc{lnk} \textsc{neg}-\textsc{ifr}-hear-\textsc{1sg} \textsc{lnk} cooking \textsc{pst}.\textsc{ipfv}-\textsc{prog}-make-\textsc{1sg} \\
\glt `Yesterday you called (me) many times (on the phone), but I did not hear it, as I was cooking.' (conversation, 14-12-2018)
\end{exe}


In combination with the Autive, the Past Imperfective can convey permansive meaning (§\ref{sec:autoben.permansive}), as in (\ref{ex:pWnWGAwu}).

\begin{exe}
\ex \label{ex:pWnWGAwu}
\gll tɕʰeme nɯ-nɯkʰɤda-t-a ri, mɯ́j-pʰɤn, tɕe pɯ-nɯ-ɣɤwu ɕti \\
girl \textsc{aor}-convince-\textsc{pst}:\textsc{tr}-\textsc{1sg} \textsc{lnk} \textsc{neg}:\textsc{sens}-be.efficient \textsc{lnk} \textsc{pst}.\textsc{ipfv}-\textsc{auto}-cry be.\textsc{aff}:\textsc{fact} \\
\glt `I comforted the girl, but to no avail, she was still crying.' (2003 zrantCWtWrme, 61)
\end{exe}

The Inferential Imperfective is the standard TAME category used in traditional stories to describe states and ongoing actions, as illustrated by the chain of verbs marked in red in (\ref{ex:pjAtu.pjAfsoR.pjAkArNici}). It is the counterpart of the Inferential Perfective in narrative function  (§\ref{sec:ifr.evd}).

\begin{exe}
\ex \label{ex:pjAtu.pjAfsoR.pjAkArNici}
\gll praʁkʰaŋ ci \rouge{pjɤ-tu} qʰe, praʁkʰaŋ ɯ-ŋgɯ nɯtɕu \rouge{pjɤ-fsoʁ}. nɯ ɯ-rkɯ nɯra rcanɯ, li nɤkinɯ, si ra \rouge{pjɤ-k-ɤrŋi-ci} ʑo. nɤkinɯ, mɯntoʁ kɯnɤ wuma ʑo \rouge{pjɤ-dɤn}, tɕe \rouge{pjɤ-sɤ-scit}. [...] ɯ-ŋgɯ lo-ɕe ri, tɕe nɯnɯtɕu rgɤtpu χsɯm kɯ, nɤki, kumbrɤl \rouge{pjɤ-k-ɤsɯ-lɤt-nɯ} \\
cave \textsc{indef} \textsc{ifr}.\textsc{ipfv}-exist \textsc{lnk} cave \textsc{3sg}.\textsc{poss}-inside \textsc{dem}:\textsc{loc} \textsc{ifr}.\textsc{ipfv}-be.bright \textsc{dem} \textsc{3sg}.\textsc{poss}-side \textsc{dem}:\textsc{loc} \textsc{unexp}:\textsc{foc} again \textsc{filler} tree \textsc{pl} \textsc{ifr}.\textsc{ipfv}-\textsc{peg}-be.green-\textsc{peg} \textsc{emph} \textsc{filler} flower also really \textsc{emph} \textsc{ifr}.\textsc{ipfv}-be.many \textsc{lnk} \textsc{ifr}.\textsc{ipfv}-\textsc{prop}-be.happy { } \textsc{3sg}.\textsc{poss}-inside \textsc{ifr}:\textsc{upstream}-go \textsc{lnk} \textsc{lnk} \textsc{dem}:\textsc{loc} old.man three \textsc{erg} \textsc{filler} chess \textsc{ifr}.\textsc{ipfv}-\textsc{peg}-\textsc{prog}-release-\textsc{pl} \\
\glt `There was a cave. Inside the cave, there was light. Around it, the trees were green, and there were many flowers, it was a very nice (place). (...) He entered (the cave), and in there there were three old men playing chess.' (150902 qixian-zh, 111-118)
\end{exe}


\subsubsection{Temporal clauses} \label{sec:pst.ifr.ipfv.temporal}
The Past Imperfective, like the Aorist (§\ref{sec:aor.temporal}), is used in temporal subordinate clauses. The clause in the Past Imperfective expresses an ongoing process in the middle of which the event referred to in the main clause (in the Aorist or in the Inferential Perfective) takes place, as illustrated by (\ref{ex:pasWndzandZi.pWpe}).

\begin{exe}
\ex \label{ex:pasWndzandZi.pWpe}
\gll wuma ʑo pɯ-asɯ-ndza-ndʑi,  pɯ-pe jamar ʑo tɕe tɕendɤre, nɯnɯ, nɤki, kʰa nɯ ɣɯ ɯ-kɯm to-ɲɟɯ\\
really \textsc{emph} \textsc{pst}.\textsc{ipfv}-\textsc{prog}-eat-\textsc{du} \textsc{pst}.\textsc{ipfv}-be.good about \textsc{emph} \textsc{lnk} \textsc{lnk} \textsc{dem} \textsc{filler} house \textsc{dem} \textsc{gen} \textsc{3sg}.\textsc{poss}-door \textsc{ifr}-\textsc{acaus}:open\\
\glt `While/Right at the moment when they were eating (the sweets) in big quantity and were (enjoying it), the door of the house opened.' (140507 tangguowu-zh, 85-87)
\end{exe}

\subsubsection{Apodosis} \label{sec:pst.ifr.ipfv.apodosis}
The Past Imperfective is also used in the apodosis of counterfactual conditionals (with an Irrealis in the protasis, §\ref{sec:irrealis.conditional}), as shown by the form \forme{pɯ-pe} in (\ref{ex:apArAt.pWpe}).

\begin{exe}
\ex \label{ex:apArAt.pWpe}
\gll nɯnɯ mɯɣʑɯsapa nɯ tɕeki nɯ tɕe a-pɯ-ɤ-rɤt tɕe pɯ-pe ma \\ 
\textsc{dem}  \textsc{topo} \textsc{dem} down \textsc{dem} \textsc{loc} \textsc{irr}-\textsc{ipfv}-\textsc{pass}-write \textsc{lnk} \textsc{pst}.\textsc{ipfv}-be.good \textsc{lnk} \\
\glt `It would have been better if (the name) \forme{mɯɣʑɯsapa} had been written down there (on a sheet of paper where many names of locations in the mountains had been written).' (140522  Kamnyu zgo, 82)
\end{exe}

In this very restricted context, all verbs can have a Past Imperfective form. For instance, \japhug{rpu}{bump into} (§\ref{sec:goal.labile}), which normally selects the \textsc{eastwards} orientation (the preverb \forme{kɤ-} in \ref{ex:kAnWrputa}), takes the \forme{pɯ-} Past Imperfective preverb in (\ref{ex:apWNu.mWpWnWrputa}).

\begin{exe}
\ex \label{ex:kAnWrputa}
\gll nɤ-kʰa jɤ-ɣe-a ri, a-ku kɤ-nɯ-rpu-t-a \\
\textsc{2sg}.\textsc{poss}-house \textsc{aor}-come[II]-\textsc{1sg} \textsc{lnk} \textsc{1sg}.\textsc{poss}-head \textsc{aor}-\textsc{auto}-bump-\textsc{pst}:\textsc{tr}-\textsc{1sg} \\
\glt `When I came to your house, I bumped my head (on the door frame).' (elicited)
\end{exe}

\begin{exe}
\ex \label{ex:apWNu.mWpWnWrputa}
\gll nɤ-kʰa jɤ-ɣe-a ri, a-ku pjɯ-pʰaβ-a a-pɯ-ŋu tɕe mɯ-pɯ-nɯ-rpu-t-a. \\
\textsc{2sg}.\textsc{poss}-house \textsc{aor}-come[II]-\textsc{1sg} \textsc{lnk} \textsc{1sg}.\textsc{poss}-head \textsc{ipfv}-lower-\textsc{1sg} \textsc{irr}-\textsc{ipfv}-be \textsc{lnk} \textsc{neg}-\textsc{pst}.\textsc{ipfv}-\textsc{auto}-bump-\textsc{pst}:\textsc{tr}-\textsc{1sg} \\
\glt `When I came to your house, if I had lowered my head, I wouldn't have bumped it (into the door frame).' (elicited)
\end{exe}

The Inferential Imperfective is not attested in this construction.

\subsubsection{Periphrastic Past and Inferential Imperfective} \label{sec:pst.ifr.ipfv.periphrastic}
Only atelic verbs, in particular stative verbs, are compatible with Past Imperfective and Inferential Imperfective (§\ref{sec:pst.ifr.ipfv.morphology}) in contexts other than the apodosis of counterfactuals (§\ref{sec:pst.ifr.ipfv.apodosis}).

To express the meanings otherwise conveyed by the Past Imperfective with telic verbs, two strategies are possible. First, the Progressive \forme{asɯ-} makes all transitive verbs compatible with these two tenses (§\ref{sec:pst.ifr.ipfv.morphology}, §\ref{sec:progressive}). Second, the Periphrastic Past Imperfective and Inferential Imperfective, which combine the Imperfective form of the verb with the copula in the Past Imperfective \forme{pɯ-ŋu} and the Inferential Imperfective \forme{pjɤ-ŋu}, respectively (§\ref{sec:ipfv.periphrastic.TAME}). For instance, instead of the incorrect form $\dagger$\forme{pjɤ-ndza} (Inferential Imperfective of the telic transitive verb \japhug{ndza}{eat}), the periphrastic construction in (\ref{ex:tundze.pjANu2}) is used.


\begin{exe}
\ex \label{ex:tundze.pjANu2}
\gll  tu-ndze pjɤ-ŋu \\
\textsc{ipfv}-eat[III] \textsc{ifr}.\textsc{ipfv}-be \\
\glt `S/he/it used to eat it/was eating it.' (many examples)
\end{exe}


Just in the same way as the Aorist can be combined with the Sensory copula \forme{ɲɯ-ŋu} to build the Periphrastic Narrative (§\ref{sec:aor.narrative}), the Past Imperfective with \forme{ɲɯ-ŋu} expresses an Periphrastic Imperfective Narrative, with the same meaning as the Inferential Imperfective. For instance, the construction in (\ref{ex:pWtaR.YWNu}) is equivalent to the Inferential Imperfective \forme{pjɤ-taʁ} (\textsc{ifr}.\textsc{ipfv}-weave) `s/he was weaving'.

\begin{exe}
\ex \label{ex:pWtaR.YWNu}
\gll tɕʰeme ci pɯ-taʁ ɲɯ-ŋu \\
girl \textsc{indef} \textsc{pst}.\textsc{ipfv}-weave \textsc{sens}-be \\
\glt `A girl was weaving.' (2003 tWxtsa, 29)
\end{exe}

\subsection{Archaic form} \label{sec:khWti}
The archaic \textsc{3sg} \forme{kʰɯ-ti} `s/he said' and \textsc{3pl} \forme{kʰɯ-ti-nɯ} `they said' forms with the isolated preverb \forme{kʰɯ\trt}, are attested in a few occurrences in stories told by Tshendzin's mother. They are glossed by Tshendzin with the Inferential forms \textsc{3sg}\fl{}3 \forme{to-ti} and \textsc{3pl}\fl{}3 \forme{to-ti-nɯ}, respectively. 

\begin{exe}
\ex \label{ex:khWtinW}
\gll ``ɕɯ kɯ nɯ-tɯ́-wɣ-mbi tɤ-ti ma mɤ-jɤɣ" kʰɯ-ti-nɯ. \\
who \textsc{erg} \textsc{aor}-2-\textsc{inv}-give \textsc{imp}-say apart.from \textsc{neg}-be.possible:\textsc{fact} ???-say-\textsc{pl} \\
\glt `Who gave this to you, you must tell us.' (2003-kWBRa, 79)
\end{exe}

These forms are not found in texts by younger speakers. No other trace of this preverb is found in Japhug. A possible origin for it would be the hearsay sentence final particle \forme{kʰi} (§\ref{sec:fsp.hearsay}), which sometimes occur in the reported speech complements (§\ref{sec:reported.speech.sfp}) of the verb \japhug{ti}{say}, sometimes even in direct contact with the stem of this verb as in (\ref{ex:khi.ti}), in its irregular preverbless inferential form (§\ref{sec:fact.periphrastic}).

\begin{exe}
\ex \label{ex:khi.ti}
 \gll ``nɤ-kɤχcɤl nɤ-ʁrɯ nɯ pjɯ-tɯ-pʰɯt tɕe kɤ-nɯ-ɕe ɲɯ-kʰɯ kʰi" ti ɲɯ-ŋu\\
 \textsc{2sg}.\textsc{poss}-top.of.head \textsc{2sg}.\textsc{poss}-horn \textsc{dem} \textsc{ipfv}-2-take.off \textsc{lnk} \textsc{inf}-\textsc{auto}-go \textsc{sens}-be.possible \textsc{sfp} say:\textsc{fact} \textsc{sens}-be \\
 \glt `He said: ``If you remove the horn on the top of your head, it will be possible for you to go wherever you want.''.' (divination 2005,101)
\end{exe}

The preverb \forme{kʰɯ-} would thus result from the procliticization and eventual absorption of a sentence final particle from a preceding clause into the verbal word, a scenario similar to that proposed for the Apprehensive \forme{ɕɯ-} (§\ref{sec:apprehensive.history}). If this hypothesis is correct, the form \forme{kʰɯ-ti} may not be the residual form from an ancient paradigm, but rather an unsuccessful Japhug innovation that has eventually died out.

  \section{Secondary Aspectual categories} \label{sec:second.aspect}
There are two secondary aspectual prefixes in Japhug, the Progressive (§\ref{sec:progressive}) and the Proximative (§\ref{sec:proximative}). They can be combined with several primary TAME categories, but cannot occur together in the same verb form. 

\subsection{Progressive} \label{sec:progressive}

\subsubsection{Morphology} \label{sec:progressive.morphology}
The Progressive prefix \forme{asɯ-} is located in slot -1 of the outer prefixal template (§\ref{sec:outer.prefixal.chain}). It is only compatible with transitive verbs, and does not appear with perfective TAME categories such as Aorist and Inferential Perfective. Its is almost only attested with finite verb forms; the only examples of non-finite Progressive forms are object participles (see example \ref{ex:pWkASWndo}, §\ref{sec:object.participle.other.prefixes}).

The Progressive prefix has six regular allomorphs, as shown in \tabref{tab:progressive.allomorphy}. The monosyllabic variants \forme{az\trt}, \forme{-ɤz-} and \forme{-oz-} are found in the same context as the \forme{z-} allomorph of the sigmatic causative (§\ref{sec:caus.z}): in non-monosyllabic verb bases, when the first syllable has a sonorant initial (\forme{mV\trt}, \forme{nV\trt}, \forme{ɣV-} or \forme{rV-}). They are illustrated with the verb \japhug{nɤjo}{wait} in \tabref{tab:progressive.allomorphy}. The disyllabic allomorphs \forme{asɯ\trt}, \forme{-ɤsɯ-} and \forme{-osɯ-} are found in all other contexts (illustrated with the monosyllabic \japhug{ndza}{eat} in \tabref{tab:progressive.allomorphy}).


\begin{table}
\caption{Allomorphs of the Progressive prefix} \label{tab:progressive.allomorphy}
\begin{tabular}{lllll}
\lsptoprule
Allomorph & Context &Examples \\
\midrule
\forme{asɯ-} &Factual & \forme{asɯ-ndza} \\
&&\textsc{prog}-eat:\textsc{fact} \\
&Past Imperfective& \forme{pɯ-asɯ-ndza} \\
&&\textsc{pst}.\textsc{ipfv}-\textsc{prog}-eat  \\
\midrule
\forme{az-} &Factual   & \forme{az-nɤjo} \\
&&\textsc{prog}-wait:\textsc{fact} \\
&Past Imperfective & \forme{pɯ-az-nɤjo} \\
&&\textsc{pst}.\textsc{ipfv}-\textsc{prog}-wait  \\
\midrule
\forme{ɤsɯ-} &Sensory & \forme{ɲɯ-ɤsɯ-ndza} \\
&& \textsc{sens}-\textsc{prog}-eat  \\
&Inferential& \forme{pjɤ-k-ɤsɯ-ndza-ci} \\
&& \textsc{ifr}.\textsc{ipfv}-\textsc{peg}-\textsc{prog}-eat-\textsc{peg}  \\
\midrule
\forme{ɤz-} &Sensory & \forme{ɲɯ-ɤz-nɤjo} \\
&& \textsc{sens}-\textsc{prog}-wait  \\
&Inferential& \forme{pjɤ-k-ɤz-nɤjo-ci} \\
&& \textsc{ifr}.\textsc{ipfv}-\textsc{peg}-\textsc{prog}-wait-\textsc{peg}  \\
\midrule
\forme{osɯ-} &Egophoric Present & \forme{ku-osɯ-ndza-a} \\
&& \textsc{prs}-\textsc{prog}-eat-\textsc{1sg}  \\
\midrule
\forme{oz-} &Egophoric Present & \forme{ku-oz-nɤjo-a} \\
&& \textsc{prs}-\textsc{prog}-wait-\textsc{1sg}  \\
\lspbottomrule
\end{tabular}
\end{table}

The vocalism of this prefix follows the same alternations as the initial \forme{a-} of contracting verbs (§\ref{sec:contraction}). 

The \forme{asɯ-}/\forme{az-} allomorphs are found in word-initial position, in first or third person Factual Non-Past (see for instance \ref{ex:WmdoR.asWndo}), following the Past Imperfective \forme{pɯ\trt}, with the negative \forme{mɤ-} and the Rhetorical Interrogative \forme{ɯβrɤ-} (§\ref{sec:WBrA}, \forme{ɯβrɤ-asɯ-ndo} \textsc{rh}.\textsc{q}-\textsc{prog}-take:\textsc{fact} `it does not have (this colour), does it?').

The \forme{-osɯ-}/\forme{-oz-} allomorphs are restricted to Egophoric Present forms with the prefix \forme{ku-} (§\ref{sec:egophoric.morphology}).

The \forme{-ɤsɯ-}/\forme{-ɤz-} allomorphs are the most common, attested with the Sensory \hbox{\forme{ɲɯ\trt},} with the peg circumfix (§\ref{sec:peg.circumfix}, in Inferential Imperfective form) and also in the Irrealis Imperfective with the preverb \forme{pɯ-} (for instance \forme{a-pɯ-ɤsɯ-ndo} \textsc{irr}-\textsc{ipfv}-\textsc{prog}-take `may it have (this colour)').

 Like contracting verbs, the Progressive prefix selects the peg circumfix \forme{kɯ-...-ci} (§\ref{sec:peg.circumfix}) in the Inferential Imperfective (§\ref{sec:pst.ifr.ipfv.morphology}), as shown by the form \forme{pjɤ-k-ɤsɯ-ɕar-ci} `it was searching/looking for it' in (\ref{ex:pjAtu.pjAkAsWCarci}).

\begin{exe}
\ex \label{ex:pjAtu.pjAkAsWCarci}
\gll sɯŋgɯ nɯtɕu, nɤkinɯ, wuma ʑo spjaŋkɯ kɯ-mtsɯr ci pjɤ-tu. tɕendɤre nɯnɯtɕu ɯ-kɤ-ndza aʁɤndɯndɤt pjɤ-k-ɤsɯ-ɕar-ci. \\
forest \textsc{dem}:\textsc{loc} \textsc{filler} really \textsc{emph} wolf \textsc{sbj}:\textsc{pcp}-be.hungry \textsc{indef} \textsc{ifr}.\textsc{ipfv}-exist \textsc{lnk} \textsc{dem}:\textsc{loc} \textsc{3sg}.\textsc{poss}-\textsc{obj}:\textsc{pcp}-eat everywhere \textsc{ifr}.\textsc{ipfv}-\textsc{peg}-\textsc{prog}-search-\textsc{peg} \\
\glt `In the forest, there was a wolf, and he was looking for food there everywhere?' (140428 xiaohongmao-zh, 29-30)
\end{exe}

The inverse (slot -1, §\ref{sec:outer.prefixal.chain}) and the autive (inner prefix, §\ref{sec:inner.prefixal.chain}) are \textit{infixed} within the progressive, as shown by (\ref{ex:YWtAwGznWkhramba}) and (\ref{ex:YAnWsWNga})  (see also \ref{ex:pjAkAwGznAjoci} in §\ref{sec:outer.prefixal.chain}, \ref{ex:konWsWndzaj} in §\ref{sec:inner.prefixal.chain} and \ref{ex:panWsWfkur} in §\ref{sec:autoben.position}). 
 
\begin{exe}
\ex \label{ex:YWtAwGznWkhramba}
\gll ɲɯ-tɯ-ɤ́<wɣ>z-nɯkʰramba \\
\textsc{sens}-2-\textsc{prog}<\textsc{inv}>-cheat \\
\glt `S/he is cheating you.' (elicited)
\end{exe} 
 
\begin{exe}
\ex \label{ex:YAnWsWNga}
\gll ɯ-ŋga cʰɤ-ɴɢraʁ ldʐɤβldʐɤβ ʑo ri, ɲɯ-ɤ<nɯ>sɯ-ŋga \\
\textsc{3sg}.\textsc{poss}-clothes \textsc{ifr}-\textsc{acaus}:tear \textsc{idph}(II):in.shreds \textsc{emph} \textsc{lnk} \textsc{sens}-\textsc{prog}<\textsc{auto}>-wear \\
\glt `His clothes are torn to shreds, but he is still wearing them.' (elicited)
\end{exe}

When both inverse and autive are combined with the Progressive, the \forme{-sɯ/z-} element is removed, and only the first vowel of the Progressive remains, as in (\ref{ex:YAwGnWnAjo}).

\begin{exe}
\ex \label{ex:YAwGnWnAjo}
\gll ɯ-zda kɯ ɲɯ-ɤ́-wɣ-nɯ-nɤjo \\
\textsc{3sg}.\textsc{poss}-companion \textsc{erg} \textsc{sens}-\textsc{prog}-\textsc{auto}-wait \\
\glt `His$_i$ companion is waiting for him$_i$.' (elicited)
\end{exe} 

An irregular allomorph \forme{-ɤs-} is found in the Sensory Progressive \forme{ɲɯ-ɤs-tɯt} (\textsc{sens}-\textsc{prog}-say[II])  `s/he is saying/said' of the verb  \japhug{ti}{say} (§\ref{sec:sensory.morphology}), a form also unusual by the presence of Stem II.

Transitive verbs with the Progressive prefix lack some of the morphological exponents of transitivity (§\ref{sec:transitivity.morphology}). 

First, in Sensory, Egophoric, Irrealis and Factual, Stem III alternation does not occur in \textsc{sg}\fl{}3 configurations of alternating verbs (§\ref{sec:stem3.distribution}); for instance, the Progressive Sensory \textsc{3sg}\flobv{} of \japhug{ndza}{eat} is \forme{ɲɯ-ɤsɯ-ndza} (\textsc{sens}-\textsc{prog}-eat) `s/he/it is eating it' with Stem I, unlike the plain Sensory \forme{ɲɯ-ndze} which selects Stem III (see example \ref{ex:YWndze.tundze}, §\ref{sec:ipfv.periphrastic.TAME}). Combining Stem III with the Progressive (something like $\dagger$\forme{ɲɯ-ɤsɯ-ndze}) is categorically rejected by native speakers. 

Second, the past \forme{-t} suffix (§\ref{sec:other.TAME}) does not occur in 1/2\textsc{sg}\fl{}3 forms of Past Imperfective and Inferential Imperfective of verbs with the Progressive prefix. For instance, the \textsc{1sg}\fl{}\textsc{3sg} form \forme{pɯ-asɯ-βzu-a} (\textsc{pst}.\textsc{ipfv}-\textsc{prog}-make-\textsc{1sg}) `I was making it' (see for instance example \ref{ex:khamu.pasWBzua}, §\ref{sec:pst.ifr.ipfv.main}) lacks the \forme{-t} suffix, and inserting it ($\dagger$\forme{pɯ-asɯ-βzu-t-a}) is not accepted by speakers. 

Other exponents of transitivity, such as C-type preverbs (§\ref{sec:indexation.non.local}), subject participles (§\ref{sec:subject.participle.possessive}) and bare infinitives (§\ref{sec:bare.inf}) cannot be tested, due to the incompatibility of the Progressive with most non-Finite forms on the one hand, and the undetectability of the \forme{pɯ-}/\forme{pa-} contrast when followed by the vowel contracting \forme{asɯ-}: the surface form \ipa{pasɯβzu} `s/he/it is making it' is equally analyzable as \forme{pɯ-asɯ-βzu} (\textsc{pst}.\textsc{ipfv}-\textsc{prog}-make) or as \forme{pa-asɯ-βzu} (\textsc{pst}.\textsc{ipfv}:3\flobv{}-\textsc{prog}-make). The former analysis is adopted in this grammar.

The 1\fl{}2 \forme{ta-} seems to be incompatible with the progressive. For instance, to express the meaning `I am waiting for you', only the simple Egophoric Present \forme{ku-ta-nɤjo} (\textsc{prs}-1\fl{}2-wait) is possible, the surface form \ipa{kutaznɤjo} can only be analyzed as a causative Imperfective \forme{ku-ta-z-nɤjo} (\textsc{ipfv}-1\fl{}2-\textsc{caus}-wait) `I will have you wait for him', not as an Egophoric Progressive $\dagger$\forme{ku-ta-ɤz-nɤjo}.

Other person configurations, including inverse mixed scenarios 3\fl{}1/2 (§\ref{sec:indexation.mixed}) and local scenario 2\fl{}1 (§\ref{sec:indexation.local}) can be elicited with the Progressive, as in (\ref{ex:YWtAwGznWkhramba}) and (\ref{ex:YWkAsWzgroRa}), though no such examples have yet been found in the corpus.

\begin{exe}
\ex \label{ex:YWkAsWzgroRa}
\gll ɲɯ-kɯ-ɤsɯ-zgroʁ-a \\
\textsc{sens}-2\fl{}1-\textsc{prog}-attach-\textsc{1sg} \\
\glt `You are attaching me.' (elicited)
 \end{exe}
 
\subsubsection{Functions} \label{sec:progressive.function}
The Progressive indicates the non-telicity of the verbal action. It occurs in the case of ongoing actions (including events be protracted for a long time period, as in \ref{ex:kosWCari}), and also habitual actions (\ref{ex:kosWBzjoz}) (see also \ref{ex:jipAri.kosWBzu} and \ref{ex:shangban.kosWBzu} in §\ref{sec:egophoric.tense}).

\begin{exe}
\ex \label{ex:kosWCari}
\gll srɤz ɣɯ jɯm ku-osɯ-ɕar-i, lu χsɯ-xpa pɯ-ŋke-j pɯ-ra ri  \\
prince \textsc{gen} wife \textsc{prs}-\textsc{prog}-search-\textsc{1pl} year three-year \textsc{pst}.\textsc{ipfv}-walk-\textsc{1pl} \textsc{pst}.\textsc{ipfv}-be.needed \textsc{lnk} \\
\glt `We are looking for the prince's wife, we had to walk for three years.'  (2003sras, 58)
 \end{exe}
 
\begin{exe}
\ex \label{ex:kosWBzjoz}
\gll  akɯ <xianzhong> ri <chuzhong> ku-osɯ-βzjoz. \\
east district.school \textsc{loc} junior.high.school \textsc{prs}-\textsc{prog}-learn \\
\glt `She is reading junior high school at the district school.' (14-siblings, 370)
 \end{exe}

With the Factual and the Sensory, the Progressive precludes an imminent future interpretation (§\ref{sec:fact.main.clauses}).  For instance \forme{tɯ-mɯ asɯ-lɤt} in (\ref{ex:tWmW.asWlAt}) means `(when) it is raining' (as part of general knowledge), while the corresponding plain Factual \forme{tɯ-mɯ lɤt} (\ref{ex:qaprAnar.tWmW.lAt}) is to be understood as `it will rain'.


\begin{exe}
\ex \label{ex:tWmW.asWlAt}
\gll ma tɤŋe me nɯtɕu tɕe, zdɯm lu-ɣi ŋu tɕe, nɯ ɯ-ngɯ nɯra, tɯ-mɯ kɯ-xtɕɯ\redp{}xtɕi ʑo asɯ-lɤt tɕe tɕe nɯnɯ βɣɤrtsʰi wuma ʑo dɤn. \\
\textsc{lnk} sun not.exist:\textsc{fact} \textsc{dem}:\textsc{loc} \textsc{lnk} cloud \textsc{ipfv}:\textsc{upstream}-come be:\textsc{fact} \textsc{lnk} \textsc{dem} \textsc{3sg}.\textsc{poss}-in \textsc{dem}:\textsc{pl} \textsc{indef}.\textsc{poss}-sky \textsc{sbj}:\textsc{pcp}-\textsc{emph}\redp{}be.small \textsc{emph} \textsc{prog}-release:\textsc{fact} \textsc{lnk} \textsc{lnk} \textsc{dem} mosquito really \textsc{emph} be.many:\textsc{fact} \\
\glt  `When the sun has not appeared (yet, in the morning), when clouds come, and there is a little rain, the mosquitoes are particularly many.' (25-RmArYWG, 38-39)
\end{exe}
 
\begin{exe}
\ex \label{ex:qaprAnar.tWmW.lAt}
\gll nɯnɯ qaprɤŋar kɯ zdɯm tu-tɕɤt ŋu tɕe tɯ-mɯ lɤt tu-ti-nɯ ŋgrɤl \\
\textsc{dem}  \textsc{topo} \textsc{erg} cloud \textsc{ipfv}-take.out be:\textsc{fact} \textsc{lnk} \textsc{indef}.\textsc{poss}-sky release:\textsc{fact} \textsc{ipfv}-say-\textsc{pl} be.usually.the.case:\textsc{fact} \\
\glt `(The snake from) Qaprangar is releasing clouds, and it will rain.' (140522 Kamnyu zgo-zh, 335)
\end{exe}

The verb \japhug{ndo}{take} requires the Progressive in all its finite forms in the stative collocation \forme{ɯ-mdoʁ+ndo} `have the colour of', as in (\ref{ex:WmdoR.asWndo}) ($\dagger$\forme{ɯ-mdoʁ ndɤm} is not attested).

\begin{exe}
\ex \label{ex:WmdoR.asWndo}
\gll kɯki ɯ-mdoʁ ʑo asɯ-ndo. \\
\textsc{dem}.\textsc{prox} \textsc{3sg}.\textsc{poss}-colour \textsc{emph} \textsc{prog}-take:\textsc{fact} \\
\glt `It has this colour.' (23-mbrAZim, 6)
\end{exe}

The Progressive makes transitive dynamic verbs compatible with the Past Imperfective and Inferential Imperfective (§\ref{sec:pst.ifr.ipfv.morphology}). These forms can express previous habitual actions (`use to $X$') that have ceased to occur, as in (\ref{ex:pasWBzu.pWnWNgra}) and (\ref{ex:pasWndzaa.kosWndzaa}), or past ongoing actions (\ref{ex:jWGi.pasWrtoRa}).

\begin{exe}
\ex \label{ex:pasWBzu.pWnWNgra}
\gll tɕe aʁɤndɯndɤt ʑo <dagong> ntsɯ pɯ-asɯ-βzu tɕe nɤki, pɯ-nɯŋgra ntsɯ ri, japa ri tɕe, [...] <kaoshi> pɯ-cʰa \\
\textsc{lnk} everywhere \textsc{emph} work.for.salary always \textsc{pst}.\textsc{ipfv}-\textsc{prog}-make \textsc{lnk} \textsc{filler} \textsc{pst}.\textsc{ipfv}-work.for.salary always \textsc{lnk} last.year \textsc{loc} \textsc{lnk} {  } examination \textsc{aor}-can \\
\glt `She used to do migrant work everywhere, but last year (...) she succeeded in the exam (to become a civil servant).' (12-BzaNsa, 77)
\end{exe}

\begin{exe}
\ex \label{ex:jWGi.pasWrtoRa}
\gll aʑo kɯkɯre ri jɯɣi pɯ-asɯ-rtoʁ-a ɕti tɕe, maka pɯ-mto-t-a maka me \\
\textsc{1sg} \textsc{dem}:\textsc{loc} \textsc{loc} book \textsc{pst}.\textsc{ipfv}-\textsc{prog}-look-\textsc{1sg} be.\textsc{aff}:\textsc{fact} \textsc{lnk} at.all \textsc{aor}-see-\textsc{pst}:\textsc{tr}-\textsc{1sg} at.all not.exist:\textsc{fact} \\
\glt `I was here reading books, I did not see anything.' (150901 dongguo xiansheng he lang-zh, 60-61)
\end{exe}

Example (\ref{ex:pasWndzaa.kosWndzaa}) illustrates the tense contrast between the Progressive Past Imperfective \forme{pɯ-asɯ-ndza-a} `I was eating/used to eat it' (previous habitual) and the Progressive Egophoric Present \forme{ku-osɯ-ndza-a} `I am eating it' (present habitual).

\begin{exe}
\ex \label{ex:pasWndzaa.kosWndzaa}
\gll woja nɯ ɕɯŋgɯ kɯmaʁ smɤn, nɯnɯ kɯ-fse <gaipian> pɯ-tu nɯnɯ pɯ-asɯ-ndza-a. tɕe nɯ tʰɯ-arɕo koʁmɯz nɤ, nɤʑo kɤ-tɯ-sɯ-ɣɯt nɯ tu-ndze-a. nɤʑo kɤ-tɯ-sɯɣɯt nɯnɯ, <gaipian> ɲɯ-ŋu ɣe, tɕe nɯ ku-osɯ-ndza-a. \\
\textsc{interj} \textsc{dem} before other medicine \textsc{dem} \textsc{sbj}:\textsc{pcp}-be.like calcium.tablet \textsc{pst}.\textsc{ipfv}-exist \textsc{dem} \textsc{pst}.\textsc{ipfv}-\textsc{prog}-eat-\textsc{1sg} \textsc{lnk} \textsc{dem} \textsc{aor}-be.finished.up only.then \textsc{add} \textsc{2sg} aor:east-2-\textsc{caus}-bring \textsc{dem} \textsc{ipfv}-eat[III]-\textsc{1sg}  \textsc{2sg} aor:east-2-\textsc{caus}-bring \textsc{dem} calcium.tablet \textsc{sens}-be \textsc{sfp} \textsc{lnk} \textsc{dem} \textsc{prs}-\textsc{prog}-eat-\textsc{1sg} \\
\glt `Before that I had another medicine, (another) calcium tablet like that, and it was the one I was taking. I (started) taking the (medicine) you have sent me when (the previous one) was finished up. The (medicine) you have sent me, calcium tablet, I am taking it (now).' (conversation 17-08-21)
\end{exe}

Temporal clauses with the Progressive and the relator noun \forme{ɯ-raŋ} `the time when, while' can express simultaneity with the action of the main clause (§\ref{sec:temporal.reference}). In this construction, the verb can be in the Factual even when the main clause is in the Aorist or the Inferential, as in (\ref{ex:aznWrdoRnW.WraN}).

\begin{exe}
\ex \label{ex:aznWrdoRnW.WraN}
\gll tɕe [tɤ-pɤtso ra kɯ az-nɯrdoʁ-nɯ ɯ-raŋ] ʑo tɕe tɕendɤre, ʑɤni daltsɯtsa ɲɤ-ʑɣɤ-sɯ-ɤrqʰi-ndʑi tɕe \\
\textsc{lnk} \textsc{indef}.\textsc{poss}-child \textsc{pl} \textsc{erg} \textsc{prog}-collect:\textsc{fact}-\textsc{pl} \textsc{3sg}.\textsc{poss}-time \textsc{emph} \textsc{loc} \textsc{lnk} \textsc{3du} slowly \textsc{ifr}-\textsc{refl}-\textsc{caus}-be.far-\textsc{du} \textsc{lnk} \\
\glt `While the children were collecting (firewood, piece by piece), (their parents) slowly moved away (from them).' (160630 poucet1, 48)
\end{exe}

When the subordinate and the main clauses share the same subjects, the verb with the Progressive prefix in the subordinate clause can express the manner of the action of the main clauses, as in (\ref{ex:YAsWBzu.YWti}).

\begin{exe}
\ex \label{ex:YAsWBzu.YWti}
\gll a-tɤ-lu to-χtɯ-ndʑi, a-paχɕi ra to-χtɯ-ndʑi, <gongxun> kɯ iʑo ji-skɤt ɲɯ-ɤsɯ-βzu tɕe, `nɤ-tɤ-lu cʰo nɤ-paχɕi tɤ-χtɯ-tɕi' nɯra ɲɯ-ti. \\
\textsc{1sg}.\textsc{poss}-\textsc{indef}.\textsc{poss}-milk \textsc{ifr}-buy-\textsc{du} \textsc{1sg}.\textsc{poss}-apple \textsc{pl} \textsc{ifr}-buy-\textsc{du}  \textsc{anthr} \textsc{erg} \textsc{1pl} \textsc{1pl}.\textsc{poss}-language \textsc{sens}-\textsc{prog}-make \textsc{lnk} \textsc{2sg}.\textsc{poss}-\textsc{indef}.\textsc{poss}-milk \textsc{comit} \textsc{2sg}.\textsc{poss}-apple \textsc{aor}-buy-\textsc{1du} \textsc{dem}:\textsc{pl} \textsc{sens}-say  \\
\glt `They bought milk and apples for me, and Gong Xun (\zh{龚勋}) said, speaking in our language: `We bought milk and apples for you'.' (conversation, 17-08-21)
\end{exe}

\subsubsection{History} \label{sec:progressive.history}
The Japhug Progressive prefix \forme{asɯ-} is obviously related to the Tshobdun \forme{ɐsɐ-} \citep[89]{jackson02rentongdengdi} and the Zbu \forme{ɐsɐ-/ɐsə-} \citep[199--201]{gong18these} progressive prefixes, but no cognates are found in the rest of Gyalrongic, even in Situ. It is a possible northern Gyalrong common innovation.

As in Japhug, the Progressive \forme{ɐsɐ-} in Zbu and Tshobdun is incompatible with Stem III and the past transitive \forme{-z} suffix (cognate to Japhug \forme{-t}, §\ref{sec:other.TAME}), but its distribution is considerably more restricted. Zbu \forme{ɐsɐ-} is not compatible with orientation preverbs and the inverse prefix; the allomorph \forme{ɐsə-} optionally occurs with verbs taking the sigmatic causative prefix (\citealt[199--200]{gong18these}).

The infixability of the inverse and the autive prefixes within the Progressive in Japhug (§\ref{sec:progressive.morphology}) is a clue that \forme{asɯ-} and its cognates are etymologically composite, comprising two elements \forme{a-} and \forme{-sɯ-}. Possible candidates for the former include the Passive (§\ref{sec:passive}) and the Denominal \forme{a-} (§\ref{sec:denom.a}), and for the latter the oblique participle \forme{sɤ-} (§\ref{sec:oblique.participle}) or the sigmatic causative \forme{sɯ-} (§\ref{sec:sig.causative}).


\subsection{Proximative} \label{sec:proximative}
The Proximative \forme{jɯ\trt}, located in slot -6 of the outer prefixal chain (§\ref{sec:outer.prefixal.chain}), corresponds to the Tshobdun \forme{jə-} prefix (Prospective \zh{前瞻体}, \citealt[142–143]{sun08shiti}) and the Zbu \forme{jə-} or \forme{wo-} prefixes (depending on the dialect, \citealt[9;201-202]{gong18these}).

It mainly appears in the corpus in combination with the Aorist (\ref{ex:zWmi.jWnWsi}, \ref{ex:zWmi.jWpWwsata}) or the Inferential (\ref{ex:Wlu.jipjAcW}, \ref{ex:XsWtAxWr.jikoCe}) expressing that the action was almost realized, but not completed. The adverb \japhug{zɯmi}{almost} is commonly used together with the proximative, as in (\ref{ex:zWmi.jWnWsi}) and (\ref{ex:zWmi.jWpWwsata}).

\begin{exe}
\ex \label{ex:zWmi.jWnWsi}
\gll ɯ-nmaʁ nɯ japandʐi ri, wuma ʑo tɤ-ngo tɕe zɯmi ʑo jɯ-nɯ-si \\
\textsc{3sg}.\textsc{poss}-husband \textsc{dem} the.year.before \textsc{loc} really \textsc{emph} \textsc{aor}-be.ill \textsc{lnk} almost \textsc{emph}  \textsc{proxm}-\textsc{aor}-die \\
\glt `Her husband became sick a few years ago, and almost died.' (14-siblings, 356)
  \end{exe}
  
  \begin{exe}
\ex \label{ex:zWmi.jWpWwsata}
\gll zɯmi jɯ-pɯ́-wɣ-sat-a. \\
almost \textsc{proxm}-\textsc{aor}-\textsc{inv}-kill-\textsc{1sg} \\
\glt `He almost killed me.' (150901 dongguo xiansheng he lang-zh, 139-141)
 \end{exe} 
 
\begin{exe}
\ex \label{ex:Wlu.jipjAcW}
\gll ɯ-lu ʑo jɯ-pjɤ-cɯ, tɯ-mu kɯ \\
\textsc{3sg}.\textsc{poss}-faint(1) \textsc{emph} \textsc{proxm}-\textsc{ifr}-faint(2) \textsc{nmlz}:\textsc{action}-fear \textsc{erg} \\
\glt `He almost fainted out of fear.' (150909 hua pi-zh, 73)
\end{exe}

The non-realized action can be deemed undesirable (as in \ref{ex:zWmi.jWnWsi} to \ref{ex:Wlu.jipjAcW} above), but may also be an aim that the subject tried but failed to realize as in (\ref{ex:XsWtAxWr.jikoCe}).

\begin{exe}
\ex \label{ex:XsWtAxWr.jikoCe}
\gll  χsɯ-tɤxɯr jɯ-ko-ɕe ʑo tɕe,  tɕe tɕe, nɯ ma mɯ-ɲɤ-cʰa tɕe \\
three-turn \textsc{proxm}-\textsc{ifr}:\textsc{east}-go \textsc{emph} \textsc{lnk} \textsc{lnk} \textsc{lnk} \textsc{dem} apart.from \textsc{neg}-\textsc{ifr}-can \textsc{lnk} \\
\glt `He almost completed (was about to complete) the third lap, but could not (run) any more.' (2003sras 110-111)
\end{exe}

The Proximative combined with the Factual Non-Past has the meaning `be about to do'. It is generally attested with a copula as in (\ref{ex:jWnWZWB}), like the Periphrastic Proximative (§\ref{sec:proximative.periphrastic}). 

\begin{exe}
\ex \label{ex:jWnWZWB}
\gll jɯ-nɯʑɯβ ʑo ɲɯ-ŋu \\
\textsc{proxm}-sleep:\textsc{fact} \textsc{emph} \textsc{sens}-be \\
\glt `He is about to fall asleep.' (elicited)
\end{exe}

With contracting verbs (§\ref{sec:contraction}) in the Factual Non-Past, the Proximative merges with the contracting vowel as \ipa{ja\trt}, as in (\ref{ex:jatAra.Zo}).

\begin{exe}
\ex \label{ex:jatAra.Zo}
\gll jɯ-atar-a ʑo ɲɯ-ŋu \\
\textsc{proxm}-fall:\textsc{fact}-\textsc{1sg} \textsc{emph} \textsc{sens}-be \\
\glt `I am about to fall down.' (elicited)
\end{exe}


This meaning is also found with participial forms, such as \forme{jɯ-tu-kɯ-wɣrum} `(the one) which is about to become white' in (\ref{ex:jWtukWwGrum}) (see also \ref {ex:YWqAt.nA.kuwum}, §\ref{sec:centripetal.centrifugal}).
 
 \begin{exe} 
\ex \label{ex:jWtukWwGrum}
\gll  kɯ-pɣi ci koŋla ʑo zɯmi jɯ-tu-kɯ-wɣrum kɯ-fse ci ɲɯ-ŋu.  \\
\textsc{sbj}:\textsc{pcp}-be.grey \textsc{indef} completely \textsc{emph} almost \textsc{proxm}-\textsc{ipfv}-\textsc{sbj}:\textsc{pcp}-be.white \textsc{sbj}:\textsc{pcp}-be.like \textsc{indef} \textsc{sens}-be \\
\glt `It is grey, almost like it is about to become white.'  (24-ZmbrWpGa, 34)
\end{exe}

A more marginal meaning of the Proximative in subordinate clauses is `as soon as' (with completion of the verbal action), as in (\ref{ex:jWCkuzonW}).

 \begin{exe} 
\ex \label{ex:jWCkuzonW}
\gll  ɯnɯnɯ ɯ-ndʑɯɣ nɯnɯ a-pɯ-tu tɕe tɕendɤre, pɣa ra jɯ-ɕ-ku-zo-nɯ tɕe tɕe kú-wɣ-ndo-nɯ tɕe pjɯ-si-nɯ pjɯ-ŋgrɤl ɲɯ-ŋu. \\
\textsc{dem} \textsc{3sg}.\textsc{poss}-resin \textsc{dem} \textsc{irr}-\textsc{ipfv}-exist \textsc{lnk} \textsc{lnk} bird \textsc{pl} \textsc{proxm}-\textsc{tral}-\textsc{ipfv}-land-\textsc{pl} \textsc{lnk} \textsc{lnk} \textsc{ipfv}-\textsc{inv}-take-\textsc{pl} \textsc{lnk} \textsc{ipfv}-die-\textsc{pl} \textsc{ipfv}-be.usually.the.case \textsc{sens}-be \\
\glt `If its resin appeared, birds would get stuck by it as soon as they land (on the tree) and would die.' (140427 yanzi yu niaolei-zh, 9)
 \end{exe}
 
%ɯ-taʁ ji-to-rpu ʑo. 150909_hua_pi, 43

 
\subsubsection{Periphrastic proximative} \label{sec:proximative.periphrastic}
The Proximative is in competition with the Periphrastic Proximative constructions, which combines the Factual Non-Past with the \textsc{3sg} copula in various TAME categories.

With the Past Imperfective copula \forme{pɯ-ŋu} (\ref{ex:sat.pjANu}) or the Inferential Imperfective \forme{pjɤ-ŋu} (\ref{ex:lAt.pjANu}), the periphrastic construction means `was about to do $X$, almost/nearly did $X$'.

  \begin{exe} 
\ex \label{ex:sat.pjANu}
\gll ʁdɯxpa naχpu ɣɯ ɯ-tɕɯ kɯ a-tɕɯ sat pɯ-ŋu ri, ɲɤ-tɯ-sɤtɯta-t tɕe  \\
\textsc{anthr}  \textsc{anthr} \textsc{gen} \textsc{3sg}.\textsc{poss}-son \textsc{erg} \textsc{1sg}.\textsc{poss}-son kill:\textsc{fact} \textsc{pst}.\textsc{ipfv}-be \textsc{lnk} \textsc{ifr}-2-separate-\textsc{pst}:\textsc{tr} \textsc{lnk} \\
\glt `The son of Klu gdugpa nagpo was about to kill (almost killed) my son, but you separated them.' (28-smAnmi, 280)
 \end{exe}
 
  \begin{exe} 
\ex \label{ex:lAt.pjANu}
\gll  ɕɤmɯɣdɯ lɤt pjɤ-ŋu ri [...] ɲɤ-sɯso tɕe \\
gun release:\textsc{fact} \textsc{ifr}.\textsc{ipfv}-be \textsc{lnk} { } \textsc{ifr}-think \textsc{lnk} \\
\glt `(The hunter) was about to shoot, but he thought `...'' (140428 xiaohongmao-zh, 145)
 \end{exe}
 
The Aorist form \forme{tɤ-ŋu} occurs in the Periphrastic Proximative construction in temporal clause to fix a point in time (§\ref{sec:aor.temporal}), either in the past or in the future as (\ref{ex:tWwGCaB.tANu}) `when $X$ will be about to $Y$'.

\begin{exe} 
\ex \label{ex:tWwGCaB.tANu}
\gll  nɤʑo tɯ́-wɣ-ɕaβ tɤ-ŋu tɕe, aʑo kɯ ʁe nɯ ɲɯ-ɕtʰɯz-a \\
\textsc{2sg} 2-\textsc{inv}-catch.up:\textsc{fact} \textsc{aor}-be \textsc{lnk} \textsc{1sg} \textsc{erg} left \textsc{dem} \textsc{ipfv}:\textsc{west}-turn.towards-\textsc{1sg} \\ 
\glt `When they will be about to catch up with you, I will turn my left (hand) in their direction. (2011-04-smanmi, 145)
 \end{exe}
 
In the non-past, the Periphrastic Proximative selects the Sensory copula \forme{ɲɯ-ŋu}, and expresses imminent future `is about to $X$' as in (\ref{ex:kWndza.YWNu}) and (\ref{ex:atara.YWNu}). With the copula in the Factual Non-Past, clear examples of proximative meaning are difficult to ascertain, as postverbal copulas have additional functions (§\ref{sec:postverbal.copulas}).

 \begin{exe} 
\ex \label{ex:kWndza.YWNu}
\gll  wo a-wɤmɯ ra mɤ-tɯ-ɣi-nɯ ɯ́-ŋu ma, kɯ-ndza ɲɯ-ŋu \\
\textsc{interj} \textsc{1sg}.\textsc{poss}-brother \textsc{pl} \textsc{neg}-2-come:\textsc{fact}-\textsc{pl} \textsc{qu}-be:\textsc{fact} \textsc{lnk} \textsc{genr}:S/O-eat:\textsc{fact} \textsc{sens}-be \\
\glt `My brothers, aren't you coming, (the râkshasî) is about to eat us!' (28-smAnmi, 399)
 \end{exe}

 \begin{exe} 
\ex \label{ex:atara.YWNu}
\gll sɤ-rŋgɯ\redp{}rŋgɯ kɯnɤ tɯ-kɤrnoʁ ɲɯ-mtɕɯr tɕe, ``atar-a ɲɯ-ŋu" kɤ-sɯso ʑo ntɕʰɤr \\
\textsc{ger}-lie.down also \textsc{genr}.\textsc{poss}-brain \textsc{ipfv}-turn \textsc{lnk} fall:\textsc{fact}-\textsc{1sg} \textsc{sens}-be \textsc{inf}-think \textsc{emph} appear:\textsc{fact} \\
\glt `Even lying down, your head is dizzy, and it feels like you are about to fall down.' (29-tAmtshAzkAkWndo, 57-58)
 \end{exe}
 

 
\section{Secondary Modal categories}  \label{sec:second.modal}
Secondary Modal categories include the Probabilitative (§\ref{sec:WmA}), Rhetorical Interrogative (§\ref{sec:WBrA}) and Interrogative (§\ref{sec:interrogative.W}) which can be combined with primary TAME categories, and the Apprehensive (§\ref{sec:apprehensive}), which presents morphological commonalities with the Aorist (§\ref{sec:aor.morphology}).

  \subsection{Apprehensive} \label{sec:apprehensive}
  
    \subsubsection{Morphology} \label{sec:apprehensive.morphology}
The Apprehensive prefix \forme{ɕɯ-} presents a series of unique morphological properties in Japhug. It occupies the slot -3 of the outer prefixal chain (§\ref{sec:outer.prefixal.chain}), the same as orientational preverbs (§\ref{sec:kamnyu.preverbs}). It is the only finite TAME category apart from the Factual Non-Past (§\ref{sec:factual}) to lack orientation preverbs.

On transitive verbs, the 3\flobv{} direct form of the Apprehensive is \forme{ɕa\trt}, with a vowel alternation identical to that found with type C preverbs in the Aorist (§\ref{sec:indexation.non.local}, §\ref{sec:kamnyu.preverbs}, §\ref{sec:preverb.TAME.morphology}) as illustrated by the contrast between the intransitive (Anticausative, §\ref{sec:anticausative.morphology}) \forme{ɕɯ-ɴɢrɯ} `(I am afraid that) it will break' and its counterpart transitive \forme{ɕa-nɯ-qrɯ} `(I am afraid that) he will break it', with the same vocalism as that on the preverb \forme{ta-} prefix on the Aorist \forme{ta-ndo}  `s/he took it'.

\begin{exe}
\ex \label{ex:NGrW.qrW}
\begin{xlist}
\ex \label{ex:CWNGrW}
\gll kɯki χɕɤl ɲɯ-ɕti tɕe, nɯ fse laχtɕʰa ɯ-ŋgɯ cʰɯ-tɯ-rke tɕe ɕɯ-ɴɢrɯ kɯ!  \\
\textsc{dem}.\textsc{prox} glass \textsc{sens}-be \textsc{lnk} \textsc{dem} be.like:\textsc{fact}  thing \textsc{3sg}.\textsc{poss}-inside \textsc{ipfv}:\textsc{downstream}-2-put.in[III] \textsc{lnk} \textsc{appr}-\textsc{acaus}:break \textsc{sfp} \\
\glt `This is (made of) glass, if you put it like that in something else, I am afraid it will break.' (elicited)
\ex \label{ex:CanWqrW}
\gll a-tɕɯ nɯ kɯ kʰɯtsa ta-ndo tɕe, ɕa-nɯ-qrɯ kɯ! \\
\textsc{1sg}.\textsc{poss}-son \textsc{dem} \textsc{erg} bowl \textsc{aor}:3\flobv{}-take \textsc{lnk} \textsc{appr}:3\flobv{}-\textsc{auto}-break \textsc{sfp} \\
\glt `My son took the bowl, I am afraid that he will break it.' (elicited)
\end{xlist}
 \end{exe}
 
 This alternation is also found on transitive verbs with dummy transitive subject (§\ref{sec:transitive.dummy}), as in (\ref{ex:CalAt}), showing that morphological transitivity, rather than semantic factors, determine the vocalism of this prefix.
 
 \begin{exe}
\ex \label{ex:CalAt}
\gll tɯ-mɯ ɕa-lɤt kɯ  \\
\textsc{indef}.\textsc{poss}-weather \textsc{appr}:3\flobv{}-release \textsc{sfp} \\
\glt `I am afraid that it will rain.' (elicited)
\end{exe}
 
 
In addition, like the Aorist, it occurs with the 1/2\textsc{sg}\fl{}3 \forme{-t} (§\ref{sec:other.TAME}) as in (\ref{ex:CWtWnAmat}), and selects Stem II (§\ref{sec:stem2}) in the verbs that have an alternation between stemI and II, such as the verb \japhug{ti}{say} in (\ref{ex:CWtWtWt.kW}) and \japhug{ɕe}{go} in (\ref{ex:CanWri.kW}).

\begin{exe}
\ex 
\begin{xlist}
\ex \label{ex:CWtWnAmat}
\gll nɯ ɕɯ-tɯ-nɤma-t kɯ \\
\textsc{dem} \textsc{appr}-2-do-\textsc{pst}:\textsc{tr} \textsc{sfp} \\
\glt  `I am afraid that you will do that.' (elicited)
\ex \label{ex:CWtWtWt.kW}
\gll ɕɯ-tɯ-tɯt kɯ \\
\textsc{appr}-2-say[II] \textsc{sfp} \\
\glt `I am afraid that you will say it.' (elicited)
\ex \label{ex:CanWri.kW}
\gll ɕɯ-a<nɯ>ri kɯ \\
\textsc{appr}-<\textsc{auto}>go[II] \textsc{sfp} \\
\glt `I am afraid that he will go away.' (elicited)
\end{xlist}
\end{exe}

\tabref{tab:comparison.apprehensive} summarizes the morphological commonalities between Apprehensive and Aorist (as opposed to Imperfective, §\ref{sec:ipfv.morphology}).

\begin{table}
\caption{Comparison of Apprehensive, Aorist and Imperfective forms}  \label{tab:comparison.apprehensive}
\begin{tabular}{llllll}
\lsptoprule
&Apprehensive & Aorist & Imperfective &Common feature\\
\midrule
\textsc{3sg}\flobv{} &\forme{ɕa-lɤt} & \forme{pa-lɤt} & \forme{pjɯ-lɤt} & \forme{a} vocalism \\
\textsc{3sg}\flobv{}&\forme{ɕa-nɯ-qrɯ} & \forme{pa-nɯ-qrɯ} & \forme{pjɯ-nɯ-qri} & \forme{a} vocalism \\
\textsc{2sg}\fl{}3 &\forme{ɕɯ-tɯ-tɯt} & \forme{tɤ-tɯ-tɯt} & \forme{tu-tɯ-ti} & Stem II \\
\textsc{2sg}\fl{}3 &\forme{ɕɯ-tɯ-nɤma-t} &\forme{tɤ-tɯ-nɤma-t} &\forme{tu-tɯ-nɤme}& \forme{-t} \\
\lspbottomrule
\end{tabular}
\end{table}

The Apprehensive is not only incompatible with all TAME markers, but also with associated motion prefixes (§\ref{sec:am.prefixes}) and non-finite verb forms. On the other hand it appears with all person indexation markers (\textsc{1sg} in \ref{ex:CWmaqʰua}, \textsc{2sg} in \ref{ex:mWCWtWcha}, 2\textsc{sg}\fl{}1\textsc{sg} in \ref{ex:CWkWmpCaa}) and the negative prefix \forme{mɯ-} (\ref{ex:mWCWtWcha} and \ref{ex:mWCWkhW}). 

  \begin{exe}
\ex \label{ex:mWCWtWcha}
\gll [nɤʑo stɤβtsʰɤt ci mɯ-ɕɯ-tɯ-cʰa] ɲɯ-sɯsam-a tɕe, nɯ ɲɯ-nɯzdɯɣ-a wo \\
\textsc{2sg} contest \textsc{indef} \textsc{neg}-\textsc{apprehensive}-2-can \textsc{sens}-think[III]-\textsc{1sg} \textsc{lnk} \textsc{dem} \textsc{sens}-be.worried-\textsc{1sg} \textsc{sfp} \\
\glt  `I fear that you will not succeed in the contest, this is what I am worried about.' (2003sras, 95)
 \end{exe}

\subsubsection{Functions} \label{sec:apprehensive.function}
The Apprehensive is used to indicate worry or even fear that the event referred to by the verb might happen. Several verbs with the meaning `fear' exist in the language (in particular the Applicative verb \japhug{nɯɣmu}{be afraid of}, §\ref{sec:applicative}), but none is a complement taking verb meaning `be afraid that $X$'.\footnote{The clause \forme{ɕɯ-maqʰu-a} \forme{kɯ} \forme{ʑo} \forme{ɲɤ-sɯso-nɯ} in (\ref{ex:CWmaqʰua}) creatively translates the Chinese expression \ch{争先恐后}{zhēngxiānkǒnghòu}{vie with each other in order to}. }     The Rhetorical Interrogative (§\ref{sec:WBrA}) can have a meaning similar to that of the Apprehensive, but conveying a milder degree of concern.

\begin{exe}
\ex \label{ex:CWmaqʰua}
\gll  [ɕɯ-maqʰu-a kɯ ʑo] ɲɤ-sɯso-nɯ tɕe rcanɯ, pɕoʁʑi kɯβde ʑo jo-ɣi-nɯ.  \\
\textsc{appr}-be.after-\textsc{1sg} \textsc{sfp} \textsc{emph} \textsc{ifr}-think \textsc{lnk} unexpectedly corner four \textsc{emph} \textsc{ifr}-come-\textsc{pl} \\
\glt `Fearing of being late (thinking `I am afraid that I will be late'), they came from the four corners of the world.' (150906 toutao, 19)
 \end{exe}
 
 The Apprehensive exclusively expresses the fear of the speaker, not that of the agent, though in the case of reported speech this may be a person different from the present speaker (as in \ref{ex:CWmaqʰua}). 
  
Despite its Aorist-like morphological features (§\ref{sec:apprehensive.morphology}), the Apprehensive is exclusively refers to fear for a future event at the time of utterance; for instance, (\ref{ex:CanWri.kW}) above in cannot be interpreted as meaning  `I am afraid that he may already be gone.'  In complement clause with a matrix verb in the Aorist, it can however refer to an event which has already happened `(I) was afraid that it might $X$', as in (\ref{ex:mWCWkhW}). 
  
\begin{exe}
\ex \label{ex:mWCWkhW}
\gll   mɯ-ɕɯ-kʰɯ kɯ nɯ-sɯso-t-a \\
\textsc{neg}-\textsc{appr}-be.possible \textsc{sfp} \textsc{aor}-think-\textsc{pst}:\textsc{tr}-\textsc{1sg} \\
\glt `I thought it would not work.'  (conversation, 02-05-2018; after opening a washing machine, expecting it would not open)
\end{exe}
 
The apprehensive prefix  only occurs either in independent clauses with the exclamative sentence final particle \forme{kɯ} (\ref{ex:CWkWmpCaa}), and in reported speech with the verb \japhug{sɯso}{think} (\ref{ex:mWCWkhW}, \ref{ex:mWCWtWcha}, \ref{ex:CWmaqʰua}). It is strictly impossible in any other type of complement clause. 

 \begin{exe}
\ex \label{ex:CWkWmpCaa}
\gll  ɕɯ-kɯ-mpɕa-a kɯ!   \\
\textsc{apprehensive}-2\fl{}1-scold-\textsc{1sg} \textsc{sfp} \\
\glt  `I am afraid that you will scold me.' (elicited)
  \end{exe}

 It is possible to use the Apprehensive in clauses with precautioning or preventive meanings (as in \ref{ex:NGrW.qrW} and \ref{ex:CWmaqʰua}), but it is not the usual way of expressing these meanings in the language (§\ref{sec:precautioning.clauses}).


\subsubsection{Historical origin} \label{sec:apprehensive.history}
The Apprehensive \forme{ɕɯ-} is superficially similar to the translocative associated motion prefix \forme{ɕɯ\trt}, which is grammaticalized from the verb \japhug{ɕe}{go} (§\ref{sec:translocative.morpho}), but the two prefixes do not occupy the same prefixal slot (-3 vs. -4, §\ref{sec:outer.prefixal.chain}) and do not share the same morphological alternations, and are not semantically close. It is unlikely that they are historical related.

A possible origin for the apprehensive \forme{ɕɯ-} might be the alternative interrogative particle \forme{ɕi} `whether ... or' (§\ref{sec:fsp.interrog}), which can coordinate two clauses (the second clause being the negative counterpart of the first one) to express hesitation or worry that an action might not take place, as in (\ref{ex:chaa.Ci}) (see also  \ref{ex:kuCea.makACea}, §\ref{sec:prohib.function}).
 
 \begin{exe}
\ex \label{ex:chaa.Ci}
\gll  cʰa-a ɕi mɤ-cʰa-a mɤ-xsi, pɤjkʰu xtɕi-a ɕti \\
can:\textsc{fact}-\textsc{1sg} \textsc{qu} \textsc{neg}-can:\textsc{fact}-\textsc{1sg} \textsc{neg}-\textsc{genr}:know:\textsc{fact} still be.small:\textsc{fact}-\textsc{1sg} be.\textsc{aff}:\textsc{fact} \\
\glt `I don't know whether I will be able to do it, I am still young.' (2010-09, 105-106)
\end{exe}
  
Although generally clause-final, the particle  \forme{ɕi}  can be prosodically attached to the second clause, and could have been absorbed into the verbal template (\citealt{laiyf20betrayal} describes similar processes in Khroskyabs). In this hypothesis, the Apprehensive was first grammaticalized in negative forms, with reordering of the Apprehensive and of the Negative prefix (\ref{ex:appr.grammaticalization}), and the non-negative Apprehensive forms were created by backformation.

 \begin{exe}
\ex \label{ex:appr.grammaticalization}
\glt \forme{ɕi mɤ-cʰa} $\Rightarrow$ \forme{*ɕɯ-mɤ-cʰa} $\Rightarrow$ \forme{mɯ-ɕɯ-cʰa} 
\glt `(I don't know) whether he (will be able to do it) or not be able to do it' $\Rightarrow$ `I worry that he might not be able to do it.' 
\end{exe}

This idea should still be viewed as speculative, since it does not account for the Aorist-like morphological features of the Apprehensive. 

The \forme{a} vocalism of the Apprehensive prefix with 3\flobv{} transitive forms is a clue that the C-type preverbs in Japhug originate from the fusion of A-type  preverbs with an Aorist 3\flobv{} \forme{*a} prefix (§\ref{sec:xtokavian.preverbs}). Given the ability of the Aorist to refer to future events in temporal clauses (§\ref{sec:aor.temporal}), it is possible that in pre-Japhug or proto-Gyalrong, there was a TAME category without preverb but taking stem II and the 3\flobv{} \forme{*a} prefix expressing perfective future tense, and that the grammaticalization of \forme{ɕi} as a prefix occurred on a form of this lost category.

\subsection{Probabilitative} \label{sec:WmA}
The Probabilitative prefix \forme{ɯmɤ\trt}, located in slot -6 of the outer prefixal template (§\ref{sec:outer.prefixal.chain}), is transparently grammaticalized from the combination of the Interrogative \forme{ɯ-} (§\ref{sec:interrogative.W}) with the negative \forme{mɤ-} (§\ref{sec:neg.allomorphs}). It is compatible with the Aorist  (\ref{ex:WmAjazGWt2}), the Imperfective (§\ref{sec:WmA.ipfv}) and the Factual Non-Past (\ref{ex:tCitaR.WmApe}), and has the epistemic modality meaning `probably, presumably'.

 \begin{exe}
 \ex \label{ex:WmAjazGWt2}
\gll ɯmɤ-jɤ-azɣɯt  \\
\textsc{prob}-\textsc{aor}-arrive \\
\glt `He probably has already arrived.' (elicited)
\end{exe}

 \begin{exe}
 \ex \label{ex:tCitaR.WmApe}
\gll tɕi-ɬaʁ nɯnɯ, tɕi-taʁ ɯmɤ-pe \\
\textsc{1du}.\textsc{poss}-MZ \textsc{dem} \textsc{1du}.\textsc{poss}-on \textsc{prob}-be.good:\textsc{fact} \\
\glt `(Now that we have money), maybe our stepmother will treat us better.' (140507 tangguowu-zh, 155)
\end{exe}

Although Probabilitative with Aorist (\ref{ex:WmAjazGWt2}) can be elicited, no such example is attested in the corpus. Instead, a periphrastic construction with the Factual Probabilitative form of the copula \forme{ɯmɤ-ŋu} `maybe it is' combined with the Aorist is found, as in (\ref{ex:mWtatWtndZi.WmANu}).
 
 \begin{exe}
 \ex \label{ex:mWtatWtndZi.WmANu}
\gll nɤ-ja ndʑi-ɕki tɤ-tɯt-a ri, mɯ-ta-tɯt-ndʑi ɯmɤ-ŋu ma \\
\textsc{2sg}.\textsc{poss}-sister \textsc{3du}.\textsc{poss}-\textsc{dat} \textsc{aor}:3\flobv{}-say[II]-\textsc{1sg} \textsc{lnk} \textsc{neg}-\textsc{aor}:3\flobv{}-say[II]-\textsc{du} \textsc{prob}-be:\textsc{fact} \textsc{sfp} \\
\glt `I have told it to your two elder sisters, but they presumably have not told (your parents) about it (since they have not given any answer).' (2014-kWlAG, 106)
\end{exe}

When the speaker hesitates in his degree of confidence regarding a particular statement, s/he can alternate between the Factual copula \forme{ŋu} and the Probabilitative form \forme{ɯmɤ-ŋu} `maybe it is', as in (\ref{ex:Nu.WmANu}).

 
 \begin{exe}
 \ex \label{ex:Nu.WmANu}
\gll ɕɤr tɕe tu-mbri tɕe, ``qaqaqaqa" tu-ti ŋu tɕe nɯnɯ pɣɤɲaʁ ŋu. ``pɣɤɲaʁ ŋu" tu-ti-nɯ, ŋu tɕe ɯmɤ-ŋu ma mɤ-xsi matɕi ɕɤr ndɤre mɯ́j-sɤ-mto tɕe. \\
night \textsc{loc} \textsc{ipfv}-cry \textsc{lnk} \textsc{interj} \textsc{ipfv}-say be:\textsc{fact} \textsc{lnk} \textsc{dem} pheasant be:\textsc{fact} pheasant be:\textsc{fact} \textsc{ipfv}-say-\textsc{pl} be:\textsc{fact} \textsc{lnk} \textsc{prob}-be:\textsc{fact} \textsc{lnk} \textsc{neg}-\textsc{genr}:know:\textsc{fact} \textsc{lnk} night \textsc{loc} \textsc{neg}:\textsc{sens}-\textsc{prop}-see \textsc{lnk} \\
\glt `It sings at night, it makes the sound \forme{qaqaqaqa}, that is the pheasant \textit{Pucrasia macrolopha}. They say it is the \textit{Pucrasia macrolopha}, maybe it is, we don't know, in the night it is not visible.' (23-pGAYaR, 8-11)
 \end{exe}
 
The Probabilitative prefix is not compatible with negative prefixes or negative copulas: forms such as $\dagger$\forme{ɯmɤ-maʁ} (intended meaning `maybe it is not') are incorrect. To express the meaning `maybe not $X$', the Rhetorical Interrogative \forme{ɯβrɤ-} with peg circumfix occurs instead (§\ref{sec:WBrA.kW.ci}).
 
\subsubsection{Possible modality with peg circumfix} \label{sec:WmA.kW.ci}
The Probabilitative prefix can be combined with the peg circumfix (§\ref{sec:peg.circumfix}), with an epistemic modality of substantial doubt  `maybe', as in (\ref{ex:WmAkWNuci}).

\begin{exe}
\ex \label{ex:WmAkWNuci}
\gll pɣa ra kɯ tu-ndza-nɯ ɯmɤ-kɯ-ŋu-ci ma \\
bird \textsc{pl} \textsc{erg} \textsc{ipfv}-eat-\textsc{pl}  \textsc{prob}-\textsc{peg}-be-\textsc{peg} \textsc{lnk} \\
\glt `Maybe birds eat it.' (17-thowum, 39)
\end{exe}

This form is compatible with verbs in the Sensory, as in (\ref{ex:GAZu.WmAkWNuci}); with the meaning `it seems, it looks like'.

\begin{exe}
\ex \label{ex:GAZu.WmAkWNuci}
\gll alo tɕu rgɯnba ci ɣɤʑu ɯmɤ-kɯ-ŋu-ci tɕe \\
upstream \textsc{loc} temple \textsc{indef} exist:\textsc{sens} \textsc{prob}-\textsc{peg}-be-\textsc{peg} \textsc{lnk} \\
\glt `It looks like there is a temple up there.' (2003 Kunbzang, 229)
\end{exe}

\subsubsection{Probabilitative with Imperfective} \label{sec:WmA.ipfv}
The combination of the Probabilitative prefix with the Imperfective form (§\ref{sec:imperfective}) expresses optative modality `if only...', as in (\ref{ex:WmAchWmACia}) and (\ref{ex:WmApjWmtama}).

\begin{exe}
\ex \label{ex:WmAchWmACia}
\gll ɯmɤ-cʰɯ-mɤɕi-a \\
\textsc{prob}-\textsc{ipfv}-be.rich-\textsc{1sg} \\
\glt `If only I could become rich!' (elicited)
\end{exe}

\begin{exe}
\ex \label{ex:WmApjWmtama}
\gll ``iɕqʰa rɟɤlpu ɯ-tɕɯ nɯ ɯmɤ-pjɯ-mtam-a" ntsɯ ɲɯ-sɯsɤm pjɤ-ŋu. \\
the.aforementioned king \textsc{3sg}.\textsc{poss}-son \textsc{dem} \textsc{prob}-\textsc{ipfv}-see[III]-\textsc{1sg} always \textsc{ipfv}-think[III] \textsc{ifr}.\textsc{ipfv}-be \\
\glt `She was thinking all the time: `If only I could see the prince!'' (150819 haidenver-zh, 154)
\end{exe}

This type of compound form is also attested however with the expected compositional meaning `it will probably be', as in (\ref{ex:WmApjWNu}).

\begin{exe}
\ex 
\begin{xlist}
\ex  
\gll ``nɯ-kʰi ɣe" ti qʰe, \\
\textsc{3pl}.\textsc{poss}-luck \textsc{sfp} say:\textsc{fact} \textsc{lnk} \\
\glt `She said: `They are so lucky.' (to have so many cattle).' 
\ex \label{ex:WmApjWNu}
\gll ``tɕiʑɤɣ ɯmɤ-pjɯ-ŋu nɤ lɤ-ɣi wo!" ntsɯ tu-ti pɯ-ŋu ɲɯ-ŋu. \\
\textsc{1du}:\textsc{gen} \textsc{prob}-\textsc{ipfv}-be \textsc{add} \textsc{imp}:\textsc{upstream}-come always \textsc{sfp} \textsc{ipfv}-say \textsc{pst}.\textsc{ipfv}-be \textsc{sens}-be \\
\glt `He said each time: `Maybe (these cattle) will be ours, come.' (2005 Kunbzang, 193-195)
\end{xlist}
 \end{exe}
 
%tɕendɤre ɯ-rʑaβ nɯ cʰɯ-nɯzdɯɣ ɯkʰɯkʰa, nɤkinɯ,
%nɯtɕu "ɯmɤ-pjɯ-si-a!" ɲɯ-sɯsɤm 
%140510 sanpian sheye-zh

\subsection{Rhetorical interrogative} \label{sec:WBrA} 

\subsubsection{Morphology} \label{sec:WBrA.morphology} 
The Rhetorical Interrogative prefix \forme{ɯβrɤ\trt}, located in slot -6 of the outer prefixal template (§\ref{sec:outer.prefixal.chain}),  mainly occurs with verbs in the Factual Non-Past (\ref{ex:WBrAsqlWm}) in the corpus, but forms with the Aorist (\ref{ex:WBrACtAtWtWt}) or the Imperfective are also attested. It is not compatible with negative prefixes.

\begin{exe}
\ex \label{ex:WBrAsqlWm}
\gll rcanɯ mti pjɯrɯ ʁɟa ʑo tɤrɤmɕkʰo pɯ-ŋu ɲɯ-ŋu. `nɯ a-mi pjɯ-te-a ri ɯβrɤ-sqlɯm ma' na-sɯso ɲɯ-ŋu \\
\textsc{unexp}:\textsc{deg} turquoise  coral completely \textsc{emph} floor \textsc{pst}.\textsc{ipfv}-be \textsc{sens}-be \textsc{dem} \textsc{1sg}.\textsc{poss}-foot \textsc{ipfv}:\textsc{down}-put[III]-\textsc{1sg} \textsc{lnk} \textsc{rh}.\textsc{q}-collapse:\textsc{fact} \textsc{sfp} \textsc{aor}:3\flobv{}-think \textsc{sens}-be \\
\glt `The floor was all tiled with and turquoise and coral. She thought `If I tread on it, it will not yield under my weight, will it?'' (2005 Kunbzang, 221)
\end{exe}

\begin{exe}
\ex \label{ex:WBrACtAtWtWt}
\gll tɯrmɯkʰa nɤ nɯ-βɣo jɤ-azɣɯt nɤ ``joβ ɯβrɤ-ɕ-tɤ-tɯ-tɯt-nɯ?" ti ɲɯ-ŋu. \\
evening \textsc{add} \textsc{3pl}.\textsc{poss}-FB \textsc{aor}-arrive \textsc{add} \textsc{interj} \textsc{rh}.\textsc{q}-\textsc{tral}-\textsc{aor}-2-say[II]-\textsc{pl} say:\textsc{fact} \textsc{sens}-be \\
\glt `In the evening, their lama arrived and said: `You did not go and say \forme{joβ} (to the girl), did you?' (2003kandZislama, 22)
 \end{exe}
 
\subsubsection{Functions} \label{sec:WBrA.functions} 
The Rhetorical Interrogative has two main functions, both in the form of negative rhetorical questions. 

First, it expresses concern on part of the speaker that the action described by the verb might happen (\ref{ex:WBrAGWrBNgW}) or might have already happened (\ref{ex:WBrACtAtWtWt} above), depending on the TAME category of the verb. In this function, it is frequently combined with the sentence final particle \forme{ma} (§\ref{sec:fsp.epistemic}) as in (\ref{ex:WBrAsqlWm}) and (\ref{ex:WBrAGWrBNgW}).
 
 \begin{exe}
\ex \label{ex:WBrAGWrBNgW}
\gll  ``a-taʁ nɯtɕu,  loŋbutɕʰi nɯ ɯβrɤ-ɣɯ-rŋgɯ ma" ɲɯ-sɯsɤm tɕe, nɯ ntsɯ ɯ-kɤ-nɯzdɯɣ ɲɯ-ŋu kʰi \\ 
\textsc{1sg}.\textsc{poss}-on \textsc{dem}:\textsc{loc} elephant \textsc{dem} \textsc{rh}.\textsc{q}-\textsc{cisl}-lie.down:\textsc{fact} \textsc{sfp} \textsc{ipfv}-think[III] \textsc{lnk} \textsc{dem} always \textsc{3sg}.\textsc{poss}-\textsc{obj}:\textsc{pcp}-worry/about \textsc{sens}-be \textsc{hearsay} \\
\glt `(The marmot)$_i$ thinks: `The elephant$_j$ won't come and sleep on my (hole), will it$_j$?', this is what it$_i$ worried about.' (28-qapar, 48)
\end{exe}

In this use, \forme{ɯβrɤ-} normally carries a negative presupposition `it will not $X$, will it?'. This negative meaning appears to be absent only  in a handful of examples such as (\ref{ex:WBrAGi.Ge}), where \forme{ɯβrɤ-} is found with the final particle \forme{ɣe}: in another version of the story from which this example is taken, the corresponding quotation has the verb \forme{ɣi} `come' in affirmative form without the Rhetorical Interrogative, and instead the sentence final particle \forme{tʰaŋ} (see \ref{ex:Gi.thaN}, §\ref{sec:fact.main.clauses}).

\begin{exe}
\ex \label{ex:WBrAGi.Ge}
\gll  a-mu, jɯɣmɯr tɕe kʰu ɯβrɤ-ɣi ɣe! \\
\textsc{1sg}.\textsc{poss}-mother this.evening \textsc{lnk} tiger \textsc{rh}.\textsc{q}-come:\textsc{fact} \textsc{sfp} \\
\glt `Mother, the tiger will come this evening (to eat us), isn't?' (X1-khu, 3)
\end{exe}
 
In this function, the prefix \forme{ɯβrɤ-} has some overlap with the Apprehensive prefix \forme{ɕɯ-} (§\ref{sec:apprehensive.function}), though the latter conveys a higher degree of worry.

Second, the Rhetorical Interrogative occurs in interrogative clauses expressing polite requests, as in (\ref{ex:WBrAkWznAmYoanW}).   

\begin{exe}
\ex \label{ex:WBrAkWznAmYoanW}
 \gll nɤki nɯ-tɤɲi ɯ-taʁ kɤ-rɤt nɯ ɯβrɤ-kɯ-z-nɤmɲo-a-nɯ? \\
 \textsc{dem}:\textsc{medial} \textsc{2pl}.\textsc{poss}-staff \textsc{3sg}.\textsc{poss}-on \textsc{obj}:\textsc{pcp}-write \textsc{dem} \textsc{rh}.\textsc{q}-2\fl{}1-\textsc{caus}-watch-\textsc{1sg}-\textsc{pl} \\
 \glt `You wouldn't show me what is written on that staff of yours, would you?' (2003sras, 61)
\end{exe}

 This function is common in particular with the Rhetorical Interrogative form \forme{ɯβrɤ-jɤɣ} of the modal verb \japhug{jɤɣ}{be possible} selecting a complement clause in the Imperfective, as in (\ref{ex:chWtWGinW.WBrAjAG}).\footnote{The noun \forme{a-rʑaβ} `my wives' in (\ref{ex:chWtWGinW.WBrAjAG}) is an essive adjunct (§\ref{sec:essive.abs}).  }
 

\begin{exe}
\ex \label{ex:chWtWGinW.WBrAjAG}
 \gll nɯʑo ra χsɯm nɯ, [a-rʑaβ cʰɯ-tɯ-ɣi-nɯ] ɯβrɤ-jɤɣ? \\
 \textsc{2pl} \textsc{pl} three \textsc{dem} \textsc{1sg}.\textsc{poss}-wife \textsc{ipfv}:\textsc{downstream}-2-come-\textsc{pl} \textsc{rh}.\textsc{q}-be.possible:\textsc{fact} \\
 \glt `The three of you, would you come (to my palace and become) my wives?' (Norzang 2005, 399)
\end{exe}

The prefix \forme{ɯβrɤ-} also occurs in polite inquiries about a person's health, expressing sincere concern about a possible illness as in (\ref{ex:nAkWmNAm.WBrAkutu}).

\begin{exe}
\ex \label{ex:nAkWmNAm.WBrAkutu}
 \gll  nɤ-kɯ-mŋɤm ɯβrɤ-ku-tu, nɤ-ɕqʰe ɯβrɤ-ku-tʰɯ? \\
\textsc{2sg}.\textsc{poss}-\textsc{sbj}:\textsc{pcp}-hurt  \textsc{rh}.\textsc{q}-\textsc{prs}-exist 
\textsc{2sg}.\textsc{poss}-cough  \textsc{rh}.\textsc{q}-\textsc{prs}-be.serious \\
\glt `You don't have any disease (these days), have you? Your cough is not serious, is it?' (conversation, 2016-03-20)
\end{exe}

A more marginal use of \forme{ɯβrɤ-} is found in (\ref{ex:WBrApjWtua}), where it appears  on two antonymic verbs linked by the adversative \forme{ri} to express the meaning `if I don't $X$, at least I will not $Y$.' In another version of the story from which (\ref{ex:WBrApjWtua}) is taken, the same meaning (\ref{ex:mAphana.mARdWGa}) is expressed by verbs in the negative Factual Non-Past with the sentence final particle \forme{tʰaŋ} (§\ref{sec:fsp.epistemic}).

 \begin{exe}
\ex \label{ex:WBrApjWtua}
\gll   pʰɤn ɯβrɤ-pjɯ-tu-a ri ʁdɯɣ nɯ ɯβrɤ-pjɯ-tu-a tɕe tɤ-rca cʰɯ-ɣi-a je \\
be.efficient:\textsc{fact} \textsc{qu}.\textsc{r}-\textsc{ipfv}-exist-\textsc{1sg} \textsc{lnk} harm:\textsc{fact} \textsc{dem} \textsc{qu}.\textsc{r}-\textsc{ipfv}-exist-\textsc{1sg}  \textsc{lnk} \textsc{indef}.\textsc{poss}-together \textsc{ipfv}:\textsc{downstream}-come-\textsc{1sg} \textsc{sfp} \\
\glt `Even if I am of no use at least I will not do any harm, let me come along!' (X1-tWJo, 43)
\end{exe}

 \begin{exe}
\ex \label{ex:mAphana.mARdWGa}
\gll  aʑo cʰɯ-ɣi-a je ma mɤ-pʰan-a nɤ mɤ-ʁdɯɣ-a tʰaŋ nɤ \\
\textsc{1sg} \textsc{ipfv}:\textsc{downstream}-come-\textsc{1sg} \textsc{sfp} \textsc{lnk} \textsc{neg}:be.efficient:\textsc{fact}-\textsc{1sg} \textsc{add} \textsc{neg}-harm:\textsc{fact}-\textsc{1sg} \textsc{sfp} \textsc{sfp} \\
\glt `Let me come along, (even) if I am of no use, I will not do any harm.' (several occurrences)
\end{exe}

\subsubsection{Rhetorical Interrogative with peg circumfix} \label{sec:WBrA.kW.ci}
The combination of the Rhetorical Interrogative with the peg circumfix (§\ref{sec:peg.circumfix}) is not compositional. Despite the absence of any negative prefix (§\ref{sec:negation}), it is the negative counterpart (in functional terms) of the Probabilitative prefix (§\ref{sec:WmA}), expressing the meaning `it looks like/seems that $X$ is/does not', as in (\ref{ex:khro.WBrAkWpeci})\footnote{In (\ref{ex:khro.WBrAkWpeci}), \forme{ɯβrɤ-kɯ-pe-ci} translates Chinese \ch{这好像不太好吧}{zhè hǎoxiàng bú tài hǎo ba}{This does not look good/nice}. } and (\ref{ex:WBrAkWtuci}).


\begin{exe}
\ex \label{ex:khro.WBrAkWpeci}
 \gll kʰro ɯβrɤ-kɯ-pe-ci \\
 much \textsc{rh}.\textsc{q}-\textsc{peg}-be.good:\textsc{fact}-\textsc{peg} \\
\glt `(What you are proposing) does not look nice (fair).' (140506 nongfu he mogui-zh, 69)
\end{exe}

\begin{exe}
\ex \label{ex:WBrAkWtuci}
 \gll ʁloŋbutɕʰi nɯ, ɯ-rme ra ɯβrɤ-kɯ-tu-ci \\
 elephant \textsc{dem} \textsc{3sg}.\textsc{poss}-hair \textsc{pl} \textsc{rh}.\textsc{q}-\textsc{peg}-exist:\textsc{fact}-\textsc{peg} \\
\glt `The elephant, it seems that it does not have hair (on its body).' (19-RloNbutChi, 9)
\end{exe}


\subsubsection{Historical origin} \label{sec:WBrA.history}
The prefix \forme{ɯβrɤ-} is probably related to the Tshobdun interrogative prefix \forme{vré-}  \citep[397]{jackson19tshobdun}, though the two prefixes are not entirely similar: Tshobdun \forme{vré-} is monosyllabic, and the vowel \forme{-e} (from ealier \forme{*a}) has not undergone reduction to \forme{ɐ} as could have been expected in the view of its Japhug cognate. The fact that Japhug \forme{ɯβrɤ-} contains two syllables, and that it occurs in the left-most (-6) slot of the verbal template suggests that it may been grammaticalized relatively recently in comparison with other elements of the prefixal chain. In view of the formal differences between the two languages, one cannot exclude at this stage the possibility of independent grammaticalization, though the source is unclear. 

Possible sources (from a formal point of view) could include the noun \japhug{ɯ-βra}{his share} (cognate to Tshobdun \forme{ó-vre}) or the  adverbial noun \japhug{ɯ-βra}{it is X's turn to...} (§\ref{sec.IPN.adverbs}). However, the semantic link between these meanings and that of the Rhetorical Interrogative is unclear. 
% Polar Interrogative  \forme{ɯ-} (§\ref{sec:interrogative.W}),

 \subsection{Interrogative} \label{sec:interrogative.W} 
 \subsubsection{Morphology} \label{sec:interrogative.W.morpho} 
 The Interrogative \forme{ɯ\trt}, found in slot -6 of the outer prefixal template (§\ref{sec:outer.prefixal.chain}), is a stress-attracting prefix (§\ref{sec:stress.prefixal.chain}). When attached to a monosyllabic base, like the Factual Non-Past \textsc{3sg} of monosyllabic stem verbs, the stress is on the Interrogative prefix itself, as in \forme{ɯ́-jɤɣ} `it is possible to'(\ref{ex:YWtWkhAm.WjAG}) (see also \ref{ex:nWnaj.WjAG}, §\ref{sec:fact.complement} above).
 
\begin{exe}
\ex \label{ex:YWtWkhAm.WjAG}
 \gll kɯki ɲɯ-tɯ-kʰɤm ɯ́-jɤɣ \\
 \textsc{dem}.\textsc{prox} \textsc{ipfv}-2-give[III] \textsc{qu}-be.possible:\textsc{fact} \\
\glt `Could you pass (me) this thing?' (2003, conversation taRrdo)
\end{exe}

If the base is polysyllabic, the stress is on the second syllable just following the Interrogative, whether this syllable is a prefix (\ref{ex:WtWsWz2}) or part of the verb stem (\ref{ex:WGAZu}).

\begin{exe}
\ex \label{ex:WtWsWz2}
 \gll nɤj ɯ-tɯ́-sɯz? \\
\textsc{2sg} \textsc{qu}-2-know:\textsc{fact} \\
\glt `Do you know it?' (19-GzW, 8)
\end{exe}

\begin{exe}
\ex \label{ex:WGAZu}
 \gll ki sakaβ ɯ-rkɯ nɯtɕu si ɯ-ɣɤ́ʑu? \\
 \textsc{dem}.\textsc{prox} well \textsc{3sg}.\textsc{poss}-side \textsc{dem}:\textsc{loc} tree \textsc{qu}-exist:\textsc{sens} \\
 \glt `Are there trees near this well?' (140517 mogui de jing-zh, 100)
\end{exe}

When occurring in direct contact with the stem of a contracting verb (§\ref{sec:contraction}), the Interrogative \forme{ɯ-} does not merge with the initial \forme{a-}. Instead, an epenthetic \forme{-j-} is inserted between the prefix and the verb stem (see examples \ref{ex:WjatsWtsu},  §\ref{sec:contraction} and \ref{ex:Wjandza}, §\ref{sec:passive.agent}).



 \subsubsection{Functions} \label{sec:interrogative.W.function}
The Interrogative prefix \forme{ɯ-} has two main functions.  First, it expresses a polar question (as in \ref{ex:atAtWthe.WjAG} for instance), expecting an answer with the same verb in affirmative or negative form (\ref{ex:jAG.jAG}), a question studied in more detail in §\ref{sec:anticipation.person} and §\ref{sec:egophoric.interrogative}. An alternative way of marking polar questions is the sentence final particle \forme{ɕi} (§\ref{sec:fsp.interrog}).


\begin{exe}
\ex 
\begin{xlist}
\ex \label{ex:atAtWthe.WjAG}
 \gll wortɕʰi ʑo a-tɤ-tɯ-tʰe ɯ́-jɤɣ? \\
please \textsc{emph} \textsc{irr}-\textsc{pfv}-2-ask[III] \textsc{qu}-be.possible:\textsc{fact} \\ 
\glt `Can you ask it (for me)?' (divination 2002, 45)
\ex \label{ex:jAG.jAG}
 \gll jɤɣ, jɤɣ! \\
be.possible:\textsc{fact} be.possible:\textsc{fact} \\ 
\glt `Yes, yes!' (divination 2002, 46)
\end{xlist}
\end{exe}

Second, it occurs in the protasis of conditionals (§\ref{sec:real.conditional}) in free variation with initial reduplication (§\ref{sec:redp.protasis}). In this construction, the verb is generally followed by the additive marker \forme{nɤ}, as in (\ref{ex:WYWtWsWsAm}).\footnote{The second person prefix \forme{nɤ-} in (\ref{ex:WYWtWsWsAm}) is a case of hybrid indirect speech (§\ref{sec:hybrid indirect}). }
 
\begin{exe}
\ex \label{ex:WYWtWsWsAm}
 \gll ``nɤ-nmaʁ nɯ a-tɤ-sɯsu'' ɯ-ɲɯ́-tɯ-sɯsɤm nɤ, kɯki tɤ-ndze \\
 \textsc{2sg}.\textsc{poss}-husband \textsc{dem} \textsc{irr}-\textsc{pfv}-live \textsc{qu}-\textsc{sens}-2-think[III] \textsc{add} \textsc{dem}.\textsc{prox} \textsc{imp}-eat[III] \\
\glt `If you want your husband to live, eat this!' (150909 hua pi-zh, 182)
\end{exe}

\subsubsection{Historical perspectives} \label{sec:interrogative.e.history}

Many languages of the Trans-Himalayan family, including Chinese, Tibetan and Gyalrongic varieties, have vocalic prefixes marking polar interrogatives similar to Japhug \forme{ɯ-} \citep{sunhk96yiwen}. However, it remains unclear whether these prefixes reflect genuine common inheritance: in particular, although some Tibetan languages do have an interrogative prefix \forme{e\trt}, it is not attested in Old Tibetan texts \citep{hoshi12e}, and may therefore be an innovative feature. If this prefix is not an archaic feature in Tibetan,  the Japhug Interrogative \forme{ɯ-} and its equivalent in other Gyalrongic languages such as Khroskyabs \forme{ə̂-} (\citealt[340]{lai17khroskyabs}) may be better analyzed as borrowings from Amdo Tibetan.

\section{Non-verbal TAME markers}  \label{sec:non.verb.TAME}
Verbal morphology and periphrastic constructions are not the only ways to encode TAME in Japhug. Three other grammatical categories contribute to mark TAME. 


\subsection{TAME adverbs} 
Tense, aspect and modality can be expressed by primary aspectual adverbs such as \japhug{pɤjkʰu}{still}, \japhug{ʑɯrɯʑɤri}{progressively} or \japhug{ntsɯ}{always} (§\ref{sec:tense.aspect.adverbs}), time ordinals (§\ref{sec:time.ordinals}), temporal counted nouns (§\ref{sec:CN.time}), time adverbs derived from nouns (§\ref{sec:time.adv}) and adverb of epistemic modality (§\ref{sec:modality.adverbs}).


\subsection{Sentence-final particles} 
Many sentence final particles contribute to the expression of modality and evidentiality. Some particles are dedicated markers: the hearsay particle \forme{kʰi} (§\ref{sec:fsp.hearsay}) is specifically encodes evidentiality,  \forme{tʰaŋ} indicates epistemic modality (§\ref{sec:fsp.epistemic}) and \forme{je} has a hortative/imperative meaning (§\ref{sec:fsp.imp}). Other particles such as \forme{wo} (§\ref{sec:fsp.imp}) have meanings that are more difficult to describe exclusively in terms of modality or evidentiality. No sentence final particle expresses tense or aspect.


\subsection{Nouns and the expression of modality} \label{sec:nouns.TAME}
Two nouns of Tibetan origin, \japhug{ɯ-mdoʁ}{colour} and \japhug{smɯlɤm}{prayer} occur sentence-finally as modal markers. These sentence-final uses probably originate from complement-taking nouns (§\ref{sec:complement.taking.IPN}) used predicatively (§\ref{sec:non.verbal.predicates}).

\subsubsection{The noun \japhug{ɯ-mdoʁ}{colour} } \label{sec:WmdoR.TAME}
The noun \japhug{ɯ-mdoʁ}{colour}, borrowed from Tibetan \tibet{མདོག་}{mdog}{colour}, occurs with the highly grammaticalized modal meaning `it looks like', as in (\ref{ex:GAZu.WmdoR}) and (\ref{ex:YWrtaR.WmdoR}).  


\begin{exe}
	\ex \label{ex:GAZu.WmdoR}
	\gll [ɯ-mɤlɤjaʁ nɯ, [...] kɯrcɤ-ldʑa jamar ɣɤʑu] ɯ-mdoʁ \\
	\textsc{3sg}.\textsc{poss}-limb \textsc{dem} { } eight-piece about exist:\textsc{sens} \textsc{3sg}.\textsc{poss}-colour \\
	\glt `It looks like (the spider) has eight limbs.' (26-mYaRmtsaR-48)
\end{exe}

\begin{exe}
	\ex \label{ex:YWrtaR.WmdoR}
	\gll    [iɕqʰa nɯ [kɤntɕʰɯ-sŋi ʑo tu-ndza-nɯ] ɲɯ-rtaʁ] ɯ-mdoʁ  \\
	the.aforementioned \textsc{dem} several-day \textsc{emph} \textsc{ipfv}-eat-\textsc{pl} \textsc{sens}-be.enough \textsc{3sg}.\textsc{poss}-colour \\
	\glt `It looks like (the meat in one of those) is enough (for the lions) to eat for several days.' (20-sWNgi, 58)
\end{exe}

This construction occurs with the Sensory in the complement clause (§\ref{sec:sensory}) as in the examples above, or with the Inferential to express a past event/situation (see \ref{ex:WmdoRo}, §\ref{sec:verb.enclitics}). The combination of Inferential with \forme{ɯ-mdoʁ} is semantically close to the Probabilitative with peg circumfix (§\ref{sec:WmA.kW.ci}, see also \ref{ex:WmAjakWzGWtci.gloss}, §\ref{sec:peg.circumfix}).

The denominal verb  \japhug{nɯmdoʁ}{look like} derived from this construction (§\ref{sec:denom.intr.nW}) has the same meaning, but rather selects a subject participle in the complement clause as in (\ref{ex:kWCe.mWpjAnWmdoR}), with subject coreference between the main clause and the complement (compare \ref{ex:kWCe.mWpjAnWmdoR} and \ref{ex:kWCe.YWtWnWmdoR}).


\begin{exe}
	\ex \label{ex:kWCe.mWpjAnWmdoR}
	\gll  χpɯn kɯ-tsʰu nɯ maka mɯ-ɲɯ-mɯnmu tɕe ku-nɯ-rɤʑi pjɤ-ɕti ma [kɯ-ɕe] mɯ-pjɤ-nɯmdoʁ. \\
	monk \textsc{sbj}:\textsc{pcp}-be.fat \textsc{dem} at.all \textsc{neg}-\textsc{ipfv}-move \textsc{lnk} \textsc{ipfv}-\textsc{auto}-stay \textsc{ifr}.\textsc{ipfv}-be.\textsc{aff} \textsc{lnk} \textsc{sbj}:\textsc{pcp}-go \textsc{neg}-\textsc{ifr}.\textsc{ipfv}-look.like \\
	\glt `The fat monk was not moving and was staying, and did not look like he was going (anywhere).'  (150830 san ge heshang-zh, 127)
\end{exe}

\begin{exe}
	\ex \label{ex:kWCe.YWtWnWmdoR}
	\gll [kɯ-ɕe] ɲɯ-tɯ-nɯmdoʁ \\
	\textsc{sbj}:\textsc{pcp}-go \textsc{sens}-2-look.like \\
	\glt `You look like you are going.' (elicited)
\end{exe}  

The verb \forme{nɯmdoʁ} can alternatively take a noun phrase as semi-object, as in (\ref{ex:kWpe.YWnWmdoR}).

\begin{exe}
	\ex \label{ex:kWpe.YWnWmdoR}
	\gll [tɯrme kɯ-pe ci] ɲɯ-nɯmdoʁ \\
	person \textsc{sbj}:\textsc{pcp}-be.good \textsc{indef} \textsc{sens}-look.like \\
	\glt `He looks like someone nice.' (elicited)
\end{exe}  

% tɕeri ku-rtoʁ tɕe nɤki, [ndʑi-sɯm kɯ-ŋɤn] mɯ-pjɤ-nɯmdoʁ qʰe, \\


\subsubsection{The noun \japhug{smɯlɤm}{prayer} } \label{sec:smWlAm.TAME}
The noun \japhug{smɯlɤm}{prayer} (\tibet{སྨོན་ལམ་}{smon.lam}{prayer}, `wish') is grammaticalized as a quasi-sentence final particle, used in combination with the Irrealis (§\ref{sec:irrealis.main}) to express a wish, as in (\ref{ex:anWtABzu.smWlAm}). A similar construction has been documented in Tshobdun \citep{jackson07irrealis}.

\begin{exe}
	\ex \label{ex:anWtABzu.smWlAm}
	\gll tɕʰeme kɯ-mpɕɯ\redp{}mpɕɤr nɤ-ɕɣa kɯ-xtɕɯ\redp{}xtɕi ʑo, tɯrme ntsɯ tɯ-ndze mɤ-kɯ-ra ci a-nɯ-tɯ-ɤβzu smɯlɤm \\
	girl \textsc{sbj}:\textsc{pcp}-\textsc{emph}\redp{}be.beautiful   \textsc{2sg}.\textsc{poss}-tooth/age  \textsc{sbj}:\textsc{pcp}-\textsc{emph}\redp{}be.small \textsc{emph}  people always 2-eat[III]:\textsc{fact} \textsc{neg}-\textsc{sbj}:\textsc{pcp}-be.needed \textsc{indef} \textsc{irr}-\textsc{pfv}-2-become prayer \\
	\glt `May you become a very beautiful, nice and young girl who does not have to eat humans.' (Norbzang 2012, 328-329)
\end{exe}

It also occurs with the Factual Non-Past (\ref{ex:tWmACi.smWlAm}), and can have scope over several clauses, comprising both verbs in the Factual Non-Past and the Irrealis, as in (\ref{ex:tWscitnW.nWkWmNAm.smWlAm}).

\begin{exe}
	\ex \label{ex:tWmACi.smWlAm}
	\gll  nɯ tɤ-nɯ-ndɤm tɕe tɯ-mɤɕi smɯlɤm  \\
	\textsc{dem} \textsc{imp}-\textsc{auto}-take[III] \textsc{lnk} 2-be.rich:\textsc{fact} prayer \\
	\glt `Take (these cattle and, and may you be rich!' (2003kAndzwsqhaj2, 142)
\end{exe}

\begin{exe}
	\ex \label{ex:tWscitnW.nWkWmNAm.smWlAm}
	\gll    nɯʑora kɯnɤ tɯ-scit-nɯ tɕe nɯ, nɯ-kɯ-mŋɤm ra a-pɯ-me smɯlɤm nɤ! \\
	\textsc{2pl} also 2-be.happy:\textsc{fact} \textsc{lnk} \textsc{dem} \textsc{2pl}.\textsc{poss}-\textsc{sbj}:\textsc{pcp}-hurt \textsc{pl} \textsc{irr}-\textsc{ipfv}-not.exist prayer \textsc{sfp} \\
	\glt `May you be happy and be spared of any disease.' (conversation, 15-01-02)
\end{exe}

The basic meaning of  \forme{smɯlɤm} is still attested in Japhug, as in (\ref{ex:smWlAm.smWlAm}), and its sentence-final function is performative, and particularly common in formal prayers and wishes.

\begin{exe}
	\ex \label{ex:smWlAm.smWlAm}
	\gll    smɯlɤm nɯ na-ndɯn-nɯ ndɤre, ``ki ɯ-ʁɤri nɯ sɯstaʁ ʑo kɯ-ɤcʰɯcʰa, ki ɯ-ʁɤri nɯ sɯstaʁ ʑo kɯ-pe [...] a-nɯ-βze smɯlɤm" ta-tɯt-nɯ \\
	prayer \textsc{dem} \textsc{aor}:3\flobv{}-read-\textsc{pl}  \textsc{lnk} \textsc{dem}.\textsc{prox} \textsc{3sg}.\textsc{poss}-before \textsc{dem} \textsc{comp} \textsc{emph} \textsc{sbj}:\textsc{pcp}-be.capable \textsc{dem}.\textsc{prox} \textsc{3sg}.\textsc{poss}-before  \textsc{dem} \textsc{comp}  \textsc{emph} \textsc{sbj}:\textsc{pcp}-be.good  {   } \textsc{irr}-\textsc{pfv}-make[III] prayer  \textsc{aor}:3\flobv{}-say[II]-\textsc{pl} \\
	\glt `They recited the prayer, and said `May you become even stronger and better than before.'' (Norbzang 2005, 410-412)
\end{exe}  
%nɯ-βde wo, tɕiʑɤɣ ɯ-mɤ-pjɯ-ŋu smɯlɤm nɯ

The noun \forme{smɯlɤm} can be denominalized with the prefix \forme{nɯ-} (§\ref {sec:denom.tr.nW}) as the transitive complement-taking verb \japhug{nɯsmɯlɤm}{wish}, which is compatible with either infinitival complements or complement clauses in the Irrealis as in (\ref{ex:athWtWmACi.YWnWsmWlama}).

\begin{exe}
	\ex \label{ex:athWtWmACi.YWnWsmWlama}
	\gll  nɤʑo a-tʰɯ-tɯ-mɤɕi  ɲɯ-nɯ-smɯlam-a \\
	\textsc{2sg} \textsc{irr}-\textsc{pfv}-2-be.rich \textsc{sens}-\textsc{denom}-prayer-\textsc{1sg} \\
	\glt `I wish that you become rich.' (elicited)
\end{exe}  
