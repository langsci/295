\chapter{Non-concatenative verbal morphology} \label{chap:nonconcatenative}
While Japhug lacks tonal alternations like most Gyalrongic languages \citep{jackson05yingao, lai17khroskyabs, gong18these, zhangsy18stem}, non-concatenative segmental alternations are common.

This chapter briefly presents derivation onset alternations (§\ref{sec:onset.alternations}), and then provides a detailed account of inflectional stem alternations (§\ref{sec:stem.alternation}), vowel contraction (§\ref{sec:contraction}) and partial reduplication (§\ref{sec:redp.verb}) in verbal morphology.



\section{Onset alternations} \label{sec:onset.alternations}
While most of the prefixal derivation morphology in Japhug is fairly concatenative, a handful of prefixal morphological processes involve consonantal alternations. 

Prenasalization is found in the anticausative (§\ref{sec:anticausative}) and a handful of frozen verb forms (§\ref{sec:fossil.prenasalization}, §\ref{sec:caus.Z}).  In all cases, it is probably the result of the fusion of a nasal prefix with an unvoiced stop/affricative, turning it into the corresponding voiced prenasalized obstruent (§\ref{sec:anticausative.direction}).

Another type of consonant alternation is the \forme{β-} \fl{} \forme{b-} fortition in the causative verb  \japhug{zbraʁ}{attach together} (§\ref{sec:sig.caus.irregular.other}).


\section{Stem alternations} \label{sec:stem.alternation}
Northern Gyalrong languages have three verbal stems, labeled I, II and III following \citet{jackson00sidaba} (Situ dialects lack stem III, but some varieties have additional stems I' and II', see \citealt{zhangsy18stem}). Stem I is the default form, stem II mainly occurs in Aorist and Past Imperfective, and stem III occurs in the Non-Past \textsc{sg}\fl{}3 direct forms of some transitive verbs (§\ref{sec:indexation.mixed}).

In Japhug, no verb has more than two different stems. Stem II is different from stem I in only a handful of irregular verbs (§\ref{sec:stem2}), while stem III has been regularized (§\ref{sec:stem3}).

\subsection{Stem II} \label{sec:stem2}

\subsubsection{Morphology} \label{sec:stem2.form}
Unlike in Tshobdun \citep{jackson00sidaba}, Zbu (\citealt{jackson04showu, gong18these}) and Situ (\citealt{linyj03tense, zhangsy18stem}), stem II in Japhug is very limited, and only found in three irregular verbs and in a few derivations from them  (\tabref{tab:stem2}).  

 \begin{table} 
\caption{Stem II alternations in Japhug Rgyalrong} \label{tab:stem2} \centering
\begin{tabular}{llllll}
\lsptoprule
Stem I   &Stem II &Derivation\\
\midrule
\japhug{ɕe}{go} &  \forme{-ari} \\
\japhug{ɣi}{come} &\forme{-ɣe} \\
\japhug{ti}{say} &\forme{-tɯt} \\
\midrule
\japhug{sɯxɕe}{send, let go}  &\forme{-sɤɣri} &Causative\\
\japhug{sɯti}{cause to say}  &\forme{-sɯtɯt} & \\
\midrule
\japhug{nɯɕe}{go back}  &\forme{-anɯri} &Vertitive, Autive \\
\japhug{nɯɣi}{come back}  &\forme{-nɯɣe} & \\
\midrule
 \japhug{nɤɕɯɕe}{go around} & \forme{-anɤrɯri} &Distributed action \\
  \japhug{nɤtɯti}{tell around} & \forme{-nɤtɯtɯt} &  \\
\lspbottomrule
\end{tabular}
\end{table}

The \forme{ɣi/ɣe} alternation has an exact correlate in Tshobdun (\forme{wi/wɛʔ}, \citealt[175]{jackson00sidaba}). The verb \japhug{ti}{say} has an irregular correspondence with other Gyalrong languages, where an affricate onset is found (Tshobdun \forme{´tsə} `say', \citealt[174]{jackson00sidaba}, Situ  \forme{tsə̂, tsīs}, \citealt[318]{zhangsy18stem}). In Zbu however, a unique alternation between affricate and stop is found in this verb (stem I \forme{´tsʰə}, stem II \forme{´tʰit},  \citealt[225]{gong18these}). The correspondence between the Japhug stem II \forme{-tɯt} and its Zbu equivalent \forme{´tʰit} is regular (apart from the aspiration). The irregular paradigm of this verb in Zbu certainly has to be reconstructed to proto-Gyalrong. Situ and Tshobdun have generalized the affricate to stem II, while Japhug has remade the stem I by generalizing the dental stop.

Suppletion is observed in the paradigm of \japhug{ɕe}{go}, whose stem II \forme{-ari} is perhaps related to the stem I of the verb `go' found in some dialects of Zbu (\forme{rî}, \citealt[274]{jackson04showu}). Xtokavian dialects of Japhug have \forme{tʰɐl} as the stem II of this verb \citep{linluo03}, a form borrowed from Tibetan \tibet{ཐལ་}{tʰal}{go beyond, pass}.

The transitive verb \japhug{sɯxɕe}{send, let go} is the causative of  \japhug{ɕe}{go}, and its stem II presents an irregular vowel merger \forme{sɯɣ+ari} $\rightarrow$ \forme{sɤɣri}, instead of expected $\dagger$\forme{sɤri} or $\dagger$\forme{asɯɣri} (§\ref{sec:caus.sAG}). Other derivations from \japhug{ɕe}{go} also have suppletion, but the alternations are regular. The autive and the vertitive \forme{nɯ-} prefixes (§\ref{sec:autobenefactive} , §\ref{sec:vertitive}) in \forme{nɯ-ɕe} `go by oneself'/`go back' are predictably infixed (§\ref{sec:inner.prefixal.chain}) within the stem II \forme{ari} as \forme{a<nɯ>ri}  (see examples \ref{ex:Wrte.YAnWBde2} in §\ref{sec:autoben.lexicalized} and \ref{ex:tWZo.kW} in §\ref{sec:genr.pro}). the distributed action derivation \japhug{nɤɕɯɕe}{go around} likewise has the expected reduplicated stem \forme{-anɤrɯri}.
 
The verb \japhug{nɯɣi}{come back}, vertitive of \japhug{ɣi}{come}, shows the expected stem II \forme{nɯɣe} with vowel alternation. Its lexicalized causative \japhug{sɯɣe}{invite} on the other hand, has lost stem alternation, but appears to have generalized stem II (§\ref{sec:sig.caus.irregular.other}).

 \subsubsection{Distribution} \label{sec:stem2.distribution}
 In Japhug, stem II is mainly found in the Aorist (§\ref{sec:aor.morphology})  and in the Apprehensive (§\ref{sec:apprehensive.morphology}). The verb   \japhug{ti}{say} also has stem II \forme{tɯt} in the irregular Progressive Sensory form  \forme{ɲɯ-ɤsɯ-tɯt}, \forme{ɲɯ-ɤs-tɯt} `he is/was saying' (§\ref{sec:sensory.morphology}). In Zbu, stem II is also found in the Progressive (\citealt{jackson00sidaba}, \citealt[196]{gong18these}).
 
 Unlike Stem III (§\ref{sec:stem3.distribution}), Stem II is insensitive to person and number in all known Gyalrong languages.
 
\subsection{Stem III}  \label{sec:stem3}
\subsubsection{Morphology} \label{sec:stem3.form}
In the Kamnyu dialect of Japhug, stem III is fully regular, and applies to all transitive verbs with an open syllable stem ending in a non-front vowel in the expected contexts (§\ref{sec:stem3.distribution}). 

As presented in \tabref{tab:stem3}, two types of alternations are attested: vowel fronting in the case of stems in \forme{-a}, \forme{-u}, \forme{-ɯ} and \forme{-m} suffixation with vowel unrounding for stems in \forme{-o}. In Xtokavian dialects, the alternations are less predictable, as some verbs in \forme{-u} and \forme{-ɯ} can take the \forme{-m} suffix (\citealt{linluo03}, \citealt[231--234]{jacques08zh}). A cognate suffix \forme{-m} in stem III  is also attested in Zbu (\citealt[228--229]{gong18these}).

 \begin{table} 
\caption{Stem III alternations in the Kamnyu dialect of Japhug} \label{tab:stem3}
\begin{tabular}{lll}
\lsptoprule
Stem I & Stem IIII& type \\
\midrule
\forme{-a} & \forme{-e} & vowel fronting\\
\forme{-u} & \forme{-e} &  \\
\forme{-ɯ} & \forme{-i} &  \\
\midrule
\forme{-o} & \forme{-ɤm} & \forme{-m} suffixation \\
\lspbottomrule
\end{tabular}
\end{table}

Stem III in \forme{-ɤm} undergoes regular vowel assimilation to \forme{-am-} when followed by a \textsc{1sg} \forme{-a} suffix (§\ref{sec:intr.1}). For instance, the \textsc{1sg}\fl{}\textsc{3sg} Factual Non-Past of \japhug{mto}{see} is \forme{mtam-a} `I (will) see it/him/her' rather than $\dagger$\forme{mtɤm-a}. Stem III with the fronted vowel \forme{-e} change to \forme{-i} when followed by \forme{-a}: \forme{ndze-a} `I (will) eat it' is realized as \phonet{ndzia}.

Vowel fronting in stem III originates from the fusion of the verb stem with a \forme{*-j} suffix (§\ref{sec:historical.phono}, \citealt[357]{jacques04these}, \citealt[234]{jacques08zh}), cognate to the `transitivity marker' \forme{-jə} in Tshobdun (\citealt[496]{jackson03caodeng}). 

Rather than analyzing the \forme{-m} suffix as a part of the stem III, it could alternatively be possible to view it as part of the suffixal chain. Under such an analysis, it would be located in slot +1 (§\ref{sec:suffixes}). Bipartite verbs provide evidence against such an analysis however (§\ref{sec:bipartite}).

\subsubsection{Distribution}\label{sec:stem3.distribution}
Stem III only occurs in finite verb forms. It is found in Factual Non-Past (§\ref{sec:fact.morphology}), Egophoric Present (§\ref{sec:egophoric.morphology}), Sensory (§\ref{sec:sensory.morphology}), Imperative (§\ref{sec:imp.morphology}), Irrealis (§\ref{sec:irrealis.morphology}) and Imperfective (§\ref{sec:ipfv.morphology}) of transitive verbs in the \textsc{1sg}\fl{}3, \textsc{2sg}\fl{}3 and \textsc{3sg}\flobv{} configurations (§\ref{sec:indexation.mixed}).  

\tabref{tab:stem3.mto.mtAm} illustrates all five person configurations of the Factual Non-Past where stem III appears and a selection of other configurations where stem I is used instead. It has exactly the same distribution in the other TAME categories listed above.

\begin{table}
\caption{Stem I vs. Stem III in the paradigm of \japhug{mto}{see}} \label{tab:stem3.mto.mtAm}
\begin{tabular}{lllll}
\lsptoprule
Person configuration & Stem & Example \\ 
\midrule
\textsc{1sg}\fl{}\textsc{3sg} & III & \forme{mtam-a} \\
\textsc{1sg}\fl{}\textsc{3du} & III & \forme{mtam-a-ndʑi} \\
\textsc{1sg}\fl{}\textsc{3pl} & III & \forme{mtam-a-nɯ} \\
\textsc{2sg}\fl{}3 & III & \forme{tɯ-mtɤm} \\
\textsc{3sg}\flobv{} & III & \forme{mtɤm} \\
\midrule
\textsc{1du}\fl{}3 & I & \forme{mto-tɕi} \\
\textsc{2du}\fl{}3 & I & \forme{tɯ-mto-ndʑi} \\
\textsc{3du}\flobv{} & I & \forme{mto-ndʑi} \\
3$'$\fl{}\textsc{3sg} & I & \forme{ɣɯ-mto} \\
\lspbottomrule
\end{tabular}
\end{table}

Stem III encodes four morpholosyntactic features, comprising both TAME and person indexation: direct configuration (§\ref{sec:direct-inverse}), singular transitive subject, third person object and Non-Past tense.

\subsubsection{Backformation} \label{sec:stem3.backformation}
An indirect consequence of the perfect regularity of stem III formation (§\ref{sec:stem3.form}) is that in some rare cases, the stem I has been generated from the stem III by applying the alternations backwards.

The stem I of \japhug{kʰo}{give, pass} (§\ref{sec:ditransitive.indirective}) regularly corresponds to Tshobdun \forme{´kʰi} (as if from proto-Gyalrong \forme{*kʰaŋ}). On the other hand, Zbu \forme{kʰɐ̂m, kʰə̂m, ´kʰəm} `give' (\citealt[229]{gong18these}) and Tangut  \tangut{𘓯}{1105}{kʰjow}{1.56} (\citealt[200--201]{jacques14esquisse}, proto-Tangut \forme{*khjVm}) match the Japhug stem III \forme{kʰɤm}. A possible hypothesis to account for these diverging correspondences would be that Japhug and Tshobdun have preserved the original verb root, and that Zbu and Tangut have generalized stem III to the whole paradigm. However, there are three reasons why such an explanation is problematic. 

First, the verb \tangut{𘓯}{1105}{kʰjow}{1.56} is highly irregular, and therefore unlikely to have been analogized. Second, Zbu is phylogenetically closer to Tshobdun and Japhug than it is to Tangut, and therefore shared features between Zbu and Tangut are more likely to be due to common retention than to common innovation. Third, there is Japhug-internal evidence that the stem I \forme{-o} is secondary.

The transitive verb \japhug{fkro}{put in order}, arrange', whose stem III is \forme{fkrɤm} is borrowed from the past tense of the Tibetan verb \tibet{འགྲེམ་བཀྲམ་}{ⁿgrem-bkram}{spread out}. The stem I \forme{fkro} is clearly backformed from the stem III, by overapplication of the \forme{-o/-ɤm} alternation. Some speakers treat this verb as having a non-alternating stem \forme{fkrɤm}, showing that backformation is still an ongoing process.

The case of \forme{fkro} offers a model to analyze Japhug \japhug{kʰo}{give, pass} and Tshobdun \forme{´kʰi} (\citealt[201]{jacques14esquisse}): this verb originally had a stem I \forme{*kʰɐm} in proto-Gyalrong, preserved in Zbu, but underwent backformation to \forme{*kʰaŋ} in the common ancestor of Japhug and Tshobdun, subsequently evolving to \forme{kʰo} and \forme{´kʰi} by the application of regular sound laws. This is one piece of evidence that Tshobdun is closer to Japhug than it is to Zbu.

The regular stem III \forme{βze} of the verb \japhug{βzu}{make} does appear in some non-finite forms such as \forme{cʰɯ-kɯ-βze} in (\ref{ex:Wmat.chWkWBze}) (see also \ref{ex:YWkWnWBze} in §\ref{sec:autoben.spontaneous}), in the dummy transitive subject construction (§\ref{sec:transitive.dummy}), meaning `grow' or in collocation with the noun \forme{ɯ-tsa} (in the meaning `be suitable', §\ref{sec:Bzu.lv}) for instance.

\begin{exe}
\ex \label{ex:Wmat.chWkWBze}
\gll <hulu> nɯ si ci ɲɯ-ŋu, ɯ-mat cʰɯ-kɯ-βze ci. tɕe ɯ-mat nɯnɯ, ɯ-taʁ ku-kɯ-xtsʰɯm, ɯ-pa ɲɯ-kɯ-jpum ci cʰɯ-βze ɲɯ-ŋu tɕe, nɯ <hulu> tu-sɤrmi-nɯ. \\
gourd \textsc{dem} tree \textsc{indef} \textsc{sens}-be \textsc{3sg}.\textsc{poss}-fruit \textsc{ipfv}-\textsc{sbj}:\textsc{pcp}-make[III] \textsc{indef} \textsc{lnk} \textsc{3sg}.\textsc{poss}-fruit \textsc{dem} \textsc{3sg}.\textsc{poss}-top \textsc{ipfv}:\textsc{east}-\textsc{sbj}:\textsc{pcp}-be.thin \textsc{3sg}.\textsc{poss}-bottom \textsc{ipfv}:\textsc{west}-\textsc{sbj}:\textsc{pcp}-be.thick \textsc{indef} \textsc{ipfv}-make[III] \textsc{sens}-be \textsc{lnk} \textsc{dem} gourd \textsc{ipfv}-call-\textsc{pl} \\
\glt `The gourd is a tree, one which grows fruits. It grows fruits that are thinner on the top part and thicker in the bottom, they are called `gourd'.' (150825 huluwa-zh, 2-3)
\end{exe}

Rather than an exception to the rules described in §\ref{sec:stem3.distribution}, it is simpler to consider that a synchronic verb root \japhug{βze}{grow} different from \japhug{βzu}{make} has been created by backformation (§\ref{sec:stem3.backformation}) from finite imperfective forms such as \forme{cʰɯ-βze} `it grows' in (\ref{ex:Wmat.chWkWBze}). The stem \forme{βze} however never occurs in past tenses (Inferential and Aorist), so that this verb is an example of partial and ongoing backformation. 

Another possible example of the same type of backformation is discussed in §\ref{sec:pWrndeta}.

\subsection{Frozen \forme{-t} suffix} \label{sec:t.free.variation}
Three verbs presented in \tabref{tab:frozen.t} have two alternative stem forms in free variation, with an optional \forme{-t} suffix. This suffix appears in all person and TAME forms without any semantic change; compare for instance the Imperfective \textsc{1pl} forms \forme{ku-rɤʑit-i} (\textsc{ipfv}-stay-\textsc{1pl}, example \ref{ex:jisAxCe.maNe}, §\ref{sec:other.oblique.participle.relatives}) and \forme{ku-rɤʑi-j} (\textsc{ipfv}-stay-\textsc{1pl}, example \ref{ex:iZo.kWBde}, §\ref{sec:uses.numerals}), uttered by the same speaker. The two variants are freely interchangeable, but some speakers use the \forme{-t} suffixed forms more frequently that others.

The optional \forme{-t} is not synchronically relatable to the past tense \forme{-t}, which is restricted to \textsc{1sg}\fl{}3 and \textsc{2sg}\fl{}3 Aorist and Inferential  (§\ref{sec:indexation.mixed}), and never appears on intransitive forms, or to any of the frozen derivation \forme{-t} suffixes (§\ref{sec:applicative.t}, §\ref{sec:antipassive.t}).  

\begin{table}
\caption{Frozen \forme{-t} suffix in free variation} \label{tab:frozen.t}
\begin{tabular}{lll}
\lsptoprule
Base form & Alternative stem & Transitivity \\
\midrule
\japhug{amdzɯ}{sit}& \forme{amdzɯt} & vi \\
\japhug{rɤʑi}{stay}& \forme{rɤʑit}& vi \\
\japhug{rɤɕi}{pull} & \forme{rɤɕit} & vt \\
\lspbottomrule
\end{tabular}
\end{table}
 
It could be a trace of  the stem II \forme{-t} suffix, attested in the verb \forme{ti, tɯt} `say' in Japhug (§\ref{sec:stem2}), and in a handful of other verbs in Zbu  (\citealt[224--225]{gong18these}). In this hypothesis, these three verbs used to have a  stem II with the \forme{-t} suffix, but at the present stage the two stems have ceased to be morphologically contrastive.
 
 
\section{Vowel contraction} \label{sec:contraction}
Contracting verbs have polysyllabic stems whose first syllable is \forme{a-}. All of these verbs are  morphologically intransitive (§\ref{sec:transitivity.morphology}), but some of them can have semi-objects or oblique arguments (§\ref{sec:semi.transitive}). The \forme{a-} element can be a valency-decreasing prefix (including the passive  §\ref{sec:passive}, reciprocal (§\ref{sec:reciprocal} or distributed property §\ref{sec:distributed.amW} prefixes), a denominal prefix (the stative denominal \forme{a-} §\ref{sec:denom.a} or a disyllabic denominal prefix with \forme{a-} as its first syllable, §\ref{sec:denom.contracting}), a deideophonic prefix (§\ref{sec:a.nA.deidph}) or a synchronically non-analyzable element (as in \japhug{aro}{own} or \japhug{amdzɯ}{sit}).
 
To represent vowel contraction in a straightforward way, the notation employed in this grammar separates the vowel of the preceding prefix and the contracting by a hyphen, the prefix is transcribed in its base form, and the result of the vowel contraction is indicated after the hyphen. For instance \forme{pɯ-a-} is to be read as `prefix \forme{pɯ-} merging with the contracting vowel as \ipa{pa}', while \forme{pɯ-ɤ-} indicates merger of the preverb with the contracting vowel as \ipa{pɤ}. This orthographic rule transcribes contracting vowels in several ways (\forme{-ɤ\trt}, \forme{-a-} or \forme{-o-}) depending on the prefix with which they merge (see \tabref{tab:preverb.contraction} below). It is important to note however that there is no contrast between \forme{-a\trt}, \forme{-ɤ-} and \forme{-o-} as contracting vowels: there are simply morphologically-conditioned allomorphs.\footnote{The \forme{a-} allomorph is used as citation form of contracting verbs. }
 
 This convention has the advantage of disambiguating verb forms that have become identical due to vowel contraction (for instance, \forme{tɤ-ɤ-} vs. \forme{tɯ-ɤ\trt}, both of which surface as \ipa{tɤ}), and making the verb forms more easily parseable in the corpus.
 
 In Factual Non-Past unprefixed verb forms, the \forme{a-} element surfaces without alternation, as in (§\ref{ex:aroa.me}). 
 
 \begin{exe}
 	\ex \label{ex:aroa.me}
 	\gll nɯ ma aro-a me qʰe, \\
 	\textsc{dem} apart.from possess:\textsc{fact}-\textsc{1sg} not.exist:\textsc{fact} \textsc{lnk} \\
 	\glt `I don't have anything else.' (2003tamukatsa, 112)
 \end{exe}
 
 The prefixes in slots -6, -5 and -4 can occur in direct contact with the contracting vowel \forme{a-} only in Factual Non-Past non-second person forms (§\ref{sec:fact.morphology}). The vowel of Rhetorical Interrogative \forme{ɯβrɤ\trt}, Proximative aspect \forme{jɯ-} (-6) negative \forme{mɤ-} (-5) and associated motion prefixes \forme{ɕɯ-} and \forme{ɣɯ-} (-4) merge as \forme{a-}: \forme{ɯβrɤ-a-} $\rightarrow$ \ipa{ɯβra\trt}, \forme{jɯ-a-} $\rightarrow$ \ipa{ja\trt}, \forme{mɤ-a-} $\rightarrow$ \ipa{ma\trt}, \forme{ɕɯ-a-} $\rightarrow$ \ipa{ɕa-} and \forme{ɣɯ-a-} $\rightarrow$ \ipa{ɣa\trt}, as shown in \forme{ɕɯ-anbaʁ-i} `we go and hide' \ipa{ɕanbaʁi} in (\ref{ex:CanbaRi}), \forme{mɤ-a-rɤt} `it has not yet been written' \ipa{marɤt} (example \ref{ex:marAt}, §\ref{sec:passive.stative}).
 
 \begin{exe}
 	\ex \label{ex:CanbaRi}
 	\gll  rŋgɯ ɯ-ŋgɯ nɯtɕu ɕɯ-anbaʁ-i ɯmɤ-kɯ-ntsʰi-ci ma, \\
 	boulder \textsc{3sg}.\textsc{poss}-in \textsc{dem}:\textsc{loc} \textsc{tral}-hide:\textsc{fact}-\textsc{1pl} \textsc{prob}-\textsc{peg}-be.better-\textsc{peg} \textsc{lnk} \\
 	\glt `It looks like we should go and hide in the (hollow) boulder.' (160706 poucet6, 65)
 \end{exe}
 
 The interrogative \forme{ɯ-} in slot -6 does not merge with \forme{a-}. Rather, an epenthetic \forme{-j-} consonant is inserted between the two, as in (\ref{ex:WjatsWtsu}) with the verb \japhug{atsɯtsu}{have time}.
 
 \begin{exe}
 	\ex \label{ex:WjatsWtsu}
 	\gll   a-wa tɯrju χsɯ-ŋka tu-ti-a ɯ-j-átsɯtsu \\
 	\textsc{1sg}.\textsc{poss}-father word three-word \textsc{ipfv}-say-\textsc{1sg} \textsc{qu}-X-have.time:\textsc{fact} \\
 	\glt `Father, do I have time to say three sentences?' (2003qachga, 173)
 \end{exe}
 
 In all finite verb forms other than the Factual non-Past, the slot -3 (§\ref{sec:outer.prefixal.chain}) is filled, and vowel contraction occurs between the vowel of the preverb and the \forme{a-} element. The rules of contraction with preverbs are detailed in §\ref{sec:preverbs.contracting.verbs}, but \tabref{tab:preverb.contraction} presents a summary.
 
 Type C preverbs, which are restricted to transitive verbs, never occur with contracting verbs. In the case of type D preverbs, vowel contraction is avoided by the insertion of the peg circumfix (see §\ref{sec:peg.circumfix}, §\ref{sec:preverbs.contracting.verbs}).
 
 \begin{table}
 	\caption{Vowel contraction rules with orientation preverbs} \label{tab:preverb.contraction}
 	\begin{tabular}{lllll}
 		\lsptoprule
 		TAME & Preverb  &Vowel of the   & Vowel  \\ 
 		&type&preverb&contraction \\
 		\midrule
 		Aorist &A & \forme{-ɤ} & \forme{ɤ-a} $\rightarrow$ \ipa{a} \\
 		&  & \forme{-ɯ} & \forme{ɯ-a} $\rightarrow$ \ipa{a} \\
 		\midrule
 		Irrealis, Imperative &A & \forme{-ɤ} & \forme{ɤ-ɤ} $\rightarrow$ \ipa{ɤ} \\
 		&  & \forme{-ɯ} & \forme{ɯ-ɤ} $\rightarrow$ \ipa{ɤ} \\
 		\midrule
 		Imperfective, Sensory, &B& \forme{-u} & \forme{u-o} $\rightarrow$ \ipa{o} \\
 		Egophoric Present&&\forme{-ɯ} & \forme{ɯ-ɤ} $\rightarrow$ \ipa{ɤ} \\
 		\lspbottomrule
 	\end{tabular}
 \end{table}
 
 These rules are not purely phonological, since the outcome of the vowel fusion with type A preverbs depends on the TAME category. For instance, in the case of the contracting verb \japhug{atɤr}{fall down}, the type A \textsc{downwards} \forme{pɯ-} preverbs merges as \ipa{pɤ-} in the Irrealis and Imperative (for example \forme{a-mɤ-pɯ-ɤtɤr} \ipa{amɤpɤtɤr} in \ref{ex:amApAtAr}) and as \ipa{pa-} in the Past Imperfective and Perfective (\forme{pɯ-atɤr} \ipa{patɤr} in \ref{ex:li.patAr}).
 
 \begin{exe}
 	\ex \label{ex:amApAtAr}
 	\gll   a-mɤ-pɯ-ɤtɤr kɤ-sɯso kɯ, \\
 	\textsc{irr}-\textsc{neg}-\textsc{pfv}-fall \textsc{inf}-think \textsc{erg} \\
 	\glt `In order to prevent (the child) from falling down...' (140426 tApAtso kAnWBdaR, 68)
 \end{exe}
 
 \begin{exe}
 	\ex \label{ex:li.patAr}
 	\gll li pɯ-atɤr ɲɯ-ŋu \\
 	again \textsc{aor}-fall \textsc{sens}-be \\
 	\glt `He fell down again.' (2003 tWxtsa, 104)
 \end{exe}
 
 In slot -2, since contracting verbs are all intransitive, only the second person \forme{tɯ-} and the generic \forme{kɯ-} prefixes and the prefixal element \forme{k(ɯ)-} of the peg circumfix can occur before the contracting vowel \forme{a-}. The outcome of vowel contraction depends on the TAME category (§\ref{sec:intr.23}: \forme{tɯ-a-} \ipa{ta-} in Factual Non-Past (\forme{tɯ-atɤr} `you will fall') and Aorist (\forme{pɯ-tɯ-atɤr} `you fell down') and \forme{tɯ-ɤ-}  \ipa{tɤ-} Imperfective (\forme{pjɯ-tɯ-ɤtɤr} `you fall down'), Sensory, Egophoric Present, Irrealis, Imperative and Prohibitive (\forme{ma-pɯ-tɯ-ɤtɤr} `don't fall down'). With the peg circumfix, which appears in the Inferential, the vowel fusion is always \forme{-k-ɤ-} (see example \ref{ex:mWYAkAtWGnWci}, §\ref{sec:peg.circumfix}).
 
 Slot -1 only contains prefixes associated with transitive verbs, which are therefore incompatible with contracting verbs.
 
 With non-finite verb form prefixes  (§\ref{chap:non-finite}),  vowel fusion is always \forme{ɯ-ɤ} / \forme{ɤ-ɤ}, as summarized in \tabref{tab:nmlz.contracting}.
 
 \begin{table}
 	\caption{Vowel contraction with non-finite prefixes} \label{tab:nmlz.contracting}
 	\begin{tabular}{lllll}
 		\lsptoprule
 		Prefix & Vowel contraction \\
 		\midrule
 		S/A participle  \forme{kɯ-} & \forme{kɯ-ɤ-} \ipa{kɤ-} \\
 		P participle  \forme{kɤ-} & \forme{kɤ-ɤ-} \ipa{kɤ-} \\
 		Oblique participle  \forme{sɤ-} & \forme{sɤ-ɤ-} \ipa{sɤ-} \\
 		Infinitive \forme{kɤ-}/\forme{kɯ-} & \forme{kɤ-ɤ\trt}, \forme{kɯ-ɤ-}  \ipa{kɤ-} \\
 		Degree nominal  \forme{-tɯ-} & \forme{-tɯ-ɤ-} \ipa{-tɤ-} \\
 		Action nominal / Dental infinitive \forme{tɯ-} & \forme{tɯ-ɤ-} \ipa{-tɤ-} \\
 		\lspbottomrule
 	\end{tabular}
 \end{table}
 
 With inner prefixes, vowel contraction is also straightforward and yields \forme{ɤ-} in all cases: for instance, the sigmatic causative, applicative and tropative of contracting verbs are \forme{sɯ-ɤ-} \ipa{sɤ-} (§\ref{sec:caus.sA}), \forme{nɯ-ɤ-} \ipa{nɤ-} (§\ref{sec:allomorphy.applicative}) and \forme{nɤ-ɤ-} \ipa{nɤ-} (§\ref{sec:tropative.allomorphy}), respectively. The autive \forme{nɯ-} cannot undergo vowel fusion and is rather infixed after the contracting vowel \forme{a-} (§\ref{sec:inner.prefixal.chain}, §\ref{sec:autoben.position}).
 
 The Progressive prefix \forme{asɯ-} (§\ref{sec:progressive.morphology}), located in slot -1 (§\ref{sec:outer.prefixal.chain}), can undergo vowel contraction in ways similar to those of contracting verbs, summarized in \tabref{tab:progressive.contraction}. The non-contracting autive \forme{nɯ-} and inverse \forme{-wɣ} prefixes are rather infixed between the \forme{a-} and the \forme{-sɯ/z-} elements.
 
 \begin{table}
 	\caption{Vowel contraction of the Progressive prefix} \label{tab:progressive.contraction}
 	\begin{tabular}{lllllll}
 		\lsptoprule
 		&Slot &Prefix & Transcription &Contracted forms& \\
 		\midrule
 		Past Imperfective & -3 &\forme{pɯ-} & \forme{pɯ-asɯ\trt}, \forme{pɯ-az-} &\ipa{pasɯ\trt}, \ipa{paz-}\\
 		Egophoric Present & -3 &\forme{ku-} & \forme{ku-osɯ\trt}, \forme{ku-oz-}&\ipa{kosɯ\trt}, \ipa{koz-} \\
 		Sensory & -3 &\forme{ɲɯ-} & \forme{ɲɯ-ɤsɯ\trt},  \forme{ɲɯ-ɤz-}& \ipa{ɲɤsɯ\trt}, \ipa{ɲɤz-} \\
 		\midrule
 		Second person & -2 &\forme{tɯ-}  & \forme{tɯ-ɤsɯ\trt}, \forme{tɯ-ɤz-} & \ipa{tɤsɯ\trt}, \ipa{tɤz-} \\
 		2\fl{}1 & -2 &\forme{kɯ-}  & \forme{kɯ-ɤsɯ\trt}, \forme{kɯ-ɤz-}& \ipa{kɤsɯ\trt}, \ipa{kɤz-} \\
 		peg & -2  & -\forme{(kɯ)-}  & \forme{k-ɤsɯ\trt}, \forme{k-ɤz-}&\ipa{kɤsɯ\trt}, \ipa{kɤz-} \\
 		\midrule
 		Inverse & -1 &\forme{wɣ-}  & \forme{-ɤ́<wɣ>sɯ\trt}, &\ipa{ó(ɣ)sɯ\trt},   \\
 		&&&\forme{ɤ́<wɣ>z-} &\ipa{ó(ɣ)z-} \\
 		\lspbottomrule
 	\end{tabular}
 \end{table}
 
 Vowel contraction is not restricted to the prefixal verbal template. Nouns whose stems begins in \forme{a-} are extremely few, but do present some instances of vowel contraction (§\ref{sec:a.nouns}). Vowel fusion involving the first person suffixes is discussed in §\ref{sec:intr.1}.
 
 
\section{Partial reduplication in verbal morphology } \label{sec:redp.verb}

\subsection{Initial reduplication} \label{sec:verb.initial.redp}
Verb-initial reduplication has a number of morphosyntactic functions, involving clause linking (conditionals §\ref{sec:redp.protasis}, temporal clauses (§\ref{sec:iterative.coincidence}), expressing increase of degree (§\ref{sec:redp.gradual.increase}), totality (§\ref{sec:totalitative.redp}) or temporal resilience (§\ref{sec:emphatic.autive}). With the possible exception of its use in the protasis of conditionals, all of the functions of the initial reduplication present an iconic component typical of reduplication crosslinguistically. 

Initial reduplication belongs to inflectional morphology. The only cases of lexicalization involve grammaticalization of reduplicated finite verb form into a discourse marker  (see §\ref{sec:delimitative} and §\ref{sec:redp.protasis} below).
 
 %{sec:vowel.harmony}
 \subsubsection{Morphophonology} \label{sec:initial.redp.morpho}
While initial reduplication follows the general rules of partial reduplication (§\ref{sec:partial.redp}), two additional rules have to be taken into account. 

First, the outcome of the reduplication of the negative prefix \forme{mɯ-} is \forme{mɯ\redp{}mɤ-} with vowel alternation (§\ref{sec:neg.allomorphs}), not distinguishable from that of the prefix \forme{mɤ-}.

Second, when the initial syllable of the word contains more than one morpheme, as in the case of the \forme{ɕ-} or \forme{ʑ-} allomorphs of the translocative prefix combined with an orientation preverb (§\ref{sec:translocative.morpho}), reduplication can disregard morpheme boundaries, resulting in a replication of the  two morphemes: \forme{ɕ-kɤ-tsʰi-t-a} (\textsc{tral}-\textsc{aor}-drink-\textsc{pst}:\textsc{tr}-\textsc{1sg}) thus yields \forme{\rouge{ɕ-kɯ}\redp{}ɕ-kɤ-tsʰi-t-a} as in  (\ref{ex:CkWCkAtshita}). Tshendzin however also accepts reduplication of the first morpheme only, reduplicated with addition of \forme{-ɯ}  (\forme{\rouge{ɕɯ}\redp{}ɕ-kɤ-tsʰi-t-a}).
 
\begin{exe}
\ex \label{ex:CkWCkAtshita}
\gll cʰa ɕkɯ\redp{}ɕ-kɤ-tsʰi-t-a ʑo lu-βzi-a ŋu \\
alcohol \textsc{iter}\redp{}\textsc{tral}-\textsc{aor}-drink-\textsc{pst}:\textsc{tr}-\textsc{1sg} \textsc{emph} \textsc{ipfv}-be.drunk-\textsc{1sg} be:\textsc{fact} \\
\glt `Each time I go and drink alcohol I get drunk.' (elicited)
\end{exe}

Example (\ref{ex:CtWtAkWtshAt}) illustrates a third possible pattern, with reduplication of the orientation preverb \forme{ɕ-tɤ-kɯ-tsʰɤt} $\Rightarrow$ \forme{ɕ-\rouge{tɯ}\redp{}tɤ-kɯ-tsʰɤt} without affecting the associated motion marker.

\begin{exe}
\ex \label{ex:CtWtAkWtshAt}
\gll kɯki tɯrme kɯra ɕ-tɯ\redp{}tɤ-kɯ-tsʰɤt ʑo nɯ pjɤ́-wɣ-sat-nɯ. \\
\textsc{dem}.\textsc{prox} person \textsc{dem}:\textsc{pl} \textsc{tral}-\textsc{total}\redp{}\textsc{aor}-\textsc{sbj}:\textsc{pcp}-try \textsc{emph} \textsc{dem} \textsc{ifr}-\textsc{inv}-kill-\textsc{pl} \\
\glt `All of these men who had went and tried (to discover the princesses' secret) were killed.' (140508 shier ge tiaowu de gongzhu-zh, 33)
\end{exe}

\subsubsection{Protasis of conditional} \label{sec:redp.protasis}
There are three possibilities in Japhug to mark the verbs in the protasis of conditionals (§\ref{sec:real.conditional}): Irrealis (§\ref{sec:irrealis.conditional}), Interrogative \forme{ɯ-} (§\ref{sec:interrogative.W.function}) and initial reduplication. In the latter two cases, the verb is generally followed by the additive linker \forme{nɤ}, as in (\ref{ex:atCW.tWtWNu.nA}).
 
\begin{exe}
\ex \label{ex:atCW.tWtWNu.nA}
\gll   a-tɕɯ tɯ\redp{}tɯ-ŋu nɤ, pɯ-ta-sɯxɕɤt nɯ ci nɯ-ndɯn ra \\
\textsc{1sg}.\textsc{poss}-son \textsc{cond}\redp{}2-be:\textsc{fact} \textsc{add} \textsc{aor}-1\fl{}2-teach \textsc{dem} once \textsc{imp}-read be.needed:\textsc{fact} \\
\glt `If you are my son, then recite (the mantra) that I have taught you.' (Norbzang 2012, 221)
\end{exe}

Conditional reduplication interacts with TAME categories. For instance, reduplicated Aorist can be used to refer to hypothetical future events as in (\ref{ex:tWtAtWtWt.nA}). Without the \forme{nɤ} linker, reduplicated Aorist can be interpreted as Iterative coincidence (§\ref{sec:iterative.coincidence}).

\begin{exe}
\ex \label{ex:tWtAtWtWt.nA}
\gll tɯ\redp{}tɤ-tɯ-tɯt nɤ tɕe pjɯ-ta-sat ŋu \\
\textsc{cond}\redp{}\textsc{aor}-2-say[II] \textsc{add} \textsc{lnk} \textsc{ipfv}-1\fl{}2-kill be:\textsc{fact} \\
\glt `If you tell (them) about (it), I will kill you.' (150901 changfamei-zh, 54)
\end{exe}

The scope of the protasis can go beyond the clause containing the reduplicated verb. In (\ref{ex:tWxtsa.mWmApWpe}), the protasis contains two clauses, the first one headed by the verb \forme{mɯ\redp{}mɤ-pɯ-pe} with reduplicated conditional, and the second one with the verb \forme{tu-kɯ-ŋke} without conditional marking.

\begin{exe}
\ex \label{ex:tWxtsa.mWmApWpe}
\gll tɕeri tɯ-mɤpa kɯnɤ, tɯ-xtsa mɯ\redp{}mɤ-pɯ-pe cʰondɤre,  tɤ-rʑaʁ kɯ-rɲɟi tu-kɯ-ŋke qʰe, tɯ-mɤpa ri cimbɤrom tu-rke ŋgrɤl. \\
\textsc{lnk} \textsc{genr}.\textsc{poss}-sole also \textsc{genr}.\textsc{poss}-shoe \textsc{cond}\redp{}\textsc{neg}-\textsc{pst}.\textsc{ipfv}-be.good  \textsc{comit} \textsc{indef}.\textsc{poss}-time \textsc{sbj}:\textsc{pcp}-be.long \textsc{ipfv}-\textsc{genr}:S/O-walk \textsc{lnk} \textsc{genr}.\textsc{poss}-sole \textsc{loc} blister \textsc{ipfv}-put.in[III] be.usually.the.case:\textsc{fact} \\
\glt `On the soles too, if one has had bad shoes, and walked for a long time, blisters will form on one's soles.' (27-tWfCAl, 139)
\end{exe}

The reduplicated conditional \forme{pɯ\redp{}pɯ-ŋu} of the Past Imperfective of the copula \japhug{ŋu}{be} has been grammaticalized as a topic marker \forme{pɯpɯŋunɤ} `as far as... is concerned' (§\ref{sec:delimitative}).


\subsubsection{Iterative coincidence} \label{sec:iterative.coincidence}
One of the meaning of initial reduplication with verbs in Aorist form is iterative coincidence (\citealt[295--296]{jacques14linking}). In this biclausal construction, the first clause with the reduplicated verb expresses the repetition of an event `every time $X$' (§\ref{sec:iterative.coincidence.clause}), and the second clause (with a main verb in the Imperfective) the resulting situation, as in (\ref{ex:tWtAGe.Zo}).

\begin{exe}
\ex \label{ex:tWtAGe.Zo}
\gll [kɯki tɕʰemɤpɯ ki si ɯ-kɯ-pʰɯt tɯ-tɤ-ɣe ʑo], nɯnɯ rgɤnmɯ nɯ ɣɯ ɯ-si ɲɯ-pʰɯt, ɯ-tɯ-ci z-ɲɯ-re, \\
\textsc{dem}.\textsc{prox} girl \textsc{dem}.\textsc{prox} wood \textsc{3sg}.\textsc{poss}-\textsc{sbj}:\textsc{pcp}-cut \textsc{iter}\redp{}\textsc{aor}:\textsc{up}-come[II] \textsc{emph} \textsc{dem} old.woman \textsc{dem} \textsc{gen} \textsc{3sg}.\textsc{poss}-wood \textsc{ipfv}-cut \textsc{3sg}.\textsc{poss}-\textsc{indef}.\textsc{poss}-water \textsc{tral}-\textsc{ipfv}:\textsc{west}-fetch[III] \\
\glt `Every time the girl came to cut firewood, she would cut firewood for the old woman and fetch water for her.' (150829 taishan zhi zhu-zh, 33)
\end{exe}

Although the semantic relationship between the two clauses is more a matter of temporal relationship rather than strict causality (for instance in example \ref{ex:tWtAGe.Zo}), this construction is clearly a subcase of the reduplicated conditional (§\ref{sec:redp.protasis}).
 
\subsubsection{Incremental} \label{sec:redp.gradual.increase}
Verb-initial reduplication with adjectival stative verbs can express a gradual increase of degree `become more and more $X$'. This function occurs with TAME categories which have an inchoative meaning when used with stative verbs: the Imperfective (§\ref{sec:ipfv.inchoative}) as in (\ref{ex:chWchWtsxot}) and (\ref{ex:chWchWmACindZi}), the Aorist (§\ref{sec:aor.inchoative}) and the Inferential (§\ref{sec:ifr.inchoative}). 
 
\begin{exe}
\ex \label{ex:chWchWtsxot}
\gll ʑŋgri mɤ-kɯ-tʂot ci, nɯ sɤz hanɯni kɯ-tʂot ci, nɯ sɤz hanɯni kɯ-tʂot ci, nɯ sɤz hanɯni kɯ-tʂot,
tɕe kɯβde ki tu-fse tɕe, [...] tʰi tʰɯ-ari ɯ-jɯja cʰɯ\redp{}cʰɯ-tʂot ʑo ɲɯ-ŋu tɕe, \\
star \textsc{neg}-\textsc{sbj}:\textsc{pcp}-be.bright \textsc{indef} \textsc{dem} \textsc{comp} a.little \textsc{sbj}:\textsc{pcp}-be.bright \textsc{indef} \textsc{dem} \textsc{comp} a.little \textsc{sbj}:\textsc{pcp}-be.bright \textsc{indef} \textsc{dem} \textsc{comp} a.little \textsc{sbj}:\textsc{pcp}-be.bright  \textsc{lnk} four \textsc{dem}.\textsc{prox} \textsc{ipfv}-be.like \textsc{lnk} { } downstream \textsc{aor}:\textsc{downstream}-go[II] \textsc{3sg}.\textsc{poss}-following \textsc{incr}\redp{}\textsc{ipfv}-be.bright \textsc{emph} \textsc{sens}-be \textsc{lnk} \\
\glt `(The constellation of the earthworm comprises) one non-bright star, another one slightly brighter, another one slightly brighter, another one slightly brighter, four (stars) like this (...), becoming brighter as they go downstream.' (29-mWBZi, 31-33)
\end{exe}

Verb-initial reduplication can be combined with verb repetition and the additive linker \forme{nɤ}  to put emphasis on the steadiness of the increase, as in (\ref{ex:chWchWmACindZi}).

\begin{exe}
\ex \label{ex:chWchWmACindZi}
\gll   ʑɯrɯʑɤri tɕe cʰɯ\redp{}cʰɯ-mɤɕi-ndʑi nɤ cʰɯ\redp{}cʰɯ-mɤɕi-ndʑi tɕe \\
progressively \textsc{lnk} \textsc{incr}\redp{}\textsc{ipfv}-be.rich-\textsc{du} \textsc{add} \textsc{incr}\redp{}\textsc{ipfv}-be.rich-\textsc{du} \textsc{lnk} \\
\glt `They progressively became richer and richer.' (02-deluge2012, 132)
\end{exe}

The negation \forme{mɯ-} of the Imperfective, reduplicated as \forme{mɯ\redp{}mɤ\trt}, can express gradual decrease, as in (\ref{ex:tWtucha.mWmAYWcha}).

\begin{exe}
\ex \label{ex:tWtucha.mWmAYWcha}
\gll ɬɤndʐi nɯ rca, tɯ\redp{}tu-cʰa ʑo pjɤ-ɕti. tɕendɤre, ɯʑo nɯnɯ tɕe ʑɯrɯʑɤri tɕe, mɯ\redp{}mɤ-ɲɯ-cʰa ʑo pjɤ-ɕti tɕe \\
demon \textsc{dem} \textsc{unexp}:\textsc{foc} \textsc{incr}\redp{}\textsc{ipfv}-can \textsc{emph} \textsc{ifr}.\textsc{ipfv}-be.\textsc{aff} \textsc{lnk} \textsc{3sg} \textsc{dem} \textsc{lnk} progressively \textsc{lnk} \textsc{incr}\redp{}\textsc{neg}-\textsc{ipfv}-can \textsc{emph} \textsc{ifr}.\textsc{ipfv}-be.\textsc{aff} \textsc{lnk} \\
\glt `The demon was becoming stronger and stronger, and he was weaker and weaker.' (140513 abide he mogui-zh, 81)
\end{exe}

\subsubsection{Totalitative} \label{sec:totalitative.redp}
Totalitative reduplication occurs on the main verb of relative clauses (§\ref{sec:totalitative.relatives}), expressing universal quantification (§\ref{sec:universal.quant}) of the relativized referent. It is attested in participial relatives (§\ref{sec:subject.participle.other.prefixes}), finite relatives (§\ref{sec:finite.relatives}) and also some partially lexicalized participles (§\ref{sec:lexicalized.subject.participle}).

In the case of participial relatives, the reduplicated syllable is the first syllable of the verb other than the possessive prefix, either the subject participle prefix \forme{kɯ-} (\ref{ex:kWkWspoR.Zo}), the object participle \forme{kɤ-} (\ref{ex:kWkAsWso}) or an orientation preverb (\ref{ex:tWtAkWrWkhArlAn}).

\begin{exe}
\ex \label{ex:kWkWspoR.Zo}
\gll laχtɕʰa ŋotɕu nɤ-kɤ-sɯso ʑo nɯnɯ, nɤ-mɲaʁ, nɤ-rna, nɤ-ɕna cʰo ra [kɯ\redp{}kɯ-spoʁ] nɯ ɯ-ŋgɯ tɕe a-kɤ-tɯ-rke qʰe \\
thing where \textsc{2sg}.\textsc{poss}-\textsc{obj}:\textsc{pcp}-think \textsc{emph} \textsc{dem} \textsc{2sg}.\textsc{poss}-eye \textsc{2sg}.\textsc{poss}-ear \textsc{2sg}.\textsc{poss}-nose \textsc{comit} \textsc{pl} \textsc{total}\redp{}\textsc{sbj}:\textsc{pcp}-have.a.hole \textsc{dem} \textsc{3sg}.\textsc{poss}-in \textsc{loc} \textsc{irr}-\textsc{pfv}:\textsc{east}-2-put.in[III] \textsc{lnk} \\
\glt `Whatever things you want (from the granary), put it in your eyes, your ears, your nose etc, all the holes in your body.' (31-deluge, 138)
\end{exe}

\begin{exe}
\ex \label{ex:kWkAsWso}
\gll wuma ʑo ɲɯ-pe ndɤre, [kɯ\redp{}kɤ-sɯso] nɯ ɲɯ-fse   \\
really \textsc{emph} \textsc{sens}-be.good \textsc{lnk} \textsc{total}\redp{}\textsc{obj}:\textsc{pcp}-think \textsc{dem} \textsc{sens}-be.like  \\
\glt `It is very nice, it is like everything that (I) want.' (2011-04-smanmi, 214)
\end{exe} 

\begin{exe}
\ex \label{ex:tWtAkWrWkhArlAn}
\gll  tɕe ɯ-tʰɤcu prɤscʰɯ ra [tɯ\redp{}tɤ-kɯ-rɯkʰɤrlɤn] kɯ nɯ-rdɤstaʁ nɯ ntsɯ s-cʰɤ-nɯ-ru-nɯ tɕe to-nɯ-ntɕʰoz-nɯ. \\
\textsc{lnk} \textsc{3sg}.\textsc{poss}-downstream  \textsc{topo} \textsc{pl} \textsc{total}\redp{}\textsc{aor}-\textsc{sbj}:\textsc{pcp}-build.house \textsc{erg} \textsc{3pl}.\textsc{poss}-stone \textsc{dem} always \textsc{tral}-\textsc{ifr}:\textsc{downstream}-\textsc{auto}-fetch-\textsc{pl} \textsc{lnk} \textsc{ifr}-\textsc{auto}-use-\textsc{pl} \\
\glt `All people from Praskyu down there who built/repaired their houses went there and took stones to use for themselves.' (140522 Kamnyu zgo, 196)
\end{exe}

In finite relatives, the verb lacks any overt nominalization marker other than the reduplication itself. In this construction, the relativized elements are either direct objects (as in \ref{ex:pWpWfCata.nWra.kW} and \ref{ex:spWspe} below), semi-objects or goals, but never subjects (§\ref{sec:finite.relatives}).

\begin{exe}
\ex \label{ex:pWpWfCata.nWra.kW}
\gll ɯ-ro nɯra [iɕqʰa pɯ\redp{}pɯ-fɕat-a] nɯra kɯ tɕe tɕe sɯjno tu-ndza-nɯ  \\
\textsc{3sg}.\textsc{poss}-rest \textsc{dem}:\textsc{pl} just.before \textsc{total}\redp{}\textsc{aor}-tell-\textsc{1sg} \textsc{dem}:\textsc{pl} \textsc{erg} \textsc{lnk}  \textsc{lnk} grass \textsc{ipfv}-eat-\textsc{pl} \\
\glt `The rest, all the (other animals) that I have told about just before eat grass.' (05-khWna, 46
\end{exe}


\begin{exe}
\ex \label{ex:spWspe}
\gll nɯ [spɯ\redp{}spe] nɯ to-nɤrmi ri tɤte <zhima> kɤ-ti nɯ ɲɤ-nɯ-jmɯt \\
\textsc{dem} \textsc{total}\redp{}be.able.to:\textsc{fact} \textsc{dem} \textsc{ifr}-call.name \textsc{lnk} that.is sesame \textsc{inf}-say \textsc{dem} \textsc{ifr}-\textsc{auto}-forget \\
\glt `He called the names of all (the crops) he knew, but forgot to say `sesame'.' (140512 alibaba-zh, 109)
\end{exe}

Finite transitive verbs with totalitative reduplication and a \textsc{1sg} subject can optionally take plural indexation (§\ref{sec:indexation.mixed}) corresponding to the relativized direct object with universal quantification. Object plural indexation in this case is rare (it is absent for instance in \ref{ex:pWpWfCata.nWra.kW} above), and only one example is attested in the corpus: \forme{pɯ\redp{}pɯ-mto-t-a-nɯ} (\textsc{total}\redp{}\textsc{aor}-see-\textsc{pst}:\textsc{tr}-\textsc{1sg}-\textsc{pl}) `all those that I have seen' (\ref{ex:pWpWmtotanW}, §\ref{sec:double.number.indexation}).


The totalitative subject participle of the existential verb \japhug{tu}{exist} can take a possessive prefix, which is interpreted as a possessor, as in \forme{a-kɯ\redp{}kɯ-tu} \textsc{1sg}.\textsc{poss}-\textsc{total}\redp{}\textsc{sbj}:\textsc{pcp}-exist `everything that I have'. No other totalitative verb form allows possessor prefixation. In particular, since the subject participles of transitive verbs require a possessive prefix coreferent with the object (§\ref{sec:subject.participle.possessive}), totalitative reduplication is incompatible with these forms. For instance, to express the meaning `all those who help me', forms such as $\dagger$\forme{a-tɯ\redp{}tu-kɯ-qur} (\textsc{1sg}.\textsc{poss}-\textsc{total}\redp{}\textsc{ipfv}-\textsc{sbj}:\textsc{pcp}-help) or $\dagger$\forme{a-kɯ\redp{}kɯ-qur} (\textsc{1sg}.\textsc{poss}-\textsc{total}\redp{}\textsc{sbj}:\textsc{pcp}-help) are unacceptable, and universal quantification has to be expressed with other means, for instance \forme{a-tu-kɯ-qur tʰamtɕɤt} (\textsc{1sg}.\textsc{poss}-\textsc{ipfv}-\textsc{sbj}:\textsc{pcp}-help all).

Totalitative reduplication of transitive subject participle is only possible if no possessive prefix is present (see \ref{ex:tWrme.pWpWkWmto}, §\ref{sec:totalitative.relatives}).

 %kɯki tɯrme kɯra ɕ-tɯ-tɤ-kɯ-tshɤt ʑo nɯ /pjɤz pjɤ/ pjɤ́-wɣ-sat-nɯ. 
 
% ndɤre, kɯ-mɤku nɯra kɯ-kɯ-sɤscit ʑo jɤ-ari-nɯ ɲɯ-ŋu.
%kɯ-maqhu ndɤ kɯ-kɯ-sɤɣmu ʑo ja-tsɯm-nɯ ɲɯ-ŋu.

\subsubsection{Emphatic autive}  \label{sec:emphatic.autive}
When occurring in word-initial position, the  autive prefix \forme{nɯ-} can be reduplicated to express emphatic permansive `still $X$ (regardless of whatever may happen)', and such verb form can be repeated with the additive linker \forme{nɤ}, as in (\ref{ex:nWnWtu}).

\begin{exe}
\ex \label{ex:nWnWtu}
\gll tɤtʰo nɯ pɯpɯŋunɤ, tɕe nɯ mɤ-fse tɕe tɕendɤre ɯ-jwaʁ nɯ nɯ\redp{}nɯ-tu nɤ nɯ\redp{}nɯ-tu qʰe \\
pine \textsc{dem} as.for \textsc{lnk} \textsc{dem} \textsc{neg}-be.like:\textsc{fact} \textsc{lnk}  \textsc{lnk} \textsc{3sg}.\textsc{poss}-leaf \textsc{dem} \textsc{emph}\redp{}\textsc{auto}-exist:\textsc{fact} \textsc{add} \textsc{emph}\redp{}\textsc{auto}-exist:\textsc{fact} \textsc{lnk} \\
\glt `As for the pine, it is not like (the other trees), (whatever happens), its leaves (needles) are still there.'  (07-tAtho, 16)
\end{exe}

The reduplicated autive should not be mistaken with the combination of a type A `westward' orientation preverb \forme{nɯ-} with the autive in spontaneous function (§\ref{sec:autoben.spontaneous})  as in (\ref{ex:nWnWme}).

\begin{exe}
\ex \label{ex:nWnWme}
\gll ɯʑo nɯ-nɯ-me ɕti. \\
\textsc{3sg} \textsc{aor}-\textsc{auto}-not.exist be.\textsc{aff}:\textsc{fact} \\
\glt `(The wart) disappeared by itself.' (24-pGArtsAG, 53)
\end{exe}
 %tʰa tɯmbri nɯ-nɯ-mbrɤt nɤ
\subsection{Verb stem reduplication as secondary exponence} \label{sec:verb.stem.redp}
Prefixal partial reduplication of the verb stem is a secondary exponence, used in combination with a prefix, in several productive verbal forms listed in \tabref{tab:verb.redp.prefix}. 

Stem reduplication occurs in two converbs, the gerund (§\ref{sec:gerund} and the purposive converb (§\ref{sec:purposive.converb}), together with a \forme{sɤ(z)-} prefix cognate to that of the oblique participle (§\ref{sec:oblique.participle}).

It is also found in four regular derivations: reciprocal (§\ref{sec:redp.reciprocal}, on the \forme{a-} prefix see §\ref{sec:applicative.history}), distributed action (§\ref{sec:distributed.action}), auto-evaluative (§\ref{sec:autoevaluative}) and attenuative (§\ref{sec:attenuative}). The attenuative differs from all other cases in that the vowel of the reduplicated syllable is in some cases \forme{-ɤ-} rather than \forme{-ɯ-}.

\begin{table}
\caption{Productive verbal forms with stem reduplication and prefixation} \label{tab:verb.redp.prefix}
\begin{tabular}{lllll}
\lsptoprule
Function  &\multicolumn{2}{l}{Example}  &  Reference \\
  \midrule
Gerund &\forme{mu} & \forme{\rouge{sɤ}-mɯ\redp{}mu} &§\ref{sec:gerund}\\
&`fear'&`fearing' \\
 \tablevspace 
Purposive converb &  \forme{jmɯt} &  \forme{ɯ-mɤ-\rouge{ɲɯ-sɤ}-jmɯ\redp{}jmɯt}&§\ref{sec:purposive.converb} \\
&`forget'&`in order not to forget' \\
 \tablevspace 
Reciprocal & \forme{rqoʁ} & \forme{\rouge{a}-rqɯ\redp{}rqoʁ} & §\ref{sec:redp.reciprocal} \\
&`hug'&`hug each other' \\
 \tablevspace 
Distributed action & \forme{mtsaʁ} & \forme{\rouge{nɤ}-mtsɯ\redp{}mtsaʁ} &§\ref{sec:distributed.action}  \\
&`jump'&`jump around' \\
 \tablevspace 
Auto-evaluative &\forme{mpɕɤr} & \forme{\rouge{znɤ}-mpɕɯ\redp{}mpɕɤr} &§\ref{sec:autoevaluative} \\
&`be beautiful'&`think of oneself as beautiful' \\
 \tablevspace 
Attenuative &\forme{wɣrum} &  \forme{\rouge{a}-ɣrɤ\redp{}ɣrum}& §\ref{sec:attenuative} \\
&`be white'&`be whitish' \\
\lspbottomrule
\end{tabular}
\end{table}

The locus of reduplication is the last syllable of the verb stem, disregarding indexation suffixes. When the verb stem contains more than one syllable, the reduplicated syllable is infixed. For instance, the gerund of \japhug{nɤre}{laugh} is \forme{sɤz-nɤ\rouge{rɯ}\redp{}re} `laughing': the partially replicated material  \forme{-rɯ-} occurs between the two syllables \forme{nɤ-} and \forme{-re} of the verb stem.  Additional reduplication is blocked on  lexically reduplicated verb stems such as \japhug{rɯru}{guard, take care of} (§\ref{sec:redp.voice}) or \japhug{nɯqambɯmbjom}{fly}: triplication is not attested in Japhug, unlike in Stau \citep{gates17triplication}.

Non-productive suffixal partial reduplication is found in the antipassive (§\ref{sec:antipassive.redp}) and distributed action derivation (§\ref{sec.distributed.action.l}).
 
Irregular partial reduplication with \forme{-oʁ} or \forme{-ɯm} in the replicated syllable instead of regular \forme{-ɯ}  are also attested (for instance \japhug{nɤʁɯmʁaʁ}{play around} from  \japhug{nɤʁaʁ}{have a good time}, §\ref{sec.distributed.action.oR}).
 
\subsection{Emphatic reduplication} \label{sec:emph.redp}
Emphatic reduplication is a partial reduplication in \forme{-ɯ} targeting the final syllable of the stem like other cases of verb stem reduplication (§\ref{sec:verb.stem.redp}). This inflectional reduplication has several related functions.

With adjectival stative verbs,  emphatic reduplication indicates a high degree, opposite of the attenuative reduplication (§\ref{sec:attenuative}): compare for instance \forme{qarŋɯ\redp{}rŋe} `be deep yellow' vs. \forme{aqarŋɯ\redp{}rŋe} `be yellowish'. Reduplicated adjectival verbs are almost always attested in participial form, as in (\ref{ex:kWmpCWmpCAr.kWpWpe}). 


\begin{exe}
\ex \label{ex:kWmpCWmpCAr.kWpWpe}
\gll tɕʰeme kɯ-mpɕɯ\redp{}mpɕɤr kɯ-pɯ\redp{}pe ci nɤ-ɕɣa  kɯ-xtɕɯ\redp{}xtɕi ci a-nɯ-tɯ-ɤβzu smɯlɤm \\
girl \textsc{sbj}:\textsc{pcp}-\textsc{emph}\redp{}be.beautiful \textsc{sbj}:\textsc{pcp}-\textsc{emph}\redp{}be.good \textsc{indef} \textsc{2sg}.\textsc{poss}-tooth/age  \textsc{sbj}:\textsc{pcp}-\textsc{emph}\redp{}be.small \textsc{indef} \textsc{irr}-\textsc{pfv}-2-become prayer \\
\glt `May you become a very beautiful, nice young girl.' (Norbzang 2012, 264)
\end{exe}

Rare examples of emphatic adjectival verbs in finite form are however also found in the corpus, in (\ref{ex:YWwGrWwGrum}).

\begin{exe}
\ex \label{ex:YWwGrWwGrum}
\gll ɯ-ku nɯra tɕe nɤki, ɯ-kɤχcɤl nɯra li ɲɯ-qarŋe qʰe ɲɯ-wɣrɯ\redp{}wɣrum kɯ-fse. \\
\textsc{3sg}.\textsc{poss}-head \textsc{dem}:\textsc{pl} \textsc{lnk} \textsc{filler} \textsc{3sg}.\textsc{poss}-top.head  \textsc{dem}:\textsc{pl} again \textsc{sens}-be.yellow \textsc{lnk} \textsc{sens}-\textsc{emph}\redp{}be.white \textsc{sbj}:\textsc{pcp}-be.like \\
\glt `It is yellow and very white on the top of its head.' (24-ZmbrWpGa, 24)
\end{exe}

With the negative existential verb \japhug{me}{not exist}, emphatic reduplication indicates radical non-existence `not (have/exist) ... at all', as in (\ref{ex:nAwa.pWnnWmWme}).\footnote{In (\ref{ex:nAwa.pWnnWmWme}), the mother of the character Padma 'Od'bar, who asked her about the identity of  father, gives this answer in the hope that her son will not try to find the truth about the disappearance of his father and thereby run into a mortal danger. }

\begin{exe}
\ex \label{ex:nAwa.pWnnWmWme}
\gll nɤ-wa pɯ-nnɯ-mɯ\redp{}me ɕti \\
 \textsc{2sg}.\textsc{poss}-father \textsc{pst}.\textsc{ipfv}-\textsc{auto}-\textsc{emph}\redp{}not.exist \\
\glt `You never had a father at all.' (Norbzang 2012, 144)
\end{exe}

Emphatic reduplication also occurs with modal verbs such as the semi-transitive \japhug{rga}{like}  (\ref{ex:kWrgWrga.tWrme}) and the transitive \japhug{spa}{be able to} (\ref{ex:kArWCmi.kWspWspa}) to express high degree.

\begin{exe}
\ex \label{ex:kWrgWrga.tWrme}
\gll  kɯ-rgɯ\redp{}rga ʑo tɯrme ra maŋe-nɯ ma  \\
\textsc{sbj}:\textsc{pcp}-\textsc{emph}\redp{}like \textsc{emph} people \textsc{pl} not.exist:\textsc{sens}-\textsc{pl} \textsc{lnk} \\
\glt `There are no people who like it a lot.' (160706 thotsi, 24)
\end{exe}

\begin{exe}
\ex \label{ex:kArWCmi.kWspWspa}
\gll pɣa kɤ-rɯɕmi kɯ-spɯ\redp{}spa nɯ kɯ ... \\
bird \textsc{inf}-speak \textsc{sbj}:\textsc{pcp}-\textsc{emph}\redp{}be.able \textsc{dem} \textsc{erg} \\
\glt `The bird who was able to speak very well (said ....).' (Norbzang 2012, 26)
\end{exe}

Often in combination with the autive \forme{nɯ-} (§\ref{sec:autoben.spontaneous}), emphatic reduplication occurs in free-choice correlatives and in universal concessive conditional constructions (§\ref{sec:universal.concessive.conditional}) as in (\ref{ex:tChi.lAwGnWjWji.Zo}), without any constraint of the verb category.\footnote{In (\ref{ex:tChi.lAwGnWjWji.Zo}), the subject of the verb \forme{pe} `(it) is good' is not the crop referred to by the interrogative pronoun \forme{tɕʰi}, but the planting activity. To analyze this example as a correlative, one would have to suppose that a complement clause such as \forme{nɯnɯ kɤ-ji} `planting that' has been elided before \forme{pe} (on \japhug{pe}{be good} as a complement-taking verb, see example \ref{ex:kAji.mApe} (§\ref{sec:adjective.complement}).  }

 \begin{exe}
\ex \label{ex:tChi.lAwGnWjWji.Zo}
\gll tɯ-ji sna tɕe, [tɕʰi lɤ́-wɣ-nɯ-jɯ\redp{}ji] ʑo pe \\
\textsc{indef}.\textsc{poss}-field be.good \textsc{lnk} what \textsc{aor}-\textsc{inv}-\textsc{auto}-\textsc{emph}\redp{}plant \textsc{emph} be.good:\textsc{fact} \\
\glt `(The type of earth called \forme{tɤtʂo}) is fit (to be used) as fields, whatever one plants in it, (the planting) will be good (successful).' (25-cWXCWz, 64)
\end{exe}


%nɤʑo tɕhi ta-tɯ-nɯ-tɯ-tɯt ʑo kɯ-nɤjtsʰɯ me ma, nɤkinɯ,
Although emphatic reduplication is highly productive and regular, a handful of verbs have irregular reduplication patterns. The transitive verb  \japhug{rɤtɕaʁ}{tread on} has the emphatic form \japhug{rɤtɕɯmtɕaʁ}{trample} (compare example \ref{ex:kWNGio.YWsaRdAt}, §\ref{sec:proprietive.generic} with \ref{ex:amAtAndonW.ma.GWznAndAGnW}, §\ref{sec:genr.3pl}) with \forme{-ɯm} reduplicated syllable (§\ref{sec.distributed.action.oR}).  The stative verb \japhug{mdi}{be complete} has the emphatic form \forme{mdoʁmdi} with \forme{-oʁ} reduplication (§\ref{sec.distributed.action.oR}), and \japhug{tsʰu}{be fat} has the suffixed \forme{-e} reduplicated form \japhug{tsʰutsʰe}{be very fat}.

The verb \japhug{fse}{be like} has the irregular reduplicated form \forme{-fsɤ\redp{}fse} alonside \forme{-fsɯ\redp{}fse} in universal concessive conditionals (\forme{tɕʰi pɯ-nɯ-fsɤ\redp{}fse}`in any case', example \ref{ex:tChi.pWnWfsAfse.Zo}, §\ref{sec:universal.concessive.conditional}).
