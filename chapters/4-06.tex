\chapter{Denominal derivations} \label{chap:denominal}


\section{Introduction}
Denominal verbalizing derivations  turn nouns into verbs. Their source can either be an inalienably or an alienably possessed noun (§\ref{sec:inalienably.possessed}), an isolated nominal root or a compound.

These derivations are referred to simply as `denominal' (without specifying `verbalizing') in this chapter and everywhere else in the grammar, since the only non-verbalizing denominal derivations, the denominal adverbs  (§\ref{sec:denominal.adverb}), are relatively marginal.

Japhug boasts a considerable number of denominal prefixes, most of which are highly productive (in particular, they can be applied to Tibetan and even Chinese loanwords). A consequence of the existence of these derivations is that conjugated verbs in Japhug are not a closed class, unlike in otherwise morphologically rich languages of the Trans-Hima\-la\-yan family such as Kiranti \citep{jacques17pkiranti},\footnote{In the case of Khaling, \citet{jacques15khaling} provides an exhaustive list of all primary verbs, which only contains two borrowings from (Thulung and Nepali) and only one clearly denominal verb.} where light verb constructions are the only productive way of forming predicates from nouns.

In the rest of the grammar, the denominal prefixes are not distinguished from the nominal root in the glosses  (the prefixes and the nominal roots are treated as a single stem). By contrast, in this chapter, the denominal prefixes are segmented and glossed as \textsc{denom}.

This chapter provides a detailed account of all attested denominal prefixes (sections §\ref{sec:denom.contracting} to §\ref{sec:denom.nW}). It also describes non-prefixal denominal derivations (§\ref{sec:denom.other}) and  deideophonic verbalizing prefixes (§\ref{sec:voice.deideophonic}).

 Some denominal  prefixes are historically related to  various valency-changing derivations, in particular the antipassive (§\ref{sec:voice.denominal}).  In addition to their basic function of deriving verbs from nouns, denominal prefixes also occur in compound verbs (§\ref{sec:denom.compound.verbs}), incorporating  verbs(§\ref{sec:incorporation}), and in verbs recently loaned from Chinese (§\ref{sec:zh.loanverbs}).

 The following subsections present a general overview of the morphological alternations observed in denominal derivations (§\ref{sec:denominal.prefixes.morph}) and on the functional correspondences between denominal derivations and light verb constructions (§\ref{sec:denominal.vs.light.verb}), two topics which are treated in more detail in each of the sections in this chapter.
 
\subsection{Morphological properties of denominal prefixes} \label{sec:denominal.prefixes.morph}
When a noun takes a denominal prefix, the root almost never exhibits any stem alternation. If the base noun is  inalienably possessed, the possessive prefix is removed (including in many cases frozen indefinite possessor prefixes in \forme{tɯ-/tɤ\trt}, §\ref{sec:frozen.indef}). For instance, the \forme{ɣɤ-} denominal verb (§\ref{sec:denom.GW}) from the inalienably possessed noun \japhug{tɤ-kʰɯ}{smoke} is \japhug{ɣɤkʰɯ}{be smoky}  (rather than $\dagger$\forme{ɣɤtɤkʰɯ}), and the same applies even to the noun \japhug{tɤrʁaʁ}{game} whose \forme{tɤ-} prefixal element is not an indefinite possessor prefix, but whose denominal form is  \japhug{ɣɤrʁaʁ}{hunt}. Likewise, when the base noun is a counted noun (§\ref{sec:counted.nouns}), the numeral prefix is also removed (§\ref{sec:denom.tr.rA}, §\ref{sec:denom.tr.GA}). Only action nominalization \forme{tɯ-} prefixes can be preserved: for example, the lexicalized action nominal \japhug{tɯsqa}{wheat gruel} (from \japhug{sqa}{cook}, §\ref{sec:denominalization.action.nominal}) has the denominal form \japhug{rɯtɯsqa}{eat wheat gruel} (§\ref{sec:denom.intr.rA}) rather than $\dagger$\forme{rɯsqa}.

Some denominal prefixes, including \forme{sɯ-/sɤ-} (§\ref{sec:denom.sW.caus.instr}),  \forme{rɯ-/rɤ-} (§\ref{sec:denom.rA}), \forme{ɣɯ-/ɣɤ-} (§\ref{sec:denom.tr.GA}), \forme{mɯ-/mɤ-} (§\ref{sec:denom.mA}) and \forme{nɯ-/nɤ-} (§\ref{sec:denom.nW}) have either \forme{ɯ} or \forme{ɤ}  vocalism (with the variant \forme{a} when the following syllables has a uvular preinitial, §\ref{sec:A.vs.a.prefixes}). This vowel contrast is partially based on that of the indefinite possessor prefix (\forme{tɯ-} vs. \forme{tɤ\trt}, §\ref{sec:inalienably.possessed.morpho}) when the base noun is  inalienably possessed, but many exceptions exist. 

A handful of denominal verbs have in addition an intrusive \forme{-ɣ-/-x-} velar fricative element (like the sigmatic causative, §\ref{sec:caus.sWG}), for instance \japhug{ɣɤxpra}{send} from  \japhug{tɤpra}{messenger} (§\ref{sec:denom.tr.GA}).
 
Denominal prefixes are nearly always located closest to the root than any non-denominal derivational prefix. Only one exception is known: the causative \japhug{sɤzmbrɯ}{anger} from the denominal verb \japhug{sɤmbrɯ}{get angry}, whose causative prefix is irregularly inserted between the denominal prefix \forme{sɤ-} and the nominal root \forme{-mbrɯ}: \forme{sɤ-z-mbrɯ} is to be parsed as  \textsc{denom}-\textsc{caus}-anger  (§\ref{sec:sig.caus.irregular.other}).

\subsection{Denominal derivations and light verb constructions} \label{sec:denominal.vs.light.verb}
Most denominal derivations described in this chapter are synonymous or near-synonymous with complex predicates involving light verbs such as \japhug{βzu}{make}, \japhug{lɤt}{release} or existential verbs like \japhug{tu}{exist}. Both types of construction serve to integrate a nominal root into a predicate. 
 
For instance, the verb \japhug{rɯqajɯ}{get worms} and  \japhug{rɤspɯ}{fester}, with the \forme{rɯ-/rɤ-} denominal prefix (§\ref{sec:denom.intr.rA}) have the same meaning as the collocation of their base noun \japhug{qajɯ}{worm} and \japhug{tɤ-spɯ}{pus} with the light verb \japhug{βzu}{make}, as illustrated by (\ref{ex:tAspW.toBzu}) where both constructions appear. This example also shows that synonymous denominal and light verb constructions from the same nouns lexically select the same orientation preverbs: \textsc{upwards} in the case of \forme{rɤspɯ}/\forme{tɤ-spɯ+βzu} `fester, have pus', and \textsc{westwards} in that of \forme{rɯqajɯ}/\forme{qajɯ+βzu} `have worms, grow worms'.
 
 \begin{exe}
\ex \label{ex:tAspW.toBzu}
 \gll tɕe [tɤ-spɯ tɕe to-βzu] tɕe [qajɯ ɲo-βzu] ma kɤ-nɯqambɯmbjom ɯ-xɕɤt kɯ nɯnɯ ɯ-ʁar ɯ-ndzom ɣɯ ɯ-qa nɯra to-rɤ-spɯ qʰe ɲɤ-rɯ-qajɯ \\
 \textsc{lnk} \textsc{indef}.\textsc{poss}-pus \textsc{lnk} \textsc{ifr}:\textsc{up}-make \textsc{lnk} worm \textsc{ifr}:\textsc{west}-make \textsc{lnk} \textsc{inf}-fly \textsc{3sg}.\textsc{poss}-strength \textsc{erg} \textsc{dem} \textsc{3sg}.\textsc{poss}-wing \textsc{3sg}.\textsc{poss}-bridge \textsc{gen} \textsc{3sg}.\textsc{poss}-bottom \textsc{dem}:\textsc{pl} \textsc{ifr}:\textsc{up}-\textsc{denom}-pus \textsc{lnk} \textsc{ifr}:\textsc{west}-\textsc{denom}-worm \\
\glt `[Its wing] had pus and worms grew in it, it flew so much that the base of its wing festered and had worms.' (22-qomndroN)
\japhdoi{0003598\#S47}
  \end{exe}
  
In each of the sections of this chapter on denominal prefixes, the corresponding light verb constructions are systematically indicated.

Although the two constructions are semantically very close, an important difference is that in the denominal construction, the noun cannot take external determiners like numeral and demonstratives (unlike denominal derivation in some Uto-Aztecan languages like Hopi for instance, see \citealt{hill.kc03hopi}). For instance, in (\ref{ex:WrZaB.chArArJit}), the verb \forme{cʰɤ-rɤ-rɟit} (from \japhug{tɤ-rɟit}{child}, §\ref{sec:denom.intr.rA}) means  `gave birth to (a) child/children', without any indication on the number of children (the unicity of the offspring in this example is deduced from the context of the story).

\begin{exe}
\ex \label{ex:WrZaB.chArArJit}
\gll [ɯ-rʑaβ nɯ ɯ-skʰrɯ mɯ-nɯ-kɯ-βdi] nɯ, cʰɤ-rɤ-rɟit \\
 \textsc{3sg}.\textsc{poss}-wide \textsc{dem} \textsc{3sg}.\textsc{poss}-body \textsc{neg}-\textsc{aor}-\textsc{sbj}:\textsc{pcp}-be.well \textsc{dem} \textsc{ifr}-\textsc{denom}-child \\
\glt `His wife, who was pregnant, gave birth to a child.' (2012 Norbzang)
\japhdoi{0003768\#S95}
\end{exe}

In order to specify the number of children that are born, a collocation with the verb \japhug{sci}{be born}, its causative \japhug{sɯsci}{give birth to} or the corresponding possessive construction with the verb \japhug{tu}{exist} are needed instead, as in (\ref{ex:kWngWt.tABzu}). The same meaning cannot be expressed by combining the verb \forme{cʰɤ-rɤ-rɟit} with a numeral.

\begin{exe}
\ex \label{ex:kWngWt.tABzu}
\gll  ɯ-rɟit kɯngɯt tɤ-tu. \\
 \textsc{3sg}.\textsc{poss}-child nine \textsc{aor}-exist \\
\glt `She had nine children.' (14-siblings) \japhdoi{0003508\#S15}
\end{exe}

The only cases when a denominal derivation can preserve the modifier of a noun is when the modifier is fused with the noun root and both undergo the derivation together. For instance, the nominal phrase \japhug{tɕʰitɕɯn paχɕi}{pear}, which comprises the noun \japhug{paχɕi}{apple} with the placename  \japhug{tɕʰitɕɯn}{Chuchen} as prenominal modifier (§\ref{ex:attributive.prenominal}, §\ref{sec:determinative.n.n}) can be verbalized as \japhug{nɯtɕʰitɕɯnpaχɕi}{collect pears} with the intransitive \forme{nɯ-} derivation (§\ref{sec:denom.tr.nW}).
 
  
 
\section{Contracting prefixes} \label{sec:denom.contracting}
Contracting denominal prefixes are mono- or disyllabic prefixes which first syllable is \forme{a\trt}, and undergoes vowel contraction in the contexts discussed in §\ref{sec:contraction} like the passive \forme{a-} (§\ref{sec:passive}) and the reciprocal derivations (§\ref{sec:reciprocal}).
 
\subsection{Stative \forme{a-}} \label{sec:denom.a}
The denominal prefix \forme{a-}  derives intransitive verbs. Most \forme{a-} denominal verbs are stative, and either mean `be $X$' or `have the property of $X$'. The stative denominal function of \forme{a-} is highly productive, and can be applied to Tibetan loanwords. 

Denominal verbs in \forme{a-} can express shape, as in \japhug{artaʁ}{be forked} from \japhug{tɤ-rtaʁ}{branch}. They can also refer to a more abstract property of the base noun, as in the case of \japhug{aci}{be wet}, which derives from \japhug{tɯ-ci}{water} (on which see §\ref{sec:earth.IPN}).
 
Denominal verbs expressing physical defects are most often built with the \forme{a-} prefix, for instance \japhug{aɕkala}{be lame}, \japhug{aʑɤwu}{be lame} and \japhug{aɕquwa}{be blind} from \japhug{ɕkala}{lame person}, \japhug{ʑɤwu}{lame person} (from \tibet{ཞ་བ་}{ʑa.ba}{cripple}) and \japhug{ɕquwa}{blind person}, respectively (an exception is \japhug{ɣɤmbɣo}{de deaf} with the \forme{ɣɤ-} prefix, §\ref{sec:denom.GW}).

Some denominal verbs in \forme{a-} express a property shared by several entities, for instance \japhug{aɕɣa}{be the same age} (§\ref{sec:intrinsically.n.sg.subject}) from \japhug{tɯ-ɕɣa}{tooth}). \footnote{The semantic extension `age' from `tooth' is also found in other Gyalrongic languages, see for instance \citet[524]{lai17khroskyabs}.} Such verbs can have a reduplicated stem, or suffixal syllables, like the verbs \forme{a-fsɯ\redp{}fsu}  and \forme{a-fsu-ja} both meaning `be of the same size'\footnote{On the causative form of \japhug{afsuja}{be of the same size}, see §\ref{sec:sig.caus.tropative} (in particular example \ref{ex:tuWGsAfsuja}). } from the inalienably possessed noun \japhug{ɯ-fsu}{of the same size} (§\ref{sec:fsu.fse}, §\ref{sec:Wfsu.equative}).

Not all \forme{a-} denominal verbs are stative. Dynamic intransitive verbs in \forme{a-} include for instance \japhug{arju}{speak} (§\ref{sec:preverb.speech}) from \japhug{tɯ-rju}{word}, `utterance', and the verbs \japhug{amqaj}{fight} (scold each other) and \japhug{aɣro}{play} (with non-singular subjects), from  \japhug{tɯ-mqaj}{scolding} and \japhug{tɤɣro}{game}\footnote{The noun \japhug{tɤɣro}{game} is however also possibly interpretable as a deverbal noun, see §\ref{sec:action.nominals}. }, which are almost always attested with infixation of the autive \forme{nɯ-} as \forme{a<nɯ>mqaj} and \forme{a<nɯ>ɣro}.

%tɤre sɤtɕɯtɕɤt
%tɕe kɤ-nɤre pjɤ-tɕɤt ri ɯ-mqaj pjɤ-tu

Compound nouns of dimension (§\ref{sec.v.v.compounds.degree}) can derive compound verbs (§\ref{sec:denom.compound.verbs}) meaning `of uneven/unequal $X$' or `spread along the dimension $X$' with the \forme{a-} prefix, as shown in \tabref{tab:denom.dimension}. Among them, \japhug{ajpomxtsʰɯm}{have uneven thickness} presents an unexplained alternation between \ipa{u} and \ipa{o} alternation: the denominal verb has \forme{o} in the first syllable, while the compound noun \japhug{jpumxtsʰɯm}{thickness} and the base verb \japhug{jpum}{be thick} have \forme{u}. 

\begin{table}
\caption{Denominal verbs from nouns of dimension} \label{tab:denom.dimension}
\begin{tabular}{llll}
 \lsptoprule 
Base compound noun & Denominal verb\\
 \midrule
\japhug{jpumxtsʰɯm}{thickness} (diameter) &\japhug{ajpomxtsʰɯm}{be of uneven thickness} \\
\japhug{jaʁmba}{thickness} (of a sheet)&\japhug{ajaʁmba}{be of uneven thickness} \\
\japhug{xtɕɯxte}{size} &\japhug{axtɕɯxte}{be of uneven size}  \\
\tablevspace
\japhug{taʁki}{up and down} &\forme{ataʁki} `be aligned on the  \\
&vertical axis \\
\japhug{lotʰi}{upstream and downstream} &\forme{alotʰi} `be aligned on the  \\
&riverine axis'\\
\japhug{kundi}{east and west} &\forme{akundi} `be aligned on the  \\
&east-west axis' \\
 \lspbottomrule
\end{tabular}
\end{table}

The causative form of these verbs, regularly built with vowel fusion  \forme{sɯ-ɤ-} (§\ref{sec:caus.sA}), means `align/put along the $X$ dimension' as in (\ref{ex:lusAlothi}) (where the upstream-downstream dimension refers to the head-tail dimension of the body of the butterfly).

%tɕe nɯnɯ ɯ-ʁar kɯβde ɣɤʑu ma,
%ɯ-phaʁ ntsi ri kɯ-wxti ci, kɯ-xtɕi ci tɕe
%kɯ-wxti tsa kɯ-fse nɯ ɯ-ku ɯ-pɕoʁ nɯre ku-ndzoʁ ɲɯ-ŋu, 
%kɯ-xtɕi tsa nɯnɯ ɯ-jme ɯ-pɕoʁ ri ku-ndzoʁ ɲɯ-ŋu tɕe,
\begin{exe}
\ex \label{ex:lusAlothi}
\gll tɕe nɯnɯ ɯ-ʁar kɯβde ɣɤʑu ma, ɯ-pʰaʁ ɯ-ntsi ri kɯ-wxti ci, kɯ-xtɕi ci tɕe [...] nɯ-nɯqambɯmbjom ri tɕe nɯnɯ lu-sɯ-ɤ-lo-tʰi ɲɯ-ŋu. \\
\textsc{lnk} \textsc{dem} \textsc{3sg}.\textsc{poss}-wing four exist:\textsc{sens} \textsc{lnk} \textsc{3sg}.\textsc{poss}-half \textsc{3sg}.\textsc{poss}-one.of.a.pair \textsc{loc} \textsc{sbj}:\textsc{pcp}-be.big \textsc{indef} \textsc{sbj}:\textsc{pcp}-be.small \textsc{indef} \textsc{lnk} { }  \textsc{aor}-fly \textsc{lnk} \textsc{lnk} \textsc{dem} \textsc{ipfv}:\textsc{upstream}-\textsc{caus}-\textsc{denom}-up-downstream \textsc{sens}-be \\
\glt `[The butterfly] has four wings, on each side a big one and a small one. (...) When it flies, it aligns [its wings] along the `upstream-downstream' axis.' (26-qambalWla)
\japhdoi{0003680\#S45}
\end{exe}

The verbs \forme{ataʁki}, \forme{alotʰi} and \forme{akundi} and their causative forms  select the \textsc{upwards}, \textsc{upstream} (as in \ref{ex:lusAlothi}) and \textsc{eastwards} preverbs, respectively.

%relationship with passive {sec:passive}

Noun-verb pairs where the verb has \forme{a-} and the noun is an inalienably possessed noun selecting the \forme{tɤ-} indefinite possessor prefix (for instance the verb \japhug{aɕqʰe}{cough} and the inalienably possessed noun \japhug{tɤ-ɕqʰe}{cough} ) can be interpreted as \forme{a-} denominal derivation, but the opposite directionality (the inalienably possessed noun as a nominalized form of the \forme{a-} prefixed verb) is also possible (see §\ref{sec:bare.action.nominals} for a more detailed discussion). Another problematic example of \forme{a-} denominal derivation is \japhug{acʰɤt}{be $X$ years apart} (§\ref{sec:a.non.passive.denominal}).

In some cases, semantic changes either in the verb or the noun have obscured the etymological relationship between them. For instance, it is possible that \japhug{aro}{own} is historically related to \japhug{tɤ-ro}{surplus, leftover} (§\ref{sec:semi.transitive}), though there is no synchronic link between these two words.
 
A certain number of intransitive verbs in \forme{a-} can be suspected of being denominal verbs whose base noun has been lost. This is particularly clear in the case of some verbs expressing shape such as \japhug{aβzɯrχsɯm}{have a triangular shape} from a noun \forme{*βzɯrχsɯm} unattested in Japhug, but attested in Tibetan as \tibet{ཟུར་གསུམ་}{zur.gsum}{triangle}. Even when the nominal source is not recoverable, as in the case of \japhug{artɯm}{be round}, the hypothesis that the \forme{a-} verb comes from a lost noun (in this case, something like \forme{*rtɯm} `round object, circle') can never be ruled out. 

The deideophonic \forme{a-} derivation (§\ref{sec:a.nA.deidph}) is also likely to have a historical relationship with the stative \forme{a-} denominal prefix.

\subsection{Similative \forme{arɯ-} } \label{sec:denom.arW}
The prefix \forme{arɯ-} derives  stative verbs meaning `be $X$-like, be similar to $X$' from a nominal base. \tabref{tab:arW.denom} presents a few representative examples of this prefix. Although these verbs (with the exception of \japhug{arɯldʑaŋkɯ}{be green}) are relatively rare in the corpus, this derivation is extremely productive and can be applied to Tibetan loanwords, for instance \japhug{ɕoʁɕoʁ}{paper}, \japhug{ldʑaŋkɯ}{green} and \japhug{mɯntoʁ}{flower} (from \tibet{ཤོག་ཤོག་}{ɕog.ɕog}{paper}, \tibet{ལྗང་གུ་}{ldʑaŋ.gu}{green} and \tibet{མེ་ཏོག་}{me.tog}{flower}, respectively. It is even found on Chinese loanwords such as \ch{喇叭}{lǎba}{horn}.

Inalienably possessed nouns such as \japhug{tɤ-tɕɯ}{son}, `boy' are denominalized together with their indefinite possessor prefix (§\ref{sec:kinship}).

The \forme{arɯ-} derivation can take not only noun roots, but also noun phrases as input. For instance, the phrase \forme{qro-mke ɯ-mdoʁ} (pigeon-neck \textsc{3sg}.\textsc{poss}-colour) `purple, colour of the pigeon's neck' can be denominalized as \japhug{arɯqromkemdoʁ}{be purple}.
 
\begin{table}
\caption{Examples of the \forme{arɯ-} similative denominal prefix} \label{tab:arW.denom}
\begin{tabular}{lllll}
\lsptoprule
Base noun & Denominal verb \\
\midrule
\japhug{ɕoʁɕoʁ}{paper}	&	\japhug{arɯɕoʁɕoʁ}{be like paper}		\\
\japhug{fsapaʁ}{animal}	&	\japhug{arɯfsapaʁ}{be like an animal}		\\
\japhug{kʰɯtsa}{bowl}	&	\japhug{arɯkʰɯtsa}{be like a bowl}		\\
\japhug{taqaβ}{needle}	&	\japhug{arɯtaqaβ}{be like a needle}		\\
\japhug{ldʑaŋkɯ}{green}	&	\japhug{arɯldʑaŋkɯ}{be green}		\\
<laba> `horn'	&	\japhug{arɯlaba}{be shaped like a horn}		\\
\japhug{sɯjno}{grass}	&	\japhug{arɯsɯjno}{be like grass}		\\
\japhug{tɤ-tɕɯ}{son}, `boy'	&	\japhug{arɯtɤtɕɯ}{be boyish}		\\
\japhug{tɤjpa}{snow}	&	\japhug{arɯtɤjpa}{be like snow}		\\
\japhug{mɯntoʁ}{flower}	&	\japhug{arɯmɯntoʁ}{be like a flower}		\\
\lspbottomrule
\end{tabular}
\end{table}


With colour nouns, including those borrowed from Tibetan like \japhug{ldʑaŋkɯ}{green} and \japhug{ʁmɤrsmɯɣ}{dark red} (§\ref{sec:tibetan.colours}) and others like \forme{qro-mke ɯ-mdoʁ} `purple', the denominal verbs in \forme{arɯ-}  simply mean `be $X$, have the colour $X$', as in (\ref{ex:arWldZaNkW}).

\begin{exe}
\ex \label{ex:arWldZaNkW}
\gll tɕe kɯmaʁ xɕaj nɯra arɯ-ldʑaŋkɯ kɯ-fse ma  \\
\textsc{lnk} other grass \textsc{dem}:\textsc{pl} \textsc{denom}-green:\textsc{fact} \textsc{sbj}:\textsc{pcp}-be.like \textsc{lnk} \\
\glt `The other plants [around it] are [very] green.' (22-BlamajmAG) \japhdoi{0003584\#S103}
\end{exe}
 
Denominal verbs in \forme{arɯ-} are often used in similative constructions with degree nominals, either as nominal predicates as in (\ref{ex:WtArWmWntoR}) (§\ref{sec:degree.nominal.predicates}) or in the degree construction (\ref{ex:YArWtAjpa}) (§\ref{sec:degree.nominal.construction}). Note that the connotation of these similative verbs can be culture-specific: although \japhug{arɯmɯntoʁ}{be like a flower}	can be understood as `be as beautiful as a flower', its most natural interpretation is surprisingly `worthless' (because flowers are viewed as not lasting long, and prone to withering).

\begin{exe}
\ex \label{ex:WtArWmWntoR}
\gll ɯ-tɯ-ɤrɯ-mɯntoʁ nɯ! \\
\textsc{3sg}.\textsc{poss}-\textsc{nmlz}:\textsc{deg}-\textsc{denom}-flower \textsc{sfp} \\
\glt `It is as [worthless] as a flower.' (elicited)
\end{exe}

 \begin{exe}
\ex \label{ex:YArWtAjpa}
\gll ɯ-tɯ-wɣrum kɯ ɲɯ-ɤrɯ-tɤjpa ʑo \\
\textsc{3sg}.\textsc{poss}-\textsc{nmlz}:\textsc{deg}-be.white \textsc{erg} \textsc{sens}-\textsc{denom}-snow \textsc{emph} \\
\glt `It is white as snow.' (elicited)
\end{exe}

They also occur in participial form as \forme{kɯ-ɤrɯ-taqaβ} `that is like a needle' in (\ref{ex:kArWtaqaB}), often with the similative stative verb \japhug{fse}{be like}.

\begin{exe}
\ex \label{ex:kArWtaqaB}
\gll ɯ-jwaʁ ɯ-tsʰɯɣa nɯnɯ kɯ-ɤrɯ-taqaβ kɯ-fse naχtɕɯɣ ri, \\
\textsc{3sg}.\textsc{poss}-leaf \textsc{3sg}.\textsc{poss}-shape \textsc{dem} \textsc{sbj}:\textsc{pcp}-\textsc{denom}-needle \textsc{sbj}:\textsc{pcp}-be.like be.identical:\textsc{fact} \textsc{lnk} \\
\glt `The shape of its leaves is similar [to those of the fir] in that they are like needles.' (11-mYAm) \japhdoi{0003474\#S65}
\end{exe}

Additional examples of similative constructions involving these verbs can be found in  \citet{jacques18similative} and §\ref{sec:denominal:similative}.

\subsection{Proprietive \forme{arɤ-} }  \label{sec:denom.arA}
The \forme{arɤ-} denominal prefix has a proprietive meaning, and is generally correlated with partial reduplication of the nominal stem as in \japhug{arɤrqʰɯrqʰioʁ}{be grooved} (\ref{ex:kArArqhWrqhioR}) from \japhug{tɤ-rqʰioʁ}{groove}.


\begin{exe}
\ex \label{ex:kArArqhWrqhioR}
\gll ɯ-ru nɯra tɤ-kɯ-ɤrɤ-rqʰɯ\redp{}rqʰioʁ kɯ-fse ci ŋu qʰe, \\
\textsc{3sg}.\textsc{poss}-stalk \textsc{dem}:\textsc{pl} \textsc{aor}-\textsc{sbj}:\textsc{pcp}-\textsc{denom}-groove \textsc{sbj}:\textsc{pcp}-be.like \textsc{indef} be:\textsc{fact} \textsc{lnk} \\
\glt `Its stalk is like it has had many groovings on it.' (14-sWNgWJu) \japhdoi{0003506\#S41}
\end{exe}

The \forme{arɤ-} prefix can also be used to derive verbs from counted nouns as for instance \japhug{arɤɕɯɕrɤz}{be striped} from \japhug{tɯ-ɕrɤz}{one stripe}. The verb \japhug{arɤkʰɯmkʰɤl}{be clustered in patches}, `not homogeneously distributed' from \japhug{tɯ-kʰɤl}{one place} has the very rare partial reduplication pattern in \forme{-ɯm} (\forme{arɤ-kʰɯm\redp{}kʰɤl}, §\ref{sec.distributed.action.oR}).

Non-reduplicated \forme{arɤ-} denominal verbs are also found, for instance   \japhug{arɤtsʰi}{be cooked like rice gruel} (from \japhug{tɯtsʰi}{rice gruel}, a lexicalized nominalization from \japhug{tsʰi}{drink}) and \japhug{arɤrɤɣ}{happen at the predicted time} from \japhug{ɯ-rɤɣ}{predicted time} (on which see §\ref{sec:relator.temporal}).

\begin{exe}
\ex \label{ex:YAkArArAGCi}
\gll aʑo a-ʑɯβ ɲɤ-k-ɤrɤ-rɤɣ-ci \\
 \textsc{1sg} \textsc{1sg}.\textsc{poss}-sleep \textsc{ifr}-\textsc{peg}-\textsc{denom}-predicted.time-\textsc{peg} \\
\glt `I feel sleepy at the expected time.' (elicited, can be said when one feels sleepy at the same hour in the afternoon when one went to sleep the previous days)
\end{exe}

Like other compound denominal prefixes (§\ref{sec:denom.aGW.caus}), the \forme{arɤ-} prefix loses its \forme{a-} element when subjected to causative derivation. For instance, \japhug{arɤtsʰi}{be cooked like rice gruel} has the causative form \japhug{zrɤtsʰi}{cook like rice gruel} (§\ref{sec:sigm.caus.a.z}).

 %arɤmgɯmgo
 %arɤmtʂɯmtʂaj, 
 %arɤphɤjqa
 %mboʁɲɟi 

The \forme{arɤ-} prefix should not be confused with the reciprocal from transitive verbs in \forme{rɤ-}. For instance \japhug{arɤzdɯzda}{call each other} (before departure) does not directly come from the noun \japhug{tɯ-zda}{companion}, but is rather the reduplicated reciprocal (§\ref{sec:redp.reciprocal}) of the denominal verb \japhug{rɤzda}{call before departure} derived from this noun.
 
\subsection{Proprietive \forme{aɣɯ-}} \label{sec:denom.aGW}  
The prefix \forme{aɣɯ-} derives proprietive stative verbs meaning `have a lot of $X$' or `produce a lot of $X$'. For instance, Tshendzin provided the definition in (\ref{ex:YAGWmdzu.def}) for the proprietive denominal \forme{aɣɯ-mdzu} from \japhug{tɤ-mdzu}{thorn}.

\begin{exe}
\ex \label{ex:YAGWmdzu.def}
\gll ɲɯ-ɤɣɯ-mdzu, ɯ-taʁ tɤ-mdzu ɲɯ-dɤn kɤ-ti ɲɯ-ŋu. \\
\textsc{sens}-\textsc{denom}:\textsc{prop}-thorn \textsc{3sg}.\textsc{poss}-on \textsc{indef}.\textsc{poss}-thorn \textsc{sens}-be.many \textsc{obj}:\textsc{pcp}-say \textsc{sens}-be  \\
\glt `[The word] \forme{ɲɯ-ɤɣɯ-mdzu} means `there are a lot of thorns on it'.' (elicited)
\end{exe}

When the base noun is inalienably possessed (§\ref{sec:inalienably.possessed}) or when it is a counted noun (§\ref{sec:counted.nouns}), the denominal prefix is directly attached to the root, and the indefinite possessor prefixes \forme{tɯ\trt}, \forme{tɤ\trt}, \forme{ta-} or the numeral prefixes are removed, as shown in the examples in \tabref{tab:aGW.denom}.  The \forme{aɣɯ-} prefix is invariable, without an allomorph such as $\dagger$\forme{aɣɤ-} as could have been expected (§\ref{sec:indef.genr.poss}) when the base noun selects the indefinite possessor prefix  \forme{tɤ-}.

This prefix is very productive, and occurs on nominal bases borrowed from Tibetan (such as \japhug{rŋɯl}{silver}, \japhug{ɯ-mdoʁ}{colour}, \japhug{tɯɣ}{poison} and \japhug{smɤn}{medicine} (respectively from \tibet{དངུལ་}{dŋul}{silver}, \tibet{མདོག་}{mdog}{colour}, \tibet{དུག་}{dug}{poison} and  \tibet{སྨན་}{sman}{medicine}) and even Chinese (\japhug{tɯ-ʂwaŋ}{a pair}	from \ch{双}{shuāng}{pair}).

\begin{table}
\caption{Examples of the \forme{aɣɯ-} proprietive denominal prefix} \label{tab:aGW.denom}
\begin{tabular}{lllll}
\lsptoprule
Base noun & Denominal verb \\
\midrule
\japhug{tɯ-ɕa}{flesh}	&	\japhug{aɣɯɕa}{have a lot of meat}		\\
\japhug{tɤ-jwaʁ}{leaf}	&	\japhug{aɣɯjwaʁ}{have a lot of leaves}		\\
\japhug{tɤ-mdzu}{thorn}	&	\japhug{aɣɯmdzu}{have a lot of thorns}		\\
\japhug{tɯ-mɲaʁ}{eye}	&	\japhug{aɣɯmɲaʁ}{have eyes}, `have a lot of holes'		\\
\japhug{tɤ-ŋgɤr}{fat}	&	\japhug{aɣɯŋgɤr}{have a lot of fat}		\\
\japhug{tɤ-rɟit}{child}	&	\japhug{aɣɯrɟit}{have many children}		\\
\japhug{tɤ-rme}{hair}	&	\japhug{aɣɯrme}{have a lot of hair}		\\
\japhug{rŋɯl}{silver}	&	\japhug{aɣɯrŋɯl}{have much money}		\\
\japhug{zrɯɣ}{louse}	&	\japhug{aɣɯzrɯɣ}{have a lot of lice}		\\
\japhug{tɤ-ntɤβ}{bubble}	&	\japhug{aɣɯntɤβ}{be sparkling}		\\
\japhug{tɤ-ndʑɯɣ}{resin}	&	\japhug{aɣɯndʑɯɣ}{be resinous}		\\
\tablevspace
\japhug{tɯ-ɣli}{dung}	&	\japhug{aɣɯɣli}{produce a lot of dung}		\\
\japhug{ta-mar}{butter}	&	\japhug{aɣɯmar}{produce a lot of butter}		\\
\japhug{ɯ-mat}{fruit}	&	\japhug{aɣɯmat}{produce a lot of fruits}		\\
\tablevspace
\japhug{ɯ-dɯχɯn}{fragrance}	&	\japhug{aɣɯdɯχɯn}{be fragrant}		\\
\japhug{tɯɣ}{poison}	&	\japhug{aɣɯtɯɣ}{be poisonous}		\\
\japhug{smɤn}{medicine}	&	\japhug{aɣɯsmɤn}{have a medical effect}		\\
\tablevspace
\japhug{ɯ-mdoʁ}{colour}	&	\japhug{aɣɯmdoʁ}{be the same colour}		\\
\tablevspace
\japhug{ɯ-ŋgɯ}{inside}	&	\japhug{aɣɯŋgɯŋgɯ}{have a lot of layers}		\\
\japhug{tɯ-jaʁ}{hand, arm}	&	\japhug{aɣɯjɯjaʁ}{have a lot of arms}		\\
\japhug{tɯ-mi}{foot, leg}	&	\japhug{aɣɯmɯmi}{have a lot of legs}		\\
\tablevspace
\japhug{tɯ-rpaʁ}{shoulder}	&	\japhug{aɣɯrpaʁ}{go along well}		\\
\japhug{tɯ-ʂwaŋ}{a pair}	 	&	\japhug{aɣɯʂwaŋ}{be a match}		\\
\lspbottomrule
\end{tabular}
\end{table}
 
Another possible meaning of the \forme{aɣɯ-} prefix is `be the same $X$', as in the case of \japhug{aɣɯmdoʁ}{be the same colour} from \japhug{ɯ-mdoʁ}{colour}. These types of verbs have non-singular intransitive subjects, as in (\ref{ex:aGWmdoR.arNi}) with a comitative phrase (§\ref{sec:comitative}).\footnote{Concerning the absence of dual indexation on the verbs in (\ref{ex:aGWmdoR.arNi}), see §\ref{sec:optional.indexation}. }

\begin{exe}
\ex \label{ex:aGWmdoR.arNi}
\gll ɯ-ru cʰo ɯ-fkaβ ra aɣɯmdoʁ ʑo arŋi. \\
\textsc{3sg}.\textsc{poss}-stalk \textsc{comit} \textsc{3sg}.\textsc{poss}-cover \textsc{pl} \textsc{denom}-colour:\textsc{fact} \textsc{emph} be.green:\textsc{fact} \\
\glt `Its stalk and its cap are the same green colour.' (of a species of mushroom) (22-BlamajmAG)
\japhdoi{0003584\#S114}
 \end{exe}

A few \forme{aɣɯ-} denominal verbs, such as \japhug{aɣɯtɯɣ}{be poisonous}	from \japhug{tɯɣ}{poison}, have a more abstract proprietive function (`have the property of $X$') like that of the \forme{sɤ-} prefix (note the synonym \japhug{sɤndɤɣ}{be poisonous}, §\ref{sec:denom.sA.proprietive}). Other semantic values of the \forme{aɣɯ-} prefix are  discussed in §\ref{sec:denom.aGW.lexicalized}.

The \forme{aɣɯ-} denominal prefix has cognates in other Gyalrong languages, for example \forme{ɐwə-} in Tshobdun \citep{jackson14morpho}.

\subsubsection{Reduplication} \label{sec:denom.aGW.redp}
Emphatic reduplication (§\ref{sec:emph.redp}) is  possible as an option on denominal verbs, but some verbs in \forme{aɣɯ-} occur with obligatory reduplication of the nominal root, for instance \japhug{aɣɯjɯjaʁ}{have a lot of arms} and \japhug{aɣɯmɯmi}{have a lot of legs} (\ref{ex:YAGWjWJaR}). Note that the alternative form \forme{aɣɯjaʁ} without reduplication is also attested, but with a completely different meaning (§\ref{sec:denom.aGW.lexicalized}).

\begin{exe}
\ex \label{ex:YAGWjWJaR}
\gll ɴɢoɕna ɲɯ-ɤɣɯ-jɯ\redp{}jaʁ ɲɯ-ɤɣɯ-mɯ\redp{}mi, ɯ-mɤlɤjaʁ ɲɯ-dɤn \\
spider \textsc{sens}-\textsc{denom}-\textsc{emph}\redp{}arm  \textsc{sens}-\textsc{denom}-\textsc{emph}\redp{}leg \textsc{3sg}.\textsc{poss}-limbs \textsc{sens}-be.many \\
\glt  `The spider has many arms and legs, its limbs are numerous.' (elicited)
 \end{exe}

\subsubsection{Causative}  \label{sec:denom.aGW.caus}
Only a few causative verbs derived from \forme{aɣɯ-} denominal verbs are attested. Instead of expected vowel fusion ($\dagger$\forme{sɯ-ɤɣɯ-} \fl{} \forme{sɤɣɯ\trt}, see §\ref{sec:caus.sA}), the \forme{a-} element is dropped and replaced by the \forme{z-} allomorph of the sigmatic causative (§\ref{sec:sigm.caus.a.z}). For instance, the stative verb \japhug{aɣɯŋgɯŋgɯ}{have a lot of layers} (from \japhug{ɯ-ŋgɯ}{inside}, with reduplication, §\ref{sec:denom.aGW.redp}) has the causative form  \japhug{zɣɯŋgɯŋgɯ}{put on a lot of layers} as in (\ref{ex:pjAzGWNgWNgW}).

\begin{exe}
\ex \label{ex:pjAzGWNgWNgW}
\gll ɯ-rʑaβ nɯ kɯ, [...] <huangdi> ɣɯ ɯ-rte nɯ χsɯm ʑo pjɤ-z-ɣɯ-ŋgɯ\redp{}ŋgɯ pjɤ-nɤrte  \\
\textsc{3sg}.\textsc{poss}-wife \textsc{dem} \textsc{erg} { } emperor \textsc{gen} \textsc{3sg}.\textsc{poss}-hat \textsc{dem} three \textsc{emph} \textsc{ifr}-\textsc{caus}-\textsc{denom}-\textsc{emph}\redp{}inside \textsc{ifr}-wear \\
\glt `His wife wore three imperial crowns one on the top of the other.' (140430 yufu he tade qizi-zh)
\japhdoi{0003900\#S217}
 \end{exe}
 
 \subsubsection{Lexicalized denominal verbs}  \label{sec:denom.aGW.lexicalized}
The semantics of the \forme{aɣɯ-} derivation is not always trivially predictable.

Some of the lexicalized meanings are relatable to one of the basic functions of this prefix. For instance, \japhug{aɣɯʂwaŋ}{be a match} (to each other) (\ref{ex:mWpjAkAGWsxWaNci}) from \japhug{tɯ-ʂwaŋ}{a pair} presumably derives from an earlier `have the property of being a pair'.

\begin{exe}
\ex \label{ex:mWpjAkAGWsxWaNci}
\gll tɕe [ɯ-ʁɤri ta-tɯt] cʰo [ɯ-qʰu ta-tɯt] nɯra maka mɯ-pjɤ-k-ɤɣɯ-ʂwaŋ-ci \\
\textsc{lnk} \textsc{3sg}.\textsc{poss}-before \textsc{aor}:3\fl{}3-say[II] \textsc{comit} \textsc{3sg}.\textsc{poss}-after  \textsc{aor}:3\fl{}3-say[II] \textsc{dem}:\textsc{pl} at.all \textsc{neg}-\textsc{ipfv}.\textsc{ifr}-\textsc{peg}-\textsc{denom}-pair-\textsc{peg} \\
\glt `The things he said in the beginning and those that he said later did not correspond to each other.' (contradicted each other) (2011-10-qajdo)
\end{exe}

Other denominal verbs have meanings that are completely unpredictable. For example, the denominal verb \forme{aɣɯjaʁ} from \japhug{tɯ-jaʁ}{hand, arm}, rather than meaning `have arms' as expected (see however §\ref{sec:denom.aGW.redp}), either occurs in the sense of `fidget, touch other peoples' things (like a thief)' or `do things (with one's hands) quickly'. The metaphorical extensions on which the unpredictable meanings of \forme{aɣɯ-} are sometimes shared with other denominal derivations. For instance, \japhug{aɣɯrpaʁ}{go along well} (with non-singular subject) from \japhug{tɯ-rpaʁ}{shoulder} presents the same semantic derivation as the transitive denominal verb \japhug{nɤrpaʁ}{go along well with} (from `carry on the shoulder', §\ref{sec:denom.nW}).

Some lexicalized \forme{aɣɯ-} denominal verbs additionally present unusual argument structure. Unlike other \forme{aɣɯ-} verbs which are plain intransitives, \japhug{aɣɯrɯz}{inherit from} (either from \japhug{ɯ-rɯz}{species}, borrowed from \tibet{རིགས}{rigs}{race}, or from \tibet{རུས}{rus}{bone, lineage}) is a semi-transitive verb (§\ref{sec:semi.transitive}), taking as semi-object the person or group of people one inherits a trait from as in (\ref{ex:tWmu.ra.anWkAGWrWz}).

\begin{exe}
\ex \label{ex:tWmu.ra.anWkAGWrWz}
\gll tɯ-mu ra a-nɯ-kɯ-ɤɣɯrɯz qʰe, nɯ nɯ-ɕɣɤrgu a-pɯ-sna, tɯʑo tɯ-ɕɣɤrgu ɲɯ-sna \\
\textsc{genr}.\textsc{poss}-mother \textsc{pl} \textsc{irr}-\textsc{pfv}-\textsc{genr}:S/O-inherit \textsc{lnk} \textsc{dem} \textsc{3pl}.\textsc{poss}-tooth.quality \textsc{irr}-\textsc{ipfv}-be.good \textsc{genr} \textsc{genr}.\textsc{poss}-tooth.quality \textsc{sens}-be.good \\
\glt `If one inherits from one's mother's [lineage], and [people from] one's mother's [lineage] have good tooth quality, then one will (also) have good tooth quality.' (27-tWCGArgu)
\japhdoi{0003708\#S11}
\end{exe}


%lexicalized:
%aɣɯrɯru zɣɯrɯru
%that has a lot of branches ; that can be done at the same time
 \subsubsection{Comitative adverbs}  \label{sec:denom.aGW.comitative}
The \forme{aɣɯ-} denominal derivation has served as the basis for the development of a separatee morphological category: denominal comitative adverbs (\citealt{jacques17comitative}, §\ref{sec:comitative.adverb}) .

Comitative adverbs are built by combining the prefix  \forme{kɤɣɯ-} (or \forme{kɤ́\trt}, a borrowing from Tshobdun, §\ref{sec:comitative.adverb}) to the partially reduplicated stem of the base noun, as in \forme{kɤɣɯ-rtɯ\tld{}rtaʁ} `together with its branches'  from \japhug{tɤ-rtaʁ}{branch} or \forme{kɤɣɯ-rɟɯ\tld{}rɟit} `together with his/her/its children'  from \japhug{tɤ-rɟit}{child}. These adverbs are formally homophonous with the \forme{kɯ-} subject participles (§\ref{sec:subject.participles}) or stative infinitives (§\ref{sec:infinitives.participles}) of the \forme{aɣɯ-} denominal verbs with emphatic reduplication (§\ref{sec:denom.aGW.redp}).

For example, the surface form \ipa{kɤɣɯrtɯrtaʁ} can be parsed either as a comitative adverb \forme{kɤɣɯ-rtɯ\tld{}rtaʁ} `together with its branches' or as the  participle \forme{kɯ-ɤɣɯ-rtɯ\redp{}rtaʁ} `the one which has many branches' as in (\ref{ex:kAGWrtWrtaR}).

  \begin{exe}
\ex \label{ex:kAGWrtWrtaR}
\gll   si kɯ-ɤɣɯ-rtɯ\redp{}rtaʁ ki kɯ-fse ɲɯ-ɕar-nɯ\\
  tree \textsc{sbj}:\textsc{pcp}-\textsc{denom}-\textsc{emph}\redp{}branch this \textsc{sbj}:\textsc{pcp}-be.this.way \textsc{ipfv}-search-\textsc{pl}\\
\glt `They search for a tree having a lot of branches like this.' (elicited)
\end{exe}

Examples (\ref{ex:kAGWrJWrJit}) and (\ref{ex:kAGWrJWrJit2}) present a minimal pair contrasting  the participle  `have many children' on the one hand and the comitative adverb  `with his/her children'  on the other hand (both pronounced \ipa{kɤɣɯrɟɯrɟit}).

\begin{exe}
\ex 
\begin{xlist}
\ex \label{ex:kAGWrJWrJit}
\gll    iɕqʰa tɕʰeme nɯ kɯ-ɤɣɯ-rɟɯ\redp{}rɟit ci pɯ-ŋu \\
the.aforementioned woman \textsc{dem} \textsc{sbj}:\textsc{pcp}-\textsc{denom}-\textsc{emph}\redp{}child \textsc{indef} \textsc{pst}.\textsc{ipfv}-be \\
\glt `This woman had a lot of children.' (elicited)
\ex \label{ex:kAGWrJWrJit2}
\gll   kɤɣɯ-rɟɯ\redp{}rɟit ʑo jo-nɯ-ɕe-nɯ \\
\textsc{comit}-children \textsc{emph} \textsc{ifr}-\textsc{vert}-go-\textsc{pl} \\
\glt `She/They went back with their children.' (elicited)
\end{xlist}
\end{exe}

The comitative adverbs result from the reanalysis of a non-finite form of proprietive \forme{aɣɯ-} denominal verb, possibly a subject participle or an infinitive in converbial function (§\ref{sec:inf.converb}), with a trivial semantic change from proprietive `have many $X$' to comitative `together with $X$' (on the semantic proximity of the two categories, see for instance \citealt{sutton76having, patz91djabugay, stassen00and, stolz06comitative, arkhipov09comitative}).
 

\subsection{Collective \forme{andʑi-} } \label{sec:denom.andZi}
The \forme{andʑi-} collective denominal prefix is attested in only one example: 
\japhug{andʑirɣa}{be neighbours} from \japhug{tɤ-rɣa}{neighbour}, as in (\ref{ex:YAndZirGandZi}). 

\begin{exe}
\ex \label{ex:YAndZirGandZi}
\gll ɲɯ-ɤndʑi-rɣa-ndʑi \\
\textsc{sens}-\textsc{denom}-neighbour-\textsc{du} \\
\glt `They are one next to the other.' (elicitation)
\end{exe}

This prefix is historically related to the \forme{andʑɯ-} reciprocal (§\ref{sec:andZW.reciprocal}), and the social collective \forme{kɤndʑi-} prefix (§\ref{sec:social.collective}) derives from the combination of the denominal \forme{andʑi-} with the \forme{kɯ-} subject participle prefix (§\ref{sec:subject.participles}). The surface form \ipa{kɤndʑirɣa} is ambiguous between the collective `neighbours' and the participle `who are neighbours to each others, placed one next to each other' as in (\ref{ex:kAndZirGa}).


\begin{exe}
\ex \label{ex:kAndZirGa}
\gll  tɕeri tɯ-tɯ-rdoʁ tɯ-tɯ-rdoʁ ɲɯ-ŋu ma kɯ-ɤndʑi-rɣa kɯ-fse kɯ-ɤrɤ-kʰɯm\redp{}kʰɤl kɯ-fse maŋe. \\
\textsc{lnk} one-one-piece one-one-piece  \textsc{sens}-be \textsc{lnk} \textsc{sbj}:\textsc{pcp}-\textsc{denom}-neighbour \textsc{sbj}:\textsc{pcp}-be.like \textsc{sbj}:\textsc{pcp}-\textsc{denom}-\textsc{emph}\redp{}place  \textsc{sbj}:\textsc{pcp}-be.like not.exist:\textsc{sens} \\
\glt `[This type of mushroom grows] one by one [in isolation], not one next to the other (clustered together), not in patches.' (24-zwArqhAjmAG) \japhdoi{0003630\#S77}
\end{exe}

Although none of the other social collective nouns has a corresponding \forme{andʑi-} stative verb synchronically, this derivation must have been more widespread at an earlier stage. It is possible that the \forme{ndʑi-} element of this prefix is related to the dual \forme{ndʑi-} found in the possessive paradigm and pronouns (§\ref{sec:pers.pronouns}), though how exactly this prefix was built remains unclear.
 
\section{Sigmatic denominal prefixes} \label{sec:sigmatic.denominal}
\subsection{Proprietive \forme{sɤ-} } \label{sec:denom.sA.proprietive}
Proprietive verbs in \forme{sɤ-} are derived from nouns in \forme{tɤ\trt}, either from inalienably possessed nouns selecting the indefinite possessor prefix \forme{tɤ-} (§\ref{sec:inalienably.possessed}) or from nouns with a frozen \forme{tɤ-} prefix. These stative verbs are almost always paired with corresponding dynamic verbs in \forme{nɤ-} (more rarely  \forme{nɯ\trt}, §\ref{sec:denom.nW.pairing}). As shown by \tabref{tab:sA.denom.proprietive}, the base nouns can either be entities (\japhug{tɤ-ndɤɣ}{poison}) or actions (\japhug{tɤ-re}{laugh (n)}). The proprietive denominal verbs, like \forme{sɤ-} proprietive verbs (§\ref{sec:proprietive}), take as intransitive subject the stimulus of the action (perceptible by or affecting other entities). They have a meaning very close to that of the \forme{aɣɯ-} proprietive denominal verbs (§\ref{sec:denom.aGW}), in particular \japhug{aɣɯtɯɣ}{be poisonous}	from \japhug{tɯɣ}{poison}, which has the same meaning as  \japhug{sɤndɤɣ}{be poisonous}.  

The corresponding \forme{nɯ-/nɤ-} verbs can be either intransitive, transitive or labile (§\ref{sec:lability.categories}). Their subject corresponds to the experiencer of the perception/feeling caused by the stimulus. When used transitively, these verbs select the stimulus as object, like tropative verbs (§\ref{sec:tropative}): the intransitive subject of  \japhug{sɤre}{be ridiculous} and \japhug{sɤmtsʰɤr}{be strange} is the same entity as the object of \forme{nɤre} `laugh at' and \japhug{nɤmtsʰɤr}{find strange}.  
 
\begin{table}
\caption{Examples of the denominal \forme{sɤ-} proprietive denominal prefix and corresponding \forme{nɤ-} denominal verbs } \label{tab:sA.denom.proprietive}
\begin{tabular}{lllll}
\lsptoprule
Base Noun& \forme{sɤ-} denominal & \forme{nɤ-} denominal \\
&(stimulus) & (experiencer) \\
\midrule
\japhug{tɤ-ndɤɣ}{poison} & \japhug{sɤndɤɣ}{be poisonous} (vi)&\japhug{nɤndɤɣ}{be poisoned} (vi)\\
\japhug{tɤ-re}{laugh (n)} & \japhug{sɤre}{be ridiculous} (vi) &\japhug{nɤre}{laugh} (vi), \\
&&`laugh at' (vt)\\
\forme{tɤmtsʰɤr}  & \japhug{sɤmtsʰɤr}{be strange} (vi)&\japhug{nɤmtsʰɤr}{find strange} (vt)\\
`strange thing' &&\\
\tablevspace
\japhug{rɤŋom}{outrage},  & \japhug{sɤrɤŋom}{outrage}  &\japhug{nɯrɤŋom}{be outraged} (vt)\\
`vexation' \\
\japhug{tɤ-mbrɯ}{anger}  & \japhug{sɤmbrɯ}{get angry} (vi)&\japhug{nɤmbrɯ}{get angry with} (vt)\\
\lspbottomrule
\end{tabular}
\end{table}
 
The presence of the loanword \forme{-mtsʰɤr} from \tibet{མཚར་}{mtsʰar}{wondrous, strange} among the verbs in \tabref{tab:sA.denom.proprietive} shows that both denominal derivations are productive.

The historical relationship between denominal \forme{sɤ-} and \forme{nɤ-} prefixes on the one hand, and proprietive and tropative prefixes on the other hand, is explored in §\ref{sec:sA.history}.

Proprietive denominal verbs in \forme{sɤ-} do not have tropative forms, and instead form the corresponding \forme{nɤ-} denominal verb instead. For instance \japhug{sɤŋaβ}{be unpleasant}, `be embarrassing' (\ref{ex:sANAB}) lacks a tropative $\dagger$\forme{nɤ-sɤŋaβ}, and instead requires the corresponding transitive denominal tropative \japhug{nɤŋaβ}{consider to be unpleasant}, `be embarrassed by/to' (\ref{ex:kWnANAB}).

\begin{exe}
\ex \label{ex:sANAB}
 \gll aɣɯmdzu tɕe, [kɤ-nɤjaʁ] sɤ-ŋaβ. \\
 have.thorns:\textsc{fact} \textsc{lnk} \textsc{inf}-touch.with.hand \textsc{denom}:\textsc{prop}-unpleasant:\textsc{fact} \\
 \glt `It is thorny, and unpleasant to touch with the hand.' (18-NGolo)
 \japhdoi{0003530\#S60}
\end{exe}

\begin{exe}
\ex \label{ex:kWnANAB}
 \gll  srɯnmɯ nɯ kɯ ʑa [kɤ-ti ɯ-rqo mɤ-kɯ-ɬoʁ] tɕe [kɤ-ti kɯ-nɤ-ŋaβ] to-ʑɣɤpa tɕe, \\
demoness \textsc{dem} \textsc{erg} early \textsc{inf}-say \textsc{3sg}.\textsc{poss}-throat \textsc{neg}-\textsc{sbj}:\textsc{pcp}-come.out \textsc{lnk} \textsc{inf}-say \textsc{sbj}:\textsc{pcp}-\textsc{denom}:\textsc{trop}-unpleasant \textsc{ifr}-pretend \textsc{lnk} \\
\glt `The demoness pretended to be to be embarrassed to say it for a long time.' (28-smAnmi)
\japhdoi{0004063\#S40}
\end{exe}

Example (\ref{ex:sANAB})  also illustrates that some denominal verbs in \forme{sɤ-} can select complement clauses as intransitive subjects. In those cases, the corresponding \forme{nɤ-} denominal verbs are found with complement clauses as objects (\ref{ex:kWnANAB}).

The verb \japhug{sɤŋaβ}{be unpleasant} is also remarkable in that it appears as first element of a the nominal compound  \japhug{sɤŋaβdi}{unpleasant smell} (§\ref{sec.v.n.compounds}). 
 
Some of the denominal verbs in \forme{sɤ-} and \forme{nɤ-} have semantic extensions that are not immediately predictable from the base verb. For instance, the denominal verbs from \japhug{tɤɣa}{visible}, `in the open'  are lexicalized: \forme{sɤɣa} is used in the sense of `be accessible and safe' (of roads, places), and \forme{nɤɣa} is a stative verb meaning either `be completely visible' or `not feel vertigo (while being in a steep place)' (\ref{ex:YWnAGA}).

\begin{exe}
\ex \label{ex:mAkWsAGa}
 \gll  tsʰɤwɤre nɯnɯ, aʁɤndɯndɤt ʑo tɕe, ɯ-jaʁ-pa nɯ wuma ʑo ɲɯ-ɲɟoʁ tɕe, tsʰi kɯ-fse ʑo mɤ-kɯ-sɤɣa tɤ-a<nɯ>ri kɯnɤ pjɯ-ɤtɤr mɯ́j-cʰa. \\
 gecko \textsc{dem} everywhere \textsc{emph} \textsc{lnk} \textsc{3sg}.\textsc{poss}-hand-under \textsc{dem} really \textsc{emph} \textsc{sens}-\textsc{pass}-glue \textsc{lnk} what \textsc{sbj}:\textsc{pcp}-be.like \textsc{emph} \textsc{neg}-\textsc{sbj}:\textsc{pcp}-be.accessible \textsc{aor}:\textsc{up}-<\textsc{auto}>go[II] also \textsc{ipfv}-fall \textsc{neg}:\textsc{sens}-can \\
\glt `The gecko, the palms of its paws are very adhesive, and no matter how steep and inaccessible the places it goes up to, it cannot fall down.' (28-tshAwAre)
\japhdoi{0003722\#S19}
 \end{exe}
 
\begin{exe}
\ex \label{ex:YWnAGA}
 \gll  praʁ ɯ-taʁ ɲɯ-nɤɣa \\
 cliff \textsc{3sg}.\textsc{poss}-on \textsc{sens}-not.feel.vertigo \\
 \glt `He does not feel vertigo on the cliff.' (elicited) 
  \end{exe}

The denominal verb \japhug{sɤmbrɯ}{get angry}, which derives from the noun \japhug{tɤ-mbrɯ}{anger} with the prefix \forme{sɤ\trt}, differs from all proprietive denominal verbs in \tabref{tab:sA.denom.proprietive} in that it selects as subject the experiencer rather than the stimulus, as shown by (\ref{ex:matAtWsAmbrW}) with \textsc{2sg} indexation. The verb used to express the corresponding proprietive meaning is \japhug{sɤmbrɯŋgɯ}{be detestable} (`cause people to get angry', a denominal incorporating verb, §\ref{sec:incorp.denom}, §\ref{sec:orphan.verb}). The corresponding transitive verb  \japhug{nɤmbrɯ}{get angry with} is the functional applicative of \japhug{sɤmbrɯ}{get angry}, as both verbs encode the same type of referent as subject.

\begin{exe}
\ex \label{ex:matAtWsAmbrW}
\gll ma-tɤ-tɯ-sɤ-mbrɯ \\
\textsc{neg}-\textsc{imp}-2-\textsc{denom}:\textsc{prop}-anger \\
\glt `Don't get angry!' (140425 shizi lang huli-zh)
\end{exe}

Another unexplained irregularity of \japhug{sɤmbrɯ}{get angry} is the causative form \japhug{sɤzmbrɯ}{anger}, with the causative \forme{z-} infixed between the denominal prefix and the nominal root (§\ref{sec:sig.caus.irregular.other}).
 

%{sec:facilitative.GA}
\subsection{Causative/instrumental \forme{sV-} } \label{sec:denom.sW.caus.instr}
The sigmatic prefix \forme{sɯ(ɣ)-}/\forme{sɤ-} has a causative `cause to be/have $X$' or instrumental `use $X$' denominal function. As illustrated in \tabref{tab:sW.denom}, most of the denominal verbs derived with this prefix are morphologically transitive, the only exceptions being \japhug{sɯʁejlu}{be left-handed} (`use the left hand') and \japhug{sɯndzɯpe}{sit without crossing legs}.

The \forme{sɯ(ɣ)-}/\forme{sɤ-} allomorphy of this denominal prefix has similarities with that of the sigmatic causative (§\ref{sec:sig.caus.allomorphs}). The allomorph \forme{sɤ-}  is selected when the base noun is an inalienably possessed noun selecting the \forme{tɤ-} indefinite possessor prefix (such as \japhug{tɤ-kʰɯ}{smoke}) or with frozen \forme{tɤ-} prefix (such as \japhug{tɤɕɤt}{comb}), \forme{sɯɣ-/sɯx-} occurs with monosyllabic nominal roots without cluster and uvular or velar onsets such as \japhug{tsʰaʁ}{sieve} (a context similar to that of the \forme{sɯɣ-} allomorph of the sigmatic causative, §\ref{sec:caus.sWG}), and \forme{sɯ-} is found in all other contexts, including inalienably possessed nouns selecting the \forme{tɯ-} indefinite prefix (\japhug{tɯ-jaʁndzu}{finger}) and/or non-inalienably possessed nouns with initial clusters.


\begin{table}
\caption{Causative/Instrumental denominal verbs} \label{tab:sW.denom}
\begin{tabular}{llll}
\lsptoprule
Base noun & Denominal verb \\
\midrule
\japhug{ʁejlu}{left-handed} &	\japhug{sɯʁejlu}{be left-handed} (vi)\\
\forme{ndzɯpe} `sitting position'&	\japhug{sɯndzɯpe}{sit without crossing legs} (vi)	\\
 (without crossing legs)& \\
\tablevspace
 \forme{tɯqartsɯ} `kicking' &  \japhug{sɯqartsɯ}{kick} (vl) \\
  \forme{laʁrdɤβ} `kicking with forelegs' &  \japhug{sɯqartsɯ}{kick} (with forelegs) (vl) \\
\tablevspace
\japhug{tɤ-kʰɯ}{smoke} &	\japhug{sɤkʰɯ}{smoke} (vt)	\\
\japhug{tɤ-rmi}{name} &	\japhug{sɤrmi}{give a name} (vt)	\\
\japhug{tɤ-ɣur}{fence} &	\japhug{sɤɣur}{enclose} (with a fence) (vt)	\\
\japhug{tɤɕɤt}{comb} (n) &	\japhug{sɤɕɤt}{comb} (vt)	\\
\japhug{tɤmcar}{tongs} &	\japhug{sɤmcar}{take with tongs} (vt) \\
\japhug{tɤtʂu}{lamp} &	\japhug{sɤtʂu}{illuminate with a lamp} (vt)	\\
\tablevspace
\japhug{tɯ-jaʁndzu}{finger} &	\japhug{sɯjaʁndzu}{point} (with the finger) (vt)	\\
\japhug{tɯ-ɕtʂi}{sweat} &	\japhug{sɯɕtʂi}{cause to sweat} (vt)	\\
\tablevspace
\japhug{fsaŋ}{fumigation} &	\japhug{sɯfsaŋ}{fumigate} (vt)	\\
\japhug{tsʰaʁ}{sieve} (n) &	\japhug{sɯxtsʰaʁ}{sieve} (vt)	\\
\japhug{tsʰwi}{dye} (n) &	\japhug{sɯxtsʰwi}{dye} (vt)\\
\japhug{ftɕaka}{method} &	\japhug{sɤftɕaka}{prepare} (vt)	\\
\lspbottomrule
\end{tabular}
\end{table}

Denominal verbs in \forme{sɯ(ɣ)-}/\forme{sɤ-} do not correspond to a single light verb construction. In the case of \japhug{sɯndzɯpe}{sit without crossing legs}, the denominal derivation has a meaning similar to that of the base noun \forme{ndzɯpe} `sitting position of women on the ground, without crossing legs' in collocations with the light verb \japhug{βzu}{make} (§\ref{sec:Bzu.lv}) as in (\ref{ex:ndzWpe.YWwGBzu}).

\begin{exe}
\ex \label{ex:ndzWpe.YWwGBzu}
 \gll tɕʰeme nɯ kɤ-kɯ-ɤ<nɯ>mdzɯ tɕe ndzɯpe ntsɯ ɲɯ́-wɣ-βzu pɯ-ra. \\
 girl \textsc{dem} \textsc{aor}-\textsc{genr}:S/O-<\textsc{auto}>sit \textsc{lnk} sitting.position always \textsc{ipfv}-\textsc{inv}-make \textsc{pst}.\textsc{ipfv}-be.needed \\
 \glt `Women, when they sat, had to sit with both legs folded on one side, without crossing legs.' (31-khAjmu)
\japhdoi{0004079\#S24}
\end{exe}

With nouns expressing striking actions such as \japhug{tɯqartsɯ}{kicking},\footnote{The \forme{tɯ-} prefix on \japhug{tɯqartsɯ}{kicking}  may either be a frozen possessor prefix, action nominal prefix or numeral `one' prefix. }  \forme{sɯ-} derives labile verbs like \japhug{sɯqartsɯ}{kick} (§\ref{sec:lability.categories}) corresponding to collocations with \japhug{lɤt}{release} (§\ref{sec:lAt.lv}).


Other verbs such as \japhug{sɯɕtʂi}{cause to sweat} and \japhug{sɤrmi}{give a name} (`cause to be named'), have the same meanings as collocations of the base nouns with the verb \japhug{tɕɤt}{take out}, as shown by the pair  (\ref{ex:tusArminW}) and (\ref{ex:Wrmi.toCAtnW}) from the same story, and example (\ref{ex:tWCtsxi.pjWtCAt}).

\begin{exe}
\ex 
\begin{xlist}
\ex \label{ex:tusArminW}
 \gll ɬɤndʐitɤlɤtsʰaʁ tu-sɤ-rmi-nɯ ɲɯ-ŋu ma, \\
 delphinium \textsc{ipfv}-\textsc{denom}:\textsc{caus}-name-\textsc{pl} \textsc{sens}-be \textsc{lnk} \\
\glt `People call it (a species of \textit{Delphinium})  \forme{ɬɤndʐitɤlɤtsʰaʁ}.' (13-NanWkWmtsWG)
\ex \label{ex:Wrmi.toCAtnW}
 \gll  tɕe nɯ ɯ-rmi nɯ ɬɤndʐitɤlɤtsʰaʁ to-tɕɤt-nɯ tɕe nɯ ɲɯ-ŋu. \\
 \textsc{lnk} \textsc{dem} \textsc{3sg}.\textsc{poss}-name \textsc{dem} delphinium \textsc{ifr}-take.out-\textsc{pl} \textsc{lnk} \textsc{dem} \textsc{sens}-be \\
\glt `This is why people call it \forme{ɬɤndʐitɤlɤtsʰaʁ} `demon milk-filter'.' (13-NanWkWmtsWG)
\end{xlist}
\end{exe}

\begin{exe}
\ex \label{ex:tWCtsxi.pjWtCAt}
 \gll  li tɯ-ɕtʂi ʑo pjɯ-tɕɤt tu-mŋɤm ŋu. \\
 again \textsc{genr}.\textsc{poss}-sweat \textsc{emph} \textsc{ipfv}-take.out \textsc{ipfv}-hurt be:\textsc{fact} \\
 \glt `It hurts again so much that it causes one to sweat.' (25-kACAl)
\japhdoi{0003640\#S86}
\end{exe}
 
The verb \forme{sɤkʰɯ} can either mean `burn to make smoke' (taking a noun such as \japhug{ɕɤɣ}{juniper} as object) or `smoke out by directing smoke towards $X$'. In the second case, this verb is equivalent to the combination of the base noun \japhug{tɤ-kʰɯ}{smoke} with the causative form \forme{sɯx-ɕe} `send, cause to go' as shown by example (\ref{ex:tAkhW.tusWxCenW}), where both the denominal \forme{sɤkʰɯ} and the noun-verb collocation appear.

\begin{exe}
\ex \label{ex:tAkhW.tusWxCenW}
 \gll  iɕqʰa si kʰoŋrɤl tɤ-k-ɤri nɯ ɯ-ŋgɯ kɯnɤ li [...] tu-ɕe qʰe, tɕe ɯ-pa tu-sɤ-kʰɯ-nɯ. ɯ-pa smi pjɯ́-wɣ-βlɯ qʰe tɤ-kʰɯ tu-sɯx-ɕe-nɯ qʰe, \\
 the.aforementioned tree hollow \textsc{aor}-\textsc{sbj}:\textsc{pcp}-go[II] \textsc{dem} \textsc{3sg}.\textsc{poss}-in also again {  } \textsc{ipfv}:\textsc{up}-go \textsc{lnk} \textsc{lnk} \textsc{3sg}.\textsc{poss}-down \textsc{ipfv}:\textsc{up}-\textsc{denom}:\textsc{caus}-smoke-\textsc{pl} \textsc{3sg}.\textsc{poss}-down fire \textsc{ipfv}-\textsc{inv}-burn \textsc{lnk} \textsc{indef}.\textsc{poss}-smoke \textsc{ipfv}:\textsc{up}-\textsc{caus}-go-\textsc{pl} \textsc{lnk}  \\
\glt `[When the bear$_i$] goes up in a hollow tree, [the hunters] smoke it$_i$ from the bottom (of the tree), they make a fire and send the smoke upwards (towards the bear, to smoke it out).' (21-pri)
\japhdoi{0003580\#S63}
\end{exe}

Other denominal verbs such as \japhug{sɯfsaŋ}{fumigate} lack a corresponding light verb construction with the same meaning: although a construction combining the noun \japhug{fsaŋ}{fumigation}  and the verb \japhug{ta}{put} exists (§\ref{sec:ta.lv}), it means `make fumigations' and cannot take a patient like the object \japhug{kʰɤdaʁ}{khatag} in (\ref{ex:khAdaR.tosWfsaN}).

\begin{exe}
\ex \label{ex:khAdaR.tosWfsaN}
 \gll kʰɤdaʁ to-sɯ-fsaŋ qʰe, cʰɤ-mqlaʁ qʰe, \\
 Khatag \textsc{ifr}-\textsc{denom}:\textsc{caus}-fumigation \textsc{lnk} \textsc{ifr}-swallow \textsc{lnk} \\
 \glt `She fumigated the khatag and swallowed it.' (Gesar 2003)
\end{exe}

The verb \japhug{sɤftɕaka}{prepare} stands out among transitive verbs with  a sigmatic denominal prefix in lacking the causative/instrumental meaning. The semantics of this denominal verb does not directly derive from that of the base noun \japhug{ftɕaka}{method}, but rather from the collocation \forme{ftɕaka+βzu}, one of whose meanings is `prepare' (§\ref{sec:nouns.manner.complement}, \citealt[240]{jacques16complementation}). The semantic near-identity between \forme{sɤftɕaka} and \forme{ftɕaka+βzu} is illustrated by the examples (\ref{ex:tusAftCake}) and (\ref{ex:ftɕaka.tABze}). Other denominal verbs derived from \forme{ftɕaka} have the same meaning `prepare'  §\ref{sec:denom.rA.pairing}.


\begin{exe}
\ex \label{ex:tusAftCake}
 \gll ɯ-ndzɤtsʰi nɯ kɯ-mɯ\redp{}mɯm ʑo tu-sɤftɕake qʰe \\
 \textsc{3sg}.\textsc{poss}-meal \textsc{dem} \textsc{sbj}:\textsc{pcp}-\textsc{emph}\redp{}be.tasty \textsc{emph} \textsc{ipfv}-prepare[III] \textsc{lnk} \\
 \glt `She would (each time) prepare a meal (for the old woman).' (2014-kWlAG)
\end{exe}
 
 \begin{exe}
\ex \label{ex:ftɕaka.tABze}
 \gll  ji-ndzɤtsʰi ftɕaka tɤ-βze \\
 \textsc{1pl}.\textsc{poss}-meal manner \textsc{imp}-make[III] \\
 \glt  `Prepare a meal for us!' (meimeidegushi)
\end{exe}

The \forme{sɤ-}  denominal derivation is productive, as shown by the presence of several words of Tibetan origin in \tabref{tab:sW.denom}, including \japhug{tɤɕɤt}{comb}, \japhug{fsaŋ}{fumigation}, \japhug{tsʰaʁ}{sieve}, \japhug{tsʰwi}{dye} and \japhug{ftɕaka}{method}  from \tibet{ཤད་}{ɕad}{comb}, \tibet{བསང་}{bsaŋ}{fumigation}, \tibet{ཚགས་}{tsʰags}{sieve}, \tibet{ཚོས་}{tsʰos}{dye} and \tibet{བཅའ་ཀ་}{bca.ka}{tool}, respectively.
  
Unlike other denominal prefixes such as \forme{nɯ-}/\forme{nɤ\trt}, which can be paired with other prefixes (§\ref{sec:denom.nW.pairing}), verbs derived by the \forme{sɯ(ɣ)-}/\forme{sɤ-} generally don't appear in sets. The groups of denominal verbs including a transitive \forme{sV-} prefix are all different from each other. There is one case of \forme{sɤ-}/\forme{ɣɤ-}/\forme{nɤ-} contrast between \forme{sɤkʰɯ} `smoke' (by directing smoke towards), `burn into smoke' (see \ref{ex:tAkhW.tusWxCenW} above) and the intransitive verbs \japhug{ɣɤkʰɯ}{be smoky} with the \forme{ɣɤ-} prefix (example \ref{ex:kha.YWGAkhWnW}, §\ref{sec:denom.GW}) and \japhug{nɤkʰɯ}{be smoked} (§\ref{sec:denom.intr.nW}). Another group of denominal verbs comprising \japhug{sɤftɕaka}{prepare} with a contrast between \forme{sɤ\trt}, \forme{rɯ-} and \forme{nɯ-} is discussed in §\ref{sec:denom.rA.pairing}.


The verb \japhug{sɤrmi}{give a name} (§\ref{sec:semi.transitive.causative}) could be analyzed as an irregular causative of \japhug{rmi}{be called} (§\ref{sec:sig.caus.irregular.other}), but the vocalism of the prefix is better accounted for by analyzing it instead as a denominal verb from the noun \japhug{tɤ-rmi}{name}, which may itself originates from the base verb \japhug{rmi}{be called}. 

The historical relationship between the sigmatic causative and the denominal \forme{sɯ-} prefix is explored in more detail in §\ref{sec:sW.caus.history}. Another related derivation is that of the \forme{sɤ-} transitive deideophonic verbs (§\ref{sec:GA.sA.deidph}).


\section{Rhotic denominal prefixes}\label{sec:denom.rA}
The \forme{rɯ-/rɤ-} prefix is among the most productive denominal prefixes in Japhug, and has a wide range of meanings. Although not used in deideophonic derivations, it is used to borrow verbs and adjectives from Chinese (§\ref{sec:zh.loanverbs}).

\subsection{Intransitive denominal verbs}  \label{sec:denom.intr.rA}
The \forme{rɯ-/rɤ-} denominal prefix is mainly used to derive intransitive verbs, of which \tabref{tab:denom.rA.intr} provides a representative sample. 
 
\begin{table}
\caption{Intransitive denominal verbs in \forme{rɯ-/rɤ-}} \label{tab:denom.rA.intr}
\begin{tabular}{llll}
\lsptoprule
Base noun & Denominal verb \\
\midrule
\japhug{mɯntoʁ}{flower} & \japhug{rɯmɯntoʁ}{bloom} \\
\japhug{kɯɕnom}{ears} (of corn) & \japhug{rɯkɯɕnom}{shoot out into ears} \\
\japhug{qajɯ}{worm} & \japhug{rɯqajɯ}{get worms} \\
\japhug{ɕom}{milk skin} & \japhug{rɤɕom}{form (of milk skin)} \\
\japhug{tɤ-jwaʁ}{leaf} & \japhug{rɤjwaʁ}{grow leaves} \\
\japhug{ɯ-mat}{fruit} & \japhug{rɤmat}{grow fruits} \\ 
\japhug{tɤ-spɯ}{pus} & \japhug{rɤspɯ}{fester}, `have pus' \\
\japhug{ɯ-cɤβ}{pod} (of beans) & \japhug{rɤcɤβ}{grow pods} \\
\japhug{tɤrka}{twins} & \japhug{rɤrka}{have twins} \\
\japhug{tɤ-pɯ}{offspring, young} (of animal) & \japhug{rɤpɯ}{have young} \\
\japhug{tɤ-rɟit}{child}   & \japhug{rɤrɟit}{have a child} \\
\tablevspace
\japhug{kʰa}{house} & \japhug{rɤkʰa}{build a house} \\
\forme{(ɣʑɤ-)zga} `honey'& \japhug{rɤzga}{make honey}, `gather pollen' \\
\japhug{ɟuli}{flute} & \japhug{rɯɟuli}{play the flute} \\
\japhug{ɯ-stu}{truth, truly} & \japhug{rɤstu}{be truthful} \\
\japhug{kʰramba}{lie} & \japhug{rɯkʰramba}{tell lies} \\
\japhug{kʰɤcɤl}{discussion} (n) & \japhug{rɯkʰɤcɤl}{chat} \\
\japhug{ndzɤtsʰi}{meal} & \japhug{rɯndzɤtsʰi}{have a meal} \\
\japhug{jɤɣɤt}{terrace}, `toilet' & \japhug{rɯjɤɣɤt}{go to the toilet} \\
\japhug{ɕoŋβzu}{woodwork} & \japhug{rɯɕoŋβzu}{do woodwork} \\
\japhug{qartsɤβ}{harvest} & \japhug{rɯqartsɤβ}{do harvesting} \\
\japhug{skɤrwa}{circumambulation} & \japhug{rɯskɤrwa}{do circumambulations} \\
\japhug{χpɯn}{monk} & \japhug{rɤχpɯn}{become a monk} \\
\japhug{ftɕaka}{method} & \japhug{rɯftɕaka}{do preparation} \\
\tablevspace
\japhug{tɯ-ɕmi}{word} & \japhug{rɯɕmi}{speak} \\
\japhug{tɯ-jroʁ}{trace} & \japhug{rɤjroʁ}{leaving traces} \\
\japhug{tɯsqa}{wheat gruel} & \japhug{rɯtɯsqa}{eat wheat gruel} \\ 
\japhug{tɯfcɤr}{pottery} & \japhug{rɤfcɤr}{do pottery} \\ 
\japhug{tɯkrɤz}{discussion} & \japhug{rɤkrɤz}{have a discussion} \\ 
\japhug{tɤ-loʁ}{nest} & \japhug{rɤloʁ}{make a nest} \\
\japhug{ta-ma}{work} (n) & \japhug{rɤma}{work} \\
\lspbottomrule
\end{tabular}
\end{table}

With the exception of adjectives from Chinese (§\ref{sec:zh.loanverbs}), denominal verbs in \forme{rɯ-/rɤ-} are dynamic, and have three main meanings. 

First, they can mean `produce/grow/get $X$' when the base noun corresponds to a part of the referent designated by the subject of the denominal verb (for instance \japhug{rɤjwaʁ}{grow leaves} from \japhug{tɤ-jwaʁ}{leaf}), or an excrescence/offspring growing out of it. The intransitive subjects of such verbs are typically plants, inanimate objects (including body parts), but also humans or animals with meanings such as `give birth to $X$' (for instance \japhug{rɤrɟit}{have a child} from \japhug{tɤ-rɟit}{child}).
 
These denominal verbs have meanings similar to collocations involving several  light verbs. For instance, \japhug{rɤjwaʁ}{grow leaves} and \japhug{rɤmat}{grow fruits} are synonymous with complex predicates comprising their base nouns \japhug{tɤ-jwaʁ}{leaf} and \japhug{ɯ-mat}{fruit} combined with the light verbs \japhug{lɤt}{release} and \japhug{βzu}{make}, as shown in (\ref{ex:YWlAt.YWBze}). Additional examples of denominal verbs belonging to this category that are synonymous with complex predicates in \japhug{βzu}{make} are discussed in §\ref{sec:denominal.vs.light.verb}.

\begin{exe}
\ex \label{ex:YWlAt.YWBze}
 \gll nɯ [ɯ-jwaʁ ɲɯ-lɤt] tɕendɤre [ɯ-mat ɲɯ-βze]  \\
 \textsc{dem} \textsc{3sg}.\textsc{poss}-leaf \textsc{ipfv}-release \textsc{lnk} \textsc{3sg}.\textsc{poss}-fruit \textsc{ipfv}-make[III] \\
 \glt `[The Zanthoxyllum] makes leaves and grows fruits.' (07-tCGom)
\japhdoi{0003434\#S21}
 \end{exe}
 
Second, a related meaning of  the \forme{rɯ-/rɤ-} denominal prefix is `build $X$', when the product of the action results from a volitional action and is separate from the body of the agent, encoded as  
intransitive subject. Denominal verbs of this type, such as \japhug{rɤloʁ}{make a nest} and  \japhug{rɤkʰa}{build a house},  systematically correspond to complex predicates with the verb \japhug{βzu}{make}, such as \forme{kʰa+βzu} `build a house' and   \forme{tɤ-loʁ+βzu} `make a nest'. These constructions select human or animal subjects.

Third, when the base noun refers to an activity rather than an object, the meaning of the \forme{rɯ-/rɤ-} denominal verb is `do $X$, perform a $X$'. There are two types of complex predicates corresponding to this subtype of denominal verbs. 

On the one hand, as in the previous subtypes, verbs such as  \japhug{rɯkʰramba}{tell lies} (from \japhug{kʰramba}{lie}) or \japhug{rɤkrɤz}{have a discussion} (from \japhug{tɯkrɤz}{discussion}), \japhug{rɯstɯnmɯ}{marry} (from \japhug{stɯnmɯ}{marriage})\footnote{Tibetan loanwords are common in this category; the three base nouns in the examples given here come from 
\tibet{ཁྲམ་པ་}{kʰram.ba}{liar}, \tibet{གྲོས་}{gros}{discussion} and \tibet{སྟོན་མོ་}{ston.mo}{banquet}, respectively.} are synonymous with constructions in \japhug{βzu}{make}, \forme{kʰramba+βzu} and \forme{tɯkrɤz+βzu}, respectively. 

On the other hand, verbs such as  \japhug{rɯskɤrwa}{do circumambulations} (from \japhug{skɤrwa}{circumambulation}) and \japhug{rɯjɤɣɤt}{go to the toilet} (from  \japhug{jɤɣɤt}{terrace}, `toilet') instead correspond to collocations with the motion verb \japhug{ɕe}{go} (§\ref{sec:motion.light.verbs}). 
 
Third, in a few cases such as  \japhug{rɯɟuli}{play the flute} (from \japhug{ɟuli}{flute}), the meaning of the denominal derivation is `use $X$', and the corresponding complex predicate selects the light verb \japhug{lɤt}{release} (\forme{ɟuli+lɤt} `play the flute', see \ref{ex:Juli.chAlAt}, §\ref{sec:preverb.speech}).
 
 Fourth, \forme{rɯ-/rɤ-} denominal verbs from nouns of profession can have the meaning `become $X$', for instance in the case of \japhug{rɤχpɯn}{become a monk} from  \japhug{χpɯn}{monk}. The corresponding complex predicate involves the transitive verb \japhug{ndo}{take} (§\ref{sec:ndo.lv}). Most nouns of this type however take the intransitive denominal \forme{nɯ-} to derive a verb with this meaning (§\ref{sec:denom.intr.nW}).
 
 
In addition to nouns, the \forme{rɯ-/rɤ-} denominal derivations also takes as input lexicalized nominalized verb forms. For example, the lexicalized participles \japhug{kɯŋu}{right thing}  and  \japhug{kɯmaʁ}{bad thing} (from the subject participles \japhug{kɯ-ŋu}{the one that is}  and  \japhug{kɯ-maʁ}{the one that is not}, §\ref{sec:lexicalized.subject.participle}) take the \forme{rɯ-} denominal prefix to form the verbs  \japhug{rɯkɯŋu}{do the right thing}, `take good care of one's family' and \japhug{rɯkɯŋu}{do bad things}, respectively.\footnote{The corresponding complex predicate selects the verb \japhug{nɤma}{do}, see \ref{ex:kWNu.mAtWnAme} in §\ref{sec:lexicalized.subject.participle}.} Another example is \japhug{rɯtɯsqa}{eat wheat gruel}, discussed in more details in §\ref{sec:denominalization.action.nominal} and §\ref{sec:denominal.prefixes.morph}.
 
The \forme{rɯ-/rɤ-} prefix also has a deadverbial function, as in the case of  \japhug{rɯʁlɤwɯr}{happen suddenly} which comes from the adverb \japhug{ʁlɤwɯr}{suddenly} (itself borrowed from \tibet{གློ་བུར་}{glo.bur}{sudden}). This verb expresses an action taking place spontaneously as in (\ref{ex:rWRlAwWr}), as opposed to that expressed by the corresponding \forme{nɯ-} verb \japhug{nɯʁlɤwɯr}{do suddenly} (§\ref{sec:denom.intr.nW}).

\begin{exe}
\ex \label{ex:rWRlAwWr}
 \gll kʰɤrɯm nɯnɯ, tɯ-mtɕʰi ɯ-taʁ ʑmbɤr ɲɯ-kɯ-ɬoʁ ŋu. tɕe wuma ʑo rɯʁlɤwɯr    \\
mouth.ulcer \textsc{dem} \textsc{indef}.\textsc{poss}-mouth \textsc{3sg}.\textsc{poss}-on ulcer \textsc{ipfv}-\textsc{sbj}:\textsc{pcp}-come.out be:\textsc{fact} \textsc{lnk} really \textsc{emph} \textsc{denom}-suddenly  \\
\glt `[The disease called] \forme{kʰɤrɯm} is an ulcer which appears on the mouth. It happens very suddenly.' (25-khArWm,1-2) \japhdoi{0003644\#S2}
\end{exe}

\subsection{Transitive  denominal verbs}   \label{sec:denom.tr.rA}
Transitive denominal verbs in \forme{rɤ-} are also attested, but are considerably rarer. \tabref{tab:denom.rA.tr} presents most attested examples.


\begin{table}
\caption{Transitive denominal verbs in \forme{rɯ-/rɤ-}} \label{tab:denom.rA.tr}
\begin{tabular}{llll}
\lsptoprule
Base noun & Denominal verb \\
\midrule
\japhug{tɯ-tɣa}{one span} & \japhug{rɤtɣa}{measure by handspan} \\
\japhug{tɯ-ɟom}{one fathom} & \japhug{rɤɟom}{measure by fathom} \\
\japhug{tɯ-ɣdɤt}{one section} & \japhug{rɤɣdɤt}{cut into sections} \\
\japhug{tɯ-rzɯɣ}{one section}, & \japhug{rɤrzɯɣ}{cut into sections} \\
\japhug{tɯ-tɤrzɯɣ}{one section} &\\
\japhug{tɯ-spra}{one handful} & \japhug{rɤspra}{take a handful of} \\
\tablevspace
\japhug{ɯ-pʰɯ}{price} & \japhug{rɤpʰɯ}{give a price for} \\
\tablevspace
\japhug{ɯ-ɴqra}{broken one} & \japhug{rɤɴqra}{do in an incomplete way} \\
\lspbottomrule
\end{tabular}
\end{table}

Most transitive denominal verbs in \forme{rɤ-} take counted nouns (§\ref{sec:counted.nouns}) as input. Three different meanings can be distinguished. First,  \forme{rɤ-} denominal verbs from counted noun expressing units of lengths (see \tabref{tab:length.cn}, §\ref{sec:measures}) mean `measure by $X$', as in the case of \japhug{rɤtɣa}{measure by handspan} (\ref{ex:chWwGrAtGa}).

\begin{exe}
\ex \label{ex:chWwGrAtGa}
 \gll  tʰɤstɯɣ kɯ-ra nɯ cʰɯ́-wɣ-rɤ-tɣa  \\
 how.much \textsc{sbj}:\textsc{pcp}-be.needed \textsc{dem} \textsc{ipfv}:\textsc{downstream}-\textsc{denom}-handspan  \\
 \glt `One measures how long [the clothes have to be weaved] by handspan.' (vid-20140429092115)
\end{exe}

Second, when the base noun has the meaning `(one) section', the \forme{rɤ-} denominal verb derived from it means `cut/turn into $X$', like corresponding complex predicates with either \japhug{lɤt}{release}, \japhug{βzu}{make} or the causative \japhug{sɯxɕe}{send}, `cause to go'. Example (\ref{ex:pjArwGrzWrzWG}) illustrates the parallel uses of the denominal verb \japhug{rɤrzɯɣ}{cut into sections}  and the complex predicate \forme{X-rzɯɣ+lɤt}.  Emphatic reduplication on the verb \forme{pjɤ́-wɣ-rɤ-rzɯ\redp{}rzɯɣ} contributes to express the fact that the patient is cut into more than two sections.

\begin{exe}
\ex \label{ex:pjArwGrzWrzWG}
 \gll  pjɤ́-wɣ-rɤ-rzɯ\redp{}rzɯɣ tɕe kɯβde-rzɯɣ ʑo tó-wɣ-lɤt, pjɤ́-wɣ-sat. \\
 \textsc{ipfv}-\textsc{inv}-\textsc{denom}-\textsc{emph}\redp{}section \textsc{lnk} four-section \textsc{emph} \textsc{ifr}-\textsc{inv}-release \textsc{ifr}-\textsc{inv}-kill \\
 \glt `[The thief] killed him by cutting him into four pieces.'  (140512 alibaba-zh) 
 \japhdoi{0003965\#S109}
\end{exe}

Third, in the case of the counted noun \japhug{tɯ-spra}{one handful} which refers to the volume contained in a handful, the denominal verb means `take a handful of', `hold in one hand' as in (\ref{ex:kurAspra}).

\begin{exe}
\ex \label{ex:kurAspra}
 \gll ɯ-jɯ, ɯ-sɤ-ndo [tɯ-jaʁ kɯ kú-wɣ-rɤ-spra kɯ-kʰɯ] jamar mɤɕtʂa cʰɯ́-wɣ-βʑoʁ \\
 \textsc{3sg}.\textsc{poss}-handle \textsc{3sg}.\textsc{poss}-\textsc{obl}:\textsc{pcp}-take \textsc{genr}.\textsc{poss}-hand \textsc{erg} \textsc{ipfv}-\textsc{inv}-\textsc{denom}-handful \textsc{sbj}:\textsc{pcp}-be.possible about until \textsc{ipfv}-\textsc{inv}-peel/sharpen \\
\glt `One sharpens its handle until it can be held in one's hand.' (13-tAsAsqAri)
\japhdoi{0003496\#S21}
\end{exe}

 %relativization:
%tɯ-ɕpɤβ kɯβde-rzɯɣ tɤ-kɤ-lɤt nɯnɯ ku-sɯ-ɤlɤɣi-a cʰa-a ɕti nɤ
% (140512 alibaba-zh)

There are two transitive denominal verbs in \forme{rɤ-} that are not derived from counted nouns: \japhug{rɤpʰɯ}{give a price for} from the inalienably possessed noun \japhug{ɯ-pʰɯ}{price} and \japhug{rɤɴqra}{do in an incomplete way} from the property noun \japhug{ɯ-ɴqra}{broken one} (§\ref{sec:property.nouns}).  The former has the lexicalized reflexive \japhug{ʑɣɤrɤpʰɯ}{act according to one's ability}\footnote{The meaning of this verb is close to that of the Chinese expression \ch{量力而行}{liànglìérxíng}{act according to one's ability}.} (probably through an intermediate meaning such as `(correctly) evaluate one's price', §\ref{sec:lexicalized.refl}).


\subsection{Pairing with other denominal prefixes}  \label{sec:denom.rA.pairing}
Intransitive denominal verbs with the \forme{rɯ-/rɤ-} denominal prefix (§\ref{sec:denom.intr.rA}) are often paired with denominal verbs in \forme{nɯ-/nɤ-}.  Several categories of \forme{rV-/nV-} pairs have to be distinguished.

\begin{table}
\caption{Pairs of denominal verbs in \forme{rɯ-/rɤ-} and \forme{nɯ-/nɤ-} } \label{tab:denom.rA.nA}
\begin{tabular}{llll}
\lsptoprule
\forme{rɯ-/rɤ-} denominal verbs & \forme{nɯ-/nɤ-} denominal verbs \\
\tablevspace
\japhug{rɯkɯɕnom}{shoot out into ears}(vi) &  \japhug{nɯkɯɕnom}{collect ears} (vi) \\
\japhug{rɯqajɯ}{get worms} (vi)&  \japhug{nɯqajɯ}{look for worms} (vi)\\
\tablevspace
\japhug{rɯftɕaka}{do preparation} (vi)&  \japhug{nɯftɕaka}{prepare} (vt)\\
\japhug{rɤkrɤz}{have a discussion} (vi)&  \japhug{nɯkrɤz}{discuss} (vt) \\
\japhug{rɤma}{work} (vi)&  \japhug{nɤma}{do} (a job) (vt)\\
\japhug{rɯkʰɤjxwi}{have a meeting} (vi)&  \japhug{nɯkʰɤjxwi}{meet about} (vt)\\
\lspbottomrule
\end{tabular}
\end{table}

There are pairs of intransitive verbs, such as \japhug{rɯqajɯ}{get worms} and \japhug{nɯqajɯ}{look for worms} from \japhug{qajɯ}{worm}, where the \forme{rV-} verb means   `produce/grow/get $X$' (§\ref{sec:denom.intr.rA}), while the \forme{nV-} verb rather has the meaning `look for/search/collect $X$' (§\ref{sec:denom.intr.nW}). 

However, in most cases the \forme{rV-} verb is intransitive and the \forme{nV-} one is transitive. For instance, \japhug{rɤma}{work} (\ref{ex:turAmandZi}) has intransitive morphology (no stem II alternation), no overt object and its subject is in absolutive form, while its counterpart  \japhug{nɤma}{do} (a job) has stem III alternation and its subject is marked in the ergative (\ref{ex:YWnAme} and \ref{ex:Wma.tunAme}) (§\ref{sec:transitivity.morphology}). The intransitive subject of the \forme{rV-} denominal verb and the transitive subject of the \forme{nV-} encode the same referent.
  
\begin{exe}
\ex \label{ex:turAmandZi}
 \gll nɯtɕu a-wi cʰo a-mu ni tu-rɤ-ma-ndʑi, \\
 \textsc{dem}:\textsc{loc} \textsc{1sg}.\textsc{poss}-grandmother \textsc{comit} \textsc{1sg}.\textsc{poss}-mother \textsc{du} \textsc{ipfv}-\textsc{denom}:\textsc{tr}-work-\textsc{du} \\
 \glt `My grandmother and my mother were working there.' (2010-09)
   \end{exe}

The additional object argument of the \forme{nV-} verb which can be a noun phrase as in (\ref{ex:YWnAme}), with a \textit{figura etymologica} \forme{ɯ-ma+nɤma} `do $X$'s work' in (\ref{ex:Wma.tunAme}). 
  
\begin{exe}
\ex \label{ex:YWnAme}
 \gll kʰa nɯra ɲɯ-nɤ-me. \\
 house \textsc{dem}:\textsc{pl} \textsc{sens}-\textsc{denom}:\textsc{tr}-work[III] \\
 \glt `[Her daughter] does the housework.' (14-siblings) \japhdoi{0003508\#S60}
  \end{exe} 
  
\begin{exe}
\ex \label{ex:Wma.tunAme}
 \gll  ɯʑo kɯ rɟɤlpu ɯ-ma tu-nɤ-me \\
 \textsc{3sg} \textsc{erg} king \textsc{3sg}.\textsc{poss}-work \textsc{ipfv}-\textsc{denom}:\textsc{tr}-work[III] \\
 \glt `He did the work of a king.' (2003 qachGA)
    \end{exe}

  In the case of other verbs pairs such as \japhug{rɤkrɤz}{have a discussion} and \japhug{nɯkrɤz}{discuss} (something), the object of the \forme{nV-} verb is an infinitive complement clause, as in (\ref{ex:tonWkrAzndZi}). In some cases the transitive verb \forme{nɯkrɤz} has the additional meaning `discuss and decide that $X$'.

\begin{exe}
\ex \label{ex:tonWkrAzndZi}
 \gll  kʰu cʰondɤre mbro ni to-rɤ-krɤz-ndʑi. [...] [tɯ-rɟit kɤ́-wɣ-nɯskʰrɯ tɕe,
ɯ-pɯ nɯ tɕʰi jamar tɕe tu-kɤ-sci nɯ], to-nɯ-krɤz-ndʑi. \\
tiger \textsc{comit} horse \textsc{du} \textsc{ifr}-\textsc{denom}:\textsc{intr}-discussion-\textsc{du} { }
\textsc{genr}.\textsc{poss}-offspring \textsc{aor}-\textsc{inv}-be.pregnant.from \textsc{lnk} \textsc{3sg}.\textsc{poss}-child \textsc{dem} what about \textsc{lnk} \textsc{ipfv}-\textsc{inf}-be.born \textsc{dem} \textsc{ifr}-\textsc{denom}:\textsc{tr}-discussion-\textsc{du} \\
\glt `The tigress and the mare had a discussion, they discussed how long it would take for the child to be born after one gets pregnant.' (20-tArka) 	\japhdoi{0003566\#S33}
 \end{exe}
 
 The relationship between the \forme{rV-} and the \forme{nV-} verbs is functionally similar to a base verb and its applicative counterpart, or to an antipassive verb and its corresponding base verb (§\ref{sec:antipassive.vs.denominal.pairs}, §\ref{sec:denom.nW.pairing}). The high degree of productivity of this type of denominal pair is shown by the presence not only of Tibetan loanwords among the examples, but also of loanwords from Chinese such as \japhug{rɯkʰɤjxwi}{have a meeting} and  \japhug{nɯkʰɤjxwi}{meet about} (from \ch{开会}{kāihuì}{have a meeting}, §\ref{sec:zh.loanverbs}).

The semantics of the verbs \japhug{rɯftɕaka}{do preparation} and \japhug{nɯftɕaka}{prepare}, like  that of \japhug{sɤftɕaka}{prepare} (§\ref{sec:denom.sW.caus.instr}), derives from the collocation \forme{ftɕaka+βzu} `prepare $X$, prepare to $X$' (the other meanings of this collocation are discussed in \citealt[240]{jacques16complementation}, §\ref{sec:inf.complementation}
and  §\ref{sec:infinitives.other.prefixes}) rather than from the noun \japhug{ftɕaka}{method} itself. Both  \forme{nɯftɕaka} and \forme{ftɕaka+βzu} can occur with infinitival complement clauses (\ref{ex:kuwGnWftCaka}) or nouns (\ref{ex:ftɕaka.tABze} in §\ref{sec:denom.sW.caus.instr}) as objects. The intransitive \forme{rɯftɕaka} occurs as the functional antipassive counterpart of both \forme{nɯftɕaka} and \forme{sɤftɕaka}.

\begin{exe}
\ex \label{ex:kuwGnWftCaka}
 \gll li kɤ-tɤβ kú-wɣ-nɯ-ftɕaka ra [...] tɤ-tɕɯ ra kɯ tɕendɤre kɤ-tɤβ ftɕaka tú-wɣ-βzu tɕe, \\
 again \textsc{inf}-thresh \textsc{ipfv}-\textsc{inv}-\textsc{denom}-manner be.needed:\textsc{fact} { } \textsc{indef}.\textsc{poss}-boy \textsc{pl} \textsc{erg} \textsc{lnk} \textsc{inf}-thresh manner \textsc{ipfv}-\textsc{inv}-make \textsc{lnk} \\
\glt `(After...), one has to prepare the threshing, the men prepare to thresh [the crops].' (2010-10)
\end{exe}
 
\section{Velar denominal prefixes} \label{sec:denom.GW}
The \forme{ɣɤ-} denominal prefix (and its rarer variant \forme{ɣɯ-} has a considerable variety of meanings, and can be used to derive both transitive and intransitive verbs. It is commonly paired with \forme{nɯ-/nɤ-} (§\ref{sec:denom.GA.pairing}). In addition, it is among the denominal prefixes used to borrow adjectives from Chinese (§\ref{sec:zh.loanverbs}) and to build incorporating verbs (§\ref{sec:incorporation}).

\subsection{Intransitive denominal verbs}  \label{sec:denom.intr.GA}
The \forme{ɣɤ-} prefix can derive several types of intransitive verbs, of which \tabref{tab:denom.GA.intr} lists a representative sample.

\begin{table}
\caption{Intransitive denominal verbs in \forme{ɣɯ-/ɣɤ-}} \label{tab:denom.GA.intr}
\begin{tabular}{llll}
\lsptoprule
Base noun & Denominal verb \\
\midrule
\japhug{rdɯl}{dust} & \japhug{ɣɤrdɯl}{be dusty} \\
\japhug{tɤrcoʁ}{mud} & \japhug{ɣɤrcoʁ}{be muddy} \\
\japhug{tɤmbɣo}{deaf person} & \japhug{ɣɤmbɣo}{be deaf} \\
\japhug{tɤndʐo}{cold} (weather) & \japhug{ɣɤndʐo}{be cold} (of weather)  \\
\japhug{tɤɕu}{coolness} & \japhug{ɣɤɕu}{be cool} (of a place)  \\
\japhug{ɯ-ʁre}{authority} & \japhug{ɣɤʁre}{be respected} (of a person)  \\
\japhug{ɯ-jlu}{uncooked} & \japhug{ɣɤjlu}{be uncooked}  \\
\japhug{tɤ-di}{smell} (n) & \japhug{ɣɤdi}{have a smell}, `stink' \\
\japhug{tɤ-mdzu}{thorn} & \japhug{ɣɤmdzu}{have thorns} \\
%\japhug{tɤ-jwaʁ}{leaf} & \japhug{ɣɤjwaʁ}{growing leaves} \\
\japhug{tɤ-tɕɯɣ}{tree shoot} & \japhug{ɣɤtɕɯɣ}{grow shoots} (of trees) \\
\japhug{tɯ-tɕa}{mistake} & \japhug{ɣɤtɕa}{be wrong}  \\
\japhug{tɤ-kʰɯ}{smoke}& \japhug{ɣɤkʰɯ}{be smoky}  \\
\tablevspace
\japhug{tɤ-tsɯr}{crack} & \japhug{ɣɤtsɯr}{have cracks}, `develop cracks' \\
\japhug{tɤwu}{cry} (n) & \japhug{ɣɤwu}{cry} (vi)  \\
\tablevspace
\japhug{tɤrʁaʁ}{game} (n) & \japhug{ɣɤrʁaʁ}{hunt} (vi)  \\
\japhug{ɕoŋtɕa}{timber} & \japhug{ɣɯɕoŋtɕa}{chop timber} \\
\lspbottomrule
\end{tabular}
\end{table}


Some \forme{ɣɤ-} denominal verbs are stative proprietive, like  those in \forme{aɣɯ-} (§\ref{sec:denom.aGW}), such as \japhug{ɣɤrcoʁ}{be muddy} from \japhug{tɤrcoʁ}{mud}. The base nouns can be inalienably possessed nouns, nouns with a frozen \forme{tɤ-} prefix, alienably possessed nouns (such as \japhug{rdɯl}{dust}) and also property nouns (such as \japhug{ɯ-jlu}{uncooked}, §\ref{tab:property.nouns}). The only example of a Tibetan loanword among these nouns is \japhug{rdɯl}{dust} (from \tibet{རྡུལ་}{rdul}{dust}). Unlike \forme{aɣɯ\trt}, the \forme{ɣɤ-} prefix can derive verbs from abstract nouns, such as \japhug{ɣɤndʐo}{be cold} from \forme{tɤndʐo} `cold weather'. 

Different nouns belonging to the same semantic categories do not necessarily select the same denominal prefix. For instance, the inalienably possessed noun \japhug{ɯ-dɯχɯn}{fragrance} is the base for the denominal verb \japhug{aɣɯdɯχɯn}{be fragrant} with the \forme{aɣɯ-} prefix, while the noun \japhug{tɤ-di}{smell} (from which \japhug{ɯ-dɯχɯn}{fragrance} is derived) takes the \forme{ɣɤ-} denominal prefix (\japhug{ɣɤdi}{have a smell}, `stink'). Similarly, while most denominal verbs expressing physical defects are built with the \forme{a-} prefix (§\ref{sec:denom.a}), the verb \japhug{ɣɤmbɣo}{be deaf} (from \japhug{tɤmbɣo}{deaf person}) has the \forme{ɣɤ-} prefix.

The meaning of the \forme{ɣɤ-} prefix is slightly different from that of \forme{aɣɯ-}. Some nouns can take both prefixes, such as \japhug{tɤ-mdzu}{thorn}, which is the base for two denominal verbs: \japhug{aɣɯmdzu}{have a lot of thorns} (see the Japhug definition of this verb in \ref{ex:YAGWmdzu.def}, §\ref{sec:denom.aGW}) and \japhug{ɣɤmdzu}{have thorns}. 

The \forme{aɣɯ-} denominal prefix indicates the presence of an important quantity of the substance or entity referred to by the base noun (`have a lot of $X$' or `produce a lot of $X$'), while \forme{ɣɤ-} prefix rather can be simply glossed as `have $X$', without specification of quantity, as shown by the definition in (\ref{ex:YWGAmdzu.def}).

\begin{exe}
\ex \label{ex:YWGAmdzu.def}
 \gll ɲɯ-ɣɤ-mdzu tɕe tɤ-mdzu ɣɤʑu kɤ-ti ŋu ma kɯ-dɤn mɤ-kɯ-dɤn nɯra maʁ\\
 \textsc{sens}-\textsc{denom}:\textsc{prop}-thorn \textsc{lnk} \textsc{indef}.\textsc{poss}-thorn exist:\textsc{sens} \textsc{obj}:\textsc{pcp}-say be:\textsc{fact}  \textsc{lnk} \textsc{sbj}:\textsc{pcp}-be.a.lot \textsc{neg}-\textsc{sbj}:\textsc{pcp}-be.a.lot \textsc{dem}:\textsc{pl} not.be:\textsc{fact}\\
 \glt `[The word] \forme{ɲɯ-ɣɤ-mdzu} means `it has thorns', it does not [specify whether] there are many or not.' (elicited)
\end{exe}

Another meaning of the \forme{ɣɤ-} prefix is `from which $X$ comes out, emitting $X$'. For instance \japhug{ɣɤrdɯl}{be dusty} has the specific meaning `emitting dust' (of a dusty road) rather than `be covered in dust' (see §\ref{sec:denom.GA.pairing}). The verb \forme{ɣɤkʰɯ} (from \japhug{tɤ-kʰɯ}{smoke}) has two different meanings: `be smoky' (of a place) or `have smoke coming out' (\ref{ex:kha.YWGAkhWnW}). 

\begin{exe}
\ex \label{ex:kha.YWGAkhWnW}
 \gll  kʰa ɲɯ-ɣɤ-kʰɯ-nɯ \\
 house \textsc{sens}-\textsc{denom}-smoke-\textsc{pl} \\
\glt `There is smoke coming out from their house.' (2012 Norbzang) \japhdoi{0003768\#S194}
\end{exe} 

Among these verbs, \japhug{ɣɤtɕa}{be wrong} is one of the few intransitive verbs that can be reflexivized with the \forme{ʑɣɤ-} prefix (§\ref{sec:refl.intr}), with a reflexive tropative meaning \japhug{ʑɣɤɣɤtɕa}{recognize one's mistake}. 

Some \forme{ɣɤ-} denominal verbs are clearly dynamic verbs. For instance, the denominal \japhug{ɣɤtsɯr}{have cracks} (\ref{ex:loGAtsWr})  has the same meaning as the collocation combining the base noun \japhug{tɤ-tsɯr}{crack} with the verb \japhug{ɕe}{go}  (\ref{ex:WtsWr.loCe}).

\begin{exe}
\ex 
\begin{xlist}
\ex \label{ex:loGAtsWr}
 \gll ɯ-rnom-ɕɤrɯ lo-ɣɤ-tsɯr \\
\textsc{3sg}.\textsc{poss}-rib-bone  \textsc{ifr}:\textsc{upstream}-\textsc{denom}-crack \\
\ex \label{ex:WtsWr.loCe}
 \gll ɯ-rnom-ɕɤrɯ ɣɯ ɯ-tsɯr lo-ɕe \\
\textsc{3sg}.\textsc{poss}-rib-bone \textsc{gen} \textsc{3sg}.\textsc{poss}-crack  \textsc{ifr}:\textsc{upstream}-go \\
 \glt `His ribs got fractured.' (elicited)
 \end{xlist}
\end{exe} 

Some of the dynamic intransitive denominal verbs in \forme{ɣɤ-} such as \japhug{ɣɤtsɯr}{have cracks} and \japhug{ɣɤwu}{cry} have patientive intransitive subjects, but a few other verbs such as \japhug{ɣɤrʁaʁ}{hunt} and \japhug{ɣɯɕoŋtɕa}{chop timber} have agentive subjects like the transitive denominal \forme{ɣɯ-/ɣɤ-} verbs treated in the following section.

The intransitive deideophonic \forme{ɣɤ-} prefix (§\ref{sec:GA.sA.deidph}) is probably historically related to the \forme{ɣɤ-} intransitive denominal prefix.

\subsection{Transitive  denominal verbs}   \label{sec:denom.tr.GA}
The \forme{ɣɯ-/ɣɤ-} prefix can also derive transitive verbs,  either from inalienably possessed nouns or counted nouns (see \tabref{tab:denom.GA.intr}). The correlation between the  vocalism of the indefinite possessor prefix \forme{tɯ-/tɤ-} and that of the denominal prefix \forme{ɣɯ-/ɣɤ-}  is less consistent than for other derivations. 
 
The verb \japhug{ɣɤxpra}{send} has the unique allomorph \forme{ɣɤx-} with an intrusive velar fricative \forme{-x-} like some voice derivations (§\ref{sec:caus.sWG}), even though this is not found in the base noun \japhug{tɤpra}{messenger} and the other denominal verb \japhug{nɤpra}{be sent} (§\ref{sec:denom.GA.pairing}).

\begin{table}
\caption{Intransitive denominal verbs in \forme{ɣɯ-/ɣɤ-}} \label{tab:denom.GA.tr}
\begin{tabular}{llll}
\lsptoprule
Base noun & Denominal verb \\
\midrule
\japhug{tɤ-ri}{thread} (n) & \japhug{ɣɯri}{thread} (vt) (beads, needle)    \\
\japhug{tɤ-fkɯm}{bag} & \japhug{ɣɯfkɯm}{put in a bag}  \\
\japhug{tɤjtsi}{pillar}, `post' & \japhug{ɣɯjtsi}{support} (as a pillar  \\
&supporting the roof)  \\
\japhug{tɯ-lɤn}{answer} (n) & \japhug{ɣɯlɤn}{answer} (vt)    \\
\japhug{tɯ-tɕʰa}{information, news} & \japhug{ɣɯtɕʰa}{answer} (vt) (to someone) \\
\tablevspace 
\japhug{tɯ-scur}{a double handful} & \japhug{ɣɯscur}{hold with both hands} \\
\japhug{tɯ-ɕkat}{one load} & \japhug{ɣɯɕkat}{load} (a burden on an animal) \\
\tablevspace 
\japhug{tɯ-jmŋo}{dream} (n) & \japhug{ɣɤjmŋo}{dream of}  \\
\japhug{tɤpra}{messenger}, `envoy' & \japhug{ɣɤxpra}{send} (someone) \\
\japhug{ɯ-tɤjɯ}{addition} (§\ref{sec:incremental.addition}) & \japhug{ɣɤjɯ}{add}  \\
\japhug{tɤ-ro}{surplus, leftover}& \japhug{ɣɤro}{add}, `do/give more'  \\
\lspbottomrule
\end{tabular}
\end{table}

When the base noun refers to a concrete object, the corresponding \forme{ɣɯ-/ɣɤ-}  denominal verb expresses the action prototypically associated with the use of that object, for instance \japhug{ɣɯri}{thread} (pass a thread through beads/a needle's eye) from \japhug{tɤ-ri}{thread}.

Denominal verbs in \forme{ɣɯ-/ɣɤ-} have meanings that are close to those of highly lexically-specific noun-verb collocations, and the correspondences between the argument structures of the denominal verbs and those of the corresponding collocations are not uniform. 
 
For instance, the verb  \japhug{ɣɯtɕʰa}{answer} is semantically close to the complex predicate involving the base noun \japhug{tɯ-tɕʰa}{information, news} and the ditransitive verb \japhug{kʰo}{give}, and the recipient (person who receives an answer to his message) is encoded as the possessor of the inalienably possessed noun \forme{tɯ-tɕʰa} in the collocation (\ref{ex:atCha.mWnakho}, §\ref{sec:biactantial.ipn}), and as direct object of the denominal \forme{ɣɯtɕʰa}. However, the possessor of the base noun does not necessarily always correspond to the object of the \forme{ɣɯ-/ɣɤ-} denominal verb. For example, in the case of the transitive verb \japhug{ɣɤjmŋo}{dream of} and its near-synonymous collocation \forme{tɯ-jmŋo+ntɕʰɤr} `appear in $X$'s dream' (from \japhug{tɯ-jmŋo}{dream} with the intransitive verb \japhug{ntɕʰɤr}{appear}), the possessor of the base noun \forme{tɯ-jmŋo} encodes the  experiencer (the person dreaming), corresponding to the \textit{transitive subject}  of \forme{ɣɤjmŋo} as in (\ref{ex:pWGAjmNota}). 

\begin{exe}
\ex \label{ex:pWGAjmNota}
 \gll [aʑo [...] qartsʰi ɲɤ-k-ɤpa-a-ci] pɯ-ɣɤ-jmŋo-t-a \\
\textsc{1sg} {  } cricket \textsc{ifr}-\textsc{peg}-become-\textsc{1sg}-\textsc{peg} \textsc{aor}-\textsc{denom}-dream-\textsc{pst}:\textsc{tr}-\textsc{1sg} \\
\glt `I dreamed that I had become a cricket.' (150904 cuzhi-zh)
\japhdoi{0006322\#S178}
\end{exe}
 
Some transitive \forme{ɣɯ-} denominal verbs can be subjected to the \forme{a-} passive derivation (§\ref{sec:passive}). The resulting verbs, for instance \japhug{aɣɯɕkat}{be loaded with}  (\ref{ex:aGWCkat}), superficially resemble \forme{aɣɯ-} proprietive denominal verbs (§\ref{sec:denom.aGW}), and it is possible that the  \forme{aɣɯ-} derivation originates from a combination of the passive with the transitive denominal \forme{ɣɯ-}.
 

\begin{exe}
\ex \label{ex:aGWCkat}
 \gll ki tɤrka ki tɤ-rɤku a-ɣɯ-ɕkat \\
 \textsc{dem}.\textsc{prox} mule  \textsc{dem}.\textsc{prox} \textsc{indef}.\textsc{poss}-crop \textsc{pass}-\textsc{denom}-\textsc{load}:\textsc{fact} \\
 \glt `This mule has been loaded with (burdens containing) crops.' (elicited)
 \end{exe}
%ɣɯrɟɤn    
 
 The \forme{ɣɤ-/ɣɯ-} denominal prefix is cognate with Tshobdun \forme{wɐ\trt}, for instance \forme{wɐ-riʔ} `thread a needle' from \forme{riʔ} `thread', \citep{jackson14morpho}, which exactly corresponds to Japhug \japhug{ɣɯri}{thread}.

\subsection{Pairing with other denominal prefixes}  \label{sec:denom.GA.pairing}
A considerable number of nouns can take both \forme{ɣɯ-/ɣɤ-} and \forme{nɯ-/nɤ-} denominal prefixes. \tabref{tab:denom.GA.nA} presents a list of pairs of denominal verbs derived with these prefixes; the base nouns are not included in this table for lack of space, but are indicated in Tables \ref{tab:denom.GA.intr} and \ref{tab:denom.GA.tr} above.
 
These pairs can be classified into three groups, depending on the transitivity of the verbs and the semantic correspondences between them.

 
\begin{table}
\caption{Pairs of denominal verbs in \forme{ɣɯ-/ɣɤ-} and \forme{nɯ-/nɤ-} } \label{tab:denom.GA.nA}
\begin{tabular}{llll}
\lsptoprule
\forme{ɣɯ-/ɣɤ-} denominal verbs & \forme{nɯ-/nɤ-} denominal verbs \\
 \midrule
 \japhug{ɣɤndʐo}{be cold} (of weather) (vi) & \japhug{nɤndʐo}{feel cold} (vi) \\
  \japhug{ɣɤɕu}{be cool}, `be shady' (vi) & \japhug{nɤɕu}{cool off} (vi) (in the shades) \\ 
 \japhug{ɣɤkʰɯ}{be smoky} (vi)& \japhug{nɤkʰɯ}{be smoked} (vi) \\
  \japhug{ɣɤrdɯl}{be dusty} (vi) & \japhug{nɯrdɯl}{be dusty} (vi) \\
  `emit dust' & `becovered in dust' \\
  \tablevspace
\japhug{ɣɤwu}{cry} (vi) &  \japhug{nɤwu}{cry for} (vt) \\
\japhug{ɣɤrʁaʁ}{hunt} (vi) &\japhug{nɤrʁaʁ}{hunt for} (vt) \\
 \tablevspace
\japhug{ɣɤxpra}{send} (vt) & \japhug{nɤpra}{be sent} (vi) \\
\japhug{ɣɤjmŋo}{dream of}(vt) & \japhug{nɯjmŋo}{appear in dream} (vi) \\
\japhug{ɣɯɕkat}{load} (vt) & \japhug{nɯɕkat}{carry loads} (vi)   \\
(a burden on an animal) &(on animals) \\
\lspbottomrule
\end{tabular}
\end{table}

First, we find pairs of intransitive verbs, where the \forme{ɣɯ-/ɣɤ-} verb expresses either a property associated with a place (`have $X$',  `emitting $X$'), or with a dummy intransitive subject (\japhug{ɣɤndʐo}{be cold}), while the  \forme{nɯ-/nɤ-} denominal verbs takes as subject an experiencer (\japhug{nɤndʐo}{feel cold}). 

In the case of  the pair of intransitive verbs \forme{ɣɤrdɯl} and \forme{nɯrdɯl} (from \japhug{rdɯl}{dust}) however, the subject of \forme{nɯrdɯl} is not an experiencer, the semantic difference between the two verbs being explained in (\ref{ex:GArdWl.def}).
 
 \begin{exe}
\ex \label{ex:GArdWl.def}
\gll kɯ-ɣɤ-rdɯl nɯ, ɯʑo ɯ-taʁ rdɯl tu-kɯ-ɬoʁ nɯ ŋu, kɯ-nɯ-rdɯl nɯ, ɯʑo ɯ-taʁ rdɯl kɤ-kɯ-ndzoʁ nɯ ŋu \\
\textsc{inf}:\textsc{stat}-\textsc{denom}-dust \textsc{dem} \textsc{3sg}  \textsc{3sg}.\textsc{poss}-on dust \textsc{ipfv}:\textsc{up}-\textsc{sbj}:\textsc{pcp}-come.out \textsc{dem} be:\textsc{fact} \textsc{inf}:\textsc{stat}-\textsc{denom}-dust \textsc{dem} \textsc{3sg}  \textsc{3sg}.\textsc{poss}-on dust \textsc{aor}-\textsc{sbj}:\textsc{pcp}-\textsc{acaus}:attach \textsc{dem} be:\textsc{fact} \\
\glt `\forme{kɯ-ɣɤ-rdɯl} means that dust is coming up from it, and \forme{kɯ-nɯ-rdɯl} means that dust is attached on it.' (elicited definition)
\end{exe} 
  
 Second, there are cases in which the  \forme{ɣɯ-/ɣɤ-} denominal verb is intransitive, and the corresponding  \forme{nɯ-/nɤ-} verb is transitive, such as \japhug{ɣɤwu}{cry} and  \japhug{nɤwu}{cry for}. The transitive subject of the  \forme{nɯ-/nɤ-} verb corresponds to the same entity as the intransitive subject of its counterpart in \forme{ɣɯ-/ɣɤ\trt}, and has the same functional relationship to it as an applicative derivation (§\ref{sec:applicative}) to its base verb, as the object of the transitive verb in \forme{nɯ-/nɤ-} expresses a patientive argument.
  
Third,  the opposite situation, with transitive denominal verbs in  \forme{ɣɯ-/ɣɤ-} and intransitive verbs in \forme{nɯ-/nɤ\trt}, is also attested. These are passive-like configurations, where the subject of the intransitive verb corresponds to the object of its transitive counterpart (as in \japhug{ɣɤxpra}{send} / \japhug{nɤpra}{be sent}), and antipassive-like configurations, where the subjects of both verbs are the same (\japhug{ɣɯɕkat}{load} / \japhug{nɯɕkat}{carry loads}).
   

%ɣɤʁre ɯ-ʁre naʁre \japhug{ɣɤʁre}{be respected} 
%ɣɤzda tɯ-zda saluer (sur le chemin)
%rɤzda saluer (avant le départ)
% sɤzda aimable 
% nɤzda inviter quelqu'un à se joindre à son groupe
 
 
\section{Labial nasal denominal prefixes} \label{sec:denom.mA}
The most common function of the \forme{mɤ-} denominal prefix to derive intransitive verbs of relative location from locative relator nouns (§\ref{sec:other.locative.relator}), as illustrated in \tabref{tab:mA.denom.location}. Note the presence of Tibetan loanwords in this list, including \japhug{ɯ-pɕi}{outside} and \japhug{ɯ-χcɤl}{center} from \tibet{ཕྱི་}{pʰʲi}{outside} and
\tibet{དཀྱིལ་}{dkʲil}{center}, a fact that demonstrates the productivity of this derivation.\footnote{The first syllable of  \japhug{ɯ-pɤrtʰɤβ}{between} is also borrowed from \tibet{བར་}{bar}{space between}. }

 
\begin{table}
\caption{Denominal verbs of location in \forme{mɤ-} } \label{tab:mA.denom.location}
\begin{tabular}{llll}
\lsptoprule
Base noun & Denominal verb \\
\midrule
\japhug{tɯ-ku}{head}, \japhug{ɯ-ku}{top of} & \japhug{mɤku}{be first} \\
\japhug{ɯ-qʰu}{after, behind} & \japhug{maqʰu}{be after} \\
\japhug{ɯ-pɕi}{outside} & \japhug{mɤpɕi}{be outside} \\
\japhug{ɯ-ŋgɯ}{inside} & \japhug{mɤŋgɯ}{be inside} \\
\japhug{ɯ-χcɤl}{center} & \japhug{mɤχcɤl}{be in the center} \\
\japhug{ɯ-pɤrtʰɤβ}{between} & \japhug{mɤpɤrtʰɤβ}{be in the middle} \\
\lspbottomrule
\end{tabular}
\end{table}

The  \forme{mɤ-} denominal prefix is possibly related to the \forme{maŋ-} prefix which derives verbs of relative locations from locational adverbs, for instance \japhug{maŋlo}{be upstream}  from \japhug{lo}{upstream} (§\ref{sec:verbs.relative.location}).

In addition to their semantic similarity, verbs of location in \forme{mɤ-} and \forme{maŋ-} have in common the fact that they are among the very few intransitive verbs that can be reflexivized (§\ref{sec:refl.intr}): \forme{ʑɣɤ-} prefixation yields volitional motion verbs  such as \japhug{ʑɣɤmɤpɤrtʰɤβ}{put oneself in between} or   \japhug{ʑɣɤmaŋlo}{put oneself upstream}, from \japhug{mɤpɤrtʰɤβ}{be in the middle} and \japhug{maŋlo}{be upstream}. 

The causativization of \forme{mɤ-} denominal verbs has several outcomes. With \japhug{mɤku}{be first} and \japhug{maqʰu}{be after}, which can have temporal meanings (§\ref{sec:ordinals}), the sigmatic causative forms \forme{z-mɤku} and  \forme{z-maqʰu} generally mean `do first' or `do after/later' with complement clauses as in (\ref{ex:Wthu.tAtWzmAkut}) (see also examples \ref{ex:pWzmAke}, §\ref{sec:sig.caus.serial} and \ref{ex:YWGAmda} §\ref{sec:facilitative.GA}).

\begin{exe}
\ex \label{ex:Wthu.tAtWzmAkut}
 \gll sɯlɤmgrɯβdɤn kɯ [ɯ-tʰu] tɤ-tɯ-z-mɤku-t ŋu tɕe \\
  \textsc{anthr} \textsc{erg} \textsc{3sg}.\textsc{poss}-\textsc{bare}.\textsc{inf}:ask \textsc{aor}-2-\textsc{caus}-be.first-\textsc{pst}:\textsc{tr} be:\textsc{fact} \textsc{lnk} \\
 \glt `Bsod.nam sgrub.ldan, you were the first to ask [her hand in marriage].' (sras 2003)
\end{exe}

 The causative \forme{z-maqʰu} also has the straightforward causative meaning `cause to be late, delay' (examples \ref{ex:tha.kWzmaqhutCi}, §\ref{sec:incl.semi.reflexive} and \ref{ex:YWwGzmaqhu}, §\ref{sec:obviation.animacy}).

With other verbs, for which a temporal interpretation is not possible, the causative derivation means `put in $X$', where $X$ corresponds to the relator noun. For instance, \forme{z-mɤŋgɯ} and  \forme{z-mɤpɕi} occur in the sense of `wear (some clothes) inside' and `wear outside'. In addition, \forme{z-mɤpɕi} can be used in the metaphorical sense of `treat as a stranger'.

In addition to generating verbs of relative location, the denominal \forme{mɯ-} and \forme{mɤ-} prefixes are also attested with other functions, illustrated by the examples in \tabref{tab:mA.denom2}.
 
 \begin{table}
\caption{Other denominal verbs  in \forme{mɯ-}/\forme{mɤ-} } \label{tab:mA.denom2}
\begin{tabular}{llll}
\lsptoprule
Base noun & Denominal verb \\
\midrule
\forme{--sti} `alone' & \japhug{mɯsti}{be alone} \\ 
\japhug{tɯ-lɯm}{size},  & \japhug{mɤlɯm}{be big in size} \\ 
`dimensions'  & \\
\japhug{tɤ-mu}{mother} & \japhug{mɤmu}{be the most important} \\
\tablevspace
\japhug{tɤ-rʑaβ}{wife} & \japhug{mɤrʑaβ}{marry} (of a girl) \\
\japhug{tɤ-tɕɯ}{son} & \japhug{mɤtɕɯ}{be adopted as a son}  \\ 
\tablevspace
\japhug{tɯ-rpaʁ}{shoulder} & \japhug{mɤrpaʁ}{carry on the shoulder} \\
\lspbottomrule
\end{tabular}
\end{table}
 
 
 First, they derive stative intransitive verbs of quantity or size such a \japhug{mɤlɯm}{be big in size} from \japhug{tɯ-lɯm}{size}. Most of the examples in this category are highly lexicalized:  \japhug{mɯsti}{be alone} derives from the stem \forme{--sti} `alone' which is only used in the Kamnyu dialect in compounds with pronouns as first element (§\ref{sec:stWsti}) or as the reduplicated adverb \japhug{stɯsti}{alone}, and \japhug{mɤmu}{be the most important}, is presumably derived from \japhug{tɤ-mu}{mother}, though not by a direct semantic change. 
 
 To these examples, it is possible to add \japhug{mɤmbɯr}{protruding}, which probably comes from a lost noun \forme{*mbɯr} borrowed from Tibetan \tibet{འབུར་}{ⁿbur}{bulge, protuberance}, and \japhug{mɯxte}{be the majority}, which may originate from the obsolete property noun \forme{*ɯ-te} `big' (§\ref{sec:augmentative}), though there is also the possibility of an isolated derivation from \japhug{wxti}{be big} with ablaut (see for instance the form \forme{-xte} in the compound noun \japhug{xtɕɯxte}{size}, §\ref{sec.v.v.compounds.degree} and §\ref{sec:denom.a} above).  

 Second, the \forme{mɤ-} prefix is used to build intransitive verbs meaning  `become someone's $X$' from kinship terms, as in  \japhug{mɤrʑaβ}{marry} (of a girl)  and \japhug{mɤtɕɯ}{be adopted as a son}\footnote{This verb has the additional meaning of Chinese \ch{入赘 }{rùzhuì}{marry into one's wife's household}.  }  from  \japhug{tɤ-rʑaβ}{wife} and \japhug{tɤ-tɕɯ}{son}, respectively.
  
 Third, there is one transitive verb derived with the \forme{mɤ-} prefix: \japhug{mɤrpaʁ}{carry on the shoulder} from the inalienably possessed body part \japhug{tɯ-rpaʁ}{shoulder} , anomalous both because of the vocalism of the denominal prefix (\forme{mɯ-} is expected) and because of its isolated meaning, similar to the \forme{nɤ-} denominal verb \japhug{nɤrpaʁ}{carry on the shoulder} (§\ref{sec:denom.tr.nW}).
 
 
\section{Dental nasal denominal prefixes}  \label{sec:denom.nW}
The denominal prefixes \forme{nɯ-/nɤ-} have a high degree of productivity, and present a wide range of functions. They can derive both intransitive and transitive verbs from nouns.
 
\subsection{Intransitive}   \label{sec:denom.intr.nW}
Intransitive denominal derivations in \forme{nɯ-/nɤ-} have at least six different meanings, illustrated in \tabref{tab:denom.nA.intr}. All six categories contain examples of borrowings from Tibetan and even Chinese in some cases, such as  \japhug{ɯ-mdoʁ}{colour}, \japhug{rdɯl}{dust} and \japhug{χpɯnbu}{master} from \tibet{མདོག་}{mdog}{colour}, \tibet{རྡུལ་}{rdul}{dust} and \tibet{དཔོན་པོ་}{dpon.po}{lord}, respectively.
 
\begin{table}
\caption{Intransitive denominal verbs in \forme{nɯ-/nɤ-}} \label{tab:denom.nA.intr}
\begin{tabular}{llll}
\lsptoprule
Base noun & Denominal verb \\
\midrule
\japhug{ɯ-ʁzɯɣ}{appearance} & \japhug{nɯʁzɯɣ}{be pleasing to the eye} \\
\japhug{ta-mar}{butter} & \japhug{nɤmar}{be oily} \\
\japhug{ɯ-mdoʁ}{colour} & \japhug{nɯmdoʁ}{look like} \\
\tablevspace 
\japhug{sɣa}{rust} & \japhug{nɯsɣa}{get rust} \\
\japhug{rdɯl}{dust} & \japhug{nɯrdɯl}{be dusty} (be covered in dust) \\
\japhug{tɯɣur}{frost} & \japhug{nɯɣur}{suffer from frost} \\
 \japhug{tɤ-kʰɯ}{smoke} & \japhug{nɤkʰɯ}{be smoked} \\
 \japhug{tɤʑri}{dew} & \japhug{nɤʑri}{get wet from the dew} \\ 
 \tablevspace
 \japhug{ɯ-χcɤl}{center} & \japhug{nɤχcɤl}{go to the middle} \\
\tablevspace 
 \japhug{tɯmtɕi}{morning} & \japhug{nɯmtɕi}{rise early} \\
 \japhug{tɯrmɯ}{evening} & \japhug{nɯrmɯ}{sleep late} \\
 \tablevspace 
\japhug{rɤɣo}{song} & \japhug{nɯrɤɣo}{sing} \\
\japhug{tɤ-pɤri}{dinner} & \japhug{nɤpɤri}{have dinner} \\
\japhug{saχsɯ}{lunch} & \japhug{nɯsaχsɯ}{have lunch} \\
\japhug{tɤ-ʁaʁ}{good time} & \japhug{nɤʁaʁ}{have a good time} \\
\tablevspace 
\japhug{χpɯnbu}{master} & \japhug{nɯχpɯnbu}{become the master} \\
\japhug{ʁmaʁmi}{soldier} & \japhug{nɯʁmaʁmi}{become a soldier}, `serve in the army' \\
\tablevspace 
\japhug{qarma}{crossoptilon} & \japhug{nɯqarma}{search for crossoptilon} \\
\japhug{mtsʰalu}{nettle} & \japhug{nɯmtsʰalu}{search for nettle} \\
\japhug{qro}{ant} & \japhug{nɯqro}{search for ants} \\
\japhug{tɤjmɤɣ}{mushroom} & \japhug{nɤjmɤɣ}{search for mushrooms} \\
\lspbottomrule
\end{tabular}
\end{table}
 
First, some denominal verbs in \forme{nV-} are stative, including proprietive verbs like  \japhug{nɤmar}{be oily} from \japhug{ta-mar}{butter}, with a meaning quite different from \japhug{aɣɯmar}{produce a lot of butter} with the \forme{aɣɯ-} prefix (§\ref{sec:denom.aGW}).  Borrowed Chinese adjectives taking the \forme{nɯ-} prefix (§\ref{sec:zh.loanverbs}) also belong to this category.
 
Among these stative verbs,  \japhug{nɯmdoʁ}{look like} from  \japhug{ɯ-mdoʁ}{colour} is semi-transitive, as shown by (\ref{ex:pjAnWmdoR}), where its semi-object is the headless relative \forme{ɯ-sni mɤ-kɯ-ɲaʁ} `who is not evil (whose heart is not black)'.  
  
\begin{exe}
\ex \label{ex:pjAnWmdoR}
 \gll <maji> nɯnɯ [ɯ-sni mɤ-kɯ-ɲaʁ] pjɤ-nɯ-mdoʁ  \\
  \textsc{anthr} \textsc{dem} \textsc{3sg}.\textsc{poss}-heart \textsc{neg}-\textsc{sbj}:\textsc{pcp}-be.black \textsc{pst}.\textsc{ifr}-\textsc{denom}-colour \\
 \glt `Ma Ji looked like someone who was not evil.' (160702 luocha) \japhdoi{0006135\#S45}
\end{exe}

The ability of this verb to take a semi-object derives from the grammaticalized use of \japhug{ɯ-mdoʁ}{colour} as a com\-ple\-ment-taking nominal predicate meaning `it looks like...' (§\ref{sec:WmdoR.TAME}). This is one of the few cases where the synthetic construction corresponding to a denominal derivation is not a noun+verb complex predicate (§\ref{sec:denominal.vs.light.verb}).

 
 
Second, the \forme{nɯ-/nɤ-} denominal derivation can mean `get (covered by) $X$, suffer from $X$', as in \japhug{nɯɣur}{suffer from frost} (from \japhug{tɯɣur}{frost}). This verb selects as subject plants (\ref{ex:pWnWGur}), and differs from the corresponding light verb construction \forme{tɯɣur+ta} (\ref{ex:tWGur.pata}) (§\ref{sec:ta.lv}) whose transitive subject is dummy (§\ref{sec:transitive.dummy}).
 
 
 \begin{exe}
\ex \label{ex:pWnWGur}
 \gll stonka tɕe li, nɤki, pɯ-nɯ-ɣur qʰe, li pjɯ-tsɣi qʰe mɯ-ɲɯ-sna ɕti. \\
autumn \textsc{lnk} again \textsc{filler} \textsc{aor}-\textsc{denom}-frost \textsc{lnk} again \textsc{ipfv}-rot \textsc{lnk} \textsc{neg}-\textsc{ipfv}-be.good be.\textsc{aff}:\textsc{fact} \\
\glt `In autumn, when [the \textit{Arisaema consanguineum}] get frosted, it rots and dies.' (14-sWNgWJu)
\japhdoi{0003506\#S166}
\end{exe}

 \begin{exe}
\ex \label{ex:tWGur.pata}
 \gll tɯɣur pa-ta, kɯ-dɤn ʑo mɯ-pa-ta. \\
 frost \textsc{aor}:3\fl{}3-put \textsc{sbj}:\textsc{pcp}-be.many \textsc{emph} \textsc{neg}-\textsc{aor}:3\fl{}3-put \\
 \glt `There was a bit of frost, but not much.' (conversation, 15-12-17)
 \end{exe}
 
 
 Third,  the \forme{nɯ-/nɤ-} denominal prefixes can also derive verbs of motion towards a location, such as \japhug{nɤχcɤl}{go to the center} (from  \japhug{ɯ-χcɤl}{center}). These verbs differ from the verbs of location derived with the \forme{mɤ-} prefix such as \japhug{mɤχcɤl}{be in the center}, which express a static position without motion (\tabref{tab:mA.denom.location}, §\ref{sec:denom.mA}).
 
  
Fourth, these prefixes can be used to build verbs describing an activity related to the base noun. The denominal verbs can have the same meaning as a light verb construction in \japhug{βzu}{make}, as in the case of  \japhug{nɯrɤɣo}{sing} (§\ref{sec:preverb.speech}) and \forme{rɤɣo+βzu} `sing'. Alternatively, the corresponding complex predicate can be an existential construction (§\ref{sec:existential.light.verbs}): compare the use of the denominal verb \japhug{nɯsaχsɯ}{have lunch}   in (\ref{ex:tAtWnWsaXsW}) with that of the base noun \japhug{saχsɯ}{lunch} combined with the existential verb \japhug{tu}{exist} in (\ref{ex:saXsW.WpWtu}).

 \begin{exe}
 \ex 
 \begin{xlist}
\ex \label{ex:tAtWnWsaXsW}
 \gll ɯ-tɤ́-tɯ-nɯ-saχsɯ? \\
 \textsc{qu}-\textsc{aor}-2-\textsc{denom}-lunch \\
 \ex \label{ex:saXsW.WpWtu}
  \gll nɤ-saχsɯ ɯ-pɯ́-tu? \\
\textsc{3sg}.\textsc{poss}-lunch \textsc{qu}-\textsc{pst}.\textsc{ipfv}-exist \\
\glt `Did you have lunch?' (both heard in context several times)
\end{xlist}
\end{exe}

The \forme{rɯ-/rɤ-} denominal prefix is more commonly used for these meanings (in examples such as \japhug{rɯndzɤtsʰi}{have a meal} from \japhug{ndzɤtsʰi}{meal}, §\ref{sec:denom.intr.rA}), and the choice between the nasal and the rhotic denominal prefixes is lexically determined.
 
 Fifth, when the base noun expresses a profession or social status, the denominal verb in \forme{nɯ-} means `become $X$' (\ref{ex:lAnWXpWnbu}), synonymous with the light verb construction with \japhug{ndo}{take} (§\ref{sec:ndo.lv}) as in (\ref{ex:XpWnbu.lando}). 
  
 \begin{exe}
\ex 
 \begin{xlist}
 \ex \label{ex:lAnWXpWnbu}
\gll jaʁmɤcʰɯqa lɤ-nɯ-χpɯnbu  \\
\textsc{anthr} \textsc{aor}:\textsc{upstream}-\textsc{denom}-master \\
\glt `Yagmakhyiqa became the master.' (2003-kWBRa)
 \ex \label{ex:XpWnbu.lando}
\gll χpɯnbu la-ndo \\
 master \textsc{aor}:3\fl{}3:\textsc{upstream}-take \\
\glt `He became the master.' (elicited)
\end{xlist}
\end{exe}

The rhotic prefixes also appear in this function, though only on two nouns: \japhug{χpɯn}{monk} and \japhug{tɕɤmɯ}{nun} (\japhug{rɤχpɯn}{become a monk} and \japhug{rɯtɕɤmɯ}{become a nun}, §\ref{sec:denom.intr.rA}). Using the \forme{nɯ-} denominal prefix instead of the rhotic prefixes with these two nouns is not ungrammatical, but less unfelicitous.

The verb \forme{nɯʁjoʁ} from  \japhug{ʁjoʁ}{servant} is labile, and means `work as a servant' in intransitive use (see §\ref{sec:lability.pass}  and §\ref{sec:denom.tr.nW} for further discussion).
  
 Sixth, when the base noun designates an animal or a plant, the \forme{nɯ-/nɤ-} prefix most often means `search/look for/collect $X$-. This highly productive function corresponds semantically to the combination with the verb \japhug{pʰɯt}{take out, cut} in the case of plants: in (\ref{ex:kWnWmtshalu}) for instance, the same action is redundantly referred to by the verb \japhug{nɯmtsʰalu}{search for nettle} and by \japhug{mtsʰalu}{nettle} followed by \forme{pʰɯt}.
 
 \begin{exe}
\ex \label{ex:kWnWmtshalu}
\gll  kɯ-nɯ-mtsʰalu, mtsʰalu ɯ-kɯ-pʰɯt jo-ɣi   \\
\textsc{sbj}:\textsc{pcp}-\textsc{denom}-nettle nettle \textsc{3sg}.\textsc{poss}-\textsc{sbj}:\textsc{pcp}-take.out \textsc{ifr}-come \\
\glt `She came to collect nettle.' (140520 ye tiane-zh) 
\japhdoi{0004044\#S345}
\end{exe}
 
When the base noun refers to animal, this denominal derivation has the same meaning as the verb \japhug{tɕɤt}{take out}, `extract' as shown by (\ref{ex:kWnWqaJy}) where both constructions redundantly occur.
 
\begin{exe}
\ex \label{ex:kWnWqaJy}
\gll lu-fsoʁ ɕɯŋgɯ tɕe tɕendɤre kɯ-nɯ-qaɟy ntsɯ ju-ɕe pjɤ-ŋu. tɕendɤre qaɟy ntsɯ ʑ-lu-tɕɤt pjɤ-ŋu tɕe, \\
\textsc{ipfv}-be.day before \textsc{lnk} \textsc{lnk} \textsc{sbj}:\textsc{pcp}-\textsc{denom}-fish always \textsc{ipfv}-go \textsc{ifr}.\textsc{ipfv}-be \textsc{lnk} fish always \textsc{tral}-\textsc{ipfv}:\textsc{upstream}-take.out \textsc{ifr}.\textsc{ipfv}-be  \textsc{lnk} \\
\glt `[The fisherman] always went fishing before daybreak.' (140512 yufu yu mogui)
\japhdoi{0003973\#S8}
\end{exe}

Like the rhotic prefix, the \forme{nɯ-} denominal prefix also has a deadverbial function, and occurs in \japhug{nɯʁlɤwɯr}{do suddenly} from \japhug{ʁlɤwɯr}{suddenly} (compare with \japhug{rɯʁlɤwɯr}{happen suddenly} , §\ref{sec:denom.intr.rA}).
  
A possible irregular reduced allomorph \forme{n-} of the intransitive denominal \forme{nɯ-} prefix is possibly found in the verb \japhug{ngo}{be ill}, which is derived from the inalienably possessed noun \japhug{tɯ-ŋgo}{disease}: the group \forme{ng-} represents \ipa{nŋg-} phonologically (§\ref{sec:NC.clusters}). This irregular allomorph appears to also be attested with some transitive verbs (§\ref{sec:denom.tr.nW}).

\subsection{Transitive}   \label{sec:denom.tr.nW}
The \forme{nɯ-/nɤ-} denominal prefix is also used to derive transitive verbs, with a considerable variety of meanings, all with high productivity. In the following, in order to clarify the glosses, $X$ represents the base noun, and $Y$ the object of the corresponding transitive \forme{nɯ-/nɤ-} denominal verb.

\begin{table}
\caption{Transitive denominal verbs in \forme{nɯ-/nɤ-}} \label{tab:denom.nA.tr}
\begin{tabular}{llll}
\lsptoprule
Base noun & Denominal verb \\
\midrule  
\japhug{tɤ-pɤtso}{child} & \japhug{nɯtɤpɤtso}{treat as a child} \\ %tropative
\japhug{ʁgra}{enemy} & \japhug{nɯʁgra}{treat as an enemy}, `be hostile to' \\
\tablevspace 
 \japhug{tɯ-me}{daughter} & \japhug{nɤme}{be adopted as a daughter}   \\%act as
\tablevspace  
\japhug{smɤn}{medicine} & \japhug{nɯsmɤn}{treat}, `heal' \\ %causative/instr
\japhug{tɯ-rpaʁ}{shoulder} & \japhug{nɤrpaʁ}{carry on the shoulder}  \\
\japhug{tɤtar}{stick}, `staff', `rod' & \japhug{nɤtar}{hit with a stick} \\
\japhug{tɤɲi}{walking stick} & \japhug{nɤɲi}{use as a walking stick} \\
\japhug{ɕɤmɯɣdɯ}{gun} & \japhug{nɯɕɤmɯɣdɯ}{shoot at} (with a gun) \\
\japhug{tɤ-βɟu}{mattress} & \japhug{nɤβɟu}{use as a mattress} \\
\tablevspace  
\japhug{tɤ-rme}{hair} & \japhug{nɤrme}{remove the hair}  \\%extract, collect pʰɯt, tɕɤt
\japhug{tɤ-qa}{paw, root} & \japhug{nɤqa}{uproot}  \\
\japhug{tɤ-rqʰu}{hull, skin} & \japhug{nɤrqʰu}{peel}  \\
\japhug{tɤ-lu}{milk} & \japhug{nɤlu}{milk} (a cow) \\
\japhug{tɯ-rdoʁ}{one piece} & \japhug{nɯrdoʁ}{collect piece by piece} \\
\tablevspace 
\japhug{tɤ-mbrɯ}{anger} & \japhug{nɤmbrɯ}{get angry with} \\ % applicative
\japhug{tɤ-re}{laugh (n)} & \japhug{nɤre}{laugh at} \\
\japhug{tɤ-sŋɯt}{bite} & \japhug{nɤsŋɯt}{gnaw}, `bite' \\
\japhug{tɤjkɯz}{secret} & \japhug{nɤjkɯz}{conceal from} \\ 
\tablevspace
\japhug{tɯ-mɢla}{one step}  & \japhug{nɯmɢla}{step over}, `cross' \\ %motion verb
\japhug{ɯ-qʰu}{after, behind}  &  \japhug{nɯɴqʰu}{go along, follow} \\
\tablevspace
 \japhug{tɯ-skʰrɯ}{body} & \japhug{nɯskʰrɯ}{be pregnant with} \\ %collocation
\lspbottomrule
\end{tabular}
\end{table}
 
First, transitive denominal verbs in \forme{nɯ-/nɤ-} can have a tropative function `treat/consider $Y$ as $X$', as in the case of \japhug{nɯʁgra}{treat as an enemy} from \japhug{ʁgra}{enemy}. The verb \forme{nɯʁjoʁ} `give orders to' (from `treat as a servant') from \japhug{ʁjoʁ}{servant} is labile, and means `work as a servant' when conjugated intransitively. It is one of the very few verbs with ergative lability in Japhug (see §\ref{sec:lability.pass}).


The opposite (anti-tropative) meaning `be treated as $X$ by $Y$' is found in \japhug{nɤme}{be adopted as a daughter}  (from \japhug{tɯ-me}{daughter}: `become $Y$'s daughter'), the transitive equivalent of the \forme{mɤ-} derivation found in the intransitive verb \japhug{mɤtɕɯ}{be adopted as a son} (from  \japhug{tɤ-tɕɯ}{son}, §\ref{sec:denom.mA}). 


%nɯskɤt 
%nɯsmɯlɤm

Second, the transitive \forme{nɯ-/nɤ-} denominal has an instrumental function like the sigmatic denominal (§\ref{sec:denom.sW.caus.instr}). This function can be subdivided into two cases: `use $X$ to do to $Y$' as in  \japhug{nɤtar}{hit with a stick} (selecting the entity being hit as object) from \japhug{tɤtar}{stick} and `use $Y$ as an $X$' as in  \japhug{nɤɲi}{use as a walking stick} from \japhug{tɤɲi}{walking stick} (selecting as object the implement used to replace a walking stick, as shown by \ref{ex:tonAYi}).

\begin{exe}
\ex \label{ex:tonAYi}
\gll  ɯ-rkoŋtoŋ nɯnɯ ci to-ndo tɕe to-nɤ-ɲi tɕe \\
\textsc{3sg}.\textsc{poss}-femur \textsc{dem} \textsc{indef} \textsc{ifr}-take \textsc{lnk} \textsc{ifr}-\textsc{denom}-walking.stick \textsc{lnk} \\
\glt `He picked one of its femurs up and used it as a walking stick.' (140511 xinbada-zh) 
\japhdoi{0003961\#S70}
\end{exe}

Third, when the base noun refers to the body part (or a substance coming from) of a plant or an animal, the denominal verb can mean `take the $Y$ from the $X$', the object $X$ being the possessor of that body part. The meaning of this derivation is similar to that of the verbs \japhug{pʰɯt}{take out, cut} and \japhug{tɕɤt}{take out} with the corresponding base nouns:\footnote{The same is true of the intransitive denominal verbs in \forme{nɯ-/nɤ-} meaning `search/look for/collect $X$' discussed above.}  for instance, \japhug{nɤrme}{remove the hair} (shear) expresses the same meaning as \japhug{tɤ-rme}{hair} with \forme{pʰɯt} in example  (\ref{ex:chWnArme}).

\begin{exe}
\ex \label{ex:chWnArme}
\gll  nɯŋa kɯ-fse, qaʑo kɯ-fse nɯra cʰɯ-nɤ-rme ŋgrɤl. nɯnɯ ɯ-rme cʰɯ-pʰɯt ŋgrɤl ma, \\
cow \textsc{sbj}:\textsc{pcp}-be.like sheep \textsc{sbj}:\textsc{pcp}-be.like \textsc{dem}:\textsc{pl} \textsc{ipfv}:\textsc{downstream}-\textsc{denom}:\textsc{tr}-hair be.usually.the.case:\textsc{fact} \textsc{dem} \textsc{3sg}.\textsc{poss}-hair \textsc{ipfv}:\textsc{downstream}-take.off  be.usually.the.case:\textsc{fact} \textsc{lnk} \\
\glt `[Bats] remove the hair of cows and sheep, they remove their hair.' (25-qarmWrwa)
\japhdoi{0003648\#S16}
\end{exe}

With the counted noun \japhug{tɯ-rdoʁ}{one piece} (§\ref{sec:CN.distributive}) the transitive denominal derivation  yields the distributed action meaning \forme{nɯrdoʁ} `pick up one by one'.

Fourth, when the base noun is an abstract noun, the transitive denominal in \forme{nɯ-/nɤ-} expresses an action directed toward a patient or stimulus, as in \japhug{nɤmbrɯ}{get angry with} (from 
\japhug{tɤ-mbrɯ}{anger}, see also §\ref{sec:denom.sA.proprietive}). Among these verbs, \forme{nɤre} `laugh, laugh at, mock' from the inalienably possessed noun \japhug{tɤ-re}{laugh (n)} is labile (§\ref{sec:lability.apass}).


%\japhug{ɯ-thɤβ}{between} & \japhug{nɤtʰɤβ}{surround from both sides} 
  
Fifth, denominal verbs in \forme{nɯ-} can have idiosyncratic meanings associated with a noun+verb collocation. For instance \japhug{nɯskʰrɯ}{be pregnant with} (\ref{ex:tonWkrAzndZi}, §\ref{sec:denom.rA.pairing}) from  \japhug{tɯ-skʰrɯ}{body} derives from the collocation \forme{tɯ-skʰrɯ+NEG+βdi} `be pregnant' (§\ref{sec:other.collocation.intr}), but it did not integrate the verbal root \japhug{βdi}{be well} and the negative prefix.

Irregular allomorphs of the transitive denominal \forme{nɯ-} prefix include the reduced form \forme{n-} in \japhug{ntsɣe}{sell} from \japhug{tɯtsɣe}{commerce} (§\ref{sec:antipassive.irr.form} , and \forme{nɯN-} with an intrusive homorganic nasal in the transitive motion verb \japhug{nɯɴqʰu}{go along, follow} (§\ref{sec:motion.verbs}) which derives from \japhug{ɯ-qʰu}{after, behind}.

\subsection{Pairing with other denominal prefixes}  \label{sec:denom.nW.pairing}
The  \forme{nV-} denominal prefixes are most commonly paired with \forme{rV-} prefixes (§\ref{sec:denom.rA.pairing}). When a transitive \forme{nV-} denominal verb (§\ref{sec:denom.tr.nW}) occurs in pair with an intransitive  \forme{rV-} verb (§\ref{sec:denom.intr.rA}), for instance  \japhug{nɯkrɤz}{discuss} vs. \japhug{rɤkrɤz}{have a discussion}, the intransitive verb serves as the functional antipassive of the transitive one, and the \forme{nV-} verb cannot take the antipassive \forme{rɤ-} prefix (§\ref{sec:antipassive.vs.denominal.pairs}).

The \forme{nɤ-} denominal prefix also derives dynamic verbs (both transitive and intransitive) paired with proprietive denominal verbs in \forme{sɤ-} (§\ref{sec:denom.sA.proprietive}). 

 These verbs are not compatible with the proprietive \forme{sɤ-} derivation (§\ref{sec:proprietive}): the corresponding stative \forme{sɤ-} denominal is used instead. For instance, the transitive denominal verb \japhug{nɤŋaβ}{consider to be unpleasant}, `be embarrassed by' (from \japhug{tɤŋaβ}{unpleasant thing, wrong}) lacks a proprietive form such as $\dagger$\forme{sɤ-nɤŋaβ}, and the denominal \japhug{sɤŋaβ}{be unpleasant}, `be embarrassing' (\ref{ex:sANAB} in §\ref{sec:denom.sA.proprietive})  is used instead.

 Denominal verbs in \forme{nɤ-} can however take the \forme{sɤ-} antipassive prefix (§\ref{sec:antipassive.sA}), for example \japhug{sɤnɤre}{laugh at people} from \japhug{nɤre}{laugh} (§\ref{sec:lability.apass}).

 Some \forme{sɤ-/nɤ-} denominal pairs have lost their base noun. For instance, there is no abstract noun $\dagger$\forme{tɤɕqa} in Kamnyu Japhug corresponding to the pair \japhug{sɤɕqa}{be bearable}/\japhug{nɤɕqa}{endure}. Nevertheless, like synchronic denominal verbs such as \japhug{nɤŋaβ}{consider to be unpleasant}, the transitive form in these pairs cannot undergo the proprietive derivation:  the denominal \forme{sɤɕqa}  (\ref{ex:mAkWsACqa})  is used instead of a putative form such as  $\dagger$\forme{sɤ-nɤɕqa}.
 
\begin{exe}
\ex \label{ex:mAkWsACqa}
\gll a-χpɯm ɲɯ-mŋɤm ri, mɤ-kɯ-sɤɕqa maŋe \\
\textsc{1sg}.\textsc{poss}-knee \textsc{sens}-hurt \textsc{lnk} \textsc{neg}-\textsc{sbj}:\textsc{pcp}-be.bearable not.exist:\textsc{sens} \\
\glt `My knee hurts, but nothing unbearable.' (elicited)
\end{exe}

Pairing with other denominal prefixes such as \forme{ɣɯ-/ɣɤ-} (§\ref{sec:denom.tr.GA}) is also attested, but with less straightforward semantic correspondences between the two verbs.

\section{Other denominal verbs}  \label{sec:denom.other}

\subsection{Zero-derivation or backformation?}  \label{sec:verb.backformation}
While there are non-finite verbs and deverbal nouns lacking any specific nominalization affix (bare infinitives §\ref{sec:bare.inf} and bare action nominals §\ref{sec:bare.action.nominals}), there is little evidence in Japhug for denominal zero-derivation. 

\tabref{tab:verb.backformation} includes cases of noun-verb pairs in which the semantics of the verb is innovative: the nouns have the same meanings as those of their cognates in other Trans-Himalayan languages such as Chinese and Tibetan (for evidence that these nouns are not borrowed from Tibetan, see \citealt[162]{jacques04these} and \citealt{hill14dz}), but there is no trace of the corresponding verbs outside of core Gyalrong languages.

\begin{table}
\caption{Verbs backformed from nouns in Japhug} \label{tab:verb.backformation}
\begin{tabular}{lllll}
\lsptoprule
Noun & Verb & Cognates \\
\midrule
\japhug{ta-mar}{butter} & \japhug{mar}{smear}  & \tibet{མར་}{mar}{butter}\\
\japhug{tɤ-mkɯm}{pillow} & \forme{mkɯm} `have one's head  &  \zh{枕} \forme{*t.kəmʔ} \fl{} \forme{tɕimX} `pillow' \\
& turned towards' \\
\tablevspace
\japhug{tɤjpɣom}{ice} & \japhug{jpɣom}{freeze} &  \zh{冰} \forme{*rpəm} \fl{} \forme{piŋ} `ice' \\
\tablevspace
\japhug{ndzom}{bridge} & \japhug{ndzom}{form a layer of ice} & \tibet{ཟམ་པ་}{zam.pa}{bridge} \\
\lspbottomrule
\end{tabular}
\end{table}

The inalienably possessed nouns \japhug{ta-mar}{butter}, \japhug{tɤ-mkɯm}{pillow} and \japhug{tɤjpɣom}{ice} formally look like bare nominalizations (§\ref{sec:tA.abstract.nouns}, §\ref{sec:bare.action.nominals}) from the corresponding verbs. The etymological relationship between \japhug{ta-mar}{butter} and \japhug{mar}{smear}, while not completely obvious, is supported by the existence of the \textit{figura etymologica} in (\ref{ex:kumarnW}) and the typological parallel provided by the French denominal verb \forme{beurrer} `smear' (not necessarily butter) from \forme{beurre} `butter'.  

\begin{exe}
\ex \label{ex:kumarnW}
 \gll ta-mar ʁɟa ʑo ku-mar-nɯ \\
 \textsc{indef}.\textsc{poss}-butter completely \textsc{emph} \textsc{ipfv}-smear-\textsc{pl} \\
 \glt `They smeared it completely with butter.' (30-komar) 
 \japhdoi{0003738\#S11}
\end{exe}
 
The inalienably possessed noun \japhug{tɤ-mkɯm}{pillow} is related to the rare orienting verb (§\ref{sec:orienting.verbs}) \forme{mkɯm} `have one's head turned towards $X$ while lying in bed', as in (\ref{ex:chWmkWm}).

\begin{exe}
\ex \label{ex:chWmkWm}
\gll  tɤ-tɕɯ nɯ, soz tɕe, rɤru tɤkʰa tɕe, tɕʰeme nɯ ɣɯ [...] ɯ-jme pɕoʁ nɯtɕu ntsɯ cʰɯ-mkɯm pjɤ-ŋu \\
\textsc{indef}.\textsc{poss}-son \textsc{dem} morning \textsc{loc} get.up:\textsc{fact} moment \textsc{loc} girl \textsc{dem} \textsc{gen} {  } \textsc{3sg}.\textsc{poss}-tail side \textsc{dem}:\textsc{loc} always \textsc{ipfv}:\textsc{downstream}-head.towards \textsc{ifr}.\textsc{ipfv}-be \\
\glt `The man, in the morning, when he was about to get up, had his head towards the tail of the woman.' (rkoNrJAl2002)
\end{exe}

In both cases, the cognates in Chinese and Tibetan suggest that the verbs in Japhug are secondary. I propose that rather than being cases of denominal zero-derivation, these verbs were back-formed from the nouns on the model of bare nominalizations.

The case of the verb \forme{ndzom} `form a bridge of ice (over a body of water)' (\ref{ex:kondzom}) is more puzzling, since the noun \japhug{ndzom}{bridge} from which this verb originates is alienably possessed and does not have a frozen \forme{tɤ-} prefix. There is no model from which this verb could have been backformed. 

\begin{exe}
\ex \label{ex:kondzom}
\gll  tɯ-ci ko-ndzom \\
\textsc{indef}.\textsc{poss}-water \textsc{ifr}-form.a.bridge.of.ice \\
\glt `A bridge of ice formed over the river.' (elicited)
\end{exe}

A possible scenario for the backformation hypothesis is that a \forme{nɯ-} denominal verb (§\ref{sec:denom.nW}) was first derived from the base noun, and its \forme{nɯ-} prefix was then reinterpreted as an autive prefix (§\ref{sec:autobenefactive}).\footnote{Both the spontaneous (§\ref{sec:autoben.spontaneous}) and the permansive (§\ref{sec:autoben.permansive}) functions of the autive would be compatible with the meaning of the verb \forme{ndzom} `form a bridge of ice'.} As a consequence of this backformation, a prefixless verb, whose stem is identical to that of the base noun, was created, following the pathway in (\ref{ex:nW.backformation}).

\begin{exe}
\ex \label{ex:nW.backformation}
\gll   ndzom $\Rightarrow$ *nɯ-ndzom $\Rightarrow$ *nɯ-ndzom $\Rightarrow$ ndzom \\
bridge $\Rightarrow$ *\textsc{denom}-bridge  $\Rightarrow$ *\textsc{auto}-form.a.bridge $\Rightarrow$  form.a.bridge \\
\end{exe}

A synchronic example of ongoing reanalysis of the \forme{nɯ-} denominal prefix as an autive prefix is provided by the transitive verb \japhug{nɯmɢla}{step over}, `cross', which derives from the counted noun \japhug{tɯ-mɢla}{one step}. An anomalous form \forme{ja-mɢla} (\ref{ex:jamGla}) without the \forme{nɯ-} prefix is found in the corpus, and can be explained as having been backformed from \forme{ja-nɯ-mɢla} (see \ref{ex:jannWpGaRndZi.jannWmGlandZi}, §\ref{sec:motion.verbs}, with a similar context) by this mechanism.
 
\begin{exe}
\ex \label{ex:jamGla}
\gll rirɤβ tʰɤstɯɣ ja-pɣaʁ, tɯ-ci tʰɤstɯɣ ja-mɢla mɤ-xsi ma, \\
mountain how.many \textsc{aor}:3\flobv{}-cross \textsc{indef}.\textsc{poss}-water how.many \textsc{aor}:3\flobv{}-step.over \textsc{neg}-\textsc{genr}:know \textsc{lnk} \\
\glt `It is not now how many mountains and rivers he crossed.' (160706 poucet6)
\japhdoi{0006109\#S44}
\end{exe}


\subsection{Vowel alternation}  \label{sec:fsu.fse}
The inalienably possessed noun \japhug{ɯ-fsu}{of the same size}, used in one of the equative constructions (§\ref{sec:Wfsu.equative}, §\ref{sec:3sg.possessive.form}, \citealt{jacques18similative}), is etymologically related to the stative verb \japhug{fse}{be like}. This \forme{u} / \forme{e} alternation, which is similar to that of Stem III (§\ref{sec:stem3.form}) and a few other isolated examples (§\ref{sec:pWrndeta}), is possibly due to a \forme{*-j} suffix.

 
\section{Deideophonic verbs}  \label{sec:voice.deideophonic}
In addition to nouns (and some adverbs such as  \japhug{ʁlɤwɯr}{suddenly}, §\ref{sec:denom.intr.rA}), some verbalizing prefixes can take ideophones as input. 


\subsection{\forme{ɣɤ-} and \forme{sɤ-} deideophonic verbs} \label{sec:GA.sA.deidph}
The most productive deideophonic prefixes are \forme{ɣɤ-} and \forme{sɤ-}. Verbs derived with these prefixes can be built from most (though not all) ideophonic roots, either from the reduplicated root (type I) or from a root with partial reduplication in \forme{l-} (type II).

Type I deideophonic verbs can be illustrated by the intransitive verb \forme{ɣɤplaʁplaʁ} `flick, extend and retract' (of a snake's tongue) (\ref{ex:YWGAplaRplaR}) and its transitive counterpart \forme{sɤplaʁplaʁ} (\ref{ex:YWsAplaRplaR}). 

\begin{exe}
\ex
\begin{xlist}
\ex \label{ex:YWGAplaRplaR}
\gll   qapri ɣɯ ɯ-mdʑu ɲɯ-ɣɤ-plaʁplaʁ ʑo \\
snake \textsc{gen} \textsc{3sg}.\textsc{poss}-tongue \textsc{sens}-\textsc{deiph}:\textsc{intr}-flicking \textsc{emph} \\
\glt `The snake's tongue is flicking.' (elicited)
\ex \label{ex:YWsAplaRplaR}
\gll   qapri kɯ ɯ-mdʑu ɲɯ-sɤ-plaʁplaʁ ʑo \\
snake \textsc{erg} \textsc{3sg}.\textsc{poss}-tongue \textsc{sens}-\textsc{deiph}:\textsc{tr}-flicking \textsc{emph} \\
\glt `The snake is flicking its tongue.' (elicited)
\end{xlist}
\end{exe}

Despite the fact that the reduplicated root \forme{-plaʁplaʁ} in these verbs resembles the type II ideophonic pattern (which has a stative meaning, §\ref{sec:ideo.II}), its meaning corresponds to that of the type III pattern `dynamic action' ideophone, namely \japhug{plaʁnɤplaʁ}{flickering} (§\ref{sec:ideo.III}).\footnote{The type II pattern \forme{plaʁplaʁ} of the ideophonic root \forme{-plaʁ} has a entirely unrelated meaning: `completely white'.} Compare (\ref{ex:YWsAplaRplaR}) with the corresponding light verb construction (\ref{ex:plaRnAplaR.YAsWstu}), which uses the similative verb \japhug{stu}{do like} (§\ref{sec:stu.idph}). In this particular case, the deideophonic verb in (\ref{ex:YWsAplaRplaR}) expresses a faster motion than the construction in (\ref{ex:plaRnAplaR.YAsWstu}), but this semantic difference is not generalizable to all deideophonic verbs.

\begin{exe}
\ex \label{ex:plaRnAplaR.YAsWstu}
\gll  qapri kɯ ɯ-mdʑu plaʁnɤplaʁ ʑo ɲɯ-ɤsɯ-stu \\
snake \textsc{erg} \textsc{3sg}.\textsc{poss}-tongue \textsc{idph}(III):flickering \textsc{emph} \textsc{sens}-\textsc{prog}-do.like \\
\glt `The snake is flicking its tongue.' (elicited)
\end{exe}

As a general rule, type I verbs in \forme{ɣɤ-} and \forme{sɤ-} have a meaning that corresponds to that of the type III ideophone based on the same root. They differ from each other in that the prefix \forme{ɣɤ-} builds intransitive verbs, whose synthetic counterpart is a light verb construction with an intransitive verb such as \forme{ʑɣɤstu} `act like' (§\ref{sec:stu.idph}), whereas \forme{sɤ-} derives transitive verbs with a causative meaning `make $X$ do/be' like the construction with \japhug{stu}{do like} (\ref{ex:plaRnAplaR.YAsWstu}).
 
%srɯnmɯ srɯnphɯ kɯ nɯ-ɕɣa pjɤ-sɤrchɯrchɯɣ-nɯ 
%ɣɤrchɯrchɯɣ
%rchɯɣnɤrchɯɣ
%rchɯɣnɤlɯɣ\\
%ɣɤrchɯrlɯɣ
%ɟɯɣnɤɟɯɣ, ɟɯɣnɤlɯɣ, sɤɟɯɣɟɯɣ, sɤɟɯɣlɯɣ

Type II deideophonic verbs have partial reduplication of the ideophonic root in \forme{l\trt}, and semantically correspond to pattern IV (§\ref{sec:ideo.IV}), expressing spatially distributed action (note that the distributed action derivation also presents the \forme{l-} reduplication, §\ref{sec:distributed.action}). Most ideophonic roots either derive type I or type II deideophonic verbs. For instance, there is no verb $\dagger$\forme{ɣɤ-plaʁ-laʁ} from the root \forme{-plaʁ} discussed above (the verb in \ref{ex:YWGAplaRplaR} is \forme{ɣɤ-plaʁ-\textbf{p}laʁ}), 
and the type II  deideophonic verb \japhug{ɣɤɕtʂaŋlaŋ}{balance} from \forme{-ɕtʂaŋ} has no correspond type I verb $\dagger$\forme{ɣɤɕtʂaŋɕtʂaŋ} despite the fact that a type III pattern ideophone \japhug{ɕtʂaŋnɤɕtʂaŋ}{balancing} does exist.

An example of an ideophonic root allowing four deideophonic patterns is \forme{-ɟɯɣ}. The type I verbs \forme{ɣɤɟɯɣɟɯɣ} (vi) and \forme{sɤɟɯɣɟɯɣ} (vt), like the corresponding ideophone \forme{ɟɯɣnɤɟɯɣ}, mean `shake', while the type II verb \forme{ɣɤɟɯɣlɯɣ} (vi) and \forme{sɤɟɯɣlɯɣ} (vt) instead mean `wiggle, squirm, creep', expressing a great number of individuals and/or motion in all directions; these verbs are particularly appropriate to refer to insects, snakes or fishes (see for instance \ref{ex:pjAGAJWGLWG}), but can also be applied to describe a crowd of people. 
 
\begin{exe}
\ex \label{ex:pjAGAJWGLWG}
\gll   tɕe qapri qaɕpa [...] pjɤ-ɣɤ-ɟɯɣlɯɣ ʑo ɕti \\
\textsc{lnk} snake frog { } \textsc{ipfv}.\textsc{ifr}-\textsc{deidph}:\textsc{intr}-wiggling \textsc{emph} be.\textsc{aff}:\textsc{fact} \\
\glt `There were [many] snakes and frogs wiggling [everywhere].' (2003 kWBRa)
\end{exe}


This meaning can also be conveyed by the type IV ideophone \forme{ɟɯɣnɤlɯɣ} `wiggling'.
%qalekɯtshi
%tɕe kɯki ɯ-ʁar ʁnɯz tu-sɤndzɯrndzɯr nɤ tu-sɤndzɯrndzɯr kɯ-fse ɲɯ-ŋu
%23-RmWrcWftsa
%150825_huluwa
%tɕendɤre to-ɣɤɕqali tɕe, nɤki, ta-ɣɤɕqali jamar tɕe rcanɯ li ɬɤndʐi pjɤ-ɕti tɕe,
%wuma ʑo pjɤ-χɕu tɕe nɯra sɤtɕha ra cho kha ra ɲɤ-sɤndzɯrndzɯr ʑo.
%150818_muzhi_guniang
%tɕe ɯ-phoŋbu ra tɤndʐo kɯ pjɤ-ɣɤndzɯrndzɯr ʑo
%ɕɯmŋɤm ɯ-tɤjɯ ɲɯ-sɤzoŋzoŋ ʑo ŋu.
%140428_mtshalu

%ɯ-jaʁ ra ɲɤ-sɤlwɤlwɤt

The deideophonic functions of \forme{ɣɤ-} and \forme{sɤ-} are certainly related to their denominal function. The prefix \forme{ɣɤ-} is also used to derive dynamic intransitive verbs such as \japhug{ɣɤwu}{cry} (from \japhug{tɤwu}{cry} (n), §\ref{sec:denom.intr.GA}), and the meaning of \forme{sɤ-} as a deideophonic prefix is reminiscent of the causative/instrumental function of sigmatic denominal prefixes (§\ref{sec:denom.sW.caus.instr}).

Tshobdun has cognate prefixes \forme{wɐ-} and \forme{sɐ-} with similar functions (\citealt{jackson04zhuangmaoci, jackson14morpho}).

\subsection{\forme{nɯ-} deideophonic verbs} \label{sec:nW.deidph}
Deideophonic verbs in \forme{nɯ-} are built on non-reduplicated ideophonic roots, but like the \forme{ɣɤ-} and \forme{sɤ-} prefixes (§\ref{sec:GA.sA.deidph}), they have a meaning based on that of type III pattern `dynamic action' ideophones. The \forme{nɯ-} derivation yields  transitive verbs which differ from \forme{sɤ-} deideophonic verbs in lacking a causative meaning. For instance, the verb \japhug{nɯxɯr}{turn around} (\ref{ex:kAnWxWra}) from the root \forme{-xɯr} `turn' expresses rotational motion of the subject (with a complex predicate involving the motion verb \forme{skɤrwa+ɕe} `make circumambulations', §\ref{sec:motion.light.verbs}
), like the type III ideophone \forme{xɯrnɤxɯr}, which occurs with the corresponding denominal verb \forme{rɯ-skɤrwa} `make circumambulations' (§\ref{sec:denom.intr.rA}).

\begin{exe}
\ex \label{ex:kAnWxWra}
\gll skɤrwa kɤ-nɯ-xɯr-a ʑo kɤ-ari-a \\
circumambulation \textsc{aor}-\textsc{denom}-turn-\textsc{1sg} \textsc{emph} \textsc{aor}-go[II]-\textsc{1sg} \\
\glt `I made circumambulations [again and again].' (elicited)
\ex \label{ex:xWrnAxWr}
\gll xɯrnɤxɯr ʑo kɤ-rɯ-skɤrwa-a \\
\textsc{idph}(III):turn \textsc{emph} \textsc{aor}-\textsc{denom}-circumambulation-\textsc{1sg} \\
\glt `I made circumambulations [again and again].' (elicited)
\end{exe}

By contrast, the corresponding transitive verb \japhug{sɤxɯrxɯr}{cause to turn} (again and again, quickly) expresses induced motion of the object.
 
In some cases the \forme{nɯ-} and \forme{sɤ-} deideophonic verbs are very close semantically, as illustrated by \forme{nɯ-bɤβ} (\ref{ex:panWbAB.paBde}) and \forme{sɤ-bɤbɤβ} (\ref{ex:pasAbAbAB.paBde}). Both of these express the action of repeatedly throwing down heavy objects which make a loud noise when reaching the ground, with no care for safety. The difference between the two verbs is subtle, \forme{sɤ-bɤbɤβ} putting more focus on the speed and the quantity of objects.
 
\begin{exe}
\ex \label{ex:nWbAB.bABnAbAB}
\begin{xlist}
\ex \label{ex:panWbAB.paBde}
\gll rdɤstaʁ pa-nɯ-bɤβ ʑo pa-βde \\
stone \textsc{aor}:3\fl{}3-\textsc{deidph}-heavy.object \textsc{emph}  \textsc{aor}:3\fl{}3:\textsc{down}-throw \\
\ex \label{ex:pasAbAbAB.paBde}
\gll   pa-sɤ-bɤbɤβ ʑo pa-βde \\
  \textsc{aor}:3\fl{}3-\textsc{deidph}-heavy.object \textsc{emph}  \textsc{aor}:3\fl{}3:\textsc{down}-throw \\
\ex \label{ex:bABnAbAB.paBde}
\gll bɤβnɤbɤβ ʑo pa-βde \\
\textsc{idph}(III):heavy.object \textsc{emph}  \textsc{aor}:3\fl{}3:\textsc{down}-throw \\
\glt `He threw [stones] down, going `boom' [on the ground] again and again.' (elicited)
\end{xlist}
\end{exe}


\subsection{\forme{a-} and \forme{nɤ-} deideophonc verbs} \label{sec:a.nA.deidph}
A few verbs in \forme{a-} can be built from reduplicated ideophonic roots, with a meaning equivalent to the type II ideophonic pattern (§\ref{sec:ideo.II}). For instance, the ideophone \forme{boʁboʁ} `in a group, in a cluster'  (\ref{ex:boRboR.kundzoRnW}) is the source of the intransitive verb \japhug{aboʁboʁ}{cluster around}, `huddle'. 

\begin{exe}
\ex \label{ex:boRboR.kundzoRnW}
\gll tɕe ɴɢoɕna me, porɤt me, ɯ-pɯ tɤ-tu tɕe, kɯki iɕqʰa, [...] ɯ-taʁ nɯnɯre boʁboʁ ʑo ku-ndzoʁ-nɯ tɕe \\
\textsc{lnk} big.spider whether small.spider whether \textsc{3sg}.\textsc{poss}-young \textsc{aor}-exist \textsc{lnk} \textsc{dem}.\textsc{prox} the.aforementioned { } \textsc{3sg}.\textsc{poss}-on \textsc{dem}:\textsc{loc} \textsc{idph}(II):in.group \textsc{emph} \textsc{ipfv}-\textsc{acaus}:attach-\textsc{pl} \textsc{lnk} \\
\glt `When spiders$_i$, whether big ones or small ones, have offspring$_j$, those$_j$ attach in clusters on them$_j$.' (26-mYaRmtsaR)  
\japhdoi{0003674\#S107}
\end{exe}

\begin{exe}
\ex \label{ex:kokAboRboRnW}
\gll kʰɯzɤpɯ ra nɯ-mu ɯ-ɕki ko-k-ɤ-boʁboʁ-nɯ-ci ma ɲɯ-nɤ-ndʐo-nɯ \\
puppy \textsc{pl} \textsc{3pl}.\textsc{poss}-mother \textsc{3sg}.\textsc{poss}-\textsc{dat} \textsc{ifr}-\textsc{peg}-\textsc{deidph}-in.group-\textsc{pl}-\textsc{peg} \textsc{lnk} \textsc{sens}-\textsc{denom}-cold-\textsc{pl} \\
\glt `The puppies huddled against their mother, as they feel cold.' (elicited)
\end{exe}

Intransitive deideophonic verbs in \forme{a-}  pair with transitive verbs in \forme{nɤ-} such as \japhug{nɤboʁboʁ}{cluster around}. The subject of transitive \forme{nɤboʁboʁ} encodes the same semantic role as that of the intransitive subject of \forme{aboʁboʁ}, while its direct object corresponds to the referent marked with the relator noun \japhug{ɯ-taʁ}{on, above} in the intransitive equivalent.

\begin{exe}
\ex \label{ex:pjAkAznAboRboRnWci}
\gll kɯ-mɤɕi ra ɣɯ, nɯ-tɕɯ nɯra kɯ tɕʰeme ci pjɤ-k-ɤz-nɤ-boʁboʁ-nɯ-ci \\
\textsc{sbj}:\textsc{pcp}-be.rich \textsc{pl} \textsc{gen} \textsc{3pl}.\textsc{poss}-son \textsc{dem}:\textsc{pl} \textsc{erg} girl \textsc{indef} \textsc{pst}.\textsc{ipfv}-\textsc{peg}-\textsc{prog}-\textsc{deidph}-in.group-\textsc{pl}-\textsc{peg} \\
\glt `Children from rich [families] were grouped around a girl.' (160630 abao-zh)
\japhdoi{0006197\#S84}
\end{exe}

It is possible to analyze \forme{nɤ-} deideophonic verbs as applicative derivations (§\ref{sec:applicative}) from \forme{a-} deideophonic verbs with regular vowel fusion \forme{nɯ-ɤ-} $\rightarrow$ \forme{nɤ-} (§\ref{sec:allomorphy.applicative}), promoting oblique arguments in \japhug{ɯ-taʁ}{on, above} to object status.

In addition to ideophones expressing `grouping, clustering' like \forme{boʁboʁ}, the \forme{a-} derivation is also compatible with ideophones of shape like \japhug{alɯlju}{be cylindrical} from \forme{ljulju} `cylindrical'. Such verbs do not have a \forme{nɤ-} counterpart.
%asqhlu aɕqhlu 

 
\section{The denominal origin of voice prefixes}  \label{sec:voice.denominal}
There is a remarkable similarity in Japhug between some valency-changing prefixes on the one hand, and denominal prefixes on the other hand, as illustrated by \tabref{tab:voice.denom}. These correspondences suggest that a historical relationship exists between these pairs of prefixes. 

\begin{table}
\caption{Voice derivations and denominal prefixes} \label{tab:voice.denom}
\begin{tabular}{lllll}
\lsptoprule
Voice& & Denominal derivation &  \\
\midrule 
Sigmatic causative  & §\ref{sec:sig.causative} & Instrumental/ & §\ref{sec:denom.sW.caus.instr}\\
\forme{sɯ(ɣ)-/z-} &&causative denominal \forme{sɯ(ɣ)-/sɤ-} &   \\
Applicative \forme{nɯ(ɣ)-} & §\ref{sec:applicative} & Transitive denominal \forme{nɯ-} & §\ref{sec:denom.tr.nW} \\
Tropative \forme{nɤ(ɣ)-} & §\ref{sec:tropative} & Transitive denominal \forme{nɤ-} & §\ref{sec:denom.nW.pairing}  \\
\tablevspace
Passive \forme{a\trt}, & §\ref{sec:passive} & Stative denominal \forme{a-} & §\ref{sec:denom.a} \\
Reciprocal \forme{a-}& §\ref{sec:redp.reciprocal} && \\
Antipassive \forme{rɤ-} & §\ref{sec:antipassive.rA} & Intransitive denominal \forme{rɯ-/rɤ-} & §\ref{sec:denom.intr.rA} \\
Antipassive \forme{sɤ-} & §\ref{sec:antipassive.sA} & Proprietive denominal \forme{sɤ-} & §\ref{sec:denom.sA.proprietive} \\
Proprietive \forme{sɤ-} & §\ref{sec:proprietive} & Proprietive denominal \forme{sɤ-} & §\ref{sec:denom.sA.proprietive} \\
\lspbottomrule
\end{tabular}
\end{table}

The resemblance between these two series of prefixes is not specific to Japhug, and found in all Gyalrongic languages, in particular Khroskyabs (\citealt[527]{lai17khroskyabs}). The following sections (in particular §\ref{sec:antipassive.history}) provide evidence that these resemblances are due to the fact that the derivations in \tabref{tab:voice.denom} actually historically originate from the corresponding denominal derivations (see also \citealt{jacques14antipassive, jacques15causative}, \citealt[527--529]{lai17khroskyabs}, \citealt{lai20denom}).

Not all valency-changing prefixes in Japhug are related to denominal derivations: the reflexive \forme{ʑɣɤ-} in particular derives instead from the incorporation of the \textsc{3sg} pronoun (§\ref{sec:reflexive.origin} and \citealt{jacques10refl}), and some derivations such as the autive and anticausative are probably inherited from proto-Trans-Himalayan  (§\ref{sec:autoben.historical}, §\ref{sec:anticausative}, \citealt{sagart12sprefix}, \citealt{jacques15spontaneous}).

\subsection{The origin of the \forme{rɤ-} antipassive prefix} \label{sec:antipassive.history} 
The resemblance between the \forme{rɤ-} applicative (§\ref{sec:antipassive.rA}) and the intransitive denominal \forme{rɤ-} prefix (§\ref{sec:denom.intr.rA}) suggests that a historical relationship between these prefixes is possible. In addition, the fact that the antipassive only has cognates in Tshobdun and Zbu (\citealt{jackson06paisheng, jackson14morpho}, \citealt{jacques21antipass}) makes it unlikely that this derivation is very ancient.

A few antipassive verbs have formal and semantic irregularities (§\ref{sec:antipassive.rA}). In this section, I show that these irregularities are shared with corresponding action nominals (§\ref{sec:antipassive.irr.form}, §\ref{sec:antipassive.irr.semantic}), and that this observation is a crucial piece of evidence to propose that the antipassive prefix originates from the denominal derivation of action nominals from transitive verbs (§\ref{sec:antipassive.pathway}).

 
\subsubsection{Formal commonalities between antipassive verbs and action nominals} \label{sec:antipassive.irr.form} 
Two antipassive verbs have irregularities is stem formation: \japhug{rɤnŋa}{have a debt}, `owe money'  from \japhug{ŋa}{buy on credit, owe} has an intrusive \forme{-n-} element between the antipassive \forme{rɤ-} prefix and the stem \forme{-ŋa}, and \japhug{rɤtsɣe}{do business} presents the opposite situation: the corresponding transitive verb \japhug{ntsɣe}{sell} has an extra prefixal \forme{n-} element (§\ref{sec:antipassive.rA}).

These morphological specificities are not isolated: the action nominals \japhug{tɯ-nŋa}{debt}  (§\ref{sec:bare.action.nominals}) and \japhug{tɯtsɣe}{commerce}  (§\ref{sec:action.nominals}) have the same stem as the corresponding antipassive verbs.

This commonality between action nominals and the corresponding antipassive verbs is explainable if one assumes that the latter derive from the former by a denominal \forme{rɤ\trt}, on the model of \japhug{rɤkrɤz}{have a discussion}  and  \japhug{rɤma}{work} from \japhug{tɯkrɤz}{discussion} or \japhug{ta-ma}{work} (n) (§\ref{sec:denom.intr.rA}), rather than directly from the corresponding transitive verbs.
 
The \forme{n-} element in \japhug{tɯ-nŋa}{debt} may be analyzable as a nasalized dental nominalization prefix (\forme{*-t-ŋa} $\rightarrow$ \forme{-nŋa},  §\ref{sec:bare.action.nominals}), and the relationship between \forme{rɤnŋa} and the base verb \forme{ŋa} is thus indirect, as shown in (\ref{ex:rAnNa}).

\begin{exe}
\ex \label{ex:rAnNa}
\glt \forme{ŋa} `owe' $\rightarrow$ \forme{-nŋa} `debt' $\rightarrow$ \japhug{rɤnŋa}{have a debt}
\end{exe}
  
The verb \japhug{rɤtsɣe}{do business} is also directly derived from \japhug{tɯtsɣe}{commerce} rather than from the transitive \japhug{ntsɣe}{sell}. The \forme{n-} element on \forme{ntsɣe} is explainable as a reduced allomorph of the transitive denominal \forme{nɯ-} prefix (§\ref{sec:denom.tr.nW}). Thus, while \japhug{rɤtsɣe}{do business} is synchronically perceived as deriving from \forme{ntsɣe}, both verbs historically actually derive from the action nominal \forme{tɯtsɣe}, as shown in \figref{fig:ntsGe}. 

   \begin{figure}
   \caption{The derivational history of \forme{ntsɣe} and \forme{rɤtsɣe}  } \label{fig:ntsGe}  
  \begin{tikzpicture}
    \node (tWtsGe) at (0,0) {  \forme{tɯ-tsɣe}  };
  \node (rAtsGe) at (5,1) { \forme{rɤ-tsɣe} `do business' };
    \node (ntsGe) at (5,-1) {\forme{n-tsɣe} `sell'   };
\tikzstyle{sur}=[->,very thick,>=latex]
\draw[sur] (tWtsGe)--(rAtsGe);
\draw[sur] (tWtsGe)--(ntsGe);
\end{tikzpicture}
\end{figure}


\subsubsection{Semantic commonalities between antipassive verbs and action nominals} \label{sec:antipassive.irr.semantic} 
Commonalities between antipassive verbs and action nominals are not restricted to stem formation as in  §\ref{sec:antipassive.irr.form}  above.

 The formally regular antipassive verb \japhug{rɤpɣaʁ}{reclaim land}, `clear land for farming' has a restricted meaning in comparison with the corresponding transitive verb  \japhug{pɣaʁ}{turn over}: while the latter can occur with a wide range of objects, the former is exclusively used to refer to turning uncultivated land into fields (from the meaning `plough' of the base verb, §\ref{sec:antipassive.lexicalized}). The same semantic restriction is also observed with the action nominal \japhug{tɯpɣaʁ}{land clearing}, used in collocation with \japhug{tɕɤt}{take out} with a meaning close to that of the antipassive \forme{rɤpɣaʁ}.

 \begin{exe}
\ex \label{ex:tWpGaR.lotCAtndZi2}
\gll mɯ-lo-nɤ-tsoʁ-ndʑi kɯ tɯpɣaʁ lo-tɕɤt-ndʑi \\
\textsc{neg}-\textsc{ifr}-\textsc{denom}-silverweed-\textsc{du}  \textsc{erg} field.clearing \textsc{ifr}:\textsc{upstream}-take.out-\textsc{du} \\
\glt `They did not collect silverweed, and cleared fields for farming (instead).' (31-deluge)
\japhdoi{0004077\#S144}
\end{exe}

This common semantic restriction suggests that the action nominal \forme{tɯpɣaʁ} and \forme{rɤpɣaʁ} are related, and can be accounted for by assuming that the latter is a denominal derivation from the former, rather than directly deriving from \japhug{pɣaʁ}{turn over}, as shown in (\ref{ex:rApGaR}).

\begin{exe}
\ex \label{ex:rApGaR}
\glt \japhug{pɣaʁ}{turn over}, `plough fields' $\rightarrow$ \forme{tɯpɣaʁ} `land clearing' $\rightarrow$ \forme{rɤpɣaʁ} `clear land for farming'
\end{exe}
 
\subsubsection{Irregular prefix} \label{sec:antipassive.irr.prefix} 
Another irregular antipassive involves the prefix rather than the stem: the verb \japhug{rɯsɯso}{think}, `ponder' (§\ref{sec:lability.apass}) from \japhug{sɯso}{think} has \forme{rɯ-} rather than the regular \forme{rɤ-} prefix. This irregular form is easily accounted for by the hypothesis that antipassive verbs are denominal derivations from action nominals: \forme{rɯsɯso} is the expected regular rhotic denominal from the bare action nominal  \japhug{tɯ-sɯso}{thought} (§\ref{sec:bare.action.nominals}).

\subsubsection{Pathway of reanalysis} \label{sec:antipassive.pathway} 
Evidence from irregular antipassive verbs presented above in §\ref{sec:antipassive.irr.form}, §\ref{sec:antipassive.irr.semantic} and §\ref{sec:antipassive.irr.prefix} suggest that the formal resemblance between the \forme{rɤ-} antipassive prefix and the intransitive rhotic denominal \forme{rɯ-/rɤ-} prefix is not simply a coincidence, but that the antipassive derivation came into being from the verbalization of an action nominal, following the pathway presented in \figref{fig:apass}.

   \begin{figure}
   \caption{The origin of the antipassive \forme{rɤ-}   } \label{fig:apass}  
  \begin{tikzpicture}
    \node (tr) at (0,0) { Transitive verb $X$ };
  \node (nmlz) at (6,0) { \begin{tabular}{l}
Action nominal \forme{tɯ-}$X$ \\ Bare action nominal $-X$\end{tabular}  };
    \node (apass) at (3,-3) {Antipassive \forme{rɤ-}$X$  };
\tikzstyle{sur}=[->,very thick,>=latex]
\tikzstyle{peutetre}=[->,dotted,very thick,>=latex]
\draw[sur] (tr)--(nmlz);
\draw[sur] (nmlz)--(apass);
\draw[peutetre] (tr)--(apass);
\end{tikzpicture}
\end{figure}

The transitive verb first undergoes nominalization into either an action nominal in \forme{tɯ-} (§\ref{sec:action.nominals}) or into a bare action nominal (taking either a a definite or an indefinite possessive \forme{tɯ-} or \forme{tɤ\trt}, §\ref{sec:bare.action.nominals}). This action nominal then takes the rhotic intransitive denominal prefix. The rhotic prefix removes the \forme{tɯ-/tɤ-} prefix irrespective of whether it is an indefinite possessor prefix or a dental nominalization prefix, following a general property of denominal derivations in Japhug (§\ref{sec:denominal.prefixes.morph}, §\ref{sec:denominalization.action.nominal}). 

With the exception of the few verbs studied in the section above, where a morphological or semantic irregularity in the action noun left a trace on the antipassive verb, the intermediate step leaves no traces. It is possible that once a sufficient number of verbalized denominal action nouns had been created by this process, the antipassive derivation became productive and  that antipassive verbs were directly derived from the transitive base verbs.
 
Even if the antipassive prefix has become functionally separate from the intransitive denominal \forme{rɤ-} in Japhug, many antipassive verbs remain synchronically ambiguous: for instance, the verb \japhug{rɤɕpʰɤt}{patch clothes} can either be analyzed as the antipassive equivalent of \japhug{ɕpʰɤt}{patch} or as a denominal from  \japhug{tɤ-ɕpʰɤt}{patch (n)} (§\ref{sec:antipassive.vs.light.verbs}).

The pathway in \figref{fig:apass}  is not specific to Gyalrong languages;  antipassive affixes from ancient light verb constructions with action nominals have been documented in Mande and elsewhere (\citealt{creissels12antip}, \citealt{sanso17antipassive}), providing a parallel example of the functional overlap between denominal verbalization and light verb constructions in Japhug (§\ref{sec:denominal.vs.light.verb}).

The same two-step mechanism can be used to account for the other resemblances between denominal prefixes and valency-changing derivations  in \tabref{tab:voice.denom}. More generally,  action nominalization neutralizes the transitivity of the verb stem, and the new argument structure of the derived verb is determined by the denominal prefix.
  
\subsection{The proprietive and antipassive \forme{sɤ-} prefixes and their tropative counterpart} \label{sec:sA.history} 
The form and meaning of the proprietive \forme{sɤ-} derivation (§\ref{sec:proprietive}) are close to that of the denominal proprietive \forme{sɤ-} (§\ref{sec:denom.sA.proprietive}). The \forme{sɤ-} denominal verbs often occur in pairs with transitive or intransitive  \forme{nɤ-} denominal verbs from the same noun (§\ref{sec:denom.nW.pairing}), as in \japhug{sɤre}{be ridiculous} and \japhug{nɤre}{laugh}, `laugh at' (from \japhug{tɤ-re}{laugh (n)}, \tabref{tab:sA.denom.proprietive}, §\ref{sec:denom.sA.proprietive}).

Both the proprietive \forme{sɤ-} (§\ref{sec:proprietive}) and the tropative \forme{nɤ-} (§\ref{sec:tropative}) can be accounted for by assuming a pathway similar to that of the antipassive (\figref{fig:apass}), by supposing \forme{sɤ-} and \forme{nɤ-} denominal derivations from abstract nouns (§\ref{sec:tA.abstract.nouns}) or degree nominals (§\ref{sec:degree.nominals}). The intermediate abstract nouns are actually attested in many cases, as in (\ref{ex:sAGmu.pathway}) and (\ref{ex:nAmpCAr.pathway}).
 
 \begin{exe}
\ex \label{ex:sAGmu.pathway}
\glt  \japhug{mu}{be afraid} $\rightarrow$ \japhug{tɯmu}{fear} $\rightarrow$ \japhug{sɤɣmu}{be frightening}
\ex \label{ex:nAmpCAr.pathway}
\glt \japhug{mpɕɤr}{be beautiful} $\rightarrow$ \japhug{tɤmpɕɤr}{beauty} $\rightarrow$ \japhug{nɤmpɕɤr}{find beautiful}
\end{exe}

In addition, the proprietive verb \japhug{sɤɣɲat}{be exhausting} (from  \japhug{ɲat}{be tired}) with the allomorph \forme{sɤɣ-} presents the same intrusive \forme{-ɣ-} element as the abstract noun  \japhug{tɤɣɲat}{tiredness}, supporting the idea that this noun was indeed the intermediate step between the base verb and its proprietive form.
  
 When the base verb undergoing action nominalization and proprietive verbalization is transitive, there is a potential ambiguity, since the proprietive derivation selects as intransitive subject a referent either possessing a particular property or having the tendency or propensity to perform a particular action. With action verbs such as \japhug{mtsɯɣ}{bite}, the most natural interpretation will be `have the propensity to bite' rather than `tend to be bitten' (\ref{ex:sAmtsWG}), yielding an antipassive derivation (§\ref{sec:antipassive.sA}).
 
 \begin{exe}
\ex \label{ex:sAmtsWG}
\glt \japhug{mtsɯɣ}{bite} $\rightarrow$ \forme{*tɯmtsɯɣ} `action of biting' $\rightarrow$ \forme{sɤmtsɯɣ} `bite people'
\end{exe}

With verbs of cognition, an ambiguity is possible; for instance the \forme{sɤ-} derivation from the transitive verb \japhug{nɯzdɯɣ}{worry about} is interpretable either as proprietive \japhug{sɤnɯzdɯɣ}{causing people to worry} (§\ref{sec:proprietive.tr}) or as the antipassive `worry about people' (§\ref{sec:proprietive.compatibility}). Both of these meanings can be explained as proprietive denominal derivations from an abstract noun \forme{*tV-nɯzdɯɣ} `worry'.

\subsection{Applicative, sigmatic causative and passive} \label{sec:applicative.history} \label{sec:sW.caus.history} 
Following the same type of pathways as for antipassive, proprietive and tropative derivations, the applicative \forme{nɯ(ɣ)-} (§\ref{sec:applicative}), the causative  \forme{sɯ(ɣ)-} (§\ref{sec:sig.causative}) and the passive \forme{a-} can be analyzed as having historically originated from the transitive \forme{nɯ-} (§\ref{sec:denom.tr.nW}), the causative/instrumental \forme{sɯ(ɣ)-}  (§\ref{sec:denom.sW.caus.instr}) and the stative \forme{a-} (§\ref{sec:denom.a}) denominalizations from an action nominal, as illustrated in (\ref{ex:pathways.denom}).

 \begin{exe}
 \ex \label{ex:pathways.denom}
  \begin{xlist}
\ex \label{ex:nWGmu.pathway}
\glt  \japhug{mu}{be afraid} $\rightarrow$ \japhug{tɯmu}{fear} $\rightarrow$ \japhug{nɯɣmu}{be afraid of}  
\ex \label{ex:sWGYat.pathway}
\glt  \japhug{ɲat}{be tired} $\rightarrow$ \japhug{tɤɣɲat}{tiredness} $\rightarrow$ \japhug{sɯɣɲat}{exhaust}, `cause to be tired'  
\ex \label{ex:atshoR.pathway}
\glt  \japhug{tsʰoʁ}{attach}$\rightarrow$ \forme{*tɯ-tsʰoʁ} `attachment'\footnote{This hypothesized action nominal is indirectly attested as the bare infinitive in (\ref{ex:bare.inf.noun}) (§\ref{sec:bare.inf.complement}). } $\rightarrow$ \japhug{atsʰoʁ}{be attached}
\end{xlist}
\end{exe}

The reduplicated reciprocal (§\ref{sec:redp.reciprocal}) originates from a stative denominal verb with reduplication expressing plural subject (as in Stau, where not only reduplication but even triplication is attested with this meaning, \citealt{gates17triplication}), from which the reciprocal function was a pragmatic inference.

While the applicative, like the antipassive, is restricted to the Core Gyalrong languages and is certainly a recent innovation, the situation is more complex in the case of the passive and the causative.

The \forme{a-} passive, which originates from \forme{*ŋa\trt}, is probably related to the valency-decreasing \forme{ŋV-} prefix found in Kuki-Chin \citep{jacques07passif}, which also expresses reciprocal, reflexive and even antipassive in some cases (\citealt[203--209]{hartmann09grammar}, \citealt[57]{mang06kcho}, \citealt{konnerth21inverse}). The sigmatic causative, which has highly complex morphology across Gyalrongic languages (\citealt{jackson07shangzhai, lai16caus}), is also widespread in the Trans-Himalayan, and has been discussed in a considerable number of works, including \citet{conrady1896}, \citet{sagart12sprefix}, \citet{mei12caus}, \citet{handel12valence} and \citet{jacques15causative}.

The existence of the cognates of the valency-changing prefixes  \forme{a-} and \forme{sɯ(ɣ)-} in other branches of Trans-Himalayan could been seen as contradicting the hypothesis that they originate from denominalization prefixes. However, two pieces of evidence suggest that the pathway of reanalysis developed to account for the origin of the antipassive is also applicable to these two prefixes.

First, the sigmatic prefix also has a denominal function in various Trans-Hi\-mal\-ayan languages from Jinghpo (\citealt[72]{dai92yufa}, \citealt[88]{kurabe16jinghpo}) to Old Chinese \citep{conrady1896}. There is consensus among scholars that this function is at least equally as old as the causative one.

Second, the irregular \forme{sɤ-} allomorph of the causative in \japhug{sɤrmi}{give a name} (§\ref{sec:sig.caus.irregular.other}) can be accounted for by analyzing this transitive verb as a causative denominal (§\ref{sec:denom.sW.caus.instr}) from the noun  \japhug{tɤ-rmi}{name} rather than a direct causative derivation from the verb \japhug{rmi}{be called}.  
 
It is therefore possible either that the postulated pathways in (\ref{ex:sWGYat.pathway}) and (\ref{ex:atshoR.pathway}) above have taken place independently in several branches, or that the reanalysis had already been completed in the ancestral language ancestral. These hypotheses do not imply that the reanalysis took place only once in each language. In all languages that have preserved the sigmatic denominal and that have some form of bare nominalization, pathways similar to (\ref{ex:sWGYat.pathway}) can have occurred repeatedly even after the sigmatic causative had been fully grammaticalized.

\section{Loan verbs} \label{sec:zh.loanverbs}
While Tibetan verbs are generally borrowed directly \citep{jacques19contact}, Chinese verbs and adjectives need to undergo denominal derivation to be compatible with verbal morphology.

Speakers born after 1990 profusely use verbs of Chinese origin.\footnote{Due to the high variability of the pronunciation of Chinese loanwords in Japhug, I make no attempt here at representing them in phonological shape, and use pinyin except in highly lexicalized examples. } Chinese adjectives can be borrowed almost freely with either the \forme{nɯ-} (§\ref{sec:denom.intr.nW}) or the \forme{rɯ-} (§\ref{sec:denom.intr.rA}) denominal prefixes, as illustrated by \forme{nɯ<yan>} `be strict' (from \ch{严}{yán}{severe, strict}) in (\ref{ex:WtWnWyan}) and \forme{rɯ<fuza>} `be complicated' (from \ch{复杂}{fùzá}{complicated}) in (\ref{ex:YWrWfuza}). The \forme{nɯ-} and \forme{rɯ-} prefixes are apparently freely interchangeable in this type of words.

\begin{exe}
\ex \label{ex:WtWnWyan}
\gll ɯʑo ɯ-tɯ-nɯ-<yan> ɯ-grɤl mu pɯ-me \\
\textsc{3sg} \textsc{3sg}.\textsc{poss}-\textsc{nmlz}:\textsc{deg}-\textsc{denom}-strict \textsc{3sg}.\textsc{poss}-order at.all \textsc{pst}.\textsc{ipfv}-not.exist \\
\glt `He was extremely strict.' (phurpa 2010)
\end{exe}

\begin{exe}
\ex \label{ex:YWrWfuza}
\gll  ɲɯ-rɯ-<fuza> wo \\
\textsc{sens}-\textsc{denom}-complicated \textsc{sfp} \\
\glt `It looks complicated!' (heard in context)
\end{exe}

For action verbs, the choice of the \forme{rɯ-} and \forme{nɯ-} prefixes is conditioned by transitivity (§\ref{sec:denom.rA.pairing}): \forme{rɯ-} denominal verbs are intransitive, for example \japhug{rɯkʰɤjxwi}{have a meeting} (from \ch{开会}{kāihuì}{have a meeting}), while their \forme{nɯ-} counterparts are transitive (\japhug{nɯkʰɤjxwi}{meet about}). In some cases a light verb construction with the Chinese verb borrowed as a noun also exists; for instance, in addition to \japhug{rɯkʰɤjxwi}{have a meeting}, it it possible to use \forme{kʰɤjxwi} `meeting' in collocation with \japhug{βzu}{make} as in (\ref{ex:kaihui}). 

\begin{exe}
\ex \label{ex:kaihui}
\gll kʰɤjxwi ɲɯ-ɤsɯ-βzu-nɯ \\
meeting \textsc{sens}-\textsc{prog}-make-\textsc{pl} \\
\glt `They were having a meeting.' (17-lhazgron)
\end{exe}


Borrowed verbs, although they have undergone denominal derivation, sometimes preserve morphosyntactic peculiarities of the corresponding verb in Chinese. For instance, although the manipulation verb \japhug{nɯtʰaj}{carry}, like \ch{抬}{tái}{lift, carry}, specifically means `lift up and carry (something heavy, of more than one person)' with a constraint on the number of the subject (see example \ref{ex:WXcAl.Zo.natsWmnW}, §\ref{sec:manipulation.verbs}).

\section{Compound verbs} \label{sec:denom.compound.verbs}

% aɕoʁri ɕe + ari `come and go'
%rɟumtɕɤr
%
Compounds verbs combine two verb roots (henceforth referred to as `$V_1$' and `$V_2$' following the Kirantological tradition) within the same stem. Unlike bipartite verbs (§\ref{sec:bipartite}, \citealt{jacques18bipartite}), the two verb roots are not separable and cannot take redundant person indexation affixes. 

Most compound verbs have a denominal prefix (\tabref{tab:compound.verbs.denom}). In some cases, the $V_1$ preserves the form of the independent verb, but in other cases occurs in bound state (§\ref{sec:status.constructus}), as in \japhug{apɤmbat}{be easy to do}, where the root of the base verb \japhug{pa}{do} undergoes \ipa{-a} $\rightarrow$ \ipa{-ɤ} vowel alternation. Alternations in the $V_2$, as in the case of \japhug{axtɕɯxte}{be of uneven size} (with \ipa{xti} $\rightarrow$ \ipa{xte} alternation, §\ref{sec:denom.a}), are much rarer.


\begin{table}
\caption{Denominal compound verbs in Japhug} \label{tab:compound.verbs.denom} 
\begin{tabular}{lllll}
\lsptoprule
Compound verb &$V_1$ & $V_2$ \\
\midrule
\japhug{rɤjoʁβzɯr}{tidy up} (vt) & \japhug{joʁ}{raise} (vt) & \japhug{βzɯr}{move} (vt)  \\ 
\japhug{nɤcɯpa}{open and close} (vt)&
\japhug{cɯ}{open} (vt) & \japhug{pa}{close}, `do' (vt)  \\ 
\japhug{axtɕɯxte}{be of uneven size}  & \japhug{xtɕi}{be small} (vi) & \japhug{wxti}{be big} (vi) \\
\tablevspace
\forme{argɤle} `be extremely & \japhug{rga}{be happy} (vi) & \forme{=le} \\ 
happy' (vi)  &&\\
\tablevspace
\japhug{andʑɤmstu}{well-ironed} (vi) & \japhug{ndʑɤm}{be warm} (vi) & \japhug{astu}{be straight} (vi) \\ 
\japhug{apɤmbat}{be easy to do} (vi) & \japhug{pa}{do} (vt) & \japhug{mbat}{be easy} (vi) \\ 
\japhug{nɤrtoχpjɤt}{observe} (vt) & \japhug{rtoʁ}{look} (vt) & \japhug{χpjɤt}{observe} (vt) \\ 
\forme{nɤscɤlɤt} `take somewhere & \japhug{sco}{see off} (vt) & \japhug{lɤt}{release}, `get so.\\
and back home' (vt)&& back home' (vt)  \\ 
\forme{nɤtsɯmɣɯt} `take away and & \japhug{tsɯm}{take away} (vt) & \japhug{ɣɯt}{bring} (vt) \\ 
bring back' (vt)\\
\forme{nɯndzɤmbɣom} `be in a  & \japhug{ndza}{eat} (vt) & \japhug{mbɣom}{be in a hurry}  \\ 
hurry to eat' (vi)&&(vi) \\
\forme{nɯndzɤqɤr} `not let & \japhug{ndza}{eat} (vt) & \japhug{qɤr}{choose} (vt) \\ 
eat together' (vt)  && \\
\japhug{nɯrkorlɯt}{be obstinate} (vi) & \japhug{rko}{be hard} (vi) & \japhug{arlɯt}{be many} (vi)\\ 
\forme{nɯrŋgɯmbri} `make noise  & \japhug{rŋgɯ}{lie down} (vi) & \japhug{mbri}{cry, sing} (vi) \\ 
in the bed' (vi) && \\
\japhug{raχtɯtsɣe}{do business} (vi) & \japhug{χtɯ}{buy} (vt) & \japhug{ntsɣe}{sell} (vt) \\ 
\lspbottomrule
\end{tabular}
\end{table}

Some compound verbs clearly derive from a compound noun by denominal derivation. This is the case of stative verbs derived from compound nouns of dimension such as \japhug{axtɕɯxte}{be of uneven size} (\tabref{tab:denom.dimension}, §\ref{sec:denom.a}, §\ref{sec.v.v.compounds.degree}), and also  \japhug{rɤjoʁβzɯr}{tidy up}, whose corresponding compound action noun \japhug{joʁβzɯr}{tidying up} is used in collocation with the light verb \japhug{βzu}{make} (§\ref{sec.v.v.compounds.action}).

The verb \japhug{argɤle}{be extremely happy} does not derive from a compound noun \forme{*rgɤle}, but rather from the bipartite verb \forme{rga=le}, which only occurs in non-finite forms (§\ref{sec:bipartite}).

The other verbs in \tabref{tab:compound.verbs.denom} lack a corresponding compound noun. For instance, although \japhug{apɤmbat}{be easy to do} presumably derives from a noun \forme{*pɤmbat} `easiness to do' rather than directly from \japhug{pa}{do} and \japhug{mbat}{be easy}, there is no such noun in Kamnyu Japhug. It is probable that such a noun used to exist, and that only its derived denominal verb was preserved. 

A clue that denominal compounds do not directly derive from their base verbs is offered by \japhug{raχtɯtsɣe}{do business}. The $V_2$ \forme{-tsɣe} has the same form as that found in the action noun \japhug{tɯtsɣe}{commerce} and the antipassive \japhug{rɤtsɣe}{do business}, `sell things' (§\ref{sec:antipassive.rA}), while the corresponding transitive verb \japhug{ntsɣe}{sell} has an additional \forme{n-} prefix. The explanation for the absence of \forme{n-} in \forme{raχtɯtsɣe} is that this verb comes from a compound \forme{*χtɯtsɣe} `commerce' directly built from the action noun \forme{tɯtsɣe} rather than from the transitive verb \japhug{ntsɣe}{sell}. Although \forme{*χtɯtsɣe} is not in common usage, it is considered to be marginally acceptable by Tshendzin.

Some compound verbs lack denominal prefixes, as shown in \tabref{tab:compound.verbs.n.denom}. However, the fact that the $V_1$ occurs in bound state in the case of \japhug{ngɤjtsʰi}{feed} (\forme{ŋgɤ-} from \japhug{ngu}{feed}) and that all the $V_1$ of all of these verbs have a prenasalized onset may suggest a denominal origin: since the denominal \forme{nɯ-} prefix has an irregular \forme{n-} or homorganic \forme{N-} allomorph (§\ref{sec:denom.intr.nW}, §\ref{sec:denom.tr.nW}), it is possible that the denominal prefix here was absorbed by the preexisting prenasalization of the $V_1$: \forme{*n-ŋgɤ-ɕtʰi} $\rightarrow$ \forme{*ŋgɤ-ɕtʰi} $\rightarrow$ \forme{ŋgɤjtsʰi}.

\begin{table}
\caption{Compound verbs without denominal prefix} \label{tab:compound.verbs.n.denom} 
\begin{tabular}{lllll}
\lsptoprule
Compound verb &$V_1$ & $V_2$ \\
\midrule
\forme{mpɯmnu} `be soft & \japhug{mpɯ}{be soft} (vi) & \japhug{mnu}{be smooth} (vi) \\ 
and smooth' (vi) &&\\
\forme{mtsɯrɕpaʁ} `be hungry  & \japhug{mtsɯr}{be hungry} (vi) & \japhug{ɕpaʁ}{be thirsty} (vi) \\ 
and thirsty' (vi) &&\\
\tablevspace
\japhug{ngɤjtsʰi}{feed} (vt) & \japhug{ngu}{feed} (vt) & \japhug{jtsʰi}{give to drink} (vt) \\ 
\forme{mbijtsʰi} `give to eat  & \japhug{mbi}{give} (vt) & \japhug{jtsʰi}{give to drink} (vt) \\ 
and drink' (vt) && \\
\lspbottomrule
\end{tabular}
\end{table}

The semantic relationship between the $V_1$ and the $V_2$ differs across compound verbs. In most cases, the compound has an additive meaning `do $V_1$ and $V_2$', as in \japhug{nɤtsɯmɣɯt}{take away and bring back} (in particular, all verbs in \tabref{tab:compound.verbs.n.denom} are of this type). Another possibility is a head-complement relationship, as that illustrated by \japhug{apɤmbat}{be easy to do}, whose meaning is equivalent to a construction with \japhug{mbat}{be easy} taking a complement clause containing the verb  \japhug{pa}{do} (§\ref{sec:facilitative}).


\section{Incorporation}  \label{sec:incorporation}
Japhug has few dozen complex verb stems comprising a nominal and a verbal root. Nearly all of these verbs  contain a denominal prefix, and incorporation is thus analyzed in Japhug as a subtype of denominal derivation \citep{jacques12incorp}.

Although noun incorporation is not a frequent phenomenon in Japhug, its productivity is undeniable, as it applies to loanwords, as shown by the verb \forme{ɣɯ<piaozi>fsoʁ} `earn money' (\ref{ex:kWGWpiaozifsoR}), whose base noun \ch{票子}{piàozi}{ticket}, `paper money' is from Chinese and whose base verb \japhug{fsoʁ}{accumulate} comes from Tibetan \tibet{བསོགས་}{bsogs}{accumulate}.

\begin{exe}
\ex \label{ex:kWGWpiaozifsoR}
\gll nɤ-mbro cʰo nɤ-rŋɯl tu-rke-a tɕe kɯ-ɣɯ-<piaozi>-fsoʁ jɤ-ɕe tɕe \\
\textsc{2sg}.\textsc{poss}-horse \textsc{comit} \textsc{2sg}.\textsc{poss}-silver \textsc{ipfv}-put.in[III]-\textsc{1sg} \textsc{lnk} \textsc{sbj}:\textsc{pcp}-\textsc{denom}-money-earn \textsc{imp}-go \textsc{lnk} \\
\glt `I will give you a horse and some silver, go and earn some money.' (Lobzang 2005)
\end{exe}

\subsection{Incorporation and denominal derivation} \label{sec:incorp.denom}
A considerable proportion of incorporating verbs in Japhug are denominal derivations from Noun-Verb action nominal compounds (§\ref{sec:object.verb.compounds}, §\ref{sec:adjunct.verb.compounds}) which are still synchronically attested. \tabref{tabe:incorp.denom.compounds} presents a sample of incorporating verbs with the corresponding action nouns, as well as the base nouns and base verbs. These action nominals also occur in collocation with light verbs, with meanings similar to those of the incorporating verbs (§\ref{sec:denominal.vs.light.verb}, §\ref{sec:incorp.vs.other}).\footnote{Japhug (pseudo-)incorporating verbs are typologically similar to the type of \grec{οἰκοδομέω} `build' in Greek \citep{benveniste66incorp}. 
}

 
 \begin{table}
 \caption{Examples of incorporating verbs from noun-verb compounds} \label{tabe:incorp.denom.compounds}
\begin{tabular}{lllll}
\lsptoprule
Incorporating verb & Noun & Verb \\
\midrule
\forme{ɣɯ-cʰɤ-tsʰi} &			\japhug{cʰa}{alcohol} &		\japhug{tsʰi}{drink} (vt) 		\\
`drink alcohol' (vi)& \multicolumn{2}{l}{$\Leftarrow$\japhug{cʰɤtsʰi}{alcohol drinking} }		\\
\tablevspace
\forme{ɣɯ-ɣlɯ-tɕɤt} &			\japhug{tɯ-ɣli}{dung} &		\japhug{tɕɤt}{take out} (vt) 		\\
`take out dung' (vi)& \multicolumn{2}{l}{$\Leftarrow$\japhug{ɣlɯtɕɤt}{dung collecting} (out of the stables) }		\\
\tablevspace
\forme{ɣɯ-cɯ-pʰɯt} &			\japhug{cɯ}{stone} &		\japhug{pʰɯt}{take out, cut} (vt) 		\\
`take out stones' (vi)& \multicolumn{2}{l}{$\Leftarrow$\japhug{cɯpʰɯt}{stone clearing} (out of the fields) }		\\
\tablevspace
\forme{ɣɯ-kʰɯ-tsʰoʁ} &			\japhug{kʰɯna}{dog} &		\japhug{tsʰoʁ}{attach} (vt) 		\\
`hunt with dogs' (vi)& \multicolumn{2}{l}{$\Leftarrow$\japhug{kʰɯtsʰoʁ}{hunting with dogs}  }		\\
\tablevspace
\forme{ɣɯ-rɟɯ-fsoʁ} &			\japhug{tɯ-rɟɯ}{fortune} &		\japhug{fsoʁ}{accumulate} (vt) &	\\
`earn a fortune' (vi)& \multicolumn{2}{l}{$\Leftarrow$\japhug{rɟɯfsoʁ}{earning money}  }		\\
\tablevspace
\forme{ɣɯ-sɯ-pʰɯt} &			\japhug{si}{wood} &		\japhug{pʰɯt}{take out, cut} (vt) 		\\
`cut firewood' (vi)& \multicolumn{2}{l}{$\Leftarrow$\japhug{sɯpʰɯt}{firewood cutting}  }		\\
\tablevspace
\forme{ɣɯ-tʂɤm-tsʰi} &			\japhug{tʂu}{road} &		\japhug{mtsʰi}{lead} (vt) 	\\
`lead the way' (vi)& \multicolumn{2}{l}{$\Leftarrow$\japhug{tʂɤmtsʰi}{leading the way}  }		\\
\tablevspace
\forme{nɯ-zgrɯ-tɕʰɯ} &			\japhug{tɯ-zgrɯ}{elbow} &		\japhug{tɕʰɯ}{gore, stab} (vt) \\
`give a nudge' (vt)& \multicolumn{2}{l}{$\Leftarrow$\japhug{zgrɯtɕʰɯ}{nudge}  }		\\
\tablevspace
\forme{nɤ-kɤ-tɕʰɯ} &			\japhug{tɯ-ku}{head} &		\japhug{tɕʰɯ}{gore, stab} (vt) \\
`give a headbutt' (vt)& \multicolumn{2}{l}{$\Leftarrow$\japhug{kɤtɕʰɯ}{headbutt} }		\\
\tablevspace
\forme{nɯ-snɯ-ɲaʁ} &			\japhug{tɯ-sni}{heart} &		\japhug{ɲaʁ}{be black} (vi) \\
`harm' (vt)& \multicolumn{2}{l}{$\Leftarrow$\japhug{snɯɲaʁ}{harming people}  }		\\
\tablevspace
\forme{nɤ-pʰɯ-xtsɯ}  & \japhug{tɤ-pʰɯ}{clod (of earth)} & \japhug{xtsɯ}{pound} (vt) \\
`break clods of earth' (vi)& \multicolumn{2}{l}{$\Leftarrow$\japhug{tɤpʰɯxtsɯ}{breaking clods of earth}  }		\\
\tablevspace
\forme{nɤ-qʰa-ru} &			\japhug{ɯ-qʰu}{after, behind} &		\japhug{ru}{look at} (vi) &		\\
`look back' (vi)& \multicolumn{2}{l}{$\Leftarrow$\japhug{qʰaru}{look back}  }		\\
\lspbottomrule
\end{tabular}
\end{table}

When the incorporating verb is transitive, its direct object corresponds in some cases to an oblique argument in the light verb construction. For instance,  the verb \japhug{nɯzgrɯtɕʰɯ}{give a nudge} encodes the patient (the person receiving the nudge) as object (\ref{ex:tAwGnWwgrWtChWa}), while in the light verb construction with the compound \japhug{zgrɯtɕʰɯ}{nudge}, the patient is marked by the relator noun \japhug{ɯ-taʁ}{on, above} (\ref{ex:zgrWtChW.matAtWlAt}) (§\ref{sec:WtaR}).
 
\begin{exe}
\ex \label{ex:tAwGnWwgrWtChWa}
\gll  tɤ́-wɣ-nɯ-zgrɯ-tɕʰɯ-a \\
\textsc{aor}-\textsc{inv}-\textsc{denom}-elbow-stab-\textsc{1sg} \\
\glt `He gave me a nudge.' (elicited)
 \ex \label{ex:zgrWtChW.matAtWlAt}
\gll  a-taʁ nɤ-zgrɯ-tɕʰɯ ma-tɤ-tɯ-lɤt ma ɲɯ-mŋɤm  \\
  \textsc{1sg}.\textsc{poss}-on \textsc{2sg}.\textsc{poss}-elbow-stab \textsc{neg}-\textsc{imp}-2-release \textsc{lnk} \textsc{sens}-hurt \\
\glt `Don't give me a nudge, it hurts.' (elicited)
\end{exe}

Alternatively, the patientive argument can be encoded as a possessive prefix on the base nominal compound. For instance, the object of the \japhug{nɯsnɯɲaʁ}{harm} (\textsc{2sg} in \ref{ex:mAtnWsnWYAR}) refers to the person who is harmed, and corresponds to the prefix on the compound noun 
\japhug{snɯɲaʁ}{harming people} in the light verb construction (\textsc{3pl} in \ref{ex:nWsnWYaR}).
 
\begin{exe}
\ex \label{ex:mAtnWsnWYAR}
\gll ma-nɯ-tɯ-mu tɕe aʑo  mɤ-ta-nɯ-snɯ-ɲaʁ \\
\textsc{neg}-\textsc{imp}-2-be.afraid \textsc{lnk} \textsc{1sg} \textsc{neg}-1\fl{}2-\textsc{denom}-heart-black:\textsc{fact} \\
 \glt `Don't be afraid, I will not do any harm to you.' (140429 jiedi-zh)
\end{exe}

\begin{exe}
\ex \label{ex:nWsnWYaR}
\gll   tɯrme ra nɯ-snɯ-ɲaʁ ɲɯ-ɤsɯ-βzu tɕe \\
people \textsc{pl} \textsc{3pl}.\textsc{poss}-heart-black \textsc{sens}-\textsc{prog}-make \textsc{lnk} \\
 \glt `It was harming people.' (150827 taisui)
 \japhdoi{0006390\#S30}
 \end{exe}  
  
\tabref{tabe:incorp.denom.compounds} only includes examples with the denominal prefixes \forme{ɣɯ-} (§\ref{sec:denom.GW}) and \forme{nɯ-/nɤ-} (§\ref{sec:denom.nW}), but other denominal prefixes are attested in this construction. In particular, the alternative forms \japhug{rɯrɟɯfsoʁ}{accumulate a fortune} (with the rhotic denominal, §\ref{sec:denom.rA}) and  \japhug{sɯzgrɯtɕʰɯ}{give a nudge} (with the instrumental sigmatic denominal, §\ref{sec:denom.sW.caus.instr}) are attested alongside \japhug{ɣɯrɟɯfsoʁ}{accumulate a fortune}  and \japhug{nɯzgrɯtɕʰɯ}{give a nudge}.
 
This use of the term `incorporation' to refer to this construction is debatable, since unlike prototypical incorporation, which originates from noun-verb coalescence \citep{mithun84incorp}, in the case of Japhug a two-step process has to be posited: (i) action nominal compounding followed by (ii) denominal derivation. However, this derivation process is not isolated cross-linguistically: similar constructions are found in Ancient Greek \citep{benveniste66incorp} and may be posited for proto-Algonquian (\citealt{garrett04stem.structure,jacques12incorp}). 
 
Not all incorporating verbs derive transparently from Noun+Verb compounds. In spite of the presence of a denominal prefix on the verb, the expected action nominal is not always attested. For instance, the intransitive verb \japhug{nɯjlɤlɤɣ}{herd hybrid yaks } (from \japhug{jla}{male hybrid yak}  and \japhug{lɤɣ}{graze} ) might have been derived from a noun such as \forme{*jlɤ-lɤɣ}, but no such noun exists anymore.


In some cases, denominal derivations are derived directly from a Noun+Verb collocation. \tabref{tab:incorp.denom.collocation} presents examples of denominal verbs in \forme{nɯ-/nɤ-} and \forme{sɤ-} coming from lexicalized complex collocations. In these constructions, the base verb is intransitive, and the base noun its subject (§\ref{sec:incorp.S}). Some of the nouns and/or verbs are orphan lexemes (§\ref{sec:orphan.noun}, §\ref{sec:orphan.verb}), and are not attested as free elements.

 \begin{table}
 \caption{Incorporating verbs from noun-verb collocations} \label{tab:incorp.denom.collocation}
\begin{tabular}{lllll}
\lsptoprule
Incorporating verb  & base noun & Noun+verb collocation \\
\midrule
\japhug{nɯɲɤmkʰe}{be skinny} (vi) &			\japhug{ɯ-ɲɤm}{flesh} &		\forme{ɯ-ɲɤm+kʰe} `be skinny'		\\
\japhug{nɯɲɤmsɯ}{be plump} (vi) &			\japhug{ɯ-ɲɤm}{flesh} &  \forme{ɯ-ɲɤm+sɯ} `be plump'		\\
\japhug{sɤʑiloʁ}{be disgusting} (vi) &			\japhug{tɯ-ʑi}{nausea} &		\forme{tɯ-ʑi+loʁ} `have nausea'			\\
\japhug{nɤʑiloʁ}{have nausea} (vi) &			\japhug{tɯ-ʑi}{nausea} &	 \\
\japhug{sɤmbrɯŋgɯ}{be detestable}  &			\japhug{tɤ-mbrɯ}{anger} &		\forme{tɤ-mbrɯ+ŋgɯ} `be angry' &		\\
(vi)&&\\
\japhug{nɤrɕɤmŋɤm}{cherish} (vt) &			&\forme{ɯ-rɕa+mŋɤm} `cherish'	\\
\japhug{sɤʁombi}{be discouraging},   &			&\japhug{ɯ-ʁo+mbi}{be discouraged},  	\\
`be hopeless' (vi) & &§\ref{sec:anticausative.collocation}\\
\japhug{nɤʁombi}{lose hope} (vt) &			& 	\\
\japhug{nɯsroʁmbrɤt}{be in agony} & \japhug{tɯ-sroʁ}{life} & \japhug{mbrɤt}{break}, `be cut' \\
&&§\ref{sec:anticausative} \\
\lspbottomrule
\end{tabular}
\end{table}

Three types of denominal derivations are found in \tabref{tab:incorp.denom.collocation}, all independently attested with simple nouns. First, we find stative intransitive verbs in \forme{nɯ-} (§\ref{sec:denom.intr.nW}), whose intransitive subject corresponds to the possessor of the base noun as in the case of \japhug{nɯɲɤmkʰe}{be skinny} (compare \ref{ex:YWtWnWYAmkhe} and \ref{ex:nAYAm.YWkhe}).

\begin{exe}
\ex 
\begin{xlist}
\ex \label{ex:YWtWnWYAmkhe}
\gll ɲɯ-tɯ-nɯ-ɲɤm-kʰe \\
\textsc{sens}-2-\textsc{denom}-flesh-be.skinny \\
\glt `You are skinny.' (heard in context)
\ex \label{ex:nAYAm.YWkhe}
\gll  nɤ-ɲɤm ɲɯ-kʰe \\
\textsc{2sg}.\textsc{poss}-flesh \textsc{sens}-be.skinny \\
\glt `You are skinny.' (140517 mogui de jing zh)
\japhdoi{0004022\#S22}
\end{xlist}
\end{exe}

 

Second, some verbs make use of the  \forme{sɤ-} proprietive denominal prefix (§\ref{sec:denom.sA.proprietive}). The verb \japhug{sɤʑiloʁ}{be disgusting} is paired with the \forme{nɤ-} denominal \japhug{nɤʑiloʁ}{have nausea} (other verb pairs of the same type are found in \tabref{tab:sA.denom.proprietive}), while \japhug{sɤmbrɯŋgɯ}{be detestable} instead corresponds to the simple denominal \japhug{sɤmbrɯ}{get angry} (§\ref{sec:denom.sA.proprietive}).

Third, the transitive verbs \japhug{nɤrɕɤmŋɤm}{cherish}  and \japhug{nɤʁombi}{lose hope} have a tropative meaning (§\ref{sec:denom.tr.nW}). Their transitive subjects corresponds to the possessor of the base noun, as shown by (\ref{ex:YWnArCAmNama}) and (\ref{ex:arCa.mNAm}).

\begin{exe}
\ex 
\begin{xlist}
\ex \label{ex:YWnArCAmNama}
\gll a-tɕɯ ɲɯ-nɤ-rɕɤ-mŋam-a \\
\textsc{1sg}.\textsc{poss}-son \textsc{sens}-\textsc{denom}-cherish(1)-cherish(2)-\textsc{1sg} \\
\glt `I cherish my son.' (elicited)
\ex \label{ex:arCa.mNAm}
\gll  a-tɕɯ a-rɕa mŋɤm \\
\textsc{1sg}.\textsc{poss}-son  \textsc{1sg}.\textsc{poss}-cherish(1)  cherish(2):\textsc{fact}   \\
\glt `I cherish my son.' (elicited)
\end{xlist}
\end{exe}

Other incorporating verbs probably originate from a Noun+Cerb collocation which does not exist anymore. From instance, \japhug{amɲaχtsʰɯm}{be petty} (\ref{ex:YWtAmYaXtshWm}) derives from \japhug{tɯ-mɲaʁ}{eye} and \japhug{xtsʰɯm}{be thin} with the denominal prefix \forme{a\trt}, but no collocation \forme{*tɯ-mɲaʁ+xtsʰɯm} exists, though the etymological relationship with the base noun and verb is still synchronically transparent.

\begin{exe}
\ex \label{ex:YWtAmYaXtshWm}
\gll ɲɯ-tɯ-ɤ-mɲaʁ-tsʰɯm \\
\textsc{sens}-2-\textsc{denom}-eye-be.thin \\
\glt `You are petty-minded.' (elicited)
\end{exe}



There are also a handful of incorporating verbs without a dedicated denominal prefix, as illustrated by \tabref{tabe:incorp.n.denom}.  

 \begin{table}
 \caption{Incorporating verbs without dedicated denominal prefix} \label{tabe:incorp.n.denom}
\begin{tabular}{lllll}
\lsptoprule
Incorporating verb  & base noun & base verb \\
\midrule
\japhug{amɤʁu}{have rickets} (vi) &			\japhug{tɯ-mi}{foot, leg} &		\japhug{ajʁu}{be bowed} (vi) &		\\	
\japhug{akɤmtɕoʁ}{be pointy-headed} (vi) &			\japhug{tɯ-ku}{head} &		\japhug{amtɕoʁ}{be pointy} (vi) &		\\
\tablevspace
 \japhug{kɤtɯpa}{tell} (vt) & \forme{kɤ-ti} `the thing   & \japhug{pa}{do} (vt) &\\
 &that is said' \\
\lspbottomrule
\end{tabular}
\end{table}
 
In the case of \japhug{amɤʁu}{have rickets} and \japhug{akɤmtɕoʁ}{be pointy-headed}, the nominal stems \forme{mɤ\trt}, \forme{kɤ-} and \forme{sɯ-} in bound state appear to be infixed within the stem of the base verbs \japhug{ajʁu}{be bent} and \japhug{amtɕoʁ}{be pointy}. The stem \forme{a<mɤ>ʁu} in addition lacks the \forme{-j-} preinitial found in \forme{ajʁu}. 

Infixation is not the only possibility to account for these two verbs however, since the initial \forme{a-} syllables themselves are probably frozen denominal prefixes  (see also §\ref{sec:a.non.passive.denominal}). An alternative  hypothesis is that both the apparent base verbs and the incorporating verbs are derived. The intransitive verb \japhug{nɯsɯzʁe}{transport wood} provides a possible model. This compound verb, which comes from the compounding of \japhug{si}{wood} (with vowel alternation \forme{sɯ\trt}, §\ref{sec:vowel.alternations.compounds}) and the root of \japhug{nɯzʁe}{transport}, at first glance seems to be a case of noun infixation within the stem \forme{nɯzʁe}. However, the existence of the action nominal  \japhug{sɯzʁe}{firewood transportation} shows that \japhug{nɯsɯzʁe}{transport wood} is a trivial example of denominal incorporation like the verbs in \tabref{tabe:incorp.denom.compounds} above. The anomaly here is the verb \japhug{nɯzʁe}{transport}, whose \forme{nɯ-} prefix must be secondary -- a likely explanation for this prefix is that \forme{nɯzʁe} is a denominal derivation from a lost action nominal \forme{*tɯ-zʁe} `transportation'. 


   \begin{figure}
   \caption{The derivational history of \japhug{nɯsɯzʁe}{transport wood} } \label{fig:nWsWzRe}  
  \begin{tikzpicture}
  \node (si) at (0,2) {\japhug{si}{wood} };
 \node (zRe) at (0,0) {\forme{*tɯ-zʁe}  };
  \node (sWzRe) at (4,1) {\japhug{sɯzʁe}{firewood transportation}   };
    \node (nWzRe) at (2,-1) {\japhug{nɯzʁe}{transport}  };
    \node (nWsWzRe) at (7,0) {\japhug{nɯsɯzʁe}{transport wood}   };
\tikzstyle{sur}=[->,very thick,>=latex]
\draw[sur] (si)--(sWzRe);
\draw[sur] (zRe)--(sWzRe);
\draw[sur] (zRe)--(nWzRe);
\draw[sur] (sWzRe)--(nWsWzRe);
\end{tikzpicture}
\end{figure}

Hence, as illustrated in \figref{fig:nWsWzRe}, \japhug{nɯsɯzʁe}{transport wood} is only indirectly related to \japhug{nɯzʁe}{transport}, and the infixation hypothesis is certainly wrong in this case. A scenario in the same lines is possible for \japhug{amɤʁu}{have rickets} and \japhug{akɤmtɕoʁ}{be pointy-headed}.

The only example of nominal incorporation without a denominal prefix is the defective verb \japhug{kɤtɯpa}{tell}  (its conjugation is presented in \tabref{tab:kAtWpa}, §\ref{sec:irregular.transitive}), which comes from the compounding of the object participle \forme{kɤ-ti} `the thing that is said' (§\ref{sec:object.participle}) with \japhug{pa}{do}. Non-denominal incorporating verbs are much more common in Khroskyabs (\citealt[388--411]{lai17khroskyabs}) than in core Gyalrong languages.

\subsection{Incorporation and other derivations} \label{sec:incorp.other}
Like other denominal verbs, incorporating verbs undergo productive voice derivations. In some cases, only the derived verb exists: for instance, the intransitive \japhug{anɯrŋɤrɯru}{look at each other's face} has the form of a reduplicated reciprocal (§\ref{sec:redp.reciprocal}) from a transitive verb \forme{*nɯ-rŋɤ-ru} `look at $X$'s face' (from \japhug{tɯ-rŋa}{face} and \japhug{ru}{look at}, §\ref{sec:incorp.goal}), but no such base verb exists at least in the Kamnyu dialect of Japhug (§\ref{sec:redp.lexicalized}).
 
\subsection{Syntactic function of the incorporated word} \label{sec:incorp.noun.function}
The incorporated nouns have four syntactic functions: intransitive subject, object/semi-object, goal/location adjunct and instrument. There are no examples of incorporated transitive subjects in Japhug. 

\subsubsection{Incorporation of intransitive subject} \label{sec:incorp.S}
Intransitive subject incorporation is found in verbs deriving from noun+in\-tran\-si\-tive verb collocations such as those listed in \tabref{tab:incorp.denom.collocation} above such as \japhug{nɯsroʁmbrɤt}{be in agony} from \japhug{tɯ-sroʁ}{life} and \japhug{mbrɤt}{break}, `be cut'. None of these verbs has a dummy intransitive subject: the incorporated subject does not saturate the subject function. Rather, either the possessor of the base noun (see examples  \ref{ex:YWtWnWYAmkhe} and \ref{ex:nAYAm.YWkhe} above in §\ref{sec:incorp.denom}) or an external referent is promoted to intransitive subject status (or to object status in the case of transitive verbs such as \japhug{nɤrɕɤmŋɤm}{cherish}).

There are no clear examples of subject incorporation among verbs deriving from noun+verb action nominals.

\subsubsection{Incorporation of object} \label{sec:incorp.O}
Object incorporation is found in some incorporating verbs that derive from noun+ verb action nominal compounds such as those presented in \tabref{tabe:incorp.denom.compounds} (§\ref{sec:incorp.denom}) and others which probably originate from lost compounds. It is a saturating incorporation, which removes the object from the argument structure of the verb: incorporating verbs of this type are all intransitive, while their base verbs are all transitive. Typical examples include \japhug{ɣɯcʰɤtsʰi}{drink alcohol}  (from \japhug{cʰa}{alcohol} and	\japhug{tsʰi}{drink}) or \japhug{nɤpʰɯxtsɯ}{break clods of earth}  (from the stem of \japhug{tɤ-pʰɯ}{clod} (of earth) with \japhug{xtsɯ}{pound}) (§\ref{sec:incorp.vs.other}).

%\japhug{nɯʁndomnɤt}{repeat the same words} (vi) &			\japhug{taʁndo}{word} &		\japhug{mnɤt}{repeat} (vt) &		\\

Some lexicalized noun+verb collocations have an incorporating form. For instance, the transitive verb \forme{ru}, which only\footnote{This verb must (at least synchronically) be distinguished from two homophonous verbs: the intransitive \japhug{ru}{look at} (§\ref{sec:orienting.verbs}) and the transitive \japhug{ru}{fetch, bring} (which requires an associated motion prefix, §\ref{sec:ru.fetch}). } occurs with the noun \japhug{zrɯɣ}{louse} with the meaning `pick lice off', can also incorporate the same noun to yield \japhug{nɯzrɯɣru}{pick lice off}.


The semi-transitive verb \japhug{rga}{like} (§\ref{sec:semi.transitive}) can incorporate a semi-object, as in \japhug{nɯɕmɯrga}{be talkative}, `like to talk a lot' (with the bound state \forme{ɕmɯ-} of the inalienably possessed noun \japhug{tɯ-ɕmi}{word}) and \japhug{nɯcʰɤrga}{like to drink alcohol} (with \forme{cʰɤ-} from \japhug{cʰa}{alcohol}).

 
\subsubsection{Incorporation of goal/locational adjunct} \label{sec:incorp.goal}
Goals or locational adjuncts can also be incorporated. Unlike objects (§\ref{sec:incorp.O}), they do not saturate the object function of the verb and the incorporating verb has the same transitivity as that of the base verb. For instance, the verb \japhug{nɤqʰɤŋga}{put on} (of clothes worn on the shoulders on the top of other clothes) is transitive as is the base verb \japhug{ŋga}{wear} and selects as object the garment that is worn, typically a raincoat  (see \ref{ex:mboR.maR.tWwWr}, §\ref{sec:disjunction.nouns}). The incorporated noun \japhug{ɯ-qʰu}{after, behind} (in bound state \forme{qʰɤ-}) here indicates the location where the piece of garment is worn (`wear on the back').

However, goal argument saturation occurs when the base verb requires a goal argument. For instance, \japhug{ru}{look at} selects a goal or dative arguments (§\ref{sec:intr.goal}), but the incorporated verb \japhug{nɤqʰaru}{look back}, which contains the locational noun \japhug{ɯ-qʰu}{after, behind} (with the bound form \forme{qʰa-} rather than \forme{qʰɤ-}), is strictly intransitive.

The intransitive verb \japhug{nɯkʰɤrŋgɯ}{lie down to rest} provides an example of locational adjunct incorporation.  It derives from the noun \japhug{kʰa}{house}  (\forme{kʰɤ-}) and the intransitive verb \japhug{rŋgɯ}{lie down}, `sleep'. The original sense of this verb was probably `lie down (somewhere) in the house' or perhaps even more specifically  `lie down in the dining room' (the term \japhug{kʰɤjmu}{dining room} contains the bound state \forme{kʰɤ-} as first element), but it has now lexicalized further, meaning `lie down to rest in a casual way' in a place unfit for this purpose and not necessarily within the house (as illustrated by \ref{ex:lonWkhArNgW}).

\begin{exe}
\ex \label{ex:lonWkhArNgW}
\gll  smɤt tɯmda rɟɤlpu nɯra ɣɯ nɯ-sakaβ tʂu nɯtɕu ɕ-kɤ-rɤʑi ɲɯ-ŋu.  lo-nɯ-kʰɤ-rŋgɯ kɯ-fse ndɤre \\
pl.n pl.n king \textsc{dem}:\textsc{pl} \textsc{gen} \textsc{3pl}.\textsc{poss}-well path \textsc{dem}:\textsc{loc} \textsc{tral}-\textsc{aor}-stay \textsc{sens}-be \textsc{aor}-\textsc{denom}-house-lie.down \textsc{sbj}:\textsc{pcp}-be.like \textsc{lnk} \\
\glt `He went to the water well of the king of Smad, and stayed there. He lay down on the ground in a casual way.' (2005-stod-kunbzang)
\end{exe}

\subsubsection{Incorporation of instrument} \label{sec:incorp.instr}
Instrument incorporation, like object incorporation (§\ref{sec:incorp.O}), is found in denominal verbs deriving from action nominal compounds (\tabref{tabe:incorp.denom.compounds}). Like goal and locational adjunct incorporation (§\ref{sec:incorp.goal}), it does not saturate the object function: the verb \japhug{nɯzgrɯtɕʰɯ}{give a nudge} from \japhug{tɯ-zgrɯ}{elbow} and \japhug{tɕʰɯ}{gore, stab} is transitive like its base verb, as shown by example (\ref{ex:tAwGnWwgrWtChWa}) (§\ref{sec:incorp.denom} above).

Some instrument incorporating verbs are lexicalized, for instance \forme{nɯ-rmbɯ-χtɕi} from \japhug{tɯ-rmbi}{urine} (regular bound state \forme{rmbɯ-}) and \japhug{χtɕi}{wash} does not mean `wash with urine'; \forme{χtɕi} is to be understood here as `drench' (as in `drenched in the rain' in \ref{ex:nAZo.rcanW}, §\ref{sec:unexpected}), and although \forme{nɯ-rmbɯ-χtɕi} probably originally meant `piss on', its present meaning is instead `spray a liquid on', in particular of a species of ants (\ref{ex:tunWrmbWXtCi}).

\begin{exe}
\ex \label{ex:tunWrmbWXtCi}
\gll tɕe nɯtɕu tɕe qro nɯ ɯ-mɲaʁ ɯ-ŋgɯ ra ku-ɕe, ɯ-mɲaʁ nɯ ku-mtsɯɣ nɤ tu-nɯ-rmbɯ-χtɕi nɯra tɕe, \\
\textsc{lnk} \textsc{dem}:\textsc{loc} \textsc{lnk} ant \textsc{dem} \textsc{3sg}.\textsc{poss}-eye \textsc{3sg}.\textsc{poss}-in \textsc{pl} \textsc{ipfv}:\textsc{east}-go \textsc{3sg}.\textsc{poss}-eye \textsc{dem} \textsc{ipfv}-bite \textsc{add} \textsc{ipfv}-\textsc{denom}-urine-wash \textsc{dem}:\textsc{pl} \textsc{lnk}  \\
\glt `(When the bear comes to eat them), the ants go into its eyes, bite its eyes and spray acid on them.' (26-qro)
\japhdoi{0003682\#S45}
\end{exe}

Instruments other than body parts can be incorporated, for instance \japhug{tɤ-jlɤβ}{steam} in the verb \japhug{nɤjlɤβsqa}{stew} from \japhug{sqa}{cook}. However, this is also a lexicalized incorporating verb, which does not mean `cook with steam' as could have been expected, but rather `stew for a long time'. 

%\japhug{nɯmbrɤrɟɯɣ}{gallop} (vi) &			\japhug{mbro}{horse} &		\japhug{rɟɯɣ}{run} (vi) &		\\


Instrument incorporation is more common with transitive verbs, but examples also exists with intransitive verbs, for instance \japhug{nɯmbrɯmtsaʁ}{skip rope} from \japhug{tɯmbri}{rope} (\forme{mbrɯ-}) and \japhug{mtsaʁ}{jump}.


\subsection{Incorporation and other constructions} \label{sec:incorp.vs.other}
Incorporating verbs occur in the same contexts as the corresponding non-com\-poun\-ded syntactic constructions when these exist, both in the case of incorporating verbs from noun+verb collocations such as \japhug{nɯɲɤmkʰe}{be skinny} (\ref{ex:qaZo.kWnWYAmkhe}) (see also \ref{ex:YWtWnWYAmkhe} and \ref{ex:nAYAm.YWkhe} above) and of those from action nominal compounds such as \japhug{nɤpʰɯxtsɯ}{break clods of earth} (\ref{ex:lukWnAphWxtsW}).

\begin{exe}
\ex \label{ex:qaZo.kWnWYAmkhe}
\gll iɕqʰa qaʑo nɯ-ɲɤm kɯ-sɯ nɯra qʰe ɲɯ-nɯɣɯ-krɤɣ, qaʑo kɯ-nɯ-ɲɤm-kʰe nɯra [...] mɯ́j-nɯɣɯ-krɤɣ. \\
the.aforementioned sheep \textsc{3pl}.\textsc{poss}-flesh \textsc{sbj}:\textsc{pcp}-be.plump \textsc{dem}:\textsc{pl} \textsc{lnk} \textsc{sens}-\textsc{facil}-shear sheep \textsc{sbj}:\textsc{pcp}-\textsc{denom}-flesh-be.skinny \textsc{dem}:\textsc{pl} {  } \textsc{neg}:\textsc{sens}-\textsc{facil}-shear \\
\glt `Sheep that are plump are easy to shear, but sheep that are skinny are difficult to shear.' (160712 smAG)
\japhdoi{0006073\#S27}
\end{exe}

\begin{exe}
\ex \label{ex:lukWnAphWxtsW}
\gll maka lu-kɯ-nɤpʰɯxtsɯ tɕe tɕe rcanɯ, tɤ-pʰɯ lú-wɣ-xtsɯ tɕe, \\
at.all \textsc{ipfv}-\textsc{genr}:S/O-\textsc{denom}-clod-pound \textsc{lnk} \textsc{lnk} \textsc{unexp}:\textsc{deg} \textsc{indef}.\textsc{poss}-clod \textsc{ipfv}-\textsc{inv}-pound \textsc{lnk} \\
\glt `When people break clods of earths, ...' (26-mYaRmtsaR)
\japhdoi{0003674\#S82}
\end{exe}

Nevertheless, these constructions are not completely equivalent. The minimal triplet in (\ref{ex:cWphWt}) can be used to illustrate some of the differences between them.

\begin{exe}
\ex \label{ex:cWphWt}
\begin{xlist}[(ii)]
\exi{(i)} 
\gll  cɯ-pʰɯt nɯ-βzu-t-a  \\
  stone-take.off \textsc{aor}-do-\textsc{pst}:\textsc{tr}-\textsc{1sg} \\
\exi{(ii)} 
\gll nɯ-ɣɯ-cɯ-pʰɯt-a \\
\textsc{aor}-\textsc{denom}-stone-take.off-\textsc{1sg} \\
\exi{(iii)} 
\gll cɯ nɯ-pʰɯt-t-a  \\
  stone \textsc{aor}-take.out-\textsc{pst}:\textsc{tr}-\textsc{1sg} \\
\end{xlist}
\glt `I cleared the stones [from the field].' (elicitation)
\end{exe}   

The non-compounded construction (iii) is the only one that can occur if the argument is referential or takes a determiner (unlike languages like Hopi which allow determiners to have scope over an incorporated noun, see \citealt{hill.kc03hopi, haugen08incorp}). In the light verb construction with the action nominal compound \japhug{cɯpʰɯt}{taking stones} (out of the fields, before ploughing) (i) and the incorporating verb \japhug{ɣɯcɯpʰɯt}{take out stones} (ii), the bound nominal element \forme{cɯ-} cannot be used to refer to specific stones that have been previously mentioned. A similar constraint is observed with other denominal verbs (§\ref{sec:denominal.vs.light.verb}). The light verb construction (i) is used to highlight that an action takes a long time or effort, or occurs many times in this particular example. 

The usage differences between (i)--(iii) in (\ref{ex:cWphWt}) however are not generalizable to all incorporating verbs, in particular because the existence of an incorporating verb does not imply that the constructions in (i) and (iii) also exist, and because compounding is not always compositional. The triplet of constructions in (\ref{ex:qharu2}) with \japhug{nɤqʰaru}{look back} (§\ref{sec:incorp.goal}), the compound \japhug{qʰaru}{look back} (§\ref{sec:incorp.denom}) and the non-compounded construction with  \japhug{ɯ-qʰu}{after, behind} and \japhug{ru}{look at}, illustrate a different situation.

\begin{exe}
\ex \label{ex:qharu2}
\begin{xlist}[(ii)]
\exi{(i)} \label{ex:qharu.mWpalAt}
\gll  ɯʑo nɯ, tatpa ta-ta ma qʰaru mucin ʑo mɯ-pa-lɤt nɤ tɤ-ari ɲɯ-ŋu. \\
\textsc{3sg} \textsc{dem} faith \textsc{aor}:3\flobv{}-put \textsc{lnk} look.back at.all \textsc{emph} \textsc{neg}-\textsc{aor}:3\flobv{}-release \textsc{add} \textsc{aor}:\textsc{up}-go[II] \textsc{sens}-be \\
\glt `He kept faith and did not look back at all and went up (did not fall down and reached the heavens).' (2005 Norbzang)
\exi{(ii)} 
\gll  tsʰoŋχpɯn nɤrɯβzaŋ ɯʑo mɯma ʑo pɯ-nɤ-qʰa-ru-nɯ ɲɯ-ŋu, kɯ\redp{}kɯ-tu ʑo pɯ-atɤr-nɯ ɲɯ-ŋu, \\
\textsc{anthr}  \textsc{anthr} \textsc{3sg} apart.from \textsc{emph} \textsc{aor}-\textsc{denom}-\textsc{back}-look-\textsc{pl} \textsc{sens}-be \textsc{total}\redp{}\textsc{sbj}:\textsc{pcp}-exist \textsc{emph} \textsc{aor}-fall-\textsc{pl} \textsc{sens}-be \\
\glt `Apart from Tshong dpon Norbzang, they looked back, and all of them fell down.' (2005 Norbzang)
\exi{(iii.a)} \label{ex:Wqhu.luru}
\gll ɯ-ʁrɯ nɯ ki tu-fse qʰe  ɯ-qʰu ri lu-ru,  \\
\textsc{3sg}.\textsc{poss}-horn \textsc{dem} \textsc{dem}.\textsc{prox} \textsc{ipfv}-be.like \textsc{lnk} \textsc{3sg}.\textsc{poss}-behind \textsc{loc} \textsc{ipfv}:\textsc{upstream}-look \\
\glt `Its horn is turned towards its back like that.' (20-RmbroN) 
\japhdoi{0003560\#S99}
\exi{(iii.b)} \label{ex:si.Wqhu.loru}
\gll tɕʰeme nɯ lo-ɕe tɕe si ɯ-qʰu nɯtɕu lo-ru ri, \\
girl \textsc{dem} \textsc{ifr}:\textsc{upstream}-go \textsc{lnk} tree \textsc{3sg}.\textsc{poss}-behind \textsc{dem}:\textsc{loc} \textsc{ifr}:\textsc{upstream}-look \textsc{lnk} \\
\glt (150901 changfamei-zh)
\glt `The girl went there and looked behind the tree.'(150901 changfamei-zh)
\japhdoi{0006352\#S207}
\exi{(iii.c)} \label{ex:ndZiqhu.koru}
\gll jo-ɕqʰlɤt-ndʑi tɕe, ndʑi-qʰu jo-ru ma nɯ ma ɯ-kɤpa pjɤ-me. \\
\textsc{ifr}-disappear-\textsc{du} \textsc{lnk} \textsc{3du}.\textsc{poss}-behind \textsc{ifr}-look \textsc{lnk} \textsc{dem} apart.from \textsc{3sg}.\textsc{poss}-manner \textsc{ifr}.\textsc{ipfv}-not.exist \\
\glt `The two them disappeared, and the only thing she could do was looking at them from behind (at their back as they were going away).' (140506 shizi he huichang de bailingniao-zh)
\japhdoi{0003927\#S208}
\end{xlist}
\end{exe}

The examples above show that the collocation of \japhug{ɯ-qʰu}{after, behind} and \japhug{ru}{look at} is semantically quite different from the compound  \japhug{qʰaru}{look back}  and its corresponding denominal incorporating verb: its range of meanings includes `be turned behind, towards the back' (iii.a), `look behind $X$' (iii.b) or `look at $X$ from behind' (iii.c). The light verb construction (i) and the incorporating verb (ii) by contrast can only be used in the sense of `look back'. Unlike in (\ref{ex:cWphWt}), the light verb construction (i) is not used to express either protracted or repeated action since  \japhug{qʰaru}{look back} is semelfactive, but rather  occurs in (\ref{ex:qharu.mWpalAt}) to put more emphasis on the negation of the action (he did not look back in the slightest).


The contrast between the three constructions in (\ref{ex:cWphWt}) and (\ref{ex:qharu2}) also has to do with the relative frequency of the verbal forms in which they appear. In particular, an important proportion of incorporating verbs in the corpus are subject participles (as in \ref{ex:qaZo.kWnWYAmkhe} above) or generic forms (as in \ref{ex:lukWnAphWxtsW}).


