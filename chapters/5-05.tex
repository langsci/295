\chapter{Degree and comparison} \label{chap:degree}
 
\section{Absolute degree and intensifiers} \label{sec:absolute.degree}
This section describes the constructions available to express the absolute degree of a property of a referent, without standard of comparison.

All degree constructions can be used with the emphatic marker \forme{ʑo}, either following a degree adverb, or in sentence-final position (§\ref{sec:postverbal.adv}).

\subsection{Degree adverbs} \label{sec:degree.adverbs}
Degree adverbs in Japhug are intensifiers meaning `really, much, very, a lot'. There is also a comparative degree adverb \japhug{mɤʑɯ}{even more} (treated in §\ref{sec:mAZW} below) and a superlative adverb \japhug{stu}{most} (§\ref{sec:stu.superlative}). There are no adverbs indicating a degree above or under a limit such as `too much' or `not enough'; the degree nominal construction must be used instead to express this meaning (§\ref{sec:degree.monoclausal}).


\subsubsection{\japhug{wuma}{real, really}} \label{sec:wuma}
The most common intensifier in Japhug is \forme{wuma} `really', from Tibetan \tibet{ངོ་མ་}{ŋo.ma}{real, true}. While marginally attested as a postnominal attribute meaning `real' (§\ref{ex:attributive.postnominal}), its most widespread function is to serve as clausal intensifier, with (\ref{ex:wuma.Zo.YWNgWtCi}) or without (\ref{ex:wuma.YWNgWtCi}) the emphatic marker \forme{ʑo}.

\begin{exe}
\ex 
\begin{xlist}
\ex \label{ex:wuma.Zo.YWNgWtCi}
\gll nɯra ɲɯ-fse tɕe wuma ʑo ɲɯ-ŋgɯ-tɕi \\
\textsc{dem}:\textsc{pl} \textsc{sens}-be.like \textsc{lnk} really \textsc{emph}  \textsc{sens}-be.poor-\textsc{1du} \\
\glt `We are very poor like that.' (divination, 14)
\ex \label{ex:wuma.YWNgWtCi}
\gll  a-mu cʰo wuma ɲɯ-ŋgɯ-tɕi  \\
\textsc{1sg}.\textsc{poss}-mother \textsc{comit} \textsc{really} \textsc{sens}-be.poor-\textsc{1du} \\
\glt `My mother and I are very poor.' (divination, 25)
\end{xlist}
\end{exe}

The intensifier is not necessarily adjacent to the verb, and some constituents can be inserted, for instance absolutive nouns in essive function (§\ref{sec:essive.abs}) such as \forme{tɯrme} `as a person' in (\ref{ex:wuma.Zo.tWrme.pe}).

\begin{exe}
\ex \label{ex:wuma.Zo.tWrme.pe}
\gll ɯ-χti nɯ wuma ʑo tɯrme pe ma \\
\textsc{3sg}.\textsc{poss}-companion \textsc{dem} really \textsc{emph} person be.good:\textsc{fact} \textsc{lnk} \\
\glt `Her husband is a very nice person.' (`he is very nice as a person.') (14-siblings, 350)
\end{exe}

The flexibility in the position of \forme{wuma} can be useful to distinguish head-internal relatives from postnominal ones (§\ref{ex:attributive.participles.stative.verbs}, §\ref{sec:head-internal.relative}, §\ref{sec:intr.subj.head.noun.position}).

The intensifier \forme{wuma} is not restricted to adjectival stative verbs. It can be used with tropative verbs (§\ref{sec:tropative}) as in (\ref{ex:wuma.YWnAmNAm}), but also modal verbs such as \japhug{cʰa}{can} (§\ref{sec:cha.verb}) and action verbs that do not have an intrinsic degree parameter such as \japhug{ndza}{eat} (\ref{ex:wuma.Zo.ndze}).

\begin{exe}
\ex \label{ex:wuma.YWnAmNAm}
\gll  tɕe nɯ tɤ́-wɣ-tɕɤt tɕe nɯŋa nɯ kɯ wuma ɲɯ-nɤ-mŋɤm  \\
\textsc{lnk} \textsc{dem} \textsc{aor}-\textsc{inv}-take.out \textsc{lnk} cow \textsc{dem} \textsc{erg} really \textsc{sens}-\textsc{trop}-hurt  \\
\glt `When one removes (the parasite by squeezing it out), the cow finds it very painful.' (25-zrW, 28)
\end{exe}

\begin{exe}
\ex \label{ex:wuma.Zo.ndze}
\gll  maka tɤɕi qaj nɯra wuma ʑo ndze. \\
at.all barley wheat \textsc{dem}:\textsc{pl} really \textsc{emph} eat[III]:\textsc{fact} \\
\glt `[The dove] eats wheat and barley a lot.' (22-CAGpGa, 30)
\end{exe}

When \japhug{cʰa}{can} takes an infinite complement, \forme{wuma} generally follows the complement clause, as in (\ref{ex:wuma.Zo.chaa}), whereas with finite complements it is generally located inside of the complement clause (\ref{ex:wuma.tusApe.cha}).

\begin{exe}
\ex \label{ex:wuma.Zo.chaa}
\gll [si kɤ-pʰaʁ] \textbf{wuma} ʑo cʰa-a \\
tree \textsc{inf}-chop really \textsc{emph} can:\textsc{fact}-\textsc{1sg} \\
\glt `I am very good at felling trees.' (2011-10-qajdo, 3)
\end{exe}

\begin{exe}
\ex \label{ex:wuma.tusApe.cha}
\gll tɯ-ji ɯ-ŋgɯ zɯ tsʰɤt ɯ-ɣli nɯ cʰɯ́-wɣ-lɤt tɕe, [tɤ-rɤku \textbf{wuma} ʑo tu-sɤpe] cʰa \\
\textsc{indef}.\textsc{poss}-field \textsc{3sg}.\textsc{poss}-in \textsc{loc} goat \textsc{3sg}.\textsc{poss}-manure \textsc{dem} \textsc{ipfv}-\textsc{inv}-release \textsc{lnk} \textsc{indef}.\textsc{poss}-crops really \textsc{emph} \textsc{ipfv}-do.well can:\textsc{fact} \\
\glt `When one puts goat manure in fields, it can do a lot of good to the crops.' (05-qaZo, 34)
\end{exe}

The intensifier \forme{wuma} can be combined with \japhug{stʰɯci}{as much} (§\ref{sec:denominal.adverb.s.prefix}).  

\begin{exe}
\ex \label{ex:wuma.sthWci}
\gll nɯ fsapaʁ ra kɯ ndza-nɯ ri, wuma stʰɯci mɤ-rga-nɯ. \\
\textsc{dem} animal \textsc{pl} \textsc{erg} eat:\textsc{fact}-\textsc{pl} \textsc{lnk} really so.much \textsc{neg}-like:\textsc{fact}-\textsc{pl} \\
\glt `The animals eat it, but don't like it so much.' (12-Zmbroko, 100)
\end{exe}
 
\subsubsection{\japhug{kʰro}{much}} \label{sec:khro}
The adverbs \japhug{kʰro}{much} and \japhug{ʑimkʰɤm}{much} can serve as intensifiers of noun phrases (§\ref{sec:nominal.intensifier}) but can also have scope over the whole clause, in pre-verbal position. Unlike \forme{wuma} and \forme{nɯstʰɯci/kɯstʰɯci} (§\ref{sec:nWtshWci}), they are more often used with action verbs, and can both refer to the quantity and intensity of actions (`much', `a lot') or to the time taken by the action (`for a long time'), as in (\ref{ex:khro.Zo.pjANke}) and (\ref{ex:khro.Zo.YACar}), respectively.

\begin{exe}
\ex \label{ex:khro.Zo.pjANke}
\gll kʰro ʑo pjɤ-ŋke pjɤ-ra tɕendɤre, pjɤ-nɯʑɯβ. \\
much \textsc{emph} \textsc{ifr}.\textsc{ipfv}-walk \textsc{ifr}.\textsc{ipfv}-be.needed \textsc{lnk} \textsc{ifr}-sleep \\
\glt `He had had to walk a lot (on that day, was very tired), and fell asleep.' (140430 yufu he tade qizi-zh, 236)
\end{exe}

\begin{exe}
\ex \label{ex:khro.Zo.YACar}
\gll kʰro ʑo ɲɤ-ɕar ri mɯ-pjɤ-mto \\
much \textsc{emph} \textsc{ifr}-search \textsc{lnk} \textsc{neg}-\textsc{ifr}-see \\
\glt `[The prince] looked for her for a long time, but could not find her.' (140504 huiguniang-zh, 190)
\end{exe}

Both intensifiers can be reduplicated as \forme{kʰɯ\redp{}kʰro} and \forme{ʑɯ\redp{}ʑimkʰɤm} as in (\ref{ex:ZWZWmkhAm.Zo.YAnAthWthu}).

\begin{exe} 
\ex \label{ex:ZWZWmkhAm.Zo.YAnAthWthu}
\gll ʑɯ\redp{}ʑimkʰɤm ʑo aʁɤndɯndɤt ʑo ɲɤ-nɤtʰɯtʰu \\
\textsc{emph}\redp{}much \textsc{emph} everywhere \textsc{emph} \textsc{ifr}-\textsc{distr}:ask \\
\glt `[The bear] asked around [about the rabbit] everywhere for a long time.' (2011-13-qala, 20)
\end{exe}

The adverb \forme{ʑimkʰɤm} comes from \tibet{ཞིང་ཁམས་}{ʑiŋ.kʰams}{country, universe} (also borrowed as the noun \japhug{ʑiŋkʰɤm}{country, realm}, sometimes also pronounced \forme{ʑimkʰɤm}), and its grammaticalization as an intensifier perhaps went through a semantic change `universe' $\Rightarrow$ `in the whole universe, universally' $\Rightarrow$ `everywhere' $\Rightarrow$ `for a long time; much'.

\subsubsection{Demonstrative+\japhug{stʰɯci}{as much}} \label{sec:nWtshWci}
The combinations of the adverbs \japhug{stʰɯci}{as much} and \japhug{stʰamtɕɤt}{as much} (§\ref{sec:denominal.adverb.s.prefix}) occur with anaphoric demonstratives (§\ref{sec:demonstrative.pronouns}) to express the meaning `so (much)'.

The demonstratives can be in free form, but can also merge with \forme{stʰɯci} and \forme{stʰamtɕɤt} into the degree adverbs \japhug{kɯstʰɯci}{this much} (\ref{ex:ri.kWsthWci.WWmpCar}), \japhug{nɯstʰɯci}{that much}, \japhug{kɯstʰamtɕɤt}{this much} and \japhug{nɯstʰamtɕɤt}{that much} (\ref{ex:nWsthamtCAt.pjAmpCAr}).


 \begin{exe}
\ex \label{ex:ri.kWsthWci.WWmpCar}
 \gll  a-rʑaβ ri kɯstʰɯci ɲɯ-mpɕɤr, a-mbro ri kɯstʰɯci ɲɯ-ʑru, a-pɣɤtɕɯ ri kɯstʰɯci ɲɯ-mpɕɤr  \\
 \textsc{1sg}.\textsc{poss}-wife also so.much \textsc{sens}-be.beautiful  \textsc{1sg}.\textsc{poss}-horse also so.much \textsc{sens}-be.strong  \textsc{1sg}.\textsc{poss}-bird also so.much \textsc{sens}-be.beautiful  \\
 \glt `My wife is so beautiful, my horse so strong, my bird so beautiful.' (2003qachga, 116)
 \end{exe}

These degree adverbs can directly precede the verb as in (\ref{ex:ri.kWsthWci.WWmpCar}), but can also occur before the subject as in (\ref{ex:nWsthamtCAt.pjAmpCAr}). 

 \begin{exe}
\ex \label{ex:nWsthamtCAt.pjAmpCAr}
 \gll tʰɯ-kɯ-ndʐaβ nɯ to-ndzur qʰe, li pjɤ-rɟaʁ. nɯstʰamtɕɤt ʑo iɕqʰa ɯ-ɟuli nɯ pjɤ-mpɕɤr. \\
 \textsc{aor}-\textsc{sbj}:\textsc{pcp}-\textsc{acaus}:cause.to.roll \textsc{dem} \textsc{ifr}-stand \textsc{lnk} again \textsc{ifr}-dance that.much \textsc{emph} the.aforementioned \textsc{3sg}.\textsc{poss}-flute \textsc{dem} \textsc{ifr}.\textsc{ipfv}-be.beautiful \\
 \glt `(The rich man$_i$ started dancing under the influence of the music andfell down). The one$_i$ who had fallen down stood up and danced again. This was the extent to which [the shepherd boy's] flute [music] was beautiful.'  (`his flute music was \textit{that} beautiful') (140513 mutong de disheng-zh, 163-164)
 \end{exe}
 
 The demonstrative can anaphorically refer to a previous entity or a previous clause, as in (\ref{ex:nWsthamtCAt.pjAmpCAr}), where it is focalized.\footnote{Although from a text translated from Chinese, this sentence was added by Tshendzin and is not in the original.  }
%nɯsthɯci ɲɯ-tɯ-kʰe ɲɯ-maʁ
% hist160708_riquet5.txt



%was grammaticalized from the  secutive relator noun \japhug{ɯ-rca}{following} (§\ref{sec:secutive})

\subsubsection{Unexpected degree} \label{sec:unexpected} 
The unexpected/high degree marker \forme{rcanɯ} or \forme{rca} indicates that the situation or action described by the predicate that follows is unexpected (\ref{ex:nAZo.rcanW}), intensifies to a noticeable (and not foreseeable) extent (\ref{ex:tokAnWmqajndZic.tCe.rcanW}) or occurs with a remarkably high degree or intensity, with  (\ref{ex:mbro.rcanW}) or without (\ref{ex:apWme.rcanW}) surprise.


\begin{exe}
	\ex \label{ex:nAZo.rcanW}
	\gll  wo nɤʑo rcanɯ tɕʰi ɲɯ-tɯ-nɤme ŋu ma,  aʑo tɯ-mɯ kɯ pɯ-kɯ-sɯ-χtɕi-a, tɤndʐo nɯ! \\
	\textsc{interj} \textsc{2sg} \textsc{unexp}:\textsc{deg} what \textsc{sens}-2-do[III] be:\textsc{fact} \textsc{lnk} \textsc{1sg} \textsc{indef}.\textsc{poss}-sky \textsc{erg} \textsc{aor}-2\fl{}1-\textsc{caus}-wash-\textsc{1sg} cold \textsc{sfp} \\
	\glt `You, what are you doing, you caused me to be drenched by the rain.' (kWlAG 2014, 157) \\
\end{exe}

\begin{exe}
	\ex \label{ex:tokAnWmqajndZic.tCe.rcanW}
	\gll to-k-ɤnɯmqaj-ndʑi-ci tɕe rcanɯ, ʑɯrɯʑɤri tɕe ko-k-ɤndɯndo-ndʑi-ci, \\
	\textsc{ifr}-\textsc{peg}-\textsc{recip}:scold-\textsc{du}-\textsc{evd} \textsc{lnk}  \textsc{unexp}:\textsc{deg} progressively \textsc{lnk}   \textsc{ifr}-\textsc{peg}-\textsc{recip}:take-\textsc{du}-\textsc{peg} \\
	\glt `They scolded each other and progressively started to fight, ' (lWlu2002, 52)
\end{exe}

\begin{exe}
	\ex \label{ex:mbro.rcanW}
	\gll mbro rcanɯ ɯ-xɕɤt kɯ-tɯ\redp{}tu ʑo nɯ-ntsʰɤr ɲɯ-nu, \\
	horse \textsc{unexp}:\textsc{deg} \textsc{3sg}.\textsc{poss}-strength \textsc{sbj}:\textsc{pcp}-\textsc{emph}\redp{}exist \textsc{emph} \textsc{aor}-neigh \textsc{sens}-be \\ 
	\glt `The horse neighed with all his strength.' (qachGa2003, 158)
\end{exe}

The adverb \forme{rcanɯ} is particularly common in the degree construction with a \forme{tɯ-} degree nominal (§\ref{sec:degree.nominals}), as in (\ref{ex:apWme.rcanW}). In this particular construction,  \forme{rcanɯ} does not necessarily express unexpectedness.

\begin{exe}
	\ex \label{ex:apWme.rcanW}
	\gll  tɕe nɯnɯ lɯlu a-pɯ-me rcanɯ, βʑɯ ɯ-tɯ-ŋɤn saχaʁ. \\
	\textsc{lnk} \textsc{dem} cat \textsc{irr}-\textsc{ipfv}-not.exist \textsc{unexp}:\textsc{deg} mouse
	\textsc{3sg}.\textsc{poss}-\textsc{nmlz}:\textsc{deg}-be.evil be.extremely:\textsc{fact} \\ 
	\glt `If there are no cats, the mice are extremely fierce (cause a lot of damages).' (21-lWlu, 32) 
\end{exe}

In addition to its function as a degree adverb,  \forme{rcanɯ} also contributes to information structure: it marks the constituent preceding it as a topic, and the part of the sentence following it as a focus.
 
\subsubsection{Emphatic} \label{sec:emphatic.Zo} 
The emphatic marker \forme{ʑo} is one of the most common words in Japhug. It is never obligatory, but frequently occurs after all the intensifiers described above. 

It is one of the few adverbs that can follow the main verb (§\ref{sec:postverbal.adv}, §\ref{sec:relative.postverbal}), in particular with the stative verb \japhug{saχaʁ}{be extremely} as in (§\ref{ex:pjasaXaR.Zo}).

\begin{exe}
	\ex \label{ex:pjasaXaR.Zo}
	\gll tɤŋe ɯ-rkɯ tɤ-armbat qʰe ɯ-tɯ-sɤ-ɕke pjɤ-saχaʁ ʑo ri \\
	sun \textsc{3sg}.\textsc{poss}-side \textsc{aor}:\textsc{up}-be.near \textsc{lnk} \textsc{3sg}.\textsc{poss}-\textsc{nmlz}:\textsc{deg}-\textsc{prop}-burn \textsc{ifr}.\textsc{ipfv}-be.extremely \textsc{emph} \textsc{lnk} \\
	\glt `When he approached the sun, it was extremely hot.' (31-deluge, 112)
	\end{exe}
	
It also occurs after ideophones (§\ref{sec:idph.syntax}) and deideophonic verbs (§\ref{sec:voice.deideophonic}).

With personal pronouns, the emphatic \forme{ʑo} can mean `by oneself', as in (\ref{ex:aZo.Zo}) (§\ref{sec:pronouns.emph}).

\begin{exe}
\ex \label{ex:aZo.Zo}
\gll aʑo ʑo nɯnɯ ɕ-pjɯ-sat-a ra \\
\textsc{1sg} \textsc{emph} \textsc{dem} \textsc{tral}-\textsc{ipfv}-kill-\textsc{1sg} be.needed:\textsc{fact} \\
\glt `I have to kill her myself.' (140504 baixuegongzhu, 117)
\end{exe}

It also appears on various types of subordinate clauses, in particular manner clauses (including serial verb constructions §\ref{sec:svc.simultaneous} and converbial clauses §\ref{sec:manner.converbs}, §\ref{sec:gerund.clauses}), some temporal and conditional clauses (§\ref{sec:immediate.converb}, §\ref{sec:universal.concessive.conditional}, §\ref{sec:iterative.coincidence.clause}) and in neutral addition clauses  (§\ref{sec:neutral.addition}).

The emphatic marker may be historically related to the \forme{-ʑo} stem of personal pronouns (§\ref{sec:pers.pronouns}) or, alternatively, be a cognate to the Tibetan particle \tibet{ཡང་}{jaŋ}{also, even}.

\subsection{Degree nominals} \label{sec:degree.nominal.subject}

\subsubsection{Monoclausal degree nominal construction} \label{sec:degree.monoclausal}
An alternative way to indicate the degree of a stative predicate is to use degree nominals (§\ref{sec:degree.nominals}, §\ref{sec:degree.nominal.construction}) as intransitive subjects of degree verbs like \japhug{saχaʁ}{be extremely}, \japhug{sɤre}{be ridiculous}, `be extremely' (\ref{ex:axtu.WtWmNAm}), \japhug{tɕʰom}{be too much} (\ref{ex:natWxtCi.tChom}) or \japhug{rtaʁ}{be enough} (\ref{ex:WtWrnaR.mWjrtaR}). The TAME is expressed on the degree verb, which is always in \textsc{3sg} (§\ref{sec:intransitive.invariable}). Degree nominals can also serve as possessors of the subject \japhug{ɯ-grɤl}{order, rule}  in the collocation \forme{ɯ-grɤl+me} `be extremely' (example \ref{ex:WtWsAGmu.WgrAl.maNe} in §\ref{sec:existential.light.verbs}). 

\begin{exe}
\ex \label{ex:axtu.WtWmNAm}
\gll aʑo kɯnɤ a-xtu ɯ-tɯ-mŋɤm ɲɯ-sɤre ʑo \\
\textsc{1sg} also \textsc{1sg}.\textsc{poss}-belly \textsc{3sg}.\textsc{poss}-\textsc{nmlz}:\textsc{deg}-hurt \textsc{sens}-be.extremely \textsc{emph} \\
\glt `Me too, my belly hurts a lot.' (literally: `the degree of my hurting is ridiculously high') (qala 2002, 17)
\end{exe}

\begin{exe}
\ex \label{ex:WtWrnaR.mWjrtaR}
\gll kɯki tɯ-ci ki ɯ-tɯ-rnaʁ mɯ́j-rtaʁ \\
\textsc{dem}.\textsc{prox} \textsc{indef}.\textsc{poss}-water \textsc{dem}.\textsc{prox} \textsc{3sg}.\textsc{poss}-\textsc{nmlz}:\textsc{deg}-be.deep \textsc{neg}:\textsc{sens}-be.enough \\
\glt `This water is not deep enough.' (2010-03-zh, 4)
\end{exe}

The use of \forme{sɤre} as degree verbs illustrates a semantic change from `funny, ridiculous' to intensifier, similar to that of French \textit{drôlement} (\textit{drôlement difficile}) or of English \textit{ridiculously} (\textit{ridiculously difficult}).

The possessive prefix is coreferent with the entity whose property is referred to by the nominalized verb, for instance \textsc{2sg} in (\ref{ex:natWxtCi.tChom}). 

\begin{exe}
\ex \label{ex:natWxtCi.tChom}
\gll nɤʑo nɤ-tɯ-xtɕi tɕʰom-o \\
\textsc{2sg} \textsc{2sg}.\textsc{poss}-\textsc{nmlz}:\textsc{deg}-be.small be.too.much:\textsc{fact}-\textsc{sfp} \\
\glt `You are too young.' (150828 huamulan-zh, 28)
\end{exe}

Causative forms of degree verbs (§\ref{sec:sig.caus.stative}), for instance \japhug{ɣɤtɕʰom}{cause to be too much}, can also be occur with degree nominals, as in (\ref{ex:WtWCqraR.tAGAtChoma}).

\begin{exe}
\ex \label{ex:WtWCqraR.tAGAtChoma}
\gll kɯki maka ɯ-tɯ-ɕqraʁ tɤ-ɣɤ-tɕʰom-a ɯβrɤ-ŋu ɣe? \\
\textsc{dem}.\textsc{prox} at.all  \textsc{3sg}.\textsc{poss}-\textsc{nmlz}:\textsc{deg}-be.intelligent \textsc{aor}-\textsc{caus}-be.too.much-\textsc{1sg} \textsc{rh}.\textsc{q}-be:\textsc{fact} \textsc{sfp} \\
\glt `I hope that I did not cause her to become too intelligent.' (hist160709 riquet6-v2)
\end{exe}

With \japhug{naχtɕɯɣ}{be the same} as degree verb, the degree construction becomes an equative construction (§\ref{sec:equative}), with the comitative postposition \japhug{cʰo}{and, with} (§\ref{sec:comitative}). 

\begin{exe}
\ex \label{ex:WtWwxti.naXtCWG}
\gll ʑmbri cʰo ɯ-tɯ-wxti naχtɕɯɣ \\
willow \textsc{comit} \textsc{3sg}.\textsc{poss}-\textsc{nmlz}:\textsc{deg}-be.big be.the.same:\textsc{fact} \\
\glt `It is as big as a willow (its degree of ``bigness" is the same as that of a willow).' (08-qaCti, 3)
\end{exe}

The lexicalized infinitive form \forme{kɤ-ti} of the verb \japhug{ti}{say} and the ergative can optionally follow the degree nominal (\ref{ex:ndZitWNgW.kAti.kW}).

\begin{exe}
\ex \label{ex:ndZitWNgW.kAti.kW}
\gll kɤtsa ni ndʑi-tɯ-ŋgɯ kɤ-ti kɯ pɯ-saχaʁ ʑo \\
parent.and.child \textsc{du} \textsc{3du}.\textsc{poss}-\textsc{nmlz}:\textsc{deg}-be.poor \textsc{inf}-say \textsc{erg} \textsc{pst}.\textsc{ipfv}-be.extremely \textsc{emph} \\
\glt `The mother and her child were extremely poor.' (Norbzang 2005, 152)
\end{exe}

Degree nouns are not the only type of nouns that can occur in this construction. Abstract nouns in \forme{tɤ-} (§\ref{sec:tA.abstract.nouns}) such as \japhug{tɤɕpaʁ}{thirst} (\ref{ex:tACpaR.kAti.kW}) or underived abstract nouns such as \japhug{rɤŋom}{outrage} can also be used, though much less commonly.  This construction however differs from the previous one in that the degree verb \forme{saχaʁ} indexes the experiencer as intransitive subject (\textsc{3pl} in \ref{ex:tACpaR.kAti.kW}).
  
\begin{exe}
\ex \label{ex:tACpaR.kAti.kW}
\gll  tɤ-ɕpaʁ kɤ-ti kɯ pɯ-saχaʁ-nɯ ʑo pɯ-rɤʑi-nɯ ɲɯ-ŋu  \\
\textsc{nmlz}:\textsc{abstract}-thirst \textsc{inf}-say \textsc{erg} \textsc{pst}.\textsc{ipfv}-be.extremely-\textsc{pl} \textsc{emph} \textsc{pst}.\textsc{ipfv}-stay-\textsc{pl} \textsc{sens}-be \\
\glt `[The people there] live in extreme thirst.' (divination 2005, 8)
\end{exe}

Example (\ref{ex:tACpaR.kAti.kW}) is in addition a hybrid construction combining the abstract nominal and the degree verb with a serial verb construction of degree (§\ref{sec:degree.svc}).

Without a degree verb, degree nominals and other abstract nouns can be used as predicates to express high degree in exclamation (§\ref{sec:degree.nominal.predicates}), often with the sentence final particle \forme{nɯ}, as in (\ref{ex:nAtWsAjndAt}).

\begin{exe}
\ex \label{ex:nAtWsAjndAt}
\gll nɤ-tɯ-sɤjndɤt nɯ! \\
\textsc{2sg}.\textsc{poss}-\textsc{nmlz}:\textsc{deg}-be.cute \textsc{sfp} \\
\glt `You are so cute!' (heard in context)
\end{exe}



\subsubsection{Consequence degree construction} \label{sec:degree.consequence}
Degree nominals and abstract nouns can also occur with the ergative marker \forme{kɯ} followed by a finite clause describing a consequence or a particular aspect of the high degree reached (`so $X$ that $Y$'; in the following, the clause $X$ is referred to as `degree clause', and $Y$ as `consequence clause'), as in (\ref{ex:WtWnAmbju.kW}). This type of construction is a subtype of causality clause linking (§\ref{sec:causality}).

\begin{exe}
\ex \label{ex:WtWnAmbju.kW}
\gll ɯ-tɯ-nɤmbju kɯ [rɟɤlpu kʰɤxtɤndo kɯ-nɯjaŋsa nɯ ɣɯ ɯ-mɲaʁ na-z-nɤmbju ʑo ɲɯ-ŋu]. \\
\textsc{3sg}.\textsc{poss}-\textsc{nmlz}:\textsc{deg}-be.shiny \textsc{erg} king side.of.the.roof \textsc{sbj}:\textsc{pcp}-be.idle \textsc{dem} \textsc{gen} \textsc{3sg}.\textsc{poss}-eye \textsc{aor}:3\flobv{}-\textsc{caus}-be.shiny \textsc{emph} \textsc{sens}-be\\
\glt `It was so shiny that it dazzled the eyes of the king, who was staying idle on the side of the roof (of the palace).' (Norbzang 2005, 177)
\end{exe}

It is possible for several degree nominals to share a consequence clause. For instance, in (\ref{ex:atAGWzrWG.kW}), the degree nouns \forme{nɯ-tɯ-ɤɣɯzrɯɣ} and \forme{a-tɯ-ɤɣɯzrɯɣ} are followed by only one marker \forme{kɯ} and have a common consequence clause \forme{mɯ́j-cha-a}. Here, the additional clause \forme{ci tu-pe tɕe} `putting together' specifies that it is the conjoined quantity (of lice) referred by the two degree nominals that is the cause of the consequence clause, and therefore implies that the absence of a consequence clause just after the first degree noun \forme{nɯ-tɯ-ɤɣɯzrɯɣ} is not due to ellipsis.

\begin{exe}
\ex \label{ex:atAGWzrWG.kW}
\gll paʁ ra ci nɯ-tɯ-ɤɣɯzrɯɣ, aʑo ci a-tɯ-ɤɣɯzrɯɣ kɯ [ci tu-pe tɕe mɯ́j-cʰa-a] \\
pig \textsc{pl} one \textsc{3sg}.\textsc{poss}-\textsc{nmlz}:\textsc{deg}-have.a.lot.of.lice \textsc{1sg} one \textsc{1sg}.\textsc{poss}-\textsc{nmlz}:\textsc{deg}-have.a.lot.of.lice \textsc{erg} one \textsc{ipfv}-do[III] \textsc{lnk} \textsc{neg}:\textsc{sens}-can-\textsc{1sg} \\
\glt `Pigs have so many lice and I have so many lice that all put together, I can't bear it.' (2005 Kunbzang, 43)
\end{exe}

As in the monoclausal construction, the lexicalized infinitive form \forme{kɤ-ti} of \japhug{ti}{say} can optionally be added before the ergative, as in (\ref{ex:kAti.kW.WCtsxi.tolhoR}) (see also \ref{ex:tWphoNbu.kAndo}, §\ref{sec:ndo.lv}).
 
\begin{exe}
\ex \label{ex:kAti.kW.WCtsxi.tolhoR}
\gll ɯ-tɯ-ɲat kɤ-ti kɯ [ɯ-ɕtʂi ra to-ɬoʁ.] \\
\textsc{3sg}.\textsc{poss}-\textsc{nmlz}:\textsc{deg}-be.tired \textsc{inf}-say \textsc{erg} \textsc{3sg}.\textsc{poss}-sweat \textsc{pl} \textsc{ifr}-come.out \\
\glt `He was so tired that he started sweating (profusely).' (140513 mutong de disheng-zh, 70)
\end{exe}

The same meaning `... is so $X$ that ...' can also be expressed by a finite clause $X$ followed by the relator noun \japhug{ɯ-xɕɤt}{strength} with the ergative \forme{kɯ} (§\ref{sec:IPN.cause}, §\ref{sec:causal.clauses}), as in (\ref{ex:YWmaqhu.WxCAt.kW}). 

\begin{exe}
\ex \label{ex:YWmaqhu.WxCAt.kW}
\gll ɲɯ-maqʰu ɯ-xɕɤt kɯ, [kɯ-fsoʁ tɤ-rɤŋgat ri kóʁmɯz tu-ɬoʁ ŋu.] \\
\textsc{ipfv}-be.after \textsc{3sg}.\textsc{poss}-strength \textsc{erg} \textsc{sbj}:\textsc{pcp}-be.light \textsc{aor}-be.about \textsc{loc} only.then \textsc{ipfv}-come.out be:\textsc{fact} \\
\glt `[This star] is so late that it only comes out when the day is about to break.' (29-mWBZi, 41)
\end{exe}


\subsection{Serial verb construction} \label{sec:degree.svc}
Some stative verbs of degree (in particular \japhug{tɕʰom}{be too much} and \japhug{rtaʁ}{be enough}) can occur in a serial verb construction (§\ref{sec:svc.degree}) with the same meaning as the monoclausal degree construction (§\ref{sec:degree.monoclausal}): examples (\ref{ex:WtWxtCi.YWtChom}) and (\ref{ex:YWxtCi.YWtChom}) are synonymous, though the former is more frequent.

\begin{exe}
\ex 
\begin{xlist}
\ex  \label{ex:WtWxtCi.YWtChom}
\gll ɯ-tɯ-xtɕi ɲɯ-tɕʰom \\
\textsc{3sg}.\textsc{poss}-\textsc{nmlz}:\textsc{deg}-be.small \textsc{sens}-be.too.much \\
\ex  \label{ex:YWxtCi.YWtChom}
\gll ɲɯ-xtɕi ɲɯ-tɕʰom \\
\textsc{sens}-be.small \textsc{sens}-be.too.much \\
\glt `It is too small.' (elicited)
\end{xlist}
\end{exe}

The serial verb construction as a whole can serve as a complement clause, as in (\ref{ex:tudAn.tutChom}) where the constituent between square brackets is the subject of the verb \forme{mɯ́j-pe} `it is not good' (§\ref{sec:adjective.complement}).

\begin{exe}
\ex \label{ex:tudAn.tutChom}
\gll [nɯŋa ra nɯ-taʁ tu-dɤn tu-tɕʰom] mɯ́j-pe ma \\
cow \textsc{pl} \textsc{3pl}.\textsc{poss}-on \textsc{ipfv}-be.many \textsc{ipfv}-be.too.much \textsc{neg}:\textsc{sens}-be.good \textsc{lnk} \\
\glt `It is not good when there are too many (ticks) on cows.' (25-xCelwi, 31)
\end{exe}

\subsection{Exceptive}
A manner clause (§\ref{sec:manner.clauses}) containing the exceptive phrase \forme{nɯ ma} `apart from that' (§\ref{sec:exceptive}) and a negative verb form can convey an extreme degree, as in (\ref{ex:nW.mA.XXX.Zo.YWrgaa}).\footnote{Example (\ref{ex:nW.ma.mWjkhW.Zo}) is a rendering of Chinese \zh{我表示十二分地满意} <wǒ biǎoshì shíèr fēn de mǎnyì>, a sentence that plays on the expression \ch{十分满意}{shífēn mǎnyì}{100\% satisfied}. 
}

\begin{exe}
\ex \label{ex:nW.mA.XXX.Zo.YWrgaa}
\begin{xlist}
\ex \label{ex:nW.ma.mWjkhW.Zo}
\gll  nɯ ma aʑo maka, kɤ-nɯ-rga mɯ́j-kʰɯ ʑo ɲɯ-rga-a \\
\textsc{dem} apart.from \textsc{1sg} at.all \textsc{inf}-\textsc{appl}-like \textsc{neg}:\textsc{sens}-be.possible \textsc{emph} \textsc{sens}-like-\textsc{1sg} \\
\glt `I like it a lot (to the extent that it is not possible to like it more than I do).' (140521 huangdi de xinzhuang-zh, 141)
\ex \label{ex:nW.ma.mAkWkhW.Zo}
\gll  nɯ ma kɤ-nɯ-rga mɤ-kɯ-kʰɯ ʑo ɲɯ-rga-a \\
\textsc{dem} apart.from \textsc{inf}-\textsc{appl}-like \textsc{neg}-\textsc{inf}:\textsc{stat}-be.possible \textsc{emph} \textsc{sens}-like-\textsc{1sg} \\
\glt `I like it a lot.' (elicitation based on \ref{ex:nW.ma.mWjkhW.Zo})
\end{xlist}
\end{exe}

\section{Comparative} \label{sec:comparison}

Adjectival stative verbs in Japhug do not have specific comparative or superlative forms,\footnote{There is  one potential example of a comparative derivation in Japhug between \japhug{sna}{be good, be worthy} and \japhug{mna}{be better} (§\ref{sec:mna.sna}), but in the additional examples, this remains speculative. } and comparison is expressed by means of postpositions, adverbs or verbs marking relative degree.


\subsection{The postpositions \forme{sɤz} and \forme{kɯ}} \label{sec:sAz.kW}
As illustrated in (\ref{ex:comparative1}), it is possible in Japhug to mark both the  standard of comparison (by the postposition \forme{sɤz} `than', §\ref{sec:comparative}), and the comparee (by  the ergative \forme{kɯ} (§\ref{sec:comparee.kW}).\footnote{In this section, the term `standard marker' corresponds to Dixon's (\citeyear{dixon08comparative}) \textsc{mark}. The `comparee marker' could be equated with Dixon's \textsc{index}. } 


\begin{exe}
\ex \label{ex:comparative1}
\glll  ɯ-ʁi sɤz ɯ-pi nɯ {                  } kɯ   mpɕɤr  \\
\textsc{3sg}.\textsc{poss}-younger.sibling \textsc{comp} \textsc{3sg}.\textsc{poss}-elder.sibling \textsc{dem} {   } \textsc{erg} be.beautiful:\textsc{fact} \\
\textsc{standard} \textsc{standard}.\textsc{marker} \textsc{comparee} { } {         } \textsc{comparee}.\textsc{marker} \textsc{parameter} \\
\glt `The elder sibling is more beautiful than the younger sibling.'  (elicited)
\end{exe}


The comparee marker is optional when the standard phrase is overt, as shown by (\ref{ex:aZo.sAz.wxti}), where the comparee \forme{ɯ-rʑaβ} `his wife' is in absolutive form.

\begin{exe}
\ex \label{ex:aZo.sAz.wxti}
\gll ɯ-rʑaβ aʑo sɤz wxti \\
\textsc{3sg}.\textsc{poss}-wife \textsc{1sg} \textsc{comp} be.big:\textsc{fact} \\
\glt `His wife is older than me.' (14-siblings, 200)
\end{exe}

This construction is exclusively used for comparisons of superiority. To express comparison of inferiority, one either has to select the antonym of the parameter, or to reformulate with another construction such as that with egressive postpositions (§\ref{sec:egressive.comparative}).

The standard marker can follow a genitive postpositional phrase, when the entities being compared are possessees, and the second one is elided, as in (\ref{ex:GW.sAz.YWwxti}). The genitive is however optional in this context, as shown by (\ref{ex:mbro.sAz.YWwxti}), when we find \forme{mbro sɤz} instead of expected \forme{mbro ɣɯ sɤz}.

\begin{exe}
\ex \label{ex:GW.sAz.YWwxti}
\gll  ɯ-rna nɯ [mbro ɣɯ] sɤz ɲɯ-wxti. \\
\textsc{3sg}.\textsc{poss}-ear \textsc{dem} horse \textsc{gen} \textsc{comp} \textsc{sens}-be.big \\
\glt `Its ears are bigger than those of the horse.' (20-tArka, 5)
\end{exe}

\begin{exe}
\ex \label{ex:mbro.sAz.YWwxti}
\gll  tɕeri ɯ-pʰoŋbu nɯ mbro sɤz koŋla ʑo ɲɯ-wxti. \\
\textsc{lnk} \textsc{3sg}.\textsc{poss}-body \textsc{dem} horse \textsc{comp} really \textsc{emph} \textsc{sens}-be.big \\
\glt `Its body is bigger than [that of] a horse.' (19-rNamoN, 38)
\end{exe}

When the standard is not overt, the comparee markers \forme{kɯ} (as in \ref{ex:qandzxe.kW.YWjpum}) or \forme{mɤʑɯ} (§\ref{sec:mAZW}). obligatory.

\begin{exe}
\ex \label{ex:qandzxe.kW.YWjpum}
\gll tɯ-ɣli kɯ-dɤn ɯ-stu qandʐe nɯ kɯ ɲɯ-jpum. \\
\textsc{indef}.\textsc{poss}-dung \textsc{sbj}:\textsc{pcp}-be.many \textsc{3sg}.\textsc{poss}-place earthworm \textsc{dem} \textsc{erg} \textsc{sens}-be.thick \\
\glt `The earthworms that are in places where there is a lot of dung (fertilizer) are fatter.' (25-akWzgumba, 128)
\end{exe}

The comparee marker \forme{kɯ} can occur on the \textit{possessor} of the comparee rather than on the comparee itself, as in (\ref{ex:aZo.kW.alaz}).

\begin{exe}
\ex \label{ex:aZo.kW.alaz}
\gll aʑo kɯ a-laz ɲɯ-sna  \\
\textsc{1sg} \textsc{erg} \textsc{1sg}.\textsc{poss}-karma \textsc{sens}-be.good \\
\glt `I am luckier [than you].' (140515 huli he yelv-zh, 19)
\end{exe}


The parameter indexes the comparee, never the standard, as shown in (\ref{ex:sAz.YWtWrZi}), where the main verb has \textsc{2sg} indexation. The comparee can be relativized like a normal intransitive subject (§\ref{comparee.relativization}), but the standard cannot (§\ref{sec:accessibility.relativization}).

\begin{exe}
\ex \label{ex:sAz.YWtWrZi}
\gll tɤru ɯ-tɕɯ nɯ sɤz ɲɯ-tɯ-rʑi! \\
chieftain \textsc{3sg}.\textsc{poss}-son \textsc{dem} \textsc{comp} \textsc{sens}-2-be.heavy \\
\glt `You are heavier than the prince!' (140506 woju guniang-zh, 134)
\end{exe}

The parameter can be combined with exceptive phrases
with \japhug{ma}{apart from} (§\ref{sec:exceptive}) containing counted nouns, to specify more precisely the difference in degree between the two referents as in (\ref{ex:sAz.YWtWrZi}).

\begin{exe}
\ex \label{ex:RWpArme.ma.mAxtCi}
\gll waŋtɕin sɤznɤ ʁnɯ-pɤrme ma mɤ-xtɕi \\
\textsc{anthr} \textsc{comp} [two-years apart.from] \textsc{neg}-be.small:\textsc{fact} \\
\glt `He is only two years younger than Dbangcan.' (14-siblings, 242)
\end{exe}
 
Comparees and standards are generally nouns, noun phrases or postpositional phrases as in the examples above, but can also be temporal adverbs (\ref{ex:comparative2}) or subordinate clauses, in the Irrealis (§\ref{sec:irrealis}), or other modal categories as in (\ref{ex:apWsi.kW}).\footnote{The original Chinese is \ch{还不如死了算了}{háibùrú sǐ le suànle}{It is better if he dies}. The use of the ergative in rectification subordinate clauses (§\ref{sec:rectification.clauses}) is historically related to this construction \citep{jacques16comparative}. }
  
\begin{exe}
\ex \label{ex:comparative2}
\gll  jɯfɕɯr sɤz jɯsŋi kɯ ɲɯ-mpja \\
yesterday \textsc{comp} today \textsc{erg} \textsc{sens}-be.warm \\
\glt `Today is warmer than yesterday.' (elicited)
\end{exe}

\begin{exe}
\ex \label{ex:apWsi.kW}
\gll nɯ sɤznɤ [a-pɯ-si] kɯ ɲɯ-mna \\
\textsc{dem} comp \textsc{irr}-\textsc{pfv}-die \textsc{erg} \textsc{sens}-be.better \\
\glt `It is better that he dies.' (150909 xiaocui-zh, 118)
\end{exe}

\subsection{Intensifier  \japhug{tsa}{a little}  } \label{sec:comparative.tsa}
The degree adverb \japhug{tsa}{a little}, one of the very rare adverbs to occur postverbally (§\ref{sec:postverbal.adv}), can by itself express comparison of superiority as in (\ref{ex:YWtCur.tsa}).

\begin{exe}
\ex \label{ex:YWtCur.tsa}
\gll  kɯ-mɤku cʰɯ-kɯ-tɯt nɯ ɲɯ-tɕur tsa \\
\textsc{sbj}:\textsc{pcp}-be.first \textsc{ipfv}-\textsc{sbj}:\textsc{pcp}-ripen \textsc{dem} \textsc{sens}-be.sour a.little\\
\glt `The [variety of apple] that ripens earlier is a bit sourer.' (07-paXCi, 48)
\end{exe}

It can be combined with an overt standard marked by either \forme{sɤz} `than' (\ref{ex:sAz.YWwxti.tsa}) or \japhug{stʰɯci}{as much} (example \ref{ex:YWrYJi.tsa}, §\ref{sec:postverbal.adv}).

\begin{exe}
\ex \label{ex:sAz.YWwxti.tsa}
\gll  ʁmɯrcɯ sɤz ɲɯ-wxti tsa \\
Garrulax \textsc{comp} \textsc{sens}-be.big a.little \\
\glt `It is a bit bigger than the Garrulax sp.' (23-qapGAmtWmtW, 13)
\end{exe}

\subsection{The negative verb \textsc{neg}+\forme{ʑɯ} `(not) just be'} \label{sec:mAZW}
The verb \japhug{\textsc{neg}+ʑɯ}{not just be}, which requires a negative prefix (§\ref{sec:obligatory.negative}) serves to build another comparative construction. It takes as semi-object the standard, and can be followed by the adverb \japhug{jamar}{about} and a verb in parataxis specifying the parameter of comparison as in (\ref{ex:mWjZW.jamar}).

\begin{exe}
\ex \label{ex:mWjZW.jamar}
 \gll tɤɕi ɯ-rdoʁ mɯ́j-ʑɯ jamar ɲɯ-wxti \\
 barley \textsc{3sg}.\textsc{poss}-grain \textsc{neg}:\textsc{sens}-just.be about \textsc{sens}-be.big \\
 \glt `It is a little bigger than a grain of barley.' (28-kWpAz,104)
\end{exe}

It can also be in infinitive form \forme{mɤ-kɯ-ʑɯ} as a manner converb (§\ref{sec:inf.converb}, §\ref{sec:manner.converbs}) as in (\ref{ex:mAkWZW.Zo.YWjpum}). 

\begin{exe}
\ex \label{ex:mAkWZW.Zo.YWjpum}
 \gll ɯ-ru nɯra tɯ-jaʁndzu mɤ-kɯ-ʑɯ ʑo ɲɯ-jpum cʰa \\
\textsc{3sg}.\textsc{poss}-stalk \textsc{dem}:\textsc{pl} \textsc{indef}.\textsc{poss}-finger \textsc{neg}-\textsc{inf}:\textsc{stat}-just.be \textsc{emph} \textsc{ipfv}-be.thick can:\textsc{fact} \\
 \glt `Its stalk can grow thicker than a finger.' (12-ndZiNgri, 5)
\end{exe}

The stative verb expressing the parameter can be combined with modal verbs as in (\ref{ex:mAkWZW.Zo.YWjpum}). Without an overt parameter, the default interpretation of the verb \forme{mɤ-ʑɯ} means `not just like that' or `bigger than'. In (\ref{ex:ndAre.mAZW}) for instance, \forme{mɤ-ʑɯ} is redundantly followed by a comparative construction in \forme{sɤz} (§\ref{sec:sAz.kW}), and the two clauses \forme{βʑɯ ndɤre mɤ-ʑɯ} and \forme{βʑɯ sɤz ndɤre wxti ŋu} have the same meaning.

\begin{exe}
\ex \label{ex:ndAre.mAZW}
 \gll nɯnɯ kɯ-xtɕɯ\redp{}xtɕi ci tɕe, βʑɯ ndɤre mɤ-ʑɯ. βʑɯ sɤz ndɤre wxti ŋu \\
 \textsc{dem} \textsc{sbj}:\textsc{pcp}-\textsc{emph}\redp{}be.small \textsc{indef} \textsc{lnk} mouse \textsc{advers} \textsc{neg}-just.be:\textsc{fact} mouse \textsc{comp} \textsc{advers} be.big:\textsc{fact} be:\textsc{fact} \\
\glt `[The weasel] is a small [animal], but bigger than a mouse. It is bigger than a mouse.' (27-spjaNkW, 32)
\end{exe}

In this construction, \forme{mɤ-ʑɯ} indexes as subject the comparee, as in (\ref{ex:mAZWa}).

\begin{exe}
\ex \label{ex:mAZWa}
 \gll ki kɯm ki mɤ-ʑɯ-a \\
\textsc{dem}.\textsc{prox} door \textsc{dem}.\textsc{prox} \textsc{neg}-just.be:\textsc{fact}-\textsc{1sg} \\
\glt `I am bigger than this door.' (elicited)
\end{exe}

Alternatively, \forme{mɤ-ʑɯ} can be combined with a degree nominal (§\ref{sec:degree.monoclausal}) as in (\ref{ex:atWmbro.mAZW}), in which case it only occurs in the \textsc{3sg}.

\begin{exe}
\ex \label{ex:atWmbro.mAZW}
 \gll a-tɯ-mbro ki kɯm ki mɤ-ʑɯ. \\
 \textsc{1sg}.\textsc{poss}-\textsc{nmlz}:\textsc{deg}-be.high \textsc{dem}.\textsc{prox} door \textsc{dem}.\textsc{prox} \textsc{neg}-just.be:\textsc{fact} \\
\glt `I am bigger than this door.' (elicited)
\end{exe}

The non-past form \forme{mɤ-ʑɯ} has been further grammaticalized as a degree adverb \japhug{mɤʑɯ}{even more}, which can be combined with the standard marker \forme{sɤz}, as in (\ref{ex:ndAre.mAZW.Zo.amtCoR}). This construction indicates that the standard already has a very high degree relative to the parameter, and that the comparee's degree is even higher.

\begin{exe}
\ex \label{ex:ndAre.mAZW.Zo.amtCoR}
 \gll qamtɕɯr nɯ ɯ-mtɕʰi nɯnɯ βʑɯ sɤznɤ mɤʑɯ ʑo amtɕoʁ \\
 shrew \textsc{dem} \textsc{3sg}.\textsc{poss}-mouth \textsc{dem} mouse \textsc{comp} even.more \textsc{emph} be.pointy:\textsc{fact} \\
\glt `The mouth of the shrew is even more pointy than that of the mouse.' (27-spjaNkW, 204)
\end{exe}

\subsection{The egressive postposition \japhug{ɕaŋtaʁ}{up from}} \label{sec:egressive.comparative}
The egressive postpositions (§\ref{sec:egressive}), in particular \japhug{ɕaŋtaʁ}{up from}, can be combined with a verb (or a complex predicate) in negative form, meaning `no more than $X$', where the standard $X$ is the noun phrase preceding \forme{ɕaŋtaʁ} as in (\ref{ex:CaNtaR.mAzri}) and (\ref{ex:CaNtaR.YWjpum.mAcha}).

\begin{exe}
\ex \label{ex:CaNtaR.mAzri}
 \gll  [tɯ-tɣa ɕaŋtaʁ] mɤ-zri. \\
 one-span up.from \textsc{neg}-be.long:\textsc{fact} \\
 \glt `It is not longer than one handspan.' (28-tshAwAre, 51)
 \end{exe}

\begin{exe}
\ex \label{ex:CaNtaR.YWjpum.mAcha}
 \gll ɯ-ru ra (...), [tɯ-jaʁndzu ɕaŋtaʁ] ɲɯ-jpum mɤ-cʰa ma \\
 \textsc{3sg}.\textsc{poss}-stalk \textsc{pl} {  } \textsc{genr}.\textsc{poss}-finger up.from \textsc{ipfv}-be.thick \textsc{neg}-can:\textsc{fact} \textsc{lnk} \\
 \glt `Its stalk cannot grow/become thicker than a finger.' (15-babW, 23)
 \end{exe}

The egressive \forme{ɕaŋtaʁ} also occurs in one of the superlative constructions (§\ref{sec:negative.existential.superlative}).

% {sec:terminative}  \japhug{mɤɕtʂa}{until}
\subsection{Negative existential verbs} \label{sec:existential.comparative}
The negative existential verb \forme{me} (§\ref{sec:existential.basic}), combined with the scalar focus marker \japhug{kɯnɤ}{even} (§\ref{sec:kWnA}), can have the meaning `not even as big as $X$', as in (\ref{ex:kWnA.me}), where the noun 
\japhug{zrɯɣ}{louse} is the standard of comparison.

\begin{exe}
\ex \label{ex:kWnA.me}
\gll qajɯβlama kɤ-ti ci tu tɕe, nɯ rca kɯ-xtɕɯ\redp{}xtɕi ci ʑo ɕti. zrɯɣ kɯnɤ me. \\
bug \textsc{obj}:\textsc{pcp}-say \textsc{indef} exist:\textsc{fact} \textsc{lnk} \textsc{dem} \textsc{unexp}:\textsc{foc} \textsc{sbj}:\textsc{pcp}-\textsc{emph}\redp{}be.small \textsc{indef} \textsc{emph} be.\textsc{aff}:\textsc{fact} louse also not.exist:\textsc{fact} \\
\glt `There is an [insect] called \forme{qajɯβlama}, it is very small. It is not even as big as a louse.' (28-kWpAz, 149-150)
\end{exe}
 
Another example of this construction is found in (\ref{ex:BZW.GW.WqiW}) in (§\ref{sec:gen.possession}).

\section{Equative and similative}


\subsection{Entity equative} \label{sec:equative}
This section discusses entity equative constructions, which express that two entities have a property in equal degree (`X is as Y as Z'; \citealt{haspelmath08equative}). No less that five constructions are available to express this meaning.

\subsubsection{Nominalized equative construction} \label{sec:nmlz.equative}
The nominalized entity equative construction is a particular case of the degree nominal construction (§\ref{sec:degree.monoclausal}), with \japhug{naχtɕɯɣ}{be the same} as degree predicate. It has three subvariants.

In the first variant (corresponding to Haspelmath's (\citeyear{haspelmath17equative}) type 5 -- Primary reach equative unified), the comparee and the standard are included in a noun phrase, with the comitative marker \forme{cʰo} (and its longer variant \forme{cʰondɤre}, §\ref{sec:comitative}) serving as the standard marker, as in (\ref{ex:chondAre.ndZitWwxti}).

\begin{exe}
\ex \label{ex:chondAre.ndZitWwxti}
\glll qalekɯtsʰi nɯnɯ cʰondɤre βʑar ni ndʑi-tɯ-wxti naχtɕɯɣ.\\
bird.sp \textsc{dem} \textsc{comit} buzzard \textsc{du} \textsc{3du}.\textsc{poss}-\textsc{nmlz}:\textsc{deg}-be.big be.identical:\textsc{fact} \\
{\textsc{comparee}} { } \textsc{standard}.\textsc{marker} {\textsc{standard}} { } \textsc{parameter} \textsc{parameter}.\textsc{marker} \\
\glt `The \forme{qalekɯtsʰi} bird is as big as the buzzard.' (literally: `The \forme{qalekɯtsʰi} bird and the buzzard are identical in their degree of bigness.')  (23-RmWrcWftsa, 34)
\end{exe}

The degree noun can be in dual/plural as in (\ref{ex:chondAre.ndZitWwxti}) and (\ref{ex:ndZitWwxti.naXtCWG}), corresponding to the sum of the numbers of the comparee and the standard, or in the  singular as in (\ref{ex:cho.WtWwxti.naXtCWG}): even though two referents are present here, the \textsc{3sg} possessive \forme{ɯ-} is used, coreferent with the comparee. 

\begin{exe} %archi
\ex \label{ex:cho.WtWwxti.naXtCWG}
\gll  qro nɯnɯ dɯdɯt cʰo ɯ-tɯ-wxti naχtɕɯɣ. \\
pigeon \textsc{dem} dove \textsc{comit} \textsc{3sg}.\textsc{poss}-\textsc{nmlz}:\textsc{deg}-be.big be.the.same:\textsc{fact} \\
\glt `The pigeon is as big as a dove.' (24-qro, 2)
\end{exe}

These two possibilities are in free variation: example (\ref{ex:ndZitWwxti.naXtCWG}), from the same text as (\ref{ex:cho.WtWwxti.naXtCWG}) and referring to the same situation, has dual marking.

\begin{exe}
\ex \label{ex:ndZitWwxti.naXtCWG}
\gll tɕe  dɯdɯt cʰo ndʑi-tɯ-wxti naχtɕɯɣ ʑo \\
\textsc{lnk} dove \textsc{comit} \textsc{3du}.\textsc{poss}-\textsc{nmlz}:\textsc{deg}-be.big be.the.same:\textsc{fact} \textsc{emph} \\
\glt `It is as big as a dove.' (24-qro, 19)
\end{exe}

In the second variant,  the nominalized parameter takes a possessive prefix only coreferent with the comparee, and the standard together with the comitative (the standard marker) \textit{follows} the parameter, as in (\ref{ex:WtWwxti.cho.naXtCWG}).

\begin{exe}
\ex \label{ex:WtWwxti.cho.naXtCWG}
\glll qaliaʁ nɯ ɯ-tɯ-wxti nɯ qandʑɣi cʰo naχtɕɯɣ tsa 	\\
eagle \textsc{dem} \textsc{3sg}.\textsc{poss}-\textsc{nmlz}:\textsc{deg}-be.big \textsc{dem} hawk \textsc{comit} be.identical:\textsc{fact} a.little  \\
{\textsc{comparee}} { } \textsc{parameter} { } {\textsc{standard}} \textsc{standard}.\textsc{marker} \textsc{parameter}.\textsc{marker}  \textsc{parameter}.\textsc{marker} \\
\glt `The eagle is about as big as the hawk.' (19-qandZGi, 36)
\end{exe}

In the third variant, the parameter takes a third person singular possessive prefix, and the comparee and standard are marked by person indexation on the verb. In (\ref{ex:WtWmWCtaR.YWnaXtCWGtCi}), the standard and the comparee are the speaker and the addressee; they are not expressed by overt pronouns, but are rather indexed on the verb by the \textsc{1du} suffix \forme{-tɕi}.

\begin{exe}
\ex \label{ex:WtWmWCtaR.YWnaXtCWGtCi}
\glll tɕe ɯ-tɯ-mɯɕtaʁ ɲɯ-naχtɕɯɣ-tɕi tɕe, qʰe nɯ-tɤjpa ɲɯ-rkɯn ma\\
\textsc{lnk} \textsc{3sg}.\textsc{poss}-\textsc{nmlz}:\textsc{deg}-be.cold \textsc{sens}-be.identical-\textsc{1du} \textsc{lnk} \textsc{lnk} \textsc{2pl}.\textsc{poss}-snow \textsc{sens}-be.few \textsc{sfr}\\
{ } \textsc{parameter} \textsc{parameter}.\textsc{marker}-\textsc{comparee+standard}\\
\glt `It is as cold here as it is in your place, you don't have a lot of snow.' (`You and I are identical as to coldness'; conversation, 2014/11)
\end{exe}
%
\subsubsection{Serial verb constructions} \label{sec:svc.equative}
Serial verb constructions with the similative verb \japhug{fse}{be like} (§\ref{sec:svc.similative.verb}), or more rarely  \japhug{naχtɕɯɣ}{be identical} and \japhug{afsuja}{be of the same size} as first verb can also be used as an entity equative.\footnote{These constructions correspond to  Haspelmath's (\citeyear{haspelmath17equative}) type 1 (Only equative standard-marker). }


The semi-transitive verb \japhug{fse}{be like} (§\ref{sec:semi.transitive})  takes the comparee as subject and the standard as semi-object. Since it is syntactically linked to the standard, it is analyzed here as the standard marker rather than as the parameter marker. 
  
\begin{exe}
\ex \label{ex:pjAfse.pjAsAjloR}
\glll nɯ li ɯ-wa fsɯfse ʑo pjɤ-fse pjɤ-sɤjloʁ \\
\textsc{dem} again \textsc{3sg}.\textsc{poss}-father completely.like \textsc{emph} \textsc{ifr}.\textsc{ipfv}-be.like \textsc{ifr}.\textsc{ipfv}-be.ugly \\
\textsc{comparee} { } \textsc{standard} \textsc{parameter}.\textsc{marker} { } \textsc{standard}.\textsc{marker} \textsc{parameter} \\
\glt `[The frog son] was as ugly as his father. ' (150818 muzhi guniang-zh, 100)
\end{exe}

The reduplicated degree adverb \forme{fsɯfse} `completely identical' which derives from \japhug{fse}{be like} optionally occurs in this construction as a parameter marker.

\begin{exe} 
	\ex \label{ex:fsWfse.Zo.fse}
	\gll ɯ-qa nɯra li kɯmaʁ tɤjmɤɣ nɯra fsɯfse ʑo fse. \\
	\textsc{3sg}.\textsc{poss}-root \textsc{dem}:\textsc{pl} again other mushroom \textsc{dem}:\textsc{pl} completely.identical \textsc{emph} be.like:\textsc{fact} \\
	\glt `Its root is completely identical to that of other mushrooms.' (23-mbrAZim, 115)
\end{exe}


Both \japhug{fse}{be like} and the stative verb occurring with it in the serial construction (the parameter) are in participial form in (\ref{ex:kWfse.kAchWcha}), forming a relative clause with the comparee as the relativized element. The superlative construction studied in §\ref{sec:negative.existential.superlative} is essentially a particular use of such relativized equative sentences.

\begin{exe}
\ex \label{ex:kWfse.kAchWcha}
\glll aʑo kɯ-fse kɯ-ɤcʰɯcʰa ʑo ʁʑɯnɯ ɣurʑa kɯrcat ra\\
\textsc{1sg} \textsc{sbj}:\textsc{pcp}-be.like \textsc{sbj}:\textsc{pcp}-be.capable \textsc{emph} young.man hundred eight be.needed:\textsc{fact} \\
\textsc{standard} \textsc{standard}.\textsc{marker} \textsc{parameter} { } \textsc{comparee} \\
\glt `I need a hundred and eight young men as able as I am.' (Norbzang 2012, 17)
\end{exe}

The verb \japhug{naχtɕɯɣ}{be the same} requires in addition the comitative \forme{cʰo} on the standard (as in the preceding construction, §\ref{sec:nmlz.equative}), as shown by (\ref{ex:cho.naXtCWG.jamar.YWmWCtaR}).

\begin{exe}
\ex \label{ex:cho.naXtCWG.jamar.YWmWCtaR}
\glll <bali> nɯ, kukutɕu iʑora cʰo naχtɕɯɣ jamar ɲɯ-mɯɕtaʁ ɲɯ-tɯ-ti tɕe \\
\textsc{topo} \textsc{dem} here \textsc{1pl} \textsc{comit} be.the.same:\textsc{fact} about \textsc{sens}-be.cold \textsc{sens}-2-say \\
\textsc{comparee} { }  \textsc{standard} \textsc{standard}.\textsc{marker} \textsc{standard}.\textsc{marker}  \textsc{parameter}.\textsc{marker} \textsc{parameter} \textsc{lnk} \\
\glt `You said that it was as cold in Paris as here by us.' (conversation, 11-08-2016)
\end{exe}




\subsubsection{Possessed noun} \label{sec:Wfsu.equative}
The inalienably possessed noun  \japhug{ɯ-fsu}{equal in size to} (§\ref{sec:3sg.possessive.form}, §\ref{sec:fsu.fse}) can be used as standard marker, as in (\ref{ex:Wfsu.jamar}) and (\ref{ex:Wfsu.jamar2}). The possessive prefix is coreferent with the standard; when the standard is the generic noun \japhug{tɯrme} {person}, the prefix can either be in \textsc{3sg} as in (\ref{ex:Wfsu.jamar}), or with the generic possessor prefix (§\ref{sec:tWrme.generic}).

\begin{exe}
\ex \label{ex:Wfsu.jamar}
\gll tu-mbro tɕe, tɯrme ɯ-fsu jamar tu-βze cʰa. \\
\textsc{ipfv}-be.high \textsc{lnk}  man \textsc{3sg}.\textsc{poss}-equal.in.size about \textsc{ipfv}-grow can:\textsc{fact} \\
\glt `When it grows, it can grow about the size of a person.' (12-ndZiNgri, 4)
\end{exe}

\begin{exe}
\ex \label{ex:Wfsu.jamar2}
\gll ɯ-tɯ-mbro nɯ tɯ-mtʰɤɣ ɯ-fsu jamar ma tu-mbro mɤ-cʰa \\
\textsc{3sg}.\textsc{poss}-\textsc{nmlz}:\textsc{deg}-be.high \textsc{dem} \textsc{genr}.\textsc{poss}-waist \textsc{3sg}.\textsc{poss}-equal.in.size about apart.from \textsc{ipfv}-be.high \textsc{neg}-can:\textsc{fact} \\
\glt `As for its size, it can grow only about as high as a person's waist.' (18-NGolo, 181)
\end{exe}

The parameter is optional in this construction. It can be expressed either as a coordinated clause as in (\ref{ex:Wfsu.jamar}), as a degree nominal or as the main predicate as in (\ref{ex:Wfsu.jamar2}). Only \japhug{mbro}{be high} and \japhug{wxti}{be big} are compatible with \forme{ɯ-fsu}.

A similar construction is reported in Situ \citep[377]{linxr93jiarong}.

 
\subsubsection{\japhug{stʰɯci}{as much} and \japhug{jamar}{about}} \label{sec:sthWci.equative}
The adverbs \japhug{stʰɯci}{so much} (on its etymology, see §\ref{sec:denominal.adverb.s.prefix}) and  \japhug{jamar}{about} (from Tibetan \tibet{ཡར་མར་}{jar.mar}{about}) are used as standard marker. The former one \forme{stʰɯci} essentially occurs in a negative equative construction `not as $X$ as $Y$' (where $X$ is the parameter and $Y$ the standard) as in (\ref{ex:sthWci.mArJum}).

\begin{exe}
\ex \label{ex:sthWci.mArJum}
\gll kɯmɕku ɯ-jwaʁ stʰɯci mɤ-rɟum \\
garlic \textsc{3sg}.\textsc{poss}-leaf so.much \textsc{neg}-be.broad \\
\glt `[Its leaves] are not as broad as garlic leaves.' (07-Cku, 91)
\end{exe}

The latter one \japhug{jamar}{about}  can be combined with either adjectival stative verbs such as \japhug{wxti}{be big} (\ref{ex:RnWz.jamar.wxti}), or with existential verbs (§\ref{sec:existential.basic}) as in (\ref{ex:ki.jamar.tu}).
 
\begin{exe}
\ex \label{ex:RnWz.jamar.wxti}
\gll qajdo ʁnɯz jamar wxti \\
crow two about be.big:\textsc{fact} \\
\glt `It is about as big as two crows.' (19-qandZGi, 9)
\end{exe}

\begin{exe}
\ex \label{ex:ki.jamar.tu}
\gll ɯ-mat ɣɯ ɯ-ru nɯ zri tɕe, tɕe tɯ-tɣa jamar, ki jamar tu tɕe \\
\textsc{3sg}.\textsc{poss}-fruit \textsc{gen} \textsc{3sg}.\textsc{poss}-stalk \textsc{dem} be.long:\textsc{fact} \textsc{lnk} \textsc{lnk} one-span about \textsc{dem}.\textsc{prox} about exist:\textsc{fact} \textsc{lnk} \\
\glt `The stalk of its fruit is long, about a handspan long, about this long.' (16-CWrNgo, 224)
\end{exe}
 

\subsection{Property equative}
In property equative constructions (`$X_i$ is as $Y$ as he/she/it$_i$ is $Z$'),  two parameters (comparee parameter $Y$ and standard parameter $Z$), rather than two entities, are compared. No example of such constructions is found in the Japhug corpus. In \citet{jacques18similative}, I used Perrault's fairy tale \textit{Riquet à la Houppe}, whose whole plot is based on property equative sentences, as a way to conduct elicitation on this topic.

Property equatives, e.g. `$X_i$ is as stupid as s/he$_i$ is beautiful' (a sentence occurring several times in the story), can be expressed in Japhug in three different ways.

First, the standard parameter is in degree nominal form, followed by the possessed noun \japhug{ɯ-fsu}{equal in size to}  (§\ref{sec:Wfsu.equative}), and the comparee parameter is the main predicate of the construction, in finite form (\ref{ex:WtWmpCAr.GW.Wfsu.jamar}).


\begin{exe}
\ex \label{ex:WtWmpCAr.GW.Wfsu.jamar}
\gll ɯ-tɯ-mpɕɤr ɣɯ ɯ-fsu jamar ci ɲɯ-kʰe ɕti \\
\textsc{3sg}.\textsc{poss}-\textsc{nmlz}:\textsc{deg}-be.beautiful \textsc{gen} \textsc{3sg}.\textsc{poss}-equal.in.size about \textsc{indef} \textsc{sens}-be.stupid be.\textsc{aff}:\textsc{fact} \\
\glt `S/he$_i$ is stupid to the extent of his/her$_i$ beauty.' (elicited)
\end{exe}

Second, the two verbs used as paramaters are in degree nominal form, linked by the comitative postposition \forme{cʰo}, and serve as subject of the verb \japhug{afsuja}{be of the same size} (\ref{ex:WtWmpCAr.cho.WtWkhe.YAfsuja}).

\begin{exe}
\ex \label{ex:WtWmpCAr.cho.WtWkhe.YAfsuja}
\gll ɯ-tɯ-mpɕɤr cʰo ɯ-tɯ-kʰe nɯ ɲɯ-ɤfsuja ɕti\\
\textsc{3sg}.\textsc{poss}-\textsc{nmlz}:\textsc{deg}-be.beautiful \textsc{comit} \textsc{3sg}.\textsc{poss}-\textsc{nmlz}:\textsc{deg}-be.stupid \textsc{dem} \textsc{sens}-be.of.the.same.size be.\textsc{aff}:\textsc{fact} \\
\glt `His/her beauty and his/her stupidity are equal.' (elicited)
\end{exe}

Third, it is possible to express the same meaning with a correlative construction, as in (\ref{ex:correlative.parameter.equative}), though this may be a calque from Chinese.

\begin{exe}
\ex \label{ex:correlative.parameter.equative}
\gll tɕʰi jamar kɯ-mpɕɤr nɯ, nɯ jamar ci ɲɯ-kʰe ɕti \\
what about \textsc{sbj}:\textsc{pcp}-be.beautiful \textsc{dem} \textsc{dem} about \textsc{indef} \textsc{sens}-be.stupid be.\textsc{aff}:\textsc{fact} \\
\glt `A much as s/he is beautiful, s/he is stupid.' (elicited)
\end{exe}

  
\subsection{Similative} \label{sec:similative}
Similative constructions express similarity in the manner in which an action is performed, rather than equal degree.

\subsubsection{Similative verbs} \label{sec:svc.deixis}

The main similative construction involves the similative verbs \japhug{fse}{be like} or \japhug{stu}{do like} in parataxis as in (\ref{ex:tWrme.YWfse}).

\begin{exe}
\ex \label{ex:tWrme.YWfse}
\gll nɯnɯ pri nɯ kɯ, nɤki, tɯrme ɲɯ-fse tɕe, nɤkinɯ iɕqʰa nɯ, tɤ-rɤku tɕi tu-ndze, ɕa tɕi tu-ndze, (...)  ɲɯ-ŋgrɤl. \\
\textsc{dem} bear \textsc{dem} \textsc{erg} \textsc{filler} man \textsc{sens}-be.like \textsc{lnk} \textsc{filler} \textsc{filler} \textsc{dem} \textsc{indef}.\textsc{poss}-crops also \textsc{ipfv}-eat meat also \textsc{ipfv}-eat { } \textsc{sens}-be.usually.the.case \\
\glt `The bear, like a man, eats grains and meat.' (21-pri, 17)
\end{exe}

Alternatively, the infinitive \forme{kɯ-fse} of \japhug{fse}{be like} as a manner converb (§\ref{sec:inf.converb}, §\ref{sec:manner.converbs}) can convey similative meaning, either with a noun phrase or an infinitive clause (\ref{ex:kAnWGro.kWfse}).

\begin{exe}
\ex \label{ex:kAnWGro.kWfse}
\gll [kɤ-ɤnɯɣro] kɯ-fse tú-wɣ-ndza ɕti ma \\
\textsc{inf}-play \textsc{inf}:\textsc{stat}-be.like \textsc{ipfv}-\textsc{inv}-eat be.\textsc{aff}:\textsc{fact} \textsc{lnk} \\
\glt `People eat it for fun (as if to play, not as part of a real meal).' (08-rasti, 59)
\end{exe}


\subsubsection{Similative denominal stative verbs} \label{sec:denominal:similative}
The denominal prefix \forme{arɯ-/ɤrɯ-} derives stative verbs meaning `be $X$-like' out of nouns (§\ref{sec:denom.arW}), as in (\ref{ex:arWsWjno}), an example in degree nominal form (§\ref{sec:degree.nominal.predicates}) spontaneously produced by Tshendzin as a side comment on a story that we were transcribing.

\begin{exe}
	\ex \label{ex:arWsWjno}
	\gll  ɯ-tɯ-ɤrɯ-sɯjno nɯ!   \\
	\textsc{3sg}.\textsc{poss}-\textsc{nmlz}:\textsc{deg}-\textsc{denom}:\textsc{similative}-grass \textsc{sfp} \\
	\glt `[The princess cuts their head as easily/casually] as if it were grass.' (heard in context)
\end{exe}

The more elaborated sentence (\ref{ex:YArWsWjno}) was given as an explanation for (\ref{ex:arWsWjno}).

\begin{exe}
	\ex \label{ex:YArWsWjno}
	\gll  kɤ-pʰɯt ɯ-tɯ-mbat kɯ ɲɯ-ɤrɯ-sɯjno ʑo \\
	\textsc{inf}-cut  \textsc{3sg}.\textsc{poss}-\textsc{nmlz}:\textsc{deg}-\textsc{denom}:\textsc{similative}-easy \textsc{erg} \textsc{sens}-be.like.grass \textsc{emph} \\
	\glt `It is as easy to cut as if it were grass.' (elicited)
\end{exe}

In this construction, the standard is the verbalized noun, the comparee is the intransitive subject, and the denominal prefix \forme{arɯ-/ɤrɯ-} is the standard marker. The parameter can be optionally indicated as a degree nominal as in (\ref{ex:YArWsWjno}).

This unusual similative construction is productive, since it can be applied to nouns from Tibetan or Chinese. It does not fit in any of Haspelmath's (\citeyear{haspelmath17equative}) six types of equative constructions, but bears some resemblance to the ``similative adjective'' derivation in \forme{-lágan} in Saami \citep[5.1]{ylikovski17similarity}. 


\section{Superlative} \label{sec:superlative}

 \subsection{Degree adverb} \label{sec:stu.superlative}
The superlative adverb \japhug{stu}{most} is generally located before the verb, and optionally takes the emphatic \forme{ʑo}. It most commonly expresses absolute superlative as in  (\ref{ex:tWmkhAz2}).
 

 \begin{exe}
\ex \label{ex:tWmkhAz2}
\gll nɤʑo stu ʑo tɯ-mkʰɤz tɕe, tɕe nɤʑo ɕ-tɤ-nɤme \\
\textsc{2sg} most \textsc{emph} 2-be.expert:\textsc{fact}   \textsc{lnk} \textsc{lnk} \textsc{2sg} \textsc{tral}-\textsc{imp}-do[III] \\
\glt `You are the best at it, do it!' (150822 laoye zuoshi zongshi duide-zh, 37)
\end{exe}

When the subject has a certain property in the highest degree only relative to a certain class (relative superlative), this class can be specified with the relator noun \japhug{ɯ-ŋgɯz}{among} (§\ref{sec:relator.postposition.location}), as in  (\ref{ex:nWNgWz.stu.xtCi}).

 
\begin{exe}
\ex \label{ex:nWNgWz.stu.xtCi}
\gll nɯ pɣɤtɕɯ nɯ-ŋgɯz stu xtɕi low.\\
\textsc{dem} bird \textsc{3pl}.\textsc{poss}-among most be.small:\textsc{fact} \textsc{sfp} \\
\glt `It is the smallest of all birds.' (24-ZmbrWpGa, 126)
\end{exe}

Most examples of this construction appear with subject participle (§\ref{sec:subject.participles}) of adjectival stative verbs as in (\ref{ex:stu.kWNAn}), and object participles (§\ref{sec:object.participle}) with transitive experiencer verbs as in (\ref{ex:stu.WkAnWrga}).

\begin{exe}
\ex \label{ex:stu.kWNAn}
\gll kɯɕɯŋgɯ tɕe <aizheng> kɤ-ti pɯ-me tɕe, kɤ-kɯ-nɤndza nɯ stu ʑo kɯ-ŋɤn kɤ-pa pɯ-ŋu.  \\
long.ago \textsc{lnk} cancer \textsc{obj}:\textsc{pcp}-say \textsc{pst}.\textsc{ipfv}-not.exist \textsc{lnk} \textsc{aor}-\textsc{sbj}:\textsc{pcp}-have.leprosy \textsc{dem} most \textsc{emph} \textsc{sbj}:\textsc{pcp}-be.evil \textsc{obj}:\textsc{pcp}-consider \textsc{pst}.\textsc{ipfv}-be \\
\glt `In former times, nobody talked about cancer, and leprosy was considered to be the most terrible [of all diseases].' (25-khArWm, 35)
\end{exe}

\begin{exe}
\ex \label{ex:stu.WkAnWrga}
\gll tɕe tɯ-ci ɯ-rkɯ tu-ɬoʁ tɕe, nɯ stu ɯ-kɤ-nɯ-rga ŋu tɕe \\
\textsc{lnk} \textsc{indef}.\textsc{poss}-water \textsc{3sg}.\textsc{poss}-side \textsc{ipfv}-come.out \textsc{lnk} \textsc{dem} most \textsc{3sg}.\textsc{poss}-\textsc{appl}-like be:\textsc{fact} \textsc{lnk} \\
\glt `It grows near the water, it is what it likes the most.' (09-mi, 6)
\end{exe}

The superlative adverb is also attested with transitive dynamic verbs of action, as in the pseudo-cleft construction (§\ref{sec:pseudo.cleft}) in (\ref{ex:stu.Zo.WkAndza}).\footnote{The phrase \forme{tɤ-rme kɯ-fse} `like hair' is incomplete; the correct way to express the meaning `clothes made of animal hair' is the head-internal relative clause \forme{tɤ-rme kɯ tɯ-ŋga tʰɯ-kɤ-βzu} (see \ref{ex:tWNga.thWkABzu}, §\ref{sec:head-internal.relative}).} 

\begin{exe}
\ex \label{ex:stu.Zo.WkAndza}
\gll [stu ʑo ɯ-kɤ-ndza] nɯnɯ tɯ-ŋga, tɤ-rme kɯ-fse, tɯ-ŋga nɯra ŋu. \\
most \textsc{emph} \textsc{3sg}.\textsc{poss}-\textsc{obj}:\textsc{pcp}-eat \textsc{dem} \textsc{indef}.\textsc{poss}-clothes \textsc{indef}.\textsc{poss}-hair \textsc{sbj}:\textsc{pcp}-be.like \textsc{indef}.\textsc{poss}-clothes \textsc{dem}:\textsc{pl} be:\textsc{fact} \\
\glt `What it eats most is clothes, clothes (made of animal fur).' (28-kWpAz, 79)
\end{exe}

Oblique participles (§\ref{sec:oblique.participle}) are also compatible with the superlative adverb, as shown in (\ref{ex:stu.WsAdAn}).

\begin{exe}
\ex \label{ex:stu.WsAdAn}
\gll  [stu ɯ-sɤ-dɤn] nɯ stɤmku nɯra ŋu-nɯ.   \\
most \textsc{3sg}.\textsc{poss}-\textsc{obl}:\textsc{pcp}-be.many \textsc{dem} grassland \textsc{dem}:\textsc{pl} be:\textsc{fact}-\textsc{pl}  \\
\glt `The place where it is most numerous is the grasslands.' (19-qachGa mWntoR, 24-25)
\end{exe}


 \subsection{Possessed participle} \label{sec:possessed.superlative}
Another possibility to express superlative meaning is with a stative verb in subject participial form with a third plural possessive marker (§\ref{sec:subject.participle.possessive}), as in (\ref{ex:thamtCAt.GW.nWkWmpCAr}), where the headless participial relative in square brackets, literally meaning `the beautiful one (among/of) all birds' is to be understood as `the most beautiful of all birds.' 

 \begin{exe} 
\ex \label{ex:thamtCAt.GW.nWkWmpCAr}
\gll tɕe [pɣa tʰamtɕɤt ɣɯ nɯ-kɯ-mpɕɤr] nɯ rmɤβja ɲɯ-ŋu.  \\
\textsc{lnk} bird all \textsc{gen} \textsc{3pl}.\textsc{poss}-\textsc{sbj}:\textsc{pcp}-be.beautiful \textsc{dem} peacock \textsc{sens}-be \\
\glt `The peacock is the most beautiful of all birds.' (24-ZmbrWpGa, 84)
\end{exe}

This construction is only attested with \japhug{mpɕɤr}{be beautiful} and \japhug{mna}{be better}.

 \subsection{Negative existential} \label{sec:negative.existential.superlative}
Another way of expressing superlative meaning in Japhug is by means of a negative existential verb (§\ref{sec:suppletive.negative}, §\ref{sec:negation.existential}) combined with a participial relative of \japhug{fse}{be like}, as in (\ref{ex:kWfse.me}). This construction is a particular use of the equative construction described in §\ref{sec:svc.equative}.


 \begin{exe}
\ex \label{ex:kWfse.me}
\gll ama, a-pi kʰu nɯ tɕʰindʐa ku-tɯ-nɤpʰɯpʰɣo tɕe [nɤʑo kɯ-fse] kɯ-sɤɣ-mu me	\\
\textsc{surprise} \textsc{1sg}.\textsc{poss}-elder.sibling tiger \textsc{dem} why \textsc{prs}-2-\textsc{distr}:flee \textsc{lnk} \textsc{2sg}  \textsc{sbj}:\textsc{pcp}-be.like   \textsc{sbj}:\textsc{pcp}-\textsc{prop}-fear  not.exist:\textsc{fact} \\
\glt `Brother tiger, why are you running away like that, you are the most dreadful [animal].' (literally: `There is no one dreadful like you.') (2005 khu, 25)
\end{exe}

This construction is potentially ambiguous: the clause \forme{$X$ kɯ-fse kɯ-sɤɣ-mu me} can be interpreted as meaning either  `$X$ is the most dreadful thing' or `there is nothing dreadful that is like $X$'. 

It is possible in some cases to use orientation preverbs to disambiguate between these two meanings. In example (\ref{ex:tusWza}),\footnote{Example  (\ref{ex:tusWza}) is translated from \ch{夜莺,我再熟悉不过了}{yèyīng, wǒ zài shúxī bùguòle}{The nightingale, I am quite familiar with it}, with the  with the construction \zh{再……不过}{zài ... bùguò} involving a negated surpass comparative, and the negative existential verb here is possibly a case of calque. However, Chinese influence cannot be a factor in the  use of orientation preverbs in (\ref{ex:tusWza}) and (\ref{ex:pjWsWza}). } the verbs \japhug{tso}{understand} and \japhug{sɯz}{know} in the superlative construction take the \textsl{upwards} prefix \forme{tu-} instead of the expected \textsc{eastwards} (\forme{ku-tso-a} \textsc{ipfv}-understand-\textsc{1sg}) and \textsc{downwards} (\forme{pjɯ-sɯz-a} \textsc{ipfv}-know-\textsc{1sg}) prefixes that they normally select.

\begin{exe}
\ex \label{ex:tusWza}
\gll aʑo [nɯ kɯ-fse] ʑo maka tu-tso-a me, tu-sɯz-a me \\
\textsc{1sg} \textsc{dem} \textsc{sbj}:\textsc{pcp}-be.like \textsc{emph} at.all \textsc{ipfv}:\textsc{up}-understand-\textsc{1sg} not.exist:\textsc{fact} \textsc{ipfv}:\textsc{up}-know-\textsc{1sg} not.exist:\textsc{fact} \\
\glt `This is what I know best.' (literally: `There is nothing that I understand, that I know like that.' 140519 yeying-zh, 62)
\end{exe}

With the \textsc{upwards} prefix \forme{tu-} as in (\ref{ex:tusWza}), only the superlative interpretation is possible, while with the \textsc{downwards} prefix \forme{pjɯ-} as in (\ref{ex:pjWsWza}) the superlative interpretation is excluded, and only the negative existential one is found.

\begin{exe}
\ex \label{ex:pjWsWza}
\gll aʑo [nɯ kɯ-fse pjɯ-sɯz-a] me \\
\textsc{1sg} \textsc{dem} \textsc{sbj}:\textsc{pcp}-be.like \textsc{ipfv}:\textsc{down}-know-\textsc{1sg} not.exist:\textsc{fact} \\
\glt `I know of no such thing.' (elicited)
\end{exe}

I interpret this difference as a matter of semantic scope. In (\ref{ex:tusWza}),  the clause \forme{nɯ kɯ-fse} `like that' is outside of the scope of the negation, and the negation applies to the minimal relative clauses \forme{tu-tso-a} `(that) I understand' (§\ref{sec:semi.tr.relativization}) and \forme{tu-sɯz-a} `(that) I know' (§\ref{sec:monotransitive.object.relativization}) exclusively. 

With the \textsc{downwards} prefix \forme{pjɯ-} on \japhug{sɯz}{know} as in (\ref{ex:pjWsWza}), the scope of the negation is different: it applies to the whole constituent indicated between square brackets (`there is nothing like that that I know').

This contrast cannot however be generalized to all verbs; more research is necessary to ascertain the extent and the functional explanation for this puzzling phenomenon.

Instead of \forme{kɯ-fse}, egressive postpositions such as \japhug{ɕaŋtaʁ}{up from} (§\ref{sec:egressive},  §\ref{sec:egressive.comparative}) can also be used in a superlative construction as in (\ref{ex:CaNtaR.kWcha.me}).

\begin{exe}
\ex \label{ex:CaNtaR.kWcha.me}
\gll  aʑo ɕaŋtaʁ kɯ-cʰa kɯ-rkaŋ me \\
\textsc{1sg} up.from \textsc{sbj}:\textsc{pcp}-can  \textsc{sbj}:\textsc{pcp}-robust not.exist:\textsc{fact} \\
\glt `I am the most able one, the most robust one.' (literally: `There is no one that is more robust/able than me') (140425 shizi huli he lu-zh, 28)
\end{exe}