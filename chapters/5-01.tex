\chapter{Simple clauses} \label{sec:basic.syntax}

This chapter focuses on the syntax of simple clauses, that is clauses which do not involve multiclausal constructions.

Five topics are discussed: word order within the clause (§\ref{sec:word.order}), sentential adverbs (§\ref{sec:sentential.adverbs}), verbless sentences (§\ref{sec:non.verbal.predicates}), noun-verb collocations (§\ref{sec:light.verb}) and copulas (§\ref{sec:copula.existential}).

\section{Word order} \label{sec:word.order}
With a few exceptions (§\ref{sec:non.verbal.predicates}), Japhug sentences require a finite verb to be complete, while all other constituents are optional. The main verb is generally sentence-final: only a handful of words (§\ref{sec:postverbal.adv}) and dislocated constituents (§\ref{sec:right.dislocation}), can occur after it. 

\subsection{Basic word order} \label{sec:basic.word.order}
This section describes the word order patterns involving verb and core arguments attested in the corpus (with the exception of right dislocation, §\ref{sec:right.dislocation}) when all arguments are overt. The question of non-overt arguments is treated in §\ref{sec:overt.non.overt}, and the position of adverbials in §\ref{sec:sentential.adverbs}.

\subsubsection{Intransitive verbs} \label{sec:word.order.intr}
When overt, core arguments and adjuncts are located before the verb. In (\ref{ex:RnaRna.YAsindZi}) for instance, the intransitive subject \forme{ɯ-mu ɯ-wa ni} `his mother and his father' (§\ref{sec:dyads}) in the dual is placed before the adverb \japhug{ʁnaʁna}{both} and the main verb \forme{ɲɤ-si-ndʑi} `they$_{du}$ died'. The dual suffix on the verb here indexes the number of the intransitive subject (§\ref{sec:intr.23}).


\begin{exe}
	\ex \label{ex:RnaRna.YAsindZi}
	\gll [ɯ-mu ɯ-wa ni] ʁnaʁna ɲɤ-si-ndʑi \\
	\textsc{3sg}.\textsc{poss}-mother 	\textsc{3sg}.\textsc{poss}-father du both \textsc{ifr}-die-\textsc{du} \\ 
	\glt `His parents had both died.' (150828 donglang)
(\japhdoi{0006312\#S8})
\end{exe}

In the case of semi-transitive verbs (§\ref{sec:semi.transitive}), which have two absolutive arguments, it is very rare to have both an overt intransitive subject and an overt semi-object (except if the semi-object is a complement clause, see §\ref{sec:complement.clause.case.marking}). The semi-object can be follow (\ref{ex:kWkWra.YWrganW}) or precede (\ref{ex:pGa.ra.mArganW}) the subject, but in the second case there is a pause, indicative of left dislocation.

\begin{exe}
\ex \label{ex:kWkWra.YWrganW}
\gll kɯkɯra <xuexiao> ɲɯ-rga-nɯ. \\
\textsc{dem}.\textsc{prox}:\textsc{pl} school \textsc{sens}-like-\textsc{pl} \\
\glt `Those (the children) like school.' (conversation, 14-05-10)
\end{exe}

\begin{exe}
\ex \label{ex:pGa.ra.mArganW}
\gll  tɤɕi nɯ, pɣa ra nɯstʰi mɤ-rga-nɯ. \\
barley \textsc{dem} bird \textsc{pl} that.much \textsc{neg}-like:\textsc{fact}-\textsc{pl} \\
\glt `The barley, the birds don't like it so much.' (23-pGAYaR)
(\japhdoi{0003606\#S23})
\end{exe}

 With intransitive verbs selecting a locative phrase (§\ref{sec:intr.goal}), the locative phrase is generally located between the intransitive subject and the verb (\ref{ex:SOblV}). 

\begin{exe}
\ex \label{ex:SOblV}
\gll [ɯ-pi ni] tʂu kɯ-wxti nɯtɕu jo-ɕe-ndʑi \\
\textsc{3sg}.\textsc{poss}-elder.sibling \textsc{du} road \textsc{sbj}:\textsc{pcp}-be.big \textsc{dem}:\textsc{loc} \textsc{ifr}-go-\textsc{du} \\
\glt `His two elder brothers went [along] the big road.' (2003qachga)
\end{exe}

However, it is possible to put the locative phrase before the subject, especially when it refers to the source of the motion (§\ref{sec:motion.verbs}), as in (\ref{ex:OblSV}) and (\ref{ex:OblSV2}).

\begin{exe}
\ex \label{ex:OblSV}
\gll tɯmɯnɤmkʰa nɯtɕu [qro χsɯm] pjɯ-ɣi-nɯ \\
heaven \textsc{dem}:\textsc{loc} pigeon three \textsc{ipfv}:\textsc{down}-come-\textsc{pl} \\
\glt `(Every day), three pigeon came down from heaven.' (02-deluge)
(\japhdoi{0003376\#S48})
\end{exe}

\begin{exe}
\ex \label{ex:OblSV2}
\gll mtsʰu ɣɯ ɯ-ŋgɯ nɯtɕu [mtsʰoʁlaŋ] to-nɯ-ɬoʁ  \\
lake \textsc{gen} \textsc{3sg}.\textsc{poss}-in \textsc{dem}:\textsc{loc} water.monster \textsc{ifr}-\textsc{auto}-come.out \\
\glt `A water monster came out of the lake.' (2011-04-smanmi)
\end{exe}

Likewise, the standard of comparison (marked by the postposition \forme{sɤz}, §\ref{sec:comparative}) can either precede or follow the intransitive subject (§\ref{sec:sAz.kW}).

Oblique phrases with the relator noun \japhug{ɯ-taʁ}{on, above} selected by the main verb (§\ref{sec:WtaR}) are located between the intransitive subject and the verb, as in (\ref{ex:tArAku.WtaR.wuma.pe}) and (\ref{ex:ZmbrW.WtaR.karindZi}).\footnote{
With the motion verb \japhug{ɕe}{go}, in (\ref{ex:ZmbrW.WtaR.karindZi}) in stem II \forme{ari}, \forme{ɯ-taʁ} specifically indicates the means of transportation (§\ref{sec:WtaR}). }

\begin{exe}
\ex \label{ex:tArAku.WtaR.wuma.pe}
\gll  [ɯ-ɣli nɯ] li tɤ-rɤku ɯ-taʁ wuma ʑo pe \\
\textsc{3sg}.\textsc{poss}-dung \textsc{dem} again \textsc{indef}.\textsc{poss}-crop \textsc{3sg}.\textsc{poss}-on really \textsc{emph} be.good:\textsc{fact} \\
\glt `Its dung is good (as fertilizer) for the crops.' (05-qaZo) (\japhdoi{0003404\#S89})
\end{exe}

\begin{exe}
\ex \label{ex:ZmbrW.WtaR.karindZi}
\gll tɕendɤre [ʑɤni pɣa ni] ʑmbrɯ ɯ-taʁ kɤ-ari-ndʑi ɲɯ-ŋu. \\
lnk \textsc{3du} bird \textsc{du} boat \textsc{3sg}.\textsc{poss}-on \textsc{aor}:\textsc{east}-go[II] \textsc{sens}-be \\
\glt `The two of them [including the bird] departed on the boat.' (2005 Norbzang)
\end{exe} 

Unlike intransitive subjects, essive adjuncts (§\ref{sec:essive.abs}) are closer to the verb than oblique arguments, as shown by \japhug{tɯrme}{person} in (\ref{ex:ataR.tWrme.penW}), which is not an intransitive subject (`the people are nice to me') but an essive phrase `as people' (see also \ref{ex:tWGli.Wuma.Zo.pe}, §\ref{sec:essive.abs}).

\begin{exe}
\ex \label{ex:ataR.tWrme.penW}
\gll tɕe <zhoujikong> tɕe mɤʑɯ li a-taʁ tɯrme pe-nɯ ma \\
\textsc{lnk} center.for.disease.control \textsc{lnk} even.more again \textsc{1sg}.\textsc{poss}-on person be.good:\textsc{fact}-\textsc{pl} \textsc{lnk} \\
\glt `At the district center for disease control, they are also very nice to me (as people).' (140501 tshering skyid)
(\japhdoi{0003902\#S139})
\end{exe}

 
\subsubsection{Monotransitive verbs} \label{sec:monotransitive.word.order}
When the main verb is transitive, the transitive subject (with the ergative postposition, §\ref{sec:A.kW}) is found before the direct object (in absolutive form, §\ref{sec:absolutive.P}), as in (\ref{ex:SOV.pjAmtonW}). 

\begin{exe}
\ex \label{ex:SOV.pjAmtonW}
\gll kɯ-ɣɤrʁaʁ nɯra kɯ qartsʰaz ɯ-pɯ nɯ pjɤ-mto-nɯ \\
\textsc{sbj}:\textsc{pcp}-hunt \textsc{dem}:\textsc{pl} \textsc{erg} deer \textsc{3sg}.\textsc{poss}-little \textsc{dem} \textsc{ifr}-see-\textsc{pl} \\
\glt `The hunters saw the little deer.' (140429 jiedi-zh)
\end{exe}

The object of a transitive verb only rarely precedes the subject, except for object complement clauses (§\ref{sec:complement.word.order}), and examples exhibiting this order are all analyzable as left-dislocated topicalized constituents, as in (\ref{ex:OSV.thanWtsWm}).

\begin{exe}
\ex \label{ex:OSV.thanWtsWm}
\gll a-tɕɯ, ji-rɣa ra nɯ-mbala kɯ tʰa-nɯtsɯm tɕe nɯnɯ ɯ-kɯ-ɕar ɕe-tɕi \\
\textsc{1sg}.\textsc{poss}-son \textsc{1pl}.\textsc{poss}-neighbour \textsc{pl} \textsc{3pl}.\textsc{poss}-ox \textsc{erg} \textsc{aor}:3\fl{}3-\textsc{auto}-take.away \textsc{lnk} \textsc{dem} \textsc{3sg}.\textsc{poss}-\textsc{sbj}:\textsc{pcp}-search go:\textsc{fact}-\textsc{1du} \\
\glt `My son$_i$, the neighbour's ox has taken him$_i$ away, and we are looking for him$_i$.' (tWJo 2012)
(\japhdoi{0004089\#S40})
\end{exe}

In (\ref{ex:OSV.pjAndo}), the object \forme{ɯ-mpʰɯz nɯnɯ} is topicalized, and in addition the ergative phrase \forme{tɤjpɣom kɯ} is not a real agent.

\begin{exe}
\ex \label{ex:OSV.pjAndo}
\gll  ɯ-mpʰɯz nɯnɯ tɤjpɣom kɯ pjɤ-ndo \\
\textsc{3sg}.\textsc{poss}-bottom \textsc{dem} ice \textsc{erg} \textsc{ifr}-take \\
\glt `His (the leopard's) bottom, it had been caught in the ice.' (140427 qala)
(\japhdoi{0003852\#S34})
\end{exe}
 
The object-subject-verb order is however found to focalize the subject, together with a postverbal copula (§\ref{sec:focalization.final.copula}) as in  (\ref{ex:OSV.janWtsWm}), where the transitive subject \forme{nɯnɯ kɯ} follows the object (Chunjie). 

 
\begin{exe}
\ex \label{ex:OSV.janWtsWm}
\gll <chunjie> \textbf{nɯnɯ} \textbf{kɯ} ja-nɯ-tsɯm ŋu tʰaŋ \\
\textsc{anthr} \textsc{dem} \textsc{erg} \textsc{aor}:3\flobv{}-\textsc{vert}-take.away be:\textsc{fact} \textsc{sfp} \\
\glt `It is probably him who took away Chunjie.' (150825 huluwa-zh)
(\japhdoi{0006346\#S115})
\end{exe}

Note that without intonation, (\ref{ex:OSV.janWtsWm}) is ambiguous, as [<chunjie> \forme{nɯnɯ kɯ}] could also be analyzed as a single constituent (since personal names can optionally take demonstrative determiners, §\ref{sec:personal.names.modifiers}), and in this case the translation of the sentence would be `Chunjie probably took him.' The above translation however is the correct one in the context of the story.


\subsubsection{Secundative verbs} \label{sec:secundative.word.order}
The recipient (direct object) generally precedes the theme (semi-object) of secundative verbs (§\ref{sec:ditransitive.secundative}) as in (\ref{ex:WRi.tWmgo.YAmbi}).


\begin{exe}
\ex \label{ex:WRi.tWmgo.YAmbi}
\gll ɯ-ʁi tɯmgo ɲɤ-mbi \\
\textsc{3sg}.\textsc{poss}-younger.sibling food \textsc{ifr}-give \\
\glt `He gave food to his younger brother.' (nyima 2003-2)
\end{exe}

In noun-verbs collocations where the verb is secundative such as \forme{tɯ-nɯ+jtsʰi} `breastfeed' as in (\ref{ex:WpW.tWnW.Yjtshi}), placing the recipient (\forme{ɯ-pɯ}) after the them (\forme{tɯ-nɯ}) is never attested.

\begin{exe}
\ex \label{ex:WpW.tWnW.Yjtshi}
\gll ɯ-pɯ tɯ-nɯ ɲɯ-jtsʰi kɯnɤ ʑo \\
\textsc{3sg}.\textsc{poss}-young \textsc{indef}.\textsc{poss}-breast \textsc{ipfv}-give.to.drink also \textsc{emph} \\
\glt `Even when [the monkey mother] nurses her baby,...' (19-GzW)
(\japhdoi{0003536\#S25})
\end{exe}


However, with the verb \japhug{mbi}{give} there are examples of the opposite order as in (\ref{ex:kWlAG.nW.nambi}), where the theme (\forme{qajɣi kɤ-kɯ-ɕke nɯra} `pieces of bread that had burnt') precedes the recipient (\forme{kɯ-lɤɣ nɯ} `the shepherd').


\begin{exe}
\ex \label{ex:kWlAG.nW.nambi}
\gll  tɕʰeme nɯ kɯ qajɣi kɤ-kɯ-ɕke nɯra kɯ-lɤɣ nɯ na-mbi \\
girl \textsc{dem} \textsc{erg} bread \textsc{aor}-\textsc{sbj}:\textsc{pcp}-burn \textsc{dem}:pl \textsc{sbj}:\textsc{pcp}-herd \textsc{dem} \textsc{aor}:3\flobv{}-give \\
\glt `The girl gave the shepherd pieces of bread that had burnt (to eat).' (2003 Kunbzang)
\end{exe}

\subsubsection{Causative verbs} \label{sec:causative.word.order}
Causative verbs (§\ref{sec:sig.causative}) derived from plain intransitive verbs behave like monotransitive verbs (§\ref{sec:monotransitive.word.order}), but those derived from transitive (§\ref{sec:ditransitive.causative}) or semi-transitive (§\ref{sec:semi.transitive.causative}) verbs are ditransitive.

Causative derivations from transitive verbs have a causee (§\ref{sec:causee.kW}), whose syntactic status is intermediate between subject and object: it can be marked with the ergative, but is indexed as if it were an object if it is first or second person while the object is third person (§\ref{sec:ditransitive.causative}). The causee in the ergative is normally placed before the object (\ref{ex:Wmbro.kW.pjAzrAtCaR}), even when it serves as instrument (§\ref{sec:sig.caus.instrumental}) as in (\ref{ex:ZmbrWBJaj.kW.tuwGsWCmi}).


\begin{exe}
\ex \label{ex:Wmbro.kW.pjAzrAtCaR}
\gll ɯ-mbro kɯ qapri tɯ-rdoʁ nɯ pjɤ-z-rɤtɕaʁ \\
\textsc{3sg}.\textsc{poss}-horse \textsc{erg} snake one-piece \textsc{dem} \textsc{ifr}-\textsc{caus}-trample \\
\glt `He made his horse trample one of the [two] snakes.' (smanmi 2003)
\end{exe}

\begin{exe}
\ex \label{ex:ZmbrWBJaj.kW.tuwGsWCmi}
\gll ``ʑmbrɯ-βɟaj nɯnɯ kɯ cʰa nɯ tú-wɣ-sɯ-ɕmi tɕe cʰa mɯm" tu-ti-nɯ pɯ-ŋu ma \\
boat-oar \textsc{dem} \textsc{erg} alcohol \textsc{dem} \textsc{ipfv}-\textsc{inv}-\textsc{caus}-mix \textsc{lnk} alcohol be.tasty:\textsc{fact} \textsc{ipfv}-say-\textsc{pl} \textsc{pst}.\textsc{ipfv}-be \textsc{lnk} \\
\glt `People used to say that if one mixes the alcohol [using] a boat oar, it is tasty.' (31-cha)
(\japhdoi{0003764\#S41})
\end{exe}

The object can be located before the causee, but it is always a left dislocated constituent, followed by a pause as \forme{ɯ-ɕnɤz nɯnɯ} `its head/extremity' in (\ref{ex:WmtChi.kW.kusWndAm}).

\begin{exe}
\ex \label{ex:WmtChi.kW.kusWndAm}
\gll  qandʐe ɣɯ, (...) ɯ-ɕnɤz nɯnɯ, ɯ-mtɕʰi kɯ ku-sɯ-ndɤm tɕe \\
earthworm \textsc{gen} {  } \textsc{3sg}.\textsc{poss}-extremity \textsc{dem} \textsc{3sg}.\textsc{poss}-mouth \textsc{erg} \textsc{ipfv}-\textsc{caus}-take[III] \textsc{lnk} \\
\glt `[The earthworm's head] extremity, [the crow] takes it with its beak.' (140511 qajdo)
(\japhdoi{0003955\#S11})
\end{exe}

In the case of causative verbs from semi-transitive verbs (§\ref{sec:semi.transitive.causative}), such as \japhug{sɤrmi}{give a name} (from \japhug{rmi}{be called}, §\ref{sec:sig.caus.irregular.other}),  the direct object (the person to whom a name is given) is always located before the semi-object (the name given), as in (\ref{ex:ndZitCW.tosArminW}).

\begin{exe}
\ex \label{ex:ndZitCW.tosArminW}
\gll tɕe ndʑi-tɕɯ nɯ ɲimawozɤr to-sɤrmi-nɯ \\
\textsc{lnk} \textsc{3du}.\textsc{poss}-son \textsc{dem}  \textsc{anthr} \textsc{ifr}-give.a.name-\textsc{pl} \\
\glt `They called their son Nyima 'Odzer.' (2011-05-nyima)
\end{exe}

\subsubsection{Indirective verbs} \label{sec:indirective.word.order}
Dative arguments (§\ref{sec:dative}) of indirective verbs (§\ref{sec:ditransitive.indirective}) follow the transitive subject as in (\ref{ex:kW.WCki.object}).\footnote{For reason of place, the theme (the reported speech clause) is not reproduced here and simply indicated by (...), but corresponds to example (\ref{ex:pWtWnWGenW.nWtCu} in §\ref{sec:aor.temporal}). }

\begin{exe}
\ex \label{ex:kW.WCki.object}
\gll  iɕqʰa nɯnɯ tɤ-tɕɯ nɯ kɯ [ɯ-mu nɯ ɯ-ɕki] (...) to-ti. \\
the.aforementioned \textsc{dem} \textsc{indef}.\textsc{poss}-son \textsc{dem} \textsc{erg} \textsc{3sg}.\textsc{poss}-mother \textsc{dem} \textsc{3sg}.\textsc{poss}-\textsc{dat} {  } \textsc{ifr}-say \\
\glt `The boy said to his mother (...).' (2014-kWlAG)
\end{exe}

Dative phrases can be located either after the theme or before it, as shown by examples (\ref{ex:tAtsxu.WCki}) and (\ref{ex:WCki.tAtsxu}) redundantly describing the same action in the same story. 

\begin{exe}
\ex 
\begin{xlist}
\ex \label{ex:tAtsxu.WCki}
\gll  <alading> nɯ kɯ, nɤkinɯ, iɕqʰa tɤtʂu nɯ [ɯ-mu ɯ-ɕki] ɲɤ-kʰo tɕe, \\
\textsc{anthr} \textsc{dem} \textsc{erg} \textsc{filler} the.aforementioned lamp \textsc{dem} \textsc{3sg}.\textsc{poss}-mother  \textsc{3sg}.\textsc{poss}-\textsc{dat} \textsc{ifr}-give \textsc{lnk} \\
\glt `Aladin gave the lamp to his mother.' (140511 alading-zh)
(\japhdoi{0003953\#S159})
\ex \label{ex:WCki.tAtsxu}
\gll  [ɯ-mu ɯ-ɕki] tɤtʂu nɯ ɲɤ-kʰo. \\
\textsc{3sg}.\textsc{poss}-mother \textsc{3sg}.\textsc{poss}-\textsc{dat} lamp \textsc{dem} \textsc{ifr}-give \\
\glt `He gave his mother the lamp.' (140511 alading-zh)
(\japhdoi{0003953\#S161})
\end{xlist}
\end{exe}

The first pattern (object-dative) is rarer than the second one, but attested for all indirective verbs, including verbs of speech, as in (\ref{ex:O.ndZiphe}), where the reported speech clause occurs before the dative phrase (compare with \ref{ex:kW.WCki.object} above, the more common pattern).

\begin{exe}
\ex \label{ex:O.ndZiphe}
\gll [nɯ-kɯ-lɤɣ lu-ɕe-a ɯ-ɲɯ́-pe a-tɤ-tɯ-ti ra] kɯ-wxti ni ndʑi-pʰe ti ɲɯ-ŋu\\
\textsc{2du}.\textsc{poss}-\textsc{sbj}:\textsc{pcp}-herd \textsc{ipfv}:\textsc{upstream}-go-\textsc{1sg} \textsc{qu}-\textsc{sens}-be.good \textsc{irr}-\textsc{pfv}-2-say be.needed:\textsc{fact}  \textsc{sbj}:\textsc{pcp}-be.big \textsc{du} \textsc{3du}-\textsc{dat} say:\textsc{fact} \textsc{sens}-be\\
\glt `He told the two elder [sisters] `Ask [your parents] whether it would be appropriate for me to go herding [cattle] for you'.' (2003 Kunbzang)
\end{exe}

Most oblique arguments also follow the subject and precede the object. It is the case of the goal argument marked by the relator \forme{ɯ-taʁ} (§\ref{sec:WtaR}) in (\ref{ex:kW.WtaR.object}).

\begin{exe}
\ex \label{ex:kW.WtaR.object}
\gll  kɯ-ɣɤrʁaʁ kɯ [pri ɯ-taʁ] tɯdi to-lɤt \\
\textsc{sbj}:\textsc{pcp}-hunt \textsc{erg} bear \textsc{3sg}.\textsc{poss}-on arrow \textsc{ifr}-release \\
\glt `The hunter shot an arrow at the bear.' (elicited)
\end{exe}

\subsection{Overt and non-overt arguments} \label{sec:overt.non.overt}
Sentences with two or three overt arguments such as those discussed in §\ref{sec:basic.word.order} are a minority in Japhug. Most clauses comprise a verb with only one, or even without any overt arguments.

\subsubsection{Core arguments} \label{sec:nonovert.core.arguments}
Non-overt core arguments, including subjects and objects, are always definite, and anaphorically refer to a previously mentioned entity. For instance, example (\ref{ex:tandza}) with a bare verb necessarily means `S/he/it ate it', and cannot be interpreted as `S/he/it ate/had a meal' with indefinite object or as `someone ate it' with indefinite subject.

\begin{exe}
\ex \label{ex:tandza}
\gll ta-ndza \\
\textsc{aor}:3\flobv{}-eat \\
\glt `S/he/it ate it.' (elicited)
\end{exe}

To express indefinite core arguments, non-overtness is an option only in the case of labile verbs (§\ref{sec:labile.tr-intr}), but requires the conversion of the verb to the intransitive conjugation. Only a handful of verbs are labile (§\ref{sec:lability.morphosyntax}, §\ref{sec:lability.categories}), and for the rest of the verbs either an indefinite pronoun (§\ref{sec:indef.pro}) or a valency-decreasing derivation such as antipassive (§\ref{sec:antipassive}), passive (§\ref{sec:passive}) or proprietive (§\ref{sec:proprietive}) must be used instead.

Semi-objects (§\ref{sec:semi.object}), like objects, are interpreted as definite where non-overt, as shown by (\ref{ex:mWjpABJAt}) with the semi-transitive verb \japhug{βɟɤt}{get, obtain} (§\ref{sec:semi.transitive}, §\ref{sec:antipassive.t}).

\begin{exe}
\ex \label{ex:mWjpABJAt}
\gll tɕʰemɤpɯ nɯ kɯ ɲɤ́-wɣ-nɯsɯkʰo tɕe mɯ-pjɤ-βɟɤt. \\
girl \textsc{dem} \textsc{erg} \textsc{ifr}-\textsc{inv}-rob \textsc{lnk} \textsc{neg}-\textsc{ifr}-obtain \\
\glt `The girl had robbed him$_i$ of it$_j$, he$_i$ had not obtained it$_j$.' (150829 taishan zhi zhu-zh)
(\japhdoi{0006350\#S131})
\end{exe}

\subsubsection{Oblique arguments}
Unlike core arguments, oblique arguments can be interpreted as indefinite when non-overt. For instance, when motion verbs like \japhug{ɕe}{go} and \japhug{ɣi}{come} (§\ref{sec:motion.verbs}), lack both an overt goal and a purposive clause (§\ref{sec:purposive.clause.motion.verbs}), two different meanings are possible. First, a previously mentioned goal can be understood as implicit as in (\ref{ex:qapri.tuCe}).

\begin{exe} 
\ex \label{ex:qapri.tuCe}
\gll qapri tu-ɕe ri mɯ́j-cʰa tɕe \\
snake \textsc{ipfv}:\textsc{up}-go also \textsc{neg}:\textsc{sens}-can \textsc{lnk} \\
\glt `Snakes cannot go up (there in the chimney) either.' (22-kumpGatCW)
(\japhdoi{0003590\#S61})
\end{exe} 

Second, there is a minority\footnote{However, when the motion verb takes a supine purposive clause, the goal is very frequently non-overt and indefinite, as can be observed in the majority of the examples in §\ref{sec:purposive.clause.motion.verbs}.  } of cases when no definite goal can be semantically recoverable, as in (\ref{ex:kuCe.nA.YWGi2}) where the two verbs \forme{ku-ɕe} and \forme{ɲɯ-ɣi} indicate back and forth distributed motion, without any clear direction.

\begin{exe}
\ex \label{ex:kuCe.nA.YWGi2}
\gll ɯ-zda ɲɯ-ɕar ɲɯ-sɯsɤm  tɕe tɕendɤre, ki kɯ-fse ku-ɕe nɤ ɲɯ-ɣi \\
\textsc{3sg}.\textsc{poss}-companion \textsc{ipfv}-search \textsc{sens}-think \textsc{lnk} \textsc{lnk} \textsc{dem} \textsc{inf}:\textsc{stat}-be.like \textsc{ipfv}:\textsc{east}-go \textsc{add} \textsc{ipfv}:west-come \\
\glt [That ant] wants to search for its companion, and goes here and fro like that.' (conversation 140501-01)
\end{exe}

Likewise, when the oblique argument of indirective verbs (§\ref{sec:ditransitive.indirective}, §\ref{sec:indirective.word.order}) is non-overt, it can be interpreted as either definite or indefinite.

For instance, the indirective collocation \forme{tɯdi+lɤt} `shoot an arrow', which selects an oblique argument (goal) in \forme{ɯ-taʁ} (§\ref{sec:WtaR}, see \ref{ex:kW.WtaR.object} above in §\ref{sec:indirective.word.order}), lacks an overt oblique in (\ref{ex:non.overt.WtaR.kolAt}). In this example, it is clear that the subject must have aimed before shooting, and therefore that a covert definite oblique is implied, as confirmed by the choice of a specific orientation `towards east' rather than the indefinite orientation preverb.

\begin{exe}
\ex \label{ex:non.overt.WtaR.kolAt}
\gll  nɯnɯ χpɯn nɯ kɯ tɯdi ci ko-lɤt tɕendɤre iɕqʰa,  qapri nɯ ɣɯ ɯ-rpaʁ ʑo to-xtsɯɣ \\
\textsc{dem} monk \textsc{dem} \textsc{erg} arrow \textsc{indef} \textsc{ifr}:\textsc{east}-release \textsc{lnk} \textsc{filler} snake \textsc{dem} \textsc{gen} \textsc{3sg}.\textsc{poss}-shoulder \textsc{emph} \textsc{ifr}-hit \\
\glt `The monk shot an arrow, and it hit the snake in the shoulder.' (150820 qaprANar)
(\japhdoi{0006246\#S22})
\end{exe}

By contrast, in (\ref{ex:tWdi.jAlAtnW}), also without overt goal, the command is to shoot arrows without aiming, simply to see whose arrows will reach the farthest, as confirmed by the choice of the indefinite orientation preverb. The absence of relator nominal clause in \forme{ɯ-taʁ} in this example is to be interpreted here as absence of goal.

\begin{exe}
\ex \label{ex:tWdi.jAlAtnW}
\gll  tɯdi jɤ-lɤt-nɯ \\
arrow \textsc{imp}-release-\textsc{pl} \\
\glt `Shoot arrows!' (elicited, based on a story)
\end{exe}

In (\ref{ex:non.overt.dat.toti}), the verb of speech \japhug{ti}{say} lacks an overt dative argument (§\ref{sec:dative}), and it is obvious in this context not only that there is no definite addressee, but that there is no addressee at all, since \forme{ti} here means `utter (a sound)'. 

\begin{exe}
\ex \label{ex:non.overt.dat.toti}
\gll ɯ-tɯ-mɯɕtaʁ ʑo pjɤ-ŋu ri, ɲɤ-nɤɕqa qʰe, ``ɯtɕʰɯtɕʰɯ" mɯ-to-ti \\
\textsc{3sg}.\textsc{poss}-\textsc{nmlz}:\textsc{deg}-be.cold \textsc{emph} \textsc{ifr}.\textsc{ipfv}-be \textsc{lnk} \textsc{ifr}-endure \textsc{lnk} \textsc{interj} \textsc{neg}-\textsc{ifr}-say \\
\glt `Although it was very cold, he (successfully) endured it, and did not say `How cold!'.' (07-deluge)
(\japhdoi{0003426\#S72})
\end{exe}

To force a definite goal/addressee interpretation, the presence of an overt relator noun with a \textsc{3sg} prefix is necessary, as in (\ref{ex:overt.oblique}).

\begin{exe}
\ex \label{ex:overt.oblique}
\begin{xlist}
\ex \label{ex:WtaR.tWdi.tolAt}
\gll \textbf{ɯ-taʁ} tɯdi to-lɤt \\
\textsc{3sg}.\textsc{poss}-on arrow \textsc{ifr}-release \\
\glt `S/he shot \textbf{at it}.' (elicited)
\ex  \label{ex:WCki.toti2}
\gll \textbf{ɯ-ɕki} (...) to-ti \\
\textsc{3sg}.\textsc{poss}-\textsc{dat} reported.speech \textsc{ifr}-say \\
\textsc{\glt} `S/he told (...) \textbf{to him/her}.' (many examples)
\end{xlist}
\end{exe}

The difference between core vs. oblique arguments is therefore distinct from the parameter of indexibility (§\ref{sec:intr.indexation}, §\ref{sec:tr.indexation}): core arguments are those with obligatory definite interpretation when non-overt, including (transitive and intransitive) subjects, objects and semi-objects, while oblique arguments lack this property.

\subsubsection{Focalization} \label{sec:focalization.overt}
Focalized noun phrases are always overt, if limited to a demonstrative pronoun. They can be formally indistinguishable from non-focalized ones even by intonation, as the demonstrative \forme{nɯ} in the last clause of example (\ref{ex:nW.mAnWXtCWG}).

\begin{exe}
\ex \label{ex:nW.mAnWXtCWG}
\gll mtʰɯmɤr ɣɯ ɯ-zbroŋ nɯnɯ, tu-nɯ-ɬoʁ ɲɯ-ŋu.  tu-mɤmbɯr kɯ-fse ɲɯ-ŋu tɕe \textbf{nɯ} mɤ-naχtɕɯɣ. \\
seal \textsc{gen} \textsc{3sg}.\textsc{poss}-pattern \textsc{dem} \textsc{ipfv}:\textsc{up}-\textsc{auto}-come.out \textsc{sens}-be \textsc{ipfv}-be.protuberant \textsc{sbj}:\textsc{pcp}-be.like \textsc{sens}-be \textsc{lnk} \textsc{dem} \textsc{neg}-be.the.same:\textsc{fact} \\
\glt `The patterns on the seals (called \forme{mtʰɯmɤr}) are coming out, protuberant, that is how they differ [from the other types of seals].' (160706 thotsi)
(\japhdoi{0006133\#S64})
\end{exe}

Focalized first or second person referents require an overt pronoun to surface, as in (\ref{ex:aZo.asni.mWjYaR}).

\begin{exe}
\ex \label{ex:aZo.asni.mWjYaR}
\gll \textbf{nɤʑo} nɤ-sni ɲɯ-ɲaʁ ma \textbf{aʑo} a-sni mɯ́j-ɲaʁ tɕe, ɬɤndʐi wuma nɯ \textbf{nɤʑo} ɲɯ-tɯ-ŋu ma \textbf{aʑo} ɬɤndʐi ɲɯ-maʁ-a \\
\textsc{2sg} \textsc{2sg}.\textsc{poss}-heart \textsc{sens}-be.black \textsc{lnk} \textsc{1sg} \textsc{1sg}.\textsc{poss}-heart  \textsc{neg}:\textsc{sens}-be.black \textsc{lnk} demon real \textsc{dem} \textsc{2sg} \textsc{sens}-2-be \textsc{lnk} \textsc{1sg} demon \textsc{sens}-not.be-\textsc{1sg} \\
\glt `You are evil, not me, you are the real demon, not me.' (2002 lhandzi)
\end{exe}

There is no dearth of strategies to explicitly mark focus, including focus particles (§\ref{sec:focus}), pseudo-cleft constructions (§\ref{sec:pseudo.cleft}) and sentence-final copulas (§\ref{sec:focalization.final.copula}).

\subsection{Right dislocation} \label{sec:right.dislocation}
While word order is rigid in Japhug, and the verb is normally located after all arguments and adjuncts (§\ref{sec:basic.word.order}), left and right dislocation is attested, though accompanied with a specific intonation.

\subsubsection{Dislocated constituents}
All arguments and adjuncts can be right dislocated, including transitive subjects with ergative marking (\ref{ex:A.right.disloc}), possessor with genitive (§\ref{ex:possessor.right.disloc}), locative and time adjuncts (§\ref{ex:time.adj.right.disloc}).

\begin{exe}
\ex \label{ex:A.right.disloc}
\gll sla tu-ngo ŋu tu-ti-nɯ ŋu, \textbf{kɯrɯ} \textbf{ra} \textbf{kɯ}. \\
moon \textsc{ipfv}-be.sick be:\textsc{fact} \textsc{ipfv}-say-\textsc{pl} be:\textsc{fact} Tibetan \textsc{pl} \textsc{erg} \\
\glt `They say that moon is getting sick, the Tibetans.' (29-mWBZi)
(\japhdoi{0003728\#S135})
\end{exe}

%ma nɯnɯ zrɯɣ nɯ li saʁnɤt. tɯ-βri ɯ-taʁ. 

\begin{exe}
\ex \label{ex:possessor.right.disloc}
\gll nɯ ma ɯ-mɯntoʁ kɤ-mto maŋe, \textbf{tɯrgi} \textbf{ɣɯ}. \\
\textsc{dem} apart.from \textsc{3sg}.\textsc{poss}-flower \textsc{inf}-see not.exist:\textsc{sens} fir \textsc{gen} \\
\glt `Apart from that it has no flowers, the fir.' (08-tWrgi)
(\japhdoi{0003464\#S68})
\end{exe}

Sentential adverbs (§\ref{sec:sentential.adverbs}) also undergo dislocation, as shown by (\ref{ex:time.adj.right.disloc}) and (\ref{ex:adv.right.disloc}).

\begin{exe}
\ex \label{ex:time.adj.right.disloc}
\gll paχɕi ndɤre pjɤ-tu, \textbf{kɯɕɯŋgɯ} \\
apples \textsc{advers} \textsc{ifr}.\textsc{ipfv}-exist former.times \\
\glt `(Unlike other fruits,) apples did exist (in our area), in former times.' (07-paXCi)
(\japhdoi{0003430\#S3})
\end{exe}

\begin{exe}
\ex \label{ex:adv.right.disloc}
\gll ɕlaʁ ʑo ɲɤ-me, \textbf{li}. \\
\textsc{idph}(I):immediately \textsc{emph} \textsc{ifr}-not.exist again \\
\glt `She suddenly disappeared, again.' (150907 niexiaoqian-zh)
(\japhdoi{0006262\#S108})
\end{exe}

%mɯ́jsɤmtshɤm wo, koŋla

Subordinate clauses can also be right-dislocated, for instance infinite complement clauses (example \ref{ex:tutWste.kAsAfstWn}, §\ref{sec:complement.morphosyntax}), supine participial clauses (\ref{ex:purposive.right.disloc}) (§\ref{sec:purposive.clause.motion.verbs}) or concessive conditionals (\ref{ex:concessive.cond.right.disloc}) (§\ref{sec:scalar.concessive.conditional}).


\begin{exe}
\ex \label{ex:purposive.right.disloc}
\gll tɕɯχtsi kɤ-ari-j, \textbf{kɯ-nɤmɲo} \\
Cogtse \textsc{aor}:\textsc{east}-go[II]-\textsc{1sg} \textsc{sbj}:\textsc{pcp}-watch \\
\glt `We went to Cogtse, to do some sightseeing.' (conversation)
\end{exe}

\begin{exe}
\ex \label{ex:concessive.cond.right.disloc}
\gll tu-mbri ɯ-kʰɯkʰa ɯ-ʁar nɯ ki ntsɯ tu-ste ɲɯ-ŋu, \textbf{lɤ-zo} \textbf{kɯnɤ}. \\
\textsc{ipfv}-make.noise \textsc{3sg}.\textsc{poss}-while \textsc{3sg}.\textsc{poss}-wing \textsc{dem} \textsc{dem}.\textsc{prox} always \textsc{ipfv}-do.like[III] \textsc{sens}-be \textsc{aor}-land also \\
\glt `It does like this with its wings when it sings, even when it lands.' (23-RmWrcWftsa)
(\japhdoi{0003610\#S106})
\end{exe}

In (\ref{ex:cause.right.disloc}), the consequence clause \forme{ʁmɯrtsɯ nɯ sɤre} `The \textit{Berchemia yunnanensis} is funny' is right-dislocated together with the linker \forme{ma} (here meaning `because'), while the causal clause (§\ref{sec:causal.clauses}) serves as the main clause. This example provides evidence that the linker \forme{ma} can form a syntactic constituent with the preceding clause, even though it is most often prosodically linked to the causal clause following it.

\begin{exe}
\ex \label{ex:cause.right.disloc}
\gll ɣɯjpa kɯ-fse nɯre ri tɕe ɯ-mɯntoʁ ɲɯ-lɤt, fsaqʰe tɕe tɕe ɯ-mat ɲɯ-βze ŋu, \textbf{ʁmɯrtsɯ} \textbf{nɯ} \textbf{sɤre} \textbf{ma}. \\
this.year \textsc{sbj}:\textsc{pcp}-be.like \textsc{dem}:\textsc{loc} \textsc{loc} \textsc{lnk} \textsc{3sg}.\textsc{poss}-flower \textsc{ipfv}-release next.year \textsc{loc} \textsc{lnk} \textsc{3sg}.\textsc{poss}-fruit \textsc{ipfv}-grow[III] be:\textsc{fact} Berchemia.yunnanensis \textsc{dem} be.funny:\textsc{fact} \textsc{lnk} \\
\glt `The \textit{Berchemia yunnanensis} is funny because it blooms this year, but its fruits grow the next year.' (11-qarGW)
(\japhdoi{0003480\#S50})
\end{exe}

The presence of a right dislocated constituent does not preclude an overt constituent with the same syntactic function in the main clause: for instance, in (\ref{ex:A.in.situ.right.disloc}) the ergative phrase \forme{tɤ-wa nɯ kɯ}`the father' and the right-dislocated constituent \forme{ɯ-nmaʁ nɯ kɯ} `her husband' refer to the same person, transitive subject of the sentence.

\begin{exe}
\ex \label{ex:A.in.situ.right.disloc}
\gll ɯʑo srɯnmɯ kɯ-ŋu nɯ \textbf{tɤ-wa} \textbf{nɯ} \textbf{kɯ} mɯ-pjɤ-sɯχsɤl, \textbf{ɯ-nmaʁ} \textbf{nɯ} \textbf{kɯ}. \\
\textsc{3sg} râkshasî \textsc{sbj}:\textsc{pcp}-be \textsc{dem} \textsc{inf}-father \textsc{dem} \textsc{erg} \textsc{neg}-\textsc{ifr}-realize \textsc{3sg}.\textsc{poss}-husband \textsc{dem} \textsc{erg} \\
\glt `The father$_i$ did not realize that she$_j$ was a râkshasî, her$_j$ husband$_i$.' (28-smAnmi)
(\japhdoi{0004063\#S57})
\end{exe}

\subsubsection{The functions of right dislocation} \label{sec:right.dislocation.function}
Right dislocation in Japhug has three main functions, similar to those identified by \citet[§13.7.2]{honkasalo19geshiza} in the Geshiza language.

The main function of right dislocation is that of afterthought, used to identify a referent that has no been mentioned previously (\ref{ex:A.right.disloc}), provide additional side comments (\ref{ex:adv.right.disloc}, \ref{ex:cause.right.disloc}), complementary information (\ref{ex:time.adj.right.disloc}) or even redundant information that can contribute to identify a referent (\ref{ex:A.in.situ.right.disloc}).  

Right dislocation is also used to reactivate a constituent in discourse (\citealt[§13.7.2]{honkasalo19geshiza}), to avoid ambiguity with other referents. In (\ref{ex:YWrko.Wkha}) for instance, the referent \forme{ɯ-kʰa} `its house/shell' is found more than twenty clauses after its previous occurrence.

\begin{exe}
\ex \label{ex:YWrko.Wkha}
\gll tɕe nɯnɯ ɲɯ-rko, \textbf{ɯ-kʰa} \textbf{nɯ}. \\
\textsc{lnk} \textsc{dem} \textsc{sens}-be.hard \textsc{3sg}.\textsc{poss}-house \textsc{dem} \\
\glt `It is hard, its house (i.e. the shell of the snail).' (26-tWcipaR)
(\japhdoi{0003694\#S68})
\end{exe}

Finally, right dislocation can serve to mark emphasis on a topical referent. In (\ref{ex:WmAlAjaR2}) for instance, the noun \forme{ɯ-mɤlɤjaʁ} is both left- and right-dislocated, showing that this instance is neither an afterthought nor a constituent reactivation.

\begin{exe}
\ex \label{ex:WmAlAjaR2}
\gll ɯ-mɤlɤjaʁ nɯ, nɯnɯ, kɯrcɤ-ldʑa jamar ɣɤʑu rca, \textbf{ɯ-mɤlɤjaʁ} \textbf{nɯ}. \\
\textsc{3sg}.\textsc{poss}-limb \textsc{dem} \textsc{dem} eight-piece about exist:\textsc{sens} \textsc{unexp}:\textsc{foc} \textsc{3sg}.\textsc{poss}-limb \textsc{dem}  \\
\glt `Its limbs, it has eight ones, its limbs.' (26-mYaRmtsaR)
\end{exe}

In some cases, the exact function of right dislocation is ambiguous. In (\ref{ex:purposive.right.disloc}) for instance, the purposive phrase could be emphasized by dislocation, but alternatively it is also possible to interpret it as an afterthought.

\section{Sentential adverbs} \label{sec:sentential.adverbs}
This section describes non-derived sentential adverbs, excluding those derived from nouns (§\ref{sec:denominal.adverb}), lexicalized converbs (§\ref{sec:converbs})  ideophones (§\ref{sec:idph}), but including adverbs with non-synchronically transparent etymology and loanwords from Tibetan.
 
\subsection{Tense and aspect} \label{sec:tense.aspect.adverbs}
Absolute tense is mainly expressed by time ordinals (§\ref{sec:time.ordinals}) and other time adverbials derived from nouns (§\ref{sec:time.adv}). There are in addition a few aspectual adverbs that are not transparently derived from nouns.

The adverb \forme{pɤjkʰu} has a permansive meaning `still' when used with a positive verb form (\ref{ex:pAjkhu.pjWnge.cha}).

\begin{exe}
\ex \label{ex:pAjkhu.pjWnge.cha}
\gll iʑo a-mu nɯ tʰamtʰam kɯrcɤsqaptɯɣ tʰɯ-azɣɯt ŋu. pɤjkʰu ji-paʁ pjɯ-nge cʰa. \\
\textsc{1pl} \textsc{1sg}.\textsc{poss}-mother \textsc{dem} now eighty.one \textsc{aor}-arrive be:\textsc{fact} still \textsc{1pl}.\textsc{poss}-pig \textsc{ipfv}-feed[III] can:\textsc{fact} \\
 \glt `Our mother is now eighty-one years old, she can still feed our pigs.' (2010, 9.2)
 \end{exe}
 
 Contrary to what could have been expected, \japhug{pɤjkʰu}{still} is only rarely used with a verb prefixed with the Autive in permansive function (§\ref{sec:autoben.permansive}). Examples like (\ref{ex:pAjkhu.pjAnnWZWB}) with redundant marking of permansive aspect are attested but uncommon.
 
\begin{exe}
\ex \label{ex:pAjkhu.pjAnnWZWB}
\gll mtsʰoʁlaŋ nɯnɯ pɤjkʰu maka ʑo pjɤ-n-nɯʑɯβ ɕti ma \\
water.monster \textsc{dem} still completely \textsc{emph} \textsc{ipfv}.\textsc{ifr}-\textsc{auto}-sleep be.\textsc{aff}:\textsc{fact} \textsc{lnk} \\
\glt `The aquatic monster was still  asleep.' (140508 benling gaoqiang de si xiongdi-zh)
(\japhdoi{0003935\#S184})
  \end{exe}
 
 With a negative predicate, \forme{pɤjkʰu} means `not...yet', as shown by (\ref{ex:pAjkhu.mWpWtsua}) and (\ref{ex:pAjkhu.mWnaCar}).
 
\begin{exe}
\ex \label{ex:pAjkhu.mWpWtsua}
\gll aʑo pɤjkʰu tɯ-sla mɯ-pɯ-tsu-a \\
 \textsc{1sg} still one-month \textsc{neg}-\textsc{pst}.\textsc{ipfv}-pass-\textsc{1sg} \\
 \glt `It has not yet been a month since I have [come here].' (2003 tWxtsa)
\end{exe}

\begin{exe}
\ex \label{ex:pAjkhu.mWnaCar}
\gll ma ɯ-me kɯnɤ ɣnɤsqamnɯz-pɤrme tʰɯ-azɣɯt, tɕe nɯnɯ pɤjkʰu ɯ-χti ra mɯ-na-ɕar ma \\
\textsc{lnk} \textsc{3sg}.\textsc{poss}-daughter also twenty.eight-years.old \textsc{aor}-arrive \textsc{lnk} \textsc{dem} yet \textsc{3sg}.\textsc{poss}-companion \textsc{pl} \textsc{neg}-\textsc{aor}:3\flobv{}-search \textsc{lnk} \\
\glt `His daughter is twenty-two years old, but she does not yet have a companion (husband).' (14-siblings)
(\japhdoi{0003508\#S284})
\end{exe}

This adverb is generally located before the object (examples \ref{ex:pAjkhu.pjWnge.cha}, \ref{ex:pAjkhu.mWnaCar}) but after the intransitive subject (\ref{ex:pAjkhu.pjAnnWZWB}, \ref{ex:pAjkhu.mWpWtsua}).  It can also occur without a main verb, in combination with with the sentence final particle \forme{je} (§\ref{sec:fsp.imp}) as \forme{pɤjkʰu je} `wait', hence the interjection \japhug{pɤkʰije}{wait!} (§\ref{sec:interjections}) with irregular fusion.
  
The reduplicated forms \japhug{ʑɯrɯʑɤri}{progressively} and the rarer \forme{ʑɤrʑɯr},\footnote{It is possible that these forms are borrowed from \tibet{ཞོར་}{ʑor}{incidentally}, though the semantics is unclear. } are frequent with gradable stative verbs to express a progressive change of state as in (\ref{ex:ZWrWZAri.ChWGWrni}), sometimes with incremental initial reduplication of the verb (§\ref{sec:redp.gradual.increase}).


\begin{exe}
\ex \label{ex:ZWrWZAri.ChWGWrni}
\gll ʑɯrɯʑɤri tɕe tɕe cʰɯ-ɣɯrni ŋu. \\
progressively \textsc{lnk} \textsc{lnk} \textsc{ipfv}-be.red be:\textsc{fact} \\
\glt `It progressively becomes red.' (11-qarGW)
(\japhdoi{0003480\#S55})
\end{exe}

They are also found with dynamic verbs, in particular with temporal clauses in \japhug{ɯ-jɯja}{along with} (§\ref{sec:simultaneity}) as in (\ref{ex:ZWrWZAri.YWmbinW}).

\begin{exe}
\ex \label{ex:ZWrWZAri.YWmbinW}
\gll  tɕe tʰɯ-wxti ɯ-jɯja nɤ, (...) kɯ-wxti ra kɯ nɯ-kɤ-ndza, tɕʰi tu-ndza-nɯ kɯ-ŋu nɯ ʑɯrɯʑɤri tɕe ɲɯ-mbi-nɯ. \\
lnk  \textsc{aor}-be.big  \textsc{3sg}.\textsc{poss}-along \textsc{add} {  } \textsc{sbj}:\textsc{pcp}-be.big \textsc{pl} \textsc{erg} \textsc{3pl}.\textsc{poss}-\textsc{obj}:\textsc{pcp}-eat what \textsc{ipfv}-eat-\textsc{pl} \textsc{sbj}:\textsc{pcp}-be \textsc{dem} progressively \textsc{lnk} \textsc{ipfv}-give-\textsc{pl} \\
\glt  `As the child$_j$ grows older, the adults$_i$ progressively give [him/her]$_j$ their$_i$ food, whatever it is they$_i$ eat.' (140426 tApAtso kAnWBdaR)
\end{exe}
 

The highly polyfunctional \forme{ci}, whose functions range from the numeral `one' (§\ref{sec:one.to.ten}) to various indefinite, partitive or non-identity markers (§\ref{sec:ci.someone}, §\ref{sec:other.pro}, §\ref{sec:partitive.pronouns}, §\ref{sec:indef.article}, §\ref{sec:identity.modifier}), is used as an aspectual adverb `once' as in (\ref{ex:qhihihi.ci}) (see also for instance \ref{ex:kWm.ci.thapa}, §\ref{sec:pa.lv}).\footnote{Semelfactive meaning is however more often expressed with a sentential counted noun (§\ref{sec:CN.iterative}) with the `one' prefix (for instance \japhug{tɯ-ɣjɤn}{one time}). } The attenuative adverb \forme{ci} presumably derives from this meaning (§\ref{sec:intensifier.adverbs}).\footnote{Chinese \ch{一下}{yīxià}{one time, a little} has a similar range of meanings. }

\begin{exe}
\ex \label{ex:qhihihi.ci}
\gll ``qʰihihi" ci ta-tɯt ɲɯ-ŋu  \\
\textsc{interj} once \textsc{aor}:3\flobv{}-say[II] \textsc{sens}-be \\
\glt `He said \forme{qʰihihi}.' (2003 qachGa)
\end{exe}

Its reduplicated form \forme{ci ci} has the meaning `sometimes, in some cases', often repeated in two or more clauses in parataxis `sometimes $X$, sometimes $Y$'  as in (\ref{ex:cici.pjWnWmtsWr}).\footnote{See also \ref{ex:tAngo.thWmdW} (§\ref{sec:aor.main}) and \ref{ex:pjWsat.juGWt} (§\ref{sec:coordination}) for representative examples of this adverbial locution. }  In  (\ref{ex:cici.pjWnWmtsWr}), \forme{ci ci} combines with the Autive prefix in spontaneous function (§\ref{sec:autoben.spontaneous}) to indicate the unpredictability of the alternative events.
 
\begin{exe}
\ex \label{ex:cici.pjWnWmtsWr}
\gll  ɯ-me mɯ́j-nɯβdaʁ qʰe tɕendɤre ɯ-me nɯ, nɤkinɯ, ci ci pjɯ-nɯ-mtsɯr, ci ci pjɯ-nɯ-fka  \\
 \textsc{3sg}.\textsc{poss}-daughter \textsc{neg}:\textsc{sens}-take.care \textsc{lnk} \textsc{lnk}  \textsc{3sg}.\textsc{poss}-daughter \textsc{dem} \textsc{filler} one one \textsc{ipfv}-\textsc{auto}-be.hungry one one \textsc{ipfv}-\textsc{auto}-be.full \\
 \glt  `He did not take care of his daughter, and his daughter would sometimes be hungry, sometimes have enough to eat.' (17-lhazgron)
 \end{exe}
 
 The adverb \forme{li} means  `again, like the previous time' as in (\ref{ex:lici.tojGAt}). It can be optionally followed by the additive \forme{nɤ}.
 
\begin{exe}
\ex \label{ex:lici.tojGAt}
\gll tɕendɤre li to-jɣɤt tɕe, smɤnmimitoʁ kuɕana cʰo li pjɤ-rɯkʰɤcɤl-ndʑi.  \\
\textsc{lnk} again \textsc{ifr}:\textsc{up}-turn.around \textsc{lnk}  \textsc{anthr} \textsc{anthr} \textsc{comit} again \textsc{ifr}-discuss-\textsc{du} \\
\glt `He returned again (up there), and had again a discussion with Smanmi Metog Koshana (as he had done previously).' (28-smAnmi)
(\japhdoi{0004063\#S183})
 \end{exe}

In combination with the adverbial \forme{ci}, it can mean `one more, another one' as in (\ref{ex:lici.pWfCAt}) (like Chinese \ch{再}{zài}{again}). Alternatively however, \forme{ci} in context can also be interpreted as an attenuative adverb (§\ref{sec:intensifier.adverbs}).
 
\begin{exe}
\ex \label{ex:lici.pWfCAt}
\gll ama ɯ-tɯ-mpɕɤr nɯ! li ci pɯ-fɕɤt! \\
\textsc{interj} \textsc{3sg}.\textsc{poss}-\textsc{nmlz}:\textsc{deg}-be.beautiful \textsc{sfp} again once \textsc{imp}-tell \\
\glt `Wow, what a beautiful [story]! Tell [me] another one.' (2005 tWJo)
(\japhdoi{0003368\#S41})
\end{exe}
 
 The adverb \japhug{ntsɯ}{always} is used as a temporal universal quantifier `all the time' (\ref{ex:ntsW.tuti}) or `every time' as in (\ref{ex:Yikuku.ntsW.tuti}), a function from which it has become a distributive quantifier (§\ref{sec:raNri}). It is one of the few adverbs that can be used postverbally (§\ref{sec:postverbal.adv}).

\begin{exe}
\ex \label{ex:ntsW.tuti}
\gll  ɯ-skɤt mpɕɤr ma {``qusput qusput"} ntsɯ tu-ti ŋu. \\
\textsc{3sg}.\textsc{poss}-voice be.beautiful:\textsc{fact} \textsc{lnk} onomatopoeia always \textsc{ipfv}-say \textsc{sens}-be \\
\glt `[The cuckoo] has a beautiful song, it says \forme{qusput qusput} all the time.' (24-qro)
(\japhdoi{0003626\#S63})
 \end{exe} 
 
\begin{exe}
\ex \label{ex:Yikuku.ntsW.tuti}
\gll   sɲikuku ʑo nɯ ntsɯ tu-ti ɲɯ-ŋu tɕe, \\
every.day \textsc{emph} \textsc{dem} always \textsc{ipfv}-say \textsc{sens}-be \textsc{lnk} \\
\glt `[The bird comes] everyday and says this every time.'  (2014-kWlAG)
 \end{exe}
 
In addition, \forme{ntsɯ} can indicate a repeated action `again and again' as in (\ref{ex:wojAr.ntsW.toti}), where the repetition is also marked by the additive \forme{nɤ} (§\ref{sec:additive.nA}).

\begin{exe}
\ex \label{ex:wojAr.ntsW.toti}
\gll ``wortɕʰi nɤ wojɤr, ma-tɤ-kɯ-ndza-a" ntsɯ to-ti ɲɯ-ŋu. \\
please \textsc{add} please \textsc{neg}-\textsc{imp}-2\fl{}1-eat-\textsc{1sg} always \textsc{ifr}-say \textsc{sens}-be \\
\glt `He said again and again `Please, don't eat me.'' (140427 bianfu yu huangshulang-zh)
(\japhdoi{0003838\#S11})
\end{exe}

Other temporal adverbs include \japhug{ʁlɤwɯr}{suddenly} from Tibetan \tibet{གློ་བུར་}{glo.bur}{sudden}, used with Aorist or Inferential verb forms to express an abrupt change of state  (\ref{ex:RlAwWr.Zo.tAmNAm}), \japhug{toʁde}{a moment}, which can also mean `suddenly' (example \ref{ex:qandZGi.pjAGi}, §\ref{sec:ifr.evd}), and also  `in a moment' referring to a future event (\ref{ex:jAlAt.je}, §\ref{sec:imp.function}) or `for a moment' (\ref{ex:toRde.kunWnaa}), and \japhug{iɕqʰa}{just now}\footnote{Its meaning is close to Chinese \ch{刚才}{gāngcái}{just now, a moment ago}.} (on which see also §\ref{sec:iCqha}).


\begin{exe}
\ex \label{ex:RlAwWr.Zo.tAmNAm}
\gll maka pɯ-nɤʁaʁ-i, ʁlɤwɯr ʑo ɯ-xtu tɤ-mŋɤm tɕe pɯ-si ɕti  \\
completely \textsc{pst}.\textsc{ipfv}-have.a.good.time-\textsc{1pl} suddenly \textsc{emph} \textsc{3sg}.\textsc{poss}-belly \textsc{aor}-hurt \textsc{lnk} \textsc{aor}-die be.\textsc{aff}:\textsc{fact} \\
\glt `As we were having a party (in the mountain), his belly suddenly started to ache, and he died.' (2012 Norbzang)
(\japhdoi{0003768\#S318})
\end{exe}
 

\begin{exe}
\ex \label{ex:toRde.kunWnaa}
\gll kɯre ri toʁde ku-nɯna-a. \\
\textsc{dem}.\textsc{prox}:\textsc{loc} \textsc{loc} a.moment \textsc{prs}-rest-\textsc{1sg} \\
\glt `I am resting for a moment.' (conversation 2019-09-16)
\end{exe}

\subsection{Quantification} \label{sec:quantification.adverbs}
  This section presents adverbial quantifiers that are not temporal or aspectual markers, as those are treated in the previous section (§\ref{sec:tense.aspect.adverbs}).
 
\subsubsection{Universal quantifiers}  \label{sec:universal.quantification.adverbs}
The universal quantifiers \japhug{kɤsɯfse}{all} and \japhug{lonba}{all} can have scope over a single noun phrase  (§\ref{sec:universal.quant}). When following an intransitive subject or object in preverbal position as in (\ref{ex:XsWm.kAsWfse}), it is not clear whether the scope of the quantifier is on the preceding noun phrase or on the whole sentence.  

\begin{exe}
\ex \label{ex:XsWm.kAsWfse}
 \gll rgɤtpu rgɤnmɯ ni kɯ [kɯki tɤ-pɤtso χsɯm ki] kɤsɯfse ʑo cʰɤ-ɣɤ-wxti-ndʑi. \\
 old.man old.woman \textsc{du} \textsc{erg} \textsc{dem}.\textsc{prox} \textsc{indef}.\textsc{poss}-child three \textsc{dem}.\textsc{prox} all \textsc{emph} \textsc{ifr}-\textsc{caus}-be.big-\textsc{du} \\
\glt `The old man and the old woman raised all these three children.' (140514 huishuohua de niao-zh)
(\japhdoi{0003992\#S58})
\end{exe}
  

The postnominal \japhug{mɯtɕʰimɯrɯz}{all kinds} (§\ref{sec:quantifiers.other}) appears in (\ref{ex:rca.mWtChimWrWz}) stranded from the noun \japhug{tɯ-ŋga}{clothes} by the unexpected/high degree marker \forme{rca} (§\ref{sec:unexpected}), suggesting that it could be analyzed here as a sentential adverb.

\begin{exe}
\ex \label{ex:rca.mWtChimWrWz}
\gll tɯ-ŋga rca mɯtɕʰimɯrɯz cʰɯ́-wɣ-βzu kʰɯ \\
\textsc{indef}.\textsc{poss}-clothes \textsc{unexp}:\textsc{foc} all.kinds \textsc{ipfv}-\textsc{inv}-make be.possible:\textsc{fact} \\
\glt `One can make all kinds of clothes (using it).' (05-qaZo)
(\japhdoi{0003404\#S65})
\end{exe}

\subsubsection{Everywhere} \label{sec:everywhere}
The adverb \japhug{aʁɤndɯndɤt}{everywhere} can be used on its own, but also together with overt locative adjuncts (or goals); it can both precede (\ref{ex:aRAndWndAt.Zo.loc}) or follow it  (\ref{ex:loc.aRAndWndAt.Zo}, \ref{ex:aRAndWndAt.Zo.loc2}). 

 \begin{exe}
\ex \label{ex:loc.aRAndWndAt.Zo}
\gll tɯ-ji ɯ-ŋgɯ aʁɤndɯndɤt ʑo tu-ɬoʁ ɕti. \\
\textsc{indef}.\textsc{poss}-field \textsc{3sg}.\textsc{poss}-in everywhere \textsc{emph} \textsc{ipfv}-come.out be.\textsc{aff}:\textsc{fact} \\
\glt `It grows everywhere in the fields.' (12-ndZiNgri)
(\japhdoi{0003488\#S146})
\end{exe} 

Locative adjuncts used with \forme{aʁɤndɯndɤt} often take the plural \forme{ra} (§\ref{sec:plural.determiners}) to mark approximate location as in (\ref{ex:aRAndWndAt.Zo.loc}) and (\ref{ex:aRAndWndAt.Zo.loc2}).

 \begin{exe}
\ex \label{ex:aRAndWndAt.Zo.loc}
\gll  aʁɤndɯndɤt sɯŋgɯ ra kɯnɤ tu-ɬoʁ ɕti. \\
everywhere forest \textsc{pl} also \textsc{ipfv}-come.out be.\textsc{aff}:\textsc{fact} \\
\glt `It also grows everywhere in the forest.' (14-sWNgWJu)
(\japhdoi{0003506\#S159})
\end{exe} 

 \begin{exe}
\ex \label{ex:aRAndWndAt.Zo.loc2}
\gll  aʁɤndɯndɤt ʑo kʰa ra cʰɯ-rɤpɯ. tɯ-ji ɯ-ngɯ ra cʰɯ-rɤpɯ \\
everywhere \textsc{emph} house \textsc{pl} \textsc{ipfv}-litter \textsc{indef}.\textsc{poss}-field \textsc{3sg}-inside \textsc{pl} \textsc{ipfv}-litter \\
\glt `[Mice] have litter everywhere in the house, in the fields.' (27-spjaNkW)
(\japhdoi{0003704\#S156})
\end{exe} 

There are a a few examples where \forme{aʁɤndɯndɤt} is followed  by the locative postposition \forme{ri} (§\ref{sec:core.locative}) as (\ref{ex:aRAndWndAt.ri}) like a locative noun phrase. However, no sentences with \japhug{aʁɤndɯndɤt}{everywhere} as core argument (like `everywhere is quiet') are found in the corpus, indicating that it would be clumsy to analyze it as a pronoun (§\ref{sec:universal.pronouns}).

 \begin{exe}
\ex \label{ex:aRAndWndAt.ri}
\gll nɯfse ʑo aʁɤndɯndɤt ri tu-nnɯ-ɬoʁ qʰe, ɯ-zrɤm nɯra kɯ-tu maŋe. \\
like.that \textsc{emph} everywhere \textsc{loc} \textsc{ipfv}-\textsc{auto}-come.out \textsc{lnk} \textsc{3sg}.\textsc{poss}-root \textsc{dem}:\textsc{pl} \textsc{sbj}:\textsc{pcp}-exist not.exist:\textsc{sens} \\
\glt `It grows simply like that everywhere, it has no roots.' (20-sWrna)
(\japhdoi{0003564\#S68})
\end{exe} 

When \japhug{aʁɤndɯndɤt}{everywhere} occurs under the scope of negation, it never expresses the meaning `nowhere', as shown by (\ref{ex:aRAndWndAt.me}) and (\ref{ex:aRAndWndAt.juCenW}).

\begin{exe}
\ex \label{ex:aRAndWndAt.me}
\gll stɤmku nɯra, tɯ-ci ɯ-rkɯ nɯra tu ma aʁɤndɯndɤt stʰɯci me \\
plain \textsc{dem}:\textsc{pl} \textsc{indef}.\textsc{poss}-water \textsc{3sg}.\textsc{poss}-side  \textsc{dem}:\textsc{pl} exist:\textsc{fact} \textsc{lnk} everywhere so.much not.exist:\textsc{fact} \\
\glt `It is found in plains, or next to rivers, but it is not found everywhere.' (14-sWNgWJu)
(\japhdoi{0003506\#S50})
\end{exe} 

\begin{exe}
\ex \label{ex:aRAndWndAt.juCenW}
\gll tɕe ɕɤr tɕe cʰɯ-nɯ-ɬoʁ-nɯ tɕe, aʁɤndɯndɤt ju-ɕe-nɯ mɤ-kɯ-kʰɯ \\
\textsc{lnk} night \textsc{lnk} \textsc{ipfv}:\textsc{downstream}-\textsc{auto}-come.out-\textsc{pl} \textsc{lnk} everywhere \textsc{ipfv}-go-\textsc{pl} \textsc{neg}-\textsc{sbj}:\textsc{pcp}-be.\textsc{possible} \\
\glt `[They make it] to prevent [animals] from coming out at night and going everywhere.' (150902 mkhoN)
(\japhdoi{0006300\#S21})
\end{exe} 

The word  \japhug{ŋotɕuŋondɤt}{everywhere} is semantically very close to \japhug{aʁɤndɯndɤt}{everywhere} but rarer; it may also be translated as `in all kinds of places'. It contains a partially reduplicated form of the interrogative pronoun \japhug{ŋotɕu}{where} (§\ref{sec:NotCu}).

 \begin{exe}
\ex \label{ex:NotCuNondAt}
\gll ɕkrɤz ɯ-ŋgɯ tɕi ɲɯ-ɬoʁ, tɯrgi ɯ-ŋgɯ tɕi ɲɯ-ɬoʁ, ʑmbri ɯ-ŋgɯ tɕi ɲɯ-ɬoʁ,  mbraj ɯ-ŋgɯ tɕi ɲɯ-ɬoʁ, tɕe sɤjku sɯŋgɯ nɯra tɕi ɲɯ-ɬoʁ, tɕe ŋotɕuŋondɤt ʑo ɣɤʑu ɕti ri, stɤmku me, sɯŋgɯ ʁɟa ʑo tu-ɬoʁ ɲɯ-ŋu. \\
oak \textsc{3sg}.\textsc{poss}-inside also \textsc{sens}-come.out fir \textsc{3sg}.\textsc{poss}-inside also \textsc{sens}-come.out willow \textsc{3sg}.\textsc{poss}-inside also \textsc{sens}-come.out red.birch \textsc{3sg}.\textsc{poss}-inside also \textsc{sens}-come.out \textsc{lnk} birch  forest \textsc{dem}:\textsc{pl} also \textsc{sens}-come.out \textsc{lnk} everywhere \textsc{emph} exist:\textsc{sens} be:\textsc{aff}  \textsc{lnk} plain whether  forest completely \textsc{emph} \textsc{ipfv}-come.out \textsc{sens}-be \\
\glt `[This mushroom] grows among oaks, among firs, among willows, among red or white birch forests, you find it everywhere, whether on plains or in forest.' (23-mbrAZim) 
(\japhdoi{0003604\#S224})
\end{exe} 


\subsubsection{Restrictive `only, always'} \label{sec:restrictive.adverbs}
 The adverb  \japhug{ʁɟa}{completely}, which can be used to mark restrictive focus on a noun phrase (§\ref{sec:restrictive.focus}), also occurs with scope over the whole sentence in the meaning `all, only, always' as in (\ref{ex:kW.RJa.Zo}), where it follows the ergative \forme{kɯ} (compare with \ref{ex:RJa.kW} in §\ref{sec:restrictive.focus}, where the ergative is located before \forme{ʁɟa}).

\begin{exe}
\ex \label{ex:kW.RJa.Zo}
 \gll tɯ-ŋga me, tɯ-xtsa me nɯ tɤ-tɕɯ ra kɯ ʁɟa ʑo cʰɯ-tʂɯβ-nɯ pjɤ-ŋu. \\
 \textsc{indef}.\textsc{poss}-clothes whether  \textsc{indef}.\textsc{poss}-shoe whether \textsc{dem} \textsc{indef}.\textsc{poss}-son \textsc{pl} \textsc{erg} completely \textsc{emph} \textsc{ipfv}-sew-\textsc{pl} \textsc{ifr}.\textsc{ipfv}-be \\
 \glt `Whether clothes or shoes, it was always the boys (not the ladies) who sewed them.' (12-kAtsxWb)
(\japhdoi{0003486\#S99})
\end{exe}

In copular sentences (§\ref{sec:copula.basic}), when \forme{ʁɟa} has scope over the nominal predicate, that nominal predicate can be preposed, as in (\ref{ex:kW.RJa.Zo}) (see also example \ref{ex:WBrAsqlWm} in §\ref{sec:WBrA.functions}).\footnote{Example \ref{ex:CkrAz.RJa.Zo} in §\ref{ex:attributive.prenominal} is superficially similar, but the constituent [noun+\forme{ʁɟa ʑo}]  in that example is a prenominal modifier of the following noun, while in (\ref{ex:tAtCW.RJa.Zo}) \forme{tɤ-tɕɯ ʁɟa ʑo} cannot be a prenominal modifier of \forme{kɯ-rɤ-tʂɯβ} `tailor', otherwise the existential verb \forme{pjɤ-tu} rather than the copula \forme{pjɤ-ɕti} would be expected. }
  
\begin{exe}
\ex \label{ex:tAtCW.RJa.Zo}
 \gll  kɯɕɯŋgɯ tɕe [tɤ-tɕɯ ʁɟa ʑo] kɯ-rɤ-tʂɯβ pjɤ-ɕti ma tɕʰeme kɯ-rɤ-tʂɯβ pjɤ-me. \\
 in.former.times \textsc{lnk} \textsc{indef}.\textsc{poss}-son completely \textsc{emph} \textsc{sbj}:\textsc{pcp}-\textsc{apass}-sew \textsc{ifr}.\textsc{ipfv}-be.\textsc{aff} \textsc{lnk} girl \textsc{sbj}:\textsc{pcp}-\textsc{apass}-sew \textsc{ifr}.\textsc{ipfv}-not.exist \\
 \glt  `In former times, only boys were tailors (all tailors were boys), there were no women tailors.' (12-kAtsxWb)
(\japhdoi{0003486\#S95})
\end{exe}

\subsubsection{Restrictive `alone'} \label{sec:stWsti}
To express the meaning `alone', two constructions based on the root \forme{-sti} `alone' are used.

The root \forme{-sti} can be directly combined with personal pronouns in forms such as \forme{aʑo-sti} with the \textsc{1sg} \japhug{aʑo}{I} or \forme{ɯʑo-sti} with the \textsc{3sg}  \japhug{ɯʑo}{he} (§\ref{sec:pronouns.possessive.markers}) as in (\ref{ex:aZosti}).

\begin{exe}
	\ex \label{ex:aZosti}
	\gll   nɤ-rca tu-ɣi-a ra ma kutɕu aʑo-sti ku-rɤʑi-a mɯ́j-cʰa-a \\
	\textsc{1sg}.\textsc{poss}-together \textsc{ipfv}:\textsc{up}-come-\textsc{1sg} be.needed:\textsc{fact} \textsc{lnk} here \textsc{1sg}-alone \textsc{ipfv}-stay-\textsc{1sg} \textsc{neg}:\textsc{sens}-can-\textsc{1sg} \\
	\glt `I am coming with you, I cannot stay here all alone.' (2-deluge2012)
	(\japhdoi{0003376\#S74})
\end{exe}

Alternatively, the reduplicated form of the root \japhug{stɯsti}{alone} occurs either on its own as in (\ref{ex:stWsti}), or as a postnominal modifier as in (\ref{ex:stWsti2}).

\begin{exe}
	\ex \label{ex:stWsti}
	\gll aʑo stɯsti ŋu-a \\
	\textsc{1sg} alone be:\textsc{fact}-\textsc{1sg} \\
	\glt `I am alone.' (conversation)
\end{exe}

\begin{exe}
	\ex \label{ex:stWsti2}
	\gll  fsapaʁ kɤ-χsu mɯ́j-mbat ma maka aki prɤku rgali stɯsti kɯ ʑo, ji-tɯβɣi lonba ʑo, nɯ kɯ\redp{}kɯ-jndʐɤz ʑo tʰa-ɕkɯt qʰe \\
	cattle \textsc{inf}-raise \textsc{neg}:\textsc{sens}-easy \textsc{lnk} completely down  \textsc{anthr} milk.cow alone \textsc{erg} \textsc{emph} \textsc{1pl}.\textsc{poss}-chaff all \textsc{emph} \textsc{dem} \textsc{total}\redp{}\textsc{sbj}:\textsc{pcp}-be.coarse \textsc{emph} \textsc{aor}:3\flobv{}-eat.completely \textsc{lnk} \\
	\glt `Raising cattle is difficult, down there in Praku, the milk cow alone ate all our chaff, all the big ones.' (taRrdo2003)
\end{exe}


In a text translated from Chinese, we do find a calque of the Chinese construction (\ch{我一个人}{wǒ yīgèrén}{I alone}) with the pronoun \japhug{aʑo}{\textsc{1sg}} followed by the counted noun \japhug{tɯ-rdoʁ}{one piece} (§\ref{sec:CN.restrictive}); this sentence is not idiomatic.

\begin{exe}
	\ex \label{ex:aZo.tWrdoR}
	\gll  aʑo tɯ-rdoʁ kɯ ntsɯ tɯ-ci ɕ-tu-re-a \\
	\textsc{1sg} one-piece \textsc{erg} always \textsc{indef}.\textsc{poss}-water \textsc{tral}-\textsc{ipfv}:\textsc{up}-fetch[III]-\textsc{1sg} \\
	\glt `It is always I alone who goes to fetch water.' (150830 san ge heshang)
(\japhdoi{0006416\#S44})
\end{exe}

\subsection{Identity} \label{sec:identity.adverbs}
The adverb \japhug{anamana}{identical} is borrowed from the Amdo Tibetan form \tibet{ཨ་ན་མ་ན་}{ʔa.na.ma.na}{identical}. It occur with the copula \forme{ŋu} (§\ref{sec:copula.basic}) as in (\ref{ex:anamana.Zo}). It is similar in meaning to the reduplicated participle \forme{kɯ-naχtɕɯ\redp{}χtɕɯɣ} `completely identical' of \japhug{naχtɕɯɣ}{be the same} (§\ref{sec:identity.modifier}) but does not occur as noun modifier. 

\begin{exe}
\ex \label{ex:anamana.Zo}
 \gll ɯ-βri rcanɯ, qaɕpa ɯ-βri nɯ anamana ʑo ɲɯ-ŋu \\
 \textsc{3sg}.\textsc{poss}-body \textsc{unexp}:\textsc{foc} frog  \textsc{3sg}.\textsc{poss}-body \textsc{dem} identical \textsc{emph} \textsc{sens}-be \\
\glt `The body of the [turtle] is identical to the body of a frog.'(140510 wugui)
(\japhdoi{0003951\#S8})
\end{exe}

\subsection{Adverbial Intensifiers} \label{sec:intensifier.adverbs}
Intensifiers used to express high degree (`much') and quantity (`much, for a long time') are discussed in §\ref{sec:degree.adverbs}. Negative intensifiers are discussed in §\ref{sec:negative.adverbs}.

The attenuative \forme{ci}, derived from the adverbial function `once' (§\ref{sec:tense.aspect.adverbs}) of the numeral  `one' (§\ref{sec:one.to.ten}),  conveys a milder and more polite tone to Imperative (§\ref{sec:imperative}) and Irrealis (§\ref{sec:irrealis}) verb forms, in particular with modal verbs in interrogative form (§\ref{sec:ra.khW.jAG.verb}) as in (\ref{ex:ci.apWmtama.WjAG}).

 \begin{exe}
\ex \label{ex:ci.apWmtama.WjAG}
\gll wortɕʰi ʑo, aʑo a-ʁi ci a-pɯ-mtam-a ɯ́-jɤɣ \\
please \textsc{emph} \textsc{1sg} \textsc{1sg}.\textsc{poss}-younger.sibling a.little \textsc{irr}-\textsc{pfv}-see[III]-\textsc{1sg} \textsc{qu}-be.allowed:\textsc{fact} \\
\glt  `Could I see my younger sister, please?' (140511 1001 yinzi-zh)
 \end{exe} 
 
 Due to the high polyfunctionality of \forme{ci}, in examples such as  (\ref{ex:li.ci.tAti}), the attenuative function is not always clearly distinguishably from its semelfactive one. 
 
 \begin{exe}
\ex \label{ex:li.ci.tAti}
\gll li ci tɤ-ti \\
again a.little.once \textsc{imp}-say \\
\glt `Say it again!' (many attestations)
 \end{exe} 
  
The intensifier \forme{koŋla}, whose first syllable originates from the reduction of the lexicalized participle \forme{kɯ-ŋu} `the one that/who is (really)' (\tabref{tab:lexicalized.S.nmlz}, §\ref{sec:lexicalized.subject.participle}),\footnote{The non-reduced form of \forme{koŋla}, \forme{kɯŋula}, corresponds to Tshobdun \forme{kəŋólɐ} `well' \citep[55]{jackson19tshobdun}. } originally means `really', a meaning still attested in (\ref{ex:koNla.jazGWta}). In this meaning it can also serve as prenominal modifier (§\ref{ex:attributive.prenominal}).

 \begin{exe}
\ex \label{ex:koNla.jazGWta}
\gll   wo a-mu, aʑo koŋla jɤ-azɣɯt-a \\
\textsc{interj} \textsc{1sg}.\textsc{poss}-mother \textsc{1sg} really \textsc{aor}-arrive-\textsc{1sg} \\
\glt `Mother, it is really me who arrived (i.e. not someone else pretending to be me).' (2012 Norbzang)
(\japhdoi{0003768\#S187})
 \end{exe} 
 
The secondary meanings `(doing) well, correctly' (\ref{ex:koNla.mWkArtoR}) and `completely' (\ref {ex:koNla.mWjtsonW})  developed out the etymological sense of `really'.
 
 \begin{exe}
\ex \label{ex:koNla.mWkArtoR}
\gll ma koŋla mɯ-kɤ-rtoʁ-a ri, ɯ-mɤlɤjaʁ nɯra ɲɯ-dɤn, \\
\textsc{lnk} really \textsc{neg}-\textsc{aor}-look-\textsc{1sg} \textsc{lnk} \textsc{3sg}.\textsc{poss}-limb \textsc{dem}:\textsc{pl} \textsc{sens}-be.many \\
\glt `I did not have a good look at [how many limbs it has], but it has many limbs.' (21-mdzadi)
(\japhdoi{0003578\#S8})
 \end{exe} 
 
 
\begin{exe}
\ex \label{ex:koNla.mWjtsonW}
\gll kupa ɯ-skɤt ri ɯ-qiɯ jamar ma mɯ́j-tso-nɯ. (...) li nɯ koŋla mɯ́j-tso-nɯ. \\
Chinese \textsc{3sg}.\textsc{poss}-language also \textsc{3sg}.\textsc{poss}-half about apart.from \textsc{neg}:\textsc{sens}-understand-\textsc{pl} {  } again \textsc{dem} completely \textsc{neg}:\textsc{sens}-understand-\textsc{pl} \\
\glt `Chinese$_i$  also, they only understand half, (...), they don't understand it$_i$ completely/well (so one has to learn to speak their language to be able to communicate).' (150901 tshuBdWnskAt)
(\japhdoi{0006242\#S13})
\end{exe} 

Finally, in some negative contexts, \forme{koŋla} acquired a meaning close to \forme{maka} `(not) ... at all', as in (\ref{ex:jiatong.koNla.pWme}).

\begin{exe}
\ex \label{ex:jiatong.koNla.pWme}
\gll  tɕe kɤ-ɤmɯ-tɯɣ ri pɯ-me ma <jiaotong> koŋla pɯ-me tɕe (...) kɤ-ŋke ʁɟa ʑo ju-kɯ-ɕe pɯ-ra.  \\
\textsc{lnk} \textsc{inf}-\textsc{recip}-meet also \textsc{pst}.\textsc{ipfv}-not.exist \textsc{lnk} completely transportation \textsc{pst}.\textsc{ipfv}-not.exist \textsc{lnk} { } \textsc{inf}-walk completely \textsc{emph} \textsc{ipfv}-\textsc{genr}:S/O-go \textsc{pst}.\textsc{ipfv}-be.needed \\
\glt `We had no opportunity to meet, as there were no transportation means at all, (...) and we had no choice but to go on foot (whenever there was a meeting).' (12-BzaNsa)
(\japhdoi{0003484\#S19})
\end{exe} 


 \subsection{Epistemic modality}\label{sec:modality.adverbs}
 The adverbs \japhug{cʰɤlɤnnɤ}{maybe} and \japhug{ʑgrɯɣ}{certainly}  can contribute to the expression of epistemic modality, together with verbal morphology (§\ref{sec:TAME.modal}, §\ref{sec:second.modal}) and sentence final particles (§\ref{sec:sfp}).
 
 The former \japhug{cʰɤlɤnnɤ}{maybe}, `perhaps' originates from an Inferential form of the verb \japhug{lɤt}{release}, followed by the additive \forme{nɤ}, perhaps originally the protasis of a conditional construction (§\ref{sec:real.conditional}). It is often combined with the sentence final particle \forme{tʰaŋ} (§\ref{sec:fsp.epistemic}) as in (\ref{ex:chAlAnnA.GWlAti}), or with a  verb in Probabilitative form (§\ref{sec:WmA}).
 
 \begin{exe}
\ex \label{ex:chAlAnnA.GWlAti}
\gll iʑora nɤki, cʰɤlɤnnɤ ɣɯ-znɯzdɯxpa-j tɕe ɣɯ-lɤt-i tʰaŋ wo \\
\textsc{1pl} \textsc{filler} maybe \textsc{inv}-have.mercy:\textsc{fact}-\textsc{1pl} \textsc{lnk} \textsc{inv}-release:\textsc{fact}-\textsc{1pl} \textsc{sfp} \textsc{sfp} \\
\glt `Maybe [the ogre] will have mercy upon us and will let us go.' (160703 poucet3)
(\japhdoi{0006107\#S38})
 \end{exe} 
 
 The latter  \japhug{ʑgrɯɣ}{certainly} generally occurs with a main verb in the Factual Non-Past (§\ref{sec:factual}). It tends to be replaced by the Chinese adverb \ch{肯定}{kěndìng}{certainly}, even by the best speakers.
  
 \begin{exe}
\ex \label{ex:ZgrWG.Zo}
\gll nɤ-wi nɯ ʑgrɯɣ ʑo rga \\
\textsc{2sg}.\textsc{poss}-grand.mother \textsc{dem} certainly \textsc{emph} like:\textsc{fact} \\
\glt `Your grandmother will certainly like it.' (140428 xiaohongmao-zh)
(\japhdoi{0003884\#S50})
 \end{exe} 
  
\subsection{Orientation adverbs} \label{sec:locative.adv}
The tridimensional system found in orientation preverbs (§\ref{sec:kamnyu.preverbs}),  egressive postpositions (§\ref{sec:egressive}, \tabref{tab:egressive}) and locative relator nouns (§\ref{sec:relator.nouns.3d}) is also reflect by locative adverbs (\tabref{tab:orientation.adverbs}). There is a one-to-one correspondence between these adverbs and the corresponds nouns and preverbs (§\ref{sec:preverbs.adverbs}).
 

 \begin{table}
\caption{Orientation adverbs} \label{tab:orientation.adverbs}
\begin{tabular}{lllllllll}
\toprule
 Orientation  &  	Bare & Basic & Distal & Approximate \\
   \midrule
Upwards   &    \forme{taʁ}  & \forme{atu}  & \forme{tɕetu}   \\  	
Downwards   &   \forme{pa} 	&\forme{aki} &  \forme{tɕeki} \\  	
\midrule
Upstream   &     \forme{lo} 	&\forme{alo}  & \forme{tɕelo} & \forme{locʰu} \\  		
Downstream     &  \forme{tʰi} 	&\forme{atʰi}    &\forme{tɕetʰi} &  \forme{tʰɯcʰu} \\  			
\midrule
Eastwards   &   \forme{kɯ} 	&\forme{akɯ}    &\forme{tɕekɯ} & \forme{kɯcʰu} \\  				
Westwards   &     \forme{ndi} 	&\forme{adi}    &\forme{tɕendi} & \forme{ndɯcʰu} \\  		
\bottomrule
\end{tabular}
\end{table}

There are four series of adverbs shown in \tabref{tab:orientation.adverbs}. With the exception of the adverbs of the vertical dimension (\textsc{upwards} and \textsc{downwards}), the other series are trivially derived from the bare adverbs by prefixation of \forme{a\trt}, \forme{tɕe-} (perhaps related to the postposition \forme{tɕe} §\ref{sec:tCe.postposition}) and \forme{-cʰu} (§\ref{sec:approximate.locative}).

The bare adverbs are most often directly placed before an orientable verb (§\ref{sec:orientable.verbs}) bearing a preverb encoding the same orientation, as in (\ref{ex:thi.chWGi}) and (\ref{ex:tCAlo.WndAcu}). The adverb \forme{kɯ}, though homophonous with the ergative in isolation (§\ref{sec:erg.kW}), is often pronounced \phonet{ku} as in the recording of (\ref{ex:tCAlo.WndAcu}), due to regressive assimilation from the preverb \forme{ko-}.

 \begin{exe}
\ex \label{ex:thi.chWGi}
\gll tɤ-mpja tɕe tʰi cʰɯ-ɣi, tɤ-ɣɤndʐo tɕe lo lu-ɕe \\
\textsc{aor}-be.warm \textsc{lnk} downstream \textsc{ipfv}:\textsc{downstream}-come \textsc{aor}-be.cold  \textsc{lnk} upstream \textsc{ipfv}:\textsc{upstream}-come \\
\glt `(Contrary to expectations), when [the weather] becomes warm it comes downstream, and when it becomes cold it goes upstream.' (24-kWmu)
(\japhdoi{0003618\#S25})
 \end{exe}

This is not the exclusive function of bare orientation adverbs however: they can also refer to a static location unrelated to the orientation of the verb, and non-adjacent to it (\ref{ex:kW.ri.ndi.ri}). However, the basic and distal adverb are more often used in this function.

\begin{exe}
\ex \label{ex:kW.ri.ndi.ri}
\gll kɯ ri ci (...) pjɯ-sɯ-ɤtsa-nɯ, ndi ri li ci pjɯ-sɯ-ɤtsa-nɯ   \\
east \textsc{loc} one {  } \textsc{ipfv}:\textsc{down}-\textsc{caus}-be.planted-\textsc{pl} 
west \textsc{loc} again one \textsc{ipfv}:\textsc{down}-\textsc{caus}-be.planted-\textsc{pl}  \\
\glt `(In former times, when people installed the loom), [they] would plant one [sharpened peg] in the east (i.e. left), another one in the west (i.e. right) (to attach the upward extremity of the warp threads).' (vid-20140429090403)
	(\japhdoi{0003776\#S119})
 \end{exe}
 
All four series of orientation adverbs can be followed by the core locative postpositions (§\ref{sec:core.locative}) as in (\ref{ex:kW.ri.ndi.ri}), (\ref{ex:akW.ri.kuru}) and (\ref{ex:tCAlo.WndAcu}).

 \begin{exe}
\ex \label{ex:akW.ri.kuru}
\gll qaliaʁ nɯtɕu lu-zo tɕe akɯ ri ku-ru, andi ri ɲɯ-ru   \\
eagle \textsc{dem}:\textsc{loc} \textsc{ipfv}-land \textsc{lnk} east \textsc{loc} \textsc{ipfv}:\textsc{east}-look west \textsc{loc} \textsc{ipfv}:\textsc{west}-look \\
\glt `Eagles land there and look around (to the east and the west, looking for food).' (140522 Kamnyu zgo)
(\japhdoi{0004059\#S214})
 \end{exe}
 
 Example (\ref{ex:tCAlo.WndAcu}) shows that distal orientation adverbs can occur in the same context as locative relator nouns in \forme{-cu} (§\ref{sec:relator.nouns.3d}).

 \begin{exe}
\ex \label{ex:tCAlo.WndAcu}
\gll ɯ-ndɤcu nɯtɕu, dɯxpakɤrpu ɣɯ ɯ-me nɯ kɯ kʰri pjɤ-ta tɕe kɯ ko-ru.
tɕe tɕelo nɯtɕu ɲimawozɤr ɣɯ ɯ-kʰri nɯ pjɤ-ta-nɯ tɕe, tʰi cʰo-ru-nɯ tɕe \\
\textsc{3sg}.\textsc{poss}-west \textsc{dem}:\textsc{loc}  \textsc{anthr} \textsc{gen} \textsc{3sg}.\textsc{poss}-daughter \textsc{dem} \textsc{erg} seat \textsc{ifr}:\textsc{down}-put \textsc{lnk} east \textsc{ifr}:\textsc{east}-look  \textsc{lnk} upstream \textsc{dem}:\textsc{loc}  \textsc{anthr} \textsc{gen} \textsc{3sg}.\textsc{poss}-seat \textsc{dem} \textsc{ifr}:\textsc{down}-put-\textsc{pl} \textsc{lnk} downstream \textsc{ifr}:\textsc{downstream}-look-\textsc{pl} \textsc{lnk} \\ 
\glt `On the west side, the daughter of Gdugpa Dkarpo placed her seat and turned it towards the east. On the downstream side, [the servants] placed Nyima 'Odzer's seat, and turned it towards the upstream direction.' (2011-04-smanmi)
(\japhdoi{0004063\#S352})
 \end{exe}

The bare adverbial stems can be compounded to build nouns of dimension (§\ref{sec.v.v.compounds.degree})

In addition to the adverbs in \tabref{tab:orientation.adverbs}, we also find \japhug{tɤton}{uphill}, \textsc{upwards} and \japhug{tɤʑɯn}{downhill}, \textsc{downwards} specifically indicating orientation upwards and downwards a slope. They can be associated with both the vertical (\textsc{upwards}, \textsc{downwards}) and the fluvial (\textsc{upstream}, \textsc{downstream}) orientations, as shown by (\ref{ex:tAton.tAZWn}), where the motion verb \japhug{ɕe}{go} and the manipulation verb \japhug{tsɯm}{take away} take the fluvial orientation \textsc{upstream} with \forme{tɤton}, but the vertical orientation \textsc{downwards} with \forme{tɤʑɯn}.

\begin{exe}
\ex \label{ex:tAton.tAZWn}
\gll tɕe tɤton lu-ɕe pɯpɯŋunɤ, snama tɤton tsa lú-wɣ-tsɯm pɯ\redp{}pɯ-ra nɤ ɯ-koŋtaʁ tu-sɯ-ɤsɯɣ-nɯ tɕe nɯ kɯ lu-rɤɕi ɲɯ-ra.  tɤʑɯn pjɯ-ɕe pɯpɯŋunɤ, ɯ-sɲɤt cʰɯ-sɯ-ɤsɯɣ-nɯ ɲɯ-ra. \\
\textsc{lnk} uphill \textsc{ipfv}:\textsc{upstream}-go \textsc{top} beast.of.burden uphill a.little \textsc{ipfv}:\textsc{upstream}-\textsc{inv}-take.away \textsc{cond}\redp{}\textsc{pst}.\textsc{ipfv}-be.needed \textsc{add} \textsc{3sg}.\textsc{poss}-front.strap \textsc{ipfv}-\textsc{caus}-be.tight-\textsc{pl} \textsc{lnk} \textsc{dem} \textsc{erg} \textsc{ipfv}:\textsc{upstream}-pull \textsc{sens}-be.needed downhill \textsc{ipfv}:\textsc{down}-go \textsc{top} \textsc{3sg}.\textsc{poss}-crupper \textsc{ipfv}:\textsc{downstream}-\textsc{caus}-be.tight-\textsc{pl} \textsc{sens}-be.needed \\
\glt `When it$_i$ goes uphill, [that is] if they need to lead the beast$_i$ of burden on an upward slope, they need to tighten the front strap of the saddle for it$_i$ to pull [the burden]. When it$_i$ goes downhill, they need to tighten the crupper.'  (30-tAsno)
(\japhdoi{0003758\#S92})
\end{exe}

Despite their \forme{-n} coda (§\ref{sec:historical.phono}), these adverbs are native words, as shown by their Tshobdun cognates \forme{tɐ́təm} `uphill' and \forme{tɐ́ɟət} `downhill' \citep[123--4]{jackson19tshobdun}. The expected Japhug forms would be $\dagger$\forme{tɤtom} and $\dagger$\forme{tɤʑɯt}, respectively. The coda \forme{-n} probably results from the coalescence of the inherited adverb with the demonstrative \forme{nɯ}. These adverbs probably originate from nominalizations of elevational motion verbs, which are lost in Northern Rgyalrong but attested in Situ (for instance Bragbar \forme{tʰɐ̂} `move upwards' and
\forme{ɟə̂} `move downwards', respectively, see \citealt[§9.1]{zhangshuya20these} and \citealt{linyj17space}).

\subsection{Postverbal elements} \label{sec:postverbal.adv}
The only part of speech that is always located after the main verb is the sentence final particles (§\ref{sec:sfp}, §\ref{sec:non.verb.TAME}). With the exception of right dislocated constituents (§\ref{sec:right.dislocation}), only a handful of other words can occur post-verbally.

Some ideophones (mainly pattern II, §\ref{sec:ideophone.plus.lexical.verb}) can be put after the verb, optionally with  the emphatic \forme{ʑo} as in (\ref{ex:V.prWNprWN.Zo}), even in some relative clauses (§\ref{sec:relative.postverbal}).

\begin{exe}
\ex \label{ex:V.prWNprWN.Zo}
\gll nɯre tu-rŋgɤβ-nɯ \textbf{prɯŋprɯŋ} \textbf{ʑo} \\
\textsc{dem}:\textsc{loc} \textsc{ipfv}-attach-\textsc{pl} \textsc{idph}(II):solidly \textsc{emph} \\
\glt `They attach [the plough] there (on the hybrid yaks' horns) solidly.' (25-stuxsi)
(\japhdoi{0003660\#S11})
\end{exe}

Only three adverbs can occupy postverbal position. First, the emphatic \forme{ʑo} (§\ref{sec:emphatic.Zo}) commonly occurs after the main verb in constructions expressing high degree (§\ref{sec:absolute.degree}), as in (\ref{ex:WtWtCur.saXaR.Zo}).

\begin{exe}
\ex \label{ex:WtWtCur.saXaR.Zo}
\gll  tɕe nɯnɯ tú-wɣ-ndza rca ɯ-tɯ-tɕur saχaʁ \textbf{ʑo}. \\
\textsc{lnk} \textsc{dem} \textsc{ipfv}-\textsc{inv}-eat \textsc{unexp}:\textsc{foc} \textsc{3sg}.\textsc{poss}-\textsc{nmlz}:\textsc{deg}-be.sour be.extremely:\textsc{fact} \textsc{emph} \\
\glt `When one eats it, it is very sour.' (16-CWrNgo)
(\japhdoi{0003518\#S219})
\end{exe}

Second, the degree adverb \japhug{tsa}{a little} is postverbal when used in a comparative construction (§\ref{sec:comparative.tsa}), as in (\ref{ex:YWrYJi.tsa}).

\begin{exe}
\ex \label{ex:YWrYJi.tsa}
\gll tɕʰitɕɯn paχɕi nɯ tɕe tɕe, ɯ-jwaʁ nɯra iʑora ji-paχɕi stʰɯci mɯ-ɲɯ-ɤrtɯm kɯ ɲɯ-rɲɟi \textbf{tsa} \\
\textsc{topo} apple \textsc{dem} \textsc{lnk} \textsc{lnk} \textsc{3sg}.\textsc{poss}-leave \textsc{dem}:\textsc{pl} \textsc{1pl} \textsc{1pl}.\textsc{poss}-apple so.much \textsc{neg}-\textsc{sens}-be.round \textsc{erg} \textsc{sens}-be.long a.little \\
\glt `Pears (Chuchen apples), their leaves are not as round as [those of] our apples, but a bit longer.' (07-paXCi)
(\japhdoi{0003430\#S48})
\end{exe}

Third, the temporal adverb and quantifier \japhug{ntsɯ}{always} has the same ranges of meaning as when in preverbal position (§\ref{sec:tense.aspect.adverbs})  when following the main verb: it can either indicate a constant state (\ref{ex:anWrNi.ntsW}) or a recurrent action (\ref{ex:postV.ntsW}).

\begin{exe}
\ex \label{ex:anWrNi.ntsW}
\gll tɯxpalɤskɤr a<nɯ>rŋi \textbf{ntsɯ}. \\
whole.year <\textsc{auto}>be.green:\textsc{fact} always \\
\glt `It remains always green the whole year.' (08-saCW)
(\japhdoi{0003462\#S19})
\end{exe}

\begin{exe}
\ex \label{ex:postV.ntsW}
\gll  tɕʰeme ɯ-skɤt kɯ-snɯ\redp{}sna ci kɯ ʑo tú-wɣ-nɯ-ɤkʰɤzŋga-nɯ \textbf{ntsɯ}. \\
girl \textsc{3sg}.\textsc{poss}-voice \textsc{sbj}:\textsc{pcp}-\textsc{emph}\redp{}be.nice \textsc{indef} \textsc{erg} \textsc{emph} \textsc{ipfv}-\textsc{inv}-\textsc{appl}-call-\textsc{pl} always \\
\glt `A girl whose voice was very nice was calling them again and again.' (2003 kandZislama)
\end{exe}

 
\section{Non-verbal predicates} \label{sec:non.verbal.predicates}
While a Japhug sentence generally requires a finite verb forms to be complete, there are nevertheless some constructions in which a noun or a nominalized verb form serves as predicate by itself.

First, zero copula predicate nominals are attested, in particular to name a referent (object or person) by showing it/him/her. For instance, in (\ref{ex:kWki.sAndAr}), the subject is the demonstrative \japhug{kɯki}{this} and the nominal predicate \japhug{sɤndɤr}{thimble}. 
 
\begin{exe}
\ex \label{ex:kWki.sAndAr}
\gll kɯki sɤndɤr. kɯki sɤndɤr. iʑora kɯrɯ ra tɕe sɤndɤr tu-kɯ-ti ŋu. \\
\textsc{dem}.\textsc{prox} thimble  \textsc{dem}.\textsc{prox} thimble \textsc{1pl} Tibetan \textsc{pl} \textsc{lnk} thimble \textsc{ipfv}-\textsc{genr}-say be:\textsc{fact} \\
\glt `(Pointing to a thimble.) This is a thimble. This is a thimble. We Tibetans call it `thimble'.' (video-2014-04-29-09-5104)
\end{exe}

However, the zero copula construction is not restricted to such contexts, and is also found when the demonstrative pronouns are anaphoric, as in (\ref{ex:tCheme.tWma}).\footnote{The generic possessor prefix in (\ref{ex:tCheme.tWma}) is due to the fact that \japhug{tɕʰeme}{girl} serves here as a generic noun, and also because the speaker includes herself in this group (§\ref{sec:indef.genr.poss}). }

\begin{exe}
\ex \label{ex:tCheme.tWma}
\gll nɯnɯ tɕʰeme tɯ-ma. \\
\textsc{dem} girl \textsc{genr}.\textsc{poss}-work \\
\glt  `This (the actions described in the previous sentences) is the work of women.' (12-kAtsxWb)
(\japhdoi{0003486\#S97})
\end{exe}

Absence of copula is also attested in multiclausal constructions, for instance in alternative concessive conditionals (§\ref{sec:alt.concessive.conditional}) as in (\ref{ex:tatChoNtChoN}).

\begin{exe}
	\ex \label{ex:tatChoNtChoN}
	\gll tɕe tɯ-ci nɯnɯ, pɯ-nɯ-xtɕi nɤ tatɕʰoŋtɕʰoŋ, pɯ-nɯ-wxti nɤ tatɕʰoŋtɕʰoŋ. \\
	\textsc{lnk} \textsc{indef}.\textsc{poss}-water \textsc{dem} \textsc{pst}.\textsc{ipfv}-\textsc{auto}-be.small \textsc{add} waterfall \textsc{pst}.\textsc{ipfv}-\textsc{auto}-be.big \textsc{add} waterfall \\
	\glt Water [falling down from a cliff], whether it is big or small, [is called] a `waterfall'.' 	(hist 180428 tatChoNtChoN)
\end{exe}


The zero copula construction is however considerably rarer than the copular construction (§\ref{sec:copula.basic}), even in the same context, when the speaker names a referent and points it/him with the finger as in (\ref{ex:kWki.Cnat.Nu}).

\begin{exe}
\ex \label{ex:kWki.Cnat.Nu}
\gll  kɯki ɕnat ŋu \\
\textsc{dem}.\textsc{prox} heddle be:\textsc{fact} \\
\glt  `(Pointing to a heddle.) This is a heddle.' (vid-2014-04-29-09-0403)
(\japhdoi{0003776\#S26})
\end{exe}

Second, exclamative expressions with a degree nominal (§\ref{sec:degree.nominal.predicates}), exclamative nouns (§\ref{sec:exclamative.IPN}) can also be used as main predicate, followed by a sentence final particle such as \forme{nɯ}, without any finite verb form.

Third, inflectionalized phatic expressions (§\ref{sec:phatic.inflectionalization}) and interjections  (§\ref{sec:interjections}) such as \japhug{mtsʰɤri}{how strange} as in (\ref{ex:mtshAri.ataR.WtWpe}) can form complete utterances on their own without any verb.

\begin{exe}
\ex \label{ex:mtshAri.ataR.WtWpe}
\gll ki mtsʰɤri, kɯki a-βɣo ki a-taʁ ɯ-tɯ-pe nɯ \\
\textsc{dem}.\textsc{prox}  how.strange \textsc{dem}:\textsc{prox} \textsc{1sg}.\textsc{poss}-FB \textsc{dem}:\textsc{prox} \textsc{1sg}.\textsc{poss}-on \textsc{3sg}.\textsc{poss}-\textsc{nmlz}:\textsc{deg}-be.good \textsc{sfp} \\
\glt `How strange, this uncle of mine is so nice to me.' (140511 alading-zh)
(\japhdoi{0003953\#S54})
\end{exe}

Fourth, the nouns \japhug{ɯ-mdoʁ}{colour} and \japhug{smɯlɤm}{prayer} have been grammaticalized as sentence-final modality markers (§\ref{sec:nouns.TAME}).
  
  
 
\section{Noun-verb collocations and light verb constructions} \label{sec:light.verb}
While Japhug has a very productive system of denominal verbalizer prefixes described in Chapter \ref{chap:denominal}, it is also possible to build predicates out of nouns using noun-verb collocations, built in particular from a few highly frequent light verbs. 

Apart from nouns, ideophones are also used with light verbs, but this question is studied in §\ref{sec:idph.syntax}.
 
\subsection{Intransitive verbs} \label{sec:intr.light.verbs}
In collocations involving intransitive verbs, the associated noun is the intransitive subject, so that the verb verb form is invariably \textsc{3sg} (§\ref{sec:intransitive.invariable}).

\subsubsection{Motion verbs} \label{sec:motion.light.verbs}
The motion verbs \japhug{ɣi}{come} and  \japhug{ɕe}{go} (§\ref{sec:motion.verbs}) are used with the  nouns of cognition \japhug{ɯ-sɯm}{mind} and \japhug{ɯ-ʁjiz}{wish} in the lexicalized collocations \forme{ɯ-sɯm+ɕe/ɣi} `want' and  \forme{ɯ-ʁjiz+ɣi} `wish', which take complement clauses (§\ref{sec:nouns.cognition.complement}). 

They also occur with temporal nouns. The verb \japhug{ɕe}{go} occurs with \japhug{tɤ-rʑaʁ}{time} and \japhug{ta-ʁa}{free time} to express the meaning `spend (one's time)', as in (\ref{ex:aRa.ChWCe}), an excerpt from a conversation where Tshendzin describes her daily activities. In this function, the whole collocation can be subjected to the facilitative \forme{ɣɤ-} derivation (§\ref{sec:facilitative.GA}).  

%<duanlian> tu-βze-a tu-nɤŋkɯŋke-a ŋu tɕe (...) ji-saχsɯ tu-βze-a, tɤ-nɯsaχsɯ-j qʰe,  tɕe tɕe aʁɤndɯndɤt ʑo cʰɯ-nɤʁaʁ-a qʰe, (...) tɕe li nɯ jamar qʰe li tɤ-pɤri tu-βze-a qʰe, 
%exercise \textsc{ipfv}-make[III]-\textsc{1sg} \textsc{ipfv}-\textsc{distr}:walk-\textsc{1sg} be:\textsc{fact} \textsc{lnk} {  } \textsc{1pl}.\textsc{poss}-lunch \textsc{ipfv}-make[III]-\textsc{1sg} \textsc{aor}-have.lunch-\textsc{1pl} \textsc{lnk} \textsc{lnk} \textsc{lnk} everywhere \textsc{emph} \textsc{ipfv}-have.a.good.time-\textsc{1sg} \textsc{lnk} {  } \textsc{lnk} again \textsc{dem} about \textsc{lnk} again \textsc{indef}.\textsc{poss}-dinner \textsc{ipfv}-make[III]-\textsc{1sg} \textsc{lnk} 
%(...) I have a walk to do some exercise, (...) then I prepare lunch, after we have eaten lunch, I take a break and rest at some place (in the courtyard, with other elderly people), (...)  then I prepare dinner,

\begin{exe}
\ex \label{ex:aRa.ChWCe}
\gll nɯ kɯ-fse ntsɯ \textbf{a-rʑaʁ} \textbf{ɲɯ-ɕe} ŋu. tɤ-nɤpɤri-a qʰe li ci <sanbu> ju-ɕe-a qʰe, tɕe nɯ kɯ-fse ntsɯ sɲikuku \textbf{a-ʁa} \textbf{cʰɯ-ɕe} ŋu. \\
\textsc{dem} \textsc{sbj}:\textsc{pcp}-be.like always \textsc{1sg}.\textsc{poss}-time \textsc{ipfv}:\textsc{west}-go be:\textsc{fact} \textsc{aor}-have.dinner-\textsc{1sg} \textsc{lnk} again once walk \textsc{ipfv}-go-\textsc{1sg} \textsc{lnk} \textsc{lnk} \textsc{dem} \textsc{sbj}:\textsc{pcp}-be.like always everyday \textsc{1sg}.\textsc{poss}-free.time \textsc{ipfv}:\textsc{downstream}-go be:\textsc{fact} \\
\glt `(...), this is how \textbf{I spend my time}. And after dinner, I go on a walk, this is how \textbf{I spend my time} every day.' (conversation 2015-12-05)
(\japhdoi{0006089\#S2})
\end{exe}
 
 The verb \japhug{ɣi}{come} on the other hand is found with nouns referring to seasons (\ref{ex:ftCar.jAGe}) or parts of the day (\ref{ex:tWrmW.koGi}).
 
\begin{exe}
\ex \label{ex:ftCar.jAGe}
\gll  ɯnɯnɯ ftɕar jɤ-ɣe qʰe, nɯ ɲɯ-ʁaʁ qʰe \\
\textsc{dem} summer \textsc{aor}-come[II] \textsc{lnk} \textsc{dem} \textsc{ipfv}-hatch \textsc{lnk} \\
\glt `When the warm season (spring) comes, it hatches, and ...' (25-akWzgumba)
(\japhdoi{0003632\#S95})
\end{exe}

\begin{exe}
\ex \label{ex:tWrmW.koGi}
\gll tɯrmɯ ko-ɣi  \\
evening \textsc{ifr}-come \\
\glt `The evening came.' (many examples)
\end{exe}

It also expresses the occurrence of  specific events like catastrophes (\ref{ex:taNi.toGi}).

\begin{exe}
\ex \label{ex:taNi.toGi}
\gll taŋi to-ɣi \\
drought \textsc{ifr}-come \\
\glt `There was a drought.' (25-kAmYW)
(\japhdoi{0003642\#S1})
\end{exe}

The combination of \japhug{ɕe}{go} with the locative noun \japhug{ɯ-pa}{below} (§\ref{sec:relator.nouns.3d}), in addition to the trivially predicatible meaning `go below $X$', is also used in the sense of `take all of $X$ for oneself (of things that do not exclusively belong to oneself)' (\ref{ex:Wpa.YAnWCe}) when the verb has the Autive \forme{nɯ-} (§\ref{sec:autoben.proper}). In this construction, the agent (the person taking the things) is encoded as possessor of \forme{ɯ-pa}, and the patient (the things taken) as the intransitive subject of \forme{nɯ-ɕe}.


\begin{exe}
\ex \label{ex:Wpa.YAnWCe}
\gll ɯʑo ɯ-pa ɲɤ-nɯ-ɕe \\
\textsc{3sg} \textsc{3sg}.\textsc{poss}-down \textsc{ifr}-\textsc{auto}-go \\
\glt `S/he took all of it for him/herself.' (elicited)
\end{exe}

The verb \japhug{ɕe}{go} also occurs in collocation with a few nouns designating places or objects to express the meaning `go and do', in particular with \japhug{skɤrwa}{circambulation} and \japhug{jɤɣɤt}{terrace}, `toilet'\footnote{In traditional houses in the Japhug-speaking area, toilets are built on a remote side of the covered terrace surrounding the house. The term as now been extended to modern flush toilets.} as in (\ref{ex:skArwa.koCe}) and (\ref{ex:jAGAt.luCea}).

 \begin{exe}
	\ex \label{ex:skArwa.koCe}
	\gll skɤrwa ko-ɕe \\
	circambulation \textsc{ifr}-go \\
	\glt `He went to do circumbulations.' (elicited)
\end{exe}

 \begin{exe}
	\ex \label{ex:jAGAt.luCea}
	\gll  aʑo jɤɣɤt ci lu-ɕe-a nɤ \\
		 toilet \textsc{indef} \textsc{ipfv}:\textsc{upstream}-go-\textsc{1sg} \textsc{sfp} \\
	\glt `I am going to the toilet.' (2005 khu) 
\end{exe}

These collocations correspond to denominal verbs in \forme{rɯ-} (§\ref{sec:denom.intr.rA}).


The ablative motion verb \japhug{ɬoʁ}{come out} (§\ref{sec:motion.verbs}) is commonly found in the meaning `grow' (of plants), but also occurs in a few more lexicalized collocations, with \japhug{tɤ-re}{laugh} as in (\ref{ex:Wre.pjAlhoR}) or with \japhug{tɤ-rmi}{name} (\ref{ex:Wrmi.tolhoR}). More of these collocations have corresponding transitive constructions with \japhug{tɕɤt}{take out} (§\ref{sec:tCAt.lv}), such as \forme{tɤ-re+tɕɤt} `mock' and \forme{tɤ-rmi+tɕɤt} `give a name'.

\begin{exe}
\ex \label{ex:Wre.pjAlhoR}
\gll ɯ-re pjɤ-ɬoʁ \\
\textsc{3sg}.\textsc{poss}-laugh \textsc{ifr}-come.out \\
\glt `S/he laughed.' (elicited; refers to an involuntary action)
\end{exe}

\begin{exe}
\ex \label{ex:Wrmi.tolhoR}
\gll ɯ-rmi to-ɬoʁ. \\
\textsc{3sg}.\textsc{poss}-name \textsc{ifr}:\textsc{up}-come.out \\
\glt `He became famous.' (elicited)
\end{exe}

 \subsubsection{The anticausative verb \japhug{ndzoʁ}{be attached}} \label{sec:ndzoR.light.verbs}
The anticausative \japhug{ndzoʁ}{be attached} (§\ref{sec:anticausative.dummy}), in addition to its function as a quasi-existential verb (§\ref{sec:existential.basic}) and also its dynamic meaning  `cling onto, lean on, grab' (example \ref{ex:WtaR.kondzoR}, §\ref{sec:anticausative.volitionality}) or `land' (with flying creatures),\footnote{The meaning `land' is more commonly expressed with the verb \japhug{zo}{land}, probably borrowed from the Tshobdun cognate of \japhug{ndzoʁ}{be attached}. } occurs in lexicalized collocations with a few nouns.  With nouns referring to  plant parts, it can mean `grow' (§\ref{sec:anticausative.dummy}); with \japhug{tɤ-rɟit}{offspring} or \japhug{tɤ-pɯ}{young}, its meaning is `become pregnant' (of humans or non-human animals), as in (\ref{ex:WrJit.konzdoR}).

 \begin{exe}
\ex \label{ex:WrJit.konzdoR}
\gll iɕqʰa <sanshengmu> nɯ ɣɯ ɯ-rɟit ko-ndzoʁ \\
the.aforementioned  \textsc{anthr} \textsc{dem} \textsc{gen} \textsc{3sg}.\textsc{poss}-offspring \textsc{ifr}-\textsc{acaus}:attach \\
\glt `Sanshengmu became pregnant with a child.' (150826 baoliandeng)
(\japhdoi{0006370\#S53})
 \end{exe} 
  
With time periods such as \japhug{qartsɯ}{winter} or \japhug{ftɕar}{summer}, it it synonymous with the motion verb \japhug{ɣi}{come} above (§\ref{sec:motion.light.verbs}) as shown by (\ref{ex:ftCar.kANdzoR}).
  
\begin{exe}
\ex \label{ex:ftCar.kANdzoR}
\gll  tɕe ɯ-fsaqʰe ftɕar kɤ-ndzoʁ qʰe li tu-ɬoʁ. \\
\textsc{lnk} \textsc{3sg}.\textsc{poss}-next.year summer \textsc{aor}-\textsc{acaus}:attach \textsc{lnk} again \textsc{ipfv}-come.out \\
\glt  `The next year, when the warm season (spring) arrives, it comes out again.' (8-qromJoR)
(\japhdoi{0003532\#S134})
 \end{exe}  

It can also be used like the dummy transitive \japhug{ta}{put} (§\ref{sec:ta.lv}) with the noun \japhug{sɣa}{rust} (\ref{ex:sGa.koNdzoR}) in the meaning `become rusted, get rust', with a function similar to that of the \forme{nɯ-} denominal prefix (compare \japhug{nɯsɣa}{get rust}, §\ref{sec:denom.intr.nW}).

\begin{exe}
\ex \label{ex:sGa.koNdzoR}
\gll sɣa ko-ndzoʁ \\
rust \textsc{ifr}-\textsc{acaus}:attach  \\
\glt `It became rusted.' (elicited)
\end{exe}  

 
\subsubsection{Existential verbs} \label{sec:existential.light.verbs}
Existential verbs (§\ref{sec:existential.basic}) commonly take inalienably possessed nouns as subjects. Some of these collocations can be analyzed as instances of the \textit{mihi est} possessive construction (§\ref{sec:possessive.constructions}). For instance, the combination of \japhug{ta-ʁa}{free time} with the existential verbs has the compositional meaning `have time' (\ref{ex:kume}, §\ref{sec:egophoric.evd}, \ref{ex:tCetha.nWmbrAt}, §\ref{sec:aor.temporal}), and apart from the fact that it can take complement clauses (§\ref{sec:free.time.complement}), this collocation does not stand out as particularly lexicalized. The present section focuses on cases where the meaning of the collocation is not straightforwardly derivable from that of its constituent parts.

The noun \japhug{tɯ-sɯmpa}{mind} (borrowed from \tibet{སེམས་པ་}{sems.pa}{mind}),  occurs with the existential verb \forme{tu} with the meaning `keep/have in mind, remember', with the experiencer marked as possessor, for instance \textsc{1sg} in (\ref{ex:asWmpa.tu}).

\begin{exe}
\ex \label{ex:asWmpa.tu}
\gll  tɤjmɤɣ ɯ-sɤ-tu nɯ a-sɯmpa tu \\
mushroom \textsc{3sg}.\textsc{poss}-\textsc{obl}:\textsc{pcp}-exist \textsc{dem} \textsc{1sg}.\textsc{poss}-mind exist:\textsc{fact} \\
\glt `I have in mind (I know, I did not forget) the places where there are mushrooms.' (elicited)
\end{exe}

The collocation of the noun \japhug{ɯ-grɤl}{order, rule} (from Tibetan \tibet{གྲལ་}{gral}{row}, a meaning still preserved in the intransitive denominal verb \japhug{nɯgrɤl}{be in a row}) with the negative existential verbs \forme{me} or \forme{maŋe} is very commonly used to indicate high degree,  in particular in the degree nominal construction (§\ref{sec:degree.monoclausal}), as in (\ref{ex:WtWsAGmu.WgrAl.maNe}).

\begin{exe}
\ex \label{ex:WtWsAGmu.WgrAl.maNe}
\gll  ɯ-tɯ-sɤɣ-mu ɯ-grɤl maŋe \\
\textsc{3sg}.\textsc{poss}-\textsc{nmlz}:\textsc{deg}-\textsc{prop}-fear \textsc{3sg}.\textsc{poss}-order not.exist:\textsc{sens} \\
\glt `It is extremely frightening/fearsome.' (khu 2012)
(\japhdoi{0004085\#S41})
\end{exe}

It is also compatible with dynamic verbs, as in (\ref{ex:tsxu.WtWmbWt.WgrAl.maNe}).

\begin{exe}
\ex \label{ex:tsxu.WtWmbWt.WgrAl.maNe}
\gll  tʂu ɯ-tɯ-mbɯt ɯ-grɤl maŋe \\
road \textsc{3sg}.\textsc{poss}-\textsc{nmlz}:\textsc{deg}-\textsc{acaus}:take.off  \textsc{3sg}.\textsc{poss}-order not.exist:\textsc{sens}  \\
\glt `The road had collapsed to a considerable extent.' (2010-1)
\end{exe}


This collocation has a second unrelated meaning: `be impudent, be shameless, act in an outrageous way' as in (\ref{ex:nAGrAl.WtWme}), where the possessive prefix can be other than \textsc{3sg}.

\begin{exe}
\ex \label{ex:nAGrAl.WtWme}
\gll tɯ-si tɤ-mda kɯnɤ nɯ kɯ-fse ɲɯ-tɯ-nɤre, nɯ nɤ-grɤl ɯ-tɯ-me nɯ! \\
2-die:\textsc{fact} \textsc{aor}-be.the.time also \textsc{dem} \textsc{sbj}:\textsc{pcp}-be.like \textsc{sens}-2-laugh \textsc{dem} \textsc{2sg}.\textsc{poss}-order \textsc{3sg}.\textsc{poss}-not.exist \textsc{sfp} \\
\glt `Even as you are about to die, you are laughing like that, what impudence!' (140516 guowang halifa-zh)
(\japhdoi{0004008\#S65})
\end{exe}

This collocation can be causativized and reflexivized (§\ref{sec:refl.caus}) as \forme{ɯ-grɤl+ʑɣɤɣɤme} `cause oneself to act shamelessly' as in (\ref{ex:WGrAl.YWZGAGAme}).

\begin{exe}
\ex \label{ex:WGrAl.YWZGAGAme}
\gll ji-wa cʰo cʰa ku-tshi-ndʑi <shafa> ɯ-taʁ zɯ tu-nɯjmɤzdɤβ-ndʑi tɕe ku-rŋgɯ-ndʑi qʰe tɕe
ɯ-grɤl ʑo ɲɯ-ʑɣɤ-ɣɤ-me-ndʑi ɲɯ-ɕti \\
\textsc{1pl}.\textsc{poss}-father \textsc{comit} alcohol \textsc{ipfv}-drink-\textsc{du} sofa \textsc{3sg}.\textsc{poss}-on \textsc{loc} \textsc{ipfv}-sleep.on.opposite.directions-\textsc{du} \textsc{lnk} \textsc{ipfv}-lie.down-\textsc{du} \textsc{lnk} \textsc{lnk} \textsc{3sg}.\textsc{poss}-order \textsc{emph} \textsc{ipfv}-\textsc{refl}-\textsc{caus}-not.exist-\textsc{du} \textsc{sens}-be.\textsc{aff}:\textsc{fact}   \\
\glt `He and my father would drink alcohol, sleep on the sofa one over the other in opposite directions, presenting themselves in a poor light.' (17-lhazgron)
\end{exe}

%\c 他们俩把自己变得不成样子了 

While the constructions above do not have any equivalent denominal derivation, the collocation of existential verbs with nouns relating to meals such as \japhug{tʂʰa}{tea, breakfast}, \japhug{saχsɯ}{lunch} or \japhug{tɤ-pɤri}{dinner} has the same meaning as the intransitive \forme{nɯ-/nɤ-} denominal prefix (§\ref{sec:denom.intr.nW}). For instance,(\ref{ex:tAnApArij}) with the denominal verb \japhug{nɤpɤri}{have dinner} is a natural answer to (\ref{ex:nApAri.WpWtu}) with the existential verb (see also \ref{ex:saXsW.WpWtu}, §\ref{sec:denom.intr.nW}).

\begin{exe}
\ex \label{ex:nApAri.WpWtu}
\gll nɤ-pɤri ɯ-pɯ́-tu? \\
\textsc{2sg}.\textsc{poss}-dinner \textsc{qu}-\textsc{pst}.\textsc{ipfv}-exist \\
\glt `Did you have dinner?'
\ex \label{ex:tAnApArij}
\gll pɯ-tu, nɯ koʁmɯz tɤ-nɤ-pɤri-j \\
\textsc{pst}.\textsc{ipfv}-exist \textsc{dem} just.before \textsc{aor}-\textsc{denom}-dinner-\textsc{1pl} \\
\glt `We did, we just had dinner.' (conversation, 2016-04-12)
\end{exe}
  
  \subsubsection{The intransitive auxiliary \forme{pa} } \label{sec:pa.intr.lv}
The transitive verb \japhug{pa}{do} (§\ref{sec:pa.lv}) has an intransitive counterpart  \forme{pa} (§\ref{sec:lability.pass}), which is used as auxiliary for ideophones (§\ref{sec:idph.pa}), and can also select a numeral as subject, meaning `pass $X$ years' (\ref{ex:40.topa}) (see also \ref{ex:12.topa}, §\ref{sec:CN.verbs}).

\begin{exe}
\ex \label{ex:40.topa}  
\gll tɕiʑo ni kɤ-amɯfse-tɕi nɯ jinde kɯβdɤsqi ɯ-ro to-pa \\
\textsc{1du} \textsc{du} \textsc{aor}-know.each.other-\textsc{1du} \textsc{dem} nowadays forty \textsc{3sg}.\textsc{poss}-excess \textsc{ifr}-pass.X.years \\
\glt  `We have known each other for more than forty years.' (`More than forty years passed', 12-BzaNsa)
(\japhdoi{0003484\#S3})
\end{exe}		

It is not possible in this construction to replace the bare numeral by the counted noun  \japhug{tɯ-xpa}{one year}. For a historical perspective on this construction, see §\ref{sec:num.prefix.paradigm.history} and §\ref{sec:CN.verbs}.

\subsubsection{Other intransitive collocations} \label{sec:other.collocation.intr}
The categories listed above do not exhaust all lexicalized collocations involving intransitive verbs.  \tabref{tab:other.intr.collocations} provides additional examples. In addition, there are also frozen collocations with nouns and/or verbs not otherwise attested (§\ref{sec:frozen.collocations}).

\begin{table}
\caption{Other intransitive collocations (excluding orphan nouns/verbs)} \label{tab:other.intr.collocations}
\begin{tabular}{llll}
\lsptoprule
Noun & Verb & Meaning \\
\midrule
\japhug{ɕɤrkʰa}{dawn} & \japhug{ɴɢraʁ}{be torn} & `break (of dawn)' \\
\japhug{tɯ-kɤrnoʁ}{brain} & \japhug{mtɕɯr}{turn} & `feel dizzy' \\
\japhug{tɯ-sɯm}{mind} & \japhug{βdi}{be well} & `be relieved' \\
\japhug{tɯ-skʰrɯ}{body} & \textsc{neg}+\japhug{βdi}{be well} & `be pregnant' \\
\japhug{tɯ-ro}{chest}, `breast' & \japhug{ŋgɤr}{be narrow} & `be petty-minded'\\
\lspbottomrule
\end{tabular}
\end{table}

Some of these collocations resemble Chinese constructions, and may have been calqued, for instance \forme{tɯ-ro+ŋgɤr} `be petty-minded' (compare with \ch{心胸狭窄}{xīnxiōng xiázhǎi}{be petty-minded}). In other case, the resemblance with Chinese may be a parallel development: the use \japhug{ɴɢraʁ}{be torn}, `break(vi)' (§\ref{sec:anticausative.morphology}) in \forme{ɕɤrkʰa+ɴɢraʁ} `break (of dawn)' reminds of \ch{破晓}{pòxiǎo}{daybreak}, but this metaphor is not limited to Chinese.

The collocation meaning \forme{tɯ-skʰrɯ+\textsc{neg}+βdi} `be pregnant' (\ref{ex:tWskhrW.mABdi}) requires a negative prefix (§\ref{sec:obligatory.negative}) on the verb form. The denominal verb  \japhug{nɯskʰrɯ}{be pregnant with} (§\ref{sec:denom.tr.nW}) derived from this construction however is only based on the noun, and does not integrate the verbal root as could have been expected (§\ref{sec:incorp.denom}).

 \begin{exe}
\ex \label{ex:tWskhrW.mABdi}
\gll tɯ-skʰrɯ mɤ-βdi tɕe nɯnɯ qartsʰaz ɯ-se ɣɯ-tsʰi mɤ-βdi ma \\
\textsc{genr}.\textsc{poss}-body \textsc{neg}-be.well:\textsc{fact} \textsc{lnk} \textsc{dem} deer \textsc{3sg}.\textsc{poss}-blood \textsc{inv}-drink:\textsc{fact} \textsc{neg}-be.well:\textsc{fact} \textsc{lnk} \\
\glt `(People say that) when one is pregnant, it is not good to drink deer blood.' (27-qartshaz)
(\japhdoi{0003702\#S106})
\end{exe}


 
\subsection{Transitive verbs}  \label{sec:tr.light.verbs}

\subsubsection{\japhug{βzu}{make} } \label{sec:Bzu.lv}
The verb \japhug{βzu}{make}, borrowed from \tibet{བཟོ་}{bzo}{make; manufacture}, serves as a causative auxiliary (§\ref{sec:sWpa.sABzu}), and occurs in a wide range of complex predicates, in particular with action nominals (§\ref{sec:action.nominal.Bzu}), simultaneous action nominals (§\ref{sec:simultaneous.action.nominal}, §\ref{sec:simult.action.nominal.Bzu}) and compound action nouns (§\ref{sec:compound.action.nominal.Bzu}). In this section, we focus on constructions with nouns that are not derived from verbs.
  
The verb \forme{βzu} is found with noun meaning `method' or `manner' such as \forme{ftɕaka}  and \forme{kowa} in complement-taking collocations meaning `try to $X$ by any means' or `prepare' (§\ref{sec:nouns.manner.complement}).
 
It also occurs with nouns of speech activity (of Tibetan origin), as in \forme{kʰramba+βzu} `tell lies' (\ref{ex:khramba.matABzea}, §\ref{sec:prohibitive.morpho}), \forme{tɯ-skɤt+βzu} `do/say something meaning $X$' (§\ref{sec:nouns.speech.complement}) or \forme{tɯ-lɤn+βzu} `give an answer' (with the addressee as possessor of the object, \ref{ex:alAN.naBzu}).

 \begin{exe}
\ex \label{ex:alAN.naBzu}
\gll  a-lɤn na-βzu  \\
\textsc{1sg}.\textsc{poss}-answer \textsc{aor}:3\flobv{}-make \\
\glt `He gave me an answer.' (elicited)
  \end{exe}
  
 With names of languages, \forme{βzu} means `speak $X$' as in (\ref{ex:mbroXpaskAt.kABzu}).  It is also found with abstract nouns and also relator nouns in a variety of collocations (\tabref{tab:Bzu.abstract}).
 
 \begin{exe}
\ex \label{ex:mbroXpaskAt.kABzu}
\gll  mbroχpa-skɤt kɤ-βzu ɲɯ-mkʰɤz-nɯ. \\
nomad-language \textsc{inf}-make \textsc{sens}-be.expert-\textsc{pl} \\
\glt `They speak the nomad language (i.e. Amdo Tibetan) very well.' (140522 RdWrJAt)
(\japhdoi{0004061\#S121})
 \end{exe}

 
  \begin{table}
  \caption{Collocations of \japhug{βzu}{make} with abstract nouns} \label{tab:Bzu.abstract}
\begin{tabular}{lllll}
\lsptoprule
Noun &   Meaning when used&Reference\\
&with \forme{βzu}&\\
\midrule
\japhug{ɯ-rtsot}{vengeance} &   `get revenge on '&\\
\forme{ɯ-tsʰɤt} `instead of, on behalf of'  &    `do instead of, replace' &§\ref{sec:deputative}  \\
 \forme{ɯ-sci} `instead of'  &  `do instead of, replace' &  \\
 &get revenge on, answer \\
\japhug{ɯ-qʰu}{after, behind}&  `support, back up' &§\ref{sec:other.locative.relator} \\
 \lspbottomrule
\end{tabular}
\end{table}
 
  
% ɯ-sɤz-nɤjo,βzu
  
The verb \japhug{βzu}{make} is attested in the dummy subject construction (§\ref{sec:transitive.dummy}) with three types of nouns.

First, it occurs with nouns of natural phenomena (\tabref{tab:Bzu.dummy}), as illustrated by (\ref{ex:qale.atABze}).
 
\begin{table}
\caption{dummy subject collocations with \japhug{βzu}{make} (natural phenomena) } \label{tab:Bzu.dummy}
\begin{tabular}{lllll}
\lsptoprule
Noun & Orientation & Meaning \\
\midrule
\japhug{qale}{wind} & upwards & `blow (of wind)' \\
\japhug{tɤŋe}{sun} & westwards,  upwards &`appear (of the sun)' \\
\japhug{tɤrmbja}{lightning}& upwards & `appear (of lightning)' \\
\japhug{mbɣɯrloʁ}{thunder}& upwards &`occur (of thunder)' \\
\japhug{tɤrtsa}{wave} & upwards &`appear (of a wave)' \\
\lspbottomrule
\end{tabular}
\end{table}
  
\begin{exe}
\ex \label{ex:qale.atABze}
\gll nɯnɯ qale a-tɤ-βze qʰe, iɕqʰa nɯ ɣɯ-nɯ-tsɯm qʰe,  \\
\textsc{dem} wind \textsc{irr}-\textsc{pfv}-make[III] \textsc{lnk}  the.aforementioned \textsc{dem} \textsc{inv}-\textsc{auto}-take.away:\textsc{fact} \textsc{lnk} \\
\glt `Whenever there is wind, it blows [this insect] away.' (28-kWpAz)
(\japhdoi{0003714\#S150})
\end{exe}

Second, it is found with parts of plants or animals in the meaning `grow' or `develop', as in \forme{tɤ-spɯ+βzu} `fester, have pus' (§\ref{sec:denominal.vs.light.verb}). A defective verb \forme{βze} `grow' based on the third stem of \forme{βzu} has been created by backformation from this function (§\ref{sec:stem3.backformation}).

Third, the abstract nouns \forme{ɯ-tsʰɤt} `with proper measure' (distinct from its use in \tabref{tab:Bzu.abstract}, see §\ref{sec:deputative}) and \forme{ɯ-tsa} `with proper measure, fit, adapted' are found in a transitive but subjectless construction meaning `$X$ fit/be the right size/quantity/degree for $Y$' which encodes the referent $Y$ as possessor, and  $X$ as an absolutive phrase: in (\ref{ex:atsa.YWBze}), the head-internal relative clause (§\ref{sec:object.causative.relativization}) in brackets does not receive ergative marking.
 
 \begin{exe}
\ex \label{ex:atsa.YWBze}
\gll  (a-xtsa lɤ-tɯ-sɯ-ɣɯt-ndʑi nɯ) a-tsa wuma ɲɯ-βze \\
\textsc{1sg}.\textsc{poss}-shoe \textsc{aor}:\textsc{upstream}-2-\textsc{caus}-bring-\textsc{du} \textsc{dem}    \textsc{1sg}.\textsc{poss}-fit really \textsc{sens}-make[III] \\
 \glt `The shoes that you have sent me fit me very well.' (conversation, 2015-04-18)
 \end{exe}

Many of the locutions in \forme{βzu} have a semantics close to denominal derivations in \forme{rɯ-/rɤ-} and \forme{nɯ-} (§\ref{sec:denominal.vs.light.verb}).  In addition, \forme{βzu} corresponds to the \forme{ɣɯ-} prefix in the case of \japhug{ɣɯlɤn}{answer} (§\ref{sec:denom.tr.GA}), a verb similar in meaning to \forme{tɯ-lɤn+βzu} (\ref{ex:alAN.naBzu}) and also to \forme{sɯ-} in \japhug{sɯndzɯpe}{sit without crossing legs} (§\ref{sec:denom.sW.caus.instr}).
 
 \subsubsection{\japhug{lɤt}{release} } \label{sec:lAt.lv}
The basic meaning of the verb \forme{lɤt} is `throw' when employed with an inanimate object (like \japhug{tɯdi}{arrow}, see \ref{ex:kW.WtaR.object}, §\ref{sec:indirective.word.order}), `pour' when applied to liquids or containers (§\ref{ex:khru.chWftsxinW}, §\ref{sec:anticausative.other.derivations}) and either `release, let go'  (\ref{ex:aZWG.nWkhAm}, §\ref{sec:gen.beneficiary}) or `see $X$ off'
with a human object (\ref{ex:akW.mACtsxa}, §\ref{sec:terminative}). It is very productively used with noun-verb compounds with \japhug{rpu}{bump into} or \japhug{tɕʰɯ}{gore}  as second element (see examples \ref{ex:RzAnrpu} and \ref{ex:RrWrpu}, §\ref{sec:action.nominal.compounds}).
 
Like \japhug{βzu}{make},  \forme{lɤt} is attested with \forme{tɯ-} action nominals (§\ref{sec:action.nominal.Bzu}), though this construction is less productive and limited to a few items such as \forme{tɯ-mɯrʁɯz+lɤt} `make a scratch' (from \japhug{mɯrʁɯz}{scratch}) or \forme{tɯ-rkɤz+lɤt} `make an engraving'  (from \japhug{rkɤz}{carve}). It also occurs with the inalienably possessed bare action nominal \japhug{tɯ-sɯso}{thought} (§\ref{sec:bare.action.nominals}). From \forme{tɯ-sɯso+lɤt} `think', the antipassive verb \japhug{rɯsɯso}{think}, `ponder' (§\ref{sec:antipassive.rA}) was derived by denominal derivation.

With iterative counted nouns (§\ref{sec:CN.iterative}) as object, \forme{lɤt} is highly common with the meaning `do $X$ times', where $X$ represents the numeral prefix on the counted noun  (\ref{ex:XsWtAxWr.talAt}).

\begin{exe}
\ex \label{ex:XsWtAxWr.talAt}
\gll mbro nɯ kɯ rirɤβ raŋri χsɯ-tɤxɯr ta-lɤt \\ 
horse \textsc{dem} \textsc{erg} mountain each three-lap \textsc{aor}:3\flobv{}-release \\
\glt `The horse made three laps around each mountain.' (2003 Kunbzang)
\end{exe}

With the counted nouns \japhug{tɯ-rzɯɣ}{one section} and \japhug{tɯ-ɣdɤt}{one section}, \forme{lɤt} means `cut into $X$ pieces', as in (\ref{ex:kWBderzWG.tolAt}).

\begin{exe}
\ex \label{ex:kWBderzWG.tolAt}
\gll  kɯβde-rzɯɣ ʑo tó-wɣ-lɤt pjɤ́-wɣ-sat \\
four-section \textsc{emph} \textsc{ifr}-\textsc{inv}-release \textsc{ifr}-\textsc{inv}-kill \\
\glt `[The thief] killed him by cutting him into four pieces.' (140512 alibaba-zh)
(\japhdoi{0003965\#S109})
\end{exe}

Example (\ref{ex:kWBderzWG.tolAt}) shows that the counted noun is not the direct object: the inverse (§\ref{sec:obviation.saliency}) on both verbs \forme{tó-wɣ-lɤt} and \forme{pjɤ́-wɣ-sat} (§\ref{sec:svc.simultaneous}) shows that both share the same (obviative) object, and therefore that the object is the patient (the person that was killed), not the counted noun. This referent is also relativized like an object (§\ref{sec:object.relativization}) using an object participial relative (\ref{ex:kWBderzWG.tAkAlAt}).

\begin{exe}
\ex \label{ex:kWBderzWG.tAkAlAt}
\gll kʰa kɯ-qanɯ\redp{}nɯ ɯ-ŋgɯ zɯ, nɤkinɯ, tɯ-ɕpɤβ kɯβde-rzɯɣ tɤ-kɤ-lɤt nɯnɯ ku-sɯ-ɤlɤɣi-a cʰa-a ɕti nɤ! \\
house \textsc{sbj}:\textsc{pcp}-\textsc{emph}\redp{}be.dark \textsc{3sg}.\textsc{poss}-in \textsc{loc} \textsc{filler} \textsc{indef}.\textsc{poss}-corpse four-section \textsc{aor}-\textsc{obj}:\textsc{pcp}-release \textsc{dem} \textsc{ipfv}-\textsc{caus}-be.connected[III]-\textsc{1sg} can:\textsc{fact}-\textsc{1sg} be.\textsc{aff}:\textsc{fact} \textsc{sfp} \\
\glt `(not only this, but) I am (even) able to put together a corpse that had been cut into four pieces in a dark house.' (140512 alibaba-zh)
(\japhdoi{0003965\#S152})
\end{exe}

With \forme{tɯ-} action nominals, \forme{lɤt} also has an iterative meaning, but the number of occurrences of the action is indicated by a numeral after the noun as in (\ref{ex:tWmWrtWG.XsWm.tolAt}).

\begin{exe}
\ex \label{ex:tWmWrtWG.XsWm.tolAt}
\gll tɯ-mɯrtsɯɣ χsɯm to-lɤt \\
\textsc{nmlz}:\textsc{action}-pinch three \textsc{ifr}-release \\
\glt `He pinched him three times.' (elicited)
\end{exe}

Noun expressing hitting actions such as \forme{tɯqartsɯ} `kicking'\footnote{The \forme{tɯ-} prefix on this noun synchronically neither an action nominal prefix nor a numeral prefix). } in collocation with \forme{lɤt} also indicate the number of hits by a free numeral (\ref{ex:tWqartsW.RnWz.tolAt}). In this construction, the patient is not encoded as direct object, but as an oblique argument with the relator \japhug{ɯ-taʁ}{on, above} (§\ref{sec:WtaR}).

\begin{exe}
\ex \label{ex:tWqartsW.RnWz.tolAt}
\gll ɯ-taʁ nɯtɕu tɯqartsɯ ʁnɯz to-lɤt \\
\textsc{3sg}.\textsc{poss}-on \textsc{dem}:\textsc{loc} kicking two \textsc{ifr}-release \\
\glt `He kicked two times on it.'  (150824 kelaosi-zh)
(\japhdoi{0006276\#S61})
\end{exe}

A handful of action nominals such as \forme{tɯ-tʂɯβ} `sewing' can however take numeral prefixes to indicate the number of iterations, as in (\ref{ex:RnWtWtsxWB.ntsW}).
 
\begin{exe}
\ex \label{ex:RnWtWtsxWB.ntsW}
\gll ʁnɯ-tɯ-tʂɯβ ntsɯ cʰɯ́-wɣ-lɤt ra.   \\
two-\textsc{nmlz}:\textsc{action}-sew always \textsc{ipfv}-\textsc{inv}-release be.needed:\textsc{fact} \\
\glt `It has to be sewed two times.' (12-kAtsxWb)
(\japhdoi{0003486\#S70})
\end{exe}
 
Table (\ref{tab:lAt.tr}) presents examples of collocations with \japhug{lɤt}{release}. From the meaning `throw, release' (an arrow), \forme{lɤt} acquired the sense of `shoot' with shooting weapons, and then `hit' with body parts or other weapons. Additionally, it came to mean `use' with various types of instruments and implements.
 
\begin{table}
\caption{Examples of collocations with \japhug{lɤt}{release} } \label{tab:lAt.tr}
\begin{tabular}{lllll}
\lsptoprule
Noun & Orientation & Meaning \\
\midrule
\japhug{tɯ-mci}{saliva} & \textsc{upwards} & `spit' \\
\japhug{tɤŋkʰɯt}{first} & \textsc{upwards} & `punch'  \\
\japhug{ɕɤmɯɣdɯ}{gun} & \textsc{upwards} & `shoot (with a gun)'  \\
\japhug{mdaʁʑɯɣ}{bow} & \textsc{upwards} & `shoot (with a bow)'  \\
\japhug{tsʰa}{salt} & \textsc{downwards} & `put salt (on)'  \\
\japhug{taqaβ}{needle} & \textsc{upwards} & `prick with a needle'  \\
\midrule
\japhug{tɯ-mke}{neck} & \textsc{downwards} & `slit $X$'s throat' \\
\midrule
\japhug{tɤ-mtsɯ}{button} & \textsc{upwards} & `button up' \\
\japhug{tɤ-mtɯ}{knot} & `up-, down-, eastwards,  & `tie a knot'  \\
&downstream' \\
\japhug{sɤcɯ}{key} & \textsc{downwards} & `lock'  \\
\midrule
\japhug{ʑɴɢro}{Jew's harp} & \textsc{westwards} & `play the Jew's harp'  \\
\japhug{ɟuli}{flute} & \textsc{downstream} & `play the flute'  \\
\midrule
\japhug{paχtsa}{piglet} & \textsc{downstream} & `bear a piglet'  \\
\lspbottomrule
\end{tabular}
\end{table}


The verb \forme{lɤt} is the productive way to derive predicates from recent Chinese loanwords  designating machines. Combined with  \ch{电话}{diànhuà}{telephone} (\ref{ex:jAlAt.je}, §\ref{sec:imp.function}, \ref{ex:jAtWlAt.mWpjAmtshama}, §\ref{sec:inf.1person}) and  \ch{汽车}{qìchē}{car} it means `phone' and `drive a car', respectively. It also occurs with nouns expressing actions derived from verbs such as  \ch{放假}{fàngjià}{have a holiday}, with a \textsc{3pl} generic subject (§\ref{sec:genr.3pl}) in (\ref{ex:fangjia.lAtnW}).

\begin{exe}
\ex \label{ex:fangjia.lAtnW}
\gll a-ɣe ra ɣɯ pɤjkʰu ʑatsa <fangjia> mɯ́j-lɤt-nɯ \\
\textsc{1sg}.\textsc{poss}-grandchild \textsc{pl} \textsc{gen} yet early vacation \textsc{neg}:\textsc{sens}-release-\textsc{pl} \\
\glt `My grandchild [and the rest of his class] are not yet on vacations.' (conversation, 2014-12-24)
\end{exe}

Some nouns occurring with \forme{lɤt} are also compatible with other light verbs, with slightly different semantics. For instance, from the noun \japhug{rɤɣo}{song}, the locution \forme{rɤɣo+βzu} has the trivial meaning `sing, sing a song' (\ref{ex:tWrJaʁ.pjWBzunW}, §\ref{sec:action.nominal.Bzu}), synonymous with the denominal  \japhug{nɯrɤɣo}{sing} (§\ref{sec:denom.intr.nW}), while \forme{rɤɣo+lɤt} means `play a tune on an instrument', reminiscent of the use of  \forme{lɤt} with musical instruments.
  
%  tɤqɤt,lɤt
 
 The verb \forme{lɤt}  is also found with nouns referring to meteorological phenomena in the dummy subject construction (§\ref{sec:transitive.dummy}), for instance with  \japhug{tɯ-mɯ}{sky, weather} (\ref{ex:tWmW.kalAt})  and \japhug{tɤjpa}{snow}  (\ref{ex:tAjpa.kalAt}). In both examples, it takes the C-type  \textsc{eastwards} preverb  \forme{ka-} (\forme{tɯ-mɯ} also occurs with the orientation \textsc{downwards}).

\begin{exe}
\ex \label{ex:tWmW.kalAt}
\gll  tɯ-mɯ ka-lɤt ɯ-mpʰru nɯ tu. \\
\textsc{indef}.\textsc{poss}-sky \textsc{aor}:3\flobv{}-release \textsc{3sg}.\textsc{poss}-after \textsc{dem} exist:\textsc{fact} \\
\glt `It is found after rain [has fallen].' (23-mbrAZim)
(\japhdoi{0003604\#S72})
\end{exe}

\begin{exe}
\ex \label{ex:tAjpa.kalAt}
\gll qartsɯ tɕe tɕendɤre tɤjpa wuma ʑo ka-lɤt tɕe \\
winter \textsc{lnk} \textsc{lnk} snow really \textsc{emph} \textsc{aor}:3\flobv{}-release \textsc{lnk} \\
\glt `In winter, when a lot of snow has fallen,' (24-kWmu)
(\japhdoi{0003618\#S20})
\end{exe}

Collocation with \forme{lɤt} correspond to various denominal derivations: \forme{rɯ-} (\forme{ɟuli+lɤt} `play the flute' $\rightarrow$ \japhug{rɯɟuli}{play the flute}, §\ref{sec:denom.intr.rA}), \forme{nɯ-} (\forme{ɕɤmɯɣdɯ+lɤt} `shoot with a gun' $\rightarrow$ \japhug{nɯɕɤmɯɣdɯ}{shoot at}, §\ref{sec:denom.tr.nW}) and \forme{sɯ-} (\forme{tɯqartsɯ+lɤt} `kick'  $\rightarrow$ \japhug{sɯqartsɯ}{kick}, §\ref{sec:denom.sW.caus.instr}).

\subsubsection{\japhug{tɕɤt}{take out} } \label{sec:tCAt.lv}
The manipulation verb \japhug{tɕɤt}{take out} (§\ref{sec:manipulation.verbs}) has a wide range of derived meanings, including the relatively straightforward `remove' (\ref{ex:wuma.YWnAmNAm}, §\ref{sec:wuma}), `expel, banish' (\ref{ex:tChi.kWNu}, §\ref{sec:interrogative.pronouns}), `take off (clothes)' (\ref{ex:tWNga.nWkAmbi}, §\ref{sec:relative.possessor.neutralization}), but also `raise (a child)' (\ref{ex:tChi.tAkWsci}, §\ref{sec:tChi}) and `earn (food)' (\ref{ex:pjWnWtCAtnW}, §\ref{sec:pronouns.emph}). It occurs as auxiliary in a marginal causative construction (\ref{ex:mAkWra.YAtWtCAt}, §\ref{sec:sWpa.sABzu}).

Unlike \forme{βzu} and \forme{lɤt}, it almost never used with \forme{tɯ-} action nominals, except with the lexicalized noun \japhug{tɯpɣaʁ}{field clearing} (§\ref{sec:lexicalized.action.nominals}), from which the irregular antipassive \japhug{rɤpɣaʁ}{reclaim land} is derived by denominal derivation (§\ref{sec:antipassive.lexicalized}).

\tabref{tab:tCAt.tr} collects the most common collocations with \forme{tɕɤt}. Most of the nouns in this table are inalienably possessed. In the case of body parts, with the meaning `stick out', the possessive prefix is coreferent with the transitive subject (\textsc{3sg} in \ref{ex:Wku.totCAt}).

\begin{exe}
\ex \label{ex:Wku.totCAt}
\gll tsʰɯtʰo-pɯ tɯ-rdoʁ kɯ ɯ-ku to-tɕɤt. \\
kid-\textsc{dim} one-piece \textsc{erg} \textsc{3sg}.\textsc{poss}-head \textsc{ifr}-take.out \\
\glt `One of the little kids stuck its head out of [the wolf's belly].' (140430 lang he qizhi xiaoshanyang-zh)
(\japhdoi{0003895\#S131})
\end{exe}

In the collocations whose gloss contains an $X$, the possessive prefix corresponds to this $X$ referent, for instance the recipient in \forme{ɯ-ftɕɤfkɤt+tɕɤt} (\ref{ex:aftCAfkAt.tAtCAt}).

\begin{exe}
\ex \label{ex:aftCAfkAt.tAtCAt}
\gll  a-ftɕɤfkɤt ci tɤ-tɕɤt tɕe jisŋi tɕʰi ʑo ɲɯ-nɤme-a pe \\
\textsc{1sg}.\textsc{poss}-idea \textsc{indef} \textsc{imp}-take.out \textsc{lnk} today what \textsc{emph} \textsc{ipfv}-do[III]-\textsc{1sg} be.good:\textsc{fact}  \\
\glt `Make me a suggestion, what should I do today?' (140515 jiesu de laoren-zh)
(\japhdoi{0004004\#S119})
\end{exe}



\begin{table}
\caption{Examples of collocations with \japhug{tɕɤt}{take out} } \label{tab:tCAt.tr}
\begin{tabular}{lllll}
\lsptoprule
Noun & Orientation & Meaning \\
\midrule
\japhug{tɤ-lu}{milk} & \textsc{downwards} & `milk (a cow)' \\
\japhug{tɤ-se}{blood} & \textsc{downwards} & `cause bleeding' \\
\japhug{tɯ-qom}{tear} & \textsc{downwards} & `shed tear(s)' \\
\japhug{tɯ-sroʁ}{life} & \textsc{upwards} & `cause $X$ to lose one's life' \\
\japhug{tɯ-ɕtʂi}{sweat} & \textsc{downwards} & `cause $X$ to sweat' \\
\japhug{tɤ-re}{laugh} &  & `mock $X$' \\
\japhug{ɯ-tɕʰaʁ}{handicap} &  \textsc{downwards} & `cause $X$ a handicap, \\
&&mutilate $X$ \\
\japhug{tɤ-rmi}{name} & \textsc{upwards} & `give a name to $X$' \\
\midrule
\japhug{tɯ-ro}{chest}, `breast' & \textsc{upwards} & `stick out one's chest' \\
\japhug{tɯ-ku}{head}  & `upwards,  & `stick one's head out' \\
&downstream'&\\
\midrule
\japhug{qaɟy}{fish} & \textsc{upstream} & `catch (a fish)' \\
\midrule
\japhug{ɯ-βlu}{trick}, `idea' & \textsc{upwards} & `suggest an idea, a trick to $X$' \\
\japhug{ɯ-ftɕɤfkɤt}{idea}, `advice' & \textsc{upwards} & `make a suggestion to $X$' \\
\midrule
\japhug{tɤndɤɣri}{illegitimate child} & \textsc{downwards} & `have an illegitimate child' \\
\midrule
\japhug{rŋama}{completion} & \textsc{downstream} & `put to completion' \\
\japhug{tɤ-qa}{root}, `paw', `bottom' & `upwards,  & `uproot, \\
& downstream' & `put $X$ to completion,   \\
&& do $X$ until the end'\\
\lspbottomrule
\end{tabular}
\end{table}


The nouns in the first section of \tabref{tab:tCAt.tr} are also compatible with \japhug{ɬoʁ}{come out} as light verb (§\ref{sec:motion.light.verbs}). In most cases, the \forme{ɬoʁ} collocations are functional anticausatives of the \forme{tɕɤt} collocations, as in \forme{ɯ-tɕʰaʁ+ɬoʁ} `have a handicap, be mutilated' vs. \forme{ɯ-tɕʰaʁ+tɕɤt} `cause a handicap, mutilate' (\ref{ex:atChaR.pjWtWtCAt}) (a noun borrowed from \tibet{ཆག་}{tɕʰag}{decrease, break}) or \forme{tɯ-sroʁ+ɬoʁ} `lose one's live' vs. \forme{tɯ-sroʁ+tɕɤt} `cause to lose one's life'. The only exception is \forme{tɤ-rmi+ɬoʁ} `become famous', whose semantics is not derivable from \forme{tɤ-rmi+tɕɤt} `give a name'.

\begin{exe}
\ex 
\begin{xlist}
\ex \label{ex:WtChaR.pWlhoR}
\gll a-mi ɯ-tɕʰaʁ pɯ-ɬoʁ \\
\textsc{1sg}.\textsc{poss}-leg \textsc{3sg}.\textsc{poss}-handicap \textsc{aor}-come.out \\
\glt `My leg became handicapped (as the result of an accident).' (elicited)
\ex \label{ex:atChaR.pjWtWtCAt}
\gll aʑo kɯnɤ, a-tɕʰaʁ pjɯ-tɯ-tɕɤt ɲɯ-ŋu \\
\textsc{1sg} also \textsc{1sg}.\textsc{poss}-handicap \textsc{ipfv}-2-take.out \textsc{sens}-be \\
\glt `(Not only did you do all these crimes), you also mutilate me.' (tou dongxi de xiaohai-zh)
\end{xlist}
\end{exe}

The noun \japhug{ɯ-βlu}{trick}, `idea' forms its corresponding intransitive construction with the existential verb \forme{tu} (example \ref{ex:aBlu.ci.tu}, §\ref{sec:existential.basic}).

The collocation of \forme{tɕɤt} with \forme{ɯ-qa} `(its) root' can have the compositional meaning `uproot' (\ref{ex:Wqa.tuwGtCAt}) like the denominal verb \japhug{nɤqa}{uproot}(§\ref{sec:denom.tr.nW}), but it also used in the sense of `do completely, until the end' as in (\ref{ex:Wqa.chWtCAt}). This meaning is also found with the nouns \forme{rŋama} (from \tibet{རྔ་མ་}{rŋa.ma}{tail}), \japhug{tɤ-jme}{hair} and the relator noun \japhug{ɯ-ndo}{edge, border} (§\ref{sec:other.locative.relator}).

\begin{exe}
\ex \label{ex:Wqa.tuwGtCAt}
\gll tʂɤɕpʰɤt nɯnɯ ɯ-qa tú-wɣ-tɕɤt tɕe, pjɯ́-wɣ-ɣɤ-la tɕe tɯ-ɕɣa kɯ-mŋɤm pʰɤn tu-ti-nɯ \\
plaintain \textsc{dem} \textsc{3sg}.\textsc{poss}-root \textsc{ipfv}:\textsc{up}-\textsc{inv}-take.out \textsc{lnk} \textsc{ipfv}-\textsc{inv}-\textsc{caus}-soak \textsc{lnk} \textsc{genr}.\textsc{poss}-tooth \textsc{sbj}:\textsc{pcp}-hurt be.efficient:\textsc{fact} \textsc{ipfv}-say-\textsc{pl} \\
\glt `People say that, if one digs out the plantain's root and soak it in water, it is efficient against toothache.' (12-ndZiNgri)
(\japhdoi{0003488\#S43})
\end{exe}


\begin{exe}
\ex \label{ex:Wqa.chWtCAt}
\gll nɯ kɤ-ndɯn lonba ɯʑo pjɯ-sɤʑe tɕe ɯ-qa cʰɯ-tɕɤt \\
\textsc{dem} \textsc{inf}-read all \textsc{3sg} \textsc{ipfv}-start[II] \textsc{lnk} \textsc{3sg}.\textsc{poss}-root \textsc{ipfv}-take.out \\
\glt `[This monk is in charge] of reading [the sûtras] from the beginning until the end.' (160721 XpWN)
(\japhdoi{0006181\#S42})
\end{exe} 

Some of the collocations in \forme{tɕɤt} correspond to denominal verbs in \forme{nɯ\trt}, for instance \japhug{nɯqaɟy}{fish} (§\ref{sec:denom.intr.nW}) or \japhug{nɤre}{laugh}, a labile verb meaning `mock' when transitive (§\ref{sec:lability.apass}) like \forme{tɤ-re+tɕɤt}. Alternatively, \forme{tɤ-rmi+tɕɤt} `give a name'and \forme{tɯ-ɕtʂi+tɕɤt} `cause to sweat' rather correspond to sigmatic causative denominal verbs (§\ref{sec:denom.sW.caus.instr}).


\subsubsection{\japhug{ndo}{take} } \label{sec:ndo.lv}
The basic meaning of \forme{ndo} is `take' (\ref{ex:nAZo.tAnWndAm}, §\ref{sec:autoben.transitivity}) or `have in the hand' with the \textsc{upwards} orientation, `catch' (\ref{ex:jWjAkWGe} §\ref{sec:subject.participle.other.prefixes}), `grab' (\ref{ex:Wku.kukWsWndo}, §\ref{sec:indexation.generic.tr}), or  `get attach to, stick on' (\ref{ex:tWjaR.kundAm}) with the \textsc{eastwards} preverbs.  
  
\begin{exe}
\ex \label{ex:tWjaR.kundAm}
\gll ɲɯ́-wɣ-nɤmɤle tɯ-jaʁ ku-ndɤm ʑo ɕti ma, ɯ-taʁ, ɯ-ndʑɯɣ ʑo kɯ-fse tu. \\
\textsc{ipfv}-\textsc{inv}-touch \textsc{genr}.\textsc{poss}-hand \textsc{ipfv}-take[III] \textsc{emph} be.\textsc{aff}:\textsc{fact} \textsc{lnk} \textsc{3sg}.\textsc{poss}-on \textsc{3sg}.\textsc{poss}-resin \textsc{emph} \textsc{sbj}:\textsc{pcp}-be.like exist:\textsc{fact} \\
\glt `When one touches it (a type of mushroom), it sticks on one's hand, as there is some kind of resin-like thing on it.' (21-kuGrummAG)
(\japhdoi{0003574\#S25})
\end{exe}

\tabref{tab:ndo.tr} presents a non-exhaustive list of lexicalized collocations with \forme{ndo}. With the noun \japhug{tɯ-jaʁ}{hand}, it can predictably mean `grab by the hand' (\ref{ex:WskAt.amAkAtWsANAm}, §\ref{sec:irrealis.delayed.imp}) or `stick on the hand' (\ref{ex:tWjaR.kundAm}), but in the imperative it can be interpreted as `don't interfere, don't meddle in it'.

 \begin{table}
\caption{Examples of collocations with \japhug{ndo}{take} } \label{tab:ndo.tr}
\begin{tabular}{lllll}
\lsptoprule
Noun & Orientation & Meaning \\
\midrule
 \japhug{ɯ-kʰrɤt}{set, determined} & \textsc{eastwards} & `control' \\
\japhug{ɯ-rtsawa}{importance}  &   &  `control'  \\
\japhug{ɯ-mdoʁ}{colour}  &   &  `have $X$'s colour'  \\
\japhug{tʂu}{road}  & \textsc{eastwards}   &  `guard the road'  \\
\japhug{tɯ-mtɕʰi}{mouth}  & \textsc{eastwards}   &  `shut up'  \\
\japhug{tɤ-pɯ}{young}  & \textsc{upwards}   &  `get pregnant' (animals)  \\
\japhug{ɯ-rtsɯz}{number},   &&  \textsc{eastwards} &  `memorize/mark down the   \\
\japhug{ɯ-χsɤr}{number} & number of' &   \\
\japhug{rɟɤlpu}{king}  & `upwards,   &  `become king'  \\
&upstream' & \\
\japhug{tɯrma}{household}  & \textsc{eastwards}   &  `establish a family'  \\
\lspbottomrule
\end{tabular}
\end{table}
 
The collocation of \forme{ndo} with  \japhug{ɯ-kʰrɤt}{set, determined}  can take finite complement clauses as in (\ref{ex:WkhrAt.kunwGnWndo}).

\begin{exe}
\ex \label{ex:WkhrAt.kunwGnWndo}
\gll tɕe kɯki χsɯ-ldʑa ra ɕi, kɯβde-ldʑa ra nɯnɯ, tɯʑo ɯ-kʰrɤt kú-wɣ-nɯ-ndo tɕe \\
\textsc{lnk} \textsc{dem}.\textsc{prox} three-long.object be.needed:\textsc{fact} \textsc{qu} four-long.object be.needed:\textsc{fact} \textsc{dem} \textsc{genr} \textsc{3sg}.\textsc{poss}-determined \textsc{ipfv}-\textsc{inv}-\textsc{auto}-take \textsc{lnk} \\
\glt `One has to control whether one needs three or four threads (for a given colour).' (vid-2014-04-29-092115)
\end{exe}
 
In addition to \japhug{rɟɤlpu}{king} in \tabref{tab:ndo.tr}, \forme{ndo} can be used with all nouns referring to a status or a charge (such as \japhug{χpɯn}{monk}) to mean `become $X$, assume the charge of $X$'. In this function, collocations in \forme{ndo} correspond either to \forme{rɯ-/rɤ-} (§\ref{sec:denom.intr.rA}) or \forme{nɯ-} (§\ref{sec:denom.intr.nW}) denominal derivations.

With numerals, \forme{ndo} indicates `be moving into one's $X$', without using the counted noun \forme{-pɤrme} `$X$ years old', as shown by (\ref{ex:42.kondo}).

\begin{exe}
\ex \label{ex:42.kondo}
\gll  kɯβdesqaptɯɣ, kɯβdesqamnɯz ko-ndo ɲɯ-ŋu ma \\
forty.one forty.two \textsc{ifr}-take \textsc{sens}-be \textsc{lnk} \\ 
\glt `She is moving into her forty-one, forty two years old.' (14-siblings)
(\japhdoi{0003508\#S304})
 \end{exe}
  
In addition, \forme{ndo} is used in the meaning `catch (a disease)' in the dummy subject construction, with \japhug{tɤmtsʰɤz}{hyperostosis} as only argument for instance.
 
Another dummy subject construction of \forme{ndo} with  the noun \japhug{tɯ-pʰoŋbu}{body},  is only found as the infinitive complement of either \japhug{kʰɯ}{be possible} or \japhug{sɤcʰa}{be possible} in negative form, the whole construction meaning `cannot help shivering' (due to cold), as in (\ref{ex:tWphoNbu.kAndo}). 

\begin{exe}
\ex \label{ex:tWphoNbu.kAndo}
\gll  ɯ-tɯ-mɯɕtaʁ kɤ-ti kɯ [tɯ-pʰoŋbu kɤ-ndo] ʑo mɯ-pɯ-sɤ-cʰa \\
\textsc{3sg}.\textsc{poss}-\textsc{nmlz}:\textsc{deg}-be.cold \textsc{inf}-say \textsc{erg} \textsc{genr}.\textsc{poss}-body \textsc{inf}-take \textsc{emph} \textsc{neg}-\textsc{pst}.\textsc{ipfv}-\textsc{prop}-can \\
\glt `It was so cold that one could not help shivering.' (29-RmGWzWn2)
(\japhdoi{0003730\#S13})
\end{exe} 

 \subsubsection{\japhug{pa}{do} } \label{sec:pa.lv}
The native verb \japhug{pa}{do}\footnote{The root \forme{pa} is cognate of Tibetan \tibet{བྱེད་བྱས་}{bʲed;bʲas}{do}; the yod medial Tibetan is unexplained, but not shared by cognates in its closest relatives \citep{jacques13yod}. } has been largely superseded by the Tibetan borrowing \japhug{βzu}{make} (§\ref{sec:Bzu.lv}) in the sense of `do, make'. It still means `do what' when used with the interrogative \japhug{tɕʰi}{what} (\ref{ex:tChi.WkWpa}, §\ref{sec:am.interrogative}), but as a complement-taking verb it rather means `discuss and decide' (§\ref{sec:pa.complements}). It is one of the few verb with ergative lability (§\ref{sec:lability.pass}), and its intransitive counterpart is a light verb used with numerals (§\ref{sec:pa.intr.lv}) and ideophones (§\ref{sec:idph.pa}).

The causative \forme{sɯpa} either has a tropative function `consider to be' (§\ref{sec:sig.caus.tropative}), or serves in various periphrastic constructions (§\ref{sec:sWpa.sABzu}; §\ref{sec:bare.dental.inf.sWpa}). Its passive \japhug{apa}{become} (§\ref{sec:passive.lexicalized}) is used as inchoative copula (§\ref{sec:copula.basic}), and its reflexive is the verb of pretense \japhug{ʑɣɤpa}{pretend} (§\ref{sec:refl.erg}, §\ref{sec:constr.participial.clause}; see \ref{ex:kWm.ci.thapa} below).

As a transitive light verb, \forme{pa} occurs in relatively fewer constructions than the previous verbs. It means `close' when used with \japhug{kɯm}{door} (\ref{ex:kWm.ci.thapa}) or \japhug{kʰɯɣɲɟɯ}{window}, but also with nouns meaning `light', `lamp' or electrical implements like phones, with either \textsc{downstream} or \textsc{eastwards} orientations.

\begin{exe}
\ex \label{ex:kWm.ci.thapa}
\gll ju-kɯ-ɕe tɤ-ʑɣɤpa nɤ, kɯm nɯ ci la-cɯ, ci tʰa-pa nɤ, kɤ-anbaʁ ɲɯ-ŋu \\
\textsc{ipfv}-\textsc{sbj}:\textsc{pcp}-go \textsc{aor}-pretend \textsc{add} door \textsc{dem} once \textsc{aor}:3\flobv{}-open once \textsc{aor}:3\flobv{}-close \textsc{add} \textsc{aor}-hide \textsc{sens}-be \\
\glt `She pretended to go [out], she opened the door, and then closed it, and hid [inside the house].' (2003 Kunbzang)
\end{exe}

Although \forme{pa} does occur with \forme{ci} in its adverbial meaning `once' (§\ref{sec:tense.aspect.adverbs}) as in (\ref{ex:kWm.ci.thapa}), a lexicalized collocation \forme{ci+pa} `get married' is also found (\ref{ex:ci.kApatCi}).

\begin{exe}
\ex \label{ex:ci.kApatCi}
\gll amaŋ, βdaʁmu, ci kɤ-pa-tɕi nɯstʰɯci ʑo tɤ-nɤrʑaʁ nɤ ri \\
\textsc{interj} lady one \textsc{aor}-do-\textsc{1du} so.much \textsc{emph} \textsc{aor}-pass(time) \textsc{add} \textsc{lnk} \\
\glt `Oh, my lady, so much time has passed since we got married, but ...' (2005 Kunbzang)
\end{exe}

With the Autive prefix (§\ref{sec:autoben.proper}), \forme{pa} occurs with nouns such as \japhug{βzaŋsa}{friend} or \japhug{ɣɯfsu}{friend} in the meaning `become friend'. This construction is also possible with social relation collectives in \forme{kɤndʑi-} (§\ref{sec:social.collective}) as in (\ref{ex:kAndZiGWfsu.nWpatCi}).
%βʑɯ chondɤre nɤki, kumpɣɤtɕɯ ni kɯ βzaŋsa to-nɯ-pa-ndʑi
%lWlu, 4

\begin{exe}
\ex \label{ex:kAndZiGWfsu.nWpatCi}
\gll a! jinde kɤndʑi-ɣɯfsu nɯ-pa-tɕi \\
\textsc{interj} now \textsc{coll}-friend \textsc{auto}-do:\textsc{fact}-\textsc{1du} \\
\glt `Let us be friends!' (smanmi 2003.1)
\end{exe}

\subsubsection{\japhug{ta}{put} }  \label{sec:ta.lv}
The manipulation verb \japhug{ta}{put} also means `leave (vt)' (\ref{ex:YotanW.Yota}, §\ref{sec:mismatch.errors}) and `let go' like \japhug{lɤt}{release} (a polysemy reminiscent of Chinese \ch{放过}{fàngguò}{let pass}).  Its  lexicalized agentless passive \japhug{ata}{be on} (§\ref{sec:passive.lexicalized}) is used as an existential verb (§\ref{sec:existential.basic}).

Unlike \forme{βzu} (§\ref{sec:Bzu.lv}) and \forme{lɤt} (§\ref{sec:lAt.lv}), it is not used with \forme{tɯ-} action nominals, except for the lexicalized nominalization \japhug{tɯpu}{moxibustion} (on which see §\ref{sec:lexicalized.action.nominals}).

\tabref{tab:ta.tr} presents a representative sample of collocation in which it occurs.

 \begin{table}
\caption{Examples of collocations with \japhug{ta}{put} } \label{tab:ta.tr}
\begin{tabular}{lllll}
\lsptoprule
Noun & Orientation & Meaning \\
\midrule
\japhug{tɯ-ku}{head} & \textsc{upstream}& `lie down' \\
\japhug{tɯ-mi}{foot, leg} & \textsc{downwards}& `tread' \\
\japhug{ɯ-taʁ}{on} & \textsc{eastwards}& `leave $X$ with, put the fault on' \\
\japhug{tɤ-ɕpʰɤt}{patch} & \textsc{eastwards}& `put a patch' \\
\japhug{tɯpu}{moxibustion} & \textsc{eastwards}& `use moxibustion' \\
\japhug{fsaŋ}{fumigation} & \textsc{downwards}& `make fumigations' \\
\midrule
\japhug{kʰon}{steamer} & \textsc{eastwards}& `cook by steam' \\
\japhug{tʂʰa}{tea} & \textsc{eastwards}& `made tea' \\
\midrule
\japhug{tɤ-rte}{hat} & \textsc{upwards}& `put on, wear' \\
\lspbottomrule
\end{tabular}
\end{table}
  
The collocations \forme{tʂʰa+ta} `make tea' and \forme{kʰon+ta} `steam' derive from `put $X$ on (the hearth tripod)', a meaning that is still obvious in examples such as  (\ref{ex:ndZitsxha.kanWtandZi}).

\begin{exe}
\ex \label{ex:ndZitsxha.kanWtandZi}
\gll tɤ-tɕɯ nɯ kɯ qapi χsɯm nɯ pa-ta ndɤre, nɯ ɯ-taʁ  ndʑi-tʂʰa ka-nɯ-ta-ndʑi   \\
\textsc{indef}.\textsc{poss}-boy \textsc{dem} \textsc{erg} white.stone three \textsc{dem} \textsc{aor}:3\flobv{}:\textsc{down}-put \textsc{lnk} \textsc{dem} \textsc{3sg}.\textsc{poss}-on \textsc{3du}.\textsc{poss}-tea \textsc{aor}:3\flobv{}-\textsc{auto}-put-\textsc{du} \\
\glt `The boy placed the three stones (on the ground), and they made their tea on it.'  (2003 Kunbzang)
\end{exe}

With head covers or implements worn on one's body (such as \japhug{χɕɤlmɯɣ}{glasses}), the verb \forme{ta} occurs with the autive in the meaning `put on, wear'.
 
In dummy subject construction (§\ref{sec:transitive.dummy}), \japhug{ta}{put} is also attested with nouns such as \japhug{ʁja}{verdigris} or \japhug{tɤ-rqʰu}{shell, hull, cuticle} to express  spontaneous growth on a surface, as in (\ref{ex:Rja.kute}) with the orientation `toward east' \forme{ku-} and stem III.

\begin{exe}
\ex \label{ex:Rja.kute}
\gll zaŋ cʰo raʁ ni ʁnaʁna ʑo ʁja ku-te ɲɯ-ŋu \\
copper \textsc{comit} brass \textsc{du} both \textsc{emph} verdigris \textsc{ipfv}-put[III] \textsc{sens}-be \\
\glt `Both copper and brass get verdigris.' (30-Com)
(\japhdoi{0003736\#S99})
\end{exe}
 
\subsubsection{\japhug{rku}{put in} } \label{sec:rku.lv}
The manipulation verb \japhug{rku}{put in} occurs in considerably fewer collocations than the previous verbs. In addition to its basic meaning, it can be used in the sense of `pour (water, grain) into a container' (example \ref{ex:atWci.tArke}, §\ref{sec:gen.beneficiary}), `give a parting present' (\ref{ex:arkuz.tarku}, §\ref{sec:z.nmlz}) and also `fix a joint dislocation'.

Prefixed with the Autive \forme{nɯ-} (§\ref{sec:autoben.proper}), it occurs with the relator noun \japhug{ɯ-pa}{below, under} (§\ref{sec:relator.nouns.3d}) and \japhug{tɯ-ku}{head} in \forme{ɯ-pa+nɯrku} `subjugate, vanquish' (\ref{ex:apa.pjWnWrkea}), literally `put $X$ under oneself' and \forme{tɯ-ku+nɯrku} `meddle into (other people's business)'. In both cases, the possessive prefix on the noun is coreferent with the subject.

\begin{exe}
\ex \label{ex:apa.pjWnWrkea}
\gll nɯ rɟɤlpu kɯ\redp{}kɯ-tu nɯ pjɯ-ɕɯ-nŋam-a, nɯnɯ a-pa pjɯ-nɯ-rke-a ra \\
\textsc{dem} king \textsc{total}\redp{}\textsc{sbj}:\textsc{pcp}-exist \textsc{dem} \textsc{ipfv}-\textsc{caus}-be.defeated[III]-\textsc{1sg} \textsc{dem} \textsc{1sg}.\textsc{poss}-under \textsc{ipfv}-\textsc{auto}-put.in[III]-\textsc{1sg} be.needed:\textsc{fact} \\
\glt `I have to defeat all the kings (in the world), to subjugate them.' (150821 edu de wangzi-zh)
(\japhdoi{0006402\#S8})
\end{exe}

%kɯ-kɯ-ɤlɯlɤt nɯ ɯ-ku ɕ-ku-nɯ-rke pjɤ-ŋu.
%150821 edu de wangzi-zh, 17

Without Autive prefix, \forme{rku} is also found in oblique nominalized form with \forme{tɯ-ku} and the negative existential \forme{me} as in (\ref{ex:nAku.sArku.kWme})

\begin{exe}
\ex \label{ex:nAku.sArku.kWme}
\gll nɤ-ku sɤ-rku kɯ-me \\
\textsc{2sg}.\textsc{poss}-head \textsc{obl}:\textsc{pcp}-put.in \textsc{sbj}:\textsc{pcp}-not.exist  \\
\glt `[Something] that is none of your business.' (elicited)
\end{exe}

In the dummy subject construction (§\ref{sec:transitive.dummy}), \forme{rku} is attested with \japhug{tɕʰɯwɯr}{blister} and \japhug{cimbɤrom}{blister} (\ref{ex:cimbArom.turke}) (see also \ref{ex:tCHWwWr.turke}, §\ref{sec:dummy.subj.object.relativization}).

\begin{exe}
\ex \label{ex:cimbArom.turke}
\gll tɤ-rʑaʁ kɯ-rɲɟi tu-kɯ-ŋke qʰe, tɯ-mɤpa ri cimbɤrom tu-rke ŋgrɤl. \\
\textsc{indef}.\textsc{poss}-time \textsc{sbj}:\textsc{pcp}-be.long \textsc{ipfv}-\textsc{genr}:S/O-walk \textsc{lnk} \textsc{genr}.\textsc{poss}-sole \textsc{loc} blister \textsc{ipfv}-put.in[III] be.usually.the.case:\textsc{fact} \\
\glt `If one walks for a long time (with bad shoes), one gets blisters on one's soles.' (27-tWfCAl)
(\japhdoi{0003710\#S131})
\end{exe}
 
\subsubsection{\japhug{tsʰoʁ}{attach} } \label{sec:tshoR.lv}
The verb  \japhug{tsʰoʁ}{attach} (§\ref{sec:anticausative.dummy}) only occurs in a handful of collocations with non-dummy subject: \forme{tɯ-χpɯm+tsʰoʁ} `knee' (§\ref{sec:sig.caus.collocations}) and \forme{kʰɯna+tsʰoʁ} `hunt with dogs'.

In the dummy subject construction (§\ref{sec:transitive.dummy}), \forme{tsʰoʁ} has the sense of `grow' with nouns referring to  plant parts (§\ref{sec:anticausative.dummy}),
in particular \forme{ɯ-mat+tsʰoʁ} `bear fruits' (see examples \ref{ex:sqamNuxpa.koRmWz}, §\ref{sec:temporal.postpositions} or \ref{ex:Wmat.kukWtshoR}, §\ref{sec:dummy.subj.object.relativization}). Its anticausative \japhug{ndzoʁ}{attach} (§\ref {sec:ndzoR.light.verbs}) also occurs with the same meaning.

\subsubsection{Other transitive collocations} \label{sec:other.collocation.tr}
\tabref{tab:other.tr.collocations} presents a list of collocations with verbs other than those discussed above. The manipulation verb \japhug{mɟa}{take}, `pick up' (whose historical morphology is discussed in §\ref{sec:antipassive.t}) and \japhug{cɯ}{open} each occur in a few non-compositional expressions. Note that \japhug{tɯtso}{experience} is the lexicalized action nominal of  \japhug{tso}{know, understand}  (§\ref{sec:semi.transitive}).
 
\begin{table}
\caption{Other transitive collocations (excluding orphan nouns/verbs)} \label{tab:other.tr.collocations}
\begin{tabular}{llll}
\lsptoprule
Noun & Verb & Meaning \\
\midrule
\japhug{ɯ-mpʰru}{after, following} & \japhug{mɟa}{take}, `pick up' & `continue (after a break), \\
&&change shift' \\
\japhug{tɯtso}{experience} &\japhug{mɟa}{take}, `pick up' & `have experience' \\
\japhug{tɯ-rnoʁ}{brain} & \japhug{cɯ}{open} & `deafen' \\
\japhug{tʂu}{road} & \japhug{cɯ}{open} & `give way' \\
\japhug{tɯ-nŋa}{debt} & \japhug{sti}{stop up} & `pay a debt by labour' \\
\japhug{tɤtsʰoʁ}{nail} & \japhug{no}{drive} (cattle)& `hammer nails' \\
\lspbottomrule
\end{tabular}
\end{table}
 
 Although the noun \japhug{tɤtsʰoʁ}{nail} probably derives from \japhug{tsʰoʁ}{attach} (§\ref{sec:tshoR.lv}), the two cannot be used as a collocation, and \forme{tɤtsʰoʁ} rather selects \japhug{no}{drive} or the causative verb \japhug{sɤtsa}{pierce} with the \textsc{downwards} preverbs to express the meaning `hammer nails into'.
 
\subsection{Frozen collocations} \label{sec:frozen.collocations}
The noun-verbs collocations studied in the previous sections involve partially grammaticalized verbs that can be combined with a considerable variety of nouns. The following focuses on more opaque collocations, which comprise either nouns or verbs that are not otherwise attested (§\ref{sec:orphan.noun}, §\ref{sec:orphan.verb}) and/or which are borrowed from Tibetan (§\ref{sec:borrowed.NV}).
 
\subsubsection{Orphan noun} \label{sec:orphan.noun}
Orphan nouns do not exist in free form, and are only found in collocations, or as members of frozen compounds.

Some orphan nouns are found in collocations with (mainly transitive) light verbs (§\ref{sec:tr.light.verbs}). A few items from this list originate from Tibetan verbs: \forme{ʂaʁ} is from \tibet{སྲེག་}{sreg}{burn}, \forme{ndaŋ} from \tibet{འདང་}{ⁿdaŋ}{think about, long for} and \forme{mtʰoŋ} from \tibet{མཐོང་}{mtʰoŋ}{see}, while  \forme{ɯ-rtsa} is from the noun \tibet{རྩ་}{rtsa}{root}. Pure Tibetan collocations are treated in §\ref{sec:borrowed.NV}. 

The orphan noun \forme{ɯ-sɲɯrʑu} is borrowed from \tibet{སྙིང་བཞོས་}{sɲiŋ.bʑos}{comfort}; the variant  \forme{ɯ-snɯrʑu} is semi-nativized by reanalyzing the \forme{sɲɯ-} from \tibet{སྙིང་}{sɲiŋ}{heart} as the \textit{status constructus} \forme{snɯ-} of its native cognate \japhug{tɯ-sni}{heart}. 

The nouns \forme{tɤ-rtɕʰɣaʁ} and \forme{ɯ-snɯrʑu} are otherwise attested as the bases from which the denominal verbs \japhug{sɤrtɕʰɣaʁ}{quibble about} and \japhug{ɣɤrtɕʰɣaʁ}{quibble} on the one hand, and \japhug{nɯsnɯrʑu}{comfort} on the other, are derived.  

\begin{table}
\caption{Orphan nouns used with light verbs} \label{tab:orphan.nouns.lv}
\begin{tabular}{lllll}
\lsptoprule
Noun & Light verb& Orientation &Meaning   \\
\midrule
\forme{ɯ-mtʰoŋ} & \japhug{ɬoʁ}{come out} & \forme{nɯ-} &`be exposed'  \\
\midrule
\forme{ɴɢartɯm} & \japhug{ɣɯt}{bring} & \forme{pɯ-} &`dive' (of birds of prey) \\
\midrule
\forme{jasa} & \japhug{ta}{put} & \forme{tɤ-} &`respect' \\
\forme{prɤdɤja} & & \forme{pɯ-} &`claw around, make a mess' \\
\forme{ʂaʁ} &   &\forme{kɤ-} & `brand' (with iron) \\
\midrule
\forme{taʁmbra} & \japhug{lɤt}{release} &\forme{tɤ-} & `jump' (of horse) \\
\forme{tɤlɟɣo} &   &\forme{kɤ-} & `catch with a lasso' \\
\forme{ɯ-ndaŋ} &&\forme{pɯ-}    & `think about' \\
\midrule
\forme{ɯ-pɯ} &\japhug{pa}{do}&\forme{tɤ-}    & `keep well, preserve' \\
\midrule
\forme{tɤ-rtɕʰɣaʁ} &\japhug{tɕɤt}{take out} &\forme{nɯ-} &  `hinder'  \\
\forme{ɯ-rtsa} &  &\forme{nɯ-} &  `investigate, get at the root of'  \\
\midrule
\forme{ɯŋaj} & \japhug{βzu}{make} &\forme{nɯ-} & `be self-satisfied'  \\
\forme{ɯ-snɯrʑu},&&\forme{nɯ-} & `comfort' \\
\forme{ɯ-sɲɯrʑu} \\
\lspbottomrule
\end{tabular}
\end{table}

Some of the orphan nouns in \tabref{tab:orphan.nouns.lv} are treated as direct object (for instance \forme{tɤlɟɣo}, \forme{ʂaʁ}, \forme{ɯ-snɯrʑu}), and other ones are semi-objects. The noun  \forme{ndaŋ} can have both grammatical functions; in  (\ref{ex:nAndaN.lata}) it is direct object (and the stimulus is encoded as its possessor), whereas in (\ref{ex:ndaN.talAt}) it is semi-object (and the stimulus is encoded as direct object and is indexed on \forme{lɤt}).

\begin{exe}
\ex 
\begin{xlist}
\ex \label{ex:nAndaN.lata}
\gll nɤ-ndaŋ lat-a ɕti \\
\textsc{2sg}.\textsc{poss}-think.about  release:\textsc{fact}-\textsc{1sg} be.\textsc{aff}:\textsc{fact} \\
\ex \label{ex:ndaN.talAt}
\gll ndaŋ ta-lɤt ɕti  \\
think.about 1\fl{}2-release:\textsc{fact} be.\textsc{aff}:\textsc{fact} \\
\end{xlist} 
\glt `I will think about/be considerate of you.' (elicited)
\end{exe}

Other collocations with orphan nouns listed in \tabref{tab:orphan.nouns.other} involve verbs not otherwise used in light verb constructions.


\begin{table}
\caption{Orphan nouns in lexicalized collocations} \label{tab:orphan.nouns.other}
\begin{tabular}{lllll}
\lsptoprule
Noun & Light verb& Meaning   \\
\midrule
\forme{ɯ-rɕa} &  \japhug{mŋɤm}{hurt}   &`cherish'  \\
 \forme{ɯ-tsɯ} & \japhug{rnaʁ}{be deep} &`keep the secret' \\
 \forme{ɯ-ndzɯɣ} & \japhug{maʁ}{not be} &`be terrible' \\
  \forme{ɯ-rka} & \japhug{ŋɤn}{be evil} &`harbour bad intentions' \\
\midrule
  \forme{ɯ-lu} & \japhug{cɯ}{open} & `lose consciousness' \\
\lspbottomrule
\end{tabular}
\end{table}

Most of the verbs in \tabref{tab:orphan.nouns.other} are intransitive; the experiencer/main referent is encoded as the possessive prefix on the intransitive subject, as the \textsc{2sg} in (\ref{ex:nAtsW.YWrnaR}).

\begin{exe}
\ex \label{ex:nAtsW.YWrnaR}
\gll nɤ-tsɯ ɲɯ-rnaʁ \\
\textsc{2sg}.\textsc{poss}-secret \textsc{sens}-be.deep \\
\glt `You are keeping the secret very well.' (elicited)
\end{exe}

The collocation \forme{ɯ-lu+cɯ}  is a transitive dummy construction (§\ref{sec:transitive.dummy}): the experiencer is marked as possessor of the object \forme{ɯ-lu} (\ref{ex:alu.pjAcW}), and no overt subject can appear. To express the meaning `cause to lose conscience', the causative \forme{sɯ-cɯ} is needed.

\begin{exe}
\ex \label{ex:alu.pjAcW}
\gll a-lu pjɤ-cɯ kʰi \\
\textsc{1sg}.\textsc{poss}-lose.conscience(1) \textsc{ifr}-lose.conscience(2) \textsc{hearsay} \\
\glt `I lost conscience (people said).' (elicited)
\end{exe}

Some of the nouns in \tabref{tab:orphan.nouns.other} have etymologies.  The noun \forme{ɯ-tsɯ}, though not otherwise attested, presumably originally meant `secret', as it serves as the base of the denominal transitive verb \japhug{nɤtsɯ}{hide}.  The nominal root \forme{-lu} is possibly related to the last syllable of \japhug{nɯkɯlu}{be lost}, `lose one's way', a denominal verb (§\ref{sec:denom.intr.nW}) from the \forme{kɯ-} participle (§\ref{sec:lexicalized.subject.participle}) of a verb \forme{*lu}, from which  \forme{ɯ-lu} would be derived (§\ref{sec:tA.abstract.nouns}, §\ref{sec:bare.action.nominals}).  

The noun \forme{ɯ-rɕa}, though non-attested on its own, occur in several collocations, respectively \forme{ɯ-rɕa+mŋɤm} `cherish' with \japhug{mŋɤm}{hurt} (example \ref{ex:arCa.mNAm},§\ref{sec:incorp.denom}),\footnote{This collocation can be denominalized as \japhug{nɤrɕɤmŋɤm}{cherish} (\tabref{tab:incorp.denom.collocation}, §\ref{sec:incorp.denom}). } \forme{ɯ-rɕa+χtɤt}  `concentrate, focus' with the transitive \japhug{χtɤt}{lean on}, and \forme{ɯ-rɕa+tsʰa} `be thoughtful and considerate of the feelings of others' with the orphan verb \forme{tsʰa} (§\ref{sec:orphan.verb}).
  

\subsubsection{Orphan verbs}  \label{sec:orphan.verb}
Orphan verbs are much fewer than orphan nouns. Among the verbs in \tabref{tab:orphan.verbs}, \forme{ri}  (example \ref{ex:asroR.kAtWrit},§\ref{sec:lability.pass}) is the transitive labile counterpart of  \japhug{ri}{remain}, `be left' (§\ref{sec:lability.pass}), and \forme{tsʰa} and \forme{loʁ} are borrowed from Tibetan \tibet{ཚ་}{tsʰa}{hot} and \tibet{ལོག་}{log}{be upside down}. The rest of the verbs are obscure.

\begin{table}
\caption{Orphan verbs } \label{tab:orphan.verbs}
\begin{tabular}{lllll}
\lsptoprule
Noun & Verb&   Meaning   \\
\midrule
  \japhug{tɯ-sroʁ}{life} &\forme{ri} (vt)   &`save $X$'s life'  \\
  \japhug{tɯ-sroʁ}{life} &\forme{nɯwɤtku} (vt)   &`risk one's life'  \\
  \japhug{tɯ-tɕa}{mistake} &\forme{nɯjɤt} (vt)   &`make amends, apologize'  \\
  \japhug{tɤ-mbrɯ}{anger} & \forme{ŋgɯ} (vi)   &`get angry'  \\
    \japhug{tɯ-mtɕʰi}{mouth} & \forme{χo} (vi)   &`talk big, exaggerate'  \\
  \midrule
\forme{ɯ-rɕa} &\forme{tsʰa} (vi)   &`be thoughtful and considerate'  \\
 \forme{tɯ-ʑi} &\forme{loʁ} (vi)   &`have nausea'  \\ 
 \forme{sala} &\forme{zrɯ} (vt)   &`be in the way, be a hindrance'  \\  
  \forme{ɯ-ʁo} &\forme{pʰi} (vt)   &`be disappointed by'  \\  
\lspbottomrule
\end{tabular}
\end{table}
 
The lower half of the Table includes collocations comprising both a orphan noun and an orphan verb. In such cases, the noun and the verb are given the same gloss, with the indices (1) on the former and (2) on the latter, as in (\ref{ex:sala.YWtWzri}). %{sec:stem3}

\begin{exe}
\ex \label{ex:sala.YWtWzri}
\gll nɯtɕu tɤ-rɤru ma sala ɲɯ-tɯ-zri ɲɯ-ŋu \\
\textsc{dem}:\textsc{loc} \textsc{imp}-get.up \textsc{lnk} be.a.hindrance(1) \textsc{sens}-2-be.a.hindrance(2) \textsc{sens}-be \\
\glt `Get up from there, you are in the way.' (elicited)
\end{exe}

Several orphan verbs are denominalized together with their noun in an incorporating construction (\tabref{tab:incorp.denom.collocation}, §\ref{sec:incorp.denom}).

Some orphan verbs can be subjected to voice derivations. For instance, \japhug{ɯ-ʁo+pʰi}{be disappointed by} has an anticausative \japhug{ɯ-ʁo+mbi}{be discouraged} (§\ref{sec:anticausative.collocation}) and a facilitative (§\ref{sec:facilitative.GA}). The verb \forme{ŋgɯ} has an irregular \forme{ɕɯ-} causative (§\ref{sec:caus.CW}).


\subsubsection{Borrowed collocations} \label{sec:borrowed.NV}
Several synchronically opaque noun-verb collocations (with orphan verbs and/or nouns) have been borrowed as a whole from Tibetan, unlike some constructions described in previous sections which combine a noun or a verb from Tibetan with a native word.

Among Tibetan collocations, \forme{tʰɯrʑi+ʑɯ} comprises the noun \japhug{tʰɯrʑi}{mercy}  (from \tibet{ཐུགས་རྗེ་ཞུ་}{tʰugs.rdʑe}{compassion}), which has many functions in Japhug (§\ref{sec:IPN.cause}), while the  transitive verb form \forme{ʑɯ} (from \tibet{ཞུ་}{ʑu}{ask}) is not otherwise attested,\footnote{Note however the verb \japhug{ndʑɯ}{accuse, report on}, which is probably borrowed from a non-classical present from \forme{*ⁿdʑu} of the verb \tibet{ཞུ་}{ʑu}{ask}. } with the meaning `ask for mercy' (\ref{ex:thWrZi.toZWnW}).

\begin{exe}
\ex \label{ex:thWrZi.toZWnW}
\gll ``wortɕʰi nɤ wojɤr" to-ti-nɯ tʰɯrʑi to-ʑɯ-nɯ. \\
please \textsc{add} please \textsc{ifr}-say-\textsc{pl} mercy \textsc{ifr}-ask-\textsc{pl} \\
\glt `[The boys] asked for mercy, saying `please' one after the other.' (160704 poucet4-v2)
(\japhdoi{0006097\#S25})
 \end{exe}  
 
In other collocations, both the noun and the verb are orphan forms. For instance, \japhug{tʰaʁ+tɕʰot}{take a decision} was borrowed from the locution \tibet{ཐག་ཆོད་}{tʰag.tɕʰod}{be decided}, built from the noun \tibet{ཐག་པ་}{tʰag.pa}{rope} and the intransitive verb \tibet{ཆོད་}{tɕʰod}{be cut off}, neither of which are attested as independent words into Japhug.   It does not take complement clause, and is better translated as `take a decision' rather than `decide'. The verbal element \forme{tɕʰot} is transitive (unlike Tibetan \tibet{ཆོད་}{tɕʰod}{be cut off}), and the person making the decision is indexed as subject (\ref{ex:thaR.pWtChota}).
 
\begin{exe}
\ex \label{ex:thaR.pWtChota}
\gll tʰaʁ pɯ-tɕʰot-a \\
decide(1) \textsc{aor}-decide(2)-\textsc{1sg} \\
\glt `I took a decision.' (elicited)
  \end{exe}  

Similarly, the intransitive impersonal constructions  \japhug{tʰo+tʰɯɣ}{match up}, `be compatible', `be conform to each other' (\ref{ex:tho.YWthWG}) from \tibet{ཐོ་ཐུག་}{tʰo.tʰug}{match}, and  \japhug{ɯ-ŋgu+tʰon}{be well-off}, `able to take care of oneself' (\ref{ex:ndZiNgu.mAthon}) from \tibet{འགོ་ཐོན་}{ⁿgo.tʰon}{able to take care of oneself} are syncrhonically opaque from a Japhug-internal perspective.
 
 
\begin{exe}
\ex \label{ex:tho.YWthWG}
\gll ɯʑo kɯ ta-tɯt cʰo nɤj tu-tɯ-ti nɯ tʰo ɲɯ-tʰɯɣ \\
\textsc{3sg} \textsc{erg} \textsc{aor}:3\flobv{}-say[II] \textsc{comit} \textsc{2sg} \textsc{ipfv}-2-say \textsc{dem} match.up(1) \textsc{sens}-match.up(2) \\
\glt `What he said and what you are saying match up.' (elicited)
\end{exe}  
 

\begin{exe}
\ex \label{ex:ndZiNgu.mAthon}
\gll   ci tʰɯ-kɯ-rgɯ\redp{}rgɤz ɲɯ-ɕti tɕe, ci kɯ-xtɕɯ\redp{}xtɕi ɲɯ-ɕti tɕe, ndʑi-ŋgu mɤ-tʰon tɕe, aʑo mɤ-ɣi-a \\
one \textsc{aor}-\textsc{sbj}:\textsc{pcp}-\textsc{emph}\redp{}be.old \textsc{sens}-be.\textsc{aff} \textsc{lnk} one \textsc{sbj}:\textsc{pcp}-\textsc{emph}\redp{}be.small \textsc{sens}-be.\textsc{aff} \textsc{lnk} \textsc{3du}.\textsc{poss}-be.well.off(1) \textsc{neg}-be.well.off(2):\textsc{fact} \textsc{lnk} \textsc{1sg} \textsc{neg}-come:\textsc{fact}-\textsc{1sg} \\
\glt `One of them is very old, the other one is very young, they are not able to take care of themselves, I will not come (to the palace and leave them behind).' (2011-05-nyima)
\end{exe}  
  
In addition, there are also partially compositional collocations, both of whose elements independently exist in Japhug. For instance, although \japhug{tɯ-sɯm+βdi}{be relieved} (§\ref{sec:other.collocation.intr}) takes its specific meaning from Tibetan \tibet{སེམས་བདེ་}{sems.bde}{be relieved}, both \japhug{tɯ-sɯm}{mind} (§\ref{sec:nouns.cognition.complement}) and \japhug{βdi}{be well} are otherwise found, and also used in a variety of constructions not calqued from Tibetan.

There is at least one case of collocation from Situ: \forme{kʰɤli+rgi} `be respected, have good reputation', from \forme{kʰalí} `wind, reputation' (cognate of \japhug{qale}{wind} and \japhug{ɯ-ʁle}{reputation}, §\ref{sec:apn.to.ipn}), with a verb from \tibet{དགེ་}{dge}{virtuous}.
 
\section{Copulas and existential verbs}  \label{sec:copula.existential}
Copulas and existential verbs stand out not simply by their morphological specificities (suppletive negation §\ref{sec:suppletive.negative}, infixed person index §\ref{sec:intr.person.irregular} and other §\ref{sec:verb.doubling}), but also by their uses as auxiliaries in periphrastic tenses (§\ref{sec:ipfv.periphrastic.TAME}), and various additional functions, including possessive constructions (§\ref{sec:possessive.constructions}), emphasis (§\ref{sec:affirmative.copula.function}), focalization (§\ref{sec:pseudo.cleft}, §\ref{sec:focalization.final.copula}), universal negation (§\ref{sec:negation.existential}), superlative (§\ref{sec:negative.existential.superlative}) and concessive conditionals (§\ref{sec:bare.inf.negative}). 

\subsection{Basic functions}  
\subsubsection{Copulas} \label{sec:copula.basic}
With the exception of a handful of predicative nouns (§\ref{sec:non.verbal.predicates}), a copula is required to make a noun phrase or a pronoun predicative in Japhug.
 
The copulas are a distinct subclass of semi-transitive verbs (§\ref{sec:semi.transitive}), whose semi-object is the nominal predicate. There are two assertive copulas, the neutral \japhug{ŋu}{be} and the emphatic affirmative \japhug{ɕti}{be}, and one suppletive negative copula \japhug{maʁ}{not be} (§\ref{sec:suppletive.negative}). The emphatic affirmative copula occurs in particular to express contrast, as in (\ref{ex:Ctia.ma.maRa}).

\begin{exe}
\ex \label{ex:Ctia.ma.maRa}
\gll  aʑo βʑɯ ɕti-a ma, nɤki, pɣa maʁ-a \\
\textsc{1sg} mouse be.\textsc{aff}:\textsc{fact} \textsc{lnk} \textsc{filler} bird not.be:\textsc{fact} \\
\glt  `(The bat said:) I am a mouse, not a bird.' (140427 bianfu yu huangshulang-zh)
(\japhdoi{0003838\#S7})
\end{exe} 

The copulas cannot be used with inchoative meanings (`become')  in the Aorist (§\ref{sec:aor.inchoative}) or the Inferential (§\ref{sec:ifr.inchoative}) unlike other stative verbs. The Aorist of the copulas are only attested to fix a point of temporal reference (§\ref{sec:temporal.reference}) and in periphrastic tenses (examples \ref{ex:tWwGCaB.tANu} in  §\ref{sec:proximative.periphrastic} and \ref{ex:tAcha.tANu} in  §\ref{sec:prerequisite.clause}). The lexicalized passives \forme{apa}  and \forme{aβzu}  (§\ref{sec:passive.lexicalized}) replace the copula to express inchoative meaning, as shown by (\ref{ex:ki.kWfse.naBzua}). 

\begin{exe}
\ex \label{ex:ki.kWfse.naBzua}
\gll   aʑo tɯmɯkɯmpɕi ra nɯ-tɕɯ pɯ-ŋu-a ri, [ki kɯ-fse] nɯ-aβzu-a \\
\textsc{1sg} heaven \textsc{pl} \textsc{3pl}.\textsc{poss}-son \textsc{pst}.\textsc{ipfv}-be-\textsc{1sg} \textsc{lnk} \textsc{dem}.\textsc{prox} \textsc{sbj}:\textsc{pcp}-be.like \textsc{aor}-become-\textsc{1sg} \\
\glt `I used to be a son of heaven, but I became like that.' (divination 2003)
\end{exe}

Person indexation on copula generally follows the subject, in particular when both subject and semi-object are non-third person as in (\ref{ex:nAZo.maRa}).

\begin{exe}
\ex \label{ex:nAZo.maRa}
\gll aʑo nɤʑo maʁ-a kɯ, a-mu mɤ-nɤmqe-a. \\
\textsc{1sg} \textsc{2sg} not.be:\textsc{fact}-\textsc{1sg} \textsc{erg} \textsc{1sg}.\textsc{poss}-mother \textsc{neg}-scold:\textsc{fact}-\textsc{1sg} \\
\glt `I am not you, I will not scold my mother.' (elicited)
\end{exe}

However, when the predicate is a first or second person pronoun and the subject third person, the copula indexes the non-third person argument. In the pseudo-cleft (\ref{ex:pWkWcha.nW})  for instance, it would not be  grammatical to  replace \forme{aʑo ɕti-a}  with a third \textsc{3sg} copula $\dagger$\forme{aʑo ɕti}. (see also  \ref{ex:akAnWrga.nW.nAZo} in §\ref{sec:pseudo.cleft}).  

\begin{exe}
\ex \label{ex:pWkWcha.nW}
\gll [pɯ-kɯ-cʰa] nɯ aʑo ɕti-a \\
\textsc{aor}-\textsc{sbj}:\textsc{pcp}-can \textsc{dem} \textsc{1sg} be.\textsc{aff}:\textsc{fact}-\textsc{1sg} \\
\glt `The one who succeeded [in doing these things] was me.' (qachGa 2003)
\end{exe}

We also find examples of indexation with a (third person) predicative noun, as in (\ref{ex:stAmku.nWra.NunW}), where the verb has plural indexation like the predicative noun phrase \forme{stɤmku nɯra}, whereas the subject is in the singular. Note that the plural \forme{nɯra} here cannot be interpreted as approximate location (§\ref{sec:plural.determiners}), because in the next sentence the grasslands are analogically referred to by the plural demonstrative pronoun \forme{nɯra} (§\ref{sec:anaphoric.demonstrative.pro}).

\begin{exe}
\ex \label{ex:stAmku.nWra.NunW}
\gll  stu ɯ-sɤ-dɤn nɯ [stɤmku nɯra] ŋu-nɯ. tɕe nɯra nɯ-ŋgɯ tɕe tu-ɬoʁ ŋu tɕe \\
most \textsc{3sg}.\textsc{poss}-\textsc{obl}:\textsc{pcp}-be.many \textsc{dem} grassland \textsc{dem}:\textsc{pl} be:\textsc{fact}-\textsc{pl} \textsc{lnk} \textsc{dem}:\textsc{pl} \textsc{3pl}.\textsc{poss}-inside \textsc{loc} \textsc{ipfv}-come.out be:\textsc{fact} \textsc{lnk} \\
\glt `The place where it is most numerous is the grasslands, and it grows in these' (19-qachGa mWntoR)
(\japhdoi{0003546\#S23})
\end{exe}

\subsubsection{Existential verbs} \label{sec:existential.basic}
Unlike other languages of the area such as Khroskyabs (\citealt[250--252]{lai17khroskyabs}), Japhug is relatively poor in existential verbs. The verb \japhug{tu}{exist} and its suppletive forms (the Sensory \forme{ɣɤʑu} §\ref{sec:intr.person.irregular}, the negative \japhug{me}{not exist} and the Sensory negative \forme{maŋe},  §\ref{sec:suppletive.negative}) have no selection restrictions on their  subjects, and are compatible  with abstract nouns (\ref{ex:aBlu.ci.tu}), inanimate objects (\ref{ex:khANqra.ci.pjAtu}) and  humans (\ref{ex:kWnApAri.aZo.ma.mea}, \ref{ex:RzAmi.pjAtundZi}).

\begin{exe}
\ex \label{ex:aBlu.ci.tu}
\gll   aʑo a-βlu ci tu \\
\textsc{1sg} \textsc{1sg}.\textsc{poss}-trick \textsc{indef} exist:\textsc{fact} \\
\glt `I have an idea.' (150829 taishan zhi zhu-zh)
(\japhdoi{0006350\#S196})
\end{exe}

\begin{exe}
\ex \label{ex:khANqra.ci.pjAtu}
\gll   kɯɕɯŋgɯ nɯnɯtɕu kʰɤɴqra ci pjɤ-tu. \\
in.former.times \textsc{dem}:\textsc{loc} house.in.ruin \textsc{indef} \textsc{ifr}.\textsc{ipfv}-exist \\
\glt `In former times, there used to be a house in ruin there.' (140522 Kamnyu zgo)
(\japhdoi{0004059\#S247})
\end{exe}

The existential verbs index the person of the intransitive subject, as shown in (\ref{ex:kWnApAri.aZo.ma.mea}) (see §\ref{sec:headless.relatives.quantification} on this construction).

\begin{exe}
\ex \label{ex:kWnApAri.aZo.ma.mea}
\gll  [jɯɣmɯr kɯ-nɤpɤri] aʑo ma me-a. \\
this.evening \textsc{sbj}:\textsc{pcp}-have.dinner \textsc{1sg} apart.from not.exist:\textsc{fact}-\textsc{1sg} \\
\glt `This evening nobody is having dinner (at home) apart from me.'   (conversation, 16-04-28)
\end{exe}

With third person subjects, number indexation is possible as in (\ref{ex:RzAmi.pjAtundZi}), but optional  (§\ref{sec:optional.indexation}), and never found in the possessive construction (§\ref{sec:possessive.constructions}).

\begin{exe}
\ex \label{ex:RzAmi.pjAtundZi}
\gll  kɯɕɯŋgɯ ʁzɤmi ci pjɤ-tu-ndʑi \\
in.former.times couple \textsc{indef} \textsc{ifr}.\textsc{ipfv}-exist-\textsc{du} \\
\glt `In former times, there was a husband and his wife.' (rkangrgyal 2002.2)
 \end{exe}
 
The assertive existential verbs can mean `be located', and take associated motion prefixes (§\ref{sec:associated.motion}) as in (\ref{ex:NotCu.CpWtWtunW}).

 \begin{exe}
\ex \label{ex:NotCu.CpWtWtunW}
\gll  ŋotɕu ɕ-pɯ-tɯ-tu-nɯ? \\
where \textsc{tral}-\textsc{pst}:\textsc{ipfv}-2-exist-\textsc{pl} \\
\glt `Where (in which places) have you been?' (2003sras)
\end{exe} 

Other meanings include `be alive' as in (\ref{ex:kutWtu}), and `be present' (especially in converbial form \forme{$X$ kɯ-tu ʑo} `in $X$'s presence', see \ref{ex:Zara.kWtu.Zo}, §\ref{sec:inf.converb}). 

\begin{exe}
\ex \label{ex:kutWtu}
\gll  nɤj ku-tɯ-tu ɯ́-ŋu \\
\textsc{2sg} \textsc{prs}-2-exist \textsc{qu}-be:\textsc{fact} \\
\glt `Are you [still] alive?' (Nyima Wodzer 2003.2)
\end{exe} 

The negative existential verb \forme{me} is also employed in a comparative construction (§\ref{sec:existential.comparative}), and with infinitives in a construction expressing impossibility (§\ref{sec:inf.exist}, §\ref{sec:participial.complements.negative}).

Unlike copulas, existential verbs can be used in Aorist, Inferential and Imperfective with an inchoative meaning (see  \ref{ex:ci.totu} in §\ref{sec:preverb.gain} and \ref{ex:mtshu.YAme} in §\ref{sec:preverb.loss}).
 
The lexicalized agentless passives (§\ref{sec:passive.lexicalized}) \japhug{ata}{be on}  and \japhug{arku}{be in}, are nascent existential verbs. They do not have semantic restrictions on the subject as in Khroskyabs or Stau (they are attested with first or second person human subjects, as in \ref{ex:parkua}, §\ref{sec:passive}), but indicate the location of the subject: on a surface or in a building in the case of \forme{ata} (\ref{ex:kha.ata}) and inside a closed and narrow space (or inside a compact matter) for \forme{arku} (\ref{ex:tArGe.ci.arku}).
 
\begin{exe}
\ex \label{ex:kha.ata}
\gll  mɯ-to-ndo-t-a, kʰa a-ta ɕti tɕe \\
\textsc{neg}-\textsc{ifr}-take-\textsc{pst}:\textsc{tr}-\textsc{1sg} house \textsc{pass}-put:\textsc{fact} be.\textsc{aff}:\textsc{fact} \textsc{lnk} \\
\glt `I did not take it [with me], [I] left it at home.' (150830 afanti-zh)
(\japhdoi{0006380\#S121})
 \end{exe}
  
\begin{exe}
\ex \label{ex:tArGe.ci.arku}
\gll a-wa ɯ-kɯr ɯ-ŋgɯ zɯ, nɯtɕu tɤ-rɣe ci a-rku tɕe \\
\textsc{1sg}.\textsc{poss}-father \textsc{3sg}.\textsc{poss}-mouth \textsc{3sg}.\textsc{poss}-in \textsc{loc} \textsc{dem}:\textsc{loc} \textsc{indef}.\textsc{poss}-pearl \textsc{indef} \textsc{pass}-put.in:\textsc{fact} \textsc{lnk} \\
\glt `In my father's mouth, there is a pearl.' (150902 hailibu-zh)
(\japhdoi{0006316\#S35})
  \end{exe}

In addition, the anticausative  verb \japhug{ndzoʁ}{be attached} (§\ref{sec:anticausative.dummy}, §\ref{sec:ndzoR.light.verbs} can be used to describe the presence of limbs, appendices, excrescence, thorns, hairs and twigs on living organisms, as in (\ref{ex:WmAlAjaR.kundzoR}) (see also \ref{ex:WmWntoR.kundzoR}, §\ref{sec:passive.agent}).

\begin{exe}
\ex \label{ex:WmAlAjaR.kundzoR}
\gll ɯ-ku ɯ-rkɯ ri ʁɟa ʑo ɯ-mɤlɤjaʁ ra ku-ndzoʁ ɲɯ-ŋu \\
\textsc{3sg}.\textsc{poss}-head \textsc{3sg}.\textsc{poss}-side \textsc{loc} completely \textsc{emph} \textsc{3sg}.\textsc{poss}-limb \textsc{pl} \textsc{ipfv}-\textsc{acaus}:attach \textsc{sens}-be \\
\glt `Its legs are located (i.e. attached) next to its head.' (26-mYaRmtsaR)
(\japhdoi{0003674\#S44})
\end{exe}

\subsection{Possessive constructions} \label{sec:possessive.constructions}
There are two main possessive constructions in Japhug. In the first one, illustrated by (\ref{ex:amkWm.tu}), the possessum is the intransitive subject of an existential verb (§\ref{sec:existential.basic}) and the possessor is indicated by a possessive prefix on the possessum (§\ref{ex:prefix.expression.of.possession}) and (optionally) a genitive phrase (example \ref{ex:11.Zo.pjAtunW} below, §\ref{sec:gen.possession}). In the second one (\ref{ex:tamkWm.aroa}), involving the semi-transitive verb  \japhug{aro}{own} (§\ref{sec:semi.transitive}), the possessor is subject, and the possessum semi-object (§\ref{sec:semi.object}); when the possessum is an inalienably possessed noun (§\ref{sec:inalienably.possessed}), it selects the indefinite possessor prefix. 

\begin{exe}
\ex \label{ex:have.pillow}
\begin{xlist}
\ex \label{ex:amkWm.tu}
\gll a-mkɯm tu \\
\textsc{1sg}.\textsc{poss}-pillow exist:\textsc{fact} \\
\ex \label{ex:tamkWm.aroa}
\gll tɤ-mkɯm aro-a \\
\textsc{indef}.\textsc{poss}-pillow have:\textsc{fact}-\textsc{1sg} \\
\glt `I have a pillow.' (elicited)
\end{xlist}
\end{exe}
 
  
 % qartshaz phu nɯ ɣɯ ɯ-ʁrɯ ɣɤʑu.
 \subsubsection{Mihi est possessive} \label{sec:possessive.mihi.est}
The \textit{mihi est}-type construction in (\ref{ex:amkWm.tu}) is by far the most frequent one.  It differs from the usual existential construction in that the number of the possessum (intransitive subject) is never indexed on the verb: in (\ref{ex:11.Zo.pjAtunW}) for instance, the subject \forme{ɯ-tɕɯ} takes the numeral \japhug{sqaptɯɣ}{eleven} but the existential verb \forme{pjɤ-tu} lacks plural indexation. Although number indexation in the existential construction is optional (§\ref{sec:existential.basic}, §\ref{sec:optional.indexation}), its presence is however more common than its absence. The non-indexation in (\ref{ex:11.Zo.pjAtunW}) is thus indicative of a significant syntactic difference between the existential and the possessive constructions, despite superficial similarity.
  
\begin{exe}
\ex \label{ex:11.Zo.pjAtunW}
\gll   rɟɤlpu nɯnɯ ɣɯ ɯ-tɕɯ sqaptɯɣ ʑo pjɤ-tu. \\
king \textsc{dem} \textsc{gen}  \textsc{3sg}.\textsc{poss}-son eleven \textsc{emph} \textsc{ifr}.\textsc{ipfv}-exist  \\
\glt `The king had eleven sons.' (140520 ye tiane-zh)
(\japhdoi{0004044\#S4})
 \end{exe}
 
Inalienably possessed possessums normally take a possessive prefix coreferent with the possessor, even when the possessor is overt and marked with the genitive. In (\ref{ex:tWtWpW.raNri.GW}), we observe a string of possessive clauses in parataxis, each sharing the same possessor with the genitive (\forme{tɯ-tɯpɯ raŋri ɣɯ}). The possessums of all of these clauses are inalienably possessed nouns (domestic animals). In the first three clauses, the expected \textsc{3pl} possessor prefix is present. In the last two however the possessive prefix is absent; this is one of the very rare cases where the possessor is not marked on the possessum in this construction in the corpus.
 
 \begin{exe}
\ex \label{ex:tWtWpW.raNri.GW}
\gll tɯ-tɯpɯ raŋri ɣɯ, nɤkinɯ, \textbf{nɯ-mbro} pjɤ-tu, \textbf{nɯ-jla} pjɤ-tu, \textbf{nɯ-nɯŋa} pjɤ-tu, \textbf{qaʑo} pjɤ-tu, \textbf{tsʰɤt} pjɤ-tu. \\
one-household each \textsc{gen} \textsc{filler} \textsc{3pl}.\textsc{poss}-horse \textsc{ifr}.\textsc{ipfv}-exist 
\textsc{3pl}.\textsc{poss}-hybrid.yak \textsc{ifr}.\textsc{ipfv}-exist \textsc{3pl}.\textsc{poss}-cow \textsc{ifr}.\textsc{ipfv}-exist sheep  \textsc{ifr}.\textsc{ipfv}-exist  goat  \textsc{ifr}.\textsc{ipfv}-exist  \\
\glt `Every household used to have horse(s), cow(s), hybrid yak(s), sheep and goat(s).' (150820 kAnWCkat)
(\japhdoi{0006256\#S2})
 \end{exe}
 
 The  \textit{mihi est} construction can be causativized by subjecting the existential \forme{tu} to the \forme{ɣɤ-} derivation (§\ref{sec:velar.caus.modal}), yielding the verb \japhug{ɣɤtu}{cause to have}, which marks the beneficiary (corresponding to the possessor of the intransitive construction) with the genitive case, as \forme{nɤʑɯɣ} in (\ref{ex:nAZWG.tAGAtuta}), or a as possessor of the object (as in \ref{ex:axCAt.Zo.tuGAte}, §\ref{sec:velar.caus.modal}).
 
 \begin{exe}
\ex \label{ex:nAZWG.tAGAtuta}
\gll tu-βze-a kɤ-cʰa nɯra lonba ʑo nɤʑɯɣ tɤ-ɣɤ-tu-t-a ɕti tɕe \\
\textsc{ipfv}-make[III]-\textsc{1sg} \textsc{inf}-can \textsc{dem}:\textsc{pl} all \textsc{emph} \textsc{2sg}:\textsc{gen} \textsc{aor}-\textsc{caus}-exist-\textsc{pst}:\textsc{tr}-\textsc{1sg} be.\textsc{aff}:\textsc{fact} \textsc{lnk} \\
\glt  `I endowed you with all the things I could make.' (140425 shizi puluomixiusi he daxiang-zh)
(\japhdoi{0003798\#S15})
 \end{exe}
 
 
When the \textit{mihi est} possessive construction undergoes relativization, there is ambiguity as to whether the relativized element is the possessum or the possessor, since both intransitive subjects (§\ref{sec:S.possessor.relativization}) and possessors (§\ref{sec:intr.subject.relativization}) are relativized by means of (mostly head-internal) subject participial relatives (§\ref{sec:subject.participle.subject.relative}). In (\ref{ex:WlaXtCha.pWkWtu}), the relativized element is the possessum, and in (\ref{ex:jla.nWRrW.kWtu}) its is the possessor, but these two relative clauses have exactly the same surface structure.
 
 \begin{exe}
\ex \label{ex:WlaXtCha.pWkWtu}
\gll  [nɯɕɯŋgɯ ɣɯ ɯ-\textbf{laχtɕʰa} pɯ-kɯ-tu] nɯra ɕ-tú-wɣ-sɯ-rtoʁ tɕe, \\
before \textsc{gen} \textsc{3sg}.\textsc{poss}-thing \textsc{pst}.\textsc{ipfv}-\textsc{sbj}:\textsc{pcp}-exist \textsc{dem}:\textsc{pl} \textsc{tral}-\textsc{ipfv}-\textsc{inv}-\textsc{caus}-look \textsc{lnk} \\
\glt `They go (there) and show [the child$_i$ objects] that he$_i$ used to have before (in his$_i$ previous life, when he$_i$ was a lama).' (160722 skWBli)
(\japhdoi{0006227\#S5})
 \end{exe}
 
\begin{exe}
\ex \label{ex:jla.nWRrW.kWtu}
\gll [\textbf{jla} nɯ-ʁrɯ kɯ-tu] ra kɯnɤ, nɯ-rpaʁ kɯ ɲɯ-z-rɤɕi-nɯ pɯ-ŋu tɕe, \\
hybrid.yak \textsc{3pl}.\textsc{poss}-horn \textsc{sbj}:\textsc{pcp}-exist \textsc{dem}:\textsc{pl} also \textsc{3pl}.\textsc{poss}-shoulder \textsc{erg} \textsc{ipfv}-\textsc{caus}-pull-\textsc{pl} \textsc{pst}.\textsc{ipfv}-be \textsc{lnk} \\
\glt `Even the hybrid yaks that had horns used to pull [the plough] with their shoulders (rather than with their horns).' (25-stuxsi)
(\japhdoi{0003660\#S21})
  \end{exe}


Possessive constructions with the existential verb \forme{kɯ-tu} in participial form can be combined with a negative existential verb \forme{kɯ-tu me/kɯ-tu maŋe} to express the meaning `not have any' as in (\ref{ex:Wmdzu.kWtu.me}).
 
\begin{exe}
\ex \label{ex:Wmdzu.kWtu.me}
\gll  ɯ-ru nɯnɯ, kɯ-mpɕɯ\redp{}mpɕu ʑo ŋu, [ɯ-mdzu ri kɯ-tu] me \\
\textsc{3sg}.\textsc{poss}-stalk \textsc{dem} \textsc{sbj}:\textsc{pcp}-\textsc{emph}\redp{}be.smooth \textsc{emph} be:\textsc{fact} \textsc{3sg}.\textsc{poss}-thorn also \textsc{sbj}:\textsc{pcp}-exist not.exist:\textsc{fact} \\
\glt `Its stalk, it is very smooth, and it does not have any thorns.' (11-qrontshom)
(\japhdoi{0003482\#S40})
\end{exe} 


The \textit{mihi est} possessive construction can be a gradable predicate and occur in a superlative construction (§\ref{sec:negative.existential.superlative}) as in (\ref{ex:WtshuxtoR.kWtu}) with the abstract noun \japhug{tsʰuxtoʁ}{loyalty}.  

\begin{exe}
\ex \label{ex:WtshuxtoR.kWtu}
\gll  fsapaʁ nɯ ɯ-ŋgɯ zɯ (...) kʰɯna kɯ-fse ʑo, nɤki, ɯ-tsʰuxtoʁ kɯ-tu me kʰi \\
animal \textsc{dem} \textsc{3sg}.\textsc{poss}-in \textsc{loc} {  } dog \textsc{sbj}:\textsc{pcp}-be.like \textsc{emph} \textsc{filler} \textsc{3sg}.\textsc{poss}-loyalty \textsc{sbj}:\textsc{pcp}-exist not.exist:\textsc{fact} \textsc{hearsay} \\
\glt  `Among domestic animals, the dog is the most loyal one.' (i.e. among domestic animals, there isn't any one whose loyalty like that of a dog) (05-khWna)
(\japhdoi{0003398\#S4})
\end{exe}


Some possessive constructions, involving in particular abstract nouns,  are lexicalized and are better described as noun-verb collocations (§\ref{sec:existential.light.verbs}).

 \subsubsection{Non-genitive possessor} \label{sec:possessive.existential2}
There is in addition a third construction, intermediate between those illustrated in (\ref{ex:have.pillow}), involving an existential verb, but in which the possessor is rather intransitive subject, indexed on the verb. It is only attested is double negative, with the negative existential verb \japhug{me}{not exist} and a negative verb participle such as \forme{mɤ-kɯ-pe} `(something) that is not good' as possessum: compare the \textit{mihi est} construction (\ref{ex:mAkWpe.kume}) with this third construction (\ref{ex:mAkWpe.kumea}), where the verb has \textsc{1sg} indexation. Only one example is found in the whole corpus (identical to example \ref{ex:mAkWpe.kume}, but with \textsc{1pl} possessor).
 
\begin{exe}
\ex 
\begin{xlist}
\ex \label{ex:mAkWpe.kume}
\gll aʑɯɣ mɤ-kɯ-pe ku-me \\
\textsc{1sg}:\textsc{gen} \textsc{neg}-\textsc{sbj}:\textsc{pcp}-be.good \textsc{prs}-not.exist \\
\ex \label{ex:mAkWpe.kumea}
\gll aʑo mɤ-kɯ-pe ku-me-a \\
 \textsc{1sg}  \textsc{neg}-\textsc{sbj}:\textsc{pcp}-be.good \textsc{prs}-not.exist-\textsc{1sg} \\
 \glt `I don't have any problem/anything bad.' (elicited, based on real examples) 
 \end{xlist} 
\end{exe}

% mɤ-kɯ-pe ku-me-j

\subsection{Postverbal copulas}  \label{sec:postverbal.copulas}
Copulas are commonly found in postverbal position, in periphrastic TAME constructions (§\ref{sec:ipfv.periphrastic.TAME}, §\ref{sec:aor.narrative}, §\ref{sec:pst.ifr.ipfv.periphrastic}, §\ref{sec:proximative.periphrastic}) and in emphatic and focalization functions, as detailed below.

These constructions have in common that the copula remains in \textsc{3sg} form, regardless of the core arguments of the preceding verb. In (\ref{ex:nWtCu.Ce.Nu}) for instance, the copula \forme{ŋu} lacks the \textsc{1sg} suffix found on the verb \forme{ɕe-a}, and using here the \textsc{1sg} form \forme{ŋu-a} would be agrammatical.


\begin{exe}
\ex \label{ex:nWtCu.Ce.Nu}
\gll nɯtɕu ɕe-a ŋu \\
\textsc{dem}:\textsc{loc} go:\textsc{fact}-\textsc{1sg} be:\textsc{fact} \\
\glt `(For all these reasons) I am going there.' (2011-04-smanmi)
\end{exe}

A question concerning the syntactic structure of these constructions is whether part of the clause preceding the copula (\forme{nɯtɕu ɕe-a}) is a subordinate clause, the copula being the main verb of the sentence. The fact that post-verbal adverbs such as \forme{ntsɯ} (§\ref{sec:postverbal.adv}) can located between the verb and the sentence-final copula, as in (\ref{ex:pe.ntsW.Cti}), is a clue that the part of the sentence preceding the copula is a syntactic constituent.

\begin{exe}
\ex \label{ex:pe.ntsW.Cti}
\gll  tɯ\redp{}tɤ-tɯ-nɤma-t rcanɯ, pe ntsɯ ɕti\\
\textsc{total}\redp{}\textsc{aor}-2-make-\textsc{pst}:\textsc{tr} \textsc{unexp}:\textsc{foc} be.good:\textsc{fact} always be.\textsc{aff}:\textsc{fact}\\
\glt `Everything that you do is always good.' (150822 laoye zuoshi zongshi duide-zh)
(\japhdoi{0006298\#S222})
\end{exe}

The pre-copula constituents (\forme{nɯtɕu ɕe-a} and \forme{pe ntsɯ}) cannot be analyzed as finite relative clauses (§\ref{sec:finite.relatives}), since this type of relatives are restricted to relativizing objects (§\ref{sec:object.relativization}) and goals, and are thus not attested with most intransitive verbs, whereas the postverbal copula constructions are found with all intransitive verbs without restriction. Another possibility would be to analyze the pre-copula constituents as complement clauses (§\ref{chap:complement.clauses}), whose function would be that of semi-object of the copula (§\ref{sec:copula.basic}). In this hypothesis, example (\ref{ex:nWtCu.Ce.Nu}) would literally be `it is (the fact that) I am going there', and (\ref{ex:pe.ntsW.Cti}) `it is that it is always good'.

 
\subsubsection{Emphatic assertion/negation} \label{sec:affirmative.copula.function}
The postverbal copulas can have scope over the whole sentence. Copulas of opposite polarity \forme{ŋu}/\forme{ɕti} and \forme{maʁ} can be used to mark an emphatic contrast between two predicates, as in (\ref{ex:Nu.tuwGskAr.maR}) and (\ref{ex:YWGAwua.maR2}) (§\ref{sec:periphrastic.negation}).

\begin{exe}
\ex \label{ex:Nu.tuwGskAr.maR}
\gll  tú-wɣ-ɕtʂo ŋu ma tú-wɣ-skɤr maʁ \\
\textsc{ipfv}-\textsc{inv}-measure be:\textsc{fact} \textsc{lnk} \textsc{ipfv}-\textsc{inv}-weigh not.be:\textsc{fact} \\
\glt `One measures (the quantity to be used) by scooping, not by weighing.' (31-cha)
(\japhdoi{0003764\#S12})
\end{exe}

\begin{exe}
\ex \label{ex:YWGAwua.maR2}
\gll ɲɯ-ɣɤkʰɯ ndʐa ɕti wo ma, tɕe a-mɲaʁ ɲɯ-ɕɯ-mŋɤm ndʐa ɕti ma, ɲɯ-ɣɤwu-a maʁ \\
\textsc{sens}-have.smoke reason be.\textsc{aff}:\textsc{fact} \textsc{sfp} \textsc{lnk} \textsc{lnk} \textsc{1sg}.\textsc{poss}-eye \textsc{sens}-\textsc{caus}-hurt reason be.\textsc{aff}:\textsc{fact} \textsc{lnk} \textsc{sens}-cry-\textsc{1sg} not.be:\textsc{fact} \\
\glt `[The reason why I shed tears] is because it is smoky, and it hurts my eyes, it is not that I am crying.' (qaCpa 202)
\end{exe}

A postverbal assertive copula can put emphasis on the trustworthiness and reliability of a statement  (\ref{ex:tWsi.Cti}).

\begin{exe}
\ex \label{ex:tWsi.Cti}
\gll  tɕetʰa nɤʑo tɯ-si ɕti \\
later \textsc{2sg} 2-die:\textsc{fact} be.\textsc{aff}:\textsc{fact} \\
\glt `Otherwise you will (certainly) die.' (2011-04-smanmi)
\end{exe}

The postverbal negation \forme{maʁ} can be combined with the interrogative form of the assertive copula \forme{ɯ́-ŋu}, as in (\ref{ex:tAtWta.maR.WNu}), to express a rhetorical question concerning something that both the speaker and the addressee are supposed to know.

\begin{exe}
\ex \label{ex:tAtWta.maR.WNu}
\gll `ɯ-kɤrme tɯ-ldʑa a-mɤ-jɤ-tɯ-ɣɯt ra' tɤ-tɯt-a maʁ ɯ́-ŋu \\
\textsc{3sg}.\textsc{poss}-hair one-long-object \textsc{irr}-\textsc{neg}-\textsc{pfv}-2-bring be.needed:\textsc{fact} \textsc{aor}-say[II]-\textsc{1sg} not.be:\textsc{fact} \textsc{qu}-be:\textsc{fact} \\
\glt `Didn't I say: `Do not bring back anything, not even a hair (from her head).'' (2014-kWlAG)
\end{exe}

\subsubsection{Focalization of a constituent} \label{sec:focalization.final.copula}
The most common focalization construction in Japhug is not a pseudo-cleft construction (§\ref{sec:pseudo.cleft}) or a focus marker (§\ref{sec:focus}), but the combination of an overt noun phrase or pronoun (§\ref{sec:focalization.overt}) with a postverbal copula, as in (\ref{ex:CkAnWru}).\footnote{These sentences are from a story where a child at school is bullied by another pupil, who forces him to bring water for him; the teacher (who guessed that the first child was being bullied) asks (\ref{ex:CtAtWnWrut}) to have him tell the one who forced him to do it, but the bullied child replies (\ref{ex:CtAtWnWrut}), as he fears reprisals.} The focalized constituents are indicated in bold. In this construction, the copula is never adjacent to the focalized constituent, which is also marked by a specific intonation. 

\begin{exe}
\ex  \label{ex:CkAnWru}
\begin{xlist}
\ex \label{ex:CtAtWnWrut}
\gll \textbf{nɤʑo} nɤ-tɯ-ci ɕ-tɤ-tɯ-nɯ-ru-t ɯ́-ŋu? \\
\textsc{2sg} \textsc{2sg}.\textsc{poss}-\textsc{indef}.\textsc{poss}-water \textsc{tral}-\textsc{aor}-\textsc{auto}-bring-\textsc{pst}:\textsc{tr}  \textsc{qu}-be:\textsc{fact} \\
\glt `Was it for yourself that you brought the water?'
\ex \label{ex:CtAnWruta}
\gll \textbf{aʑo} ɕ-tɤ-nɯ-ru-t-a ŋu \\
\textsc{1sg} \textsc{tral}-\textsc{aor}-\textsc{auto}-bring-\textsc{pst}:\textsc{tr}-\textsc{1sg} be:\textsc{fact} \\
\glt `I brought it for myself.'
\end{xlist}
\end{exe}


Given the fact that many periphrastic TAME categories use copulas (§\ref{sec:ipfv.periphrastic.TAME}, §\ref{sec:aor.narrative}, §\ref{sec:pst.ifr.ipfv.periphrastic}, §\ref{sec:proximative.periphrastic}), there are many cases where this focalizing function of the copula is ambiguous (for this reason, most of the examples presented below involve verbs in Aorist and Inferential form, which do not occur with the copulas in periphrastic tenses). 


The postverbal copula construction is used to indicate focus on all core arguments, including intransitive subject (\ref{ex:Wpi.mWpjArAzi}), transitive subject (\ref{ex:pAnmawombAr.kW.Nu}, \ref{ex:aZo.pWnWmtota}, with optional ergative on the overt pronoun, §\ref{sec:absolutive.A}) and object (\ref{ex:tutia.Cti}, \ref{ex:nW.YWnArea}). 

In (§\ref{ex:CtAnWruta}), although the focalized argument is in transitive subject function, it is not focalized as subject (`it is I who brought it') but as beneficiary (`it is for myself that I brought it'), as this referent has both functions, the latter marked by the autive prefix \forme{nɯ-} (§\ref{sec:autoben.proper}).

All three copulas (§\ref{sec:copula.basic}) are found in this construction, including the emphatic affirmative \forme{ɕti} (\ref{ex:Wpi.mWpjArAzi}, \ref{ex:tutia.Cti}) and the negative \forme{maʁ} (\ref{ex:tutia.Cti}, \ref{ex:nW.Wndzxa.kW.pWmaR}).

\begin{exe}
\ex \label{ex:Wpi.mWpjArAzi}
\gll tɕeri, ɯ-pi mɯ-pjɤ-rɤʑi qʰendɤre,  \textbf{ɯ-ɬaʁ} \textbf{nɯ} pjɤ-rɤʑi ɕti qʰe  \\
\textsc{lnk} \textsc{3sg}.\textsc{poss}-elder.sibling \textsc{neg}-\textsc{ifr}.\textsc{ipfv}-stay \textsc{lnk} \textsc{3sg}.\textsc{poss}-aunt \textsc{dem} \textsc{neg}-\textsc{ifr}.\textsc{ipfv}-stay be.\textsc{aff}:\textsc{fact} \textsc{lnk} \\
\glt  `But his elder brother was not there, it was his brother's wife who was there.' (140512 alibaba)
(\japhdoi{0003965\#S57})
\end{exe}

\begin{exe}
\ex \label{ex:pAnmawombAr.kW.Nu}
\gll  \textbf{pɤnmawombɤr} \textbf{kɯ} [...] tɤ́-wɣ-sɯ-χtɯ-a ŋu \\
\textsc{anthr} \textsc{erg} { } \textsc{aor}-\textsc{inv}-\textsc{caus}-sell-\textsc{1sg} be:\textsc{fact} \\
\glt `It is Padma 'Od'bar who sold it to me.' (2012 Norbzang)
(\japhdoi{0003768\#S133})
\end{exe} 

Focalized first or second person core arguments, in addition to being marked by an overt pronoun, are also obligatorily indexed on the verb (§\ref{sec:intr.indexation}, §\ref{sec:tr.indexation}), as shown by \textsc{1sg} and \textsc{2sg} marking in (\ref{ex:aZo.pWnWmtota}) and (\ref{ex:tutia.Cti}) below and in (\ref{ex:CkAnWru}) above.

\begin{exe}
\ex \label{ex:aZo.pWnWmtota}
\gll  kɯki tɕʰeme ki ndɤre aʑɯɣ a-pɯ-ŋu tʂaŋ ma tɕe \textbf{aʑo} pɯ-nɯ-mto-t-a ɕti tɕe \\
\textsc{dem}.\textsc{prox} girl  \textsc{dem}.\textsc{prox} \textsc{lnk} \textsc{1sg}.\textsc{gen} \textsc{irr}-\textsc{ipfv}-be be.fair:\textsc{fact} \textsc{lnk} \textsc{lnk} \textsc{1sg} \textsc{aor}-\textsc{auto}-see-\textsc{pst}:\textsc{tr}-\textsc{1sg} be.\textsc{aff}:\textsc{fact} \textsc{lnk} \\
\glt `This girl, it would be fair if she were mine, as it was I who found her.' (140517 buaishuohua)
(\japhdoi{0004018\#S98})
\end{exe}

\begin{exe}
\ex \label{ex:tutia.Cti}
\gll aʑo \textbf{pɤnmawombɤr} tu-ti-a ɕti ma, \textbf{nɤj} ɲɯ-ta-nɯ-ɤkʰɤzŋga maʁ  \\
\textsc{1sg}  \textsc{anthr} \textsc{ipfv}-say-\textsc{1sg} be.\textsc{aff}:\textsc{fact} \textsc{lnk} \textsc{2sg} \textsc{ipfv}-1\fl{}2-\textsc{appl}-call not.be:\textsc{fact} \\
\glt `I am saying ``Padma 'Od 'bar", it is not you that I am calling.' (2012 Norbzang)
(\japhdoi{0003768\#S139})
\end{exe}

 \begin{exe}
\ex \label{ex:nW.YWnArea}
\gll  kʰɯɣɲɟɯ ri pɣɤtɕɯ ni ɲɯ-ɤnɯɣro-ndʑi tɕe, \textbf{nɯ} ɲɯ-nɤre-a ɕti wo  \\
window \textsc{loc} bird \textsc{du} \textsc{sens}-play-\textsc{du} \textsc{lnk} \textsc{dem} \textsc{ipfv}--\textsc{1sg}  be.\textsc{aff}:\textsc{fact} \textsc{sfp} \\
\glt `On the window two birds were playing, this is what I was laughing about.' (2014-kWlAG)
\end{exe}

This construction is also attested to focalize adjuncts, such as the causal phrase marked with the ergative (§\ref{sec:manner.nominal.kW}) in (\ref{ex:nW.Wndzxa.kW.pWmaR}).

\begin{exe}
\ex \label{ex:nW.Wndzxa.kW.pWmaR}
\gll  \textbf{nɯ} \textbf{ɯ-ndʐa} \textbf{kɯ} nɯ-si ʁo pɯ-maʁ. \\
\textsc{dem} \textsc{3sg}.\textsc{poss}-reason \textsc{erg} \textsc{aor}-die \textsc{advers} \textsc{pst}.\textsc{ipfv}-not.be \\
\glt `It was not because of this that shed died.' (150907 srWn)
(\japhdoi{0006360\#S22})
\end{exe}


The postverbal copula can be nominalized, and the whole sentence turned into a non-finite clause. This type of construction is also used for constituent focalization  as in (\ref{ex:qapri.bWxsi.pjAsat}). The form \forme{kɯ-ŋu} is ambiguous between a stative infinitive and a subject participle (§\ref{sec:infinitives.participles}), and therefore the non-finite clause is either analyzable as an infinitival complement clause (§\ref{sec:inf.complementation}) or a participial clause in semi-object function (§\ref{sec:relative.core.arg}).

\begin{exe}
\ex \label{ex:qapri.bWxsi.pjAsat}
\gll [\textbf{xɕiri} \textbf{nɯ} \textbf{kɯ} qapri bɯxsi pjɤ-sat kɯ-ŋu] nɯnɯ ko-tso \\
weasel \textsc{dem} \textsc{erg} snake python \textsc{ifr}-kill \textsc{sbj}:\textsc{pcp}-be \textsc{dem} \textsc{ifr}-understand \\
\glt `He realized that it was the weasel that had killed the python.' (140518 xuezhe he huangshulang-zh)
(\japhdoi{0004032\#S34})
\end{exe}

\subsection{Postverbal negative existential verb} \label{sec:negation.existential}
Like the negative copula \forme{maʁ} (§\ref{sec:affirmative.copula.function}), the negative existential verbs \forme{me} and \forme{maŋe}  occur postverbally as periphrastic negative construction (§\ref{sec:periphrastic.negation}).

With transitive verbs, this construction can express universal negative object (`nothing'), in particular with the adverbial intensifier \forme{maka} (§\ref{sec:intensifier.adverbs}), which can be located either before the main verb (\ref{ex:maka.V.me}) or directly before the negative existential verb  (\ref{ex:V.maka.me}). 

\begin{exe}
\ex 
\begin{xlist}
\ex \label{ex:maka.V.me}
\gll maka ʑo pɯ-mto-t-a me \\
at.all \textsc{emph} \textsc{aor}-see-\textsc{pst}:\textsc{tr}-\textsc{1sg} not.exist:\textsc{fact} \\
\ex \label{ex:V.maka.me}
\gll pɯ-mto-t-a maka me \\
\textsc{emph} \textsc{aor}-see-\textsc{pst}:\textsc{tr}-\textsc{1sg} at.all  not.exist:\textsc{fact} \\
\glt `I did not see anything/any of it at all.' (\japhdoi{0006336\#S58})
\end{xlist}
\end{exe}

With secundative verbs (§\ref{sec:ditransitive.secundative}), negative meaning can apply either to the object or to the theme (semi-object). In (\ref{ex:rNWl.stea}), the object (\japhug{rŋɯl}{silver}) is definite, while the non-overt theme (the manner) is universal negative.


\begin{exe}
\ex \label{ex:rNWl.stea}
\gll aʑo rŋgɯ ɕti-a qhe, rŋɯl ste-a me qʰe, nɤʑo nɤ-ŋgra a-pɯ-ŋu tɕe tɤ-nɯ-ndɤm \\
\textsc{1sg} boulder be.\textsc{aff}:\textsc{fact}-\textsc{1sg} \textsc{lnk} silver do.like:\textsc{fact}-\textsc{1sg} not.exist:\textsc{fact} \textsc{lnk} \textsc{2sg} \textsc{2sg}.\textsc{poss}-salary \textsc{irr}-\textsc{pfv}-be \textsc{lnk} \textsc{imp}-\textsc{auto}-take[III] \\
\glt `I am a boulder, I have no use of [this] silver, may this be your reward (for helping me), take it.' (divination)
(\japhdoi{0003364\#S98})
\end{exe}

The postverbal negative construction can also indicate a non-specific, but not completely indefinite entity (`not any $X$, none of the $X$' rather than `nothing'). In (\ref{ex:pWmtota.me.nonspec}), the zero object anaphorically refers to \japhug{qɯmdroŋ}{crane} (§\ref{sec:nonovert.core.arguments}), the main topic of the story from which this sentence is taken.

\begin{exe}
\ex \label{ex:pWmtota.me.nonspec}
\gll tɕe jinde aj pɯ-mto-t-a me ri, pɯ-kɯ-xtɕi tɕe pɯ́-wɣ-mto \\
\textsc{lnk} nowadays \textsc{1sg} \textsc{aor}-see-\textsc{pst}:\textsc{tr}-\textsc{1sg} \textsc{lnk} not.exist:\textsc{fact} \textsc{aor}-\textsc{genr}:S/O-be.small \textsc{lnk} \textsc{aor}-\textsc{inv}-see \\
\glt `These days I have not seen any [crane], but I did see [some] when I was young.' (22-qomndroN)
(\japhdoi{0003598\#S31})
\end{exe}

The scope of the negation is however not necessarily restricted to the object. In (\ref{ex:jinW.maNe}) for instance, the non-overt object anaphorically refers to \japhug{jima}{maize} mentioned in the previous sentences, but the scope of negation is rather on the essive participial clause \forme{ʑara nɯ-kɤ-nɯ-ndza} `as food for themselves/for themselves to eat' (§\ref{sec:essive.abs}): the villagers do  plant maize, but for a different purpose.
 
\begin{exe}
\ex \label{ex:jinW.maNe}
\gll tɕe ʑara nɯ-kɤ-nɯ-ndza tɕe koŋla ji-nɯ maŋe woma \\
\textsc{lnk} \textsc{3pl} \textsc{2pl}.\textsc{poss}-\textsc{obj}:\textsc{pcp}-\textsc{auto}-eat \textsc{lnk} completely plant:\textsc{fact}-\textsc{pl} not.exist:\textsc{sens} \textsc{sfp} \\
\glt `They don't  plant any [maize] for themselves to  eat, (they plant it to feed their pigs).'  (140522 kAmYW tWji)
(\japhdoi{0004055\#S64})
\end{exe}

With intransitive verbs, this construction is used to indicate the complete absence of the action `not ... at all' (\ref{ex:WNgW.ri.karia.me}).

\begin{exe}
\ex \label{ex:WNgW.ri.karia.me} 
\gll  tɯ-ci ɯ-rkɯ zɯ ku-rɤʑi-a tɕe nɯre ri ku-nɯ-tsʰi-a ɕti ma ɯ-ŋgɯ ri kɤ-ari-a me, ɕ-tu-nɤŋkɯŋke-a ri me \\
\textsc{indef}.\textsc{poss}-water \textsc{3sg}.\textsc{poss}-side \textsc{loc} \textsc{ipfv}-stay-\textsc{1sg} \textsc{lnk} \textsc{dem}:\textsc{loc} \textsc{loc} \textsc{ipfv}-\textsc{auto}-drink-\textsc{1sg} be.\textsc{aff}:\textsc{fact} \textsc{lnk} \textsc{3sg}.\textsc{poss}-in also \textsc{aor}:\textsc{east}-go[II]-\textsc{1sg} not.exist:\textsc{fact} \textsc{tral}-\textsc{ipfv}-\textsc{distr}:walk-\textsc{1sg} also not.exist:\textsc{fact} \\
\glt `I am staying near the water and drinking (water) there, I did not go into the water at all, I am not walking around in it.' (aesop lang he yang 16-17)
\end{exe}

The intransitive stative verb \japhug{sna}{be good, be worthy} (§\ref{sec:mna.sna}) is commonly used with negative existential verbs, with the meaning `not good for anything', as in (\ref{ex:sna.me}). The scope of universal negation in this case is on the essive adjunct (for instance \forme{nɯŋa ɯ-ndza} `cow fodder' in the first clause of \ref{ex:sna.me}).

\begin{exe}
\ex \label{ex:sna.me} 
\gll nɯŋa ɯ-ndza sna ma nɯ ma sna me \\
cow \textsc{3sg}.\textsc{poss}-food be.good:\textsc{fact} \textsc{lnk} \textsc{dem} apart.from be.good:\textsc{fact} not.exist:\textsc{fact} \\
\glt `[Oat] is good as fodder for cattle, but apart from that it is not good for anything.' (08-qaJAGi)
(\japhdoi{0003458\#S8})
\end{exe}

The clause preceding the negative existential verb can be embedded within a participial clause with the existential verb \forme{kɯ-tu}, as in (\ref{ex:CtAnWtWta.kWtu.me}). A similar use of \forme{kɯ-tu} is also found in the possessive construction (example \ref{ex:Wmdzu.kWtu.me}, §\ref{sec:possessive.constructions}).

\begin{exe}
\ex \label{ex:CtAnWtWta.kWtu.me} 
\gll [[aʑo joβ tɕi ɕ-tɤ-nɯ-tɯt-a] kɯ-tu] me qʰe  [aʑo ndɤre ɲɯ-ɤβzu-a ri kɯ-ra] me qʰe, nɤ-rcɯ\redp{}rca ʑo ɣi-a ŋu ma nɯ ma [[aʑɯɣ kɯ-ra] kɯ-tu] me \\
\textsc{1sg} \textsc{interj} also \textsc{tral}-\textsc{aor}-\textsc{auto}-say[II]-\textsc{1sg} \textsc{sbj}:\textsc{pcp}-exist not.exist:\textsc{fact} \textsc{lnk} \textsc{1sg} \textsc{advers} \textsc{ipfv}-become-\textsc{1sg} also \textsc{sbj}:\textsc{pcp}-be.needed not.exist:\textsc{fact} \textsc{lnk} \textsc{2sg}.\textsc{poss}-\textsc{emph}\redp{}together.with \textsc{emph} come:\textsc{fact}-\textsc{1sg} be:\textsc{fact} \textsc{lnk} \textsc{dem} apart.from \textsc{1sg}:\textsc{gen} \textsc{sbj}:\textsc{pcp}-be.needed \textsc{sbj}:\textsc{pcp}-exist not.exist:\textsc{fact} \\
\glt `I did not go and heed [the girl who was calling us], and I don't need to become anything special, I don't need anything, apart from following you.' (2003 kandZislama)
\end{exe}

The postverbal negative existential verb can be combined with a verb taking the negative prefix, forming a double negation (§\ref{sec:double.negation}) expressing universal quantification (\ref{ex:mAnAmandZi.me}).

\begin{exe}
\ex \label{ex:mAnAmandZi.me}
\gll kɯ-ŋɤn mɤ-nɤma-ndʑi maka me\\
\textsc{sbj}:\textsc{pcp}-be.evil \textsc{neg}-do:\textsc{fact}-\textsc{du} at.all not.exist:\textsc{fact}   \\
\glt `There is no evil thing that they do not do/They do all kinds of evil things.' (140428 yonggan de xiaocaifeng-zh)
(\japhdoi{0003886\#S162})
 \end{exe}

In addition, postverbal negative existential verbs are used in one of the superlative constructions (§\ref{sec:negative.existential.superlative}).
(§\ref{sec:bare.inf.negative})

\subsection{Verb doubling} \label{sec:verb.doubling}
Copulas are exceptional in being the only verbs that can be repeated to express emphasis. This usage is only found in the traditional story register, as in (\ref{ex:tWNu.tWNu}). This construction differs from partial reduplication, which only targets one syllable (§\ref{sec:partial.redp}).

\begin{exe}
\ex \label{ex:tWNu.tWNu}
\gll nɯ kɯ-fse ci tɯ-ŋu, nɯ mɤ-kɯ-naχtɕɯɣ ci \textbf{tɯ-ŋu} \textbf{tɯ-ŋu} \\
\textsc{dem} \textsc{sbj}:\textsc{pcp}-be.like \textsc{indef} 2-be:\textsc{fact} \textsc{dem} \textsc{neg}-\textsc{sbj}:\textsc{pcp}-be.the.same \textsc{indef} 2-be:\textsc{fact} 2-be:\textsc{fact} \\
\glt `You are (someone) like that, you are different (from normal people).' (2011-04-smanmi)
\end{exe}

%"wo nɯ kɯ-fse ci ŋu ŋu ma,


\subsection{Other constructions}
Apart from the constructions discussed above, existential verbs are attested in an unusual type of alternative concessive conditional (§\ref{sec:alt.concessive.conditional}), combined with the bare stem of the verb (§\ref{sec:bare.inf.negative}), and in periphrastic modal constructions (§\ref{sec:cimame.cinAmaRkW}).
 
%other ndza mɤ-ndza me-a
 Existential verbs can also occur with the adverb \japhug{jamar}{about} to build an equative construction `as big as $X$' (§\ref{sec:sthWci.equative}) as in (\ref{ex:ajaR.ki.jamar.GAZu}).  

\begin{exe}
\ex \label{ex:ajaR.ki.jamar.GAZu}
\gll ki aʑo a-jaʁ ki jamar ɣɤʑu. \\
\textsc{dem}.\textsc{prox} \textsc{1sg} \textsc{1sg}.\textsc{poss}-hand \textsc{dem}.\textsc{prox} about exist:\textsc{sens} \\
\glt `The wild yak horn is about as big as my hand here.' (20-RmbroN)
(\japhdoi{0003560\#S26})
 \end{exe}


