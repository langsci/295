\chapter{Consonant clusters and partial reduplication} \label{sec:clusters.redp}


\section{Partial Reduplication}  \label{sec:partial.redp}
A useful test to analyse and classify clusters is partial reduplication \citep{jacques07redupl}, a very productive process which can be applied to both verb and noun stems and has a variety of morphosyntactic functions depending on the position of the syllable affected by reduplication. 

Partial reduplication of the first syllable of the verbal complex can mark the protasis of a conditional construction (§\ref{sec:redp.protasis}), iterative coincidence (§\ref{sec:iterative.coincidence}), degree incrementation (§\ref{sec:redp.gradual.increase}) and totalitative relativization (§\ref{sec:totalitative.redp}).

When the target of partial reduplication is the last syllable of the verb, it can indicate emphasis (§\ref{sec:emphatic.autive}, §\ref{sec:emph.redp}), and also serve as a secondary exponent (§\ref{sec:verb.stem.redp}) of several derivational processes, such as reciprocal (§\ref{sec:redp.reciprocal}), distributed action (§\ref{sec:distributed.action}), auto-evaluative (§\ref{sec:autoevaluative}), attenuative (§\ref{sec:attenuative}), as well as some non-finite verb forms (gerund §\ref{sec:gerund} and purposive converb §\ref{sec:purposive.converb}). Outside of verbal morphology, we find partial reduplication in comitative adverbs (§\ref{sec:comitative.adverb}), perlative (§\ref{sec:perlative}) and collectives (§\ref{sec:redp.coll}).  There are in addition a few sporadic cases of full reduplication (§\ref{sec:redp.voice}, §\ref{sec:antipassive.redp}).  

When partial reduplication is applied to a syllable, the rhyme of the reduplicated syllable is  changed to \ipa{ɯ} in the reduplicant (other types of partial reduplication are treated in §\ref{sec:irregular.reduplication}). This vowel can undergo rounded assimilation to \phonet{u} when the base syllable has the main vowel \ipa{u} or \ipa{o} (§\ref{sec:vowel.harmony}).

Some clusters are only partially copied: when the last consonant  of a cluster is one of the non-nasal sonorants (\ipa{r}, \ipa{l}, \ipa{j}, \ipa{w}, \ipa{ɣ} or \ipa{ʁ}), and the preceding consonant is neither a non-nasal sonorant nor a sibilant or alveolo-palatal fricative, the sonorant is deleted, as in the gerund of \japhug{mbɣom}{cut}, which yields \forme{sɤ-\textbf{mbɯ}\redp{}mbɣom} (example \ref{ex:sAmbWmbGom.YABdenW}, §\ref{sec:gerund.clauses}) instead of $\dagger$\forme{sɤ-mbɣɯ\redp{}mbɣom}, or the emphatic form  \forme{\textbf{kʰɯ}\redp{}kʰro} from \japhug{kʰro}{much} (§\ref{sec:khro}), instead of $\dagger$\forme{kʰrɯ\redp{}kʰro}.


However, in \textit{some} of the clusters ending in a non-nasal sonorant, but whose penultimate consonant is either a non-nasal sonorant, a sibilant (\ipa{s}, \ipa{z}) or an alveolo-palatal fricative (\ipa{ɕ}, \ipa{ʑ}), the final sonorant is not deleted in partial reduplication. For instance, the emphatic reduplication of the infinitive of \japhug{aɕɤrɣi}{be quick} is \forme{kɯ-ɤɕɤrɣɯ\redp{}rɣi} (\ref{ex:kACArGWrGi}), not $\dagger$\forme{kɯ-ɤɕɤ\textbf{rɯ}\redp{}rɣi}, and the perlative (§\ref{sec:perlative}) of \japhug{tɤ-jroʁ}{trace} is  \forme{ɯ-\textbf{jrɯ}\redp{}jroʁ} `following X's trace', not $\dagger$\forme{ɯ-jɯ\redp{}jroʁ}.

\begin{exe}
	\ex \label{ex:kACArGWrGi}
	\gll kɯ-ɤɕɤrɣɯ\redp{}rɣi ʑo, tu-sɯ-mtɕɯr ŋgrɤl \\
	\textsc{inf}:\textsc{stat}-\textsc{emph}\redp{}be.quick \textsc{emph} \textsc{ipfv}-\textsc{caus}-turn be.usually.the.case:\textsc{fact} \\
	\glt `It makes it turn very quickly.' (18-NGolo, 140)
\end{exe}

However, in the cluster \forme{rj-} for instance, the glide \forme{j} is deleted, as in the negative gerund \japhug{masɤrɯrju}{quietly, in secret} from \japhug{arju}{speak} (§\ref{sec:gerund.neg}). A precise inventory of the clusters with sonorant deletion is provided in  §\ref{sec:medials}  and §\ref{sec:summary.clusters}.


Partial reduplication is crucial in analyzing and classifying consonant clusters. Sonorants that undergo deletion when partial reduplication is applied are henceforth designated as \textit{medial} consonants (corresponding to the terms \ch{介音}{ jièyīn}{medial} in the Chinese phonological tradition and \tibet{འདོགས་ཅན་}{ⁿdogs.tɕan}{attached letter} in the Tibetan one) and do not belong to the same phonological constituent as the rest of the onset.  


\section{Inventory of consonant clusters} \label{sec:inventory.clusters}
Japhug counts at least 423 clusters in syllable onset position:  319 clusters with two consonants and 104 with three consonants.\footnote{A few new clusters have been discovered since the article \citet{jacques19ipa} was written. } In addition, there are additional types of consonant clusters across syllable boundaries (§\ref{sec:heterosyllabic.clusters}) and secondary clusters due to vowel fusion (§\ref{sec:synizesis}), which are too numerous to be systematically surveyed. While the number of clusters in Japhug is considerably smaller than that of Khroskyabs, which counts as many as 757 \citep[101]{lai17khroskyabs}, it is still much richer than most languages in the Trans-Himalayan family. 

In the following discussion, phonemes or clusters found exclusively in ideophones (§\ref{sec:idph}) or Tibetan loanwords are systematically indicated, in order to bring out the phonotactics of inherited Japhug vocabulary. For each cluster, an example is provided; in the case of verb roots, the base stem is indicated, rather than a conjugated form (unlike what was done in  \citealt{jacques19ipa}).

%\&[ \t]*\\ipa\{([^}]*)\} *\&[\t ]*([^\&]*)
%\& \\japhug{\1}{\2} 
%regex for converting tables

\newcounter{2clusters}
\newcounter{3clusters}

\subsection{Preinitials} \label{sec:preinitials}
This section deals with consonant clusters whose last consonant is not one of the six non-nasal sonorants (\ipa{r}, \ipa{l}, \ipa{j}, \ipa{w}, \ipa{ɣ} or \ipa{ʁ}) and thus cannot contain a medial consonant. 

Clusters of this type have a limited number of possible consonants in first position: \ipa{w}, \ipa{s}-\ipa{z}, \ipa{ɕ}-\ipa{ʑ}, \ipa{l}, \ipa{ʂ}-\ipa{r}, \ipa{j}, \ipa{x}-\ipa{ɣ}, \ipa{χ}-\ipa{ʁ}, \ipa{n}, \ipa{m} and the homorganic nasal, to which must be added a few clusters in stop+\ipa{ɕ}. These consonants, which correspond to both \tibet{སྔོན་འཇུག་}{sŋon.ⁿdʑug}{prefixed letter} and \tibet{མགོ་ཅན་}{mgo.tɕan}{superscript letter} in Tibetan traditional grammar, are referred to as \textit{preinitial}. The last consonant of these clusters is called the \textit{initial} (\tibet{མིང་གཞི་}{miŋ.gʑi}{radical letter} in Tibetan). The following sections list clusters by preinitial, and in each table present them by place and mode of articulation of the initial. Clusters with three consonants are dealt with separately.

It is striking that (except in the stop+\forme{ɕ-} clusters, which are secondary, §\ref{sec:kC.pC.clusters}), all preinitials are either sonorants or fricatives. There are no stop preinitials in Japhug, unlike in Situ, which allows labial and velar stops \citep[44]{zhang16bragdbar}. The preinitial stops  \forme{p-} and \forme{k-} in Situ regularly correspond to Japhug \ipa{w-} and \ipa{x/ɣ-} or \ipa{χ/ʁ\trt}, suggesting that a general rule of preinitial fricativization took place in Japhug \citep[273]{jacques04these}.


\subsubsection{\ipa{w}+C clusters}  \label{sec:wC.clusters}
\tabref{prein.w} lists all consonant clusters in Japhug with \ipa{w} as first element not ending in a non-nasal sonorant.

\begin{table}
	\caption{List of consonant clusters with \ipa{w} as a first element (15+8); clusters only found in Tibetan loanwords are shaded in light grey. }  \centering \label{prein.w}
	\begin{tabular}{Xlll}
		\lsptoprule
		%\ipa{p}  & 	  & 	  & 	  & 	 \\
		%\ipa{pʰ}  & 	  & 	  & 	  & 	 \\
		%\ipa{b}  & 	  & 	  & 	  & 	 \\
		%\ipa{mb}  & 	  & 	  & 	  & 	 \\
		%\ipa{m}  & 	  & 	  & 	  & 	 \\
		\ipa{t}  & 	 \deux{wt}  &  	 \japhug{ɯ-ftaʁ}{sign} \\
		%\ipa{tʰ}  & 	  & 	  & 	  & 	 \\
		\ipa{d}  & 	 \deux{wd}  & 	\japhug{βdɯt}{demon} \\
		%%\ipa{nd}  & 	  & 	  & 	  & 	 \\
		%%\ipa{n}  & 	  & 	  & 	  & 	 \\
		\ipa{ts}  & 	 \deux{wts}  & 	\japhug{ftsoʁ}{female hybrid yak} \\
		\ipa{tsʰ}  & 	 \deux{wtsʰ}  & 	\japhug{ftsʰi}{feel better} \\
		%%\ipa{dz}  & 	  & 	  & 	  & 	 \\
		%%\ipa{ndz}  & 	  & 	  & 	  & 	 \\
		\ipa{s}  & 	 \deux{ws}  & 	\japhug{fsaŋ}{fumigation} \\
		\ipa{z}  & 	 \deux{wz} \tib{}  & 	\japhug{βzaŋsa}{friend} \\
		%\ipa{ɬ}  & 	  & 	  & 	  & 	 \\
		\ipa{tɕ}  & 	 \deux{wtɕ}  & 	\japhug{ftɕar}{summer} \\
		\ipa{tɕʰ}  & 	 \deux{wtɕʰ}  & 	\japhug{ftɕʰur}{put vertically}, `pour down' \\
		%\ipa{dʑ}  & 	  & 	  & 	  & 	 \\
		%\ipa{ndʑ}  & 	  & 	  & 	  & 	 \\
		\ipa{ɕ}  & 	 \deux{wɕ}  & 	\japhug{fɕaʁ}{repent} \\
		\ipa{ʑ}  & 	 \deux{wʑ}  & 	\japhug{βʑar}{buzzard} \\
		\ipa{tʂ}  & 	 \deux{wtʂ}  & 	\japhug{ftʂi}{melt} \\
		%\ipa{tʂʰ}  & 	  & 	  & 	  & 	 \\
		%\ipa{dʐ}  & 	  & 	  & 	  & 	 \\
		%\ipa{ndʐ}  & 	  & 	  & 	  & 	 \\
		%\ipa{ʂ}  & 	  & 	  & 	  & 	 \\
		\ipa{c}  & 	 \deux{wc}  & 	\japhug{tɯ-fcaʁ}{dorsal mat} \\
		%\ipa{cʰ}  & 	  & 	  & 	  & 	 \\
		\ipa{ɟ}  & 	 \deux{wɟ}  & 	\japhug{βɟi}{chase, catch up with} \\
		%\ipa{ɲɟ}  & 	  & 	  & 	  & 	 \\
		%\ipa{ɲ}  & 	  & 	  & 	  & 	 \\
		\ipa{k}  & 	 \deux{wk}  & 	\japhug{fka}{be full} \\
		%\ipa{kʰ}  & 	  & 	  & 	  & 	 \\
		\ipa{g}  & 	 \deux{wg} \tib{} & 	\japhug{βgoz}{prepare} \\
		\midrule
		&	\trois{wxt}  &	\japhug{wxti}{it is big} \\
		&	\trois{wst} \tib{} &	\japhug{fstɯn}{serve}, `treat' \\
		&	\trois{wrt}  \tib{} &	\japhug{frtɤn}{be trustworthy} \\
		&	\trois{wsk}  \tib{} &	\japhug{fskɤr}{go around} \\
		&	\trois{wzg}  \tib{} &	\japhug{βzgɤr}{delay} \\
		&	\trois{wzd}  \tib{} &	\japhug{βzdɯ}{collect} \\
		&	\trois{wzɟ}  \tib{} &	\japhug{βzɟɯr}{transform} \\
		&	\trois{wrɟ}  \tib{} &	\japhug{βrɟaŋ}{stretch tight} (skin) \\					
		\lspbottomrule
	\end{tabular} 
\end{table}
\resetcounters{2wC}{3wC}

The phoneme  \ipa{w}  has fricativized allophones when occurring as first member of a consonant cluster (§\ref{sec:consonant.phonemes}), transcribed here as \forme{f-} before unvoiced obstruents, and \forme{β-} before all voiced consonants.   It does not appear before nasal or prenasalized segments, due to a nasalization rule \forme{*wN-} \fl{} \forme{mN\trt}, attested notably in the irregular \forme{m-} allomorph of the \forme{ɣɤ-} causative (§\ref{sec:causative.m}). It cannot be followed by any of the labial consonants: for instance $\dagger$\forme{fp-} and $\dagger$\forme{βb-} are not acceptable onsets.  Some clusters with  \ipa{w}  + voiced obstruents (\forme{βz-} and \forme{βg-}) are only attested in Tibetan loanwords. 

Clusters with three consonants whose first element is  \ipa{w}  and the last one is not a sonorant are all restricted to Tibetan borrowings (verb forms with the past tense \forme{b-} prefix, for example \forme{βzɟɯr} from \tibet{བསྒྱུར་}{bsgʲur}{change}) except for \ipa{wxt}, which is realized as \phonet{xʷt} with a labiovelarized fricative, rounding the preceding unrounded vowels \ipa{ɯ} and \ipa{ɤ}: the participle \forme{kɯ-wxti} `the big one' is realized as \phonet{kuxʷti}. Not all speakers maintain the contrast between \ipa{wxt} and \ipa{xt}; the verb \japhug{wxti}{be big} is the only item with this cluster.

Non-segmental realizations of \ipa{w}, limited to vowel rounding, are also found in the group \forme{wɣr-} (treated in §\ref{sec:Cr.clusters}) and the inverse prefix \forme{wɣ-} (§\ref{sec:allomorphy.inv}).

\subsubsection{Sibilant+C clusters}  \label{sec:sC.clusters}
\tabref{prein.sz} lists all consonant clusters with a sibilant fricative \ipa{s} or \ipa{z} as first element not ending in a non-nasal sonorant.  


\begin{table}
	\caption{List of consonant clusters with \ipa{s} or \ipa{z} as a first element (23+0)}  \centering \label{prein.sz}
	\begin{tabular}{Xlll}
		\lsptoprule
		\ipa{p}  & 	 \deux{sp}  & \japhug{spoz}{incense} 	  \\
		%\ipa{pʰ}  & 	  & 	  & 	  & 	  & 	  & 	  \\
		\ipa{b}  & 	 	 \deux{zb}  & \japhug{zbaʁ}{be dry} \\
		\ipa{mb}  & 	 \deux{zmb}  & \japhug{tɤzmbɯr}{silt}  \\
		\ipa{m}  & 	 \deux{sm}  & \japhug{smar}{river} \\
		& 	 \deux{zm}  & \japhug{zmaqʰu}{cause to be late} \\
		\ipa{t}  & 	 \deux{st}  & \japhug{staχpɯ}{pea} 	  \\
		\ipa{tʰ}  & 	 \deux{stʰ}  & \japhug{stʰaβ}{touch} 	  \\
		\ipa{d}  & 	   \deux{zd}  & \japhug{zdɯm}{cloud} \\
		\ipa{nd}  & 	 	 \deux{znd}  & \japhug{znde}{wall} \\
		\ipa{n}  & 	 \deux{sn}  & \japhug{sna}{be good} \\
		&	 \deux{zn}  & \japhug{znɤja}{find $X$ a shame} \\
		%\ipa{ts}  & 	  & 	  & 	  & 	  & 	  & 	  \\
		%\ipa{tsʰ}  & 	  & 	  & 	  & 	  & 	  & 	  \\
		%\ipa{dz}  & 	  & 	  & 	  & 	  & 	  & 	  \\
		%\ipa{ndz}  & 	  & 	  & 	  & 	  & 	  & 	  \\
		%\ipa{s}  & 	  & 	  & 	  & 	  & 	  & 	  \\
		%\ipa{z}  & 	  & 	  & 	  & 	  & 	  & 	  \\
		%\ipa{ɬ}  & 	  & 	  & 	  & 	  & 	  & 	  \\
		%\ipa{tɕ}  & 	  & 	  & 	  & 	  & 	  & 	  \\
		%\ipa{tɕʰ}  & 	  & 	  & 	  & 	  & 	  & 	  \\
		%\ipa{dʑ}  & 	  & 	  & 	  & 	  & 	  & 	  \\
		%\ipa{ndʑ}  & 	  & 	  & 	  & 	  & 	  & 	  \\
		%\ipa{ɕ}  & 	  & 	  & 	  & 	  & 	  & 	  \\
		%\ipa{ʑ}  & 	  & 	  & 	  & 	  & 	  & 	  \\
		%\ipa{tʂ}  & 	  & 	  & 	  & 	  & 	  & 	  \\
		%\ipa{tʂʰ}  & 	  & 	  & 	  & 	  & 	  & 	  \\
		%\ipa{dʐ}  & 	  & 	  & 	  & 	  & 	  & 	  \\
		%\ipa{ndʐ}  & 	  & 	  & 	  & 	  & 	  & 	  \\
		%\ipa{ʂ}  & 	  & 	  & 	  & 	  & 	  & 	  \\
		\ipa{c}  & 	 \deux{sc}  & \japhug{scoʁ}{scoop} 	  \\
		\ipa{cʰ}  & 	 \deux{scʰ}  & \japhug{scʰɤt}{recede} (water)	  \\
		\ipa{ɟ}  & 	   	 \deux{zɟ}  & \japhug{nɯzɟɯ}{suffer losses} \\
		\ipa{ɲɟ}  & 		 \deux{zɲɟ}  & \japhug{zɲɟa}{plant sp.} \\
		\ipa{ɲ}  & 	 \deux{sɲ}  & \japhug{sɲaŋne}{fasting} 	  \\
		%\midrule  \\
		\ipa{k}  & 	 \deux{sk}  & \japhug{skɤm}{ox}  	  \\
		\ipa{kʰ}  & 	 \deux{skʰ}  & \japhug{rɟɤskʰi}{pan}   \\
		\ipa{g}  & 		 \deux{zg}  & \japhug{zga}{sauce} \\
		\ipa{ŋg}  & 		  	 \deux{zŋg}  & \japhug{akʰɤzŋga}{shout, call} \\
		\ipa{ŋ}  & 	 \deux{sŋ}  & \japhug{sŋaʁ}{curse} 	  \\
		%\ipa{x}  & 	  & 	  & 	  & 	  & 	  & 	  \\
		\ipa{q}  & 	 \deux{sq}  & \japhug{sqamnɯz}{twelve} 	  \\
		\ipa{qʰ}  & 	 \deux{sqʰ}  & \japhug{sqʰi}{tripod} 	  \\
		%\ipa{ɴɢ}  & 	  & 	  & 	  & 	  & 	  & 	  \\
		%\ipa{χ}  & 	  & 	  & 	  & 	  & 	  & 	  \\
		\lspbottomrule
	\end{tabular} 
\end{table}
\resetcounters{2szC}{3szC}

The contrast between \ipa{s} and \ipa{z} is neutralized before obstruents: we find \forme{s-} before unvoiced stops, fricatives and affricates and \forme{z-} before voiced ones (including prenasalized stops). With nasals, a contrast is found between \forme{sm-} and \forme{sn-} on the one hand, and their voiced counterparts  \forme{zm-} and \forme{zn-} on the other hand. The latter are only found in morphologically complex nouns or verbs, as several sibilant prefixes. These include the causative (§\ref{sec:caus.z}), the oblique participle (§\ref{sec:oblique.participle.allomorphy}), the gerund (§\ref{sec:gerund.allomorphs}) and the abilitative (§\ref{sec:abilitative}) prefixes, all of which have \forme{z-} allomorphs when prefixed to polysyllabic stems whose first syllable has a single sonorant followed by a vowel. In particular, \japhug{zmaqʰu}{cause to be late}  and \japhug{znɤja}{find $X$ a shame} in \tabref{prein.sz} are the causative derivations of \japhug{maqʰu}{be after}, `be late' (§\ref{sec:denom.mA}) and \japhug{nɤja}{be a shame} (§\ref{sec:sig.caus.tropative}), respectively.

Before non-nasal sonorants, a voicing contrast is attested between \ipa{sj-} vs. \ipa{zj-} (§\ref{sec:Cj.clusters}), \ipa{sl-} vs. \ipa{zl-} (§\ref{sec:Cl.clusters}), \ipa{sr-} vs. \ipa{zr-} (§\ref{sec:Cr.clusters}) and \ipa{sɣ-} vs. \ipa{zɣ-} (§\ref{sec:CG.clusters}).

The sibilant preinitials are compatible with all stops, but do not occur with affricates, whether dental, alveolo-palatal or retroflex, except in heterosyllabic clusters (§\ref{sec:heterosyllabic.clusters}). The loanword \japhug{koxtɕɯn}{brocade} from \tibet{གོས་ཆེན་}{gos.tɕʰen}{brocade} suggests that a sound change \forme{*stɕ-} \fl{} \forme{xtɕ-} has removed all instances of sibilant+al\-veolo-palatal clusters.

The aspiration contrast in unvoiced stops and affricates is not neutralized after the \forme{s-} preinitial, as shown, for instance, by the minimal pair  \japhug{stʰoʁ}{push}, `press' vs. \japhug{stoʁ}{broad bean}.

\subsubsection{\ipa{l}+C clusters}  \label{sec:lC.clusters}


There are considerably fewer clusters with an \ipa{l} preinitial
than with other preinitials. These clusters are listed in \tabref{prein.l}. They contain a high proportion of ideophones and Tibetan loanwords, as the genuine proto-Gyalrong \forme{*l-} preinitial has changed to \forme{j-} in most contexts (§\ref{sec:jC.clusters}).

\begin{table}
	\caption{List of consonant clusters with \ipa{l}  as a first element (17+1); light grey shading indicates clusters only attested in Tibetan loanwords, and dark grey clusters only found in ideophones.} \label{prein.l}  \centering
	\begin{tabular}{Xlll}
		\lsptoprule
		\ipa{p}   & 	 	 \deux{lp}   & \japhug{tɯ-lpɤɣ}{one piece}  \\ 
		%\ipa{pʰ}   & 	 	   & 	 	   & 	 	   \\ 
		%\ipa{b}   & 	 	   & 	 	   & 	 	   \\ 
		%\ipa{mb}   & 	 	   & 	 	   & 	 	   \\ 
		\ipa{m}   & 	 	 \deux{lm}   & \japhug{tɤlmɯz}{straw covering the balcony}  \\ 
		\ipa{t}   & 	 	 \deux{lt}   & \japhug{ltɤβ}{fold}  \\ 
		\ipa{tʰ}   & 	 	 \deux{ltʰ} \idph{}   & \japhug{ltʰɯmɯmi}{coming slowly (sleep)}  \\ 
		\ipa{d}   & 	 	 \deux{ld}   & \japhug{ldɯɣi}{bharal}  \\ 
		%\ipa{nd}   & 	 	   & 	 	   & 	 	   \\ 
		\ipa{n}   & 	 	 \deux{ln}   & \japhug{lni}{wither}  \\ 
		\ipa{ts}   & 	 	 \deux{lts}   & \japhug{ɕɤltsaʁ}{leather coat}  \\ 
		\ipa{tsʰ}   & 	 	 \deux{ltsʰ} \idph{}   & \japhug{ltsʰɤltsʰɤt}{small and weak}  \\ 
		%\ipa{dz}   & 	 	   & 	 	   & 	 	   \\ 
		%\ipa{ndz}   & 	 	   & 	 	   & 	 	   \\ 
		%\ipa{s}   & 	 	   & 	 	   & 	 	   \\ 
		%\ipa{z}   & 	 	   & 	 	   & 	 	   \\ 
		%\ipa{ɬ}   & 	 	   & 	 	   & 	 	   \\ 
		\ipa{tɕ}   & 	 	 \deux{ltɕ}  \tib{}  & \japhug{rtɤltɕaʁ}{horse whip}  \\ 
		\ipa{tɕʰ}   & 	 	 \deux{ltɕʰ} \idph{}   & \japhug{ltɕʰɤltɕʰɤt}{hanging}  (of fluffy objects)\\ 
		\ipa{dʑ}   & 	 	 \deux{ldʑ} \tib{}   & \japhug{ldʑaŋkɯ}{green}  \\ 
		%\ipa{ndʑ}   & 	 	   & 	 	   & 	 	   \\ 
		%\ipa{ɕ}   & 	 	   & 	 	   & 	 	   \\ 
		%\ipa{ʑ}   & 	 	   & 	 	   & 	 	   \\ 
		%\ipa{tʂ}   & 	 	   & 	 	   & 	 	   \\ 
		%\ipa{tʂʰ}   & 	 	   & 	 	   & 	 	   \\ 
		\ipa{dʐ}   & 	 	 \deux{ldʐ} \idph{}   & \japhug{ldʐaŋldʐaŋ}{hanging} (big object) \\ 
		%\ipa{ndʐ}   & 	 	   & 	 	   & 	 	   \\ 
		%\ipa{ʂ}   & 	 	   & 	 	   & 	 	   \\ 
		\ipa{c}   & 	 	 \deux{lc} \idph{}  & \japhug{lcɯɣlcɯɣ}{drenching}  \\ 
		\ipa{cʰ}   & 	 	 \deux{lcʰ}   & \japhug{tɯ-lcʰɯɣ}{one section} (of a bag) \\ 
		%\ipa{ɟ}   & 	 	   & 	 	   & 	 	   \\ 
		%\ipa{ɲɟ}   & 	 	   & 	 	   & 	 	   \\ 
		%\ipa{ɲ}   & 	 	   & 	 	   & 	 	   \\ 
		%\ipa{k}   & 	 	   & 	 	   & 	 	   \\ 
		%\ipa{kʰ}   & 	 	   & 	 	   & 	 	   \\ 
		%\ipa{g}   & 	 	   & 	 	   & 	 	   \\ 
		%\ipa{ŋg}   & 	 	   & 	 	   & 	 	   \\ 
		\ipa{ŋ}   & 	 	 \deux{lŋ} \idph{}   & \japhug{lŋɤlŋɤt}{hanging} (fruit)  \\ 
		\ipa{x}   & 	 	 \deux{lx} \idph{}   & \japhug{lxɤβlxɤβ}{thick}  (clothes) \\ 
		\ipa{q}   & 	 	\deux{lq}   & 	 \japhug{lqɤnɤlqɤt}{toddling}	   \\ 
		%\ipa{qʰ}   & 	 	   & 	 	   & 	 	   \\ 
		%\ipa{ɴɢ}   & 	 	   & 	 	   & 	 	   \\ 
		%\ipa{χ}   & 	 	   & 	 	   & 	 	   \\ 
		% & 	 & 	 & 	 \\ 
		\midrule
		&\trois{lpɕ}	&\japhug{qalpɕa}{open} (fern leaf)\\
		\lspbottomrule
	\end{tabular}
\end{table}
\resetcounters{2lC}{3lC} %deux

A possible example of an additional cluster \ipa{lŋ-} is found in the verb \japhug{nɯndzɯ\-l\-ŋɯz}{doze off}, but both \ipa{nɯ.ndzɯl.ŋɯz} and \forme{nɯ.ndzɯ.lŋɯz} are possible syllabifications.

The cluster \forme{lpɕ\trt}, only attested in \japhug{qalpɕa}{open} (of fern leaf), is discussed further in §\ref{sec:kC.pC.clusters}.

\subsubsection{\ipa{r}+C clusters} \label{sec:rC.clusters}
\tabref{prein.r} lists all consonant clusters not ending in a non-nasal sonorant with the rhotic \ipa{r} or its unvoiced counterpart, the retroflex fricative \ipa{ʂ}, as first element.  

\begin{table}
	\caption{List of consonant clusters with \ipa{r} and \ipa{ʂ}  as a first element (35+0)} \label{prein.r}  \centering
	\begin{tabular}{Xlll}
		\lsptoprule
		\ipa{p}   & \deux{ʂp}   & \japhug{tɯ-rpa}{axe}  \\ 
		\ipa{pʰ}   & \deux{ʂpʰ} \idph{}   & \japhug{rpʰɤβrpʰɤβ}{flapping wings}  \\ 
		%\ipa{b}   &   \\ 
		\ipa{mb}   & \deux{rmb}   & \japhug{armbat}{near}  \\ 
		\ipa{m}   & \deux{rm}   & \japhug{rmɤβja}{peacock}  \\ 
		\ipa{t}   & \deux{ʂt}   & \japhug{rtalu}{horse year}  \\ 
		\ipa{tʰ}   & \deux{ʂtʰ}   & \japhug{ɯ-pɤrtʰɤβ}{middle}  \\ 
		\ipa{d}   & \deux{rd}   & \japhug{rdɤstaʁ}{stone}  \\ 
		\ipa{nd}   & \deux{rnd}   & \japhug{rnde}{find}  \\ 
		\ipa{n}   & \deux{rn}   & \japhug{rnaʁ}{be deep}  \\ 
		\ipa{ts}   & \deux{ʂts}   & \japhug{rtsot}{vengeance}  \\ 
		\ipa{tsʰ}   & \deux{ʂtsʰ}   & \japhug{rtsʰom}{have a crack} (bucket) \\ 
		\ipa{dz}   & \deux{rdz} \idph{}   & \japhug{rdzardza}{insolent}  \\ 
		\ipa{ndz}   & \deux{rndz}   & \japhug{rndzɤkɤŋe}{shade of the mountain}  \\ 
		\ipa{s}   & \deux{ʂs} \idph{}   & \japhug{rsɯβrsɯβ}{hairy}  \\ 
		\ipa{z}   & \deux{rz}   & \japhug{tɯ-rzɯɣ}{one section}  \\ 
		%\ipa{ɬ}   &  &  &  \\ 
		\ipa{tɕ}   & \deux{ʂtɕ}   & \japhug{nɯrtɕa}{tease}  \\ 
		\ipa{tɕʰ}   & \deux{ʂtɕʰ}   & \japhug{rtɕʰɯʁjɯ}{caterpillar}  \\ 
		%\ipa{dʑ}   &  &  &  \\ 
		\ipa{ndʑ}   & \deux{rndʑ}   & \japhug{cirndʑi}{sand}  \\ 
		\ipa{ɕ}   & \deux{ʂɕ}   & \japhug{rɕɯβrɕɯβ}{rough}  \\ 
		\ipa{ʑ}   & \deux{rʑ}   & \japhug{tɤ-rʑaβ}{wife}  \\ 
		%\ipa{tʂ}   &  &  &  \\ 
		%\ipa{tʂʰ}   &  &  &  \\ 
		%\ipa{dʐ}   &  &  &  \\ 
		%\ipa{ndʐ}   &  &  &  \\ 
		%\ipa{ʂ}   &  &  &  \\ 
		\ipa{c}   & \deux{ʂc}   & \japhug{tɤ-rcoʁ}{mud}  \\ 
		\ipa{cʰ}   & \deux{ʂcʰ}   & \japhug{ɯ-rcʰarcʰɤβ}{interstice}  \\ 
		\ipa{ɟ}   & \deux{rɟ}   & \japhug{rɟaʁ}{dance}  \\ 
		\ipa{ɲɟ}   & \deux{rɲɟ}   & \japhug{rɲɟaʁlo}{bolt}  \\ 
		\ipa{ɲ}   & \deux{ʂɲ} \idph{}   & \japhug{ʂɲoʁʂɲoʁ}{long and thin}  \\ 
		& \deux{rɲ} \tib{}   & \japhug{rɲaŋ}{be ancient}  \\ 
		\ipa{k}   & \deux{ʂk}   & \japhug{rko}{be hard}  \\ 
		\ipa{kʰ}   & \deux{ʂkʰ}   & \japhug{tɤ-rkʰom}{feather rachis}  \\ 
		\ipa{g}   & \deux{rg}   & \japhug{rga}{like}  \\ 
		\ipa{ŋg}   & \deux{rŋg}   & \japhug{rŋgɤm}{hard piece}  \\ 
		\ipa{ŋ}   & \deux{rŋ}   & \japhug{tɯ-rŋa}{face}  \\ 
		%\ipa{x}   &  &  &  \\ 
		\ipa{q}   & \deux{ʂq}   & \japhug{rqoʁ}{hug}  \\ 
		\ipa{qʰ}   & \deux{ʂqʰ}   & \japhug{tɤ-rqʰu}{bark, skin}  \\ 
		\ipa{ɴɢ}   & \deux{rɴɢ}   & \japhug{ɕɯrɴɢo}{Anisodus tanguticus}  \\ 
		\ipa{χ}   & \deux{ʂχ}   & \japhug{ʂχɯʂχi}{with big nostrils}  \\ 
		\lspbottomrule
	\end{tabular}
\end{table}
\resetcounters{2rC}{3rC} %deux 


The contrast between \ipa{r} and \ipa{ʂ} is almost completely neutralized in preinitial position, the former occurring with voiced obstruents, and the latter with unvoiced ones. The symbol \forme{r} is used in the orthography employed in this grammar for the archiphoneme \archi{r,ʂ} preceding obstruents, except in the group \forme{ʂχ-}.

However, a contrast between \ipa{r} and \ipa{ʂ} is found before sonorant initials,  in particular with the palatal nasal (\ipa{ʂɲ-} vs. \ipa{rɲ-}), the velar nasal (\ipa{ʂŋ-} vs. \ipa{rŋ-}) and the velar fricative (\ipa{ʂɣ-} vs. \ipa{rɣ-} see  §\ref{sec:CG.clusters}).

When preceding sonorants, \ipa{r} is initial when occurring before \forme{w} and \forme{j} (§\ref{sec:Cw.clusters}, §\ref{sec:Cj.clusters}), but preinitial in \forme{rl\trt}, \forme{rɣ-} and \forme{rʁ-} (§\ref{sec:Cl.clusters}, §\ref{sec:CG.clusters}, §\ref{sec:CRR.clusters}).

There is one case of a preinitial \forme{r}  apparently originating from a rhotacized sibilant, in the verb form \japhug{arɴɢlɯm}{be caved in} (§\ref{sec:Cl.clusters}, §\ref{sec:fossil.prenasalization}).

\subsubsection{Alveolo-palatal+C clusters}  \label{sec:shC.clusters}
\tabref{prein.C.Z} lists all consonant clusters with an alveolo-palatal fricative \ipa{ɕ} or \ipa{ʑ} as preinitial and not ending in a non-nasal sonorant.  

\begin{table}
	\caption{List of consonant clusters with \ipa{ɕ} and \ipa{ʑ}  as a first element (18+0)} \label{prein.C.Z}  
	\begin{tabular}{Xlll}
		\lsptoprule
		\ipa{p} & \deux{ɕp} & \japhug{ɕpaʁ}{be thirsty} \\ 
		\ipa{pʰ} & \deux{ɕpʰ} & \japhug{ɕpʰɤt}{patch} (vt) \\ 
		%\ipa{b} & & & \\ 
		\ipa{mb} & \deux{ʑmb} & \japhug{ʑmbɤr}{ulcer} \\ 
		\ipa{m} & \deux{ɕm} & \japhug{ɕmi}{mix} \\ 
		\ipa{t} & \deux{ɕt} & \japhug{ɕte}{contaminate} \\ 
		\ipa{tʰ} & \deux{ɕtʰ} & \japhug{ɕtʰɯz}{turn towards} \\ 
		\ipa{d} & \deux{ʑd} \idph{} & \japhug{ʑdɯɣʑdɯɣ}{strong, tough} \\ 
		%\ipa{nd} & & & \\ 
		\ipa{n} & \deux{ɕn} & \japhug{ɕnat}{heddle} \\ 
		& \deux{ʑn} & \forme{ʑ-nɯ-ɕar} `go and look for it' \\ 
		\ipa{ts} & \deux{ɕts}&\japhug{ŋɤjɕtsa}{foreman} \\ 
		%\ipa{tsʰ} & & & \\ 
		%\ipa{dz} & & & \\ 
		%\ipa{ndz} & & & \\ 
		%\ipa{s} & & & \\ 
		%\ipa{z} & & & \\ 
		%\ipa{ɬ} & & & \\ 
		\ipa{tɕ} &\deux{ɕtɕ} & \japhug{sɤɕtɕɯɣ}{strap} (to carry children on the back) \\ 
		%\ipa{tɕʰ} & & & \\ 
		%\ipa{dʑ} & & & \\ 
		%\ipa{ndʑ} & & & \\ 
		%\ipa{ɕ} & & & \\ 
		%\ipa{ʑ} & & & \\ 
		\ipa{tʂ} & \deux{ɕtʂ} \idph{} & \japhug{ɕtʂaŋlaŋ}{hanging and swinging} \\ 
		%\ipa{tʂʰ} & & & \\ 
		%\ipa{dʐ} & & & \\ 
		%\ipa{ndʐ} & & & \\ 
		%\ipa{ʂ} & & & \\ 
		%\ipa{c} & & & \\ 
		%\ipa{cʰ} & & & \\ 
		%\ipa{ɟ} & & & \\ 
		%\ipa{ɲɟ} & & & \\ 
		%\ipa{ɲ} & & & \\ 
		\ipa{k} &   \deux{ɕk} & \japhug{ɕkom}{muntjac} \\ 
		\ipa{kʰ} &   \deux{ɕkʰ} & \japhug{ɕkʰo}{spread} \\ 
		\ipa{g} &   \deux{ʑg} & \japhug{ʑgaʁ}{exactly} \\ 
		\ipa{ŋg} &   \deux{ʑŋg} & \japhug{ʑŋgu}{cross river} \\ 
		\ipa{ŋ} &   \deux{ɕŋ} \idph{} & \japhug{ɕŋaʁɕŋaʁ}{bright yellow} \\ 
		%\ipa{x} & 	 & & \\ 
		\ipa{q} &   \deux{ɕq} & \japhug{ɕqɤjɤr}{cross-eyed} \\ 
		\ipa{qʰ} &   \deux{ɕqʰ} & \japhug{ɕqʰaloʁ}{latch} \\ 
		\ipa{ɴɢ} &   \deux{ʑɴɢ} & \japhug{ʑɴɢɯloʁ}{walnut} \\ 
		%\ipa{χ} & 	 & & \\ 
		\lspbottomrule
	\end{tabular}
\end{table}
\resetcounters{2CZC}{3CZC} %deux 

As in the case of the sibilant preinitials \forme{s-} and \forme{z\trt}, the voicing contrast between \ipa{ɕ} and \ipa{ʑ} is neutralized in preinitial position before obstruents, the former occurring when followed by unvoiced stops, fricatives and affricates, and the latter when followed by voiced ones.

A voicing contrast is attested between \ipa{ɕn-} and \ipa{ʑn\trt}, but the latter only occurs when the \forme{ʑ-} allomorph of the translocative prefix (§\ref{sec:translocative.morpho}) precedes the A-type \forme{nɯ-} or C-type   \forme{na-}  `westward' orientation preverbs (§\ref{sec:kamnyu.preverbs}), as in examples (\ref{ex:iCqha.ZnWzmWnmuta}) (§\ref{sec:iCqha}) and (\ref{ex:ZnasAmWmtshWmtshAmnW2}) (§\ref{sec:inner.prefixal.chain}). 

Before non-nasal sonorants, the voiced contrast is attested between \ipa{ɕl-} vs. \ipa{ʑl-} (§\ref{sec:Cl.clusters}, also only with the translocative prefix), \ipa{ɕr-} vs. \ipa{ʑr-} (§\ref{sec:Cr.clusters}) and \ipa{ɕɣ-} vs. \ipa{ʑɣ-} (§\ref{sec:CG.clusters}).

Unlike  sibilant preinitials, alveolo-palatal preinitials are compatible with affricates of all places of articulation, even alveolo-palatal affricates in the cluster \forme{ɕtɕ-} found in the noun \japhug{sɤɕtɕɯɣ}{strap} (to carry children). There is evidence, however, that \forme{*ɕtsʰ-} dissimilated to \forme{jtsʰ-} in the irregular causative \japhug{jtsʰi}{give to drink} (§\ref{sec:jC.clusters}, §\ref{sec:caus.j}). 

There are two examples of alveolo-palatal preinitials followed by a palatal segment: \forme{ʑɲ-} and \forme{ɕcʰ\trt}, but they are not stable and hence not included in \tabref{prein.C.Z}. They only occur when the \forme{ʑ-} and \forme{ɕ-} allomorphs of the translocative prefix are found before the B-type \forme{ɲɯ-} and \forme{cʰɯ-} or the D-type \forme{ɲɤ-} and \forme{cʰɤ-} `westward' and \textsc{downstream} orientation preverbs. These clusters are always in free variation with \forme{zɲ-} and \forme{scʰ\trt}, respectively, which are considerably more common (\tabref{tab:transloc.allomorphs.counts}, §\ref{sec:translocative.morpho}), since the translocative prefix generally undergoes a palatal dissimilation rule to a dental fricative \forme{s-} or \forme{z-} when it precedes a palatal consonant. The non-dissimilated clusters \forme{ɕcʰ-} and \forme{ʑɲ-} are extremely rare: the former is only attested in one single example in the corpus (\ref{ex:CchWZGABdea}), and the second in a handful of verb forms (see for instance \ref{ex:ZYWlata}, §\ref{sec:translocative.morpho}). These non-dissimilated forms are undoubtedly due to analogical levelling, which generalized the alveolo-palatal allomorphs of the translocative.

\begin{exe}
	\ex \label{ex:CchWZGABdea}
	\gll a-tʰɯ-ɤ<nɯ>kʰu ɕti ma aʑo ɕ-cʰɯ-ʑɣɤ-βde-a ɕti ma \\
	\textsc{irr}-\textsc{pfv}:\textsc{downstream}-<\textsc{auto}>call be.\textsc{aff}:\textsc{fact} \textsc{lnk} \textsc{1sg} \textsc{tral}-\textsc{ipfv}:\textsc{downstream}-\textsc{refl}-throw-\textsc{1sg} be.\textsc{aff}:\textsc{fact} \textsc{lnk} \\
	\glt `Let her call (me), I am going to throw myself (in the river).' (22-qajdo, 55)
\end{exe}

Secondary clusters with alveolo-palatal preinitials are created by prefixes with the shape \forme{ɕ-} or \forme{ʑ\trt}, including allomorphs of the translocative as mentioned above (§\ref{sec:translocative.morpho}), but also irregular causatives (§\ref{sec:caus.C}, §\ref{sec:caus.Z}) and perhaps lexicalized oblique participles (§\ref{sec:lexicalized.oblique.participle}).

\subsubsection{\ipa{j}+C clusters}  \label{sec:jC.clusters}
\tabref{prein.j}  lists all consonant clusters with the semi-vowel \ipa{j} as first element not ending in a non-nasal sonorant. It cannot precede alveolo-palatal obstruents and palatal stops.

\begin{table}
	\caption{List of consonant clusters with \ipa{j}  as a first element (12+1)} \label{prein.j}  \centering
	\begin{tabular}{Xlll}
		\lsptoprule
		\ipa{p}   & 	 	 \deux{jp}   & \japhug{jpum}{be thick}  \\ 
		%\ipa{pʰ}   & 	 	   & 	 	   & 	 	   \\ 
		%\ipa{b}   & 	 	   & 	 	   & 	 	   \\ 
		%\ipa{mb}   & 	 	   & 	 	   & 	 	   \\ 
		\ipa{m}   & 	 	 \deux{jm}   & \japhug{jmɯt}{forget}  \\ 
		\ipa{t}   & 	 	 \deux{jt}   & \japhug{ajtɯ}{accumulate}  \\ 
		%\ipa{tʰ}   & 	 	   & 	 	   & 	 	   \\ 
		%\ipa{d}   & 	 	   & 	 	   & 	 	   \\ 
		\ipa{nd}   & 	 	 \deux{jnd} 	   & 	  \japhug{sɤjndɤt}{be cute} 	   \\ 
		\ipa{n}   & 	 	 \deux{jn}   & \japhug{jnom}{be flexible}  \\ 
		\ipa{ts}   & 	 	 \deux{jts}   & \japhug{tɤ-jtsi}{pillar}  \\ 
		\ipa{tsʰ}   & 	 	 \deux{jtsʰ}   & \japhug{jtsʰi}{give to drink}  \\ 
		%\ipa{dz}   & 	 	   & 	 	   & 	 	   \\ 
		%\ipa{ndz}   & 	 	   & 	 	   & 	 	   \\ 
		%\ipa{s}   & 	 	   & 	 	   & 	 	   \\ 
		%\ipa{z}   & 	 	   & 	 	   & 	 	   \\ 
		%\ipa{ɬ}   & 	 	   & 	 	   & 	 	   \\ 
		%\ipa{tɕ}   & 	 	   & 	 	   & 	 	   \\ 
		%\ipa{tɕʰ}   & 	 	   & 	 	   & 	 	   \\ 
		%\ipa{dʑ}   & 	 	   & 	 	   & 	 	   \\ 
		%\ipa{ndʑ}   & 	 	   & 	 	   & 	 	   \\ 
		%\ipa{ɕ}   & 	 	   & 	 	   & 	 	   \\ 
		%\ipa{ʑ}   & 	 	   & 	 	   & 	 	   \\ 
		%\ipa{tʂ}   & 	 	   & 	 	   & 	 	   \\ 
		\ipa{tʂʰ}   & 	 	 \deux{jtʂʰ}   & \japhug{qajtʂʰa}{vulture}  \\ 
		%\ipa{dʐ}   & 	 	   & 	 	   & 	 	   \\ 
		\ipa{ndʐ}   & 	 	 \deux{jndʐ}   & \japhug{jndʐɤz}{be thick} (powder) \\ 
		%\ipa{ʂ}   & 	 	   & 	 	   & 	 	   \\ 
		%\ipa{c}   & 	 	   & 	 	   & 	 	   \\ 
		%\ipa{cʰ}   & 	 	   & 	 	   & 	 	   \\ 
		%\ipa{ɟ}   & 	 	   & 	 	   & 	 	   \\ 
		%\ipa{ɲɟ}   & 	 	   & 	 	   & 	 	   \\ 
		%\ipa{ɲ}   & 	 	   & 	 	   & 	 	   \\ 
		\ipa{k}   & 		 \deux{jk}   & \japhug{tɤ-jkɯz}{secret}  \\ 
		%\ipa{kʰ}   & 		   & 		   & 		   \\ 
		%\ipa{g}   & 		   & 		   & 		   \\ 
		%\ipa{ŋg}   & 		   & 		   & 		   \\ 
		\ipa{ŋ}   & 		 \deux{jŋ}   & \japhug{tɤ-jŋoʁ}{hook}  \\ 
		%\ipa{x}   & 		   & 		   & 		   \\ 
		\ipa{q}   & 		 \deux{jq}   & \japhug{jqu}{be able to lift}  \\ 
		%\ipa{qʰ}   & 		   & 		   & 		   \\ 
		%\ipa{ɴɢ}   & 		   & 		   & 		   \\ 
		\ipa{χ}   & 		 \deux{jχ}   & \japhug{ajχoʁ}{be flat} (belly)  \\ 
		&\trois{jmŋ} & \japhug{tɯ-jmŋo}{dream}  \\  
		\lspbottomrule
	\end{tabular}
\end{table}
\resetcounters{2jC}{3jC} 

Most of these clusters violate the sonority sequencing principle (§\ref{sec:ssp}), but those that are attested in verb roots (such as \japhug{jpum}{be thick}) can occur in word-initial position in the Factual Non-Past (§\ref{sec:factual}).

Comparison with Zbu reveals that one of the sources of the Japhug \forme{j-} preinitial is proto-Gyalrong \forme{*l-} \citep[271--272]{jacques04these}: for instance \japhug{jmɯt}{forget},  \japhug{tɤ-jtsi}{pillar}, \japhug{tɤ-jme}{tail} and  \japhug{tɯ-jmŋo}{dream}  correspond to Zbu \forme{lmɑ̂t}, \forme{tɐltsíʔ}, \forme{-lméʔ} and \forme{-lmɑ́ʔ}, respectively \citep[43; 53; 288]{gong18these}. In other cases, \forme{j-} comes from a dissimilated coronal fricative, as in \japhug{jtsʰi}{give to drink} (§\ref{sec:caus.j}) and perhaps \japhug{jqu}{be able to lift} (§\ref{sec:abilitative.lexicalized}).

The group \forme{jmŋ-} in \japhug{tɯ-jmŋo}{dream} has two preinitials \ipa{j} and \ipa{m}. The labial nasal is the ancient initial (as shown by the Zbu cognate \forme{-lmɑ́ʔ}), and the nasal velar arose as a trace of proto-Gyalrong vowel velarization \forme{*-lmaˠŋ} \fl{} \forme{*-jmɣo} \fl{} \forme{-jmŋo} (\citealt[44]{jacques04these}; on velarized vowels in proto-Gyalrong, see §\ref{sec:CG.clusters}).

When preceding sonorants, \ipa{j} is preinitial in all cases: \forme{jw\trt}, \forme{jl\trt}, \forme{jr\trt}, \forme{jɣ\trt}, \forme{jʁ-} and \forme{rʁ-} (§\ref{sec:Cw.clusters}, §\ref{sec:Cl.clusters}, §\ref{sec:Cr.clusters}, §\ref{sec:CG.clusters}, §\ref{sec:CRR.clusters}).

\subsubsection{Velar+C clusters}  \label{sec:xC.clusters}
\tabref{prein.x} lists all consonant clusters with the velar fricatives \ipa{x} and \ipa{ɣ} as preinitial and not ending in a non-nasal sonorant.  

\begin{table}
	\caption{List of consonant clusters with  \ipa{x} and \ipa{ɣ} as a first element (23+0)} \label{prein.x}  \centering
	\begin{tabular}{Xlll}
		\lsptoprule
		\ipa{p}	 & 	 	 \deux{xp}	 & \japhug{tɯ-xpa}{one year} \\ 
		%\ipa{pʰ}	 & 		 & 		 & 		 \\ 
		%\ipa{b}	 & 		 & 		 & 		 \\ 
		\ipa{mb}	 & 	 	 \deux{ɣmb}	 & \japhug{tɯ-ɣmba}{cheek}  \\ 
		\ipa{m}	 & 	 	 \deux{ɣm}	 & \japhug{tɯ-ɣmaz}{wound}  \\ 
		\ipa{t}	 & 	 	 \deux{xt}	 & \japhug{xtɯt}{wild cat}  \\ 
		\ipa{tʰ}	 & 	 	 \deux{xtʰ}	 & \japhug{xtʰom}{put horizontally}  \\ 
		\ipa{d}	 & 	 	 \deux{ɣd}	 & \japhug{ɣdɤso}{species of grub}  \\ 
		\ipa{nd}	 & 	 	 \deux{ɣnd}	 & \japhug{ɣnda}{ram}  \\ 
		\ipa{n}	 & 	 	 \deux{ɣn}	 & \japhug{ɣnɤsqi}{twenty}  \\ 
		\ipa{ts}	 & 	 	 \deux{xts}	 & \japhug{xtsɤɕna}{tip of boot}  \\ 
		\ipa{tsʰ}	 & 	 	 \deux{xtsʰ}	 & \japhug{xtsʰɯm}{be thin}  \\ 
		%\ipa{dz} 	 & 		 & 		 & 		 \\ 
		%\ipa{ndz}	 & 		 & 		 & 		 \\ 
		\ipa{s}	 & 	 	 \deux{xs}	 & \japhug{xsar}{goral}  \\ 
		\ipa{z}	 & 	 	 \deux{ɣz}	 & \japhug{ɣzɯ}{monkey}  \\ 
		%\ipa{ɬ} 	 & 		 & 		 & 		 \\ 
		\ipa{tɕ}	 & 	 	 \deux{xtɕ}	 & \japhug{xtɕi}{be small}  \\ 
		\ipa{tɕʰ}	 & 	 	 \deux{xtɕʰ}	 & \japhug{xtɕʰɯt}{can contain}  \\ 
		%\ipa{dʑ} 	 & 		 & 		 & 		 \\ 
		\ipa{ndʑ}	 & 	 	 \deux{ɣndʑ}	 & \japhug{ɣndʑɤβ}{fire}  \\ 
		\ipa{ɕ}	 & 	 	 \deux{xɕ}	 & \japhug{xɕaj}{grass}  \\ 
		\ipa{ʑ}	 & 	 	 \deux{ɣʑ}	 & \japhug{ɣʑo}{bee}  \\ 
		\ipa{tʂ}	 & 	 	 \deux{xtʂ}	 & \japhug{nɤxtʂɯ}{bring in passing}  \\ 
		%\ipa{tʂʰ}	 & 		 & 		 & 		 \\ 
		%\ipa{dʐ}	 & 		 & 		 & 		 \\ 
		%\ipa{ndʐ}	 & 		 & 		 & 		 \\ 
		\ipa{ʂ}	 & 	 	 \deux{xʂ} \idph{}	 & \japhug{xʂɤxʂɤt}{long and thin}  \\ 
		\ipa{c}	 & 	 	 \deux{xc}	 & \japhug{xcat}{be many}  \\ 
		\ipa{cʰ}	 & 	 	 \deux{xcʰ}	 & \japhug{tɤlɤxcʰi}{curdled milk}  \\ 
		\ipa{ɟ}	 & 	 	 \deux{ɣɟ}	 & \japhug{ɣɟaβ}{churn}  \\ 
		%\ipa{ɲɟ}	 &	 & 	 	 	 & 	 	 	 \\ 
		\ipa{ɲ}	 & 	 	 \deux{ɣɲ}	 & \japhug{ɯ-ɣɲaʁ}{disaster}  \\ 
		\lspbottomrule
	\end{tabular}
\end{table}
\resetcounters{2xGC}{3xGC} %deux 

Velar preinitials are not compatible with dorsal (velar and velar) initial consonants, but are attested with all other places of articulations. The complementary distribution between \forme{x} (before unvoiced segments) and \forme{ɣ} (before voiced segments, including obstruents and nasals) is complete, and the voicing contrast can be considered to be completely neutralized in preinitial position.

Comparison with Situ shows that some of these velar fricative preinitials originate from velar presyllables \forme{*kə-} (\citealt[6]{jacques14antipassive}). Two non-productive prefixes are sources of velar+C clusters: the animal velar prefix (§\ref{sec:velar.class.prefix}) and the lexicalized participle \forme{x/ɣ-} prefix (§\ref{sec:G.nmlz}). Another source of secondary velar preinitials is the intrusive \forme{ɣ} found as element of some derivation prefixes, including the \forme{sɯɣ-} (§\ref{sec:caus.sWG}) and \forme{ɕɯɣ-} (§\ref{sec:caus.CWG}) allomorphs of the causative, the \forme{nɯɣ-} allomorph of the applicative (§\ref{sec:allomorphy.applicative}), the \forme{nɤɣ-} allomorph of the tropative (§\ref{sec:tropative.allomorphy}) and the \forme{sɤɣ-} allomorph of the proprietive (§\ref{sec:proprietive.allomorphy}).

In addition, the velar preinitial in some counted nouns like \japhug{tɯ-xpa}{one year} is due to a \textit{fausse coupe} (§\ref{sec:num.prefix.paradigm.history}).

\subsubsection{Uvular+C clusters}  \label{sec:XC.clusters}
\tabref{prein.X.R} lists all consonant clusters with the uvular fricatives \ipa{χ} and \ipa{ʁ} as first element and not ending in a non-nasal sonorant. They cannot precede velars except in heterosyllabic clusters (§\ref{sec:heterosyllabic.clusters}).


\begin{table}
	\caption{List of consonant clusters with  \ipa{χ} and \ipa{ʁ} as a first element (25+0)} \label{prein.X.R}  
	\begin{tabular}{Xllllllll}
		\lsptoprule
		\ipa{p}	 &	   \deux{χp} \tib{}	 & \japhug{χpi}{story}  &	   		 \\
		\ipa{pʰ}	 &	 	 \deux{χpʰ}	 & \japhug{taχpʰe}{slap}  &	   	 \\
		\ipa{b}	 &	\deux{ʁb}  \idph{}	 & \japhug{ʁbɤʁbɤβ}{thick and big}  \\
		\ipa{mb}	 &	 	  \deux{ʁmb}	 & \japhug{aʁmbɯm}{be concave}  \\
		\ipa{m}	 &	 	 \deux{ʁm}	 & \japhug{ʁmaʁ}{army}  \\
		\ipa{t}	 &	 	 \deux{χt}	 & \japhug{χtɤrma}{offerings}  &	   	 \\
		\ipa{tʰ}	 &	 	 \deux{χtʰ}	 & \japhug{naχtʰɤβ}{seize the opportunity}  &	  	 \\
		\ipa{d}	 &	 	 \deux{ʁd} \tib{} 	 & \japhug{ʁdɯɣ}{umbrella}  \\
		\ipa{nd}	 &	 	 \deux{ʁnd}	 & \japhug{ʁndɤr}{scatter}  (anticausative)\\
		\ipa{n}	 &	 \deux{ʁn}	 & \japhug{ʁnaʁna}{both}  \\
		\ipa{ts}	 &	 	 \deux{χts}	 & \japhug{χtso}{it is clean}  &	   	 \\
		\ipa{tsʰ}	 &	 	 \deux{χtsʰ} \idph{}	 & \japhug{χtsʰɤχtsʰɤt}{small and active}  &	  	 \\
		%\ipa{dz} 	 &	 	    \\
		\ipa{ndz}	 &		 \deux{ʁndz}	 & \japhug{ʁndzɤr}{cut} (with scissors) \\
		\ipa{s}	 &	 	 \deux{χs}	 & \japhug{χsɤr}{gold}  &	  	 \\
		\ipa{z}	 &	 \deux{ʁz}  \tib{}	 &	   \japhug{ʁzɤβ}{be careful}   		 \\
		%\ipa{ɬ} 	 &	 	    \\
		\ipa{tɕ}	 &	 	 \deux{χtɕ}  \tib{} 	 & \japhug{χtɕoŋ}{rheumatism}  &	   	 \\
		%\ipa{tɕʰ}	 &	   		 \\
		%\ipa{dʑ} 	 &	 	    \\
		%\ipa{ndʑ}	 &		   	 \\
		\ipa{ɕ}	 &	 	 \deux{χɕ}	 & \japhug{χɕu}{be strong}  &	  	 \\
		\ipa{ʑ}	 &	\ipa{ʁʑ}  &\japhug{ʁʑɯnɯ}{young man}   		 \\
		\ipa{tʂ}	 &	 	 \deux{χtʂ}  \tib{}	 & \japhug{χtʂɯɣdʑa}{butter tea}  &	  	 \\
		%\ipa{tʂʰ}	 &		   	 \\
		%\ipa{dʐ}	 &		   	 \\
		%\ipa{ndʐ}	 &		   	 \\
		\ipa{ʂ}	 &	 	 \deux{χʂ} \idph{}	 & \japhug{χʂɤχʂɤt}{light (clothes)}  &	   	 \\
		\ipa{c}	 &	 	 \deux{χc} \tib{}	 & \japhug{χcoŋkroŋ}{cross-legged (sitting)}  &	  	 \\
		\ipa{cʰ}	 &	 	 \deux{χcʰ}	 & \japhug{χcʰa}{right}  &	  	 \\
		\ipa{ɟ}	 &		 \deux{ʁɟ}	 & \japhug{ʁɟa}{completely}  \\
		\ipa{ɲɟ}	 &		 \deux{ʁɲɟ}	 & \japhug{ʁɲɟiʁɲɟi}{enormous}  \\
		\ipa{ɲ}	 &	 	\deux{χɲ} \idph{}	 & \japhug{χɲɤχɲɤr}{without energy} \\
		& \deux{ʁɲ}\tib{}	 & \japhug{ʁɲɤrpa}{steward (monastery)}  \\
		\lspbottomrule
	\end{tabular}
\end{table}
\resetcounters{2XRC}{3XRC} %deux 

The voicing contrast between \ipa{χ} and \ipa{ʁ} is only attested before sonorants: a handful of ideophones have the cluster \ipa{χɲ} contrasting with \ipa{ʁɲ}. Uvular preinitials are particularly common in Tibetan loanwords, where they correspond to the \forme{g-} and \forme{d-} \forme{sŋon.ⁿdʑug}. The voiced fricative \ipa{ʁ} is more often realized as an epiglottal \phonet{ʕ} in this context.

Some uvular preinitials come from the reduced allomorphs \forme{χ-} and \forme{ʁ-} of the uvular animal prefix \forme{qa-} (§\ref{sec:uvular.animal}).


\subsubsection{Nasal+C clusters}  \label{sec:NC.clusters}
The Tables \ref{prein.nasal} and \ref{prein.nh.nasal} list all consonant clusters with a nasal consonant as preinitial and without medial consonant. 

Unlike the previous preinitials, nasals can never be directly followed by any non-nasal sonorant: groups such as \forme{*mj\trt}, \forme{*mw\trt}, \forme{*mr\trt}, \forme{*ml\trt}, \forme{*mɣ-} and \forme{*mʁ-} are not attested in onset position (they are, however, found as heterosyllabic clusters, §\ref{sec:heterosyllabic.clusters}). This gap is due to a combination of two sound changes. 

First, the medials \forme{*j} and \forme{*ɣ} were nasalized when preceded by a nasal, so that pre-Japhug \forme{*mj-} and \forme{*mɣ-} became \forme{mɲ-} (in \japhug{tɯ-mɲa}{arrow}, compare \tibet{མདའ་}{mda}{arrow} from \forme{*mla}, \citealt[18]{hill2019phonology}) and \forme{mŋ-} (in \japhug{tɯ-jmŋo}{dream}, see §\ref{sec:jC.clusters}), respectively.  

Second, pre-Japhug \forme{*mr-} became \forme{mbr-} by epenthesis, as shown by \japhug{mbro}{horse}, \japhug{mbro}{be high} and \japhug{mbri}{cry, sing} (in Burmese \forme{mraṅḥ}, \forme{mraṅʔ} and \forme{mraññ}, \citealt[60;259]{hill2019phonology}).

Additionally, nasal preinitials cannot be followed by any fricative. As in Tibetan \citep{lifk33}, this absence is due to the conversion of post-nasal fricatives to the corresponding affricates, as in \japhug{tɯ-mtsʰi}{liver} (from \forme{*m-si}, Burmese \forme{asaññḥ}, \citealt[56]{hill2019phonology}).

\begin{table} 
	\caption{List of consonant clusters with a homorganic nasal as  first element (14+1)} \label{prein.nasal}   
	\begin{tabular}{Xlll}
		\lsptoprule
		\ipa{p} 	 &	 \deux{mp} 	 & \japhug{mpɯ}{be soft} \\	
		\ipa{pʰ} 	 &	 \deux{mpʰ} 	 & \japhug{mpʰɯl}{reproduce} \\	
		%\ipa{b} 	 &	 	 &	 	 &	 	 \\	
		%\ipa{mb} 	 &	 	 &	 	 &	 	 \\	
		%\ipa{m} 	 &	 	 &	 	 &	 	 \\	
		\ipa{t} 	 &	 \deux{nt} 	 & \japhug{ntaβ}{be stable} \\	
		\ipa{tʰ} 	 &	 \deux{ntʰ} 	 & \japhug{ntʰɤβ}{be caught between} \\	
		%\ipa{d} 	 &	 	 &	 	 &	 	 \\	
		%\ipa{nd} 	 &	 	 &	 	 &	 	 \\	
		%\ipa{n} 	 &	 	 &	 	 &	 	 \\	
		\ipa{ts} 	 &	 \deux{nts} 	 & \japhug{ntsɯ}{always} \\	
		\ipa{tsʰ} 	 &	 \deux{ntsʰ} 	 & \japhug{ntsʰɤr}{neigh} (of horse) \\	
		%\ipa{dz} 	 &	 	 &	 	 &	 	 \\	
		%\ipa{ndz} 	 &	 	 &	 	 &	 	 \\	
		%\ipa{s} 	 &	 	 &	 	 &	 	 \\	
		%\ipa{z} 	 &	 	 &	 	 &	 	 \\	
		%\ipa{ɬ} 	 &	 	 &	 	 &	 	 \\	
		%\ipa{tɕ} 	 &	 	 &	 	 &	 	 \\	
		\ipa{tɕʰ} 	 &	 \deux{ntɕʰ} 	 & \japhug{ntɕʰoz}{use} \\	
		%\ipa{dʑ} 	 &	 	 &	 	 &	 	 \\	
		%\ipa{ndʑ} 	 &	 	 &	 	 &	 	 \\	
		%\ipa{ɕ} 	 &	 	 &	 	 &	 	 \\	
		%\ipa{ʑ} 	 &	 	 &	 	 &	 	 \\	
		\ipa{tʂ} 	 &	 \deux{ntʂ} 	 & \japhug{ntʂu}{weed} (vt) \\	
		%\ipa{tʂʰ} 	 &	 	 &	 	 &	 	 \\	
		%\ipa{dʐ} 	 &	 	 &	 	 &	 	 \\	
		%\ipa{ndʐ} 	 &	 	 &	 	 &	 	 \\	
		%\ipa{ʂ} 	 &	 	 &	 	 &	 	 \\	
		\ipa{c} 	 &	 \deux{ɲc} 	 & \japhug{ɲcɤr}{press} \\	
		\ipa{cʰ} 	 &	 \deux{ɲcʰ} 	 & \japhug{ɲcʰoʁ}{shrink} \\	
		%\ipa{ɟ} 	 &	 	 &	 	 &	 	 \\	
		%\ipa{ɲɟ} 	 &	 	 &	 	 &	 	 \\	
		%\ipa{ɲ} 	 &	 	 &	 	 &	 	 \\	
		\ipa{k} 	 &	 \deux{ŋk} 	 & \japhug{ŋke}{walk} \\	
		\ipa{kʰ} 	 &	 \deux{ŋkʰ} 	 & \japhug{ŋkʰor}{arrive} \\	
		%\ipa{g} 	 &	 	 &	 	 &	 	 \\	
		%\ipa{ŋg} 	 &	 	 &	 	 &	 	 \\	
		%\ipa{ŋ} 	 &	 	 &	 	 &	 	 \\	
		%\ipa{x} 	 &	 	 &	 	 &	 	 \\	
		\ipa{q} 	 &	   \deux{ɴq} 	 & \japhug{ɴqa}{be difficult} \\	
		\ipa{qʰ}   	 &	   \deux{ɴqʰ} 	 & \japhug{ɴqʰi}{be dirty} \\	
		%\ipa{ɴɢ}  	 &	 	 &		 &		 \\	
		\midrule
		&\trois{mpɕ} &\japhug{mpɕɤr}{be beautiful} \\
		\lspbottomrule
	\end{tabular}
\end{table}		
\resetcounters{2NC}{3NC}

While voiced stops and affricates with homorganic prenasalization are monophonemic (§\ref{sec:consonant.phonemes}), unaspirated and aspirated ones (\tabref{prein.nasal}) are better analyzed as clusters. All places of articulation are possible, from labial (\forme{mp(ʰ)-}) to uvular (\forme{ɴq(ʰ)-}). The special case of \forme{mpɕ-} is discussed in §\ref{sec:kC.pC.clusters}.

Two non-homorganic prenasalized preinitials are also attested, \ipa{m-} before all places of articulation except labials (where \forme{m-} corresponds to the homorganic nasal preinitial), and \forme{n-} before (voiced) labials and velars (\tabref{prein.nh.nasal}). When preceded by \ipa{m} or \ipa{n}, the contrast between plain voiced and prenasalized voiced stops and affricates is neutralized; for instance the \forme{b} in \forme{nb-} stands for the archiphoneme \archi{b,mb}. In the case of the cluster \forme{mɢ\trt}, since the prenasalized phoneme \ipa{ɴɢ}  lacks a plain voiced counterpart  (§\ref{sec:consonant.phonemes}), the only analysis possible is \ipa{mɴɢ-}. In the orthography used in this grammar, the simplified (but non-ambiguous) transcription \forme{mɢ-} is used. 

The other nasal phonemes \ipa{ɲ} and \ipa{ŋ} are not attested as preinitials: for instance, clusters such as $\dagger$\forme{ŋp-} or $\dagger$\forme{ɲt-} are prohibited.

\begin{table} 
	\caption{List of consonant clusters with a non-homorganic nasal as  first element (24+0)} \label{prein.nh.nasal} 
	\begin{tabular}{Xlll}
		\lsptoprule
		%\ipa{p} & & & \\
		%\ipa{pʰ} & & & \\
		%\ipa{b} &\\
		\ipa{t} & \deux{mt} & \japhug{tɤ-mtɯ}{knot} \\
		\ipa{tʰ} & \deux{mtʰ}\tib{} & \japhug{mtʰɯ}{spell} \\
		%\ipa{d} & & & \\
		\ipa{nd} & \deux{md} & \japhug{mda}{arrive} (of time)\\
		\ipa{n} & \deux{mn} & \japhug{mna}{be better}, `heal' \\
		\ipa{ts} & \deux{mts} & \japhug{tɤ-mtsɯ}{button} \\
		\ipa{tsʰ} & \deux{mtsʰ} & \japhug{mtsʰɤm}{hear} \\
		%\ipa{dz} & & & \\
		\ipa{ndz} & \deux{mdz} & \japhug{mdzadi}{flea} \\
		%\ipa{s} & & & \\
		%\ipa{z} & & & \\
		%\ipa{ɬ} & & & \\
		\ipa{tɕ} & \deux{mtɕ} & \japhug{mtɕoʁ}{be sharp} \\
		\ipa{tɕʰ} & \deux{mtɕʰ} & \japhug{tɤ-mtɕʰo}{wedge} \\
		%\ipa{dʑ} & & & \\
		\ipa{ndʑ} & \deux{mdʑ} & \japhug{tɯ-mdʑu}{tongue} \\
		%\ipa{ɕ} & & & \\
		%\ipa{ʑ} & & & \\
		\ipa{tʂ} & \deux{mtʂ} & \japhug{mtʂɤkʰoz}{bib} \\
		%\ipa{tʂʰ} & & & \\
		%\ipa{dʐ} & & & \\
		\ipa{ndʐ} & \deux{mdʐ} & \japhug{mdʐɯɕɯɣ}{bedbug} \\
		%\ipa{ʂ} & & & \\
		\ipa{c} & \deux{mc} & \japhug{tɤmcar}{tongs} \\
		\ipa{cʰ} & \deux{mcʰ} & \japhug{tɯ-mcʰi}{gall} \\
		%\ipa{ɟ} & & & \\
		\ipa{ɲɟ} & \deux{mɟ} & \japhug{tɯ-mɟa}{jaw} \\
		\ipa{ɲ} & \deux{mɲ} & \japhug{mɲɤm}{species of tree} \\
		\ipa{k} & \deux{mk} & \japhug{tɯ-mke}{neck} \\
		\ipa{kʰ} & \deux{mkʰ} & \japhug{mkʰɤz}{be expert} \\
		%\ipa{g} & \\
		\ipa{ŋg} & \deux{mg} & \japhug{tɯ-mga}{advantage} \\
		\ipa{ŋ} & \deux{mŋ} & \japhug{mŋɤm}{hurt} \\
		%\ipa{x} & & & \\
		%\ipa{q} & & & \\
		%\ipa{qʰ} & & & \\
		\ipa{ɴɢ} & \deux{mɢ} & \japhug{tamɢom}{clamp} \\
		\midrule
		\ipa{mb} &  \deux{nb} 	& \japhug{anbaʁ}{hide}  \\
		\ipa{m} &  \deux{nm} 	& \japhug{tɤ-nmaʁ}{husband}  \\
		\ipa{ŋg} &\deux{ng} 	& \japhug{ngɯt}{be strong}, `be resistant' \\
		\ipa{ŋ}& \deux{nŋ} & \japhug{nŋo}{lose} \\
		\lspbottomrule
	\end{tabular}
\end{table}		
\resetcounters{2mnC}{3mnC}

Some \forme{m-} preinitials come from the nasalization of \forme{*w-} before nasal or prenasalized stops (as in the irregular causative \japhug{mɲo}{prepare} from \japhug{ɲo}{be ready}, see §\ref{sec:causative.m}), a sound law accounting for the absence of preinitial \ipa{w} in this context (§\ref{sec:wC.clusters}). This sound law does not operate across syllables (§\ref{sec:heterosyllabic.clusters}), and heterosyllabic clusters with a non-nasalized coda \forme{-β} followed by a nasal do exist.

Some \forme{n-} preinitials originate from former \forme{*t-} before nasal or prenasalized stops, as in \japhug{tɯ-nŋa}{debt} (§\ref{sec:NC.clusters}; see also \citealt{jacques14antipassive}). Another source is a reduced \forme{n-} allomorph of denominal \forme{nɯ-} (as in \japhug{ngo}{be ill} from the noun \japhug{tɯ-ŋgo}{disease}, §\ref{sec:denom.intr.nW}).


\subsubsection{Stop+\forme{ɕ} clusters}  \label{sec:kC.pC.clusters}
A handful of onsets have an unvoiced stop followed by \forme{ɕ}, including \forme{kɕ-} (as in \japhug{kɕilu}{year of the dog} from \tibet{ཁྱི་ལོ་}{kʰʲi.lo}{year of the dog}, \forme{pɕ\trt}, (\japhug{tɯpɕi}{flax}) as well as \forme{lpɕ-} (§\ref{sec:lC.clusters}) and \forme{mpɕ-} (§\ref{sec:NC.clusters}). These clusters, found in both native vocabulary and borrowings from Tibetan, mainly originate from the combination of aspirated stops \forme{pʰ} and \forme{kʰ} with the medial \forme{-j-}. 

Evidence that these groups should not be synchronically analyzed as \ipa{kʰj-} and \ipa{pʰj} underlyingly comes from two facts. First, the groups \forme{pʰj-} and \forme{kʰj-} are otherwise attested (§\ref{sec:Cj.clusters}). Second, the clusters \forme{(C)(p|k)ɕ-} are not affected by partial reduplication; for instance \japhug{mpɕɤr}{be beautiful} is reduplicated as \forme{mpɕɯ\redp{}mpɕɤr} (see for instance \ref{ex:anWtABzu.smWlAm}, §\ref{sec:smWlAm.TAME}), not $\dagger$\forme{mpʰɯ\redp{}mpɕɤr} as would have been expected if \forme{mpɕɤr} were really $\dagger$\ipa{mpʰjɤr}.\footnote{In addition, \japhug{mpɕɤr}{be beautiful} is probably borrowed from \tibet{མཆོར་པོ་}{mtɕʰor.po}{handsome}, so that the analysis as \ipa{mpʰj} is wrong even historically.  }

\subsection{Medials} \label{sec:medials}
This section deals with consonant clusters whose last consonant is one of the six non-nasal sonorants (\ipa{r}, \ipa{l}, \ipa{j}, \ipa{w}, \ipa{ɣ} or \ipa{ʁ}). These sonorants can either be medials (§\ref{sec:partial.redp}) or initials, depending on whether they are deleted when the cluster is partially reduplicated. They are always medials when the preceding consonant is not one of the possible preinitials (§\ref{sec:preinitials}). Clusters ending in a non-nasal sonorant whose penultimate element is a possible preinitial (sibilant/alveolo-palatal fricative or non-nasal sonorant)\footnote{Nasal sonorants  cannot directly precede non-nasal sonorants in syllable onset position, due to a series of sound changes (§\ref{sec:NC.clusters}), so they are not possible preinitials when the initial is a non-nasal sonorant.  } are referred to as \textit{ambiguous clusters}, as they need to be systematically tested with partial reduplication to determine the phonological status of the final sonorant (initial or medial).

The secondary clusters with medial glides created by the effect of synizesis (§\ref{sec:synizesis}) with the \textsc{1sg} suffix \forme{-a} (§\ref{sec:intr.1}) are not included in the listing below, even though what is transcribed as \forme{Ce|i-a} and \forme{Co|u-a} in the orthography is homophonous with \forme{Cja} and \forme{Cwa}, respectively. Including them would have the effect of introducing an enormous number of secondary and predictable clusters. In addition, synizesis of \forme{Cɯ-a} generates the \phonet{ɰ} allomorph of \ipa{ɯ}, constituting a seventh glide not found in the vocabulary.

\subsubsection{C+\ipa{w} clusters} \label{sec:Cw.clusters}
\tabref{med.w} lists all clusters ending in \ipa{w}. Clusters of this type are very rare, as a combination of two sound changes have removed all instances of proto-Gyalrong \forme{*Cw}. First, the labiovelars, still preserved in Zbu, have merged with plain velars (for instance \japhug{nɯŋa}{cow}, Zbu \forme{ŋwéʔ}, \citealt[40]{gong18these}), and the remaining instances of \forme{*w} have become \forme{ɣ} (as in \japhug{tɯ-ɕɣa}{tooth} from \forme{*-ɕwa}, §\ref{sec:CG.clusters}).


\begin{table}
	\caption{List of consonant clusters ending in \ipa{w} (10+0)} \label{med.w} 
	\begin{tabular}{Xlll}
		\lsptoprule
		%\ipa{p}   &    &    &   \\
		%\ipa{pʰ}   &    &    &   \\
		%\ipa{b}   &    &    &   \\
		%\ipa{mb}   &    &    &   \\
		%\ipa{m}   &    &    &   \\
		%\ipa{w}   &    &    &   \\
		%\ipa{t}   &    &    &   \\
		%\ipa{tʰ}   &    &    &   \\
		\ipa{d}   & \deux{dw}\idph{}   & \japhug{dwaŋdwaŋ}{out of his head} \\
		%\ipa{nd}   &    &    &   \\
		%\ipa{n}   &    &    &   \\
		%\ipa{ts}   &    &    &   \\
		%\ipa{tsʰ}   &    &    &   \\
		%\ipa{dz}   &    &    &   \\
		%\ipa{ndz}   &    &    &   \\
		%\ipa{s}   &    &    &   \\
		\ipa{z}   & \deux{zw}   & \japhug{zwɤr}{mugwort} \\
		\ipa{l}   & \deux{lw}   & \japhug{lwɤz}{become sick again} \\
		%\ipa{ɬ}   &    &    &   \\
		%\ipa{tɕ}   &    &    &   \\
		%\ipa{tɕʰ}   &    &    &   \\
		%\ipa{dʑ}   &    &    &   \\
		%\ipa{ndʑ}   &    &    &   \\
		%\ipa{ɕ}   &    &    &   \\
		%\ipa{ʑ}   &    &    &   \\
		%\ipa{tʂ}   &    &    &   \\
		%\ipa{tʂʰ}   &    &    &   \\
		%\ipa{dʐ}   &    &    &   \\
		%\ipa{ndʐ}   &    &    &   \\
		\ipa{r}   & \deux{rw}\tib{}   & \japhug{rwa}{yak felt tent} \\
		\ipa{ʂ}   & \deux{ʂw} \tib{}  & \japhug{aɣɯʂwaŋ}{correspond well} \\
		%\ipa{c}   &    &    &   \\
		%\ipa{cʰ}   &    &    &   \\
		%\ipa{ɟ}   &    &    &   \\
		%\ipa{ɲɟ}   &    &    &   \\
		%\ipa{ɲ}   &    &    &   \\
		\ipa{j}   & \deux{jw}   & \japhug{jwajwa}{very thin} \\
		\ipa{k}   & \deux{kw}\tib{}   & \japhug{kwitsɯt}{cupboard} \\
		%\ipa{kʰ}   &     &    &   \\
		%\ipa{g}   &    &    &   \\
		%\ipa{ŋg}   &    &    &   \\
		%\ipa{ŋ}   &    &    &   \\
		\ipa{x}   & \deux{xw}\idph{}   & \japhug{xwɤrnɤxwɤr}{rotating quickly} \\
		%\ipa{ɣ}   &    &    &   \\
		%\ipa{q}   &    &    &   \\
		%\ipa{qʰ}   &    &    &   \\
		%\ipa{ɴɢ}   &    &    &   \\
		\ipa{χ}   & \deux{χw} \tib{}  & \japhug{χwɤr}{Hor} (place name) \\
		%\ipa{ʁ}   &    &    &   \\
		\ipa{h}   & \deux{hw} \idph{}  & \japhug{hwɤrhwɤr}{wide-mouthed} \\
		\lspbottomrule
	\end{tabular}
\end{table}		
\resetcounters{2Cw}{3Cw}

There are no clusters with labial consonants followed by \forme{w}, but there is a triple contrast between \forme{hw\trt}, \forme{χw-} and \forme{xw-}: it is the only context where minimal pairs between the glottal \ipa{h}, the uvular \ipa{χ} and the velar \ipa{x} unvoiced fricatives can be found.

The combination of cluster-final \ipa{w} with the rhyme \forme{-a} is homophonous with the result of synizesis of the \textsc{1sg} \forme{-a} with verb stems in \forme{-o} and \forme{-u}; for instance, the \textsc{1sg} possessive \forme{a-rwa} `my tent'  is not distinguishable from \forme{aro-a} `I own' (§\ref{sec:synizesis}). Both are phonetically realized as \phonet{arwa}.

Clusters in \forme{w} are mainly found in ideophones and Chinese loanwords (including nativized ones such as \japhug{kwitsɯt}{cupboard} from \ch{柜子}{guìzǐ}{cupboard} with a mysterious coda \forme{-t}). The only native clusters with \forme{w} as final element are \forme{jw-} (as in \japhug{tɤ-jwaʁ}{leaf}) and \forme{zw-} (\japhug{zwɤr}{mugwort}), where the \forme{w} comes from the lenition of a voiced labial stop (\citealt[325--329]{jacques04these}).

In ambiguous clusters, cluster-final \forme{-w-} is a medial consonant in \ipa{zw} and \ipa{lw}, as shown by the distributed action derivation (§\ref{sec:distributed.action}) \forme{nɤ-zɯ\redp{}zwɤr} `burn everywhere' and \forme{nɤ-lɯ\redp{}lwoʁ} `spill everywhere' from \japhug{zwɤr}{burn} and \japhug{lwoʁ}{spill} (not $\dagger$\forme{nɤ-zwɯ\redp{}zwɤr} or $\dagger$\forme{nɤ-lwɯ\redp{}lwoʁ}). 

In \forme{jw}, the labial glide is initial, as it is not deleted in reduplication, for example in the comitative adverb \forme{kɤ́-jwɯ\redp{}jwaʁ} `together with (its) leaves' (§\ref{sec:comitative.adverb}) from \japhug{tɤ-jwaʁ}{leaf}.

The cluster \forme{rw} is intriguing, as it presents two alternative reduplication patterns. The labial glide is treated as initial in the distributed action derivation \forme{nɤ-rwɯ\redp{}rwɤt} `dig everywhere' from \japhug{rwɤt}{dig} (not $\dagger$\forme{nɤ-rɯ\redp{}rwɤt}), but in the comitative derivation, \japhug{rwa}{yak felt tent} can either be reduplicated as \forme{kɤ́-rɯ\redp{}rwa} or \forme{kɤ́-rwɯ\redp{}rwa} `together with the felt tent'.

\subsubsection{C+\ipa{j} clusters} \label{sec:Cj.clusters}
Tables \ref{med.j} and \ref{med.j.3} list all clusters ending in \ipa{j}. This glide can follow consonants of all places of articulation except palatals, alveolo-palatals and retroflex consonants (other than \ipa{r}). In the orthographical system used in this grammar and previous publications on Japhug, \ipa{j} is transcribed as <\forme{j}> after \ipa{ʁ}, \ipa{ɣ}, labial consonants and dental fricatives, and as  <\forme{i}> after velar and uvular stops, as well as dental stops and affricates and \ipa{l}.

\begin{table}
	\caption{List of biconsonant clusters ending in \ipa{j} (20)} \label{med.j} 
	\begin{tabular}{Xlll}
		\lsptoprule
		\ipa{p}   &    \deux{pj}   & \japhug{pjalu}{year of the cock} \\  
		%\ipa{pʰ}   &       &    & \\  
		\ipa{b}   &    \deux{bj}\idph{}   & \japhug{bjɯbjɯɣ}{hanging in great number} \\  
		\ipa{mb}   &    \deux{mbj}   & \japhug{mbjom}{be fast} \\  
		%\ipa{m}   &       &    & \\  
		\ipa{w}   &    \deux{wj}   & \japhug{tɕʰiβja}{duck} \\  
		%\ipa{t}   &       &    & \\  
		%\ipa{tʰ}   &       &    & \\  
		\ipa{d}   &    \deux{dj} \idph{}  & \japhug{dioʁdioʁ}{evenly mixed} \\  
		\ipa{nd}   &    \deux{ndj} \idph{}  & \japhug{ndiɤndiɤt}{gracious} \\  
		%\ipa{n}   &       &    & \\  
		\ipa{ts}   &     \deux{tsj}   & \japhug{tsiaŋnɤtsiaŋ}{very tall, moving} \\  
		%\ipa{tsʰ}   &       &    & \\  
		%\ipa{dz}   &       &    & \\  
		\ipa{ndz}   &    \deux{ndzj}   & \japhug{ndziaʁ}{be tight} (knot) \\  
		\ipa{s}   &    \deux{sj} \idph{}  & \japhug{sjaŋnɤsjaŋ}{shaking one's head} \\  
		\ipa{z}   &    \deux{zj} \idph{}  & \japhug{zjaŋzjaŋ}{big} \\  
		\ipa{l}   &    \deux{lj}   & \japhug{qaliaʁ}{eagle} \\  
		%\ipa{ɬ}   &       &    & \\  
		%\ipa{tɕ}   &       &    & \\  
		%\ipa{tɕʰ}   &       &    & \\  
		%\ipa{dʑ}   &       &    & \\  
		%\ipa{ndʑ}   &       &    & \\  
		%\ipa{ɕ}   &       &    & \\  
		%\ipa{ʑ}   &       &    & \\  
		%\ipa{tʂ}   &       &    & \\  
		%\ipa{tʂʰ}   &       &    & \\  
		%\ipa{dʐ}   &       &    & \\  
		%\ipa{ndʐ}   &       &    & \\  
		\ipa{r}   &    \deux{rj}   & \japhug{tɯ-rju}{word} \\  
		%\ipa{ʂ}   &       &    & \\  
		%\ipa{c}   &       &    & \\  
		%\ipa{cʰ}   &       &    & \\  
		%\ipa{ɟ}   &       &    & \\  
		%\ipa{ɲɟ}   &       &    & \\  
		%\ipa{ɲ}   &       &    & \\  
		%\ipa{j}   &       &    & \\  
		\ipa{k}   &    \deux{kj}   & \japhug{kio}{caused to glide} \\  
		\ipa{kʰ}   &       \deux{kʰj} \idph{} & \japhug{kʰɯkʰju}{oval} \\  
		%\ipa{g}   &       &    & \\  
		\ipa{ŋg}   &    \deux{ŋgj}   & \japhug{ŋgio}{slip}, `glide' \\  
		%\ipa{ŋ}   &       &    & \\  
		%\ipa{x}   &       &    & \\  
		\ipa{ɣ}   &    \deux{ɣj}   & \japhug{tɯ-ɣjɤn}{one time} \\  
		\ipa{q}   &    \deux{qj}   & \japhug{qiaβ}{be bitter} \\  
		\ipa{qʰ}   &    \deux{qʰj} \idph{}  & \japhug{qʰiɯqʰiɯ}{blunt (colour)} \\  
		\ipa{ɴɢ}   &    \deux{ɴɢj}   & \japhug{ɴɢia}{come loose} \\  
		%\ipa{χ}   &       &    & \\  
		\ipa{ʁ}   &    \deux{ʁj}   & \japhug{ʁjit}{think of}, `miss', `remember' \\ 
		%\ipa{h}   &       &    & \\  
		\lspbottomrule
	\end{tabular}
\end{table}		

In ambiguous clusters, cluster-final \ipa{j} is medial when following \ipa{r}, \ipa{l} and \ipa{s} (\ipa{rj\trt}, \ipa{lj-} and \ipa{sj-}), and initial when preceded by \ipa{w}, \ipa{ɣ} and \ipa{ʁ} (\ipa{wj\trt}, \ipa{ɣj-} and \ipa{ʁj-}).

There is in addition a highly unstable cluster \forme{ʑj-} when the alveolo-palatal allomorph of the translocative prefix (§\ref{sec:translocative.morpho}) precedes the `unspecified' orientation preverbs. As in the case of the other palatal preverbs, the dissimilated dental allomorph \forme{z-} is more commonly found (§\ref{sec:shC.clusters}, §\ref{sec:translocative.morpho}). There is no example with \forme{ʑj-} in the corpus, and all instances of this cluster come from elicitation of verb forms with the translocative.

Clusters with velar stops+\ipa{j}  and dental affricates+\ipa{j} are clearly distinctive with palatal stops (\japhug{kio}{cause to glide} vs. \japhug{co}{valley}, cf. \tabref{sec:consonant.phonemes}, §\ref{tab:coronal.dorsal}) and alveolo-palatal affricates (\japhug{ndziaʁ}{be tight} (of knot), `be deep' (of colour), `be completed' vs. \japhug{ndʑaʁ}{swim}). There are no perfect minimal pairs between alveolo-palatal affricates and dental stop+\ipa{j} clusters, as they are very rare and restricted to a handful of ideophones, but quasi-minimal pairs such as \japhug{ndiɤndiɤt}{gracious} and \japhug{ndʑɤβ}{be burned}(§\ref{sec:anticausative.morphology}) suffice to establish the reality of this phonological contrast.

Aspirated velar and labial stops followed by \ipa{j} are rare, since pre-Japhug \forme{*kʰj-} and \forme{*pʰj-} have changed to \forme{kɕ-} and \forme{pɕ\trt}, respectively (§\ref{sec:kC.pC.clusters}). This sound change, however, did not affect the cluster \forme{spʰj-} (\tabref{med.j.3}), and new instances of \ipa{kʰj-} have been created from interjections.

The combination of cluster-final \ipa{j} with the rhyme \forme{-a} is homophonous with the result of synizesis of the \textsc{1sg} \forme{-a} with verb stems in \forme{-e} and \forme{-i}.

\begin{table}
	\caption{List of triconsonantal clusters ending in \ipa{j} (18)} \label{med.j.3} 
	\begin{tabular}{Xlll}
		\lsptoprule
		&    \trois{wsj}    & \japhug{tɤ-fsjit}{whistle} \\ 
		&    \trois{wzj}  \tib{}   & \japhug{βzjoz}{learns} \\ 
		\hline
		&    \trois{spj}    & \japhug{spjaŋkɯ}{wolf} \\ 
		&    \trois{spʰj}    & \japhug{spʰjar}{spread out to dry} \\ 
		&    \trois{stj}  \idph{}   & \japhug{stiaŋnɤstiaŋ}{jumping} \\ 
		&    \trois{sqʰj}    & \japhug{sqʰiar}{stretches} \\ 
		\hline
		&    \trois{ltʰj}  \idph{}   & \japhug{ltʰiɤltʰiɤt}{well-ironed (clothes)} \\ 
		&    \trois{lbj} \idph{}   & \japhug{lbjɯlbjɯɣ}{hanging} \\ 
		\hline
		&    \trois{ʂpj}    & \japhug{rpjɯ}{spoil} (milk) \\ 
		&    \trois{rmbj}    & \japhug{tɤ-rmbja}{flash of lightning} \\ 
		&    \trois{ʂtsj}    & \japhug{rtsiaʁ}{be steep} (road) \\ 
		&    \trois{ʂqʰj}    & \japhug{tɤ-rqʰioʁ}{groove} \\ 
		&    \trois{rɴɢj}    & \japhug{arɤrɴɢioʁ}{be grooved} \\ 
		\hline
		&    \trois{χtsj}    & \japhug{χtsiɯ}{pint} \\ 
		&    \trois{χpj} \tib{}   & \japhug{χpjɤt}{observe} \\ 
		&    \trois{χsj}    & \japhug{ɯ-χsjɯβ}{slough} \\ 
		\hline
		&    \trois{mpj}    & \japhug{mpja}{be warm} \\ 
		&    \trois{mtsj}    & \japhug{tɤ-mtsioʁ}{beak} \\ 
		&    \trois{ɴqj}    & \japhug{ɴqiaβ}{dark side of the mountain} \\ 
		\lspbottomrule
	\end{tabular}
\end{table}		

\resetcounters{2Cj}{3Cj}

In triconsonantal groups, \ipa{j} can co-occur with all preinitials except velar and alveolo-palatal fricatives and itself (\tabref{med.j.3}).

A group \forme{ɕpj-} with  alveolo-palatal preinitial does exist, but it is not completely stable. It only occurs when the \forme{ɕ-} allomorph of the translocative prefix (§\ref{sec:translocative.morpho}) precedes the B-type \forme{pjɯ-} or the D-type \forme{pjɤ-} \textsc{downwards} preverbs (§\ref{sec:kamnyu.preverbs}), as in examples such as (\ref{ex:tWmWchW}) (§\ref{sec:approximate.locative}). It is in free variation with \forme{spj-} in this context, as the translocative has another \forme{s-} allomorph when preceding \forme{pj-} and \forme{cʰ-} (§\ref{sec:translocative.morpho}). For instance in (\ref{ex:spjWlAtnW}) we find \forme{s-pjɯ-lɤt-nɯ} instead of \forme{ɕ-pjɯ-lɤt-nɯ} (the more common form).

\begin{exe}
	\ex \label{ex:spjWlAtnW}
	\gll tɯsqar tu-ndo-nɯ tɕe, qrormbɯ ɲɯ-ɕar-nɯ tɕe nɯre s-pjɯ-lɤt-nɯ ŋgrɤl. \\
	tsampa \textsc{ipfv}-\textsc{ipfv}-take-\textsc{pl} \textsc{lnk} anthill \textsc{ipfv}-look.for-\textsc{pl} \textsc{lnk} \textsc{dem}:\textsc{loc} \textsc{tral}-\textsc{ipfv}:\textsc{down}-release-\textsc{pl} be.usually.the.case:\textsc{fact} \\
	\glt `(Faithful Buddhist people) take tsampa$_i$, look for an anthill$_j$ and spill it$_i$ there$_j$.' (26-qro, 74)
\end{exe}

\subsubsection{C+\ipa{l} clusters} \label{sec:Cl.clusters}
Tables \ref{med.l} and \ref{med.l.3} list all clusters ending in \ipa{l}. The lateral sonorant cannot follow retroflex affricates, dental stops and alveolo-palatal affricates.

\begin{table}
	\caption{List of biconsonant clusters ending in \ipa{l} (18)} \label{med.l}  
	\begin{tabular}{Xlll}
		\lsptoprule
		\ipa{p}  &  \deux{pl}  & \japhug{plɯt}{destroy} \\ 
		%\ipa{pʰ}  &    &    &    \\ 
		%\ipa{b}  &    &    &    \\ 
		\ipa{mb}  &  \deux{mbl}  & \japhug{mblɯt}{be destroyed} \\ 
		%\ipa{m}  &    &    &    \\ 
		\ipa{w}  &  \deux{wl}  & \japhug{βlɯ}{burn} \\ 
		%\ipa{t}  &    &    &    \\ 
		%\ipa{tʰ}  &    &    &    \\ 
		%\ipa{d}  &    &    &    \\ 
		%\ipa{nd}  &    &    &    \\ 
		%\ipa{n}  &    &    &    \\ 
		\ipa{ts}   &  \deux{tsl}\idph{}  & \japhug{tslɯɣtslɯɣ}{completely wrapped up} \\ 
		%\ipa{tsʰ}  &    &    &    \\ 
		%\ipa{dz}  &    &    &    \\ 
		%\ipa{ndz}  &    &    &    \\ 
		\ipa{s}  &  \deux{sl}  & \japhug{sloʁ}{dig} (with snout) \\ 
		\ipa{z}  &  \deux{zl} \tib{}  & \japhug{tɯ-zloʁ}{one time} \\ 
		%\ipa{l}  &    &    &    \\ 
		%\ipa{ɬ}  &    &    &    \\ 
		%\ipa{tɕ}  &    &    &    \\ 
		%\ipa{tɕʰ}  &    &    &    \\ 
		%\ipa{dʑ}  &    &    &    \\ 
		%\ipa{ndʑ}  &    &    &    \\ 
		\ipa{ɕ}  &  \deux{ɕl}  & \japhug{ɕlu}{plough} \\ 
		\ipa{ʑ}  & \deux{ʑ}   & \forme{ʑ-lo-ru} `s/he went and looked upstream'   \\ 
		%\ipa{tʂ}  &    &    &    \\ 
		%\ipa{tʂʰ}  &    &    &    \\ 
		%\ipa{dʐ}  &    &    &    \\ 
		%\ipa{ndʐ}  &    &    &    \\ 
		\ipa{r}  &  \deux{rl}  & \japhug{rlaʁ}{disappear} \\ 
		%\ipa{ʂ}  &    &    &    \\ 
		\ipa{c}  &  \deux{cl} \idph{} & \japhug{claŋclaŋ}{round and smooth} \\ 
		%\ipa{cʰ}  &    &    &    \\ 
		%\ipa{ɟ}  &    &    &    \\ 
		%\ipa{ɲɟ}  &    &    &    \\ 
		%\ipa{ɲ}  &    &    &    \\ 
		\ipa{j}  &  \deux{jl}  & \japhug{jla}{hybrid yak} \\ 
		\ipa{k}  &  \deux{kl}  & \japhug{klɯklɯɣ}{stiff} \\ 
		%\ipa{kʰ}  &    &    &    \\ 
		\ipa{g}  &  \deux{gl} \idph{} & \japhug{glɤɣglɤɣ}{pressed} \\ 
		\ipa{ŋg}  &  \deux{ŋgl}  & \japhug{cɯŋglɯɣ}{pestle} \\ 
		%\ipa{ŋ}  &    &    &    \\ 
		%\ipa{x}  &    &    &    \\ 
		\ipa{ɣ}  &  \deux{ɣl}  & \japhug{ɣle}{knead}, `rub' \\ 
		\ipa{q}  &  \deux{ql}  & \japhug{qlɯt}{break} (vt) \\ 
		\ipa{qʰ}  &  \deux{qʰl} \tib{} & \japhug{qʰlɯ}{naga} \\ 
		\ipa{ɴɢ}  &  \deux{ɴɢl}  & \japhug{ɴɢlɯt}{break} (vi) \\ 
		%\ipa{χ}  &    &    &    \\ 
		\ipa{ʁ}  &  \deux{ʁl}  & \japhug{tɯ-ʁla}{forearm} \\ 
		\lspbottomrule
	\end{tabular}
\end{table} 

The cluster \forme{ʑl-} is only attested when the translocative prefix occurs before \textsc{upstream} preverbs (§\ref{sec:translocative.morpho}).

In the clusters  with the fricative \ipa{s} or a sonorant as first element (\ipa{sl}, \ipa{wl}, \ipa{jl}, \ipa{rl}, \ipa{ɣl} and \ipa{ʁl}), \ipa{l} is initial. For instance, the distributed action derivation (§\ref{sec:distributed.action}) of \japhug{sloʁ}{dig} and \japhug{ɣle}{rub} are \forme{nɤ-slɯ\redp{}sloʁ} `dig everywhere with snout' \forme{nɤ-ɣlɯ\redp{}ɣle} `rub again and again' (not $\dagger$\forme{nɤ-sɯ\redp{}sloʁ} and $\dagger$\forme{nɤ-ɣɯ\redp{}ɣle}), and the comitative adverb (§\ref{sec:comitative.adverb}) from \japhug{jla}{male hybrid yak} is \forme{kɤ́-jlɯ\redp{}jla} `together with his/her/their hybrid yak(s)' (not $\dagger$\forme{kɤ́-jɯ\redp{}jla}).


On the other hand, \ipa{l} is medial in \ipa{ɕl}, as shown by the fact that the verb \japhug{ɕlu}{plough} has the distributed action derivation \forme{nɤ-ɕɯ\redp{}ɕlu} `plough everywhere'  (not $\dagger$\forme{nɤ-ɕlɯ\redp{}ɕlu} as would have been expected is \ipa{l} were initial).\footnote{In \citet[25:59]{jacques04these}, I claimed that \ipa{ɕl} could be reduplicated either with or without deletion of \ipa{l}, but I have not been able to confirm my earlier data and it may have been an error. } The groups \ipa{zl} and \ipa{ʑl} are not found in words that can be subjected to partial reduplication, and their status is undecidable at the present moment.


\begin{table}
	\caption{List of triconsonantal clusters ending in  \ipa{l} (12)} \label{med.l.3}  
	\begin{tabular}{Xlll}
		\lsptoprule
		& \trois{scl}  \idph{} & \japhug{sclaŋsclaŋ}{bald} \\ 
		& \trois{sql}   & \japhug{sqlɯm}{collapse} \\ 
		& \trois{sqʰl}   & \japhug{asqʰlu}{be concave} \\ 
		\hline 
		& \trois{ɕpl} & \japhug{ɕploʁɕploʁ}{round and smooth} \\ 
		& \trois{ɕkl} & \japhug{ɕkliɕkli}{round and stiff} \\ 
		& \trois{ɕql}  \idph{} & \japhug{ɕqlɯβnɤɕqlɯβ}{walking in the water} \\ 
		& \trois{ɕqʰl}   & \japhug{ɕqʰlɤt}{disappear} \\ 
		\hline 
		&  \trois{rɴɢl}   & \japhug{arɴɢlɯm}{be caved in} \\ 
		\hline 
		& \trois{χpl} \idph{} & \japhug{χploʁχploʁ}{round like a ball} \\ 
		& \trois{ʁɲɟl}  \idph{} & \japhug{ʁɲɟliʁɲɟli}{big and tall} \\ 
		\hline 
		& \trois{mql}   & \japhug{mqlaʁ}{swallow} \\ 
		& \trois{mɢl}   & \japhug{tɯ-mɢla}{one step} \\ 
		\lspbottomrule
	\end{tabular}
\end{table}		
\resetcounters{2Cl}{3Cl}

In triconsonantal clusters (\tabref{med.l.3}), the medial \ipa{l} co-occurs with dental, alveolo-palatal and uvular fricatives, as well as \ipa{r} and \ipa{m}.  The only word with a preinitial \ipa{r} and a medial \forme{l}, \japhug{arɴɢlɯm}{be caved in}, derives from \japhug{sqlɯm}{collapse} by prenasalization alternation (§\ref{sec:fossil.prenasalization}): the \forme{r-} originates here from rhotacism of \forme{s-} (through \forme{*z}).

\subsubsection{C+\ipa{r} clusters} \label{sec:Cr.clusters}
The Tables \ref{med.r2}, \ref{med.r3} and \ref{med.r3.n} list all clusters ending in \ipa{r}. The rhotic sonorant can follow all places of articulations except retroflex fricatives and affricates.

\begin{table}
	\caption{List of consonant clusters with two elements ending in  \ipa{r} (26)} \label{med.r2}  
	\begin{tabular}{Xlll}
		\lsptoprule
		\ipa{p} &  \deux{pr} & \japhug{pri}{bear} \\ 
		\ipa{pʰ} &  \deux{pʰr} & \japhug{kʰɤpʰrɯ}{spraying water with the mouth} \\ 
		\ipa{b} &  \deux{br} \idph{} & \japhug{brɯbrɯz}{having pimples} \\ 
		\ipa{mb} &  \deux{mbr} & \japhug{mbrɤt}{break} (vi) \\ 
		%\ipa{m} &  & & \\ 
		\ipa{w} &  \deux{wr} & \japhug{βraʁ}{attach} \\ 
		%\ipa{t} &  & & \\ 
		%\ipa{tʰ} &  & & \\ 
		\ipa{d} &  \deux{dr} \idph{} & \japhug{droŋdroŋ}{big and dirty} \\ 
		%\ipa{n} &  & & \\ 
		\ipa{ts} &  \deux{tsr} & \japhug{tsri}{be salty} \\ 
		%\ipa{tsʰ} &  & & \\ 
		%\ipa{dz} &  & & \\ 
		\ipa{ndz} &  \deux{ndzr} & \japhug{ndzri}{wring} \\ 
		\ipa{s} &  \deux{sr} & \japhug{srɯn}{cotton} \\ 
		\ipa{z} &  \deux{zr} & \japhug{zrɯ}{sunny side of the mountain} \\ 
		%\ipa{l} &  & & \\ 
		%\ipa{ɬ} &  & & \\ 
		\ipa{tɕ} &  \deux{tɕr} \idph{} & \japhug{tɕrɯɣnɤtɕrɯɣ}{crunching} \\ 
		%\ipa{tɕʰ} &  & & \\ 
		%\ipa{dʑ} &  & & \\ 
		%\ipa{ndʑ} &  & & \\ 
		\ipa{ɕ} &  \deux{ɕr} & \japhug{ɕri}{leak} \\ 
		\ipa{ʑ} &  \deux{ʑr} & \japhug{ʑru}{be strong} \\ 
		%\ipa{tʂ} &  & & \\ 
		%\ipa{tʂʰ} &  & & \\ 
		%\ipa{dʐ} &  & & \\ 
		%\ipa{ndʐ} &  & & \\ 
		%\ipa{r} &  & & \\ 
		%\ipa{ʂ} &  & & \\ 
		\ipa{c} &  \deux{cr} \idph{} & \japhug{crɯɣcrɯɣ}{in a mess} \\ 
		\ipa{cʰ} &  \deux{cʰr}\idph{} & \japhug{cʰrɤβcʰrɤβ}{messy and dirty} \\ 
		\ipa{ɟ} &  \deux{ɟr} \idph{} & \japhug{ɟrɯɣɟrɯɣ}{gurgling} \\ 
		%\ipa{ɲɟ} &  & & \\ 
		%\ipa{ɲ} &  & & \\ 
		\ipa{j} &  \deux{jr} & \japhug{tɤ-jroʁ}{trace} \\ 
		\ipa{k} &  \deux{kr} & \japhug{krɤɣ}{cut}, `shear', `mow' \\ 
		\ipa{kʰ} &  \deux{kʰr} & \japhug{kʰro}{much} \\ 
		\ipa{g} &  \deux{gr} & \japhug{grɯβgrɯβ}{matsutake} \\ 
		\ipa{ŋg} &  \deux{ŋgr} & \japhug{ŋgrɤl}{be usually the case} \\ 
		%\ipa{ŋ} &  & & \\ 
		%\ipa{x} &  & & \\ 
		\ipa{ɣ} &  \deux{ɣr} & \japhug{ɣro}{suffocate} \\ 
		\ipa{q} &  \deux{qr} & \japhug{qro}{pigeon} \\ 
		%\ipa{qʰ} &  & & \\ 
		\ipa{ɴɢ} &  \deux{ɴɢr} & \japhug{ɴɢraʁ}{be torn} \\ 
		%\ipa{χ} &  & & \\ 
		\ipa{ʁ} &  \deux{ʁr} & \japhug{ʁrɯlu}{without horns} \\ 
		\lspbottomrule
	\end{tabular}
\end{table}

In the ambiguous clusters with a sonorant as first element (\ipa{wr}, \ipa{jr}, \ipa{ɣr} and \ipa{ʁr}), \ipa{r} is initial, as shown by the perlative \forme{ɯ-jrɯ\redp{}jroʁ} `following X's trace' (§\ref{sec:perlative}), the emphatic participle \forme{kɯ-wɣrɯ\redp{}wɣrum} `very white (one)' (\ref{ex:khWlu}, §\ref{sec:possessive.n.n}) and the reciprocal \forme{a-zɣɤʁrɯ\redp{}ʁre} `respect each other' from \japhug{zɣɤʁre}{respect}. 


\begin{table}
	\caption{List of consonant clusters with three elements ending in \ipa{r} with a non-nasal preinitial (32)} \label{med.r3}  
	\begin{tabular}{Xlll}
		\lsptoprule
		& \trois{wkr} \tib{} & \japhug{fkrɯz}{be greedy} \\ 
		& \trois{wɣr} & \japhug{wɣrum}{be white} \\ 
		& \trois{wsr} & \japhug{fsraŋ}{he protects} \\ 
		\hline
		& \trois{spr} & \japhug{sprɯskɯ}{reincarnated} \\ 
		& \trois{zbr} \tib{} & \japhug{zbrilu}{year of the snake} \\ 
		& \trois{zmbr} & \japhug{sɤzmbrɯ}{make angry} \\ 
		& \trois{stʰr} \idph{} & \japhug{stʰrɯβ}{dangling (of snot)} \\ 
		& \trois{scr} \idph{} & \japhug{scraʁscraʁ}{very small} \\
		& \trois{zɟr} \idph{} & \japhug{zɟraŋzɟraŋ}{soft and bloated} \\ 
		& \trois{skr} & \japhug{skraskra}{impolite} \\ 
		& \trois{skʰr} & \japhug{tɯ-skʰrɯ}{body} \\ 
		& \trois{zgr} \tib{} & \japhug{zgrawa}{leather sack} \\ 
		& \trois{sqr} & \japhug{sɤsqra}{limit} \\ 
		\hline
		& \trois{ɕpr} & \japhug{aɕprɯm}{be badly sewed} \\ 
		& \trois{ʑmbr} & \japhug{ʑmbri}{willow} \\ 
		& \trois{ɕtr} \idph{} & \japhug{ɕtraŋɕtraŋ}{long and soft} \\ 
		& \trois{ʑdr} \idph{} & \japhug{ʑdraŋʑdraŋ}{long and soft} \\ 
		& \trois{ɕkr} & \japhug{ɕkrɤz}{oak} \\ 
		& \trois{ʑgr} & \japhug{ʑgrɯɣ}{certainly} \\ 
		& \trois{ʑŋgr} & \japhug{ʑŋgri}{star} \\ 
		& \trois{ɕqr} & \japhug{ɕqraʁ}{be intelligent} \\ 
		&  \trois{ʑɴɢr} & \japhug{ʑɴɢro}{Jew's harp} \\ 
		\hline
		& \trois{jkr} & \japhug{jkrɯt}{congeal} \\ 
		& \trois{jtsr} & \japhug{jtsraβ}{delay departure} \\ 
		\hline 
		&  \trois{xpr} & \japhug{ɣɤxpra}{dispatch} \\ 
		\hline
		& \trois{χpr} & \japhug{tɕʰɯχpri}{newt} \\ 
		& \trois{ʁmbr} & \forme{taʁmbra} `jump' (of horse) \\ 
		& \trois{χsr} & \japhug{ɣɤχsrɯ}{be handsome} \\ 
		& \trois{ʁzr} & \japhug{ʁzraŋʁzraŋ}{dishevelled} \\ 
		& \trois{χcr} \idph{} & \japhug{χcɯχcri}{thin, diluted} \\ 
		& \trois{ʁɟr} \idph{} & \japhug{ʁɟɯʁɟri}{fat and soft} \\ 
		& \trois{ʁgr} \tib{} & \japhug{ʁgra}{enemy} \\ 
		\lspbottomrule
	\end{tabular}
\end{table}		

By contrast, in ambiguous clusters with a coronal fricative as first element (\ipa{sr}, \ipa{zr}, \ipa{ɕr}, \ipa{ʑr}), \ipa{r} is medial, as illustrated by the emphatic participles \forme{kɯ-zɯ\redp{}zri} `very long (one)' (in \ref{ex:Wmi.kWzWzri}, §\ref{sec:S.possessor.relativization}), \forme{kɯ-ʑɯ\redp{}ʑru} `very strong (one)' and \forme{kɯ-ɣɤχsɯ\redp{}χsrɯ} `very handsome (one)' (\ref{ex:kWZWZru.kWGAXsWXsrW}).\footnote{Reduplication of the simple \ipa{sr} cluster in the comitative adverb \forme{kɤ́-sɯ\redp{}srɯn} `together with (its) cotton wool' (§\ref{sec:comitative.adverb}) confirms that \ipa{r} is medial in \ipa{sr}, not only in \ipa{χsr}. } Tshendzin rejects reduplication of these clusters without deletion: $\dagger$\forme{kɯ-zrɯ\redp{}zri} and $\dagger$\forme{kɯ-ʑrɯ\redp{}ʑru} are not grammatical.


\begin{exe}
	\ex \label{ex:kWZWZru.kWGAXsWXsrW}
	\gll ʁʑɯnɯ kɯ-ʑɯ\redp{}ʑru ʑo kɯ-ɣɤχsɯ\redp{}χsrɯ ʑo ɲɤ-k-ɤβzu-ci \\
	young.man \textsc{sbj}:\textsc{pcp}-\textsc{emph}\redp{}strong \textsc{emph} \textsc{sbj}:\textsc{pcp}-\textsc{emph}\redp{}handsome \textsc{emph} \textsc{ifr}-\textsc{peg}-become-\textsc{peg} \\
	\glt `(The fox) became a strong and handsome young man.' (2012 qachGa, 187)
\end{exe}


The \ipa{w} preinitial in the group \forme{wɣr-} is not realized as a separate segment (§\ref{sec:wC.clusters}); it turns preceding back unrounded vowels \ipa{ɯ} and \ipa{ɤ} into their  rounded counterparts. For instance, \forme{kɯ-wɣrɯ\redp{}wɣrum} `very white (one)' is pronounced \phonet{kuɣruɣrum}.

An extra-short \textit{svarabhakti} vowel can be heard in some clusters with \forme{-r-} medial, in particular after coronal affricates and dorsal stops. For instance, \forme{cʰɤ-kro} `s/he distributed it' can be realized as \phonet{cʰɤkə̌ro}.

The retroflex affricates originate at least in part from clusters of dental stops followed by \forme{r}; direct evidence for the sound change \forme{*tr-} \fl{} \forme{tʂ-} comes from alternations between \ipa{tʂ} and \ipa{r}, as that between the numeral \japhug{kɯtʂɤɣ}{six} and \japhug{sqaprɤɣ}{sixteen} (§\ref{sec:teens}) and  between \japhug{tʂu}{path} and \japhug{ftɕɤru}{path in the middle of the fields} is a compound of \japhug{ftɕar}{summer} and \japhug{tʂu}{path}
(§\ref{sec:second.member.alternation}). The gap in the system caused by this sound change has been filled by some ideophones in \forme{dr-} (§\ref{sec:idph.onsets}); the only non-ideophone with a cluster of this type is \japhug{qɯmdroŋ}{crane}, whose onset can either be analyzed as \ipa{dr}  or as \ipa{ndr}, as the contrast between \ipa{d} and \ipa{nd} is neutralized due to assimilation with the \ipa{m} coda of the preceding syllable.

\begin{table}
	\caption{List of consonant clusters with three elements ending in \ipa{r} with a nasal preinitial (9)} \label{med.r3.n}  
	\begin{tabular}{Xlll}
		\lsptoprule
		& \trois{ɲcr} \idph{} & \japhug{ɲcɯɲcri}{thin, diluted} \\ 
		& \trois{ŋkʰr} & \japhug{ŋkʰrɯli}{screw} \\ 
		& \trois{ngr} & \japhug{ngrɯβ}{accomplish} \\
		& \trois{ɴqr} & \japhug{ɯ-ɴqra}{shabby} \\ 
		\hline
		& \trois{mtsr} & \japhug{mɯmtsrɯɣ}{drink with a straw} \\ 
		& \trois{mpʰr} & \japhug{mpʰrɯmɯ}{divination} \\ 
		& \trois{mkʰr} & \japhug{mkʰroŋ}{be reincarnated} \\ 
		& \trois{mgr} & \japhug{mgrɯn}{treat}, `invite' \\ 
		\hline
		&\trois{nbr} & \japhug{nbraʁ}{hoe} (vt) \\ 
		\lspbottomrule
	\end{tabular}
\end{table}		
\resetcounters{2Cr}{3Cr}


In triconsonantal clusters (Tables \ref{med.r3} and \ref{med.r3.n}), the medial \ipa{r} is compatible with all preinitials except itself.


\subsubsection{C+\ipa{ɣ} clusters} \label{sec:CG.clusters}
The Tables \ref{med.G2} and \ref{med.G3} list all clusters ending in \ipa{ɣ}. The velar fricative can follow all places of articulations except retroflex affricates, velars and uvulars.

\begin{table}
	\caption{List of consonant clusters with two elements ending in \ipa{ɣ} (25)} \label{med.G2}  \centering
	\begin{tabular}{Xllll}
		\lsptoprule
		\ipa{p}    &    \deux{pɣ}    & \japhug{pɣa}{bird} \\ 
		\ipa{pʰ}    &    \deux{pʰɣ}    & \japhug{pʰɣo}{flee} \\ 
		\ipa{b}    &   \deux{bɣ}     &    \forme{sɯbɣi} `species of bush'      \\ 
		\ipa{mb}    &    \deux{mbɣ}    & \japhug{mbɣaʁ}{turn over} (vi)\\ 
		%\ipa{m}    &        &        &      \\ 
		\ipa{w}    &    \deux{wɣ}    & \japhug{βɣa}{mill} \\ 
		\ipa{t}    &    \deux{tɣ}    & \japhug{tɯ-tɣa}{one span} \\ 
		\ipa{tʰ}    &    \deux{tʰɣ}    & \japhug{tʰɣe}{acorn} \\ 
		\ipa{d}    &    \deux{dɣ}  \idph{}  & \japhug{dɣɤrdɣɤr}{dumb} \\ 
		\ipa{nd}    &    \deux{ndɣ}  \idph{}  & \japhug{ndɣɤndɣɤt}{shaking} \\ 
		%\ipa{n}    &        &        &      \\ 
		\ipa{ts}    &    \deux{tsɣ}    & \japhug{tsɣi}{rot} \\ 
		%\ipa{tsʰ}    &        &        &      \\ 
		%\ipa{dz}    &        &        &      \\ 
		\ipa{ndz}    &    \deux{ndzɣ}    & \japhug{tɯ-ndzɣi}{fang} \\ 
		\ipa{s}    &    \deux{sɣ}    & \japhug{sɣa}{rust} \\ 
		\ipa{z}    &    \deux{zɣ}    & \japhug{zɣɯt}{reach} \\ 
		\ipa{l}    &    \deux{lɣ}    & \japhug{lɣa}{dig} \\ 
		%\ipa{ɬ}    &        &        &      \\ 
		\ipa{tɕ}    &    \deux{tɕɣ}    & \japhug{tɕɣaʁ}{squeeze out} \\ 
		\ipa{tɕʰ}    &    \deux{tɕʰɣ} \idph{}   & \japhug{tɕʰɣaʁtɕʰɣaʁ}{completely} \\ 
		%\ipa{dʑ}    &        &        &      \\ 
		\ipa{ndʑ}    &    \deux{ndʑɣ}    & \japhug{ndʑɣaʁ}{be squeezed out} \\ 
		\ipa{ɕ}    &    \deux{ɕɣ}    & \japhug{tɯ-ɕɣa}{tooth} \\ 
		\ipa{ʑ}    &    \deux{ʑɣ}    & \forme{ʑɣɤ-} reflexive prefix \\ 
		%\ipa{tʂ}    &        &        &      \\ 
		%\ipa{tʂʰ}    &        &        &      \\ 
		%\ipa{dʐ}    &        &        &      \\ 
		%\ipa{ndʐ}    &        &        &      \\ 
		\ipa{r}    &    \deux{rɣ}    & \japhug{tɯ-rɣi}{seed} \\ 
		\ipa{ʂ}    &    \deux{ʂɣ} \idph{}   & \japhug{ʂɣɤlʂɣɤl}{transparent and round} \\ 
		%\ipa{c}    &        &        &      \\ 
		\ipa{cʰ}    &    \deux{cʰɣ}    & \japhug{qacʰɣa}{fox} \\ 
		%\ipa{ɟ}    &        &        &      \\ 
		\ipa{ɲɟ}    &    \deux{ɲɟɣ} \idph{}   & \japhug{ɲɟɣɤrɲɟɣɤr}{plump and huge} \\ 
		%\ipa{ɲ}    &        &        &      \\ 
		\ipa{j}    &    \deux{jɣ}    & \japhug{jɣɤt}{come back} \\ 
		\lspbottomrule
	\end{tabular}
\end{table}		

In ambiguous clusters, the \ipa{ɣ} is initial in \ipa{zɣ} and when preceded by a non-nasal sonorant (\ipa{wɣ}, \ipa{jɣ}, \ipa{lɣ} and \ipa{rɣ}, as illustrated by the distributed action 
derivation (§\ref{sec:distributed.action}) \forme{nɤ-lɣɯ\redp{}lɣa} `dig everywhere' from \japhug{lɣa}{dig} (not $\dagger$\forme{nɤ-lɯ\redp{}lɣa}). %\ref{ex:kACArGWrGi} in §\ref{sec:partial.redp})?

The velar sonorant is medial in \ipa{ɕɣ}, as shown by the comitative adverb (§\ref{sec:comitative.adverb}) \forme{kɤ́-ɕɯ\redp{}ɕɣa} `together with its teeth' (not $\dagger$\forme{kɤ́-ɕɣɯ\redp{}ɕɣa}) from \japhug{tɯ-ɕɣa}{tooth}.\footnote{In \citet[59]{jacques04these}, I claimed that \ipa{ɕɣ} could be reduplicated either with or without deletion of \ipa{ɣ}, but in my more recent data only the variant with deletion is attested. } The group \ipa{ʑɣ}  cannot be tested with partial reduplication.


\begin{table}
	\caption{List of consonant clusters with three elements ending in \ipa{ɣ} (21)} \label{med.G3} 
	\begin{tabular}{Xlll}
		\lsptoprule
		&  \trois{spɣ}    & \japhug{spɣi}{storehouse} \\ 
		&  \trois{zbɣ}    & \japhug{tɤkɤzbɣaʁ}{headache} \\ 
		&  \trois{stɣ}    & \japhug{stɣɤrnɤstɣɤr}{jumping} \\ 
		\hline
		&  \trois{lcʰɣ} \idph{}   & \japhug{lcʰɣaʁlcʰɣaʁ}{nice to wear} \\ 
		&  \trois{ldzɣ}    & \japhug{stoʁldzɣɤm}{straw from broad beans} \\ 
		\hline
		&  \deux{ɕpɣ}   & \japhug{ɕpɣo}{unit of measure} \\ 
		&  \trois{ɕpʰɣ}    &  \japhug{ɕpʰɣo}{flee with}     \\ 
		\hline
		&  \trois{jmbɣ}    & \japhug{nɤjmbɣom}{have vertigo} \\ 
		&  \trois{jpɣ}    & \japhug{jpɣom}{freeze} \\ 
		&\trois{jndɣ} & \japhug{nɯjndɣo}{echo} (vi) \\  
		\hline
		&  \trois{rmbɣ}    & \japhug{tɤ-rmbɣo}{drum} \\ 
		&  \trois{rpɣ}    & \japhug{rpɣo}{up on the mountain} \\ 
		&  \trois{rpʰɣ}    & \japhug{ɯ-rpʰɣɤt}{upper door frame} \\ 		
		\hline
		&  \trois{χpɣ}    & \japhug{tɯ-χpɣi}{thigh} \\ 
		&  \trois{ʁmbɣ}    & \japhug{ʁmbɣi}{sun} \\ 
		\hline
		&  \trois{mpʰɣ}\idph{}     & \japhug{mpʰɣaʁmpʰɣaʁ}{very tight} \\ 
		&  \trois{ntɕʰɣ}    & \japhug{ntɕʰɣaʁ}{splash} \\ 
		&  \trois{ntʰɣ}    & \japhug{antʰɣar}{bounce} \\ 
		&  \trois{ntsɣ}    & \japhug{ntsɣe}{sell} \\ 
		&  \trois{ntsʰɣ}    & \japhug{nɤntsʰɣɤz}{bump into} \\ 
		&  \trois{ɲcɣ}\idph{}    & \japhug{ɲcɣɤɲcɣɤt}{many people, very noisy} \\ 
		&  \trois{ɲcʰɣ}    & \japhug{ɲcʰɣaʁ}{birchbark} \\ 
		\lspbottomrule
	\end{tabular}
\end{table}		
\resetcounters{2CG}{3CG}

Cluster-final \forme{ɣ} is secondary, and has at least three proto-Gyalrong origins. 


First, it comes from proto-Gyalrong medial \forme{*-w\trt},  as in \japhug{lɣa}{dig} (Situ \forme{rwâ}, \citealt{huangsun02}) or \japhug{tɯ-tɣa}{one span} (Situ \forme{tə-təwá}; the Tibetan cognate \tibet{མཐོ་}{mtʰo}{span} underwent Laufer's law, \citealt{jacques09wazur, hill11laws}). Most \forme{*-w-} medials in the inherited vocabulary have undergone this sound change (except after dorsals, where they have disappeared), and the present \forme{w} medials are secondary (§\ref{sec:Cw.clusters}).  

Second, it originates from the lenition of velar stops in clusters with two stops in proto-Gyalrong, as in \japhug{pɣa}{bird} (from \forme{*pk\trt}, see Cogtse Situ \forme{pká}, \citealt{huangsun02}).

Third, it is a secondary trace of velarized vowels (\citealt[231]{jacques04these}), as in \japhug{tɤjpɣom}{ice} (from \forme{*tɐ-lpaˠŋ}, see Zbu \forme{tɑlvɑ́mʔ}, \citealt[13]{gong18these}).

In triconsonantal clusters (\tabref{med.G3}),  medial \forme{ɣ} can occur with all preinitials except velar fricatives and the labial \forme{w-} and \forme{m\trt}, a further clue of its diachronic origin from \forme{*w}. Combinations \ipa{-w.Cɣ-} are possible, however, in heterosyllabic clusters (§\ref{sec:heterosyllabic.clusters}), as in the noun \japhug{laβzɣi}{steamed turnip}.

\subsubsection{C+\ipa{ʁ} clusters} \label{sec:CRR.clusters}
Clusters with \forme{ʁ} as last element are listed in \tabref{med.R}. The uvular fricative is medial only in a handful of clusters with dental or alveolo-palatal affricates, where it contrasts with \forme{ɣ}, as shown by the minimal pair \japhug{tɯ-ndzʁi}{collar bone} vs. \japhug{tɯ-ndzɣi}{fang}.

\begin{table}
	\caption{List of consonant clusters ending in \ipa{ʁ} (8)} \label{med.R} 
	\begin{tabular}{Xlll}
		\lsptoprule
		%\ipa{p}    &         &       &   \\ 
		%\ipa{pʰ}    &       &    &   \\ 
		%\ipa{b}    &       &    &   \\ 
		%\ipa{mb}    &       &    &   \\ 
		%\ipa{m}    &       &    &   \\ 
		\ipa{w}    &      \deux{wʁ}    & \japhug{βʁa}{prevail} \\ 
		%\ipa{t}    &       &    &   \\ 
		%\ipa{tʰ}    &       &    &   \\ 
		%\ipa{d}    &       &    &   \\ 
		%\ipa{nd}    &       &    &   \\ 
		%\ipa{n}    &       &    &   \\ 
		%\ipa{ts}    &        &      &     
		%\ipa{tsʰ}    &       &    &   \\ 
		%\ipa{dz}    &       &    &   \\ 
		\ipa{ndz}    &     \deux{ndzʁ}    &    \japhug{tɯ-ndzʁi}{collar bone} \\
		%\ipa{s}    &       &    &   \\ 
		\ipa{z}    &      \deux{zʁ}    & \japhug{zʁɤɲcɯ}{sling} \\ 
		\ipa{l}    &     \deux{lʁ}   & \japhug{lʁa}{gunny bag} \\ 
		%\ipa{ɬ}    &       &    &   \\ 
		\ipa{tɕ}    &     \deux{tɕʁ}  \idph{}  &    \japhug{tɕʁɯznɤtɕʁɯz}{crunchy} \\
		\ipa{tɕʰ}    &     \deux{tɕʰʁ}  \idph{}  &    \japhug{tɕʰʁɯznɤtɕʰʁɯz}{crunchy} \\
		%\ipa{dʑ}    &       &    &   \\ 
		%\ipa{ndʑ}    &       &    &   \\ 
		%\ipa{ɕ}    &       &    &   \\ 
		%\ipa{ʑ}    &       &    &   \\ 
		%\ipa{tʂ}    &       &    &   \\ 
		%\ipa{tʂʰ}    &       &    &   \\ 
		%\ipa{dʐ}    &       &    &   \\ 
		%\ipa{ndʐ}    &       &    &   \\ 
		\ipa{r}    &     \deux{rʁ}  & \japhug{rʁom}{be rough} \\ 
		%\ipa{ʂ}    &       &    &   \\ 
		%\ipa{c}    &       &    &   \\ 
		%\ipa{cʰ}    &       &    &   \\ 
		%\ipa{ɟ}    &       &    &   \\ 
		%\ipa{ɲɟ}    &       &    &   \\ 
		%\ipa{ɲ}    &       &    &   \\ 
		\ipa{j}    &     \deux{jʁ}  & \japhug{ajʁu}{be bowed} (of legs, trees etc) \\ 
		\lspbottomrule
	\end{tabular}
\end{table}		
\resetcounters{2CR}{3CR}

In ambiguous clusters, with the dental fricative \ipa{zʁ} and non-nasal sonorants (\ipa{wʁ}, \ipa{lʁ}, \ipa{rʁ} and \ipa{jʁ}), \ipa{ʁ} is always initial, as illustrated by \forme{rɤβʁɯ\redp{}βʁa} from \japhug{rɤβʁa}{roar}, \forme{kɯ-ʁrɯ\redp{}rʁom} `very rough (one)' from \japhug{rʁom}{be rough} and \forme{kɯ-ɤjʁɯ\redp{}jʁu} `very bent (one)' from \japhug{ajʁu}{be bent}.

One of the origins of cluster-final \forme{ʁ} is the result of the lenition of the uvular stop \forme{*q} in double stop clusters, as in \japhug{βʁa}{prevail} (from \forme{*pq\trt}, \citealt[330]{jacques04these}, as shown by the Situ cognate \forme{pkâ}, \citealt[603]{huangsun02}).

\subsection{Summary} \label{sec:summary.clusters}
\ADD{\value{2wC}}{\value{2szC}}{\totdeux}
\ADD{\totdeux}{\value{2lC}}{\totdeux}
\ADD{\totdeux}{\value{2rC}}{\totdeux}
\ADD{\totdeux}{\value{2jC}}{\totdeux}
\ADD{\totdeux}{\value{2CZC}}{\totdeux}
\ADD{\totdeux}{\value{2xGC}}{\totdeux}
\ADD{\totdeux}{\value{2XRC}}{\totdeux}
\ADD{\totdeux}{\value{2NC}}{\totdeux}
\ADD{\totdeux}{\value{2mnC}}{\totdeux}
\ADD{\totdeux}{\value{2Cw}}{\totdeux}
\ADD{\totdeux}{\value{2Cj}}{\totdeux}
\ADD{\totdeux}{\value{2Cr}}{\totdeux}
\ADD{\totdeux}{\value{2Cl}}{\totdeux}
\ADD{\totdeux}{\value{2CG}}{\totdeux}
\ADD{\totdeux}{\value{2CR}}{\totdeux}
\ADD{\totdeux}{2}{\totdeux}

\ADD{\value{3wC}}{\value{3szC}}{\tottrois}
\ADD{\tottrois}{\value{3lC}}{\tottrois}
\ADD{\tottrois}{\value{3rC}}{\tottrois}
\ADD{\tottrois}{\value{3jC}}{\tottrois}
\ADD{\tottrois}{\value{3CZC}}{\tottrois}
\ADD{\tottrois}{\value{3xGC}}{\tottrois}
\ADD{\tottrois}{\value{3XRC}}{\tottrois}
\ADD{\tottrois}{\value{3NC}}{\tottrois}
\ADD{\tottrois}{\value{3mnC}}{\tottrois}
\ADD{\tottrois}{\value{3Cw}}{\tottrois}
\ADD{\tottrois}{\value{3Cj}}{\tottrois}
\ADD{\tottrois}{\value{3Cr}}{\tottrois}
\ADD{\tottrois}{\value{3Cl}}{\tottrois}
\ADD{\tottrois}{\value{3CG}}{\tottrois}
\ADD{\tottrois}{\value{3CR}}{\tottrois}

\tabref{tab:clusters.tot}  summarizes the numbers of clusters identified in §\ref{sec:preinitials} and §\ref{sec:medials}. Since in ambiguous clusters (§\ref{sec:medials}) partial reduplication is not always available to determine whether the final non-nasal sonorant is medial or initial, no attempt is made at distinguishing between [initial+medial] and [preinitial+initial] clusters in this table.

\begin{table}
	\caption{Count of consonant clusters} \label{tab:clusters.tot}   
	\begin{tabular}{lrrrr}
		\lsptoprule	
		type &CC& CCC& total\\		
		\midrule
		\ipab{wC}  & 	\arabic{2wC}  & \arabic{3wC}  &   \addition{2wC}{3wC}  & 	\\	
		\ipab{s/zC}  & 	\arabic{2szC}  & \arabic{3szC}  &   \addition{2szC}{3szC}  & 	\\	
		\ipab{lC}  & 	\arabic{2lC}  & \arabic{3lC}  &   \addition{2lC}{3lC}  & 	\\	
		\ipab{ʂ/rC}  & 	\arabic{2rC}  & \arabic{3rC}  &   \addition{2rC}{3rC}  & 	\\	
		\ipab{jC}  & 	\arabic{2jC}  & \arabic{3jC}  &   \addition{2jC}{3jC}  & 	\\	
		\ipab{ɕ/ʑC}  & 	\arabic{2CZC}  & \arabic{3CZC}  &   \addition{2CZC}{3CZC}  & 	\\	
		\ipab{x/ɣC}  & 	\arabic{2xGC}  & \arabic{3xGC}  &   \addition{2xGC}{3xGC}  & 	\\	
		\ipab{χ/ʁC}  & 	\arabic{2XRC}  & \arabic{3XRC}  &   \addition{2XRC}{3XRC}  & 	\\	
		\ipab{NC}  & \arabic{2NC}  & \arabic{3NC}  &   \addition{2NC}{3NC}  & 	\\	
		\ipab{m/nC}  & \arabic{2mnC}  & \arabic{3mnC}  &   \addition{2mnC}{3mnC}  & 	\\	
		\midrule
		\ipab{Cɕ}  & 	2  & 	  & 	  2& 	\\	
		\midrule
		\ipab{Cw}  & 	 \arabic{2Cw}  & \arabic{3Cw}  &   \addition{2Cw}{3Cw}  & 	\\
		\ipab{Cj}  & 	 \arabic{2Cj}  & \arabic{3Cj}  &   \addition{2Cj}{3Cj}  & 	\\
		\ipab{Cl}  & 	 \arabic{2Cl}  & \arabic{3Cl}  &   \addition{2Cl}{3Cl}  & 	\\
		\ipab{Cr}  & 	 \arabic{2Cr}  & \arabic{3Cr}  &   \addition{2Cr}{3Cr}  & 	\\
		\ipab{Cɣ} & \arabic{2CG}  & \arabic{3CG}  &   \addition{2CG}{3CG}  & 	\\
		\ipab{Cʁ} & \arabic{2CR}  & \arabic{3CR}  &   \addition{2CR}{3CR}  & 	\\
		\midrule
		total & \totdeux & \tottrois & \ADD{\totdeux}{\tottrois}{\total}\total \\
		\lspbottomrule
	\end{tabular}
\end{table}

Clusters with four elements can appear at least at the phonetic level if the last syllable of a verb stem with a triconsonantal onset undergoes synizesis with the \textsc{1sg} \forme{-a} suffix (§\ref{sec:synizesis}). The denominal verb \japhug{ɣɤjmŋo}{dream of} (§\ref{sec:denom.tr.GA}) from the noun \japhug{tɯ-jmŋo}{dream} (§\ref{sec:jC.clusters}) provides examples like \forme{pjɤ́-wɣ-ɣɤ-jmŋo-a} (\textsc{ifr}-\textsc{inv}-\textsc{denom}-dream-\textsc{1sg}) `s/he dreamed of me' realized as \phonet{pjó.ɣe.jmŋwa} with a complex onset \phonet{jmŋw} comprising four segments.


Attested ambiguous clusters are listed in \tabref{tab:ambiguous.clusters}. Grey shading indicates phonotactically impossible combinations, due in particular to the constraints against the combination of velars and uvulars (§\ref{sec:CG.clusters}, §\ref{sec:CRR.clusters}), and against the combination of alveolo-palatals with \ipa{j} (§\ref{sec:jC.clusters}, §\ref{sec:Cj.clusters}). Orange colour indicates [initial+medial] clusters, whose final sonorant is removed in partial reduplication. Blue colour marks [preinitial+initial] clusters, whose final sonorant is unaffected by reduplication. Clusters left in white are those for which partial reduplication cannot be tested. The cluster \forme{rw} has two possible reduplication patterns (§\ref{sec:Cw.clusters}) and is thus left unmarked.


\begin{table}
	\caption{Ambiguous clusters} \label{tab:ambiguous.clusters}   
	\begin{tabular}{Xllllllllll}
		\lsptoprule
		& \ipa{w}	& \ipa{j}	& \ipa{r}	& \ipa{l}	& \ipa{ɣ}	& \ipa{ʁ}	\\
		\midrule
		\ipa{ɕ}	&	&\grise{}	& \ipa{ɕr}	 \medial{}& \ipa{ɕl}	\medial& \ipa{ɕɣ}	\medial{}& \ipa{ɕʁ}	\\
		\ipa{ʑ}	&&	\grise{}	& \ipa{ʑr}\medial{}& \ipa{ʑl}	& \ipa{ʑɣ}	&	\\
		\hline
		\ipa{s}	&	& \ipa{sj}	\medial{}& \ipa{sr}	\medial{}& \ipa{sl}	\initial{}& \ipa{sɣ}	&	\\
		\ipa{z}	& \ipa{zw}	\medial{}& \ipa{zj}	& \ipa{zr}	\medial{}& \ipa{zl}	& \ipa{zɣ}	\initial{}& \ipa{zʁ}	\\
		\ipa{l}	& \ipa{lw}	\medial{}& \ipa{lj}	\medial{}&	\grise{}	&\grise{}		& \ipa{lɣ}	\initial{}& \ipa{lʁ}	\initial{}\\
		\ipa{r}	& \ipa{rw}	 & \ipa{rj}	\medial{}&	\grise{}	& \ipa{rl}	\initial{}& \ipa{rɣ}	\initial{}& \ipa{rʁ}	\initial{}\\
		\hline
		\ipa{w}	&\grise{}		& \ipa{wj}	\initial{} & \ipa{wr}	\initial{}& \ipa{wl}	\initial{}& \ipa{wɣ}	\initial{}& \ipa{wʁ}	\initial{}\\
		\ipa{j}	& \ipa{jw}	\initial{}&\grise{}		& \ipa{jr}	\initial{}& \ipa{jl}	\initial{}& \ipa{jɣ}	\initial{}& \ipa{jʁ}	\initial{}\\
		\ipa{ɣ}	&\grise{}		& \ipa{ɣj}	\initial{}& \ipa{ɣr}	 \initial{}& \ipa{ɣl}	\initial{}&\grise{}		&\grise{}	\\
		\ipa{ʁ}	&\grise{}		& \ipa{ʁj} \initial{}	& \ipa{ʁr}	 \initial{}& \ipa{ʁl}	\initial{}&\grise{}		&\grise{}		\\
		\lspbottomrule
	\end{tabular}
\end{table}

Even though some of the clusters cannot be subjected to testing with partial reduplication, the data in \tabref{tab:ambiguous.clusters} suggest the following rules:

\begin{itemize}
	\item If the first element of the ambiguous cluster is an alveolo-palatal fricative  (\ipa{ɕ}, \ipa{ʑ}), the following sonorant is \textit{medial}.
	\item If the first element of the ambiguous cluster is a glide  (\ipa{w}, \ipa{j}) or a dorsal sonorant (\ipa{ɣ}, \ipa{ʁ}), the following sonorant is \textit{initial}.
	\item If the first element of the ambiguous cluster is a dental fricative  (\ipa{s}, \ipa{z}), the rhotic \ipa{r} or the lateral \ipa{l}, the following sonorant is \textit{medial} if it is a glide or \ipa{l}, and it is \textit{initial} if it is the rhotic or a dorsal sonorant. 
\end{itemize}


\subsubsection{Heterosyllabic clusters} \label{sec:heterosyllabic.clusters}  
In addition to the clusters attested in onset position described in this chapter, another type of consonant clusters is found whenever a closed syllable is in non-final position in the word. These heterosyllabic clusters, whose first element is the coda of the first syllable, present considerably fewer phonotactic constraints than onset clusters.


The most obvious difference is the fact that, while a strict prohibition against having the same segment as preinitial and a medial is observed in onset clusters, this constraint does not apply in heterosyllabic ones. For instance, the noun \japhug{tɕʰɤrprɯ}{rain shelter}, a compound from Tibetan \tibet{ཆར་}{tɕʰar}{rain} and the native word \japhug{tɤprɯ}{rain shelter}, has the sequence \forme{-r.pr\trt}, which would be completely impossible as syllable onset (§\ref{sec:rhotic.dissimilation}).

The phoneme \ipa{w}, realized as \forme{-β} in coda position (§\ref{sec:codas.inventory}), does not become an unvoiced fricative when followed by an unvoiced obstruent in the next syllable, as in the noun \japhug{slɤβkʰaŋ}{school}, which is not realized as $\dagger$\forme{slɤfkʰaŋ} as would have been expected if the \ipa{w} were in preinitial position (§\ref{sec:wC.clusters}). The coda \forme{-β} is generally deleted when it precedes labial stops, especially the voiced \ipa{b} in ideophones. For instance, the pattern II ideophone (§\ref{sec:ideo.II}) from the root  \idroot{bɤβ} is \forme{bɤbɤβ} `stubborn, bulky'; the alternative realization \forme{bɤβbɤβ} is also possible.

Unlike \ipa{w} in preinitial position (§\ref{sec:NC.clusters}), the coda \forme{-β} does not assimilate with nasals or prenasalized onsets that follow it. For instance,  the coda \forme{-β} in the last syllable of verb stems is not nasalized by the dual \forme{-ndʑi} or plural \forme{-nɯ} indexation suffixes (§\ref{sec:intr.23}): \forme{pjɯ-fkaβ-nɯ} (\textsc{ipfv}-cover-\textsc{pl}) `they cover it' (example \ref{ex:WtaR.kutanW}, §\ref{sec:preverb.cover}) cannot be realized as $\dagger$\phonet{pjifkámnɯ}. The same is true in noun compounds; for example \japhug{jlɤβndʑu}{weft stick} from \japhug{tɯ-jlɤβ}{weft} and \japhug{ndʑu}{stick}.

A puzzling cluster \forme{-p.t-} is observed in the numeral \japhug{sqaptɯɣ}{eleven} (§\ref{sec:teens}). The \forme{-p-} is a linking element between the bound form \forme{sqa-} `ten' and the following numeral, also found in other Gyalrongic languages \citep{jacques17num}, but its expected form would be the preinitial \forme{f-} (from \ipa{w}) in this position. It is better analyzed as the coda of the first syllable.

The dental \ipa{t}, which exists as coda (§\ref{sec:codas.inventory}), but not as preinitial (§\ref{sec:preinitials}), is attested in heterosyllabic clusters followed by unvoiced labial or velar stops, mainly in Tibetan loanwords such as \japhug{rgɤtpu}{old man} (from \tibet{རྒད་པོ་}{rgad.po}{old man}) or \japhug{mtɕʰɤtkʰo}{house shrine} (on which see §\ref{sec:historical.phono}). In compounds, the coda \ipa{t} is either lost in non-final syllables (§\ref{sec:loss.codas.compounds}), or nasalizes to \phonet{n} if the following segment is a nasal sonorant or a prenasalized obstruent (§\ref{sec:internal.sandhi.compounds}), in particular with the dual \forme{-ndʑi} and plural \forme{-nɯ} indexation suffixes (§\ref{sec:intr.23}), so that even heterosyllabic clusters with \ipa{t} a first element are infrequent. 

Exceptions to the loss of \forme{-t} before coronal stops and affricates do exist, however, for instance \forme{tɤɕphɤtta}, the name of a type of sewing method, compound of \japhug{tɤ-ɕpʰɤt}{patch} (§\ref{sec:bare.action.nominals}) with the verb \japhug{ta}{put} (§\ref{sec:ta.lv}). In this noun, a geminated \forme{-t.t-} occurs across syllable boundaries.

While dental fricative preinitials cannot precede affricates (§\ref{sec:sC.clusters}), the coda \ipa{z} does occurs before the dental affricate \forme{tsʰ-} in the cluster \forme{-s.tsʰ-} in \japhug{mbrɤstsʰi}{rice soup} (from \japhug{mbrɤz}{rice} and \japhug{tɯtsʰi}{rice gruel}), assimilating in voice (§\ref{sec:internal.sandhi.compounds}).

The preinitial \ipa{j} is not compatible with palatal and alveolo-palatal initials (§\ref{sec:jC.clusters}), but the coda \ipa{j} can precede those segments across syllable boundary, as in \japhug{qajʑmbraʁ}{ear of wheat} from \japhug{qaj}{wheat} and \japhug{ʑmbraʁ}{ear} (of corn).

The velar coda \ipa{ɣ} is generally deleted before velar stops, even in reduplicated ideophones. For instance, the pattern II ideophone (§\ref{sec:ideo.II}) from the root \idroot{gɤɣ} `curved; moving with difficulty' is \forme{gɤgɤɣ}, though \forme{gɤɣgɤɣ} is also possible. There is only one example of \forme{ɣ} preceding a uvular segment even across syllable boundaries: the sigmatic causative \japhug{sɯɣʁaʁ}{cause to hatch} from \japhug{ʁaʁ}{hatch} (§\ref{sec:caus.sWG}). The Tibetan form  \tibet{འབྲུག་གློག་}{ⁿbrug.glog}{thunder and lightning} would have been expected to  yield $\dagger$\forme{mbrɯɣ.ʁloʁ} in Japhug, but we find instead \forme{mbrɯɣloʁ} (in dialects other than Kamnyu) or \japhug{mbɣɯrloʁ}{thunderstorm} in Kamnyu with irregular metathesis of \forme{ɣ} and \forme{r}.

The uvular \ipa{ʁ} can surface before velar and uvular stops in heterosyllabic clusters (unlike when it occurs as preinitial, §\ref{sec:XC.clusters}), as in \japhug{taʁki}{up and down} (§\ref{sec.v.v.compounds.degree}; realized as \phonet{taχki}) and \japhug{qʰoʁqʰoʁ}{ingot}. The uvular coda is deleted, however, in the compound \japhug{paskɤɣ}{pig to be fattened} from \japhug{paʁ}{pig} and \japhug{skɤɣ}{fatten} (pig), avoiding the non-attested cluster \forme{-ʁ.sk-}.

When preceding unvoiced obstruents, the fricative codas \ipa{z}, \ipa{ɣ} and \ipa{ʁ} assimilate in voice and are converted to \phonet{s}, \phonet{x} and \phonet{ʁ}, respectively (§\ref{sec:internal.sandhi.compounds}).

Onset clusters with nasal preinitials followed by non-nasal sonorants have been completely eliminated by a series of sound changes detailed in §\ref{sec:NC.clusters}. However, the nasal codas \ipa{m}, \ipa{n} and \ipa{ŋ} can occur before syllables with \forme{l\trt}, \forme{r\trt}, \forme{j\trt}, \forme{w\trt}, \forme{ɣ-} or \forme{ʁ-} as onset. For instance, we find \forme{-n.l-} in \japhug{srɯnloʁ}{ring}, \forme{-m.ɣ-} (instead of \forme{-mbr-}) in \japhug{nɤtsɯmɣɯt}{take away and bring back} (from  \japhug{tsɯm}{take away} and \japhug{ɣɯt}{bring}, §\ref{sec:denom.compound.verbs}), \forme{-n.r-} in \japhug{smɤnrɯɣ}{medicinal plants} (from \tibet{སྨན་རིགས་}{sman.rigs}{type of materia medica}) or \forme{-m.j-} (instead of \forme{-mɲ-}) in \japhug{zɯmjɯ}{barrel handle} (compound of \japhug{zɯm}{bucket} and \japhug{ɯ-jɯ}{its handle}, from \tibet{ཟོམ་}{zom}{bucket} and \tibet{ཡུ་བ་}{ju.ba}{handle}, respectively).

These clusters, although heterosyllabic, are reduplicated as a whole when partial reduplication is applied, as in the comitative adverb (§\ref{sec:comitative.adverb}) \forme{kɤ́-srɯnlɯ\redp{}nloʁ} `together with his/her ring' from \japhug{srɯnloʁ}{ring} (instead of $\dagger$\forme{kɤ́-srɯnlɯ\redp{}loʁ}, an incorrect form).

More complex clusters with a nasal as first element are found when the second member of the compound has a complex onset, for instance  \forme{-n.rʑ-} in \japhug{χɕɯnrʑi}{Yama} (from \tibet{གཤིན་རྗེ་}{gɕin.rdʑe}{Yama}, the Buddhist god of the underworld) and \mbox{\forme{-m.xtsʰ-}} in \japhug{jpumxtsʰɯm}{thickness} (a compound noun derived from \japhug{jpum}{be thick} and \japhug{xtsʰɯm}{be thin}, §\ref{sec.v.v.compounds.degree}).

Some heterosyllabic clusters are avoided by insertion of an anaptyctic vowel \ipa{ɯ}, in particular in Tibetan loanwords. For instance, \tibet{སེམས་སྡུག་}{sems.sdug}{sadness, worry} is borrowed as \japhug{sɯmɯzdɯɣ}{worry} instead of expected $\dagger$\forme{sɯmzdɯɣ}, with the \forme{ɯ} breaking the cluster \forme{-m.-zd-}.

The  anaptyctic  vowel can be \ipa{i} if the cluster in the second syllable has a palatal or alveolo-palatal preinitial. For instance,  \japhug{ɕom}{iron} and \japhug{tɤ-jŋoʁ}{hook} are compounded as \japhug{ɕɤmiŋoʁ}{iron hook}, in which the preinitial \ipa{j} is converted to \ipa{i}.

\subsubsection{The sonority sequencing principle in Japhug} \label{sec:ssp}
A considerable amount of work in phonology supports the idea that the segments of the world's languages follow a universal scale of sonority (for instance \citealt{vennemann88syllable, blevins95syllable}; see \citealt{ohala90sonority} for an opposing view). Several versions of this sonority hierarchy have been proposed, for instance  (§\ref{sec:sonority.hierarchy}) (\citealt[235]{parker02sonority}).

\begin{exe}
	\ex \label{sec:sonority.hierarchy}
	\glt low vowels > mid vowels > high vowels > \ipa{ə} > glides > laterals > flaps
	> trills > nasals > \ipa{h} > voiced fricatives > voiced stops > voiceless
	fricatives > voiceless stops and affricates
\end{exe}

The notion of sonority is invoked in particular to account for observed generalizations in the structure of consonant clusters: in many languages, clusters follow the so-called \textit{sonority sequencing principle} (\textsc{ssp}, \citealt[210]{blevins95syllable}):

\begin{exe}
	\ex \label{ex:ssp}
	\glt `Between any member of a syllable and the syllable peak, a sonority
	rise or plateau must occur.'
\end{exe}

According to this hierarchy, in onset clusters, sonorants are expected to be closer the syllable nucleus than obstruents (\forme{krV} is favoured over \forme{rkV}), and glides to be closer to the nucleus than any other consonant (\forme{mjV} is preferred over \forme{jmV}).

Japhug, like other Gyalrongic languages (see in particular \citealt{jackson00puxi} and \citealt[73]{lai17khroskyabs}), is rich in \textsc{ssp}-infringing clusters. 


For instance, the 17 biconsonantal clusters with \ipa{j} as first element (\arabic{2jC} in §\ref{sec:jC.clusters} and 5 in §\ref{sec:medials}) all contravene the \textsc{ssp} except for \ipa{jw}. Moreover, some of these clusters, such as \ipa{jtʂʰ}, \ipa{jm} and \ipa{jŋ}, have no \textsc{ssp}-compliant counterpart: $\dagger$\ipa{tʂʰj}, $\dagger$\ipa{mj},  and $\dagger$\ipa{ŋj} are not attested as syllable onsets (though the last two can occur across syllables, §\ref{sec:heterosyllabic.clusters}). 

A certain number of \textsc{ssp}-compliant clusters have been removed by sound changes, in particular \forme{*mj-} \fl{} \forme{mɲ\trt}, \forme{*mr-} \fl{} \forme{mbr-} (§\ref{sec:NC.clusters}) and  \forme{*tr-} \fl{} \forme{tʂ-} (§\ref{sec:Cr.clusters}).


\section{Sandhi} \label{sec:sandhi.word}
Word boundaries in Japhug can be defined by a combination of criteria, including stress (§\ref{sec:stress}) and also morphology in the case of verbs (§\ref{sec:wordhood.criteria.verb}), but they are not completely impenetrable from a phonological point of view: some sandhi phenomena, including assimilation and resyllabification, do occur across word boundaries.

First, the codas \ipa{-z},  \ipa{-r}, \ipa{-ɣ}, \ipa{-ʁ} can assimilate in voicing to the following consonant (§\ref{sec:codas.inventory}), as in heterosyllabic clusters (§\ref{sec:heterosyllabic.clusters}).

Second, the coda \ipa{-t} is nasalized when directly followed by a nasal or prenasalized segment, and tends to drop when followed by a dental, alveolo-palatal and retroflex segment. In (§\ref{sec:rWmtChAt.cthWwGlAt}) for example, the \forme{-t} coda of \japhug{rɯmtɕʰɤt}{offering to the mountain} drops before the \forme{ɕ-} translocative prefix and that of \forme{lɤt} merges with the following affricate \forme{tɕ}.

\begin{exe}
	\ex \label{sec:rWmtChAt.cthWwGlAt}
	\gll  rɯmtɕʰɤt ɕ-cʰɯ́-wɣ-lɤt tɕe ɲɯ-pʰɤn  \\
	mountain.offering \textsc{tral}-\textsc{ipfv}:\textsc{downstream}-\textsc{inv}-release \textsc{lnk} \textsc{sens}-be.efficient   \\
	\glt `If you go and make offerings to the mountain (god), it will solve (your problem).'  (Lobzang, 26)
\end{exe}

These rules are optional, and it is possible to find examples in the corpus where they do not take place. For instance, in (§\ref{sec:tolAt.ri}), the coda \forme{-t} of \forme{to-lɤt} is clearly audible even before \forme{r-}.

\begin{exe}
	\ex \label{sec:tolAt.ri}
	\gll tɯrpa ci to-lɤt ri \\
	axe once \textsc{ifr}-release \textsc{lnk} \\
	\glt `He wielded his axe, but ...' (140430 jin e-zh, 39)
\end{exe}

Third, when a [preinitial+initial] cluster is preceded by an open syllable, the preinitial can be resyllabified as coda of the previous syllable. In (§\ref{sec:Wrna.XchoRe}) for instance, the preinitial \forme{χ-} of \japhug{χcʰoʁe}{right and left} becomes the coda of \forme{-rna} and voices to \phonet{ʁ}, and yields a sequence \phonet{-rnaʁ.cʰo-}. This type of phenomenon is considerably rarer than the previous ones.

\begin{exe}
	\ex \label{sec:Wrna.XchoRe}
	\gll ɯ-rna χcʰoʁe ʑo ko-sti. \\
	\textsc{3sg}.\textsc{poss}-ear left.and.right \textsc{emph} \textsc{ifr}-plug \\
	\glt  `He plugged both hid left and his right ears.' (140514 huishuohua de niao-zh, 195)
\end{exe}

Fourth, when a word ending in open syllable precedes a vowel-initial word (§\ref{sec:zero.onset}), the initial vowel \forme{a-} and \forme{ɯ-} of the latter very commonly merge with the final vowel of the former. For instance, the sentence in (§\ref{sec:amu.Wxtsa}) is realized as \phonet{amuxtsaːpɯŋu}: the \forme{ɯ-} prefix is elided by merging with the preceding \forme{u}, and the \forme{-a} rhyme of \forme{ɯ-xtsa} merges with the \forme{a-} prefix of the following word, yielding a phonetically long \phonet{aː}.

\begin{exe}
	\ex \label{sec:amu.Wxtsa}
	\gll a-mu ɯ-xtsa a-pɯ-ŋu ɲɯ-ra \\
	\textsc{1sg}.\textsc{poss}-mother \textsc{3sg}.\textsc{poss}-shoe \textsc{irr}-\textsc{ipfv}-be \textsc{sens}-be.needed \\
	\glt `Let (me take this) shoe (as a present) for my mother.' (tWxtsa 2003, 11)
\end{exe}

External sandhi is not represented in the transcription used in this grammar.


\section{Tibetan script-based orthography} \label{sec:tibetan.script}
The IPA orthography chosen to write Japhug in this grammar is probably not viable for use by native speakers, and an alternative writing system based on Tibetan script is preferable. Fortunately, it is relatively easy to transcribe Japhug into Tibetan script.

Concerning the vowels (§\ref{sec:vowels}), \ipa{i}, \ipa{u}, \ipa{e} and \ipa{o} can be represented using the \tibt{གི་གུ་}{gi.gu}, \tibt{ཞབས་ཀྱུ་}{ʑabs.kʲu}, \tibt{འགྲེང་བུ་}{ⁿgreŋ.bu} and \tibt{ན་རོ་}{na.ro} symbols, respectively. The vowel \ipa{ɤ}, being very common word-internally, can be taken as base vowel of the aksharas. As for \ipa{ɯ}, the Old Tibetan \tibt{གི་གུ་ལོག་}{gi.gu.log} can be employed. Finally, the \ipa{a} can be represented by the long vowel symbol \tibt{འ་ཆུང་}{ɦa.tɕʰuŋ}.

A few consonants found in Japhug do not have Tibetan equivalents. The letter \tibt{ཝ}{w} can be reserved for the velar fricative \ipa{ɣ}. The uvulars can be differentiated from the velars by the \tibt{ཚ་རྟགས་}{tsʰa.rtags}, for instance \tibn{ཀ༹} \ipa{q}, \tibn{འག༹} \ipa{ɴɢ} and \tibn{ཝ༹}  \ipa{ʁ}. In coda position however, the contrast between \ipa{ɣ} and \ipa{ʁ} does not need to be transcribed, since these two codas are in complementary distribution with the preceding vowels (§\ref{sec:rhyme.inventory}). The unvoiced velar fricative can be written as \tibn{ཧ} \ipa{x} and the uvular one as \tibn{ཧ༹} \ipa{χ} in initial position (for the rare phoneme \ipa{h} an additional diacritic symbol such as \tibn{ཧ྇} can be employed). In preinitial position, however (§\ref{sec:xC.clusters}, §\ref{sec:XC.clusters}), the voiced stop symbols should be preferred, since there is no contrast between fricatives and stops in this context.

The retroflex affricates can be represented by the Sanskrit symbols for retroflex stops (\tibn{ཊ} \ipa{tʂ}, \tibn{ཋ} \ipa{tʂʰ} and \tibn{ཌ} \ipa{dʐ}) and the retroflex fricative by the corresponding Sankrit consonant (\tibn{ཥ} \ipa{ʂ} ).


The glide \ipa{w} in initial position can be written as \tibn{བ༹} \ipa{w} with \tibt{ཚ་རྟགས་}{tsʰa.rtags}, but as medial (§\ref{sec:Cw.clusters}) the \tibt{ཝ་ཟུར་}{wa.zur} can be employed. Medial \forme{-ɣ-} and \forme{-ʁ-} can be represented by underscript \tibn{ཝ} and \tibn{འ}, for instance in \tibt{པྺཱག}{pɣaʁ} `turn over' and \tibt{ཏྀ་འཛྰི་}{tɯ-ndzʁi} `collar bone'.

The clusters with fricatives preceding prenasalized obstruents can be transcribed by combining a prefixed {\tibetain{འ}} marking prenasalization with a superfixed letter marking the fricative, as in \tibt{འཤྦྲྀ་}{ʑmbrɯ} `boat',  \tibt{འཤྒྲི་}{ʑŋgri} `star' or \tibet{འསྡེ}{znde} `wall'. Since the voicing contrast is neutralized in this position (§\ref{sec:sC.clusters}, §\ref{sec:shC.clusters}), the unvoiced fricative symbols \tibt{ས}{s} and \tibt{ཤ}{ɕ} can be used.

The following examples illustrate how this script can be used to write complete sentences. 

\begin{exe}
	\ex 
	\gll ɕ-tɤ-χtɯ-t-a \\
	\textsc{tral}-\textsc{aor}-buy-\textsc{pst}:\textsc{tr}-\textsc{1sg} \\
	\glt \tibt{ཤྟ་ག༹ཏྀ་ཏཱ།} 
	\glt `I went and bought it.' 
\end{exe}

\begin{exe}
	\ex 
	\gll  tɤ-rʑaβ nɯ pjɤ-rɤ-ɕpʰɤt \\
	\textsc{indef}.\textsc{poss}-wide \textsc{dem} \textsc{ifr}.\textsc{ipfv}-\textsc{apass}-patch \\
	\glt \tibt{ཏ་རྮཱབ་ནྀ་པྱ་ར་ཤྥད།}
	\glt `The wife was patching clothes.' 
\end{exe}


\begin{exe}
	\ex 
	\gll  ɯ-me nɯ kɯ andi paχɕi ɲɯ-ntsɣe ŋu\\
	\textsc{3sg}.\textsc{poss}-daughter \textsc{dem} \textsc{erg} west apple \textsc{ipfv}-sell be:\textsc{fact} \\
	\glt \tibt{འྀ་མེ་ནྀ་ཀྀ་འ་འདི་པཱག་ཤི་ཉི་འཙྺེ་ངུ།}
	\glt `Her daughter is selling apples over there.' 
\end{exe}

