\chapter{Pronouns} \label{chap:pronouns}
\section{Personal pronouns} \label{sec:pers.pronouns}

%\\ipa\{([\w-]*)\}
%\1

%(\d)\\(\w\w)\{\}
%\\textsc{\1\2}

The pronominal system of Japhug distinguishes singular, dual and plural. Alongside free pronouns, a system of pronominal prefixes is used not only to express possession on nouns (§\ref{sec:gen.possession}), but also appears in various non-finite verb forms (§\ref{sec:subject.participle.possessive}, §\ref{sec:object.participle.possessive}, §\ref{sec:oblique.participle.possessive}, §\ref{sec:bare.inf}, §\ref{sec:bare.action.nominals}). These prefixes do not distinguish the second and the third person in the dual and plural forms; their use is described in §\ref{sec:possessive.paradigm}.

\begin{table}[h] 
\caption{Pronouns and possessive prefixes }\label{tab:pronoun}
\begin{tabular}{lllll} 
\lsptoprule
& Possessive   & \multicolumn{2}{c}{Pronouns}    \\
&prefixes &(Kamnyu dialect) &(Tatshi dialect) \\
\midrule
1\sg{}  &\forme{a-}  &	 \forme{aʑo},    \forme{aj} &\forme{ŋa} 	\\
2\sg{} &\forme{nɤ-}  &		\forme{nɤʑo},  \forme{nɤj} & \forme{naʒo} \\
3\sg{}& \forme{ɯ-}  &\forme{ɯʑo}	&  \forme{mi} \\
\midrule
1\du{} &\forme{tɕi-}   & \forme{tɕiʑo}  & \forme{tsəʒo} \\
2\du{}&\forme{ndʑi-}  &	\forme{ndʑiʑo} 	& \forme{ndzəʒo} \\	
3\du{}&\forme{ndʑi-}  &	\forme{ʑɤni} &	 \forme{mindzɐ} \\
\midrule
1\pl{} & \forme{i-}  &	\forme{iʑo}, \forme{iʑora},   \forme{iʑɤra}  &	  \forme{jiʒo} \\
2\pl{}&\forme{nɯ-}  &		\forme{nɯʑo}, \forme{nɯʑora},   \forme{nɯʑɤra} &  \forme{əʒo} \\	
3\pl{}&\forme{nɯ-}  &		\forme{ʑara} 	 & \forme{mijo} \\
\midrule
generic&\forme{tɯ-}  &		\forme{tɯʑo}	& -- \\
\lspbottomrule
\end{tabular}
\end{table}
 
Free pronouns and possessive prefixes are remarkably similar in Kamnyu Japhug. In this dialect, all the pronouns except the third person dual and plural are formed by adding the root \forme{-ʑo} to the corresponding possessive prefix. In the eastern Japhug dialects (represented by Tatshi in \tabref{tab:pronoun}), the \textsc{1sg}  \forme{ŋa} and \textsc{3sg}  \forme{mi} pronouns differ from the corresponding possessive prefixes (the former is possibly borrowed from Situ \forme{ŋā}). 

The first and second person singular pronouns  \forme{aʑo} and \forme{nɤʑo} also have the shorter monosyllabic forms \forme{aj} and \forme{nɤj}, respectively. These short forms are considerably less common in stories (in the reported speech of the characters), but appear frequently in free conversations.

Japhug lacks any inclusive/exclusive distinction, unlike other Gyalrongic languages such as Tshobdun, Situ or Khroskyabs (see \citealt{jackson98morphology}, \citealt[177]{linxr93jiarong}, \citealt[92]{prins16kyomkyo}, \citealt[170]{lai17khroskyabs}). Example  (\ref{ex:tCiZo.CetCi}) shows the dual pronoun \japhug{tɕiʑo}{we (dual)} in inclusive use (it is clear from the context that the son tells his mother to come with him), and (\ref{ex:tCiZo.tCitAYi}) illustrates the same pronoun in exclusive use. Similar pairs of examples can be found with the first plural pronoun \japhug{iʑo}{we (plural)} and its variants.

\begin{exe}
\ex \label{ex:tCiZo.CetCi}
\gll a-mu tɕetʰa tɕiʑo kɯnɤ ɕe-tɕi \\
\textsc{1sg}.\textsc{poss}-mother later \textsc{1du} also go:\textsc{fact}-\textsc{1du} \\
\glt `Mother, you and I will go too.' (2003tWxtsa, 138)
\end{exe}

\begin{exe}
\ex \label{ex:tCiZo.tCitAYi}
\gll nɯʑora ɣɯ nɯ-ɕɤmɯɣdɯ cʰo kɯ-fse nɯ ɯ-tsʰɤt nɯ, tɕiʑo ɣɯ tɕi-tɤɲi tɯ-ldʑa pɯ-tu tɕe, nɯ kɤ-nɯ-tʰɯ-tɕi ɕti wo \\
\textsc{2pl} \textsc{gen} \textsc{2pl}.\textsc{poss}-gun \textsc{comit} \textsc{sbj}:\textsc{pcp}-be.like \textsc{dem} \textsc{3sg}-instead \textsc{dem} \textsc{1du} \textsc{gen} \textsc{1du}.\textsc{poss}-staff one-long.object \textsc{pst}.\textsc{ipfv}-exist \textsc{lnk} \textsc{dem} \textsc{aor}-\textsc{auto}-spread-\textsc{1du} be:\textsc{aff}:\textsc{fact} \textsc{sfp} \\
\glt `Instead of guns and other things like you, we$_{DU.EXCL}$ only had a staff, and we$_{DU.EXCL}$ used it as a bridge (to cross the river).' (2003kunbzang, 164)
\end{exe}

Third person pronouns can be used with inanimate referents, as the third person dual \forme{ʑɤni} in example (\ref{ex:rNgW.tokAmWrpundZici}).

\begin{exe}
\ex \label{ex:rNgW.tokAmWrpundZici}
\gll tɕe rŋgɯ nɯ  to-k-ɤmɯ-rpu-ndʑi-ci tɕe, tɕendɤre ʑɤni pjɤ-nɯ-ɴɢrɯ-ndʑi   \\
\textsc{lnk} boulder \textsc{dem} \textsc{ifr}-\textsc{peg}-\textsc{recip}-bump.into-\textsc{du}-\textsc{peg} \textsc{lnk} \textsc{lnk} \textsc{3du} \textsc{ifr}-\textsc{auto}-\textsc{acaus}:shatter-\textsc{du} \\
\glt `The boulders bumped into each other and they were pulverized.' (smanmi4.82-83)
\end{exe}

In some contexts, demonstrative pronouns rather than person pronouns are used to refer to a third person, even human (see §\ref{sec:demonstrative.pronouns}).

Personal pronouns are not used as head of relative clauses (as in Chinese \ch{……的你}{... de  nǐ }{you who are ...}), though there are case of relativization of first or second person possessor, as in (\ref{ex:amu.kWme}) (§\ref{sec:S.possessor.relativization}).

\begin{exe}
\ex \label{ex:amu.kWme}
\gll aʑo nɯ [a-mu kɯ-me] ŋu-a tɕe tɕe \\
\textsc{1sg} \textsc{dem} \textsc{1sg}.\textsc{poss}-mother \textsc{sbj}:\textsc{pcp}-exist be:\textsc{fact}-\textsc{1sg} \textsc{lnk} \textsc{lnk} \\
\glt `I am someone who does not have a mother.' (2003Nyimawodzer2, 12)
\end{exe}

Personal pronouns can take determiners, in particular the demonstrative \forme{nɯ} as in (\ref{ex:amu.kWme}), numerals (§\ref{sec:uses.numerals}) and can also precede a noun in apposition, in expressions such as \forme{iʑo kɯrɯ} `we, Tibetans' (\ref{ex:iZo.kWrW}) or \forme{nɤʑo qaɕpa} `you frog' in (\ref{ex:nAZo.qaCpa}).

\begin{exe}
\ex \label{ex:iZo.kWrW}
\gll iʑo kɯrɯ tɕe pɤjka tu-nɯ-ti-j ŋu tɕe, \\
\textsc{1pl} Tibetan \textsc{lnk} species.of.squash \textsc{ipfv}-\textsc{auto}-say-\textsc{1pl} \textsc{lnk} \\
\glt `We Tibetans call it \forme{pɤjka}.' (16-CWrNgo, 71)
\end{exe}

\begin{exe}
\ex  \label{ex:nAZo.qaCpa}
\gll  nɤʑo qaɕpa nɤ-rʑaβ ɕɯ kɯ tɯ́-wɣ-mbi  \\
 \textsc{2sg} frog \textsc{2sg}.\textsc{poss}-wife who \textsc{erg} 2-\textsc{inv}-give:\textsc{fact} \\
\glt `Who will give you a wife, you frog.'   (2002 qaCpa, 15)
\end{exe} 

The relationship between the pronouns and the following noun is not necessarily appositional; for instance, in (\ref{ex:iZora.tuhanzu}), despite the absence of possessive prefix on the borrowing \ch{土汉族}{tǔhànzú}{local Chinese}, the meaning of \forme{iʑora <tuhanzu>} is `the local Chinese living among us' rather than `We local Chinese'.

\begin{exe}
\ex \label{ex:iZora.tuhanzu}
\gll iʑora <tuhanzu> ra kɯ, <qingyang> tu-ti-nɯ ŋu. \\
\textsc{1pl} local.Chinese \textsc{pl} \textsc{erg} bharal \textsc{ipfv}-say-\textsc{pl} be:\textsc{fact} \\
\glt `The local Chinese among us call it ``qingyang''.' (20-xsar, 40)
\end{exe} 

As in most languages with polypersonal indexation, pronouns (especially first and second person pronouns) are never obligatory, and a finite verb form without overt argument NPs is a perfectly well-formed sentence (§\ref{sec:word.order}). Overt pronouns are obligatorily overt as core arguments only when focalized (§\ref{sec:focalization.overt}).

Japhug presents a very common subtype of split ergativity: the ergative \forme{kɯ} being obligatory on transitive subject third person pronouns (except in the case of the emphatic use of pronouns, §\ref{sec:pronouns.emph}) but optional on first and second person pronouns (§\ref{sec:absolutive.A} and §\ref{sec:A.kW}).

\subsection{Honorific plural} \label{sec:honorific.pronouns}
The second plural pronoun \forme{nɯʑo} can be used as  honorific pronoun, as in (\ref{ex:nWZo.tChi.WrWG}), where the addressee is unambiguously singular.

\begin{exe}
\ex \label{ex:nWZo.tChi.WrWG}
\gll ɯtɤz nɯʑo tɕʰi ɯ-rɯɣ tɯ-ŋu-nɯ? \\
actually \textsc{2pl} what \textsc{3sg}.\textsc{poss}-race 2-be:\textsc{fact}-\textsc{pl} \\
\glt `Actually, which type of being are you?' (2003smanmi, 162)
\end{exe}

This pronoun correlates with plural honorific indexation \forme{-nɯ} (§\ref{sec:honorific.indexation}). Note that the plural \forme{ra} also occurs as a honorific marker on nouns (§\ref{sec:ra.honorific}).

\subsection{Personal pronouns as possessive markers} \label{sec:pronouns.possessive.markers}
Personal pronouns occur instead of possessive prefixes (§\ref{sec:possessive.prefixes}) as first member of compounds  with the adverbs \forme{-sɯso} `as $X$ wish' (from the verb \japhug{sɯso}{think}), as in example (\ref{ex:ʑara.sWso}) and \forme{-sti} `$X$ alone'.  

\begin{exe}
\ex \label{ex:ʑara.sWso}
\gll a-zda ra ʑara-sɯso tu-nɯ-nɤŋkɯŋke-nɯ ɲɯ-kʰɯ \\
\textsc{1sg}.\textsc{poss}-companion \textsc{pl} \textsc{3pl}-as.wish \textsc{ipfv}-\textsc{auto}-go.here.and.there-\textsc{pl} \textsc{sens}-be.possible \\
\glt `The other (snakes) can go here and there as they wish.' (The divination, 43)
\end{exe}

The noun \forme{-βra} `it is $X$'s turn to' can either take a regular possessive prefix, or a pronoun in \textit{status constructus} form as in (\ref{ex:nAZoBra}), where the bound form \forme{nɤʑɤ-} occurs instead of \japhug{nɤʑo}{\textsc{2sg}}.
\begin{exe}
\ex \label{ex:nAZoBra}
\gll  wortɕʰi nɤʑɤ-βra a-tɤ-tɯ-ti ra \\
please \textsc{2sg}.\textsc{poss}-turn \textsc{irr}-\textsc{pfv}-2-say be.needed:\textsc{fact} \\
\glt `It is your turn to say it.' (2014-kWlAG, 90)
\end{exe}


These constructions are discussed in more detail in §\ref{sec.IPN.adverbs} and §\ref{sec:stWsti}.

\section{Generic pronouns}  

\subsection{\japhug{tɯʑo}{one}}  \label{sec:genr.pro}
The generic pronoun \japhug{tɯʑo}{one} has the same morphological structure as personal pronouns as seen in the previous section, combining the generic possessive prefix \forme{tɯ-} with the pronominal root \forme{-ʑo}. Note that this generic possessive has to be strictly distinguished from the homophonous indefinite possessor prefix \forme{tɯ-} (see §\ref{sec:indef.genr.poss}). It has a rare plural variant \japhug{tɯʑɤra}{one} (see  §\ref{sec:partitive.indexation}, example \ref{ex:tWZAra.pWxtCij}).

In Japhug, sentences have at most one generic human referent (§\ref{sec:indef.genr.poss}). If this referent is core argument, the verb has generic indexation (\forme{kɯ-} for intransitive subject and object and \forme{wɣ-} for transitive subject, as in the following example, §\ref{sec:indexation.generic.tr}). The generic argument can be realized as the generic pronoun \forme{tɯʑo} as in (\ref{ex:pjWkWZGAGANgi}) or by a generic noun (such as \japhug{tɯrme}{person}, §\ref{sec:tWrme.generic}).

\begin{exe}
\ex \label{ex:pjWkWZGAGANgi}
\gll tɯ-zda pjɯ́-wɣ-z-ɣɤtɕa, \textbf{tɯʑo} ntsɯ pjɯ-kɯ-ʑɣɤ-ɣɤŋgi tɕe, pɯ-kɯ-nɯ-ɣɤtɕa kɯ́nɤ pjɯ-kɯ-ʑɣɤ-ɣɤŋgi tɕe, ɯ-mbrɤzɯ kɯ-tu me tu-kɯ-ti ɲɯ-ŋu.   \\
\textsc{genr}.\textsc{poss}-companion \textsc{ipfv}-\textsc{inv}-\textsc{caus}-be.wrong oneself always \textsc{ipfv}-\textsc{genr}:S/O-\textsc{refl}-be.right \textsc{lnk} \textsc{aor}-\textsc{genr}:S/O-\textsc{auto}-be.wrong also \textsc{ipfv}-\textsc{genr}:S/O-\textsc{refl}-be.right \textsc{lnk} \textsc{3sg}.\textsc{poss}-result \textsc{sbj}:\textsc{pcp}-have  not.exist:\textsc{fact} \textsc{ipfv}-\textsc{genr}-say \textsc{sens}-be \\
\glt  `If one considers that one's companion is wrong, and always considers himself to be right even if one is wrong, there is can be no good result.' (Mouse and sparrow, 80-82)
\end{exe} 

The generic pronoun can occur before a noun with the generic possessive as in \forme{tɯʑo tɯ-skɤt}  `one's language' in example (\ref{ex:tWZo.tWskAt}); this contributes to disambiguating between the indefinite possessive and the generic possessive in the case of inalienably possessed nouns (thus on its own \forme{tɯ-skɤt} can mean either `a language' or `one's language').

\begin{exe}
\ex \label{ex:tWZo.tWskAt}
\gll tɕendɤre tɯʑo tɯ-skɤt ʑara ɣɯ-sɯxɕɤt ɲɯ-ra, ʑara nɯ-skɤt tɯʑo kɯ-sɯxɕɤt ɲɯ-ra \\
\textsc{lnk} \textsc{genr} \textsc{genr}.\textsc{poss}-language \textsc{3pl} \textsc{inv}-teach:\textsc{fact} \textsc{sens}-be.needed \textsc{3pl} \textsc{3pl}.\textsc{poss}-language \textsc{genr} \textsc{genr}:S/O-teach:\textsc{fact} \textsc{sens}-be.needed \\
\glt `One has to teach them one's language, and they have to teach you their language.'  (150901 tshuBdWnskAt, 29)
\end{exe} 

When occurring in A function, the generic pronoun \forme{tɯʑo} obligatorily receives the ergative \forme{kɯ} as in (\ref{ex:tWZo.kW}) (note that in example \ref{ex:tWZo.tWskAt}, although the generic referent is A in the first clause, \forme{tɯʑo} does not take ergative because it is a determiner of \forme{tɯ-skɤt}). 

\begin{exe}
\ex \label{ex:tWZo.kW}
\gll tɯʑo kɯ tɯ-χti ɲɯ́-wɣ-nɯ-ɕar kɯ-maʁ kɯ,  tɯ-pʰama ra kɯ tɯ-χti ɲɯ-ɕar-nɯ \\
\textsc{genr} \textsc{erg} \textsc{genr}.\textsc{poss}-spouse \textsc{ipfv}-\textsc{inv}-\textsc{auto}-search \textsc{inf}:\textsc{stat}-not.be \textsc{erg} \textsc{genr}.\textsc{poss}-parent \textsc{pl} \textsc{erg} \textsc{genr}.\textsc{poss}-spouse \textsc{ipfv}-search-\textsc{pl}  \\
\glt `One could not choose one's spouse, one's parents chose one's spouse.' (14-siblings, 212-213)
\end{exe} 

Relator nouns (§\ref{sec:relator.nouns}), for example the dative \forme{ɯ-ɕki} (§\ref{sec:dative}),  are grammaticalized from inalienably possessed nouns and like them, take a generic possessive prefix (\forme{tɯ-ɕki} or \forme{tɯ-pʰe} `to one') when following \forme{tɯʑo}, as in example (\ref{ex:tWZo.tWCki}), which describes the Omaha-type skewing rule in the kinship system (§\ref{sec:omaha}).

\begin{exe}
\ex \label{ex:tWZo.tWCki}
\gll
tɯ-ɲi ɣɯ ɯ-rɟit nɯra kɯ tɯʑo tɯ-ɕki ``a-rpɯ", tɤ-tɕɯ pɯ-kɯ-ŋu nɤ ``a-rpɯ" tu-ti-nɯ, tɕʰeme pɯ-kɯ-ŋu nɤ ``a-ɬaʁ" tu-ti-nɯ kɯ-ra ŋu \\
\textsc{genr}.\textsc{poss}-FZ \textsc{gen} \textsc{3sg}.\textsc{poss}-child \textsc{dem}:\textsc{pl} \textsc{erg} \textsc{genr} \textsc{genr}-dat \textsc{1sg}.\textsc{poss}-MB \textsc{indef}.\textsc{poss}-son \textsc{pst}.\textsc{ipfv}-\textsc{genr}:S/O-be if \textsc{1sg}.\textsc{poss}-MB \textsc{ipfv}-say-\textsc{pl} girl \textsc{pst}.\textsc{ipfv}-\textsc{genr}:S/O-be  if \textsc{1sg}.\textsc{poss}-MZ \textsc{ipfv}-say-\textsc{pl} \textsc{inf}.\textsc{stat}-be.needed be:\textsc{fact} \\
\glt `One's father's sister's children have to call oneself ``my maternal uncle'' if one is a boy, ``my maternal aunt'' if one is a girl.'  (140425kWmdza03, 1)
\end{exe} 

As examples (\ref{ex:pjWkWZGAGANgi}) to (\ref{ex:tWZo.tWCki}) illustrate, generic agreement between pronoun, possessive prefix and verb indexation is very systematic. Examples of \textsc{1pl} indexation with generic pronouns or vice-versa are, however, attested (§\ref{sec:partitive.indexation}).

Due to the constraint against more than one generic argument per clause, the only case that the generic pronoun can appear two times in the same clause occurs in reflexive constructions (§\ref{sec:reflexive}), as in (\ref{ex:tWZo.kW.tWZo}).

\begin{exe}
\ex \label{ex:tWZo.kW.tWZo}
\gll tɯʑo kɯ tɯʑo tu-kɯ-nɯ-ʑɣɤ-βri ra kɤ-ti ɲɯ-ŋu \\
\textsc{genr} \textsc{erg} \textsc{genr} \textsc{ipfv}-\textsc{genr}:S/O-\textsc{auto}-\textsc{refl}-protect be.needed:\textsc{fact} \textsc{inf}-say \textsc{sens}-be \\
\glt `One has to protect oneself.' (04-qala1, 25)
\end{exe} 

\subsection{The generic noun \japhug{tɯrme}{person}} \label{sec:tWrme.generic}
The noun \japhug{tɯrme}{person}, also attested to express indefinite humans (§\ref{sec:ci.someone}) or in the meaning `someone else' (§\ref{sec:other.pro}), can occur as a marker of generic person, as in (\ref{ex:kWnA.tuwGndza}).
 
\begin{exe}
\ex \label{ex:kWnA.tuwGndza}
\gll  tɯrme kɯnɤ tú-wɣ-ndza sna. \\
 people also \textsc{ipfv}-\textsc{inv}-eat be.good:\textsc{fact} \\
\glt `It is also good for people to eat.' (12-Zmbroko, 31)
\end{exe}
 
In this function, \japhug{tɯrme}{person} can be indexed on the verb by either generic person (as in \ref{ex:kWnA.tuwGndza}) or \textsc{3pl} markers, a question explored in more detail in §\ref{sec:genr.3pl}.

There is a slight difference of usage between \japhug{tɯrme}{person} and \japhug{tɯʑo}{one} in clauses with a generic argument. Both can be followed by a noun or a case marker taking the generic possessive prefix, as in (\ref{ex:genr.tWrme3}) and (\ref{ex:tWZo.tWfsu}). 

\begin{exe}
\ex  \label{ex:genr.tWrme3}
\gll tɯrme tɯ-fsu ɕaŋtaʁ tu-mbro mɤ-cʰa. \\
people \textsc{genr}.\textsc{poss}-equal.to until \textsc{ipfv}-be.big \textsc{neg}-\textsc{fact}:can \\ 
\glt `It cannot grow as big as a person (as oneself).' (11-qarGW, 24)
\end{exe}

\begin{exe}
\ex  \label{ex:tWZo.tWfsu}
\gll ɯ-tɯ-mbro nɯnɯ tɯʑo tɯ-fsu jamar tu-zɣɯt cʰa.  \\
\textsc{3sg}-\textsc{nmlz}:\textsc{deg}-be.high \textsc{dem} \textsc{genr} \textsc{genr}-equal.to about \textsc{ipfv}:\textsc{up}-reach can:\textsc{fact} \\
\glt `As for its size, it can reach one's (a person's) size.' (16-CWrNgo, 18)
\end{exe}

However, \japhug{tɯrme}{person} as a generic noun can alternatively be used with a possessee or a case marker with the third person singular \forme{ɯ-} prefix, as in (\ref{ex:tWrme.Wfsu}), while this option does not exist for \japhug{tɯʑo}{one}.

\begin{exe}
\ex  \label{ex:tWrme.Wfsu}
\gll tɯrme ɯ-fsu jamar tu-βze cʰa. \\
people \textsc{3sg}-equal.to about \textsc{ipfv}-do[III] can:\textsc{fact} \\
\glt `It can grow about the size of a person.' (12-ndZiNgri, 4)
\end{exe}

\section{Genitive forms} \label{sec:pronouns.gen}
The form of pronouns and personal prefixes undergoes few morphophonological changes in combination with postpositions and relational nouns. However, in combination with the genitive postposition \forme{ɣɯ} (cf §\ref{sec:genitive}), some  personal pronouns have special forms indicated in Table  \ref{tab:pronoun.gen}.

\begin{table}[h] \centering
\caption{Pronouns and possessive prefixes }\label{tab:pronoun.gen}
\begin{tabular}{lllllllll} \lsptoprule
 Free pronoun & Genitive & \\
\midrule
 \forme{aʑo}  &	\forme{aʑɯɣ}  &		\textsc{1sg} \\ 
\forme{nɤʑo}  &	\forme{nɤʑɯɣ}  &			\textsc{2sg} \\ 
\forme{ɯʑo}  &	\forme{ɯʑɤɣ}  &			\textsc{3sg} \\ 
\forme{tɕiʑo}  &	\forme{tɕiʑɤɣ}  &			\textsc{1du} \\ 
\forme{ndʑiʑo}  &	\forme{ndʑiʑɤɣ}  &		\textsc{2du} \\	 
\forme{ʑɤni}  &	\forme{ʑɤniɣɯ}  &		\textsc{3du} \\	 
\forme{iʑo}  &	\forme{iʑɤɣ}, 	\forme{iʑɤra ɣɯ}   &			\textsc{1pl} \\ 
\forme{nɯʑo}  &	\forme{nɯʑɤɣ}, 	\forme{nɯʑɤra ɣɯ}  &			\textsc{2pl} \\ 
\forme{ʑara}  &	\forme{ʑaraɣ},   \forme{ʑara ɣɯ}&			\textsc{3pl}  \\  
\lspbottomrule
\end{tabular}
\end{table}

While some degree of variation exists with dual and plural pronouns (for instance the regular \forme{iʑo ɣɯ} is found alongside \forme{iʑɤɣ} and \forme{iʑɤra ɣɯ}), for the singular pronouns only one form is attested.

\begin{exe}
\ex
\gll aʑɯɣ 	ndʐa 	ŋu 	ɕi, 	nɤʑɯɣ 	ndʐa 	ŋu, 	aj 	mɯ́j-tso-a   \\
\textsc{1sg}:\textsc{gen} reason be:\textsc{fact} \textsc{qu} \textsc{2sg}:\textsc{gen} reason be:\textsc{fact} \textsc{1sg} \textsc{neg}:\textsc{sens}-understand-\textsc{1sg} \\
\glt  `I don't know if it is because of me, or because of you.' (that the phone line is not working well) (phone conversation, 2011) %\wav{8_ndzxa})
\end{exe} 

In the genitive forms of the pronouns, the vowel of the genitive marker is generally dropped, and the pronominal root \forme{-ʑo} undergoes vowel change to \forme{-ʑɯɣ} (in the case of first and second person) and \forme{-ʑɤɣ} (in other forms). Note that \forme{ʑaraɣ} is the only case of the rhyme \ipa{aɣ} in Japhug.

When genitive pronouns occur as determiners of nouns (including in the possessive existential construction, §\ref{sec:possessive.mihi.est}), these nouns almost always take a possessive prefix coreferent with the genitive pronoun, as in (\ref{ex:tɕithAfkAlAGi}).

\begin{exe}
\ex \label{ex:tɕithAfkAlAGi}
\gll 
tɕiʑɤɣ tɕi-tʰɤfkɤlɤɣi tɯ-ɕkat pɯ-tu tɕe, nɯ kɤ-nɯ-χtɤr-tɕi ɕti wo \\
\textsc{1du}:\textsc{gen} \textsc{2du}-plant.ash one-load \textsc{pst}.\textsc{ipfv}-exist \textsc{lnk} \textsc{dem} \textsc{aor}-\textsc{auto}-spread-\textsc{1du} be:\textsc{aff}:\textsc{fact} \textsc{sfp} \\
\glt `We had one load of ash, and we spilled it there.' (2003 Kunbzang, 171)
\end{exe} 

The genitive pronouns can be used as possessive pronouns (`mine', `my own' etc) and take the determiner \forme{nɯ} and the plural \forme{ra}, as in (\ref{ex:aZWG.nW}) and (\ref{ex:WZAG.nWra}).

\begin{exe}
\ex \label{ex:aZWG.nW}
\gll ``tɕe ɣnɤsqaptɯ-rʑaʁ tu-tsu tɕe ɲɯ-ʁaʁ ŋu" ɲɯ-ti-nɯ ri, aʑɯɣ nɯ ɣnɤsqamnɯz tɤ-rʑaʁ mɤɕtʂa mɯ-nɯ-ʁaʁ. \\
\textsc{lnk} twenty.one-night \textsc{ipfv}-pass \textsc{lnk} \textsc{ipfv}-hatch be:\textsc{fact} \textsc{sens}-say-\textsc{pl} \textsc{lnk} \textsc{1sg}:\textsc{gen} \textsc{dem} twenty.two one-night  until \textsc{neg}-\textsc{aor}-hatch \\
\glt `People say that (chicken eggs) hatch after twenty-one days, mine took twenty-two days to hatch.' (150819 kumpGa)
\end{exe} 

\begin{exe}
\ex \label{ex:WZAG.nWra}
\gll ɯʑɤɣ nɯra tu-nɯ-ɣɤ-βdi tɕe, ɕɯ-sɤ-sqɤr mɤ-ra \\
\textsc{3sg}:\textsc{gen} \textsc{dem}:\textsc{pl} \textsc{ipfv}-\textsc{auto}-\textsc{caus}-be.well \textsc{lnk} \textsc{tral}-\textsc{apass}-hire:\textsc{fact} \textsc{neg}-be.needed:\textsc{fact} \\
\glt `He repairs his own (machines) himself, he does not need to ask other people.' (14-siblings, 168)
\end{exe} 

\section{The emphatic use of pronouns} \label{sec:pronouns.emph}
In addition to their referential and anaphoric functions, pronouns in Japhug can be used in an emphatic way in combination with the emphatic particle \forme{ʑo} (§\ref{sec:emphatic.Zo}), as in  (\ref{ex:WZo.Zo}).

\begin{exe}
\ex \label{ex:WZo.Zo}
\gll aʑo ɯʑo ʑo kɤ-mto mɯ-pɯ-rɲo-t-a. \\
\textsc{1sg} \textsc{3sg} \textsc{emph} \textsc{inf}-see \textsc{neg}-\textsc{aor}-experience-\textsc{pst}:\textsc{tr}-\textsc{1sg} \\
\glt `I never saw it itself.' (24-kWmu, 7)
\end{exe} 

In combination with the autive \forme{nɯ-} on the verb, pronouns express the meaning `do $X$ on one's own' (§\ref{sec:autoben.proper}). In the case of transitive verbs, third person pronouns in this function does not take the ergative even if the referent is the transitive subject (example \ref{ex:pjWnWtCAtnW}, where \japhug{tɕɤt}{take out} is transitive, §\ref{sec:tCAt.lv}).
 
\begin{exe}
\ex \label{ex:pjWnWtCAtnW}
\gll tɕe lu-nɯ-rɤji-nɯ tɕe, nɯ-kɤ-ndza nɯra ʑara pjɯ-nɯ-tɕɤt-nɯ pjɤ-ŋu tɕe \\
\textsc{lnk} \textsc{ipfv}-\textsc{auto}-plant.crops-\textsc{pl} \textsc{lnk} \textsc{3pl}.\textsc{poss}-\textsc{obj}.\textsc{pcp}-eat \textsc{dem}:\textsc{pl} \textsc{3pl} \textsc{ipfv}-\textsc{auto}-take.out-\textsc{pl} \textsc{ifr}.\textsc{ipfv}-be \textsc{lnk} \\
\glt `They planted crops, and earned their food on their own.' (about lepers, who were settled in the special place by the government, 25-khArWm, 70)
\end{exe} 

The emphatic pronoun \japhug{raŋ}{oneself} borrowed from Tibetan \tibet{རང་}{raŋ}{oneself}, can also be used with any person, though this usage is not very common. It can occur with the autive (\ref{ex:aZo.raN}) or without it (\ref{ex:aZo.raN.Zo}).

\begin{exe}
\ex \label{ex:aZo.raN}
\gll
nɤʑo tu-tɯ-ti mɤ-ra ma aʑo raŋ tu-nɯ-ti-a jɤɣ \\
\textsc{2sg} \textsc{ipfv}-2-say \textsc{neg}-be.neededo:\textsc{fact} \textsc{lnk} \textsc{1sg} oneself \textsc{ipfv}-\textsc{auto}-say-\textsc{1sg} be.possible:\textsc{fact} \\
\glt `You don't need to say it, I can say it myself.' (elicited)
\end{exe}

\begin{exe}
\ex \label{ex:aZo.raN.Zo}
\gll aʑo raŋ ʑo ju-ɕe-a ra \\
\textsc{1sg} oneself \textsc{emph} \textsc{ipfv}-go-\textsc{1sg} be.needed:\textsc{fact} \\
\glt `I have to go there myself.' (150830 afanti-zh, 96)
\end{exe}

\section{Interrogative pronouns} \label{sec:interrogative.pronouns}
The interrogative pronouns in Japhug are indicated in \tabref{tab:interrog.pronoun}. These pronouns are used in independent interrogative clauses (\ref{ex:tChi.pWNu}), in subordinate clauses (\ref{ex:tChi.kWNu}), and also in correlatives (\ref{ex:NotCu.WsAzrAZi}) (§\ref{sec:interrogative.relative}), and also occur to express non-specific referents (these uses are described in § §\ref{sec:interrogative.indef}, after the indefinite pronouns).

\begin{exe}
\ex \label{ex:tChi.pWNu}
\gll
tɕe mɤʑɯ tɕʰi pɯ-ŋu? \\
\textsc{lnk} yet what \textsc{pst}.\textsc{ipfv}-be \\
\glt `What was there (after this one)?' (12-ndZiNgri, 100)
\end{exe}  

\begin{exe}
\ex \label{ex:tChi.kWNu}
\gll ɯʑo tɕʰi kɯ-ŋu nɯ ko-tso-nɯ tɕe tɕe cʰɤ́-wɣ-tɕɤt \\
\textsc{3sg} what \textsc{sbj}:\textsc{pcp}-be \textsc{dem} \textsc{ifr}-understand-\textsc{pl} \textsc{lnk} \textsc{lnk} \textsc{ifr}:\textsc{downstream}-\textsc{inv}-take.out \\
\glt `They understood what he was, and expelled him (from their group).' (140427 hanya yu gezi-zh, 19)
\end{exe}  

\begin{exe}
\ex \label{ex:NotCu.WsAzrAZi}
\gll  ɯ-pʰoŋbu tɕʰi kɯ-fse nɯ, ŋotɕu ɯ-sɤz-rɤʑi nɯnɯ ɣɯ kɯ-nɯtsa kɯ-fse ɲɯ-ɕti tɕe \\
\textsc{3sg}.\textsc{poss}-body what \textsc{sbj}:\textsc{pcp}-be.like \textsc{dem} where \textsc{3sg}-\textsc{obl}:\textsc{pcp}-remain \textsc{dem} \textsc{gen}  \textsc{sbj}:\textsc{pcp}-fit \textsc{sbj}:\textsc{pcp}-be.like \textsc{sens}-be:\textsc{aff} \textsc{lnk} \\
\glt `The way its body is like is well-adapted to the place where it lives.' (19-rNamoN, 24)
\end{exe}  

\begin{table}[h] \centering
\caption{Interrogative pronouns }\label{tab:interrog.pronoun}
\begin{tabular}{lllllllll} \lsptoprule
\japhug{tɕʰi}{what} \\
\japhug{ɕɯ}{who} \\
\japhug{tʰɤstɯɣ}{how many} \\
\japhug{tʰɤjtɕu}{when} \\
\japhug{ŋotɕu}{where}, \japhug{ŋoj}{where} \\
\japhug{tɕʰindʐa}{why} \\
\lspbottomrule
\end{tabular}
\end{table}

In addition to these pronouns, some indefinite pronouns are also marginally used in questions, see for instance (\ref{ex:thWthAci.totia}) in §\ref{sec:thWci}.

\subsection{\japhug{tɕʰi}{what}} \label{sec:tChi}
The interrogative pronoun  `what' considerably varies across Japhug dialects. In Kamnyu we find \forme{tɕʰi}, apparently borrowed from Tibetan \forme{tɕʰi}. Neighbouring dialects of Gdongbrgyad area have either \forme{tsʰi} (in Mangi) or \forme{tʰi} (in Rqaco), which represents the original Rgyalrongic root for this interrogative pronoun (cognate with Tibetan \tibet{ཆི་}{tɕʰi}{what} and Limbu \forme{the}). Even in the Kamnyu dialect, the form \forme{tsʰi-} is directly attested in the indefinite \japhug{tsʰitsuku}{some} (§\ref{sec:tshitsuku}). Mangi Japhug shares with Kamnyu the sound change \forme{*tʰi} \fl{} \forme{tsʰi} which also affects the verb \japhug{tsʰi}{drink} (this sound change occurred after the pronoun  \forme{*tʰi} underwent \textit{status constructus} alternation to \forme{tʰɯ-} and was used to build the indefinite pronoun \japhug{tʰɯci}{something}, see §\ref{sec:thWci}). Note that Kamnyu Japhug \japhug{tɕʰi}{what} is homophonous with the noun \japhug{tɕʰi}{tree-trunk stairs} attested for instance in example (\ref{ex:tChi.tukWndW}) -- the readers of Japhug texts have to be aware of this potential ambiguity.

\begin{exe}
\ex \label{ex:tChi.tukWndW}
\gll ɕom ɣɟɯ kɯ-mbɯ\redp{}mbro ʑo,  ɯ-ɣmbaj zɯ tɕʰi tu-kɯ-ndɯ ci pɯ-tu ɲɯ-ŋu \\
iron tower \textsc{sbj}:\textsc{pcp}-\textsc{emph}\redp{}be.high \textsc{emph} \textsc{3sg}.\textsc{poss}-side \textsc{loc} treetrunk.stairs \textsc{ipfv}:\textsc{up}-\textsc{sbj}:\textsc{pcp}-\textsc{acaus}:spread \textsc{indef} \textsc{pst}.\textsc{ipfv}-be \textsc{sens}-be \\
\glt `There was a huge iron toward, with a tree-trunk stairs on its side.' (Norbzang05, 65)
\end{exe}  

The Eastern dialects of Gsardzong and Datshang have \forme{xto} instead, a word of unknown etymology; these dialects, which share many additional morphosyntactic and phonological commonalities, can be collectively referred to as ``Xtokavian''.

In the Kamnyu dialect, \japhug{tɕʰi}{what} is by far the most common interrogative pronoun in the corpus. In interrogative clauses, it can be used to ask about objects, non-human animals (\ref{ex:nAmbro}) and names of persons (\ref{ex:tChi.tWrmi}), and can occur as prenominal determiner meaning `what (type of)' as in (\ref{ex:tChi.Cku}).

\begin{exe}
\ex \label{ex:nAmbro}
\gll nɤʑo nɤ-mbro nɯ tɕʰi ŋu \\
\textsc{2sg} \textsc{2sg}.\textsc{poss}-horse \textsc{dem} what be:\textsc{fact} \\
\glt `Who is your horse?' (about a sentient horse, 2003smanmi-tamu, 53)
\end{exe}  

\begin{exe}
\ex \label{ex:tChi.tWrmi}
\gll tɕʰi tɯ-rmi? \\
what 2-be.called:\textsc{fact} \\
\glt `What is your name?' (heard in context)
\end{exe}    

\begin{exe}
\ex \label{ex:tChi.Cku}
\gll  tɕendɤre mɤʑɯ tɕʰi ɕku tu \\
\textsc{lnk} yet what allium exist:\textsc{fact} \\
\glt `What other (plants of the genus) \textit{Allium} are there?' (07-Cku, 97)
\end{exe}   

As in many languages, this interrogative pronoun (instead of the pronoun \japhug{ɕɯ}{who}) is also used in questions about classification of persons \citep{idiatov07nonselective}, including social affiliation (\ref{ex:tChi.WrWG}, and \ref{ex:tChi.kWNu} above) and biological affiliation (\ref{ex:tChi.tosci}).

\begin{exe}
\ex \label{ex:tChi.WrWG}
\gll ɯtɤz nɯʑo tɕʰi ɯ-rɯɣ tɯ-ŋu-nɯ? \\
finally \textsc{2pl} what  \textsc{3sg}.\textsc{poss}-race  2-be:\textsc{fact}-\textsc{pl} \\
\glt `Finally, what race (of being) are you?' (smanmi2003, 172)
\end{exe}  

There is no specific interrogative pronoun to ask about manner like English `how', and Japhug expresses this meaning by combining \forme{tɕʰi} with the verbs \japhug{fse}{be like} or \japhug{stu}{do like} (§\ref{sec:similative.verb.complementation}) as in (\ref{ex:tChi.Zo.tuwGBzu.phAn}).


\begin{exe}
\ex \label{ex:tChi.Zo.tuwGBzu.phAn}
\gll nɤ-smɤn tɤ-sɯ-βzu-t-a ri maka mɯ́j-pʰɤn, tɕe tɕʰi ʑo tú-wɣ-stu pʰɤn \\
\textsc{3sg}.\textsc{poss}-medicine \textsc{aor}-\textsc{caus}-make-\textsc{pst}:\textsc{tr}-\textsc{1sg} but at.all \textsc{neg}:\textsc{sens}-be.efficient \textsc{lnk} what \textsc{emph} \textsc{ipfv}-\textsc{inv}-do.like be.efficient:\textsc{fact} \\
\glt `I had medicine made for you but it does not work, how should we do for it to work?' (nyima wodzer 2002, 22) 
\end{exe}  

When used with \forme{tɕʰi}, the similative verbs often take an infinitival complement as in (\ref{ex:tChi.tAtWfsendZi}).

\begin{exe}
\ex \label{ex:tChi.tAtWfsendZi}
\gll a-ʁi, ki kɯ-fse tɤjpɣom kɯ-wxti nɯtɕu, kɤ-ɕe tɕʰi tɤ-tɯ-fse-ndʑi? \\
\textsc{1sg}.\textsc{poss}-younger.sibling this \textsc{sbj}:\textsc{pcp}-be.like ice \textsc{sbj}:\textsc{pcp}-be.big \textsc{dem}:\textsc{loc} \textsc{inf}-go what \textsc{aor}-2-be.like-\textsc{du} \\
\glt `Sister, how did you cross such a big block of ice?' (Kunbzang 2005, 156)
\end{exe}  
 
The pronoun \japhug{tɕʰi}{what} on its own can occur in questions about the reason or the purpose of a particular state of affair, as in (\ref{ex:tChi.apWNua}) and (\ref{ex:tChi.YWtWnAre}).

\begin{exe}
\ex \label{ex:tChi.apWNua}
\gll  aʑo tɕʰi a-pɯ-ŋu-a? \\
\textsc{1sg} what \textsc{irr}-\textsc{ipfv}-be-\textsc{1sg} \\
\glt `How can it be me?' (2003sras, 61)
\end{exe}  

\begin{exe}
\ex \label{ex:tChi.YWtWnAre}
\gll  a-tɤɕime, tɕʰi ɲɯ-tɯ-nɤre ŋu? \\
 \textsc{1sg}.\textsc{poss}-lady what \textsc{sens}-2-laugh be:\textsc{fact} \\
 \glt `My lady, why are you laughing?'  (Not `what are you laughing at?', 2002 qaCpa, 102)
\end{exe}  

When referring to purpose or reason, it is possible to combine  \japhug{tɕʰi}{what} with the nouns \japhug{ɯ-spa}{its material} and \japhug{ɯ-ndʐa}{its reason} (as the pronoun \japhug{tɕʰindʐa}{why}) , as in (\ref{ex:tChi.Wspa.pWNu}) and (\ref{ex:tChi.YWtWɣAwu}). Note that examples (\ref{ex:tChi.YWtWnAre}) and (\ref{ex:tChi.YWtWɣAwu}) are from the same story, just a few lines away, in the same context; the construction in (\ref{ex:tChi.YWtWɣAwu}) is a more explicit variant of that in (\ref{ex:tChi.YWtWnAre}).

\begin{exe}
\ex \label{ex:tChi.Wspa.pWNu}
\gll tɕe tɕʰi ɯ-spa pɯ-ŋu mɤ-xsi ma tɕe nɯ kɯ-fse pjɤ-tu  \\
\textsc{lnk} what \textsc{3sg}.\textsc{poss}-material \textsc{pst}.\textsc{ipfv}-be \textsc{neg}-\textsc{genr}:know \textsc{lnk} \textsc{lnk} \textsc{dem} \textsc{sbj}:\textsc{pcp}-be.like \textsc{ifr}.\textsc{ipfv}-exist \\
\glt `It is not known what it was for, but there was something like that.' (hist140522 GJW, 18)
\end{exe}  

\begin{exe}
\ex \label{ex:tChi.YWtWɣAwu}
\gll tɕʰindʐa ɲɯ-tɯ-ɣɤwu ŋu? \\
why \textsc{sens}-2-cry be:\textsc{fact} \\
\glt `Why are you crying?' (2002 qaCpa, 94)
\end{exe} 

The pronoun \forme{tɕʰi} takes case marking with genitive \forme{ɣɯ} and the instrumental/ergative \forme{kɯ}, as in (\ref{ex:tChi.kW}).

\begin{exe}
\ex \label{ex:tChi.kW}
\gll tɕe tɕʰi kɯ tu-sɯ-βze ŋu mɤxsi ma nɯ kɯ-fse nɯ, sɯku ri ku-ndzoʁ ŋu \\
\textsc{lnk} what \textsc{erg} \textsc{ipfv}-\textsc{caus}-make[III] be:\textsc{fact} \textsc{neg}-\textsc{genr}-know \textsc{lnk} \textsc{dem} \textsc{sbj}:\textsc{pcp}-be.like \textsc{dem} top.of.trees \textsc{loc} \textsc{ipfv}-\textsc{acaus}:attach be:\textsc{fact} \\
\glt `I don't what it (the wasp) uses to make it (its nest), it is attached on trees.' (26-ndzWrnaR, 55)
\end{exe} 

In combination with the adverb \forme{jarma} / \japhug{jamar}{about}, it can be used to indicate a quantity, instead of \japhug{tʰɤstɯɣ}{how many} (§\ref{sec:thAstWG}), as illustrated by (\ref{ex:tChi.jamar}), and (\ref{ex:tChi.jamar.kondza}).

\begin{exe}
\ex \label{ex:tChi.jamar}
\gll tu-ɕtʂam-a tɕe tɕʰi jamar ʑo ɣɤʑu kɯ? \\
\textsc{ipfv}-measure[III]-\textsc{1sg} \textsc{lnk} what about \textsc{emph} exist:\textsc{sens} \textsc{sfp} \\
\glt `I will measure it with a scoop to see how much (gold) there is.' (140512 alibaba-zh, 59)
\end{exe}  

\begin{exe}
\ex \label{ex:tChi.jamar.kondza}
\gll kʰɯtsa ɯ-ŋgɯ tɯ-ci tu-rku-nɯ tɕe, nɯnɯtɕu tɤŋe nɯ pjɯ-sɯ-ntɕʰɤr-nɯ tɕe, tɕe tɕʰi jamar ko-ndza nɯnɯ, nɯnɯ ɯ-ŋgɯ nɯtɕu pjɯ-ru-nɯ tɕe,  nɯnɯ tu-rtoʁ-nɯ pjɤ-ŋgrɤl.   \\
bowl \textsc{3sg}-inside \textsc{indef}.\textsc{poss}-water \textsc{ipfv}-put.in-\textsc{pl} \textsc{lnk} \textsc{dem}:\textsc{loc} sun \textsc{dem} \textsc{ipfv}-\textsc{caus}-illuminate-\textsc{pl} \textsc{lnk} \textsc{lnk} what about \textsc{ifr}-eat \textsc{dem} \textsc{dem} \textsc{3sg}-inside \textsc{dem}:\textsc{loc} \textsc{ipfv}:\textsc{down}-look.at-\textsc{pl} \textsc{lnk} \textsc{dem} \textsc{ipfv}-see-\textsc{pl} \textsc{ifr}.\textsc{ipfv}-be.usually.the.case \\
\glt `They used to put water in a bowl and let the sunlight reflect into it; they could see how much (of the sun) had been occulted (`eaten' by the eclipse).' (29-mWBZi, 130)
\end{exe}  

It is possible to combine \forme{tɕʰi jamar} with a adjective to express approximate comparison, as in (\ref{ex:tChi.kWzri}).

\begin{exe}
\ex \label{ex:tChi.kWzri}
\gll lɯlu ɣɯ tɕe ɯʑo ɯ-pʰoŋbu tɕʰi kɯ-zri jamar ɯ-jme nɯ kɯnɤ zri ri \\
cat \textsc{gen} \textsc{lnk} \textsc{3sg} \textsc{3sg}.\textsc{poss}-body what \textsc{sbj}:\textsc{pcp}-be.long about \textsc{3sg}.\textsc{poss}-tail \textsc{dem} also be.long\textsc{fact} but \\
\glt `The cat, its body is about as long as its tail, but...' (27-qartshAz, 219)
\end{exe}  

In correlative clauses, the pronoun \japhug{tɕʰi}{what} can also be used to refer to a quantity without the adverb \japhug{jamar}{about} (example \ref{ex:tChi.tAkWsci}).

\begin{exe}
\ex \label{ex:tChi.tAkWsci}
\gll  tɤ-rɟit tɕʰi tɤ-kɯ-sci nɯ ʑo ɣɯ-tɕɤt kɯ-ra pjɤ-ɕti tɕe,   \\
\textsc{indef}.\textsc{poss}-child what \textsc{aor}-\textsc{sbj}:\textsc{pcp}-be.born \textsc{dem} \textsc{emph} \textsc{inv}-take.out:\textsc{fact} \textsc{inf}:\textsc{stat}-be.needed \textsc{ipfv}.\textsc{ifr}-be:\textsc{aff} \textsc{lnk} \\
\glt `However many children were born, one had to raise them.' (tApAtso kAnWBdaR I, 9)
\end{exe}  

However, in independent interrogative clauses, \japhug{tɕʰi}{what} cannot refer to quantities. Sentence (\ref{ex:tChi.tosci}) thus can only mean `Was it a boy or a girl' not `How many children did she have?'.

\begin{exe}
\ex \label{ex:tChi.tosci}
\gll  ɯ-rɟit tɕʰi to-sci \\
\textsc{3sg}.\textsc{poss}-child what \textsc{ifr}-be.born \\
\glt `Was it a boy or a girl?' (elicited)
\end{exe}  


The interrogative \japhug{tɕʰi}{what} occurs in topicalized clauses with an adjective stative verb in perfective form, meaning `as for how $X$ it becomes' as in examples (\ref{ex:tChi.nWjpum}) and (\ref{ex:tChi.tAmbro}).

\begin{exe}
\ex \label{ex:tChi.nWjpum}
\gll tɕʰi nɯ-jpum ki ɕaŋtaʁ ɲɯ-jpum mɯ́j-cʰa \\
what \textsc{aor}-be.thick \textsc{dem}.\textsc{prox} above \textsc{ipfv}-be.thick \textsc{neg}:\textsc{sens}-can \\
\glt `As for how thick it can grow, it cannot grow thicker than this.' (16-CWrNgo, 154)
\end{exe}


\begin{exe}
\ex \label{ex:tChi.tAmbro}
\gll tɕʰi tɤ-mbro, ʁnɯ-rtsɤɣ ɕaŋtaʁ tu-mbro mɯ́j-cʰa.  \\
what \textsc{aor}-be.tall two-stairs above \textsc{ipfv}-be.tall \textsc{neg}:\textsc{sens}-can \\
\glt `As for how tall it can grow, it cannot grow taller than two stairs.' (07-paXCi, 8)
\end{exe}

\subsection{\japhug{ɕɯ}{who}} \label{sec:CW.pronoun}
The interrogative pronoun \japhug{ɕɯ}{who} occurs in questions about the identification of a human referent. It can occur in all syntactic roles, and does not have special ergative or genitive forms (see examples \ref{ex:CW.kW.tWwGmbi} and \ref{ex:CW.GW}). It is the probable cognate of a etymon widespread in the Trans-Himalayan family (for instance, Tibetan \tibet{སུ་}{su}{who}).

\begin{exe}
\ex  \label{ex:CW.tWNu}
\gll ma-tɯ-nɯqaɟy ma ɕɯ tɯ-ŋu mɤ-xsi \\
\textsc{neg}:\textsc{imp}-2-fish \textsc{lnk} who 2-be:\textsc{fact} \textsc{neg}-\textsc{genr}:know   \\
\glt `Don't fish, I don't who you are.' (gesar, 369)
\end{exe}  

\begin{exe}
\ex  \label{ex:CW.kW.tWwGmbi}
\gll  mɤ-ta-mbi nɤʑo qaɕpa ɕɯ kɯ tɯ́-wɣ-mbi    \\
\textsc{neg}-1\fl2-give:\textsc{fact} \textsc{2sg} frog who \textsc{erg} 2-\textsc{inv}-give:\textsc{fact}  \\
\glt `We won't give her to you, who would give her to you, a frog?'   (2002 qaCpa, 09)
\end{exe} 
 
\begin{exe}
\ex  \label{ex:CW.GW}
\gll  ɕɯ ɣɯ ʑo ɲɯ-kʰam-a ra kɯɣe?    \\
who \textsc{gen} \textsc{emph} \textsc{ipfv}-give:III-\textsc{1sg} be.needed:\textsc{fact} \textsc{sfp} \\
\glt `Whom should I give (her) to (in marriage)?' (140508 benling gaoqiang de si xiongdi-zh, 222)
\end{exe}  

The pronoun  \japhug{ɕɯ}{who} can be used in one context with non-human referents, when asking about which object (out of two or more) has the highest value as to a property described by the main verb, as in (\ref{ex:CW.kW.YWzrindZi}); in this construction, the verb receives non-singular indexation (§\ref{sec:partitive.indexation}), such as the dual  \forme{-ndʑi} in this example. Concerning the use of the ergative \forme{kɯ} in this sentence see §\ref{sec:comparative}, §\ref{sec:sAz.kW} and \citet{jacques16comparative}.  

\begin{exe}
\ex  \label{ex:CW.kW.YWzrindZi}
\gll nɯ nɤ-ku ɯ-tɯ-rɲɟi nɯ, aki ɕe-tɕi tɕe, mbro ɯ-jme cʰonɤ tú-wɣ-sɤfsu, ɕɯ kɯ ɲɯ-zri-ndʑi kɯ \\
\textsc{dem} \textsc{2sg}.\textsc{poss}-head \textsc{3sg}.\textsc{poss}-\textsc{nmlz}:\textsc{deg}-be.long  \textsc{sfp} down go:\textsc{fact}-\textsc{1du} \textsc{lnk} horse \textsc{3sg}.\textsc{poss}-tail \textsc{comit}  \textsc{ipfv}-\textsc{inv}-compare who \textsc{erg} \textsc{sens}-be.long-\textsc{du} \textsc{sfp} \\
\glt `Your hair is very long, let us go downstairs, and compare it with a horse's tail.' (2002 qaCpa, 292)
\end{exe}  

Forms related to \japhug{ɕɯ}{who} in Japhug include the indefinite pronoun \japhug{ɕɯmɤɕɯ}{whoever, anybody} (\ref{sex:CWmACW}) and \japhug{ɕɯŋarɯra}{each better than the other}.

\subsection{\japhug{tʰɤstɯɣ}{how many} and \japhug{tʰɤjtɕu}{when}} \label{sec:thAstWG}
To ask about precise quantities, \japhug{tʰɤstɯɣ}{how many} (or `how much') occurs rather than \forme{tɕʰi jamar} as seen above (§\ref{ex:tChi.jamar}).

\begin{exe}
\ex \label{ex:thAstWG.tWkhAm}
 \gll    nɤʑo 	tʰɤstɯɣ 	tɯ-kʰɤm?    \\
 you how.much 2-give[III]:\textsc{fact}  \\
\glt  `How much (money) do you give (for it)?' (Bargaining, 13)
\end{exe} 

It can be used for any countable quantity, including for people, as in (\ref{ex:thAstWG.tWtunW}).

\begin{exe}
\ex \label{ex:thAstWG.tWtunW}
 \gll tsʰupa tʰɤstɯɣ tɯ-tu-nɯ ŋu? \\
village how.much 2-exist:\textsc{fact}-\textsc{pl} be:\textsc{fact} \\
\glt `How many (people) are you in the village?' (conversation, 140501)
\end{exe} 

The pronoun \forme{tʰɤstɯɣ} has a conjunct form \forme{tʰɤstɯ-} when used with counted nouns, including Chinese borrowings (§\ref{sec:other.numeral.prefixes}). For instance, in (\ref{ex:thAstWmaR}) and (\ref{ex:thAstWdian}), it occurs with the classifiers \japhug{X-maʁ}{size of shoes} (from Chinese \zh{码} \forme{mǎ}), and in (\ref{ex:thAstWdian}) with the non-nativized form of \ch{点}{diǎn}{hour} (the same meaning can be expressed without borrowing from Chinese, as in \ref{ex:thAstWG.kozGWt}).

 \begin{exe}
\ex \label{ex:thAstWmaR}
 \gll   nɤ-xtsa nɯ tʰɤstɯ-maʁ tu-tɯ-ŋge ŋu   \\
\textsc{2sg}.\textsc{poss}-shoe \textsc{dem} how.many-size \textsc{ipfv}-2-wear[III] be:\textsc{fact} \\ 
\glt `What is the size of your shoes?'  (Conversation, 2015)
\end{exe} 
 
 \begin{exe}
\ex \label{ex:thAstWdian}
 \gll  nɯʑo nɯtɕu tʰɤstɯ-<dian>  ŋu? \\
 \textsc{2pl} \textsc{dem}:\textsc{loc} hour how.many-hour be:\textsc{fact} \\
 \glt `What time is it at your place?' (conversation 12-11-2018)
\end{exe} 

Combined with the noun \japhug{tɤ-rʑaʁ}{time}, 	\forme{tʰɤstɯɣ} can be used to ask about a length of time (\ref{ex:thAstWG}).

\begin{exe}
\ex \label{ex:thAstWG}
 \gll   nɤʑo tɤ-rʑaʁ tʰɤstɯɣ jamar tɤ-tsu tɕe kɤ-tɯ-spa-t?  \\
 you \textsc{indef}.\textsc{poss}-time how.many about \textsc{aor}-pass \textsc{lnk} \textsc{aor}-2-be.able-\textsc{pst}:\textsc{tr} \\
\glt   `How long did it take you to learn it?' (elicited)
\end{exe} 

The phrase \forme{tɤ-rʑaʁ tʰɤstɯɣ} (or alternatively \forme{tɯtsʰot tʰɤstɯɣ}) in collocation with the verb \japhug{zɣɯt}{reach}, is also employed for asking about clock time, as in (\ref{ex:thAstWG.kozGWt}) (see §\ref{sec:hours}) or dates. %The nouns \japhug{tɤ-rʑaʁ}{time} and \japhug{tɯtsʰot}{time, hour, clock} are not even obligatory, as shown by  

 \begin{exe}
\ex \label{ex:thAstWG.kozGWt}
 \gll   tɤ-rʑaʁ tʰɤstɯɣ ko-zɣɯt? \\
  \textsc{indef}.\textsc{poss}-time how.many  \textsc{ifr}-reach \\
  \glt `What is the time?' (heard in context)
  \end{exe} 
    
Questions about time can also be expressed by the pronoun \japhug{tʰɤjtɕu}{when}, as in  (\ref{ex:thAjtCu}) and (\ref{ex:thAjtCu.GW}).

\begin{exe}
\ex \label{ex:thAjtCu}
\gll  tʰɤjtɕu lɤ-tɯ-nɯɣe pɯ-ŋu ra nɤ?    \\
 when \textsc{aor}-2-come.back[II] \textsc{pst}.\textsc{ipfv}-be \textsc{pl} \textsc{sfp} \\
\glt  `When did you come back home?' (taRrdo conversation, 01)
\end{exe} 

As shown by (\ref{ex:thAjtCu.GW}), \japhug{tʰɤjtɕu}{when} can be used with the genitive \forme{ɣɯ}.

\begin{exe}
\ex \label{ex:thAjtCu.GW}
\gll <jipiao> nɯ, tʰɤjtɕu ɣɯ tɤ-tɯ-χtɯ-t? \\
plane.ticket \textsc{dem} when \textsc{gen} \textsc{aor}-2-buy-\textsc{pst}:\textsc{tr} \\
\glt `Your plane ticket, for what date did you buy it?' (conversation, 2014.03.19)
\end{exe} 

 
The pronoun \japhug{tʰɤjtɕu}{when} also occurs in the meaning `since when' to express impossible events as in (\ref{ex:thAjtCu.pjANgrAl}).

\begin{exe}
\ex \label{ex:thAjtCu.pjANgrAl}
\gll  nɤ-wɯ kɯ-rɤrɟit jɤ-ari kɯ ɕi, tɤ-tɕɯ cʰɯ-kɯ-rɤrɟit tʰɤjtɕu pjɤ-ŋgrɤl? \\
\textsc{3sg}.\textsc{poss}-grandfather \textsc{sbj}:\textsc{pcp}-have.a.child \textsc{aor}-go[II] \textsc{sfp} \textsc{qu} \textsc{indef}.\textsc{poss}-son \textsc{ipfv}-\textsc{sbj}:\textsc{pcp}-have.a.child when \textsc{ifr}.\textsc{ipfv}-be.usually.the.case \\
\glt `(You say) that your father-in-law went away to give birth to a child, since when can a man bear children?' (tAwa kWCqraR, 99)
\end{exe} 

The element \forme{tʰɤ-} in the pronouns \japhug{tʰɤjtɕu}{when}  and \japhug{tʰɤstɯɣ}{how many} is the \textit{status constructus} form of proto-Japhug \forme{*tʰi}, the inherited form of the pronoun `what' (see §\ref{sec:tChi}). The element \forme{-tɕu} in \japhug{tʰɤjtɕu}{when} is related to the locative \forme{tɕu} (§\ref{sec:core.locative}).

The prefixal form \forme{ɕɯstɤ-} instead of \forme{tʰɤstɯ-} is attested in one story told by Kunbzang Mtsho, probably influence from Tshobdun (\ref{ex:CWstAJom}).

\begin{exe}
\ex \label{ex:CWstAJom}
\gll nɤmkʰa ɕɯstɤ-ɟom ku-nɯ-tu kɯɣe?\\
sky how.many-fathom \textsc{dubit}-\textsc{auto}-exist \textsc{sfp} \\
\glt `How many fathoms (high) is the sky?' (tAwa kWCqraR, 106)
\end{exe}%930

\subsection{\japhug{ŋotɕu}{where}} \label{sec:NotCu}

The interrogative pronoun \japhug{ŋotɕu}{where} and its variant form \forme{ŋoj} (§\ref{sec:locative.j}) can be used to ask either about a location (\ref{ex:NotCu.kutWrAZi}), a direction towards (examples \ref{ex:NotCu.tWCe} and \ref{ex:Noj.nari}) or from (\ref{ex:NotCu.jAtWGenW}) a certain place. The second syllable of this pronoun \forme{-tɕu} comes from the locative postposition \forme{tɕu}, but the first part is etymologically obscure.
 
\begin{exe}
\ex \label{ex:NotCu.kutWrAZi}
\gll     ŋotɕu ku-tɯ-rɤʑi?   \\
  where \textsc{pres}.\textsc{egoph}-2-stay \\
\glt `Where are you?' (Conversation, 2005)
\end{exe} 

\begin{exe}
\ex \label{ex:NotCu.tWCe}
\gll   ŋotɕu tɯ-ɕe? \\
 where 2-go:\textsc{fact} \\
\glt `Where are you going to?' (Common greeting used when one meets someone on the road)
 \end{exe} 
 
\begin{exe}
\ex \label{ex:Noj.nari}
\gll     qala ŋoj nɯ-ari?  \\
  rabbit where \textsc{aor}:\textsc{west}-go[II] \\
\glt `Where did the rabbit go?'  (qala2002, 21)
\end{exe} 

\begin{exe}
\ex \label{ex:NotCu.jAtWGenW}
\gll  nɯʑɤra ŋotɕu jɤ-tɯ-ɣe-nɯ? ŋotɕu ɕ-pɯ-tɯ-tu-nɯ? \\
\textsc{2pl} where \textsc{aor}-2-come[II]-\textsc{pl} where \textsc{tral}-\textsc{pst}.\textsc{ipfv}-2-exist-\textsc{pl} \\
\glt `Where are you from? Where have you been?' (2003sras, 57)
\end{exe} 

Likewise, with verbs of manipulation, \forme{ŋotɕu} can be used both from questions about origin (as in \ref{ex:NotCu.jAtWGWt}) or destination.

\begin{exe}
\ex \label{ex:NotCu.jAtWGWt}
\gll nɯnɯ ŋotɕu nɤ-jaʁ nɯ-ɣe ŋu, ŋotɕu jɤ-tɯ-ɣɯt? \\
\textsc{dem} where \textsc{2sg}.\textsc{poss}-hand \textsc{aor}-come[II] be:\textsc{fact} where \textsc{aor}-2-bring \\
\glt `Where did you get it from, from where did you bring it?' (150831 renshen wawa-zh, 53)
\end{exe} 

With the determiner \forme{nɯ}, the pronoun \forme{ŋotɕu} means `which (of several places)', as in (\ref{ex:NotCu.nW.Nu}) and (\ref{ex:NotCu.nW.Wku.Nu}).

\begin{exe}
\ex \label{ex:NotCu.nW.Nu}
\gll kʰa raŋri ɣɯ ʑo ɯ-ftaʁ pjɤ-tu ɕti ma, tɕe ŋotɕu nɯ ŋu, ŋotɕu nɯ maʁ mɯ-pjɤ-saχsɤl. \\
house each \textsc{gen} \textsc{emph} \textsc{3sg}.\textsc{poss}-mark \textsc{ifr}.\textsc{ipfv}-exist be.\textsc{aff}:\textsc{fact} \textsc{lnk} \textsc{lnk} where \textsc{dem} be:\textsc{fact}  where \textsc{dem} be:\textsc{fact} \textsc{neg}-\textsc{ifr}.\textsc{ipfv}-be.clear \\
\glt `There was a mark on each of the houses, and one could not tell which (house) was (Alibaba's) and which was not.' (140512 alibaba-zh, 189-190)
\end{exe} 

\begin{exe}
\ex \label{ex:NotCu.nW.Wku.Nu}
\gll qaprɤftsa nɯnɯ, cici jɤ-ari tɕe ɯ-ku ju-z-mɤke, cici tɕe ɯ-jme ju-z-mɤke ɲɯ-ɕti tɕe ŋotɕu nɯ ɯ-ku ŋu, ŋotɕu ɯ-jme ŋu, mɯ́j-saχsɤl \\
centipede \textsc{dem} sometimes \textsc{aor}-go \textsc{lnk} \textsc{3sg}.\textsc{poss}-head \textsc{ipfv}-\textsc{caus}-be.first[III] sometimes \textsc{lnk} \textsc{3sg}.\textsc{poss}-tail \textsc{ipfv}-\textsc{caus}-be.first[III] \textsc{sens}-be:\textsc{aff} \textsc{lnk} where \textsc{dem} \textsc{3sg}.\textsc{poss}-head be:\textsc{fact} where \textsc{3sg}.\textsc{poss}-head be:\textsc{fact} \textsc{neg}.\textsc{sens}-be.clear \\
\glt `The centipede, when it moves, sometimes its head goes first, sometimes its tail goes first, it is not each to tell which is its head and which is its tail.' (21-qaprAftsa, 12)
\end{exe} 


With generic nouns such as \japhug{tɯrme}{person}, \forme{ŋotɕu} can serve as prenominal determiner to mean `a person from where', as in (\ref{ex:NotCu.tWrme}).

\begin{exe}
\ex \label{ex:NotCu.tWrme}
\gll ŋotɕu tɯrme tɯ-ŋu? \\
where person 2-be:\textsc{fact} \\
\glt `Where are you from?' (2011-05-nyima, 83)
\end{exe} 

The pronoun \forme{ŋotɕu} however has non-spatial uses, in particular to ask about a particular individual (or a subgroup) within a group, as in (\ref{ex:NotCu.Zo.nW}) and (\ref{ex:NotCu.nW.Zo.sAndAG}).

\begin{exe}
\ex \label{ex:NotCu.Zo.nW}
\gll  aʑo a-wa nɯ ŋotɕu ʑo nɯ ŋu kɯ? \\
\textsc{1sg} \textsc{1sg}.\textsc{poss}-father \textsc{dem} where \textsc{emph} \textsc{dem} be:\textsc{fact} \textsc{sfp} \\
\glt `Which one is my father?' (150831 jubaopen-zh, 177)
\end{exe} 

\begin{exe}
\ex \label{ex:NotCu.nW.Zo.sAndAG}
\gll  mɤʑɯ ŋotɕu nɯ ʑo sɤndɤɣ mɤ-xsi ma \\
more where \textsc{dem} \textsc{emph} be.poisonous:\textsc{fact} \textsc{neg}-know:\textsc{genr} \textsc{lnk} \\
\glt `(Apart from this one), I don't which other (mushrooms) are poisonous.' (23-grWBgrWBftsa, 31)
\end{exe} 

In participial relatives with subject participle (in \forme{kɯ\trt}, see §\ref{sec:subject.participle.subject.relative}), \japhug{ŋotɕu}{where} can occur to express relativization of locative adjuncts, as in  (\ref{ex:NotCu.kWtu}); see §\ref{sec:interrogative.relative} for a discussion of the other available constructions.

\begin{exe}
\ex \label{ex:NotCu.kWtu}
\gll kɯ-me nɯra qʰe me,  ŋotɕu kɯ-tu nɯ qʰe kɯ-dɯ\redp{}dɤn tu-ɬoʁ ŋu. \\
\textsc{sbj}:\textsc{pcp}-not.exist \textsc{dem}:\textsc{pl} \textsc{lnk} not.exist:\textsc{fact} where \textsc{sbj}:\textsc{pcp}-exist \textsc{dem} \textsc{lnk} \textsc{sbj}:\textsc{pcp}-\textsc{emph}\redp{}be.many \textsc{ipfv}-come.out be:\textsc{fact} \\
\glt `In (places) where it is not found, there is none, but in (places) where it is found, it grows in great number.' (21-jmAGni, 91)
\end{exe} 

The pronoun \japhug{ŋotɕu}{where} is not exclusively used in question about place or direction, we also find it in the expression in (\ref{ex:NotCu.YWNgrAl}).

 \begin{exe}
\ex \label{ex:NotCu.YWNgrAl}
\gll  kɯki ŋotɕu ɲɯ-ŋgrɤl?   \\
 this where \textsc{ipfv}-be.usually.the.case \\
\glt `How could this be possible?'  (qajdoskAt 2002, 32)
\end{exe} 

This sentence is used to express indignation (as in Chinese \zh{哪有这样的道理?}).\footnote{In the story from which it is quoted, the husband says this sentence after his wife, quoting the words of a raven, says that she will have luck, not her husband, who thus reacts in anger. }




\section{Indefinite pronouns} \label{sec:indef.pro}
 Japhug has a handful of indefinite pronouns, indicated in \tabref{tab:indef.pronoun}. They do not form a complete paradigm, and other constructions, in particular generic nouns and free relatives occur to express meanings for which no indefinite pronoun exists (see §\ref{sec:headless.relatives.quantification}).

There are no negative indefinite pronouns, and indefinite pronouns are almost never under the scope of negation (except in translations from Chinese). They also never occur as standard of comparison.\footnote{Examples such as `In Freiburg the weather is better than anywhere in Germany' (\citealt[2]{haspelmath97indef}) would not be expressible with an indefinite pronoun.}
 

\begin{table}[H] \centering
\caption{Indefinite pronouns }\label{tab:indef.pronoun}
\begin{tabular}{lllllll} \lsptoprule
\japhug{ci}{one, someone} \\
\forme{tʰɯci}, \japhug{tʰɯtʰɤci}{something} \\
\japhug{tsʰitsuku}{whatever} \\
\japhug{ɕɯmɤɕɯ}{whoever, anybody} \\
\japhug{ciscʰiz}{somewhere} \\ 
\lspbottomrule
\end{tabular}
\end{table}

\subsection{\japhug{ci}{someone} } \label{sec:ci.someone} 
There is no indefinite pronoun `someone' in Japhug, but the numeral \japhug{ci}{one}, which has many additional functions (indefinite article, modifier and partitive pronoun §\ref{sec:indef.article}, §\ref{sec:other.pro}, §\ref{sec:partitive.pronouns}, §\ref{sec:identity.modifier}, §\ref{sec:one.to.ten} and §\ref{sec:tense.aspect.adverbs}), can express this meaning as in (\ref{ci.kW.thaGWt}) and (\ref{ci.kW.tWrdoR}).

\begin{exe}
\ex \label{ci.kW.thaGWt}
\gll ɯ-lɤcu nɯtɕu qaʑo kɤtsa ci, ci kɯ kɤ-ntsɣe tʰa-ɣɯt ɲɯ-ŋu. \\
\textsc{3sg}.\textsc{poss}-upstream \textsc{dem}:\textsc{loc} sheep parent.and.child \textsc{indef} one \textsc{erg} \textsc{obj}:\textsc{pcp}-sell \textsc{aor}:3\flobv{}-bring \textsc{sens}-be \\
\glt `Upstream from there, (there was) a ewe and her young, that someone had brought them to sell.' (2003 kandZislama, 202)
\end{exe}

\begin{exe}
\ex \label{ci.kW.tWrdoR}
\gll  tɯ-xpa tɕe ci kɯ tɯ-rdoʁ pjɤ-sat. \\
 one-year \textsc{lnk} one \textsc{erg} one-piece \textsc{ifr}-kill \\
 \glt `One year, someone killed one of them (wils geese).' (22-qomndroN, 43)
\end{exe}

This use of \forme{ci} is rare. The preferred construction to express the meaning `someone' involves the combination of the generic noun \japhug{tɯrme}{person} with the indefinite \forme{ci}.

\subsection{\japhug{tʰɯci}{something} } \label{sec:thWci} 
The indefinite pronoun \japhug{tʰɯci}{something} derives from the \textit{status constructus} of the proto-Japhug pronoun \forme{*tʰi} `what' (see §\ref{sec:tChi} above) with the indefinite determiner and numeral \japhug{ci}{one}. Note that vowel alternation bleeds the sound change \ipa{*tʰi} \fl{}  \ipa{tsʰi}, otherwise a form such as $\dagger$\forme{tsʰɯci} would have been expected. Its reduplicated form \forme{tʰɯtʰɤci} has an irregular vocalism \ipa{ɤ} ($\dagger$\forme{tʰɯtʰɯci} would have been expected instead).

 It can designate specific referents, whose nature is known to the speaker but unknown to the addressee (as in \ref{ex:thWthAci.Zo.pjWtu}),\footnote{Example (\ref{ex:thWthAci.Zo.pjWtu}) is from a tale about a rabbit tricking a snow leopard; the difference of knowledge between the speaker and the addressee concerning the nature of the `something' is crucial to the plot. }.

\begin{exe}
\ex  \label{ex:thWthAci.Zo.pjWtu}
\gll tu-nɯsman-a jɤɣ ri, mɤʑɯ ɯ-ftɕaka tsuku pjɯ-tu ra wo, tɕe tʰɯtʰɤci ʑo pjɯ-tu ra \\
\textsc{ipfv}-treat-\textsc{1sg} be.possible:\textsc{fact} but yet \textsc{3sg}.\textsc{poss}-manner some \textsc{ipfv}-exist be.needed:\textsc{fact} \textsc{sfp} \textsc{lnk} something \textsc{emph} \textsc{ipfv}-exist be.needed:\textsc{fact} \\
\glt `I can treat (your illness), but yet another method is needed, something (else) is needed.'  (140427 qala cho kWrtsAg, 48-49)
\end{exe}

The pronoun \forme{tʰɯci} also occurs to refer to things whose name is unknown to the speaker (as in \ref{ex:gser.zhwa} and \ref{ex:thWci.khWtsa}), even if he/she may have seen the object.
 
\begin{exe}
\ex \label{ex:gser.zhwa}
\gll tɕe nɯ nɯ-rte nɯ tɕʰi ŋu ma tʰɯci ci ``-ʑa" tu-ti ŋu, χsɤrʑa! \\
\textsc{lnk} \textsc{dem} \textsc{3pl}.\textsc{poss}-hat \textsc{dem} what be:\textsc{fact} \textsc{lnk} something \textsc{indef} {  ...} \textsc{ipfv}-say be:\textsc{fact} golden.hat \\
\glt `How is their hat (called), something in `ʑa'.... yes, \tibet{གསེར་ཞྭ་}{gser.ʑʷa}{golden hat}!' (30-mboR, 102)
\end{exe}

\begin{exe}
\ex \label{ex:thWci.khWtsa}
\gll  tɕe tɤ-ndʑɯɣ nɯ kɯnɤ, tʰɯci kʰɯtsa kɯ-fse ɯ-ŋgɯ tu-rku-nɯ tɕe   \\
\textsc{lnk} \textsc{indef}.\textsc{poss}-resin \textsc{dem} also something bowl \textsc{sbj}:\textsc{pcp}be.like \textsc{3sg}-inside \textsc{ipfv}-put.in-\textsc{pl} \textsc{lnk}   \\
\glt `The resin, people put it into something like a bowl.'' (07-tAtho, 44)
\end{exe}

It is also used for non-specific referents whose nature is entirely unknown, as in  (\ref{ex:thWthAci.tannWrkunW}) and (\ref{ex:thWmqlaR}).

\begin{exe}
\ex \label{ex:thWthAci.tannWrkunW}
\gll   tɕe mɤʑɯ tʰɯtʰɤci ta-nnɯ-rku-nɯ kɯma  \\
\textsc{lnk} yet something \textsc{aor}:3\flobv{}-\textsc{auto}-put.in-\textsc{pl} \textsc{sfp} \\
\glt `They also probably gave them something else.' (02-deluge2012, 120)
 \end{exe}
 
  \begin{exe}
\ex \label{ex:thWmqlaR}
\gll  tʰɯ-mqlaʁ tʰɯ-mqlaʁ ma tʰɯci fse ci ndʐa cʰɯ-ɕe ɕti \\
 \textsc{imp}:swallow  \textsc{imp}:swallow \textsc{lnk} something be.like:\textsc{fact} \textsc{indef} reason \textsc{ipfv}:\textsc{downstream}-go be.\textsc{aff}:\textsc{fact} \\
\glt `Swallow it, swallow it, it comes down (into your throat) for some reason.' (2005-stod-kunbzang, 87)
  \end{exe}

The reduplicated form \forme{tʰɯ\redp{}tʰɤci}, especially in combination with \japhug{fse}{be like}, can also mean `whatever (happened)', as in (\ref{ex:thWthAci.kWfse}).
 
 \begin{exe}
\ex \label{ex:thWthAci.kWfse}
\gll  slama ra ɣɯ tʰɯtʰɤci kɯ-fse, kɤ-rɤ-βzjoz ra ɲɯ-stu mɯ́j-stu-nɯ, nɯ-stu ɲɯ-nɤma-nɯ mɯ́j-nɤma-nɯ,  nɯnɯra nɯ-pʰama ra nɯ-ɕki kɯ-rɤfɕɤt ɲɯ-ra. \\
student \textsc{pl} \textsc{gen} something \textsc{sbj}:\textsc{pcp}-be.like \textsc{inf}-\textsc{antipass}-learn \textsc{pl} \textsc{sens}-try.hard-\textsc{pl} \textsc{neg}:\textsc{sens}-try.hard-\textsc{pl} \textsc{3sg}.\textsc{poss}-right \textsc{sens}-do-\textsc{pl} \textsc{neg}:\textsc{sens}-do-\textsc{pl} \textsc{dem}:\textsc{pl} \textsc{3pl}.\textsc{poss}-parent \textsc{pl} \textsc{3pl}-\textsc{dat} \textsc{genr}:S/O-tell \textsc{sens}-be.needed \\
\glt `One has to tell the parents whatever concerns the students, whether they study seriously and try hard or not.'   (150901 tshuBdWnskAt, 18)
 \end{exe}
  
The non-reduplicated form \forme{tʰɯci} occurs in a correlative construction with the form \forme{mɯci} to mean `this and that', an expression that is used especially in reporting speech from another person when the speaker does not want to bother reporting in details the exact words that have been said.

\begin{exe}
\ex \label{ex:thWci.mWci}
\gll tʰɯci nɤme-a ra, mɯci nɤ-me-a ra \\
something do[III]:fact-\textsc{1sg} be.needed:\textsc{fact} something do[III]:fact-\textsc{1sg} be.needed:\textsc{fact} \\
\glt `I have to do this and that (so I cannot do X).' (elicitation)
 \end{exe}
 
The pronoun \japhug{tʰɯci}{something}  can also occur as head of a relative clause as in (\ref{ex:thWci.khWtsa}) above with the relative \forme{tʰɯci kʰɯtsa kɯ-fse} `something which is like a bowl'. This use is most common in texts translated from Chinese, with the indefinite article \japhug{ci}{one} (§\ref{sec:indef.article}) following relative clause, as in (\ref{ex:thWci.akAspa}). 

\begin{exe}
\ex \label{ex:thWci.akAspa}
\gll  laχɕi ci pjɯ-βzjoz-a, tʰɯci a-kɤ-spa ci a-pɯ-tu ɲɯ-ra  \\
 trade \textsc{indef} \textsc{ipfv}-learn-\textsc{1sg} something \textsc{1sg}.\textsc{poss}-\textsc{obj}:\textsc{pcp}-be.able \textsc{indef} \textsc{irr}-\textsc{pfv}-exist \textsc{sens}-be.needed \\
 \glt `I have to learn a trade, to have something I am able to do.' (150902 luban-zh, 12)
\end{exe}
 
With stative verbs in the relative as in (\ref{ex:thWci.kApGWlu}), this construction has a low degree meaning `a little X'.

\begin{exe}
\ex \label{ex:thWci.kApGWlu}
\gll   tɕe kɯ-wɣrum ɯ-ŋgɯz kɯnɤ tʰɯci kɯ-ɤpɣɯlu kɯ-fse ci ŋu tɕe, \\
\textsc{lnk} \textsc{sbj}:\textsc{pcp}-be.white \textsc{3sg}.\textsc{poss}-inside:\textsc{loc} also something \textsc{sbj}:\textsc{pcp}-greyish \textsc{sbj}:\textsc{pcp}-be.like \textsc{indef} be:\textsc{fact} \textsc{lnk} \\
\glt `(Silver) is white with a little greyish colour.' (30-Com, 176)
\end{exe}
  
 The reduplicated form of the the indefinite pronoun \forme{tʰɯtʰɤci} can be used as an interrogative pronoun, as in (\ref{ex:thWthAci.totia}). This construction is similar in meaning to Chinese \ch{一些什么}{yīxiēshénme}{what kinds of things}, and is attested in particular with the verbs \japhug{ti}{say} and \japhug{ra}{have to, need}. By using this form, the speaker implies that the addressee necessarily knows the answer to the question. For instance,  in (\ref{ex:thWthAci.totia}), a sentence from a text enumerating the mountain names in Kamnyu, the names had been written before hand on a piece of paper, and I was reading them one by one to Tshendzin; given the fact that the name had been written down, it was obvious that I necessarily knew the answer to that question (on the use of the Inferential in this example, see §\ref{sec:inf.1person}).
  
 \begin{exe}
\ex \label{ex:thWthAci.totia}
 \gll  nɯ ɯ-pa tʰɯtʰɤci to-ti-a? \\
 \textsc{dem} \textsc{3sg}.\textsc{poss}-down something \textsc{ifr}-say-\textsc{1sg} \\
 \glt `What did I say after that?' (140522 Kamnyu zgo, 58)
\end{exe}

 There are very marginal examples of \japhug{tʰɯci}{something} used as an indefinite prenominal determiner (§\ref{sec:indefinite}).

The pronoun \japhug{tʰɯci}{something} can take various modifiers, for instance the identity modifier \japhug{kɯmaʁ}{other} (§\ref{sec:identity.modifier})  as in (\ref{ex:kWmaR.thWci}). 
 
\begin{exe}
\ex \label{ex:kWmaR.thWci}
\gll    ki mbro ki ɲɯ-kɤ-ntsɣe tɕe, [kɯmaʁ tʰɯci] ɲɯ-kɤ-sɤndu to-nɯkrɤz-ndʑi \\
\textsc{dem}:\textsc{prox} horse \textsc{dem}:\textsc{prox} \textsc{ipfv}-\textsc{inf}-sell \textsc{lnk} other  something   \textsc{ipfv}-\textsc{inf}-exchange \textsc{ifr}-discuss-\textsc{du} \\
 \glt `They discussed about selling their horse, and exchanging it for something else.' (150822 laoye zuoshi zongshi duide-zh, 41)
\end{exe}

No example of \forme{tʰɯci} with topic markers contributing to mark definiteness such as \forme{nɯ} or \forme{iɕqʰa} (§\ref{sec:definiteness}) have been found in the corpus.

\subsection{\japhug{tsʰitsuku}{whatever}} \label{sec:tshitsuku}
The pronoun \japhug{tsʰitsuku}{whatever} combines the  interrogative pronoun \japhug{tsʰi}{what} (replaced by \japhug{tɕʰi}{what}, a borrowing from Tibetan, in Kamnyu Japhug, but still attested in Mangi village, see §\ref{sec:tChi} above) with the mid-scalar quantifier  \japhug{tsuku}{some} (see §\ref{sec:tsuku}; also found as a partitive pronoun, §\ref{sec:partitive.pronouns}).  Unlike  \japhug{tʰɯci}{something}, is not used for specific referents.  Example (\ref{ex:tshitsuku.kuwGsqa}) illustrates its most common use. The variant form \forme{tʰitsuku}, without the sound change \forme{*tʰi} \fl{} \forme{tsʰi}, is also used by speakers of the Kamnyu dialect.

\begin{exe}
\ex \label{ex:tshitsuku.kuwGsqa}
\gll  
kɤ-nɯ-βlɯ tɕe ɕkrɤz wuma ʑo pe ma nɯnɯ, nɯnɯ ɣɯ ɯ-smɯmba nɯ sɤɕke, tɕendɤre tsʰitsuku kú-wɣ-sqa tɕe, ʑaʑa ʑo ku-ɣɤ-smi cʰa, tsʰitsuku tú-wɣ-sɯ-ɤla tɕe, ʑaʑa tu-sɯ-ɤle cʰa. \\
\textsc{inf}-\textsc{auto}-burn \textsc{lnk} oak really \textsc{emph} be.good:\textsc{fact} \textsc{lnk} \textsc{dem} \textsc{dem} \textsc{gen} \textsc{3sg}.\textsc{poss}-flame \textsc{dem} burning \textsc{lnk} whatever \textsc{ipfv}-\textsc{inv}-cook \textsc{lnk} soon \textsc{emph}  \textsc{ipfv}-\textsc{caus}-be.cooked can:\textsc{fact} whatever \textsc{ipfv}-\textsc{inv}-\textsc{caus}-be.boiling \textsc{lnk} soon  \textsc{ipfv}-\textsc{caus}-\textsc{caus}-be.boiling[III] can:\textsc{fact} \\
\glt `For burning, oak is very good, the flames (from its wood) are very hot, whatever one cooks, it cooks it quickly, whatever one boils, it boils it quickly.' (08-CkrAz, 4-5)
\end{exe}
%nɤʑo kɯ rcanɯ, tɯ-tso ɯ-tɯ-me nɯ, maka /ji/ tɕi-rca jɤ-ɣi tɕe, nɯ sɤznɤ tshitsuku a-pɯ-tɯ-mtɤm tɕe a-pɯ-tɯ-nɯtɯtso ɲɯ-mna
%140510_fengwang, 15
 
In many cases, it is better translated as `all kinds of things', as in (\ref{ex:tshitsuku.YWznAme}).

\begin{exe}
\ex \label{ex:tshitsuku.YWznAme}
\gll  
tɕe nɯtɕu kɯnɤ ɯ-jaʁ ɯ-ntsi tɤɲi pjɯ-sɤtse  ɯ-jaʁ ɯ-ntsi kɯ tsʰitsuku ɲɯ-z-nɤme qhe, ʑara nɯ-ndzɤtsʰi tu-βze, fsapaʁ ra nɯ-ndzɤtsʰi ɲɯ-βze \\
\textsc{lnk} \textsc{dem}:\textsc{loc} also \textsc{3sg}.\textsc{poss}-hand \textsc{3sg}.\textsc{poss}-one.of.a.pair staff \textsc{ipfv}-plant[III]  \textsc{3sg}.\textsc{poss}-hand \textsc{3sg}.\textsc{poss}-one.of.a.pair \textsc{erg} whatever \textsc{ipfv}-\textsc{caus}-do[III] \textsc{lnk} \textsc{3pl} \textsc{3pl}.\textsc{poss}-food \textsc{ipfv}-make[III] animal \textsc{pl} \textsc{3pl}.\textsc{poss}-food \textsc{ipfv}-make[III]  \\
\glt `Even like that, she supports herself with a staff in one hand, and with the other hand she does all kinds of things, makes their food, she makes food for the animals.' (14-siblings, 54)
\end{exe}
%tshitsuku ɲɯ́-wɣ-mbi, ɲɯ́-wɣ-jtshi, tú-wɣ-raχtɕɤz tɕe, ʑɯrɯʑɤri tɕe tɕendɤre ku-kɯ-nɯfse ɲɯ-ŋu

As other indefinite pronouns, \japhug{tsʰitsuku}{whatever} is not normally used with negation, but such sentences do occur in the corpus in translations from Chinese, as (\ref{ex:tshitsuku.mWtoti}). They are not idiomatic Japhug, and only marginally grammatical.

\begin{exe}
\ex \label{ex:tshitsuku.mWtoti}
\gll   tsʰitsuku mɯ-to-ti, qʰe tɕendɤre kɯ-rŋgɯ jo-nɯɕe qʰe ko-nɯ-rŋgɯ. \\
whatever \textsc{neg}-\textsc{ifr}-say \textsc{lnk} \textsc{lnk} \textsc{sbj}:\textsc{pcp}-lay.down \textsc{ifr}-go.back \textsc{lnk} \textsc{ifr}-\textsc{auto}-lay.down \\
\glt `He did not said anything, went back to sleep and laid down in bed.' (150902 qixian-zh, 91)
\end{exe}

 \subsection{\japhug{ɕɯmɤɕɯ}{whoever, anybody}} \label{sex:CWmACW}
 There is no indefinite pronoun for human referents `somebody' in Japhug  corresponding to \japhug{tʰɯci}{something} -- a generic noun with the indefinite determiner \japhug{ci}{one} such as \forme{tɯrme ci} `a man' is used instead. There is nevertheless a free choice pronoun \japhug{ɕɯmɤɕɯ}{whoever, anybody} (see \citealt[48--52]{haspelmath97indef} on the differences with universal quantifiers), which, however, is not very common. As example (\ref{ex:CWmACW.kW}) shows, it can take the ergative \forme{kɯ}, and the verb receives plural indexation (§\ref{sec:genr.3pl}). 
 
 \begin{exe}
\ex \label{ex:CWmACW.kW}
\gll tɕaχkɤr kʰɯtsa nɯ ʁo tʰam qʰe ɕɯmɤɕɯ kɯ ku-nɯ-ntɕʰoz-nɯ ɕti \\
tin bowl \textsc{dem} \textsc{advers} now \textsc{lnk} anybody \textsc{erg} \textsc{ipfv}-\textsc{auto}-use-\textsc{pl} be.\textsc{aff}:\textsc{fact} \\
\glt `Now anybody can use tin bowls.' (unlike before, when only important people could use it, 160702 khWtsa, 26)
 \end{exe}
 
  \subsection{\japhug{ciscʰiz}{somewhere}} \label{sec:cischiz}
The indefinite pronoun \japhug{ciscʰiz}{somewhere} comprises the indefinite \japhug{ci}{one} and the approximate locative \forme{(s)cʰiz} (see §\ref{sec:approximate.locative}). It occurs with or without the locative postposition \forme{ri}, as in (\ref{ex:cischiz}) and (\ref{ex:cischiz.ri}). It can refer to static location, or motion from or towards a direction.

 \begin{exe}
\ex \label{ex:cischiz}
\gll
ciscʰiz, tɤtsʰoʁ ɯ-taʁ kɯ-fse, tɤ-jtsi ɯ-taʁ kɯ-fse, nɯnɯra, nɯnɯtɕu kú-wɣ-βraʁ tɕe, \\
somewhere nail \textsc{3sg}-\textsc{on} \textsc{sbj}:\textsc{pcp}-be.like, \textsc{indef}.\textsc{poss}-pillar \textsc{3sg}-\textsc{on} \textsc{sbj}:\textsc{pcp}-be.like,  \textsc{dem}:\textsc{pl} \textsc{dem}:\textsc{loc} \textsc{ipfv}-\textsc{inv}-attach \textsc{lnk} \\
\glt `One attaches (their noseband) somewhere, like on a nail, on a pillar.' (150902 kAxtCAr, 6)
 \end{exe}
 
 \begin{exe}
\ex \label{ex:cischiz.ri}
\gll nɯnɯ ciscʰiz ri tú-wɣ-z-nɯndzɯ tɕe ɲɯ́-wɣ-ta.\\
\textsc{dem} somewhere  \textsc{loc} \textsc{ipfv}-\textsc{inv}-\textsc{caus}-be.vertical \textsc{lnk} \textsc{ipfv}:\textsc{west}-\textsc{inv}-put\\
\glt `One puts it vertically somewhere.' (14-tasa, 62)
 \end{exe}
  
 
\subsection{Interrogative pronouns used as free-choice indefinites} \label{sec:interrogative.indef}
Non-specific free-choice (\citealt[48]{haspelmath97indef}) indefinite referents can be expressed by interrogative pronouns in Japhug. Constructions where this function is attested include correlatives (§\ref{sec:interrogative.relative}), as in (\ref{ex:NotCu.lAtWrNgW}) and universal concessive conditionals (§\ref{sec:universal.concessive.conditional}) as in (\ref{ex:thAjtCu.fsaN}).
 
\begin{exe}
\ex \label{ex:NotCu.lAtWrNgW}
\gll a-pɯwɯ, [ŋotɕu lɤ-tɯ-rŋgɯ ʑo qʰe], nɯtɕu rɤʑi-tɕi ŋu ma, \\
\textsc{1sg}-donkey where \textsc{aor}:\textsc{upstream}-2-lay.down \textsc{emph} \textsc{lnk} \textsc{dem}:\textsc{loc} stay:\textsc{fact}-\textsc{1du} be:\textsc{fact} because \\
\glt `My donkey, we will stay wherever you lay down.' (28-qAjdoskAt, 38)
\end{exe}  
 
\begin{exe}
\ex \label{ex:thAjtCu.fsaN}
\gll [tʰɤjtɕu fsaŋ kɤ-ta tɤ-ra] ʑo tɕe nɯnɯ tu-βlɯ-nɯ tɕe, \\
when fumigation \textsc{inf}-put \textsc{aor}-be.needed \textsc{emph} \textsc{lnk} \textsc{dem} \textsc{ipfv}-burn-\textsc{pl} \textsc{lnk} \\
\glt `Whenever there is need to make fumigations, they burn it.' (15-YaBrWG, 31)
\end{exe}  

This meaning also occurs in infinitival subordinate clauses, in particular in the expression \forme{tɕʰi kɤ-cʰa} `do whatever $X$ can to $Y$', as in example (\ref{ex:tChi.kAcha.Zo}).

\begin{exe}
\ex \label{ex:tChi.kAcha.Zo}
\gll  [tɕʰi kɤ-cʰa] ʑo cʰɯ-pʰɯt-nɯ, \\
what \textsc{inf}-can \textsc{emph} \textsc{ipfv}-remove-\textsc{pl} \\
\glt `People do whatever they can to remove (this plant).' (12-Zmbroko, 119)
\end{exe}

In both correlative relatives and universal concessive conditionals, the interrogative pronoun often occurs with verb with partial reduplication on the last syllable of the stem and/or the autive \forme{nɯ-} prefix (§\ref{sec:autoben.spontaneous}, §\ref{sec:universal.concessive.conditional}).

With \japhug{tɕʰi}{what}, this construction expresses the meaning `whatever; no matter what' in intransitive subject (\ref{ex:tChi.pWnWNWNu}), object (\ref{ex:tChi.tAtWnWtWtWt}) or semi-object (\ref{ex:tChi.kWstWstua}, see §\ref{sec:ditransitive.secundative}) functions.

\begin{exe}
\ex \label{ex:tChi.pWnWNWNu}
\gll lú-wɣ-sti tɕe tɕe nɯ ɯ-ŋgɯ [tɕʰi pɯ-nɯ-ŋɯ\redp{}ŋu] nɯ ɲɯ-mɲɤt mɯ́j-cʰa \\
\textsc{ipfv}-\textsc{inv}-block \textsc{lnk} \textsc{lnk} \textsc{dem} \textsc{3sg}-inside what \textsc{pst}.\textsc{ipfv}-\textsc{auto}-be \textsc{dem} \textsc{ipfv}-be.spoiled \textsc{neg}:\textsc{sens}-can \\
\glt `One seals (its opening) and whatever (food) is inside will not be spoiled.' (150828 kodAt, 14)
\end{exe}  

\begin{exe}
\ex \label{ex:tChi.tAtWnWtWtWt}
\gll [tɕʰi tɤ-tɯ-nɯ-tɯ\redp{}tɯt] ʑo ju-ɣi ɕti \\
what \textsc{aor}-2-\textsc{auto}-say[II] \textsc{emph} \textsc{ipfv}-come be.\textsc{aff}:\textsc{fact} \\
\glt  `Whatever you say will come.' (2003twxtsa, 117)
\end{exe}  

\begin{exe}
\ex \label{ex:tChi.kWstWstua}
\gll nɤʑo tɕʰi kɯ-stɯ\redp{}stu-a ʑo ŋu \\
\textsc{2sg} what 2\fl1-do.like-\textsc{1sg} \textsc{emph} be:\textsc{fact} \\
\glt `Whatever you do to me (will be fine).' (28-qAjdoskAt, 40)
\end{exe}


With \japhug{ɕɯ}{who}, the construction means `whoever; regardless of who; no matter who'. Examples are found with the non-specific referent in intransitive subject (\ref{ex:CW.pWnWNWNu}), transitive subject (\ref{ex:CW.kW.panWmtWmtonW}), or oblique argument (\ref{ex:CW.GW.nWnWkhWkhota}) functions. Note that it often occurs with plural indexation.

\begin{exe}
\ex \label{ex:CW.pWnWNWNu}
\gll tɯsqar nɯ kɯrɯ tɯrme ra mɤ-kɯ-rga maka ʑo me, [ɕɯ pɯ-nɯ-ŋɯ\redp{}ŋu] ʑo, tɯsqar a-pɯ-tu qʰe, tɕendɤre, nɯ-kɤ-ndza tu-rtaʁ ɕti, \\
tsampa \textsc{dem} Tibetan person \textsc{pl} \textsc{neg}-\textsc{sbj}:\textsc{pcp}-like at.all \textsc{emph} not.exist:\textsc{fact} who  \textsc{pst}.\textsc{ipfv}-\textsc{auto}-be \textsc{emph} tsampa \textsc{irr}-\textsc{ipfv}-exist \textsc{lnk} \textsc{lnk} \textsc{3pl}.\textsc{poss}-\textsc{obj}:\textsc{pcp}-eat \textsc{ipfv}-be.enough be:\textsc{aff}:\textsc{fact} \\
\glt `Among Tibetan people, everybody likes tsampa (`there is no one who does not like it'), no matter who, if they have tsampa, they have enough to eat.' (2002tWsqar2, 9)
\end{exe}

\begin{exe}
\ex \label{ex:CW.kW.panWmtWmtonW}
\gll tɕe [ɕɯ kɯ pa-nɯ-mtɯ\redp{}mto-nɯ] ʑo kɯki ɣɯ, nɯ-kʰa ɣɯ nɯ-mɯntoʁ nɯ cʰondɤre nɯ-ɕoŋpʰu nɯra tɕe, mɤʑɯ nɯ-<cai> nɯra, pjɯ-ɣɤmɯ-nɯ tɕe, \\
\textsc{lnk} who \textsc{erg} \textsc{aor}:3\flobv{}-see-\textsc{pl} \textsc{emph} \textsc{dem}.\textsc{prox} \textsc{gen} \textsc{3pl}.\textsc{poss}-house \textsc{gen} \textsc{3pl}.\textsc{poss}-flower \textsc{dem} \textsc{comit} \textsc{3pl}.\textsc{poss}-tree \textsc{dem}:\textsc{pl} \textsc{lnk} yet \textsc{3pl}.\textsc{poss}-vegetable \textsc{dem}:\textsc{pl} \textsc{ipfv}-praise-\textsc{pl} \textsc{lnk} \\
\glt `Whoever saw it, the flowers and the trees and the vegetables of their house, they praised it.' (150824 yuanding-zh, 30)
\end{exe}


\begin{exe}
\ex \label{ex:CW.GW.nWnWkhWkhota}
\gll tɤɕime ri tɯ-rdoʁ ma me, tɕendɤre nɯʑo [ɕɯ ɣɯ nɯ-nɯ-kʰɯ\redp{}kʰo-t-a] ʑo mɯ́j-nɯtɯtʂaŋ ɕti tɕe, \\
lady also one-piece apart.from not.exist:\textsc{fact} \textsc{lnk} \textsc{2pl} who \textsc{gen} \textsc{aor}-\textsc{auto}-give-\textsc{pst}:\textsc{tr}-\textsc{1sg} \textsc{emph} \textsc{neg}:\textsc{sens}-be.fair be.\textsc{aff}:\textsc{fact} \textsc{lnk} \\
\glt `There is only one princess, and regardless of whom among you all I give her hand to, it will be unfair.' (140508 benling gaoqiang de si xiongdi-zh, 227)
\end{exe}

Universal concessive conditionals are found with the pronoun \japhug{ŋotɕu}{where}, with the meaning `no matter where, wherever' (location or direction from or to), as in (\ref{ex:nWGtWta}). This free-choice indefinite meaning of  \forme{ŋotɕu} in also found in the delocutive expression \japhug{ŋɤtɕɯkɤti,kʰɯ}{obey to everything}, where the interrogative pronoun occurs in  \textit{status constructus} form \forme{ŋɤtɕɯ-} with the infinitive of \japhug{ti}{speak}  (§\ref{sec:lexicalized.velar.inf}).

\begin{exe}
\ex \label{ex:nWGtWta}
\gll  [ŋotɕu nɯ́-wɣ-tɯ\redp{}ta] ʑo kɯpɤz ɲɯ-βze ɲɯ-ɕti\\
 where \textsc{ipfv}-\textsc{inv}-\textsc{indefinite}\textasciitilde{}put \textsc{emph} type.of.bug \textsc{ipfv}-grow \textsc{sens}-be.\textsc{aff}\\
\glt `Bugs will grow wherever you put (the meat).' (28-kWpAz, 48)
\end{exe}


No example of multiple partitive use of interrogatives (as in French \textit{qui apportait un fromage, qui un sac de noix}, \citealt[177]{haspelmath97indef} ) is attested in the data at hand; mid-scalar quantifiers such as \japhug{tsuku}{some} occur instead as partitive pronouns (§\ref{sec:partitive.pronouns}). 


\section{Quantifiers} \label{sec:quantifiers.pronouns} \label{sec:aRandWndAt}


\subsection{Universal quantifiers} \label{sec:universal.pronouns}
Several quantifiers meaning `all' exist in Japhug (§\ref{sec:universal.quant}). Among them, \japhug{kɤsɯfse}{all} can be used in the meaning `everybody', as in example (\ref{ex:kAsWfse.kW}).

\begin{exe}
\ex \label{ex:kAsWfse.kW}
\gll kɤsɯfse kɯ ʑo ta-nɯ maʁ \\
all \textsc{erg} \textsc{emph} put:\textsc{fact}-\textsc{pl} not.be:\textsc{fact} \\
\glt `Not everybody puts it.' (160706 thotsi, 21)
\end{exe}

The less common form \japhug{mɲɯrɯri}{everybody, each person} (from Tibetan \tibet{མི་རེ་རེ་}{mi.re.re}{each man} also serves as a universal quantifier, as in (\ref{ex:mYWrWri}).

\begin{exe}
\ex \label{ex:mYWrWri}
\gll mɲɯrɯri kɯ `nɯnɯ ɲɯ-pe' ntsɯ to-ti-nɯ \\
everybody \textsc{erg} \textsc{dem} \textsc{sens}-be.good always \textsc{ifr}-say-\textsc{pl} \\
\glt `Everybody said `It is nice!'' (140521 huangdi de xinzhuang, 214)
\end{exe}

The interrogative pronoun \japhug{tɕʰi}{what}, appears with the plural demonstrative determiner \forme{kɯra} to mean `everything', as in example (\ref{ex:tChi.kWra}). It is not possible to express meanings such as `everybody' or `everywhere'  by combining the other pronouns \japhug{ɕɯ}{who} or \japhug{ŋotɕu}{where} with the same demonstrative.

\begin{exe}
\ex \label{ex:tChi.kWra}
\gll ɯʑo tɕʰi kɯra ko-tso \\
\textsc{3sg} what \textsc{dem}:\textsc{prox:pl} \textsc{ifr}-understand \\
\glt `He understood everything.' (2002qajdoskAt, 115)
\end{exe}

There are several words meaning `everywhere', such as \japhug{aʁɤndɯndɤt}{everywhere}, but they are treated as adverbs rather than pronouns (§\ref{sec:everywhere}).

\subsection{Partitive pronouns} \label{sec:partitive.pronouns}
The mid-scalar quantifier \japhug{tsuku}{some} is used both as a noun determiner (§\ref{sec:tsuku}) and as a partitive pronoun, taking case markers and determiners. This construction expresses a meaning close to that obtained by combining a relative clause with an existential verb (`there is someone who...', §\ref{sec:existential.basic}), as can be seen in (\ref{ex:zgri.mWrkuj}) where both constructions are used one after the other. 

\begin{exe}
\ex \label{ex:zgri.mWrkuj}
\gll tsuku kɯ zgri tu-ti-nɯ ŋu, tsuku kɯ mɯrkuj tu-ti-nɯ ŋu. mɯrkuj tu-kɯ-ti tɕi tu, zgri tu-kɯ-ti tɕi tu ma, \\
some \textsc{erg} plant.name \textsc{ipfv}-say-\textsc{pl} some \textsc{erg} plant.name \textsc{ipfv}-say-\textsc{pl} plant.name  \textsc{ipfv}-\textsc{sbj}:\textsc{pcp}-say also exist:\textsc{fact} plant.name  \textsc{ipfv}-\textsc{sbj}:\textsc{pcp}-say also exist:\textsc{fact}  \textsc{lnk} \\
\glt  `Some call it \forme{zgri}, some call it \forme{mɯrkuj}; there are people who call it \forme{mɯrkuj}, and also people who call it \forme{zgri}.' (19-qachGa mWntoR, 168)
\end{exe}
 
The quantifier \japhug{tsuku}{some} used as a pronoun generally refers to humans in the corpus, but (\ref{ex:tsuku.YaR}) shows that it can also denote plants for instance. 

\begin{exe}
\ex \label{ex:tsuku.YaR}
\gll tsuku ɲaʁ, tsuku aqarŋɯrŋe, \\
some be.black:\textsc{fact} some be.light.yellow:\textsc{fact} \\
\glt `Some are black, some are light yellow.' (140505 stonka mWntoR, 5)
\end{exe}

Numerals (in particular \japhug{ci}{one}) and also counted nouns (§\ref{sec:CN.quantifier}) can be used without head noun with a partitive meaning `one of (a group)' as in (\ref{ex:ci.GW})  and (\ref{ex:ci.thWkWrgAz}).

\begin{exe}
\ex \label{ex:ci.GW}
\gll ci ɣɯ 	tɤ-tɕɯ,  ci ɣɯ tɕʰeme tɯ\redp{}tɤ-tu nɤ, ʁzɤmi ku-kɤ-sɯ-βzu \\
one \textsc{gen} \textsc{indef}.\textsc{poss}-son one \textsc{gen} \textsc{indef}.\textsc{poss}-son \textsc{cond}\redp{}\textsc{aor}-exist \textsc{lnk} husband.and.wife \textsc{ipfv}-\textsc{inf}-\textsc{caus}-make \\
\glt `If one of them has a boy, and the other one has a girl, let us make them husband and wife.' (zrAntCW 5)
\end{exe}

\begin{exe}
\ex \label{ex:ci.thWkWrgAz}
\gll ci tʰɯ-kɯ-rgɯ\redp{}rgɤz ɲɯ-ɕti tɕe, ci kɯ-xtɕɯ\redp{}xtɕi ɲɯ-ɕti tɕe, \\
one \textsc{aor}-\textsc{sbj}:\textsc{pcp}-\textsc{emph}\redp{}be.old C-be.\textsc{aff} \textsc{lnk} one \textsc{aor}-\textsc{sbj}:\textsc{pcp}-\textsc{emph}\redp{}be.old C-be.\textsc{aff} \textsc{lnk}  \\
\glt `One of them has grown very old, and one of them is very small.' (2011-05-nyima, 140)
\end{exe}

This partitive function is also found in combination with personal pronouns, as in (\ref{ex:nWZora.ci}).

\begin{exe}
\ex \label{ex:nWZora.ci}
\gll nɯʑora ci kɯ a-tɯci ci ju-tɯ-ɣɯt-nɯ ɯ́-jɤɣ \\
\textsc{2pl} one \textsc{erg} \textsc{1sg}.\textsc{poss}-\textsc{indef}.\textsc{poss}-water a.little \textsc{ipfv}-2-bring-\textsc{pl} \textsc{qu}-be.possible:\textsc{fact} \\
\glt `Could one of you bring me some water?' (150904 zhongli-zh, 51)
\end{exe}

\subsection{Distributive pronouns} \label{sec:distributive.pronouns}
The pronoun \japhug{ʑaka}{each his own} and its variant \japhug{ʑakastaka}{each his own} occur as pronouns, especially as possessors in an possessive existential construction. It can be correlated with a third singular \forme{ɯ-} (\ref{ex:Zaka.WmdoR}) or a third plural \forme{nɯ-} (\ref{ex:Zaka.nWrmi}) prefix on the possessum.


\begin{exe}
\ex \label{ex:Zaka.WmdoR}
\gll tɕe nɯnɯ li qaʑo nɯ kɯ-ɲaʁ tu, kɯ-wɣrum tu, kɯ-ɤɣɯnɯɕɯr kɯ-fse tu, kɯ-ɤrŋɯlɯz tu, tɕe nɯnɯ ʑaka ɯ-mdoʁ tu ma \\
\textsc{lnk} \textsc{dem} again sheep \textsc{dem} \textsc{sbj}:\textsc{pcp}-be.black exist:\textsc{fact} \textsc{sbj}:\textsc{pcp}-be.white exist:\textsc{fact} \textsc{sbj}:\textsc{pcp}-be.reddish \textsc{sbj}:\textsc{pcp}-be.like exist:\textsc{fact}  \textsc{sbj}:\textsc{pcp}-be.blueish exist:\textsc{fact} \textsc{lnk} \textsc{dem} each.his.own \textsc{3sg}.\textsc{poss}-colour  exist:\textsc{fact} \textsc{lnk} \\
\glt `There are black sheep, white ones, reddish ones, blueish ones, each has his own colour (they come in all types of colors).' (05-qaZo, 64-66)
\end{exe}

\begin{exe}
\ex \label{ex:Zaka.nWrmi}
\gll li ʑaka nɯ-rmi tu, \\
again each.his.own \textsc{3pl}.\textsc{poss}-name exist:\textsc{fact} \\
\glt `Each have their own names.' (150903 tWmNu, 11)
\end{exe}

The pronoun \forme{ʑaka} is built by combining the \textit{status constructus} of the pronominal root \forme{-ʑo} (§\ref{sec:pers.pronouns}) with the root \forme{-ka} found in the distributive modifier \japhug{tɯka}{each} (which follows possessums, see §\ref{sec:raNri}).


\section{Identity pronoun} \label{sec:other.pro}
The words \japhug{kɯmaʁ}{other} and \japhug{kɯɕte}{other} occur as prenominal determiners (see §\ref{sec:identity.modifier}, also for a discussion on the etymology of the former), but it can also be used as a pronoun and take determiners as in (\ref{ex:kWmaR.nWra}).

\begin{exe}
\ex \label{ex:kWmaR.nWra}
\gll ma kɯmaʁ nɯra aj mɯ́j-sɯχsal-a ri, tɤkʰepɣɤtɕɯ nɯ sɯχsal-a  \\
\textsc{lnk} other \textsc{dem}:\textsc{pl} \textsc{1sg} \textsc{neg}:\textsc{sens}-recognize but bird.sp \textsc{dem} recognize:\textsc{fact}-\textsc{1sg} \\
\glt `The other ones I don't recognize them, but the \forme{tɤkʰepɣɤtɕɯ} bird, I do recognize it.' (23-scuz, 46)
\end{exe}

The interpretation of both \japhug{kɯmaʁ}{other} and \japhug{kɯɕte}{other} can be locative  `somewhere else' as in (\ref{ex:kWmaR.nWtWtat}), when the main verb (\japhug{ta}{put} in this example) selects a goal or a locative adjunct (since locative noun phrases are often unmarked, §\ref{absolutive.goal}, §\ref{absolutive.locative}).

\begin{exe}
\ex \label{ex:kWmaR.nWtWtat}
\gll kɯmaʁ/kɯɕte   nɯ-tɯ-ta-t ŋu ɯ-maʁ? \\
other  \textsc{aor}-2-put-\textsc{tr}:\textsc{pst} be:\textsc{fact} \textsc{qu}-not.be:\textsc{fact} \\
\glt `Did you put it somewhere else?' (elicitation)
\end{exe}

Adding the indefinite determiner \japhug{ci}{one} is necessary in this context to convey the meaning `something else':

\begin{exe}
\ex \label{ex:kWmaR.ci.nWtWtat}
\gll kɯmaʁ/kɯɕte ci nɯ-tɯ-ta-t ŋu ɯ-maʁ? \\
other \textsc{indef} \textsc{aor}-2-put-\textsc{tr}:\textsc{pst} be:\textsc{fact} \textsc{qu}-not.be:\textsc{fact} \\
\glt `Did you put something else?' (elicitation)
\end{exe}

Example (\ref{ex:kWmaR.ci.juGWt}) illustrates that both \forme{kɯmaʁ} and \forme{kɯmaʁ ci} can occur in the meaning `another one' in some contexts (here with the verb \japhug{ɕar}{search}).

 \begin{exe}
\ex \label{ex:kWmaR.ci.juGWt}
\gll  χsɯ-sŋi mɤ-kɯ-tsu qʰe li kɯmaʁ ci ju-ɣɯt qʰe, li ɯ-zda ɲɯ-nɯ-ɕar ɲɯ-ɕti.
tɕe nɯnɯ maka kɯjka nɯ ŋɤn ma, ɯ-zda nɯ nɯ-me ɯ-qʰu mɤ-kɯ-nɤrʑaʁ tɕe kɯmaʁ ɲɯ-ɕar ɲɯ-ɕti tɕe mɯ́j-pe tu-ti-nɯ ŋgrɤl. \\
three-day \textsc{neg}-\textsc{inf}:\textsc{stat}-pass \textsc{lnk} again other \textsc{indef} \textsc{ipfv}-bring \textsc{lnk} again \textsc{3sg}.\textsc{poss}-companion \textsc{ipfv}-\textsc{auto}-search \textsc{sens}-be.\textsc{aff} \textsc{lnk} \textsc{dem} completely pyrrhocorax \textsc{dem} be.evil:\textsc{fact} \textsc{lnk} \textsc{3sg}.\textsc{poss}-companion \textsc{dem}  \textsc{aor}-not.exist \textsc{3sg}.\textsc{poss}-after \textsc{neg}-\textsc{inf}:\textsc{stat}-spend.time \textsc{lnk} other \textsc{ipfv}-search \textsc{sens}-be.\textsc{aff} \textsc{lnk} \textsc{neg}:\textsc{sens}-be.good \textsc{ipfv}-say-\textsc{pl} be.usually.the.case:\textsc{fact} \\
\glt `Not even three days (after hunters kill its mate, the \textit{Pyrrhocorax}) brings another one, it looks for another mate. People say that the \textit{Pyrrhocorax} is not nice, because not long after its mate has died, it looks for another one, it is not good.' (22-CAGpGa, 84)
\end{exe}

The indefinite \japhug{ci}{one} combined with the demonstrative determiner \forme{nɯ} (or \forme{nɯnɯ}) has the meaning `the other one' (the definite counterpart of \japhug{kɯmaʁ}{other}), as in (\ref{ex:ci.nW.kW}) and (\ref{ex:tWrdoR.ci.nW}).

\begin{exe}
\ex \label{ex:ci.nW.kW}
 \gll tɕe ɯ-jaʁ kɯ ki tu-ste lu-z-naʁje ɲɯ-ŋu ri, tɕe ci nɯ kɯ ɯ-jaʁ ku-mtsɯɣ ɲɯ-ɕti qʰe, \\
 \textsc{lnk} \textsc{3sg}.\textsc{poss}-hand \textsc{dem}:\textsc{prox} \textsc{ipfv}-do.like[III]  \textsc{ipfv}-reach.into[III] \textsc{sens}-be but \textsc{lnk} \textsc{indef} \textsc{dem} \textsc{erg} \textsc{3sg}.\textsc{poss}-hand  \textsc{ipfv}-bite \textsc{sens}-be.\textsc{aff}:\textsc{fact} \textsc{lnk} \\
\glt `(The cat) reaches with its paw (into the whole) like this, but the other one (the weasel) bites its paw.' (27-spjaNkW, 48)
\end{exe}

\begin{exe}
\ex \label{ex:tWrdoR.ci.nW}
 \gll 
ɯ-me ʁnɯz pjɤ-tu tɕe, tɯ-rdoʁ nɯ χsɤrlɤsmɤn pjɤ-rmi, ci nɯ rŋɯlɤsmɤn pjɤ-rmi tɕe, \\
\textsc{3sg}.\textsc{poss}-daughter two \textsc{ifr}.\textsc{ipfv}-exist \textsc{lnk} one-piece \textsc{dem} gser.la.sman \textsc{ifr}.\textsc{ipfv}-be.called \textsc{indef} \textsc{dem} dngul.la.sman \textsc{ifr}.\textsc{ipfv}-be.called \textsc{lnk} \\
\glt `He had two daughters, one of them was called Gser.la.sman, and the other Dngul.la.sman.' (2003-kWBRa, 1-2)
\end{exe}

Alternatively to the construction in (\ref{ex:tWrdoR.ci.nW}) with \japhug{tɯ-rdoʁ}{one piece} and \forme{ci nɯ} to express the meaning `one of them .... and the other ...', it is possible to use \forme{ci nɯ} two times in the same sentence to refer to more than one persons or animals, as in (\ref{ex:ci.nW.2}).

\begin{exe}
\ex \label{ex:ci.nW.2}
 \gll tɕe ci nɯnɯ ju-ɕe ɯ-kʰɯkʰa ci nɯ kɯ ɯ-pu tu-ndze, cʰɯ-rɤɕi. \\
\textsc{lnk} \textsc{indef} \textsc{dem} \textsc{ipfv}-go \textsc{3sg}-while \textsc{indef} \textsc{dem} \textsc{erg} \textsc{3sg}.\textsc{poss}-intestine \textsc{ipfv}-eat[III] \textsc{ipfv}-pull \\
\glt `While one of the two (the prey) is (still) going, the other one (the predator) eats and pulls its intestine.'  (20-RmbroN, 76)
\end{exe}

The dual \forme{ci nɯni} `the other two' and plural \forme{ci nɯra} `the other ones' are also attested, as in (\ref{ex:ci.nWni}), showing that \forme{ci} is here completely bleached of numeral meaning.

 \begin{exe}
\ex \label{ex:ci.nWni}
 \gll  ci nɯni ɣɯ nɯ, ndʑi-ta-mar rɟɤɣi pjɤ-ŋu tɕe tɕe nɯnɯ ɯʑo kɯ to-ndza \\
\textsc{indef} \textsc{dem}:\textsc{du} \textsc{gen} \textsc{dem} \textsc{3du}.\textsc{poss}-\textsc{indef}.\textsc{poss}-butter tsampa \textsc{ipfv}.\textsc{ifr}-be:\textsc{fact} \textsc{lnk} \textsc{lnk} \textsc{dem} \textsc{3sg} \textsc{erg} \textsc{ifr}-eat \\
\glt `The tsampa of the other two (sisters) was butter tsampa, and she ate it.' (2003-kWBRa, 20)
\end{exe}

It is also possible in this function to use other modifiers such as numerals, as in (\ref{ex:ci.XsWm}) with \japhug{χsɯm}{three}. 

 \begin{exe}
\ex \label{ex:ci.XsWm}
 \gll iɕqʰa ci χsɯm nɯ mɯ-jo-ɣi-nɯ kɯ  \\
 the.aforementioned \textsc{indef} three \textsc{dem} \textsc{neg}-\textsc{ifr}-come-\textsc{pl} \textsc{erg} \\
\glt `The three other ones, without coming, (said...)' (140515 congming de wusui xiaohai-zh, 45)
 \end{exe}

As a prenominal determiner, \forme{ci} also has the meaning `the other $X$' (see §\ref{sec:identity.modifier}).


Finally, the noun \japhug{tɯrme}{person} (which also occurs to express generic person, §\ref{sec:tWrme.generic}) can be used in the meaning `someone else' or `other people',  in particular in genitival constructions as in (\ref{ex:tWrme.WkhApa.zW}).

\begin{exe}
\ex \label{ex:tWrme.WkhApa.zW}
\gll tɯrme ɯ-kʰɤpa zɯ, ki kɯ-fse tɯ-rʑaʁ lu-znɯfsoʁspat-a ku-omdzɯ-a. \\
people \textsc{3sg}.\textsc{poss}-yard \textsc{loc} \textsc{dem}.\textsc{prox} \textsc{sbj}:\textsc{pcp}-be.like one-night \textsc{ipfv}-do.the.whole.night-\textsc{1sg} \textsc{ipfv}-sit-\textsc{1sg} \\
\glt `I would spend an entire night from dusk till dawn sitting in someone else's animal yard.' (2010-histoire09, 34)
\end{exe} 

\section{Demonstrative pronouns} \label{sec:demonstrative.pronouns}
There are two basic demonstratives in Japhug, the proximal \japhug{ki}{this} and the distal one \japhug{nɯ}{that}, which also occur as demonstrative determiners (see §\ref{sec:demonstrative.determiners}). \tabref{tab:dem.pronoun} illustrates the various demonstrative pronouns that are derived from these basic forms, with reduplicated and emphatic forms. There is in addition a cataphoric pronoun \forme{nɤki}, discussed in §\ref{sec:cataph.pron}.

Plural and dual forms, as in the case of determiners, are formed by adding \forme{-ra} and \forme{-ni} suffixes (§\ref{sec:number.determiners}) to the demonstrative root, which undergoes \textit{status constructus} change \ipa{i} \fl{} \ipa{ɯ} in the case of proximal demonstratives. Plural forms are given in the table; dual forms are attested but rare and can be predicted (\japhug{kɯni}{these two} etc).

\begin{table}
\caption{Demonstrative pronouns}\label{tab:dem.pronoun}
\begin{tabular}{lllll} 
\lsptoprule
&Base form & Reduplicated & Emphatic \\
\midrule
Proximal, singular & \forme{ki} & \forme{kɯki} &  \forme{ɯkɯki}  \\
Distal, singular & \forme{nɯ} &  \forme{nɯnɯ} & \forme{ɯnɯnɯ} \\
\midrule
Proximal, plural & \forme{kɯra} & \forme{kɯkɯra} &  \forme{ɯkɯkɯra}  \\
Distal, plural & \forme{nɯra} &  \forme{nɯnɯra} & \forme{ɯnɯnɯra} \\
\lspbottomrule
\end{tabular}
\end{table}

The distal demonstratives, being the default forms, are simply glossed as \textsc{dem} in the examples: only proximal demonstratives are explicitely marked as such in the glosses.



\subsection{Anaphoric demonstrative pronouns} \label{sec:anaphoric.demonstrative.pro}

The basic demonstratives \forme{ki} and \forme{nɯ} are less often used as pronouns that the other ones (they mainly occur as determiners). They nevertheless do occur in all syntactic functions, including object (in particular with the verb \japhug{ti}{say}, as in \ref{ex:nW.toti}, where it refers to words that have been previously told to another animal),  and semi-object (in particular with the verb \japhug{stu}{do like} as in \ref{ex:ki.tuste}).

\begin{exe}
\ex \label{ex:nW.toti}
 \gll  li nɯ to-ti ri, \\
 again \textsc{dem} \textsc{ifr}-say \textsc{lnk} \\
\glt `(Gesar) said the same thing to the (snow leopard).' (gesar, 286)
\end{exe}

\begin{exe}
\ex \label{ex:ki.tuste}
 \gll ɯ-mu nɯ ku-rqoʁ tɕe ki tu-ste tɕe \\
\textsc{3sg}.\textsc{poss}-mother  \textsc{dem} \textsc{ipfv}-hug \textsc{lnk} \textsc{dem}:\textsc{prox} \textsc{ipfv}-do.like[III] \textsc{lnk} \\
\glt `It hugs its mother like that.' (19-GzW, 30)
\end{exe}

The distal demonstratives \forme{nɯ} and \forme{nɯnɯ} serve as anaphoric pronouns with any type of referent, including humans, but also abstract concepts, inanimate objects or plants as in (\ref{ex:nWnW.kW.smi}), though as mentioned in §\ref{sec:pers.pronouns},  third person pronouns such as \japhug{ɯʑo}{he} can also have inanimate antecedents.

\begin{exe}
\ex \label{ex:nWnW.kW.smi}
 \gll tʂʰa kɤ-nɯ-ta tɤ-ra, smi kɤ-βlɯ tɤ-ra pɯ-nɯ-ŋu, tʰamaka sko-nɯ pɯ-nɯ-ŋu, tɕe \textbf{nɯnɯ} kɯ smi tu-sɯ-tɕɤt-nɯ. \\
 tea \textsc{inf}-\textsc{auto}-put \textsc{aor}-be.needed fire  \textsc{aor}-burn \textsc{aor}-be.needed \textsc{pst}.\textsc{ipfv}-\textsc{auto}-be tobacco smoke:\textsc{fact}-\textsc{pl} \textsc{pst}.\textsc{ipfv}-\textsc{auto}-be \textsc{lnk} \textsc{dem} \textsc{erg} fire \textsc{ipfv}-\textsc{caus}-take.out-\textsc{pl} \\
 \glt `When they need to boil tea, to make a fire or smoke tobacco, people light up the fire with it.' (15-babW, 226-229)
\end{exe}

When a third person mentioned in a discussion is present, the pronoun \japhug{ɯʑo}{he} is not the optimal way of referring to him/her, and a proximal demonstrative, in particular the reduplicated \japhug{kɯki}{this one}, is used instead. It can occur to present someone to someone else (\ref{ex:kWki.aslama}) (note that a similar usage exists in Western languages such as English in the same context) and even to talk about the actions of this person, as in  (\ref{ex:kWki.kW.taBzu}) and (\ref{ex:kWki.nW.ftsWntCi}).

\begin{exe}
\ex \label{ex:kWki.aslama}
 \gll kɯki a-slama ŋu \\
\textsc{dem}.\textsc{prox} \textsc{1sg}.\textsc{poss}-student be:\textsc{fact} \\
\glt `This a (former) student of mine.' (conversation 140510, 17)
\end{exe}

\begin{exe}
\ex \label{ex:kWki.kW.taBzu}
 \gll  kɯki kɯ ta-βzu? \\
 \textsc{dem}.\textsc{prox} \textsc{erg} \textsc{aor}:3\flobv{}-make \\
 \glt `Did she make it?' (conversation 140510, 152)
\end{exe}

As other pronouns (see §\ref{sec:pers.pronouns}), demonstrative pronouns can take the demonstrative determiner \forme{nɯ}, as in (\ref{ex:kWki.nW.ftsWntCi}).

\begin{exe}
\ex \label{ex:kWki.nW.ftsWntCi}
 \gll mɯ\redp{}mɤ-pɯ-jɤɣ tɕe mɤ-ɣi-tɕi ma \textbf{kɯki} \textbf{nɯ} fstɯn-tɕi ra ma tɕi-βɣe ɯ-ku tʰɯ-kɯ-ɣɤrndi  \\
\textsc{cond}\redp{}\textsc{neg}-\textsc{pst}.\textsc{ipfv}-be.acceptable \textsc{lnk} \textsc{neg}-come:\textsc{fact}-\textsc{1du} \textsc{lnk} \textsc{dem}:\textsc{prox} \textsc{dem} serve:\textsc{fact}-\textsc{1du} be.needed:\textsc{fact} \textsc{lnk} \textsc{1du}.\textsc{poss}-orphan \textsc{3sg}.\textsc{poss}-head \textsc{aor}-\textsc{sbj}:\textsc{pcp}-support \\
\glt `If it is not possible (to take the old man with us) we will not come, as we have to serve him, he is the one who adopted us orphans when we were in dire straits.' (The old man is presumably present when this sentence is uttered; 2003nyima2, 122)
\end{exe}

The emphatic demonstrative pronouns (which are also used as determiners, §\ref{sec:demonstrative.determiners}) are built by combining the reduplicated forms of demonstratives with the third person possessive prefix \forme{ɯ-}. They are about fifty times less common than corresponding reduplicated forms, but their function is essentially the same. In (\ref{ex:WnWnW.kW}), \forme{ɯnɯnɯ} is an anaphoric pronoun whose antecedent is present in the immediately preceding clause.

\begin{exe}
\ex \label{ex:WnWnW.kW}
 \gll
tɕe ɯ-rqʰu kɯ-fse ci ɣɤʑu tɕe, \textbf{ɯnɯnɯ} kɯ ɯ-rdu nɯ tu-ɕɯ-fkaβ kɯ-fse ɲɯ-ŋu. \\
\textsc{lnk} \textsc{3sg}.\textsc{poss}-hull \textsc{sbj}:\textsc{pcp}-be.like \textsc{indef}  exist:\textsc{sens} \textsc{lnk} \textsc{dem}:\textsc{emph} \textsc{erg} \textsc{3sg}.\textsc{poss}-eyeball \textsc{dem} \textsc{ipfv}-\textsc{caus}-cover  \textsc{sbj}:\textsc{pcp}-be.like \textsc{sens}-be \\
\glt `It has something like a membrane, and it covers its eyeball with it.' (description of the nictitating membrane of birds, 140513 sWNgWrmABja, 9)
\end{exe}

The demonstrative \forme{nɯ} may not refer anaphorically to a particular entity , but also to an entire situation, as in (\ref{ex:nW.pjArAZindZi}) where it occurs as an adjunct in absolutive form, meaning `this way, like that' (in another version of the same story, we find \forme{nɯ kɯ-fse} `like that' instead of \forme{nɯ} in the same context).

 \begin{exe}
\ex \label{ex:nW.pjArAZindZi}
\gll   nɯnɯ ɯ-mŋu nɯtɕu zɯ li, qapri, nɤki, kɯ-ɲaʁ nɯ kɯ kɯ-wɣrum nɯ ɯ-qiɯ ʑo cʰɤ-mqlaʁ tɕe nɯ pjɤ-rɤʑi-ndʑi. \\
\textsc{dem} \textsc{3sg}.\textsc{poss}-bank \textsc{dem}:\textsc{loc} \textsc{loc} again snake \textsc{filler} \textsc{sbj}:\textsc{pcp}-be.black \textsc{dem} \textsc{erg}  \textsc{sbj}:\textsc{pcp}-be.white \textsc{dem} \textsc{3sg}.\textsc{poss}-half \textsc{emph} \textsc{ifr}-swallow \textsc{lnk} \textsc{dem} \textsc{ifr}.\textsc{ipfv}-stay-\textsc{du} \\
\glt `On the bank (of the lake), there was again a black snake that had swallowed half of a white snake, and they were staying (stuck) like that.' (28-smAnmi, 104)
\end{exe}


\subsection{Medial and cataphoric pronoun} 

\subsubsection{Medial demonstrative} \label{sec:medial.dem.pro}
In addition to the proximal and distal demonstratives,  there is a considerably rarer medial demonstrative \forme{nɤki}, in examples such as (\ref{ex:nAki.nWtɕu}) and (\ref{ex:nAki.nW.aZWG}), which means `your place, near you', as opposed to `here'.

\begin{exe}
	\ex \label{ex:nAki.nWtɕu}
	\gll kutɕu ko-qanɯ pʰoʁpʰoʁ ʑo, nɤki nɯtɕu ɯ-kó-qanɯ? \\
	\textsc{dem}.\textsc{prox}:\textsc{loc} \textsc{ifr}-be.dark \textsc{idph}(II):completely \textsc{emph} \textsc{dem}:\textsc{medial} \textsc{dem}:\textsc{loc} \textsc{qu}-\textsc{ifr}-be.dark \\
	\glt `Here it is already dark, is it (also) dark in your place?' (conversation, 14.12.24, referring to the time lag between Paris and Mbarkham)
\end{exe}

\begin{exe}
	\ex \label{ex:nAki.nW.aZWG}
	\gll  ki kɯra ɲɯ-kʰam-a tɕe nɤki nɯ aʑɯɣ nɯ-kʰɤm je \\
	\textsc{dem}:\textsc{prox} \textsc{dem}:\textsc{prox}:\textsc{pl}  \textsc{ipfv}-give-\textsc{1sg} \textsc{lnk} \textsc{dem}:\textsc{medial} \textsc{dem} \textsc{1sg}:\textsc{gen} \textsc{imp}-give \textsc{sfp} \\
	\glt `I give (you) these (toys), give me that one.'  (2012 Norbzang, 135)
\end{exe}

The medial demonstrative \forme{nɤki} can be historically analyzed as a combination of the proximal demonstrative \forme{ki} with the second person possessive \forme{nɤ-}. However, equivalent dual or plural forms such as  $\dagger$\forme{ndʑiki} or $\dagger$\forme{nɯki} are impossible (the equivalent meaning can only be expressed with the dative, using a form such as \forme{ndʑiʑo ndʑi-pʰe} `at your$_{du}$ place', §\ref{sec:dative} ).


\subsubsection{Cataphoric pronoun}
\label{sec:cataph.pron}
In addition to its function as a medial demonstrative (§\ref{sec:medial.dem.pro}), the demonstrative \forme{nɤki} also occurs to express cataphoric reference. It occurs especially when the speaker hesitates and uses it as a filler, followed by a clause with the same verb (examples \ref{ex:nAki.YWNu} and \ref{ex:nAki.YAXtAr}) or just with the same auxiliary (\ref{ex:nAki.Nu.Ci}).

\begin{exe}
\ex \label{ex:nAki.YWNu}
 \gll qra nɯ kɯ, mbala na-lɤt nɤ tɕe \textbf{nɤki} ɲɯ-ŋu, jla ɲɯ-ŋu, \\
female.yak \textsc{dem} \textsc{erg} male.young.bovid \textsc{aor}:3\flobv{} \textsc{lnk} \textsc{lnk} \textsc{dem}:\textsc{cataph} \textsc{sens}-be male.hybrid.yak \textsc{sens}-be \\
\glt `When a female yak has a young (with a bull), it is..., it is a hybrid yak.' (05-qambrW, 64)
\end{exe}

\begin{exe}
\ex \label{ex:nAki.YAXtAr}
 \gll tɕe nɯ tɯ-ci ɣɯ ɯ-taʁ nɯnɯtɕu, \textbf{nɤki} ɲɤ-χtɤr, iɕqʰa <yujinxiang> kɤ-ti mɯntoʁ nɯ ɣɯ  ɯ-jwaʁ nɯ ɲɤ-χtɤr.  \\
\textsc{lnk} \textsc{dem} \textsc{indef}.\textsc{poss}-water \textsc{gen} \textsc{3sg}-on \textsc{dem}:\textsc{loc} \textsc{dem}:\textsc{cataph} \textsc{ifr}-spread the.aforementionned tulip \textsc{obj}:\textsc{pcp}-say flower \textsc{dem} \textsc{gen} \textsc{3sg}.\textsc{poss}-leaf \textsc{dem} \textsc{ifr}-spread \\
\glt `She spilled on the water...  she spilled the petals of the flower called ``tulip''.' (150818 muzhi guniang-zh, 69)
\end{exe}

\begin{exe}
\ex \label{ex:nAki.Nu.Ci}
 \gll tɕe nɯ-nɯŋa ra nɤki ŋu ɕi, tɕe ɲɯ-tɯ-nɤm qʰe, tɕe ʑara ku-n-nɯ-ɣi-nɯ ŋu ɕi? \\
\textsc{lnk} \textsc{2sg}.\textsc{poss}-cow \textsc{pl} \textsc{dem}:\textsc{cataph} be:\textsc{fact} \textsc{qu} \textsc{lnk} \textsc{ipfv}:\textsc{west}-2-chase[III] \textsc{lnk} \textsc{lnk} \textsc{3pl} \textsc{ipfv}:\textsc{east}-\textsc{auto}-\textsc{vert}-come-\textsc{pl} be:\textsc{fact} \textsc{qu} \\
\glt `And your cows, are they (still) like that, you let them out of the pen (in the morning), and  they come back home on their own (in the evening)?' (taRrdo conversation, 28-29)
\end{exe}

It is also used  when the speaker alerts the addressee that a long description follows as in (\ref{ex:nAki.tustunW}), as in English `(he said) the following'. Given the fact the Japhug is strictly verb-final and has pre-verbal complements (§\ref{sec:basic.word.order}), this is a strategy employed to avoid relegating the main verb to the end of the description.

 \begin{exe}
\ex \label{ex:nAki.tustunW}
 \gll kʰopi kɯ ɴqiazwɤr ci ɲɯ-mɯm rca ɲɯ-saχaʁ ʑo tɕe \textbf{nɤki} tu-stu-nɯ ɲɯ-ŋu ɲɯ-ti, ɲɯ-pʰɯt-nɯ qʰe kɯ-zri... ki jamar ʑo kɯ-zri ɲɯ-pʰɯt-nɯ qʰe nɤki, ɯ-ku ɯ-mtɯ kɯ-fse nɯtɕu kú-wɣ-ndo qʰe tɕe ɯ-pa nɯ, ɯ-jwaʁ nɯ cʰɯ-χɕoʁ-nɯ ɲɯ-ŋu...  \\
\textsc{anthr} \textsc{erg} bitter.wormwood \textsc{indef} \textsc{sens}-be.tasty \textsc{unexpect} \textsc{sens}-be.extremely \textsc{emph} \textsc{lnk} \textsc{dem}:\textsc{cataph} \textsc{ipfv}-do.like-\textsc{pl} \textsc{sens}-be \textsc{sens}-say \textsc{ipfv}-take.out-\textsc{pl} \textsc{lnk} \textsc{sbj}:\textsc{pcp}-be.long \textsc{dem}:\textsc{prox} about \textsc{emph} \textsc{sbj}:\textsc{pcp}-be.long \textsc{ipfv}-take.out-\textsc{pl} \textsc{lnk} \textsc{dem}:\textsc{cataph} \textsc{3sg}.\textsc{poss}-head  \textsc{3sg}.\textsc{poss}-crest \textsc{sbj}:\textsc{pcp}-be.like  \textsc{dem}:\textsc{loc} \textsc{ipfv}-\textsc{inv}-take \textsc{lnk} \textsc{lnk} \textsc{3sg}.\textsc{poss}-under \textsc{dem}  \textsc{3sg}.\textsc{poss}-leaf \textsc{dem} \textsc{ipfv}:\textsc{downstream}-take.out-\textsc{pl}  \textsc{sens}-be \\
\glt `Kebei says that bitter wormwood is very tasty, and that they prepare it in the following way: they pluck (wormwoods) that are this big, take it by something that looks like a crest on the top, and prune away the leaves under it... (continued by several paragraphs)' (conversation 140510)
\end{exe}

The pronoun \forme{nɤki} is also used as a determiner (§\ref{sec:demonstrative.determiners}) and the speech filler \forme{nɤkinɯ} (§\ref{sec:fillers}) derives from the combination of  \forme{nɤki}  with the determiner \forme{nɯ}. There are no plural or dual forms of \forme{nɤki}, but it can be combined with dual or plural determiners as in (\ref{ex:nAki.nWra}).

\begin{exe}
\ex \label{ex:nAki.nWra}
\gll  tɕe nɤki nɯra, mkʰɤrmaŋ ra pjɤ-rɯsɯso-nɯ tɕe, \\
\textsc{lnk} \textsc{dem}:\textsc{cataph} \textsc{dem}:\textsc{pl} people \textsc{pl} \textsc{ifr}-think-\textsc{pl} \textsc{lnk} \\
\glt `And these, the people thought about it.' (150829 jidian-zh, 138)
\end{exe}

It is likely that the cataphoric demonstrative use of \forme{nɤki} derives  from its function as a medial demonstrative, suggesting the historical pathway in (\ref{ex:nAki.hist}).

\begin{exe}
\ex \label{ex:nAki.hist}
\glt \textsc{2sg}+\textsc{dem}:\textsc{prox} $\Rightarrow$ \textsc{dem}:\textsc{medial} $\Rightarrow$ \textsc{dem}:\textsc{cataphoric} $\Rightarrow$ \textsc{speech filler}
\end{exe}

\subsection{Locative forms of the demonstrative pronouns} \label{sec:locative.pronoun}
The locative postposition \forme{tɕu} (§\ref{sec:core.locative}) can be combined with the demonstrative pronouns \forme{nɯ} and \forme{ki} and their reduplicated and emphatic forms, as shown in \tabref{tab:loc.dem.pronoun}.  

The locative pronouns in \forme{-tɕu} can be followed by the postposition \forme{zɯ} as in (\ref{ex:nWtCu.zW.mbro}), but not by the locative \forme{ri}.
 
\begin{exe}
\ex \label{ex:nWtCu.zW.mbro}
 \gll mbrosta ci tu tɕe, nɯtɕu zɯ mbro nɯ a-ja. \\
 stable \textsc{indef} exist:\textsc{fact} \textsc{lnk} \textsc{dem}:\textsc{loc} \textsc{loc} horse \textsc{dem} \textsc{pass}-keep.attached:\textsc{fact} \\
 \glt `There are stables (in this palace), and the horse is kept there.' (140507 jinniao, 174)
\end{exe} 

The proximal demonstrative \forme{ki} undergoes \textit{status constructus} (§\ref{sec:status.constructus}) alternation and changes to \forme{kɯ-} when combined with the locative postposition \forme{tɕu} (§\ref{sec:core.locative}), with further assimilation to [\forme{u}] due to the regressive vowel assimilation (§\ref{sec:vowel.harmony}) when followed by  \forme{-tɕu}.

 In addition to the locative pronouns in  \forme{-tɕu}, there is an entirely parallel series of pronouns in \forme{-re}; this suffix is probably unrelated to the locative postposition \forme{ri}, and may rather reflect the plural marker \forme{ra} (which can be used to mark vague location, see §\ref{sec:plural.determiners}) with the proto-Gyalrong locative suffix \forme{*-j} and regular vowel fusion (§\ref{sec:locative.j}). These locative pronouns are much less commonly used in the corpus than those of the  \forme{-tɕu} series.
 
\begin{table}
\caption{Locative demonstrative pronouns}\label{tab:loc.dem.pronoun}
\begin{tabular}{lllll} 
\lsptoprule
&Base form & Reduplicated & Emphatic \\
\midrule
\textsc{prox}.\textsc{sg} & \forme{kutɕu} & \forme{kukutɕu} &  --  \\
\textsc{dist}.\textsc{sg} & \forme{nɯtɕu} &  \forme{nɯnɯtɕu} & \forme{ɯnɯnɯtɕu} \\
\midrule
\textsc{prox}.\textsc{sg} & \forme{kɯre} & \forme{kɯkɯre} &  --  \\
\textsc{dist}.\textsc{sg} & \forme{nɯre} &  \forme{nɯnɯre} & \forme{ɯnɯnɯre} \\
\lspbottomrule
\end{tabular}
\end{table}

Locative pronouns in \forme{-re} can appear on their own as in (\ref{ex:kWre}), or with the locative postposition \forme{ri} as in (\ref{ex:ndWchu.kWre.ri}), but never with the other postpositions \forme{zɯ} and \forme{tɕu}. 

\begin{exe}
\ex \label{ex:kWre}
 \gll aʑo mɯ-pɯ-rɤʑi-a, kɯre pɯ-ɕti-a. \\
 \textsc{1sg} \textsc{neg}-\textsc{pst}.\textsc{ipfv}-stay \textsc{dem}.\textsc{prox}:\textsc{loc} \textsc{pst}.\textsc{ipfv}-be.\textsc{aff}-\textsc{1sg} \\
\glt `I was not present (there), I was here.' (conversation140510 , 84)
\end{exe}

\begin{exe}
\ex \label{ex:ndWchu.kWre.ri}
 \gll tɕe kukutɕu, <zhuanmen>, nɤkinɯ, tɯ-ɕɣa ɯ-kɯ-nɯsmɤn, tɕetʰi, ndɯcʰu kɯre ri rɤʑi ma,
nɯnɯ wuma ʑo mkʰɤz tɕe, \\
\textsc{lnk} \textsc{dem}.\textsc{prox}:\textsc{loc} specially \textsc{filler} \textsc{indef}.\textsc{poss}-tooth \textsc{3sg}.\textsc{poss}-\textsc{sbj}:\textsc{pcp}-treat downstream west:\textsc{approx}.\textsc{loc} \textsc{dem}.\textsc{prox}:\textsc{loc} \textsc{loc} stay:\textsc{fact} \textsc{lnk} \textsc{dem} really \textsc{emph} be.expert:\textsc{fact} \textsc{lnk} \\
\glt `Here (in Mbarkham), there is someone who specially treats teeth in the west (of Mbarkham).' (27-tApGi, 139-140)
\end{exe}

 Locative adverbs, such as those based on the approximate locative \forme{-cʰu} (§\ref{sec:approximate.locative}) can be combined with the locative pronouns in \forme{-re} as shown by the phrase \forme{ndɯcʰu kɯre ri} `in the west side' in (\ref{ex:ndWchu.kWre.ri}).

Both series of locative pronouns can express static location as in examples (\ref{ex:kWre}) to  (\ref{ex:nWtCu.kurAzinW}), or motion towards a place as in (\ref{ex:kWre.nWnWGea}) and (\ref{ex:nWnWtCu.pjWwGlAt}).

\begin{exe}
\ex \label{ex:nWtCu.kurAzinW}
 \gll nɯtɕu ku-rɤʑi-nɯ ɲɯ-ŋu. \\
 \textsc{dem}:\textsc{loc} \textsc{ipfv}-stay-\textsc{pl} \textsc{sens}-be \\
 \glt `They live there.' (20-RmbroN, 4)
\end{exe}

\begin{exe}
\ex \label{ex:kWre.nWnWGea}
 \gll aʑo akɯ kɤntɕʰaʁ ri kɤ-ari-a tɕe, nɯ kóʁmɯz kɯre nɯ-nɯ-ɣe-a \\
 \textsc{1sg} east street \textsc{loc} \textsc{aor}:\textsc{east}-go[II]-\textsc{1sg} \textsc{lnk} \textsc{dem} only.after \textsc{dem}.\textsc{prox}:\textsc{loc} \textsc{aor}:\textsc{west}-\textsc{vert}-come[II]-\textsc{1sg} \\
\glt `I went there on the street, I just came back here.' (conversation, 2013-12-02)
\end{exe}

\begin{exe}
\ex \label{ex:nWnWtCu.pjWwGlAt}
 \gll βɣɤtu nɯ ɣɯ ɯ-χcɤl ri spoʁ. [...] nɯnɯtɕu tɕe kɤ-ɣndʑɯr ɯ-spa nɯra pjɯ́-wɣ-lɤt. \\
upper.grindstone \textsc{dem} \textsc{gen} \textsc{3sg}.\textsc{poss}-middle \textsc{loc} have.a.hole:\textsc{fact}  [...] \textsc{dem}:\textsc{loc} \textsc{lnk} \textsc{obj}:\textsc{pcp}-grind \textsc{3sg}.\textsc{poss}-material \textsc{dem}:\textsc{pl} \textsc{ipfv}-\textsc{inv}-throw \\
\glt `There is a hole in the middle of the upper grindstone, into which one pours (the grains) that are to be ground.' (160705 khABGa, 14)
\end{exe}

Apart from its locative uses, \forme{nɯtɕu} can express a temporal meaning `at that time' as in (\ref{ex:nWtCu.temporal}) and (\ref{ex:nWtCu.temporal2}).

\begin{exe}
\ex \label{ex:nWtCu.temporal}
\gll <qidian> tɕe tɤ-mŋɤm ta-ʑa a-pɯ-ŋu tɕe, tɕe nɯnɯ tɯ-sŋi nɯ tu-mŋɤm, tɯ-rʑaʁ nɯ tu-mŋɤm tɕe, ɯ-fso <qidian> mɤɕtʂa nɯ mɯ́j-ʑi tɕe \textbf{nɯtɕu} tɕe kɯ-xtɕɯ\redp{}xtɕi tɯ-ʑi ɲɯ-ʑe ɲɯ-ŋu tɕe \\
seven.o'clock \textsc{lnk} \textsc{aor}-hurt \textsc{aor}:3\flobv{}-start \textsc{irr}-\textsc{ipfv}-be \textsc{lnk} \textsc{lnk} \textsc{dem} one-day \textsc{dem} \textsc{ipfv}-hurt  one-night \textsc{dem} \textsc{ipfv}-hurt  \textsc{lnk}  \textsc{3sg}.\textsc{poss}-tomorrow seven.o'clock until \textsc{dem} \textsc{neg}:\textsc{sens}-subside \textsc{lnk} \textsc{dem}:\textsc{loc} \textsc{lnk} \textsc{sbj}:\textsc{pcp}-\textsc{emph}\redp{}be.small \textsc{inf}-subside \textsc{ipfv}-start[III] \textsc{sens}-be \textsc{lnk} \\
\glt `(For instance), if (the headache) starts at seven o'clock, it hurts for one day and one night, and subsides only in the next day at seven, \textbf{at that time} it starts to subside a little.' (24-pGArtsAG, 93-96)
\end{exe}

\begin{exe}
\ex \label{ex:nWtCu.temporal2}
\gll  nɯtɕu tɯrme nɯra pɤjkʰu pjɤ-me ɲɯ-ŋu tɕe. \\
\textsc{dem}:\textsc{loc} people \textsc{dem}:\textsc{pl} yet \textsc{ifr}.\textsc{ipfv}-not.exist \textsc{sens}-be \textsc{lnk}  \\
\glt `\textbf{At that time} (the time of the dinosaurs), humans did not exist yet.' (180421 bawanglong, 33)
\end{exe}

In addition, it can convey in some contexts a more abstract meaning like `in those circumstances', as in (\ref{ex:nWtCu.kWnA}).

\begin{exe}
\ex \label{ex:nWtCu.kWnA}
\gll tɕeri nɯtɕu kɯnɤ tɕiʑo kɤndʑiβzaŋsa nɯ mɯ-pɯ-nɯ-qia-tɕi \\
but \textsc{dem}:\textsc{loc} also \textsc{1du} \textsc{coll}:friend \textsc{dem} \textsc{neg}-\textsc{aor}-\textsc{auto}-tear.down-\textsc{1du} \\
\glt `But even in those circumstances (working in different places, and meeting only once a year), we did not lose our friendship.' (12-BzaNsa, 42)
\end{exe}

As for \forme{kutɕu}, it is almost exclusively used for spatial location; a metaphorical usage is attested in (\ref{ex:kutCu.tCe.nW.tutia}), where it means `in this story'.

\begin{exe}
\ex \label{ex:kutCu.tCe.nW.tutia}
\gll iɕqʰa nɤkinɯ <piqiu> nɯ ɯ-rmi kɤ-spa-t-a. nɤki, tsʰuβdɯn ra kɯ rgoŋlu tu-ti-nɯ ɲɯ-ŋu. iʑora, tɕe kutɕu tɕe nɯ tu-ti-a ŋu. \\
\textsc{filler} \textsc{filler} ball \textsc{dem} \textsc{3sg}.\textsc{poss}-name \textsc{aor}-be.able-\textsc{pst}:\textsc{tr}-\textsc{1sg} \textsc{filler}  \textsc{topo} \textsc{pl} \textsc{erg} ball \textsc{ipfv}-say-\textsc{pl} \textsc{sens}-be \textsc{1pl} \textsc{lnk} \textsc{dem}.\textsc{prox}:\textsc{loc} \textsc{lnk} \textsc{dem} \textsc{ipfv}-say-\textsc{1pl} be:\textsc{fact} \\
\glt `I have learned how to say ``ball'', people from Tshobdun call it \forme{rgoŋlu}, we (do not have this word but) this is how I am going to say it here (in this story). (140514 huishuohua de niao, 4)
\end{exe}

The forms \forme{nɯtɕu} and \forme{nɯnɯtɕu} following a noun phrase result from the fusion of the postnominal demonstrative determiners \forme{nɯ} and \forme{nɯnɯ} with the locative postposition \forme{tɕu} (§\ref{sec:core.locative}), and are not be to analyzed as locative pronouns.

