\chapter{A grammatical sketch} \label{chap:sketch}

This chapter offers a short introduction to the phonology and morphosyntax of the Japhug language. It is written for typologists and comparative linguists wanting to get a quick general picture of this language, before delving into the core of the grammar.

Morphosyntactic phenomena are illustrated mainly using simplified elicited examples rather than examples gleaned from texts and conversations as in the rest of the grammar.

\section{Phonology and word structure} \label{sec:phono.introduction}
\is{vowel} \is{consonant}
Japhug has 8 vowels (§\ref{sec:vowels}) and 50 consonant phonemes (§\ref{sec:consonant.phonemes}), which can be combined into more than 400 biconsonantal or triconsonantal clusters in the onset (§\ref{sec:inventory.clusters}). Additional clusters are attested across syllable boundaries (§\ref{sec:heterosyllabic.clusters}). In coda position, only 12 consonants are found, and no clusters are possible: several phonological contrasts are neutralized (§\ref{sec:codas.inventory}).

The IPA-based transcription in this grammar is a spelling system that is not strictly phonological: it uses different symbols to represent the allophones of some phonemes (in particular \ipa{w}, §\ref{sec:consonant.phonemes}, §\ref{sec:wC.clusters}). An alternative Tibetan-based orthography for use by native speakers (§\ref{sec:tibetan.script}) is also provided.

Japhug is very far removed from the isolating, tonal and ``monosyllabic'' type once considered to be typical of Sino-Tibetan languages. Japhug lacks tonal contrasts (§\ref{sec:stress}), and monosyllabic and monomorphemic words are a minority (§\ref{sec:wordhood.verb}). Words of six syllables or more are not rare in the corpus, as shown by the verb form (\ref{ex:akAtWnWrAZinW}) below.

\begin{exe}
\ex \label{ex:akAtWnWrAZinW}
\gll a-kɤ-tɯ-nɯ-rɤʑí-nɯ \\
\textsc{irr}-\textsc{pfv}-2-\textsc{auto}-stay-\textsc{pl} \\
\glt `May you stay (here).'
\end{exe}

Non-final syllables have strong phonotactic constraints at least in the native vocabulary (§\ref{sec:non.final.syllable}). The last syllable of verbal and nominal stems generally receives stress (as the syllable \forme{-ʑi-} in \ref{ex:akAtWnWrAZinW}, §\ref{sec:stress}) and allows the maximal number of vowel and consonant contrasts.


\section{Parts of speech} \label{sec:parts.speech.introduction}
\is{parts of speech}
Unlike other Trans-Himalayan languages such as Sinitic, where the identification of word classes requires extensive syntactic analysis \citep{gabelentz1881chinesische, chao68chinese}, most parts of speech in Japhug can be straightforwardly defined on the basis of morphology. 

Japhug has open and closed parts of speech. The three main open parts of speech are \textit{nouns}, \textit{verbs} and \textit{ideophones}.

\is{noun}
Nouns can in their turn be subdivided into four classes with different morphological properties. First, inalienably possessed nouns (§\ref{sec:inalienably.possessed}) require the presence of a possessive prefix (§\ref{sec:possessive.paradigm}). Second, alienably possessed nouns allow a possessive prefix, but do not require it. Third, counted noun (§\ref{sec:counted.nouns}) take numeral prefixes (§\ref{sec:numeral.prefixes}). Fourth, unpossessible nouns cannot be prefixed (§\ref{sec:unpossessible.nouns}).

\is{verb}
Verbs are at the core of Japhug grammar, and have the richest morphology of all parts of speech -- no less than nine chapters (from \ref{chap:verb.template} to \ref{chap:tame}), about half of the grammar, is devoted to verbal morphology. Unlike other Trans-Himalayan languages such as Khaling, 
whose verbs are a closed class \citep{jacques12khaling}, Japhug has a productive system of denominal and deideophonic morphology (§\ref{chap:denominal}), allowing a constant creation of new verbs. Apart from a handful of defective verbs (§\ref{sec:intr.person.irregular}, §\ref{sec:irregular.transitive}), all verbs require orientation preverbs (§\ref{sec:orientation.preverbs}) in most finite forms, and are the only part of speech compatible with these preverbs. Person indexation morphology (§\ref{sec:intransitive.paradigm}) is also a criterion for identifying verbs, but it is not always applicable since some verbs only occur in \textsc{3sg} with no indexation affixes (§\ref{sec:intransitive.invariable}), and since a few predicative words have adopted number indexation suffixes by analogy with verbs (§\ref{sec:non.finite.indexation}).

\is{ideophone}
Ideophones (§\ref{sec:idph}) can also be distinguished from other parts of speech by specific morphological patterns (§\ref{sec:ideo.regular}).

\is{adjective}
There is no specific class of ``adjective'' in Japhug. Words describing properties belong to three different parts of speech: adjectival stative verbs, which can be distinguished from other stative verbs by their ability to undergo the tropative derivation (§\ref{sec:tropative}), ideophones, and also property nouns, a subclass of inalienably possessed nouns (§\ref{sec:property.nouns}).

Closed parts of speech comprise numerals (§\ref{sec:plain.numerals}) and pronouns (chapter \ref{chap:pronouns}), which also have specific morphology, and invariable words including postpositions (§\ref{ex:postpositions}), determiners (§\ref{sec:determiners}), adverbs (§\ref{sec:sentential.adverbs}, §\ref{sec:degree.adverbs}), linkers (§\ref{sec:coordination}, §\ref{sec:tCe.topic}), interjections and calling sounds (§\ref{sec:other.expressive}) and sentence final particles (§\ref{sec:sfp}). 

\section{Nominal morphology} \label{sec:noun.introduction}
\is{noun}
This section presents nominal inflection, derivation as well as grammatical categories expressed by syntactic rather than morphological means within noun phrases and postpositional phrases.

\subsection{Non-attested nominal morphological categories}
There is no gender, definiteness, obviation, negation and tense-aspect-modality-evidential inflection or derivation on nouns in Japhug. 

The whole category of gender is completely absent from Japhug grammar; there are only suffixes for male and female animals, and even these are borrowed from Tibetan (§\ref{sec:gender}). 
\is{gender}

There is no dedicated marker of definiteness (§\ref{sec:definiteness}), but the aforementioned topic marker \forme{iɕqʰa} only occurs on definite referents (§\ref{sec:iCqha}). Indefiniteness can be marked by the indefinite article \forme{ci} (§\ref{sec:indef.article}) derived from the numeral `one' (§\ref{sec:one.to.ten}). 
\is{definite}

While an obviative/proximative contrast is reflected by some uses of the inverse prefix in the inflection of transitive verbs (§\ref{sec:obviation.possessor}), Japhug lacks obviative morphology on nouns (§\ref{sec:possessive.prefix.obv.def}), unlike Algonquian languages (for instance \citealt[183]{valentine01grammar}).
\is{obviative} \is{procimate}
 
Polarity and Tense-aspect-modality-evidentiality, which are prominently encoded by verbal morphology (§\ref{sec:negation}, chapter \ref{chap:tame}), even on nominalized verb forms (§\ref{sec:subject.participle.other.prefixes}), are completely absent from nominal morphology, and various types of participial clauses have to be used instead (§\ref{sec:non.existing.derivation}).
  

\subsection{Number}
\is{number} \is{dual}
Number can be indicated by the dual \forme{ni} (§\ref{sec:dual.determiners}) and plural \forme{ra} (§\ref{sec:plural.determiners}) determiners. These determiners are not mutually incompatible with numerals, as shown by (\ref{ex:jla.RnWz.ni}), where the redundant \japhug{ʁnɯz}{two} can be added (these redundant forms, though grammatical, are not very common). The fact that numerals (and other postnominal modifiers, including relative clauses) can be inserted between nouns and number determiners show that they are not analyzable as number suffixes.

\begin{exe}
\ex \label{ex:jla.RnWz.ni}
\gll jla (ʁnɯz) ni \\
male.hybrid.yak two \textsc{du} \\
\glt `The two male hybrid yaks' 	\japhdoi{0003660\#S7}
\end{exe}

Some nouns however do have an inflectional number category, expressed by a prefixal paradigm partially illustrated in Table \ref{tab:num.prefixes.introduction} (§\ref{sec:numeral.prefixes}). 

\begin{table}
\caption{Numeral prefixes of counted nouns}  \label{tab:num.prefixes.introduction} 
\begin{tabular}{llll}
\lsptoprule
 & Numeral &  \forme{-sŋi} `day'   \\
\midrule
 1	& \forme{ci}  &	\forme{tɯ-sŋi}&  `one day'	\\
2	&	\forme{ʁnɯz}  &	\forme{ʁnɯ-sŋi}&  `two days'	\\
3	&	\forme{χsɯm}  &	\forme{χsɯ-sŋi}&  `three days'	\\
4	&	\forme{kɯβde}  &	\forme{kɯβde-sŋi}, \forme{kɯβdɤ-sŋi}  &`four days'	\\
5	&	\forme{kɯmŋu}  &	\forme{kɯmŋu-sŋi}, \forme{kɯmŋɤ-sŋi}  &	`five days'\\
6	&	\forme{kɯtʂɤɣ}  &	\forme{kɯtʂɤ-sŋi}  &`six days'	\\
7	&	\forme{kɯɕnɯz}  &	\forme{kɯɕnɯ-sŋi}  &`seven days'	\\
8	&	\forme{kɯrcat}  &	\forme{kɯrcɤ-sŋi}  &`eight days'	\\
9	&	\forme{kɯngɯt}  &	\forme{kɯngɯ-sŋi}  &`nine days'	\\
10	&	\forme{sqi}  &	\forme{sqɯ-sŋi}   & `ten days'\\
\lspbottomrule
\end{tabular}
\end{table}

This type of nouns corresponds to the category called `classifiers' (\citealt[518]{chao68chinese}, \citealt{aikhenvald00classifiers}) (\zh{量词} <liàngcí> in Chinese) in works on the grammar of Chinese, Japanese and other languages of East Asia. However, this terminology is particularly clumsy in the case of Japhug.  Unlike in languages such as Chinese or Thai, nouns in Japhug do not require a `classifier' to be used with a numeral, as shown by (\ref{ex:tCheme.ci})\footnote{The numeral \japhug{ci}{one} is grammaticalized as an indefinite marker (§\ref{sec:indef.article}).} and (\ref{ex:tCheme.XsWm}).


\begin{exe}
\ex \label{ex:tCheme.ci}
\gll tɕʰeme ci \\
girl one/\textsc{indef} \\
\glt `A/one girl'
\ex \label{ex:tCheme.XsWm}
\gll tɕʰeme χsɯm \\
girl three \\
\glt `Three girls'
\end{exe}

When occurring as postnominal modifiers (\ref{ex:tCheme.tWrdoR}), the two main functions of nouns with numerals prefixes are partitive (`one of the $X$') and distributive (`each $X$') depending on the constructions where they appear (§\ref{sec:CN.quantifier}). 

\begin{exe}
\ex \label{ex:tCheme.tWrdoR}
\gll tɕʰeme tɯ-rdoʁ \\
girl one-piece \\
\glt `One of the girls'; `Each girl...'; `One girl' \japhdoi{0004053\#S36}
\end{exe}

In addition, although a handful of  `classifiers' are indeed specific to a particular semantic category of nouns (§\ref{sec:CN.classification}), the generic `classifier' \japhug{tɯ-rdoʁ}{one piece} can be used as modifier with nearly all referents.

For these reasons, this grammar favours the term \textit{counted noun} based on morphology, rather than ``classifier'' (an extremely marginal function of these words) or ``quantifier'' (not specific enough, since there are many quantifiers that are not counted nouns, §\ref{sec:quantifiers.determiners}) to refer to nouns with numeral prefixes such as \japhug{tɯ-sŋi}{one day} or \japhug{tɯ-rdoʁ}{one piece}.

\subsection{Case marking} \label{sec:case.intro}
Japhug lacks case inflection, but has a few regular adverbializing derivations (§\ref{sec:denominal.adverb}), whose functions resemble that of oblique cases: the comitative \forme{kɤ́-} (§\ref{sec:comitative.adverb}) and the perlative (§\ref{sec:perlative}).

Grammatical relations on noun phrases are encoded by postpositions such as the ergative \forme{kɯ} (§\ref{sec:erg.kW}), the genitive \forme{ɣɯ} (§\ref{sec:genitive}) and the comitative \forme{cʰo} (§\ref{sec:comitative}), as well as relator nouns (§\ref{sec:relator.nouns}) such as the dative \forme{ɯ-ɕki} or \forme{ɯ-pʰe} (§\ref{sec:dative}). With the sole exception of the genitive forms of a few pronouns such as \forme{aʑɯɣ} \textsc{1sg}:\textsc{gen} (from \japhug{aʑo}{\textsc{1sg}} and \forme{ɣɯ}, §\ref{sec:pronouns.gen}), the postpositions do not merge phonologically with the previous word. As shown by (\ref{ex:Wpi.ni.kW}) and (\ref{ex:ra.GW.nWfsapaR}), they are located at the end of the noun phrase, further away from the head noun than all determiners, including number markers. 

\begin{exe}
\ex \label{ex:Wpi.ni.kW}
\gll [[ɯ-pi ni] \textbf{kɯ}] pɣa nɯ pa-mto-ndʑi\\
\textsc{3sg}.\textsc{poss}-elder.sibling \textsc{du} \textsc{erg} bird \textsc{dem} 3\flobv{}:\textsc{aor}-see-\textsc{du} \\
\glt `His two elder siblings saw the bird.' 
\end{exe}

\begin{exe}
\ex \label{ex:ra.GW.nWfsapaR}
\gll [[tɯrme ra] \textbf{ɣɯ}] nɯ-fsapaʁ \\
person \textsc{pl} \textsc{gen} \textsc{3pl}.\textsc{poss}-cattle \\
\glt `People's cattle' 
\end{exe}

 \is{genitive}
In addition, it is possible to make a pause between the noun and the postposition \forme{kɯ} or \forme{ɣɯ} that follows: the postposition can be procliticized to the following word. A considerable number of examples can be found in the corpus (§\ref{ex:word.vs.clitic.postp}).

\is{locative}
While Situ Gyalrong does have locative suffixes (\citealt[325--331]{linxr93jiarong}), Japhug only uses postpositions (§\ref{sec:core.locative}) and/or relator nouns (§\ref{sec:relator.postposition.location}) to express location, goal and source of motion. The only traces of the proto-Gyalrong suffixes are the locative postposition \forme{zɯ}, which was degrammaticalized from a suffix \forme{*-s} (§\ref{sec:core.locative}), and a few isolated lexicalized forms (§\ref{sec:locative.j}, §\ref{sec:relator.postposition.location}).

\subsection{Possession}
The main nominal morphosyntactic category expressed by an inflectional paradigm in Japhug is possession, encoded by a series of possessive prefixes  (§\ref{sec:possessive.paradigm}).

\subsubsection{Possessive paradigm} \label{sec:possessive.paradigm.intro}
\is{prefix!possessive} \is{possession!prefix}
\begin{table}
\caption{Possessive paradigms} \label{tab:possessive.paradigms.intro}
\begin{tabular}{lllll} 
\lsptoprule
Person & Prefix & \japhug{tɯ-ku}{head} & \japhug{kʰa}{house}  \\
\midrule
1\sg{}  &\forme{a-} &\forme{a-ku} & \forme{a-kʰa} \\
2\sg{} &\forme{nɤ-} & \forme{nɤ-ku} & \forme{nɤ-kʰa} \\
3\sg{}& \forme{ɯ-}   &\forme{ɯ-ku} & \forme{ɯ-kʰa} \\
\midrule
1\du{} &\forme{tɕi-}   &\forme{tɕi-ku} & \forme{tɕi-kʰa} \\
2/3\du{}&\forme{ndʑi-} &\forme{ndʑi-ku} & \forme{ndʑi-kʰa} \\
\midrule
1\pl{} & \forme{ji-} 	&\forme{ji-ku} & \forme{ji-kʰa} \\
2/3\pl{}&\forme{nɯ-}   &\forme{nɯ-ku} & \forme{nɯ-kʰa} \\
\midrule
indefinite&\forme{tɯ-/tɤ-/ta-}    &\textbf{\forme{tɯ-ku}} & \textbf{\forme{kʰa}} \\
generic&\forme{tɯ-}    &\forme{tɯ-ku} & \forme{tɯ-kʰa} \\
\lspbottomrule
\end{tabular}
\end{table}

The possessive paradigm is nearly the same for all nouns (\tabref{tab:possessive.paradigms.intro}), but some nouns such as \japhug{tɯ-ku}{head} require a indefinite possessor prefix \forme{tɯ\trt}, \forme{tɤ-} or \forme{ta-} when no definite possessor is present, while other nouns like \japhug{kʰa}{house} can occur in bare stem form. The former are \textit{inalienably possessed nouns}, comprising in particular body parts (§\ref{sec:body.part}) and kinship terms (§\ref{sec:kinship}), while the latter are \textit{alienably possessed nouns}. Some inalienably possessed nouns have been grammaticalized as relator nouns (§\ref{sec:relator.nouns}) marking the syntactic function of noun phrases. Furthermore, a handful of nouns have become TAME markers (§\ref{sec:WmdoR.TAME}).

Possessors are obligatorily indicated by possessive prefixes. An overt possessor can be optionally added. For instance, the meaning `my cow' can be expressed by the noun form \forme{a-nɯŋa} with a simple possessive prefix, but this noun can be additionally preceded by the genitive pronoun \forme{aʑɯɣ} (\ref{ex:aZWG.anWNa}) or even by the absolutive \forme{aʑo}.

\begin{exe}
\ex \label{ex:aZWG.anWNa}
\gll (aʑɯɣ) a-nɯŋa \\
\textsc{1sg} \textsc{1sg}.\textsc{poss}-cow \\
\glt `My cow' 
\end{exe}

The possessive prefix cannot be elided, even in the case of alienably possessed nouns like \japhug{nɯŋa}{cow}. In (\ref{ex:aZWG.anWNa}), removing the \forme{a-} prefix would result in an agrammatical form ($\dagger$\forme{aʑɯɣ nɯŋa}).

The phrase expressing the possessor always precedes the possessum. Genitive marking on the possessor is optional: in (\ref{ex:ni.GW.ndZinWNa}), the genitive postposition \forme{ɣɯ} (§\ref{sec:gen.possession}) can be elided.

\begin{exe}
\ex \label{ex:ni.GW.ndZinWNa}
\gll [a-mu a-wa ni] (ɣɯ) ndʑi-nɯŋa \\
\textsc{1sg}.\textsc{poss}-mother \textsc{1sg}.\textsc{poss}-father \textsc{du} \textsc{gen} \textsc{3du}.\textsc{poss}-cow \\
\glt `My parents' cow' 
\end{exe}

The person and number of the possessive prefix on the possessum is the same as that of the noun phrase or pronoun marking the possessor: in (\ref{ex:ni.GW.ndZinWNa}) for instance, the third dual possessive prefix \forme{ndʑi-} agrees in number with the dual possessor phrase \forme{a-mu a-wa ni}. Number agreement mismatch is only attested in very restricted contexts (§\ref{ex:prefix.expression.of.possession}).

\subsubsection{Alienabilization} \label{sec:alienabilization.intro}
\is{alienabilization}
Nouns can only take one single possessive prefix, except when inalienably possessed nouns are turned into alienably possessed nouns (§\ref{sec:alienabilization}), by stacking a definite possessor prefix (any of the prefixes in \tabref{tab:possessive.paradigms.intro} except the indefinite ones) on the indefinite possessive form. For instance, the possessed form \forme{ɯ-lu} (\ref{ex:nWNa.Wlu}) of \japhug{tɤ-lu}{milk} without prefix stacking is used when the possessor is the cow \textit{producing} the milk, but the alienabilized possessive forms \forme{ɯ-tɤ-lu} `his/her/its milk, the milk for him/her/it' (\ref{ex:lWlu.WtAlu}) or \forme{a-tɤ-lu} `my milk' (\ref{ex:atAlu}), expressing the person or animal \textit{drinking} the milk as possessor, have a combination of two prefixes.

\begin{exe}
\ex 
\begin{xlist}
\ex \label{ex:nWNa.Wlu}
\gll nɯŋa (ɣɯ) ɯ-lu \\
cow \textsc{gen} \textsc{3sg}.\textsc{poss}-milk \\
\glt `(The/a) cow's milk' 
\ex \label{ex:lWlu.WtAlu}
\gll lɯlu (ɣɯ) ɯ-tɤ-lu \\
cat \textsc{gen} \textsc{3sg}.\textsc{poss}-\textsc{indef}.\textsc{poss}-milk \\
\glt `The milk for the cat (i.e. given to the cat to drink)' 
\ex \label{ex:atAlu}
\gll (aʑo) a-tɤ-lu \\
\textsc{1sg} \textsc{1sg}.\textsc{poss}-\textsc{indef}.\textsc{poss}-milk \\
\glt `My milk (i.e. for me to drink)' 
\end{xlist}
\end{exe}

Stacking of two definite possessor prefixes, or of a numeral prefix with a definite possessor prefix, are not grammatical.

\subsubsection{Generic possessors}
\is{generic}
The indefinite possessor prefix has three allomorphs \forme{tɯ-} (as in in \japhug{tɯ-ku}{head}), \forme{tɤ-} (as in \japhug{tɤ-se}{blood}) or \forme{ta-} (as in \japhug{ta-ma}{work}). It has to be distinguished from the generic possessor prefix \forme{tɯ-} (§\ref{sec:indef.genr.poss}). Generic possessors are identical with indefinite possessors in the case of inalienably possessed nouns selecting the \forme{tɯ-} prefix, for instance \forme{tɯ-ku} can either mean `head' or `one's head'. With inalienably possessed nouns selecting the \forme{tɤ-} or \forme{ta-} allomorphs, a contrast is found between \forme{tɤ-se} `blood' and \forme{tɯ-se} `one's blood' (\ref{ex:tWse.kutshi}) for instance.

\begin{exe}
\ex \label{ex:tWse.kutshi}
\gll qajɯsmɤnba kɯ tɯ-se ku-tsʰi ŋu \\
leech \textsc{erg} \textsc{genr}.\textsc{poss}-blood \textsc{ipfv}-drink be:\textsc{fact} \\
\glt `The leech drinks people's (i.e. one's) blood.' 
\end{exe}

The generic possessor prefix can also occur on alienably possessed nouns, as in \forme{tɯ-kʰa} `one's house'. 

No more than one generic referent is possible per clause, so that if a noun with generic possessor prefix is found in the same clause as a verb with generic indexation, there is obligatory co-reference (§\ref{sec:indexation.generic.tr}), as in (\ref{ex:tWrpW.WrZaB}) between the possessor of \forme{tɯ-rpɯ} `one's mother's brother' (indefinite form \forme{tɤ-rpɯ}, §\ref{sec:kinship}) and the transitive subject of the verb \forme{tu-kɯ-ti} `one says' (§\ref{sec:irregular.transitive}).

\begin{exe}
\ex \label{ex:tWrpW.WrZaB}
\gll \textbf{tɯ}-rpɯ ɣɯ ɯ-rʑaβ ɯ-ɕki tɕe ``a-ɬaʁ" tu-\textbf{kɯ}-ti ŋu \\
\textsc{\textbf{genr}}.\textsc{poss}-mother's.brother \textsc{gen} \textsc{3sg}.\textsc{poss}-wife \textsc{3sg}.\textsc{poss}-\textsc{dat} \textsc{loc} \textsc{1sg}.\textsc{poss}-aunt \textsc{ipfv}-\textsc{\textbf{genr}}-say be:\textsc{fact} \\
\glt `\textbf{One}$_i$ calls \textbf{one}$_i$'s mother's brother's wife `my aunt' (i.e. one says `my aunt' to one's mother's brother's wife).' 
\end{exe}

\subsubsection{Possessive existential construction}
\is{predicative possession}
Predicative possession can be expressed by the verb \japhug{aro}{own} (§\ref{sec:semi.transitive}, §\ref{sec:possessive.constructions}), encoding the possessor as subject, but the most frequent construction involves an existential verb (§\ref{sec:existential.basic}) with the possessum as subject and the possessor marked by a possessive prefix on the possessum, optionally with a genitive phrase (§\ref{sec:possessive.mihi.est}). The construction is the same for alienably (\ref{ex:nWnWNa.XsWm.pjAtu}) and inalienably (\ref{ex:nWtCW.XsWm.pjAtu}) possessed nouns.
 

\begin{exe}
\ex 
\begin{xlist}
\ex \label{ex:nWnWNa.XsWm.pjAtu}
\gll  kɯβʁa ra ɣɯ nɯ-nɯŋa kɯ-dɤn pjɤ-tu \\
noble \textsc{pl} \textsc{gen} \textsc{3pl}.\textsc{poss}-cow \textsc{sbj}:\textsc{pcp}-be.many \textsc{ifr}.\textsc{ipfv}-exist \\
\glt `The nobles had many cows.'  
\ex \label{ex:nWtCW.XsWm.pjAtu}
\gll  kɯβʁa ra ɣɯ nɯ-tɕɯ χsɯm pjɤ-tu \\
noble \textsc{pl} \textsc{gen} \textsc{3pl}.\textsc{poss}-son three \textsc{ifr}.\textsc{ipfv}-exist \\
\glt `The nobles had three sons.'  
\end{xlist}
\end{exe}

In this construction, the existential verbs rarely agree in number with the possessum: in both (\ref{ex:nWnWNa.XsWm.pjAtu}) and (\ref{ex:nWtCW.XsWm.pjAtu}), the singular verb form \forme{pjɤ-tu} is by far more commonly used than its plural counterpart (§\ref{sec:optional.indexation}).

\subsection{Compounding}
\is{compounding}
Compounding can be realized by the simple concatenation of noun (or verb) stems.

In compounds comprising two noun stems (§\ref{sec.n.n.compounds}), whose order is modifier-modified, no possessive prefixes occur between the nominal stems, even if the second noun is inalienably possessed. For instance, the compound built from \japhug{kɯrɯ}{Tibetan} and the inalienably possessed \japhug{tɯ-ŋga}{clothes} is \forme{kɯrɯ-ŋga} `Tibetan clothes', rather than $\dagger$\forme{kɯrɯ-tɯŋga}.

In many cases, the non-final elements of the compound appear in a bound form, the bound state (§\ref{sec:status.constructus}), which is characterized by a vocalic change to either \forme{-ɤ} or \forme{-ɯ} (§\ref{sec:vowel.alternations.compounds}) For instance, compounding \japhug{tɯ-ku}{head} with \japhug{tɤ-rme}{hair} yields \forme{tɤ-kɤ-rme} `head hair' with the bound state \forme{kɤ-} from \forme{-ku}. The compound inherits the allomorph \forme{tɤ-} of the indefinite possessor prefix of the head of the compound \japhug{tɤ-rme}{hair}.
 

\subsection{Derivations}
Some suffixal nominal derivations come from compounds with a grammaticalized noun as second element, in particular the diminutive suffixes (§\ref{sec:diminutive}). The recent origin of these suffixes can be shown by the fact that a free form still co-exists with the corresponding compound in some cases. For instance, the compound \forme{χpɯn-pɯ} `little monk' from \japhug{χpɯn}{monk} with the diminutive \forme{-pɯ} occurs in free variation with the phrase (\ref{ex:XpWn.WpW}), in which the main noun is followed by the alienably possessed property noun \japhug{ɯ-pɯ}{little one} (§\ref{sec:property.nouns}) derived from \japhug{tɤ-pɯ}{offspring, young}.

\begin{exe}
\ex \label{ex:XpWn.WpW}
\gll χpɯn ɯ-pɯ \\
monk \textsc{3sg}.\textsc{poss}-little.one  \\
\glt `(The/a) little monk' 
\end{exe}

Not all suffixal derivations have transparent origins. For instance, the privative \forme{-lu} suffix (§\ref{sec:privative}) is not related to any independently attested nominal or verbal root. It turns an inalienably possessed noun into a non-possessible one mainly used as postnominal modifier (§\ref{ex:attributive.postnominal}), removing possessive prefixes and subjecting the nominal stem to bound state alternation (\japhug{tɯ-ku}{head} \fl{} \forme{kɤ-lu} `headless').

Prefixal nominal derivations, however, do not originate from elements of compounds, but rather from participial forms of denominal verbs (§\ref{sec:denom.aGW}, §\ref{sec:denom.andZi}). 

The social relation collective \forme{kɤndʑi-} prefix (§\ref{sec:social.collective}) occurs on the bare stem of kinship terms and a few other terms of social relationship to indicate a group of people. When the group members are  related to each other by a symmetrical relationship (\japhug{tɤ-xtɤɣ}{brother} (of a male) \fl{} \forme{kɤndʑi-xtɤɣ} `group of brothers'), the social relation collective is based on only one nominal stem, but when the relationship is asymmetrical, \forme{kɤndʑi-} can be prefixed to two compounded noun stems, as in \forme{kɤndʑi-wɤmɯ-snom} `group of siblings' from \japhug{tɤ-wɤmɯ}{brother} (of a female) and \japhug{tɤ-snom}{sister} (of a male).
 
The comitative derivation (§\ref{sec:comitative.adverb}) is built by adding the prefixes \forme{kɤ́-} or \forme{kɤɣɯ-} to reduplicated noun stems. It derives an adverb meaning `together with $X$' which can have scope over the whole clause, or be restricted to a noun phrase. It can apply to both inalienably possessed nouns (\japhug{tɯ-ŋga}{clothes} \fl{} \forme{kɤ́-ŋgɯ\redp{}ŋga} `together with his/her clothes') or alienably possessed ones (\japhug{jla}{male hybrid yak} \fl{} \forme{kɤ́-jlɯ\redp{}jla} `together with the/his/her hybrid yak'). Comitative derivation can preserve the indefinite possessor prefix of inalienably possessed nouns, causing alienabilization (§\ref{sec:alienabilization.intro}, §\ref{sec:alienabilization}). For instance, \japhug{tɤ-rte}{hat} has two comitative forms: \forme{kɤ́-rtɯ\redp{}rte} `together with his/her hat' (wearing it)  and \forme{kɤ́-tɤ-rtɯ\redp{}rte} `together with a/the hat' (not wearing it).

\section{Verbal morphology} \label{sec:verb.introduction}

\subsection{Overview} \label{sec:verb.overview.intro}
Verbs have a considerably more elaborate morphology than all other parts of speech (§\ref{sec:verb.intro}). Verbal morphology is strongly prefixal (§\ref{sec:prefixal.chain}), with some vowel contractions (§\ref{sec:contraction}). Non-concatenative morphology includes stem alternations (§\ref{sec:stem.alternation}) as well as infixation (§\ref{sec:inner.prefixal.chain}, §\ref{sec:intr.person.irregular}, §\ref{sec:autoben.position}). 

Japhug verbal morphology is considerably more regular than that of other Gyalrong languages, in particular Zbu (\citealt{jackson04showu, gong18these}) and Situ \citep{zhangsy18stem}. Most alternations are productive and predictable, and irregular verbs are limited in number (§\ref{sec:intr.person.irregular}, §\ref{sec:irregular.transitive}).

 As illustrated by (\ref{ex:amAGWnWtWwGznAre}), a verb form can comprise six inflectional prefixes (in blue) arranged in a rigid template (§\ref{sec:outer.prefixal.chain}) and several derivational prefixes (in red).
 
\begin{exe}
\ex \label{ex:amAGWnWtWwGznAre}
\gll   \bleu{a$^{-6}$-mɤ$^{-5}$-ɣɯ$^{-4}$-nɯ$^{-3}$-tɯ́$^{-2}$-wɣ$^{-1}$}-\rouge{z-nɤ}-re \\
\bleu{\textsc{irr}$^{-6}$-\textsc{neg}$^{-5}$-\textsc{cisl}$^{-4}$-\textsc{pfv}$^{-3}$-2$^{-2}$-\textsc{inv}$^{-1}$}-\rouge{\textsc{caus}-\textsc{denom}}-laughter \\
\glt `Don't let him come and make you laugh.'
\end{exe}
 
The suffixal chain (§\ref{sec:suffixes}) only includes inflectional suffixes, with a maximal number of four slots (\ref{ex:WmApWkWmtoandZici.intro}).  

 \begin{exe}
\ex \label{ex:WmApWkWmtoandZici.intro}
\gll  \bleu{ɯmɤ$^{-6}$-pɯ$^{-3}$-kɯ$^{-2}$}-mto-\bleu{t$^{+1}$-a$^{+2}$-ndʑi$^{+3}$-ci$^{+4}$} \\
 \bleu{\textsc{prob}$^{-6}$-\textsc{aor}$^{-3}$-\textsc{peg}$^{-2}$}-see-\bleu{\textsc{pst}:\textsc{tr}$^{+1}$-\textsc{1sg}$^{+2}$-\textsc{du}$^{+3}$-\textsc{peg}$^{+4}$} \\
 \glt `It looks like I have seen the two of them.'  
\end{exe} 
 
 Numerous non-adjacent dependencies (§\ref{sec:templatic.verb}) are observed across the prefixal and the suffixal chains.
 
Inflectional verbal morphology encodes person and number of one or two core arguments (chapter \ref{chap:indexation}, §\ref{sec:indexation.tr.intro}), orientation (§\ref{sec:orientation.preverbs}), associated motion (§\ref{sec:associated.motion}), negation (§\ref{sec:negation}) and Tense-Aspect-Modality-Evidentiality (chapter \ref{chap:tame}). 
 
 
\subsection{Indexation} \label{sec:indexation.intro}
 \is{indexation} 
All finite verb forms in Japhug have obligatory person indexation. Since the \textsc{3sg} has zero marking as in many languages of the world \citep[227--236]{benveniste66problemes1}, indexation is not conspicuous on intransitive verbs requiring a \textsc{3sg} subject (§\ref{sec:intransitive.invariable}), but indirectly observable even on transitive dummy verbs (§\ref{sec:transitive.dummy}) due to the presence of stem alternation (§\ref{sec:stem3}).

Transitive and intransitive verbs are clearly distinguished by a series of seven morphological parameters (§\ref{sec:transitivity.morphology}). Intransitive verbs only index one argument (the intransitive subject, S), and transitive verbs index two arguments (the transitive subject A and the object O). Semi-transitive verbs have intransitive indexation (§\ref{sec:semi.transitive}), but select a second core argument (§\ref{sec:semi.object}). Only a handful of verbs are labile, and can be conjugated either transitively or intransitively (§\ref{sec:lability}).

\subsubsection{The intransitive paradigm}
\is{verb!intransitive}
Person indexation in Japhug is best introduced with the intransitive paradigm (§\ref{sec:intransitive.paradigm}), since it it considerably smaller than the transitive one. The regular paradigm is illustrated in \tabref{tab:intransitive.indexation.intro} with the verb \japhug{mbɣom}{be in a hurry} in the Factual Non-Past, the only TAME without orientation preverb (§\ref{sec:factual}).
 
\begin{table}[H]  
\caption{The intransitive indexation paradigm} \label{tab:intransitive.indexation.intro}
\begin{tabular}{llll} \lsptoprule
Person & Form & Example   \\
\midrule
\textsc{1sg} & \ro{}-\forme{a} & \forme{mbɣom-a} \\
\textsc{1du} & \ro{}-\forme{tɕi} & \forme{mbɣom-tɕi} \\
\textsc{1pl} & \ro{}-\forme{ji} & \forme{mbɣom-i} \\
\midrule
\textsc{2sg} & \forme{tɯ}-\ro{} & \forme{tɯ-mbɣom} \\
\textsc{2du} & \forme{tɯ}-\ro{}-\forme{ndʑi} & \forme{tɯ-mbɣom-ndʑi} \\
\textsc{2pl} & \forme{tɯ}-\ro{}-\forme{nɯ} & \forme{tɯ-mbɣom-nɯ} \\
\midrule
\textsc{3sg} & \ro{} & \forme{mbɣom} \\
\textsc{3du} & \ro{}-\forme{ndʑi} & \forme{mbɣom-ndʑi} \\
\textsc{3pl} & \ro{}-\forme{nɯ} & \forme{mbɣom-nɯ} \\
\lspbottomrule
\end{tabular}
\end{table}

The intransitive paradigm has different forms for singular, dual and plural. First persons have dedicated suffixes encoding both person and number; there is no inclusive/exclusive contrast. Second and third person forms are distinguished by the second person \forme{tɯ-} prefix (on the historical significance of this prefix, see \citealt{jacques12agreement} and \citealt{delancey14second}). There is no overt third person marker on intransitive verbs. Number markers are shared by second and third person forms: absence of suffix for the singular, \forme{-ndʑi} for the dual and \forme{-nɯ} for the plural. Slightly different forms are found in dialects of Japhug other than Kamnyu (§\ref{sec:indexation.suffixes.history}). In addition to the paradigm in \tabref{tab:intransitive.indexation.intro}, a generic person \forme{kɯ-} prefix also occurs on intransitive verbs (§\ref{sec:indexation.generic.tr}). 

A handful of intransitive verbs infix rather than prefix the second person (§\ref{sec:intr.person.irregular}). It is the only irregularity related to person indexation in intransitive verbs in Japhug.


\subsubsection{The transitive paradigm} \label{sec:indexation.tr.intro}
\is{verb!transitive}
The transitive paradigm is too large to be described in this introductory chapter in its entirety (§\ref{sec:polypersonal}). \tabref{tab:transitive.paradigm.singular} presents the singular forms of the paradigm of the transitive verb \japhug{sat}{kill} (a verb lacking stem alternations) in the Factual Non-Past, which are sufficient to illustrate its basic structure. In this table, the columns represent the objects (O), and the rows the subjects (A). The shaded cells indicate configurations with coreferent subject and object, which are expressed by the reflexive derivation (§\ref{sec:reflexive}) and do not belong to the transitive paradigm.

\begin{table}[H] 
\caption{The transitive paradigm (singular forms) of the non-alternating verb \japhug{sat}{kill} in the Factual Non-Past} 
 \centering \label{tab:transitive.paradigm.singular}
\begin{tabular}{l|l|l|lll} 
\lsptoprule
&1O & 2O &3O&3$'$O\\
\hline
1A&\grise{}& \forme{ta-\textbf{sat}} & \forme{\textbf{sat}-a} & \\
\hline
2A&\forme{kɯ-\textbf{sat}-a} & \grise{} & \forme{tɯ-\textbf{sat}} & \\
\hline
3A& \forme{ɣɯ́-\textbf{sat}-a} & \forme{tɯ́-wɣ-\textbf{sat}} & \grise{} &\forme{\textbf{sat}} \\
3$'$A & & &\forme{ɣɯ́-\textbf{sat}} &\grise{} \\
\lspbottomrule
\end{tabular}
\end{table}

The comparison of Tables \ref{tab:intransitive.indexation.intro} and \ref{tab:transitive.paradigm.singular} shows that the \textsc{1sg} \forme{-a} suffix and the second person \forme{tɯ-} prefix have neutral alignment: they can index intransitive subject, transitive subject or object. In the following, the person configurations are referred to by $X \rightarrow Y$, where $X$ represents the subject and $Y$ the object (for instance 1\fl{}3 means `first person subject, third person object').  The eight non-shaded cells of the paradigm in \tabref{tab:transitive.paradigm.singular} can be divided into three groups. 

First, 1\fl{}3, 2\fl{}3 and 3\flobv{} are the \textit{direct} configurations with a third person object, whose forms resemble the 1, 2 and 3 forms of the intransitive paradigm (at least in the Factual Non-Past). 

Second, 3\fl{}1, 3\fl{}2 and 3$'$\fl{}3 are the \textit{inverse} configurations, which have the same prefixes or suffixes as the corresponding intransitive and direct forms (at least in this paradigm), but take in addition the inverse prefix \forme{ɣɯ-}/\forme{wɣ-} (the allomorphy of this prefix is explained in §\ref{sec:allomorphy.inv}). 

Third, 1\fl{}2 and 2\fl{}1 (without third person) are the the \textit{local} configurations (§\ref{sec:indexation.local}). They are characterized by the presence of the portmanteau \forme{ta-} and \forme{kɯ-} prefixes (§\ref{sec:portmanteau.prefixes.history}) not found in the intransitive paradigm.

The reasons for using the terms ``direct'' and ``inverse'' to describe the transitive paradigm are discussed in §\ref{sec:direct-inverse}.

When both arguments are third person (§\ref{sec:indexation.non.local}), there is a contrast between direct  3\flobv{} and inverse 3$'$\fl{}3 configurations, whose meaning is not entirely straightforward  (§\ref{sec:inverse.3.3.saliency}). The subject of the direct configuration, and object of the inverse one is called \textit{proximate} (3), and the other argument \textit{obviative} (3$'$). The direct 3\flobv{} configuration is by far the most common one in narratives and conversation. The inverse 3$'$\fl{}3 is more restricted; it occurs in particular to index a generic subject with a third person object (§\ref{sec:indexation.generic.tr}), and also when the subject is inanimate and the object animate (§\ref{sec:obviation.animacy}).

The majority of transitive verbs (all verbs ending in closed syllables or with front vowels) are non-alternating like \japhug{sat}{kill}. However, about a third of all transitive verbs (those ending in \forme{-a}, \forme{-u}, \forme{-o} and \forme{-ɯ}) have stem alternation in the \textit{direct} configurations with a singular subject in the Factual Non-Past and a few other tenses (§\ref{sec:stem3}).
 
\begin{table}[H] 
\caption{The transitive paradigm (singular forms) of the alternating verb \japhug{ʁndɯ}{hit} in the Factual Non-Past} 
 \centering \label{tab:transitive.paradigm.singular.alternating}
\begin{tabular}{l|l|l|lll} 
\lsptoprule
&1O & 2O &3O&3$'$O\\
\hline
1A&\grise{}& \forme{ta-\textbf{ʁndɯ}} & \forme{\textbf{ʁndi}-a} \acell & \acell \\
\hline
2A&\forme{kɯ-\textbf{ʁndɯ}-a} & \grise{} & \forme{tɯ-\textbf{ʁndi}} \acell & \acell \\
\hline
3A& \forme{ɣɯ́-\textbf{ʁndɯ}-a} & \forme{tɯ́-wɣ-\textbf{ʁndɯ}} & \grise{} &\forme{\textbf{ʁndi}} \acell \\
3$'$A & & &\forme{ɣɯ́-\textbf{ʁndɯ}} &\grise{} \\
\lspbottomrule
\end{tabular}
\end{table}

\tabref{tab:transitive.paradigm.singular.alternating} illustrates the singular forms of the alternating verb \japhug{ʁndɯ}{hit}, which has additional stem \forme{-ʁndi} in the \textsc{1sg}\fl{}3, \textsc{2sg}\fl{}3 and \textsc{3sg}\flobv{} configurations (with blue colouring), which are thus different from the corresponding intransitive forms.

In the Aorist (§\ref{sec:aor.morphology}), a slightly different paradigm is found, illustrated in \tabref{tab:transitive.paradigm.singular.aorist} with the \japhug{ʁndɯ}{hit}. Unlike the Factual Non-Past shown in the previous tables, the Aorist requires an orientation preverb (§\ref{sec:kamnyu.preverbs}), here the \textsc{upwards} \forme{tɤ-}. There is no stem alternation marking the direct forms, but all verbs have a different series of preverbs (here \forme{ta-}) in the direct 3\flobv{} forms (§\ref{sec:aor.morphology}, §\ref{sec:kamnyu.preverbs}). In addition, the \textsc{1sg}\fl{}3 and \textsc{2sg}\fl{}3 forms require a \forme{-t} suffix which redundantly encodes both person and tense-aspect (§\ref{sec:suffixes}).

\begin{table}[H] 
\caption{The transitive paradigm (singular forms) of the verb \japhug{ʁndɯ}{hit} in the Aorist} 
 \centering \label{tab:transitive.paradigm.singular.aorist}
\begin{tabular}{l|l|l|lll} 
\lsptoprule
&1O & 2O &3O&3$'$O\\
\hline
1A&\grise{}& \forme{tɤ-ta-\textbf{ʁndɯ}} & \forme{tɤ-\textbf{ʁndɯ}-\rouge{t}-a}  &  \\
\hline
2A&\forme{tɤ-kɯ-\textbf{ʁndɯ}-a} & \grise{} & \forme{tɤ-tɯ-\textbf{ʁndɯ}-\rouge{t}}  &  \\
\hline
3A& \forme{tɤ́-wɣ-\textbf{ʁndɯ}-a} & \forme{tɤ-tɯ́-wɣ-\textbf{ʁndɯ}} & \grise{} &\forme{\rouge{ta}-\textbf{ʁndɯ}}  \\
3$'$A & & &\forme{tɤ́-wɣ-\textbf{ʁndɯ}} &\grise{} \\
\lspbottomrule
\end{tabular}
\end{table}

There is no ambiguity in person indexation in Japhug (unlike for instance in Khaling where 2\fl{}1 and 3\fl{}1 configurations are identical, \citealt{jacques12khaling}), but there are strong restrictions on number indexation: unless the \textsc{1sg} suffix is present, only one of the two arguments can be indexed for both person and number, the subject in direct configurations, and the object in inverse and local configurations. Double number indexation only occurs in forms with the \textsc{1sg} \forme{-a} suffix, to which additional number suffixes can be added (§\ref{sec:double.number.indexation}), for instance \forme{ɣɯ́-ʁndɯ-a-nɯ}  `they will hit me' (\textsc{3pl}\fl{}\textsc{1sg}) where the plural morpheme \forme{-nɯ}, indexing the number of the subject, follows the \textsc{1sg}.

\subsubsection{Person indexation and finiteness} \label{sec:indexation.finiteness.intro}
\is{indexation} 
Person indexation markers (including person-indexing stem alternation and orientation preverbs) are not found on participles, infinitives and other non-finite verb forms (chapter \ref{chap:non-finite}). A handful of phatic and exclamative words of nominal origin have however developed the ability to take number suffixes (§\ref{sec:inflectionalization.intro}, §\ref{sec:non.finite.indexation}). Apart from these, words belonging to parts of speech other than verbs are incompatible with the indexation affixes described in this section.


\subsection{Orientation preverbs and TAME}
This section focuses on morphology. Since the use of the TAME categories involve sometimes subtle semantic nuances, and have to be explained on the basis of examples with a clear context, the discussion of the semantic function of each category is deferred to chapter \ref{chap:tame}.

\subsubsection{The morphology of orientation preverbs}
\is{orientation!preverb}
The main morphological exponents of tense, aspect, modality and evidentiality (henceforth TAME) in Japhug are the orientation preverbs (§\ref{sec:orientation.preverbs}). All regular finite verb forms require \textit{one and only one} preverb (§\ref{sec:outer.prefixal.chain}), except the Factual Non-Past (§\ref{sec:factual}) which does not take any preverb. The stacking of two or more preverbs is ungrammatical. Only a handful of irregular defective verbs are incompatible with orientation preverbs (§\ref{sec:intr.person.irregular}, §\ref{sec:irregular.transitive}).

Preverbs encode one out of seven orientations (\tabref{tab:orientation.preverbs.intro}), divided into three dimensions: vertical (§\ref{sec:vertical.dimension}), riverine (§\ref{sec:riverine.dimension}) and solar (§\ref{sec:solar.dimension}), to which an unspecified orientation is added. This tridimensional system is not restricted to verbal morphology: locative relator nouns (§\ref{sec:relator.nouns.3d}), egressive postpositions (§\ref{sec:egressive}) and locative adverbs (§\ref{sec:preverbs.adverbs}) have similar systems with six orientations, built from morphemes that are historically related to the preverbs.

There are four series of preverbs (\tabref{tab:orientation.preverbs.intro} includes two of them, the series A and B) in the Kamnyu dialect, used in different TAME categories (§\ref{sec:kamnyu.preverbs}).  Some dialects of Japhug have a slightly different system \begin{scriptsize}\begin{footnotesize}\end{footnotesize}\end{scriptsize} (§\ref{sec:xtokavian.preverbs}).

\begin{table}
\caption{Orientation preverbs in Kamnyu Japhug} \label{tab:orientation.preverbs.intro}
\begin{tabular}{llllll}
\lsptoprule
Dimension& Orientation  &  	A &   B    \\  	
   \midrule
Vertical &Up   &  	\forme{tɤ-}   &  	\forme{tu-}   &    \\  	
  & Down   &  	\forme{pɯ-}   &  	\forme{pjɯ-}  &   \\  	
\midrule
Riverine &Upstream   &  	\forme{lɤ-}   &  	\forme{lu-}   &  	   \\  	
  &Downstream   &  	\forme{tʰɯ-}   &  	\forme{cʰɯ-}      \\  	
\midrule
Solar &Eastwards   &  	\forme{kɤ-}   &  	\forme{ku-}       \\  	
  &Westwards   &  	\forme{nɯ-}   &  	\forme{ɲɯ-}      \\  	
\midrule
&Unspecified  &\forme{jɤ-}   &  	\forme{ju-}      \\  	
\lspbottomrule
\end{tabular}
\end{table}

 \textit{Orientable} verbs (§\ref{sec:orientable.verbs}) are compatible with all orientations; this includes in particular motion verbs like \japhug{ɣi}{come} and \japhug{ɬoʁ}{come out} (§\ref{sec:motion.verbs}). With this type of verbs, the preverbs indicate either the absolute direction of the motion (§\ref{sec:tridimensional.preverb}), for instance  \textsc{upwards} in (\ref{ex:tANe.tAlhoR}) or have extended meanings (§\ref{sec:orientation.extended}), such as the illative function (§\ref{sec:illative.elative}) of the \textsc{upstream} preverb in (\ref{ex:WNgW.lAGi}).
 
 \begin{exe}
\ex \label{ex:tANe.tAlhoR}
\gll tɤŋe tɤ-ɬoʁ \\
sun \textsc{aor}:\textsc{up}-come.up \\
\glt `The sun rose.' 
\end{exe} 

 \begin{exe}
\ex \label{ex:WNgW.lAGi}
\gll ɯ-ŋgɯ lɤ-ɣi \\
\textsc{3sg}.\textsc{poss}-in \textsc{imp}:\textsc{upstream}-come \\
\glt `Come in! (for instance, inside a house)' \japhdoi{0003884\#S90}
 \end{exe} 
 
 Non-orientable verbs only select a restricted number of lexically determined orientations, sometimes only one (§\ref{sec:lexicalized.orientation}). For instance, the verb \japhug{ndza}{eat} and  \japhug{mto}{see} require the \textsc{upwards} (§\ref{sec:preverb.ingestion}) and \textsc{downwards} (§\ref{sec:preverb.perception}) preverbs, respectively.

\subsubsection{The morphology of TAME categories}
There are eleven primary TAME categories, which can be divided into four main groups: Non-Past (Factual Non-Past, Egophoric Present, Sensory, §\ref{sec:TAME.npst}), Imperfective (§\ref{sec:imperfective}), Past (Aorist, Inferential, Past Imperfective, and Inferential Imperfective, §\ref{sec:TAME.pst}) and Modal (Irrealis, Imperative, Prohibitive, Dubitative, §\ref{sec:TAME.modal}) categories. 

In the finite TAME categories, the B-type  preverbs are found in the Imperfective (§\ref{sec:imperfective}) and the A-type preverbs in the Imperative (§\ref{sec:imp.morphology}), the Irrealis (§\ref{sec:irrealis.morphology}) and the Aorist (§\ref{sec:aor}), though with slightly different vowel contraction rules (§\ref{sec:contraction}). In the Aorist paradigm of transitive verbs, another series of preverbs (C) is found in the direct 3\flobv{} forms (§\ref{sec:indexation.tr.intro}), based on the A-type preverbs but with \forme{-a} vocalism instead of \forme{-ɯ} and \forme{-ɤ}  (originating from fusion with another prefix, §\ref{sec:xtokavian.preverbs}). For instance, the \textsc{2pl}\fl{}3 Aorist of \japhug{ndza}{eat} is \forme{tɤ-tɯ-ndza-nɯ} (\textsc{aor}:\textsc{up}-2-eat-\textsc{pl} `you$_{pl}$ ate it') with the \textsc{upwards} A-type \forme{tɤ-} preverb, but the corresponding \textsc{3pl}\flobv{} form is \forme{\textbf{ta}-ndza-nɯ} (\textbf{\textsc{aor}:3\flobv{}:\textsc{up}}-eat-\textsc{pl} `they ate it') with the C-type preverb \forme{ta-}. It is the only case when an orientation preverb encodes person in addition to TAME.

The Inferential (§\ref{sec:ifr}) has a series of preverbs (series D) based on series B, but with \forme{-o} and \forme{-ɤ} vocalism instead of \forme{-u} and \forme{-ɯ}, respectively (§\ref{sec:xtokavian.preverbs}). For instance, the  \textsc{upwards} and \textsc{downwards} D-type preverbs are \forme{to-} and \forme{pjɤ\trt}, corresponding to the B-type \forme{tu-} and \forme{pjɯ\trt}, respectively (§\ref{sec:kamnyu.preverbs}).

Five TAME categories neutralize the orientation contrast, and require the same marker for all verbs: the Sensory evidential \forme{ɲɯ-} (§\ref{sec:sensory}), from the B-type \textsc{westwards} preverb (\tabref{tab:orientation.preverbs.intro}), the Egophoric Present \forme{ku-} (§\ref{sec:egophoric}), the Dubitative \forme{ku-} (§\ref{sec:dubitative}), from the B-type \textsc{eastwards} preverb, the Past Imperfective \forme{pɯ-} from the A-type \textsc{downwards} preverb (§\ref{sec:pst.ifr.ipfv.morphology}, \citealt{lin11direction}), and the Inferential Imperfective \forme{pjɤ-} from the D-type \textsc{downwards} preverb (§\ref{sec:pst.ifr.ipfv.morphology}).
 
In addition to orientation preverbs, TAME categories are marked by several morphological exponents, including stem alternations (§\ref{sec:stem.alternation}), allomorphy of negative prefixes (§\ref{sec:neg.allomorphs}) and additional affixes: the Irrealis \forme{a-} prefix (§\ref{sec:outer.prefixal.chain}) and the Past transitive \forme{-t} suffix (§\ref{sec:suffixes}).
 
In the Non-Past, Imperfective and Modal categories (all except Past), transitive alternating verbs have a specific stem in direct configurations with a singular subject (see \tabref{tab:transitive.paradigm.singular.alternating} above and §\ref{sec:stem3}). Another stem is found in the Aorist of a handful of verbs (§\ref{sec:stem2}).
 
Some of the primary categories can be combined with the copula \japhug{ŋu}{be} to form periphrastic TAME categories (§\ref{sec:ipfv.periphrastic.TAME}), for instance the Periphrastic Past Imperfective (§\ref{sec:pst.ifr.ipfv.periphrastic}) illustrated in (\ref{ex:tundzea.pWNu2}), built from the Imperfective (\forme{tu-ndze-a}, with the B-type \textsc{upwards} preverb \forme{tu\trt}, the alternating stem \forme{ndze} from \japhug{ndza}{eat} and the \textsc{1sg} suffix) and the Past Imperfective of the copula \forme{pɯ-ŋu}.
 
 \begin{exe}
\ex \label{ex:tundzea.pWNu2}
\gll  tɤ-mtʰɯm tu-ndze-a pɯ-ŋu   \\
\textsc{indef}.\textsc{poss}-meat \textsc{ipfv}-eat[III]-\textsc{1sg} \textsc{pst}.\textsc{ipfv}-be \\
\glt `I was eating meat/I used to eat meat.' 
  \end{exe} 

In addition, primary TAME categories can be combined with prefixes expressing secondary aspectual (§\ref{sec:second.aspect}) or modal (§\ref{sec:second.modal}) meanings.

There is a robust contrast between Past and Non-Past tenses in Japhug, but no grammaticalized future tense. Future events in main clauses are mainly expressed by the Factual Non-Past (§\ref{sec:fact.main.clauses}) or the Irrealis (§\ref{sec:irrealis.main}).
 
The tripartite evidential system between Egophoric Present, Sensory (or Testimonial)  and Factual observed in the non-past is structurally very similar to that found in some Tibetic languages (\citealt{tournadre08conjunct, hill17evidential}).


\subsubsection{Stative vs. dynamic verbs}
\is{verb!stative} \is{verb!dynamic}
TAME morphology presents a contrast between \textit{stative} and \textit{dynamic} verbs. In the  Imperfective (§\ref{sec:ipfv.inchoative}), the Aorist (§\ref{sec:aor.inchoative}) and the Inferential (§\ref{sec:ifr.inchoative}), stative verbs have an inchoative meaning, different from their meaning in Non-Past tenses.

For example, the verb \japhug{zri}{be long} means `become long(er)' in the Imperfective (\forme{tu-zri} \textsc{ipfv}:\textsc{up}-be.long `it becomes longer') or the Aorist (\forme{tɤ-zri} \textsc{aor}:\textsc{up}-be.long `(when) it became longer'). By contrast, in the Sensory (\forme{ɲɯ-zri} \textsc{sens}-be.long `it is long') and the other Non-Past tenses, it retains its basic stative meaning.

Stative verbs also differ from most dynamic verbs in being compatible with Past Imperfective and Inferential Imperfective in all contexts (\forme{pɯ-zri} \textsc{pst}.\textsc{ipfv}-be.long `it was/used to be long'), while dynamic verbs generally require the periphrastic Past Imperfective instead (§\ref{sec:pst.ifr.ipfv.periphrastic}), except in some limited contexts (§\ref{sec:pst.ifr.ipfv.apodosis}).\footnote{This criterion is however not absolute, since a few atelic dynamic verbs can occurs in the non-periphrastic Past Imperfective (§\ref{sec:pst.ifr.ipfv.morphology}). }

The stative/dynamic contrast is not completely independent from transitivity (§\ref{sec:transitivity.morphology}). Only \textit{intransitive} stative verbs have a distinctive morphological marking: the \forme{kɯ-} infinitive  appears in some contexts (§\ref{sec:infinitives.participles}). Transitive stative verbs include in particular verbs derived from adjectival stative verbs by the tropative derivation (§\ref{sec:tropative.pst.ipfv}).

 
 

\subsection{Non-finite verb forms}
\is{verb!finite}
Finiteness can be defined in Japhug by the ability of a given verb form to occur in the person indexation paradigms (§\ref{sec:indexation.intro}). Non-finite verbs forms cannot take indexation affixes, though some of them can mark the person and number of at most \textit{one} argument by means of possessive prefixes like nouns (§\ref{sec:possessive.paradigm.intro}). The distinction between finite and non-finite verbal forms is categorical: there are no intermediate semi-finite forms, unlike in Situ \citep{jacksonlin07} where some participles take person indexation in specific contexts.

The main verb of a complete sentence has to be in a finite form, and non-finite verbs are restricted to subordinate clauses (including relative and complement clauses). 

\subsubsection{Participles} \label{sec:participles.intro}
\is{participle}
There are three types of participle in Japhug: subject, object and oblique, respectively marked by the prefixes \forme{kɯ-} (\ref{ex:nAkWqur}), \forme{kɤ-} (\ref{ex:nAkAqur}) and \forme{sɤ/z-} (\ref{ex:WsAthu}). They are fully productive, and only a handful of defective verbs lack participles (§\ref{sec:intr.person.irregular}, §\ref{sec:irregular.transitive}).

\begin{exe}
\ex 
\begin{xlist}
\ex \label{ex:nAkWqur}
\gll nɤ-kɯ-qur \\
\textsc{2sg}.\textsc{poss}-\textsc{sbj}:\textsc{pcp}-help \\
\glt `The one/someone who helps you' 
\ex \label{ex:nAkAqur}
\gll nɤ-kɤ-qur \\
\textsc{2sg}.\textsc{poss}-\textsc{obj}:\textsc{pcp}-help \\
\glt `The one/someone that you help' 
\ex \label{ex:WsAthu}
\gll a-sɤ-tʰu \\
\textsc{1sg}.\textsc{poss}-\textsc{obl}:\textsc{pcp}-ask \\
\glt `The one whom I ask' 
\end{xlist}
\end{exe}

Participles can take a possessive prefix (§\ref{sec:subject.participle.possessive}, §\ref{sec:object.participle.possessive}, §\ref{sec:oblique.participle.possessive}), marking either the subject (\ref{ex:nAkAqur} and \ref{ex:WsAthu}) or the object (\ref{ex:nAkWqur}).  

Participles can in addition be combined with negative, orientation and associated motion prefixes (§\ref{sec:subject.participle.other.prefixes}, §\ref{sec:object.participle.other.prefixes}), as shown by (\ref{ex:WmApjWkWnWfkAB2}).

\begin{exe}
\ex \label{ex:WmApjWkWnWfkAB2}
\gll ɯ-mɤ-pjɯ-kɯ-nɯ-fkaβ \\
\textsc{3sg}.\textsc{poss}-\textsc{neg}-\textsc{ipfv}:\textsc{down}-\textsc{sbj}:\textsc{pcp}-\textsc{auto}-cover \\
\glt `The one/those who do(es) not cover it' (from example \ref{ex:WmApjWkWnWfkAB}, §\ref{sec:subject.participle.other.prefixes}) 
\japhdoi{0003604\#S20}
\end{exe}

One of the main function of participles is to build participial relative clauses (§\ref{sec:participial.relatives}). Participial clauses can relativize core arguments (§\ref{sec:subject.participle.subject.relative}, §\ref{sec:object.participle.relatives}) and various oblique arguments and adjuncts (§\ref{sec:subject.participle.other.relative}, §\ref{sec:object.participle.relatives}, §\ref{sec:locative.participle.relatives}, §\ref{sec:instrumental.participle.relatives}, §\ref{sec:other.oblique.participle.relatives}).

In particular, the Japhug equivalent of attributive adjectives are adjectival stative verbs in subject participle form (§\ref{ex:attributive.participles.stative.verbs}, §\ref{sec:intr.subject.relativization}), occurring in generally head-internal (§\ref{sec:head-internal.relative.postnominal}) relative clauses as in (\ref{ex:tCheme.kWmpCAr}).

\begin{exe}
\ex \label{ex:tCheme.kWmpCAr}
\gll tɕʰeme kɯ-mpɕɤr \\
girl \textsc{sbj}:\textsc{pcp}-be.beautiful \\
\glt `A/the beautiful girl' 
\japhdoi{0006312\#S89}
\end{exe}

The subjects of both intransitive (§\ref{sec:intr.subject.relativization}) and transitive (§\ref{sec:tr.subject.relativization}) verbs are relativized by means of a participial relative clause in \forme{kɯ-}. Subject participles of transitive verbs take an obligatory possessive prefix coreferent with the object (unless another prefix is present) as in (\ref{ex:nAkWqur}) and (\ref{ex:WkWrAt.part}), while those of intransitive verbs lack possessive prefixes as in (\ref{ex:kWrArAt.part}),\footnote{The verb \forme{rɤ-rɤt} in (\ref{ex:kWrArAt.part}) is the antipassive derivation (§\ref{sec:antipassive.rA}) from \japhug{rɤt}{write, draw} (\ref{ex:WkWrAt.part}). } except in very restricted cases (§\ref{sec:subject.participle.possessive}).

\begin{exe}
\ex 
\begin{xlist}
\ex \label{ex:WkWrAt.part}
\gll ɯ-kɯ-rɤt \\
\textsc{3sg}.\textsc{poss}-\textsc{sbj}:\textsc{pcp}-write \\
\glt `The one/someone who writes it' 
\ex \label{ex:kWrArAt.part}
\gll kɯ-rɤ-rɤt \\
\textsc{sbj}:\textsc{pcp}-\textsc{apass}-write \\
\glt `The one/someone who writes things, a writer.' 
\end{xlist}
\end{exe}

Another important function of participles is to build the purposive complements of motion verbs (§\ref{sec:subject.participle.complementation}, §\ref{sec:purposive.clause.motion.verbs}, §\ref{sec:am.vs.mvc}), as in  (\ref{ex:akWtoR.jAGe}) (see also \ref{ex:WkWXtW.jaria} in §\ref{sec:am.intro} below).


\begin{exe}
\ex \label{ex:akWtoR.jAGe}
\gll a-kɯ-rtoʁ jɤ-ɣe \\
\textsc{1sg}.\textsc{poss}-\textsc{sbj}:\textsc{pcp}-look \textsc{aor}-come[II] \\
\glt `S/he came to see me.' 
\end{exe}

Participles have several additional morphosyntactic functions, presented in §\ref{sec:participial.clause.complementation strategies}.

\subsubsection{Infinitives and converbs} \label{sec:inf.intro}
\is{infinitive}  \is{converb}
The infinitives in \forme{kɯ-} and \forme{kɤ\trt}, called ``velar infinitives'' in this grammar, serve as the citation forms of verbs (§\ref{sec:inf.citation}). They occur in some complement clauses as in (\ref{ex:kAtaR.rgaa}) (§\ref{sec:inf.complementation}, §\ref{sec:velar.infinitives.complement.clauses})  and also serve as converbs (§\ref{sec:inf.converb}, §\ref{sec:manner.converbs}). They are easily confused with subject or object participles (§\ref{sec:infinitives.participles}).

\begin{exe}
\ex \label{ex:kAtaR.rgaa}
\gll kɤ-taʁ rga-a \\
\textsc{inf}-weave like:\textsc{fact}-\textsc{1sg} \\
\glt `I like to weave.' 
\end{exe}

In addition to velar infinitives, two other types of infinitives are attested: the bare (§\ref{sec:bare.inf}) and the  dental (§\ref{sec:dental.inf}) infinitives. These forms are only used in the complement clauses of a handful of verbs such as \japhug{ʑa}{begin} (§\ref{sec:bare.dental.inf}, §\ref{sec:aspectual.complement}, §\ref{sec:bare.dental.inf.sWpa}). Bare and dental infinitives are in complementary distribution (§\ref{sec:transitivity.morphology}): the former is found with transitive verbs (\ref{ex:Wndza.toZa2}), and the latter with intransitive ones (\ref{ex:tWrJaR.pjAZa}).

\begin{exe}
\ex 
\begin{xlist}
\ex \label{ex:Wndza.toZa2}
\gll  ɯ-ndza to-ʑa \\
\textsc{3sg}.\textsc{poss}-\textsc{bare}.\textsc{inf}:eat \textsc{ifr}:\textsc{up}-start \\
\glt `S/he/it started eating it.' 
\ex \label{ex:tWrJaR.pjAZa}
\gll tɯ-rɟaʁ pjɤ-ʑa \\
\textsc{inf}:II-dance \textsc{ifr}:\textsc{down}-start \\
\glt `S/he started dancing.' 
\end{xlist}
\end{exe}

In this construction (and a few other ones), the com\-ple\-ment-taking verb takes the orientation that is lexically selected by the verb in the complement clause (§\ref{sec:orientation.raising}), for instance \textsc{upwards} like \japhug{ndza}{eat} in (\ref{ex:Wndza.toZa2}) and \textsc{downwards} like \japhug{rɟaʁ}{dance} in (\ref{ex:tWrJaR.pjAZa}).
\is{gerund}
In addition to the infinitives, three converbs are attested: the reduplicated gerund \forme{sɤ-} (§\ref{sec:gerund}) expressing temporal simultaneity (§\ref{sec:simultaneity}), the purposive converb (§\ref{sec:purposive.converb}, §\ref{sec:purposive.clauses}) and the converb of immediate subsequence (§\ref{sec:immediate.converb}, §\ref{sec:immediate.subsequence}), which stands out in having a perfective meaning `as soon as ...' (for instance  \forme{pjɯ-tɯ-mto} `as soon as $X$ saw $Y$') while selecting the B-type preverbs which usually mark Imperfective and Non-Past tenses. Despite the existence of these converbs, temporal (§\ref{sec:temporal.clauses}), manner (§\ref{sec:manner.clauses}) and causal (§\ref{sec:causality}) subordinate clauses mainly select finite verb forms in Japhug narratives and conversations.

\subsubsection{Other nominalizations}
Several productive abstract nominalizations are found in Japhug. The degree nominals (§\ref{sec:degree.nominals}), combining a \forme{tɯ-} with a possessive prefix coreferent with the subject (for example \forme{a-tɯ-mtsɯr} \textsc{1sg}-\textsc{nmlz}:\textsc{deg}-be.hungry `my degree of hunger'), are highly common and occur in degree and equative constructions (§\ref{sec:degree.nominal.subject}).

Action nominals (§\ref{sec:tW.action.nominal}) and abstract nouns (§\ref{sec:tA.abstract.nouns}) can be formed by prefixation of \forme{tɯ-} and \forme{tɤ\trt}, for instance the noun \forme{tɯ-rɟaʁ} `dance' from the intransitive verb \japhug{rɟaʁ}{dance}, and \forme{tɤ-mtsɯr} `hunger' from \japhug{mtsɯr}{be hungry}. These derivations are not rare, but not fully productive either. The \forme{tɯ-} and \forme{tɤ-} prefixes here are not analyzable as indefinite possessor prefixes with which they are homophonous (§\ref{sec:possessive.paradigm}), as they cannot be replaced by definite possessor prefixes.

\subsection{Associated motion} \label{sec:am.intro}
\is{associated motion}
Japhug and other Gyalrong languages stand out in Trans-Himalayan by having a system of associated motion clearly different from orientation markers (\citealt{jacques20am-st}). Unlike Arandic \citep{koch84associated.motion} or Tacanan \citep{guillaume09mouv.assoc}, the category of associated motion in Japhug only comprises two different prefixes (§\ref{sec:am.prefixes}) marking either cislocative or translocative motion of the subject (§\ref{sec:AM.argument.motion}) prior to the action expressed by the verb root.

Although associated motion prefixes (\ref{ex:CtAXtWta}) seems at first glance semantically similar to motion verbs with a purposive clause (\ref{ex:WkWXtW.jaria}), there are systematic differences between these two constructions (§\ref{sec:am.vs.mvc}), both in terms of presuppositions (§\ref{sec:am.concessive}) and in syntactic constraints of relativization (§\ref{sec:AM.mvc.relativizability}).

\begin{exe} 
\ex 
\begin{xlist}
\ex \label{ex:CtAXtWta}
\gll ɕ-tɤ-χtɯ-t-a   \\
\textsc{tral}-\textsc{aor}-buy-\textsc{pst}:\textsc{tr}-\textsc{1sg} \\
\glt `I went and bought it.' 
\ex \label{ex:WkWXtW.jaria}
\gll ɯ-kɯ-χtɯ jɤ-ari-a \\
\textsc{3sg}.\textsc{poss}-\textsc{sbj}:\textsc{pcp}-buy \textsc{aor}-go[II]-\textsc{1sg} \\
\glt `I went to buy it.' 
\end{xlist}
\end{exe} 

\subsection{Voice} \label{sec:voice.intro}

\subsubsection{Overview}
The rich and redundant morphological expression of transitivity (§\ref{sec:transitivity.morphology}) and the rarity of labile verbs (§\ref{sec:lability}) in Japhug are correlated with a highly productive system of voice derivations, treated in chapters \ref{chap:valency.increasing.derivation}, \ref{chap:valency.decreasing.derivation} and \ref{chap:other.derivations}. There are eleven fully productive valency-changing prefixes,\footnote{These derivations are not fully productive in the sense that they can be applied to any verb (since there are transitivity and semantic restrictions on their uses), but in the sense that some recent loanwords can be subjected to them. } summarized in \tabref{tab:voice.overview}, to which a certain number of non-productive derivations such as the applicative \forme{nɯ-} (§\ref{sec:applicative}) and the anticausative (§\ref{sec:anticausative}) can be added. 

\begin{table}
\caption{Productive valency-changing verbal derivations in Japhug} \label{tab:voice.overview}
\begin{tabular}{lllll}
\lsptoprule
Voice& Prefix & Section  \\
\midrule 
Sigmatic causative  & \forme{sɯ(ɣ)-/z-} & §\ref{sec:sig.causative}  \\
Velar causative  & \forme{ɣɤ-} & §\ref{sec:velar.causative}  \\
Tropative &\forme{nɤ(ɣ)-} & §\ref{sec:tropative}    \\
\midrule
Passive &\forme{a-} & §\ref{sec:passive}   \\
Reciprocal &\forme{a-}+reduplication& §\ref{sec:redp.reciprocal}  \\
Reflexive & \forme{ʑɣɤ-} & §\ref{sec:reflexive}  \\
Antipassive &\forme{rɤ-} & §\ref{sec:antipassive.rA}   \\
 &\forme{sɤ-} & §\ref{sec:antipassive.sA}   \\
Proprietive &\forme{sɤ-} & §\ref{sec:passive}  \\
Facilitative & \forme{ɣɤ-} & §\ref{sec:facilitative.GA}  \\
& \forme{nɯɣɯ-} & §\ref{sec:facilitative.nWGW} \\
\lspbottomrule
\end{tabular}
\end{table}

All productive voice derivations are marked by prefixes (§\ref{sec:inner.prefixal.chain}). Only fossil traces of derivational suffixes are found (in particular the applicative \forme{-t}, §\ref{sec:applicative.t}). The anticausative is marked not by a prefix, but by an alternation whereby unvoiced obstruents are converted into their voiced prenasalized counterparts (see §\ref{sec:anticausative.morphology}).

There are, in addition, productive verbal derivations that do not change valency, such as the autive \forme{nɯ-} (§\ref{sec:autobenefactive}) and the distributed action derivation  (§\ref{sec:distributed.action}).

The following sections present a representative sample of voice derivations and their main morphosyntactic functions.

\subsubsection{Causative}
\is{causative}
There are two productive causative derivations, the sigmatic causative (which has four productive allomorphs \forme{sɯ\trt}, \forme{sɯɣ\trt}, \forme{s-} and \forme{z-} depending on the phonological and morphological context, §\ref{sec:sig.caus.allomorphs}) and the velar causative \forme{ɣɤ-} (from \forme{wɤ\trt}, a form still found in some Japhug dialects).

The latter is restricted to a subset of stative verbs. Some stative verbs are compatible with both causative derivations: for instance \japhug{zbaʁ}{be dry} can be causativized as both \forme{ɣɤ-zbaʁ} or \forme{sɯ-zbaʁ}. The semantic contrast between the two causatives in this context remains unclear (§\ref{sec:velar.causative.vs.sigmatic.causative}).

The sigmatic causative is the most productive derivation in Japhug. It is compatible with intransitive, transitive and even ditransitive verbs (§\ref{sec:ditransitive.causative}), and has a wide range of meanings (§\ref{sec:sig.caus.function}), from coercion (§\ref{sec:sig.caus.coercitive}) as in (\ref{ex:kW.tuznAme}) to indirect causation (§\ref{sec:sig.caus.indirect}).

\begin{exe}
\ex  \label{ex:kW.tuznAme}
\gll rɟɤlpu kɯ  ɯ-ma nɯ  \textbf{mkʰɤrmaŋ} \textbf{ra} tu-\textbf{z}-nɤme pjɤ-ŋu  \\
king \textsc{erg} \textsc{3sg}.\textsc{poss}-work \textsc{dem} people \textsc{pl}   \textsc{ipfv}-\textsc{caus}-do[III] \textsc{ifr}.\textsc{ipfv}-be \\
\glt `The king used to make the people do work for him.' 
\japhdoi{0006248\#S5}
\end{exe}

The sigmatic causative is also used to mark instruments (§\ref{sec:sig.caus.instrumental}), for example in (\ref{ex:kW.chAsWtsxWB}), where the instrument \japhug{taqaβ}{needle} receives  ergative marking (§\ref{sec:instr.kW}) like a causee (§\ref{sec:causee.kW}): the construction literally means `s/he made the needle sew the clothes'.


\begin{exe}
\ex  \label{ex:kW.chAsWtsxWB}
\gll  \textbf{ki} \textbf{taqaβ} \textbf{ki} \textbf{kɯ} tɯ-ŋga cʰɤ-\textbf{sɯ}-tʂɯβ  \\
 \textsc{dem}.\textsc{prox} needle \textsc{dem}.\textsc{prox} \textsc{erg} \textsc{indef}.\textsc{poss}-clothes \textsc{ifr}-\textsc{caus}-sew \\
\glt `S/he sewed the clothes with this needle.' 
\end{exe}

In addition to the productive causatives, there are irregular causative forms (§\ref{sec:sig.caus.irregular}, §\ref{sec:causative.m}), some of which co-exist with their regular counterparts, but with a more lexicalized meaning. For instance, the verb \japhug{tsʰi}{drink} has the irregular causative \japhug{jtsʰi}{give to drink} (§\ref{sec:caus.j}) as opposed to the regular one \forme{sɯ-tsʰi} `make drink, drink with'.

The sigmatic causative prefix can be combined with nearly all other derivational prefixes (§\ref{sec:sig.caus.other.derivations}). It can precede the velar causative, as in \forme{z-ɣɤ-mpja} `heat up $X$ with $Y$, make/let $Y$ heat up $X$' from \forme{ɣɤ-mpja} `heat up', causative of \japhug{mpja}{be warm}. 

It is also the only prefix that can occur more than once in a single verb form,\footnote{A double reciprocal form is attested (§\ref{sec:sAmW}), but not with the same reciprocal prefixes. } as shown by examples such as \forme{sɯ-sɯ-spoʁ} `make a hole with' from \forme{sɯ-spoʁ} `make a hole', causative of the intransitive verb \japhug{spoʁ}{have a hole} (§\ref{sec:sig.caus.other.recursion}). 


\subsubsection{Tropative}
\is{tropative}
The tropative \forme{nɤ-} prefix (§\ref{sec:tropative}), like a causative derivation, turns an intransitive verb into a transitive, but its meaning differs: the added argument is not a causer, but an experiencer feeling/perceiving the state expressed by the base verb. For instance, the tropative of \japhug{mpɕɤr}{be beautiful} is \forme{nɤ-mpɕɤr} `find beautiful', with the experiencer encoded as subject and the stimulus (corresponding to the intransitive subject of the base verb) as object, as shown by (\ref{ex:YWtanAmpCAr}).

\begin{exe}
\ex  \label{ex:YWtanAmpCAr}
\gll ɲɯ-ta-nɤ-mpɕɤr \\
\textsc{sens}-1\fl{}2-\textsc{trop}-be.beautiful \\
\glt `I find you very beautiful.' 
\end{exe}

The tropative can be used to define \textit{adjectives} as a sub-class of stative verbs: only adjectival stative verbs can undergo this derivation, unlike for example existential verbs and copulas (§\ref{sec:copula.existential}).

\subsubsection{Antipassive}
\is{antipassive}
When the object of a morphologically transitive verb (thus excluding labile verbs in intransitive conjugation, §\ref{sec:lability}) is non-overt, it is necessarily interpreted as definite (§\ref{sec:nonovert.core.arguments}). For instance, example (\ref{ex:kW.chAtsxWB}) can only be used if the referent that has been sewn has been previously mentioned or is retrievable from the context, and cannot be understood as `sewed something' with an indefinite object. To express this meaning, several strategies are possible, including the antipassive \forme{rɤ-} derivation (§\ref{sec:antipassive.rA}). 

\begin{exe}
\ex 
\begin{xlist}
\ex \label{ex:kW.chAtsxWB}
\gll tɕʰeme nɯ kɯ cʰɤ-tʂɯβ \\
girl \textsc{dem} \textsc{erg}  \textsc{ifr}-sew \\
\glt `The girl sewed it.' 
\ex \label{ex:chArAtsxWB}
\gll  tɕʰeme nɯ cʰɤ-rɤ-tʂɯβ \\
 girl \textsc{dem}  \textsc{ifr}-\textsc{apass}-sew \\
 \glt `The girl sewed / did sewing.' 
 \end{xlist}
  \end{exe}

The \forme{rɤ-} turns a transitive verb into an intransitive one, whose only argument is semantically the agent, but does not take ergative marking  (\ref{ex:chArAtsxWB}).

\subsubsection{Reflexive and reciprocal}
\is{reflexive} 
Japhug has a dedicated reflexive prefix \forme{ʑɣɤ-} (§\ref{sec:reflexive}), different from other valency-decreasing derivations. The reflexive verb is conjugated intransitively, as shown by (\ref{ex:thWZGArkua})\footnote{If the verb in (\ref{ex:thWZGArkua}) were transitive, a past transitive \forme{-t} suffix would be inserted (§\ref  {sec:suffixes}, §\ref{sec:indexation.mixed}) and the expected form would be $\dagger$\forme{tʰɯ-ʑɣɤ-rku-t-a}. }
 
\begin{exe}
\ex \label{ex:thWZGArkua}
\gll  tʰɯ-ʑɣɤ-rku-a \\
\textsc{aor}:\textsc{downstream}-\textsc{refl}-put.in-\textsc{1sg} \\
\glt `I put myself (in the bag).'
\end{exe}
  
 The reflexive is frequently combined with the sigmatic causative (§\ref{sec:refl.caus}) to express an unintentional indirect causation affecting oneself (\ref{ex:tAZGAsWmpCaa}).
 
\begin{exe}
\ex \label{ex:tAZGAsWmpCaa}
\gll tɤ-ʑɣɤ-sɯ-mpɕa-a \\
\textsc{aor}-\textsc{refl}-\textsc{caus}-scold-\textsc{1sg} \\
\glt `I got myself scolded.' 
\end{exe}
 
 \is{reciprocal}
The reciprocal derivation (§\ref{sec:redp.reciprocal}), entirely different from the reflexive, is built by prefixing \forme{a-} and reduplicating the verb stem (§\ref{sec:partial.redp}). For instance the transitive verb \japhug{rqoʁ}{hug} yields \forme{a-rqɯ\redp{}rqoʁ} `hug each other'. Reciprocal verbs generally require a non-singular intransitive subject, and can also select a comitative postpositional phrase (§\ref{sec:comitative}).

\subsubsection{Autive}
\is{autive}
The autive \forme{nɯ-} (§\ref{sec:autobenefactive}) is a highly productive derivation, which 
does not affect verbal transitivity unlike the previous ones (§\ref{sec:autoben.transitivity}). 

\is{autobenefactive}
Its most basic function is self-affectedness or autobenefactive (§\ref{sec:autoben.proper}). In particular, the autive on transitive verbs taking a inalienably possessed object can be used to specify that the subject and the possessor of the object are coreferent (\ref{ex:WsroR.konWri}), whereas the absence of the autive is generally interpreted as indicating the absence of coreference (\ref{ex:WsroR.kori}). This is not an absolute syntactic rule, however, since the Autive has additional unrelated uses (see below and §\ref{sec:autoben.spontaneous}, §\ref{sec:autoben.permansive}) which can interfere with this particular function.

\begin{exe}
\ex 
\begin{xlist}
\ex \label{ex:WsroR.konWri}
\gll ɯʑo kɯ ɯ-sroʁ ko-nɯ-ri  \\
\textsc{3sg} \textsc{erg} \textsc{3sg}.\textsc{poss}-life \textsc{ifr}-\textsc{auto}-save  \\
\glt `S/he$_i$ saved his/her$_i$ own life.' 
\ex \label{ex:WsroR.kori}
\gll ɯʑo kɯ ɯ-sroʁ ko-ri  \\
\textsc{3sg} \textsc{erg} \textsc{3sg}.\textsc{poss}-life \textsc{ifr}-save  \\
\glt `S/he$_i$ saved his/her$_j$ life.'
\end{xlist}
\end{exe}

\is{volitionality} \is{spontaneous}
Another function of the Autive is to indicate spontaneous or non-volitional actions (§\ref{sec:autoben.spontaneous}), occurring for example by mistake, as illustrated by the minimal pair between (\ref{ex:pWnWprata}) and (\ref{ex:pWprata}).


\begin{exe}
\ex 
\begin{xlist}
\ex \label{ex:pWnWprata}
\gll tɤ-rɣe pɯ-nɯ-prat-a  \\
\textsc{indef}.\textsc{poss}-necklace \textsc{aor}-\textsc{auto}-break-\textsc{1sg}  \\
\glt `I broke the pearl necklace (by mistake).'
\ex \label{ex:pWprata}
\gll tɤ-rɣe pɯ-prat-a  \\
\textsc{indef}.\textsc{poss}-necklace \textsc{aor}-break-\textsc{1sg}  \\
\glt `I broke the pearl necklace (on purpose).' 
\end{xlist} 
\end{exe}

The third main function of the Autive is to express permansive aspect (§\ref{sec:autoben.permansive}).

\subsection{Denominal derivations}
\is{denominal derivation} 
Verbalizing denominal (chapter \ref{chap:denominal}) and deideophonic (§\ref{sec:voice.deideophonic}) derivations are rich and productive in Japhug. 

A considerable number of denominal prefixes can be identified. Some of them have a well-identifiable meaning, for instance the proprietive \forme{aɣɯ-} (§\ref{sec:denom.aGW}) deriving verbs meaning `having a lot of $X$' or `producing a lot of $X$' (such as \japhug{aɣɯlu}{producing a lot of milk} (of a cow) from \japhug{tɤ-lu}{milk}) or the similative \forme{arɯ-} (§\ref{sec:denom.arW}). For the prefixes \forme{rɯ/ɤ-} (§\ref{sec:denom.rA}), \forme{nɯ/ɤ-} (§\ref{sec:denom.nW}) and \forme{ɣɯ/ɤ-}  (§\ref{sec:denom.GW}), several different functions have to be postulated, since these prefixes can derive both intransitive (§\ref{sec:denom.intr.rA}, §\ref{sec:denom.intr.nW}, §\ref{sec:denom.intr.GA}) and transitive (§\ref{sec:denom.tr.rA}, §\ref{sec:denom.tr.nW}, §\ref{sec:denom.tr.GA}) verbs.

Some denominal derivations occur in pairs (§\ref{sec:denom.nW.pairing}, §\ref{sec:denom.rA.pairing}). For instance, when a noun has both \forme{rɯ/ɤ-} and \forme{nɯ/ɤ-} denominal verbs, the former is usually dynamic intransitive, and the latter transitive, as illustrated by the pair comprising the intransitive verb \forme{rɤ-ma} `do (some) work' and its transitive counterpart \forme{nɤ-ma} `do (a work)', both from the noun \japhug{ta-ma}{work}.

Denominal derivations compete with light verb constructions to turn nouns into predicates (§\ref{sec:denominal.vs.light.verb}, §\ref{sec:light.verb}). For instance, the meaning `tell lies, cheat' from the noun \japhug{kʰramba}{lie} can be expressed either by a collocation with the light verb \japhug{βzu}{make} (§\ref{sec:Bzu.lv}) or by the denominal verbs \japhug{rɯkʰramba}{tell lies} (§\ref{sec:denom.intr.rA}, intransitive) and \japhug{nɯkʰramba}{cheat} (§\ref{sec:denom.tr.nW}, transitive).

\is{reanalysis} 
An important proportion of voice prefixes originate from the reanalysis of denominal derivations from bare nominalized forms (§\ref{sec:voice.denominal}), most clearly in the case of the \forme{rɤ-} antipassive (§\ref{sec:antipassive.history}, \citealt{jacques14antipassive}).

When applied to noun-verb compounds (§\ref{sec.n.v.compounds}, §\ref{sec:action.nominal.compounds}), denominal derivations can serve to build incorporating verbs (§\ref{sec:incorp.denom}). For instance, the intransitive verb \japhug{ɣɯsɯpʰɯt}{cut firewood} is derived by the prefix \forme{ɣɯ-} (§\ref{sec:denom.intr.GA}) from the compound \japhug{sɯpʰɯt}{cutting firewood}, itself made from the bound state \forme{sɯ-} of the noun \japhug{si}{wood} compounded with the transitive verb \japhug{pʰɯt}{take out, cut}, `cut'. A tripartite contrast exists between the basic transitive construction (\forme{si+pʰɯt}), a light verb construction combining \japhug{βzu}{make} with the corresponding action nominal compound (\forme{sɯpʰɯt+βzu}) and the denominal incorporating verb \forme{ɣɯ-sɯpʰɯt}, all three meaning `cut firewood' (§\ref{sec:incorp.vs.other}).


 \section{Core and oblique arguments } \label{sec:alignment.introduction}
The morphosyntactic properties of arguments can be studied from the point of view of flagging (§\ref{chap:postpositions.relators}, §\ref{ex:postpositions}, §\ref{sec:relator.nouns}) and indexation (§\ref{sec:intr.indexation}, §\ref{sec:polypersonal}), but also relativization (§\ref{sec:function.relativization}) and coreference restrictions between complement and matrix clauses (§\ref{sec:complement.types}).
 
\is{argument!core}
Core arguments are defined as those that are indexed by the verb morphology. In the case of intransitive verbs (§\ref{sec:intr.indexation}), the only argument indexed on the verb is the \textit{intransitive subject}, but the argument structure of intransitive verbs can contain up to two additional oblique arguments (§\ref{sec:semi.transitive.dative}). Morphologically transitive verbs index two arguments (including in some cases a dummy one, §\ref{sec:transitive.dummy}). Although there are no different morphological slots for transitive subjects and objects (§\ref{sec:indexation.tr.intro}, §\ref{sec:direct-inverse}), the transitive paradigm contains no ambiguity in person configurations (§\ref{sec:polypersonal}), and the person of the agentive and patientive core arguments can always be clearly identified in finite verb forms. The core argument indexed like the agentive argument of verbs of action such as \japhug{sat}{kill} or \japhug{ʁndɯ}{hit} (as in §\ref{sec:indexation.tr.intro}) is called \textit{transitive subject}, and the other one is the \textit{object}.\footnote{The term \textit{object} in this grammar is restricted to this particular core argument, to the exclusion of all object-like patientive arguments. }

Outside of verb indexation, the three basic core arguments (intransitive subject, transitive subject and object) are encoded in various ways, and present different types of alignments.
 
 \subsection{Neutral alignment}
 \is{alignment!neutral}
Person indexation affixes in general have neutral alignment: for instance the \textsc{1sg} suffix \forme{-a}  indexes the subject of intransitive verbs, and is also found in \textsc{1sg}\fl{}3, 3\fl{}\textsc{1sg} and 2\fl{}\textsc{1sg} configurations (§\ref{sec:indexation.tr.intro}, §\ref{sec:direct-inverse}).

The absence of strict coreference restrictions between matrix and subordinate clauses in some categories of complement clauses (§\ref{sec:velar.inf.coreference}, §\ref{sec:rYo.complements}, §\ref{sec:finite.complement.coref}) and in manner (§\ref{sec:manner.converbs}) and temporal clauses (§\ref{sec:temporal.clauses}) could be interpreted as a type of neutral alignment, but it appears that the neutralization in those cases is not limited to the three core arguments: there can also be complete absence of coreference, or coreference with an oblique or a possessor of an argument.

\subsection{Nominative-accusative alignment}
 \is{alignment!nominative-accusative}
 
 \subsection{Subjecthood}
  \is{subject!intransitive}  \is{subject!transitive}
Nominative-accusative alignment appears in several unrelated constructions in Japhug. 

 \is{participle!subject}
The first piece of evidence for this type of alignment in Gyalrong languages to have been proposed \citep{jackson03caodeng} is the fact that the \forme{kɯ-} (subject) participle (§\ref{sec:subject.participle.subject.relative}) is the only form that can be used to relativize both intransitive (§\ref{sec:intr.subject.relativization}) and transitive subjects (§\ref{sec:tr.subject.relativization}). However, the  \forme{kɯ-} participles are not exclusively used to relativize intransitive and transitive subjects: they can also relativize possessors of intransitive subjects (§\ref{sec:possessor.relativization}), and are the only option to do so, resulting in ambiguities (see examples \ref{ex:WlaXtCha.pWkWtu} and \ref{ex:jla.nWRrW.kWtu} in §\ref{sec:possessive.mihi.est}).

 \is{lability}
The fact that nearly all labile verbs are subject-preserving (§\ref{sec:lability.apass})  could also be adduced as evidence of nominative-accusative alignment, but the existence of a handful of  object-preserving labile verbs (§\ref{sec:lability.pass}) makes it less compelling.
 
 Clearer cases of constructions where strict nominative-accusative alignment is observed include associated motion and complementation.
 
First, the argument performing the motion encoded by the associated prefixes (§\ref{sec:am.prefixes}) is always the intransitive subject in the case of intransitive verbs, and the transitive subject in the case of transitive ones (§\ref{sec:AM.argument.motion}), and can never be the object, a possessor of the subject or any oblique argument (but it can be the causee of a causative verb).
 
 Second, several subtypes of complement clauses require coreference between the (intransitive or transitive) subjects of the complement clause and that of the main clause, in particular bare and dental infinitives (§\ref{sec:bare.inf.coreference}).
 
 \subsection{Objecthood}
  \is{object}
 While evidence for \textit{subjects} independent from person indexation can be identified, evidence for \textit{objects} (as opposed to other non-subject arguments in absolutive form, §\ref{sec:semi.object}, §\ref{sec:theme.ditransitive}, §\ref{absolutive.goal}) is more elusive. 
 
In subject participles of transitive verbs, the object is marked by a possessive prefix  (§\ref{sec:subject.participle.possessive}), obligatorily if no other prefix (of orientation, negation or associated motion) is present, as in (\ref{ex:WkWndza2}).  

\begin{exe} 
\ex 
\begin{xlist}
\ex \label{ex:Ca.YWndze}
\gll spjaŋkɯ nɯ kɯ ɕa ɲɯ-ndze \\
wolf \textsc{dem} \textsc{erg} meat \textsc{sens}-eat[III] \\
\glt `The wolf eats meat.' 
\ex \label{ex:WkWndza2}
\gll ɯ-kɯ-ndza  \\
\textsc{3sg}.\textsc{poss}-\textsc{sbj}:\textsc{pcp}-eat \\
\glt `The one/someone who eats it' 
\end{xlist}
\end{exe}
 
However, possessive prefixes on subject participles cannot be used as evidence for objects outside of finite indexation, as they can mark other grammatical functions. For example the verb \japhug{rga}{like} is conjugated intransitively as shown by (\ref{ex:WkWrga}): its subject \forme{spjaŋkɯ} has no ergative marking, and no stem alternation is observed on the verb (a form like $\dagger$\forme{ɲɯ-rge} would be expected if this verb were transitive as in \ref{ex:Ca.YWndze}, §\ref{sec:stem3.form}, §\ref{sec:transitivity.morphology}). However, in addition to its subject, this verb takes an absolutive argument (\japhug{ɕa}{meat} in \ref{ex:Ca.YWrga}) that cannot be indexed: the \textit{semi-object}. This argument is also marked as a possessive prefix on the subject participle like a real object, as in (\ref{ex:WkWrga}).
 
\begin{exe} 
\ex 
\begin{xlist}
\ex \label{ex:Ca.YWrga}
\gll spjaŋkɯ nɯ ɕa ɲɯ-rga \\
wolf \textsc{dem} meat \textsc{sens}-like \\
\glt `The wolf likes meat.' 
\ex \label{ex:WkWrga}
\gll ɯ-kɯ-rga  \\
\textsc{3sg}.\textsc{poss}-\textsc{sbj}:\textsc{pcp}-like \\
\glt `The one/someone who likes him/her/it.' 
\end{xlist}
\end{exe}
 
Moreover, objects do not have any relativization construction that is specific to them. They can be relativized by object participial relatives (§\ref{sec:object.participle.relatives}) and finite relatives (§\ref{sec:finite.relatives}), but these categories of clauses can also be used to relativize semi-objects (§\ref{sec:semi.tr.relativization}, §\ref{sec:object.participle.other.relative}) and other participants  such as goals (§\ref{sec:locative.relativization.object}, §\ref{sec:locative.relativization.finite}) and possessors of objects (§\ref{sec:possessor.relativization}). For instance, headless participial clauses relativizing the object of \japhug{ndza}{eat} (\ref{ex:akAndza}) or the semi-object of \japhug{rga}{like} (\ref{ex:akArga}) have the same structure.
 
\begin{exe} 
\ex 
\begin{xlist}
\ex \label{ex:akAndza}
\gll a-kɤ-ndza \\
\textsc{1sg}.\textsc{poss}-\textsc{obj}:\textsc{pcp}-eat \\
\glt `[The things] that I eat.' 
\ex \label{ex:akArga}
\gll a-kɤ-rga  \\
\textsc{1sg}.\textsc{poss}-\textsc{sbj}:\textsc{pcp}-like \\
\glt `[The things] that I like.' 
\end{xlist}
\end{exe}

\subsection{Absolutive-ergative alignment} \label{sec:ergativity}
\is{alignment!ergative-absolutive}

Although the terms ``absolutive'' (§\ref{sec:absolutive}) and ``ergative'' (§\ref{sec:erg.kW}) are used in this grammar to refer to flagging, Japhug does not display perfect absolutive-ergative alignment in case marking.

 \is{absolutive}
Absolutive form (absence of postposition or relator noun) does indeed serve to mark intransitive subjects (§\ref{sec:absolutive.S}) and objects (§\ref{sec:absolutive.P}) by default, and transitive subjects normally take the ergative \forme{kɯ} postposition (§\ref{sec:A.kW}). However, both absolutive (§\ref{sec:semi.object}, §\ref{sec:essive.abs}, §\ref{absolutive.goal}) and ergative forms (§\ref{sec:instr.kW}, §\ref{sec:comparee.kW}) have many functions other than marking core arguments. In addition, there is some fluidity in the use of the ergative postposition: some intransitive subjects can also be marked by it (§\ref{sec:S.kW}), in particular due to anticipation of  a transitive verb in the following clauses (§\ref{sec:long.distance.kW}).


The clearest evidence of absolutive-ergative alignment in Japhug is found in generic person indexation (§\ref{sec:indexation.generic.tr}): generic intransitive subjects and objects are both indexed by \forme{kɯ\trt}, while generic transitive subjects are marked by the inverse \forme{wɣ-} (§\ref{sec:direct-inverse}).
\is{alignment!ergative-absolutive}

In addition, in local configurations (§\ref{sec:indexation.local}), the fact that the suffix closest to the verb stem indexes the object (like an intransitive subject) can also be analyzed as ergative alignment in this sub-part of the indexation system.

There are no cases of syntactic pivot with exclusive neutralization of intransitive subject and object in complementation or relativization. 

\subsection{Ditransitive verbs}
Most ditransitive verbs have indirective alignment (§\ref{sec:ditransitive.indirective}): the theme is indexed (treated as the object of a monotransitive verb), and the recipient marked with the genitive (§\ref{sec:genitive}) or the dative (§\ref{sec:dative}). There are however a few highly common verbs such as \japhug{mbi}{give} which have secundative alignment (§\ref{sec:ditransitive.secundative}) and index the recipient.

Causativized transitive verbs have a different indexation pattern: either the causee or the patientive argument is indexed as object, depending on factors such as person hierarchy (§\ref{sec:ditransitive.causative}).

\section{Word order} \label{sec:word.order.introduction}
\is{word order}
Japhug has a strict verb-final order, with only a limited number of exceptions: right dislocated constituents (§\ref{sec:right.dislocation}), only a few adverbs, particles (§\ref{sec:sfp}) and ideophones (§\ref{sec:idph.syntax}) can occur postverbally (§\ref{sec:postverbal.adv}).

When both the subject and the object of a monotransitive verb are overt as in (\ref{ex:lWlu.nW.kW.tondza}), the former is placed by default before the latter (§\ref{sec:monotransitive.word.order}). For ditransitive verbs, the order between theme and recipient is not rigid (§\ref{sec:secundative.word.order}, §\ref{sec:indirective.word.order}).

\begin{exe}
\ex \label{ex:lWlu.nW.kW.tondza}
\gll lɯlu nɯ kɯ pɣa nɯ to-ndza \\
cat \textsc{dem} \textsc{erg} bird \textsc{dem} \textsc{ifr}-eat \\
\glt `The cat ate the bird.' 
\end{exe}

Noun phrases have by default (Demonstrative)-Noun-Adjective-Numeral-De\-mon\-strative order (§\ref{sec:noun.phrases.word.order}) as illustrated by (\ref{ex:tCheme.kWmpCAr.XsWm}).

\begin{exe}
\ex \label{ex:tCheme.kWmpCAr.XsWm}
\gll (kɯki) tɕʰeme kɯ-mpɕɤr χsɯm kɯra \\
\textsc{dem}.\textsc{prox} girl \textsc{sbj}:\textsc{pcp}-be.beautiful three \textsc{dem}:\textsc{pl} \\
\glt `These three beautiful girls'
\end{exe}


\section{Subordination} \label{sec:subordination.introduction}

\subsection{Relative clauses}
\is{relative clause}
Japhug lacks relative pronouns, and most relative clauses are participial (§\ref{sec:participial.relatives}). Finite relative clauses are also attested (§\ref{sec:finite.relatives}).

Most arguments and adjuncts are relativizable, though there are constraints on relativizability (§\ref{sec:accessibility.relativization}).

In the text corpus, an important proportion of relative clauses are headless (§\ref{sec:headless.relative}). When the head noun is overt, it is either internal to the relative clause (§\ref{sec:head-internal.relative}) or follows it (§\ref{sec:prenominal.relative}), depending in part on the syntactic function of the relativized element and on the type of relative clause.

Example (\ref{ex:lWlu.nW.kW.tAkAndza}) illustrates a head-internal finite relative: the relativized object \japhug{pɣa}{bird} is located between the transitive subject \forme{lɯlu nɯ kɯ} and the finite verb (an object participle \forme{tɤ-kɤ-ndza} (\textsc{aor}-\textsc{obj}:\textsc{pcp}-eat) would also be possible to express the same meaning), at the position it would normally occupy in the corresponding independent sentence (see \ref{ex:lWlu.nW.kW.tondza} above).

\begin{exe}
\ex \label{ex:lWlu.nW.kW.tAkAndza}
\gll [lɯlu nɯ kɯ \textbf{pɣa} ta-ndza] nɯ  \\
cat \textsc{dem} \textsc{erg} \textbf{bird} \textsc{aor}:3\flobv{}-eat \textsc{dem}   \\
\glt `The \textbf{bird} that the cat ate' 
\end{exe}

In (\ref{ex:pGA.WkWndza}) on the other hand, the relative clause precedes the relativized transitive subject \japhug{lɯlu}{cat}.

\begin{exe}
\ex \label{ex:pGA.WkWndza}
\gll [pɣa ɯ-tɤ-kɯ-ndza] \textbf{lɯlu} nɯ   \\
bird \textsc{3sg}.\textsc{poss}-\textsc{aor}-\textsc{sbj}:\textsc{pcp}-eat \textbf{cat} \textsc{dem}  \\
\glt `The \textbf{cat} that ate the bird' 
\end{exe}

\subsection{Complement clauses}
\is{complement clause}
Complement clauses in Japhug (chapter \ref{chap:complement.clauses}) occur in intransitive subject (\ref{ex:kABzu.mApe}, §\ref{sec:adjective.complement}), object (\ref{ex:kABzu.mAspe}, §\ref{sec:spa.verb}) and semi-object (\ref{ex:kABzu.mAnAz}, §\ref{sec:nAz.verb}) functions. There are also com\-ple\-ment-taking nouns (\ref{ex:kABzu.WRjiz}, §\ref{sec:complement.taking.nouns}).

\begin{exe} 
\ex \label{ex:kABzu.complements}
\begin{xlist}
\ex \label{ex:kABzu.mApe}
\gll [kʰramba kɤ-βzu] mɤ-pe \\
lie \textsc{inf}-make \textsc{neg}-be.good:\textsc{fact} \\
\glt `Telling lies is bad.' 
\ex \label{ex:kABzu.mAspe}
\gll ɯʑo kɯ [kʰramba kɤ-βzu] mɤ-spe \\
\textsc{3sg} \textsc{erg} lie \textsc{inf}-make \textsc{neg}-be.able[III] \\
\glt `S/he is not able to tell lies.' 
\ex \label{ex:kABzu.mAnAz}
\gll [ɯʑo kɯ kʰramba kɤ-βzu] mɤ-nɤz \\
\textsc{3sg} \textsc{erg} lie \textsc{inf}-make \textsc{neg}-dare \\
\glt `S/he does not dare to tell lies.' 
\ex \label{ex:kABzu.WRjiz}
\gll [kʰramba kɤ-βzu] a-ʁjiz mɤ-ɣi \\
lie \textsc{inf}-make \textsc{3sg}.\textsc{poss}-wish \textsc{neg}-come:\textsc{fact} \\
\glt `I do not want to tell lies.' 
\end{xlist}
\end{exe} 

As shown by (\ref{ex:kABzu.mAnAz}), when the com\-ple\-ment-taking verb is semi-transitive (and thus morphologically intransitive, §\ref{sec:semi.transitive}), the verb in the complement clause is transitive and they share their subjects, the subject can take ergative case following the verb in the complement clause rather than absolutive (§\ref{sec:case.infinitive}).

Not all complements are infinitive clauses as in (\ref{ex:kABzu.complements}) (§\ref{sec:inf.intro}, §\ref{sec:velar.infinitives.complement.clauses}). Some verbs are also compatible with finite complements (§\ref{sec:finite.complement}) as in (\ref{ex:pjWtasWXCAt.YWspea}), with subject coreference.

\begin{exe} 
\ex \label{ex:pjWtasWXCAt.YWspea}
\gll [pjɯ-ta-sɯxɕɤt] ɲɯ-spe-a  \\
\textsc{ipfv}-1\fl{}2-teach \textsc{sens}-be.able[III]-\textsc{1sg} \\
\glt `I am able to teach you.' 
\end{exe} 

Verbs of speech also take reported speech clauses (§\ref{sec:reported.speech}), which present morphosyntactic properties different from  regular finite complements (§\ref{sec:hybrid indirect}).

Aside from complement clauses proper, various complementation strategies are also found, including relative clauses in core argument function (which resemble, and can even be ambiguous with, complement clauses, §\ref{sec:relative.core.arg}), participial clauses (§\ref{sec:participial.clause.complementation strategies}), action nominals (§\ref{sec:complementation.strategy.action.nominals}) and simple coordination (§\ref{sec:coordination.comp.str}).

\subsection{Other subordinate clauses}
Subordinate clauses other than relative and complement clauses (treated in chapter \ref{chap:temporal.conditional}) tend to be expressed by finite clauses headed by a relator noun or a postposition. For instance, clauses of temporal precedence (§\ref{sec:precedence.CWNgW}) require a finite verb in the Imperfective regardless of the TAME category of the verb in the main clause (§\ref{sec:ipfv.temporal}) followed the postposition \japhug{ɕɯŋgɯ}{before} (§\ref{sec:temporal.postpositions}) as in (\ref{ex:chWstanW.CWNgW}).


\begin{exe} 
\ex \label{ex:chWstanW.CWNgW}
\gll [cʰɯ-sta-nɯ ɕɯŋgɯ] tɕe, ɯʑo cʰɤ-rɤru  \\
\textsc{ipfv}-wake.up-\textsc{pl} before \textsc{lnk} \textsc{3sg} \textsc{ifr}-get.up \\
\glt `S/he got up before they had woken up.' 
\japhdoi{0006065\#S35}
\end{exe} 

Converbial clauses do exist (§\ref{sec:converbs}), but are all in competition with a finite clause type. For instance, the immediate converb (§\ref{sec:immediate.converb}) illustrated in (\ref{ex:tutWlhoR}) has a corresponding construction (\ref{ex:tAlhoR.CimWma2}) with a finite verb in the Aorist (§\ref{sec:immediate.subsequence}).

\begin{exe} 
\ex 
\begin{xlist}
\ex \label{ex:tutWlhoR}
\gll tu-tɯ-ɬoʁ \\
\textsc{ipfv}:\textsc{up}-\textsc{imm}:\textsc{conv}-come.out \\
\ex \label{ex:tAlhoR.CimWma2}
\gll tɤ-ɬoʁ ɕimɯma \\
\textsc{aor}:\textsc{up}-come.out immediately.after  \\
\glt `As soon as it comes out / immediately after it has come out'
\end{xlist}
\end{exe} 

Some finite clauses not followed by postpositions or relator nouns still have clues of a subordinate status. In particular, periphrastic TAME constructions involving an Imperfective verb (§\ref{sec:imperfective}) and a copula (§\ref{sec:ipfv.periphrastic.TAME}), when they occur in chains, tend to elide the auxiliary in the non-final clauses (§\ref{sec:periphrastic.subordination}), resulting in constructions like  (\ref{ex:turAma.kunWrNgW}), where the first clause \forme{ɕɤr tɕe tu-rɤma} `it is active during the night' is incomplete (lacking the auxiliary \forme{ɲɯ-ŋu}), and the copula \forme{ɲɯ-ŋu} has scope over the two clauses preceding it (see also \ref{ex:chain.pjANu}, §\ref{sec:ipfv.periphrastic.TAME}, for a example of the same type with more than ten clauses sharing a single auxiliary).

\begin{exe} 
\ex \label{ex:turAma.kunWrNgW}
\gll lɯlu nɯ, [ɕɤr tɕe tu-rɤma], sŋi tɕe ku-nɯ-rŋgɯ ɲɯ-ŋu \\
cat \textsc{dem} night \textsc{loc} \textsc{ipfv}-work day \textsc{loc} \textsc{ipfv}-\textsc{auto}-lie.down \textsc{sens}-be \\
\glt `The cat, active during the during the night, sleeps during the day.' 
\japhdoi{0003576\#S33}
\end{exe} 

We also find a special type of serial verb construction expressing manner (§\ref{sec:svc.manner}), in which the verb in the manner clause (with a similative verb §\ref{sec:svc.similative.verb} or a deideophonic verb §\ref{sec:svc.deideophonic}) shares the same TAME, subject (and often object) as the verb in the other clause.  In (\ref{ex:ki.tAstuta.nWXtCita}) for example,  \japhug{stu}{do like} and the \japhug{χtɕi}{wash} are in Aorist \textsc{1sg}\fl{}\textsc{3sg} form (§\ref{sec:aor.morphology}) and share \forme{a-ŋga} `my clothes' as object (the demonstrative \forme{ki} is semi-object, §\ref{sec:ditransitive.secundative}).
 

\begin{exe} 
\ex \label{ex:ki.tAstuta.nWXtCita}
\gll a-ŋga nɯ ki tɤ-stu-t-a tɕe nɯ-χtɕi-t-a \\
\textsc{1sg}.\textsc{poss}-clothes \textsc{dem} \textsc{dem}.\textsc{prox} \textsc{aor}-do.like-\textsc{pst}:\textsc{tr}-\textsc{1sg} \textsc{lnk}   \textsc{aor}-wash-\textsc{pst}:\textsc{tr}-\textsc{1sg} \\
\glt `I washed my clothes like this.' 
\end{exe} 

\section{Remarkable features} \label{sec:remarkable.features}
After this overview of the core features of Japhug grammar, this section presents a selection of topics of particular interest to linguistic typology and comparative grammar in Japhug.

\subsection{Consonant clusters}
\is{consonant!cluster}
The Kamnyu dialect of Japhug has over four hundred consonant clusters in onset position (§\ref{sec:inventory.clusters}, \citealt{jacques19ipa}), whose internal structure can be studied through alternations in reduplication patterns (§\ref{sec:partial.redp}, \citealt{jacques07redupl}).
 
 In addition to the size of the inventory, which in itself is significant, because Japhug is one of the languages with the greatest number of clusters in the Trans-Himalayan family. Only Khroskyabs boasts more clusters, \citep[101]{lai17khroskyabs}. The system of clusters presents two main points of interest.
 
 First,  syllable onsets in Japhug are rich in clusters violating the  \textit{sonority sequencing principle} (\textsc{ssp}, §\ref{sec:ssp}, \citealt[210]{blevins95syllable}), and in particular present cases of \textsc{ssp}-infringing clusters without corresponding \textsc{ssp}-compliant equivalents. For instance, the \textsc{ssp}-infringing clusters  \ipa{rm-} and \ipa{rt-} are relatively common, while $\dagger$\ipa{mr-} and $\dagger$\ipa{tr-} are unattested (except across syllables, §\ref{sec:heterosyllabic.clusters}) due to a series of sound changes  (§\ref{sec:NC.clusters}, §\ref{sec:Cr.clusters}).
 
 Second, some of these clusters are of considerable antiquity, as they appear to be preservations from proto-Trans-Himalayan (\citealt{jacques15sr,zhangsy19cognates}), and thus of considerable importance for the reconstruction of syllable structure in this family in general (\citealt[212]{hill2019phonology}) and in Old Chinese in particular \citep{gong17clusters}.
 
\subsection{Direct-inverse}
\is{direct-inverse} \is{inverse}
Direct-inverse systems, while relatively well-attested in languages of the Americas \citep{zuniga06}, are very rare in the Old World. Japhug and the other Gyalrong languages (\citealt{delancey81direction, jackson02rentongdengdi, jacques10inverse, gongxun14agreement}) are in fact the only languages in Eurasia to have a near-canonical direct-inverse indexation system (§\ref{sec:direct-inverse}, \citealt{jacques14inverse}), in particular with a direct-inverse contrast in both mixed (§\ref{sec:indexation.mixed}) and non-local domains (§\ref{sec:inverse.3.3.saliency})
.\footnote{Inverse-like phenomena are observed in Japanese \citep{koga2008}, Circassian \citep{arkadiev17inverse} and some Trans-Himalayan languages, especially Kuki-Chin and Northern Naga \citep{konnerth19agreement}, but the inverse morphemes in these languages are in the process of being grammaticalized, and still retain a cislocative meaning, whereas the inverse \forme{wɣ-} in Japhug is dedicated to the expression of person indexation.  Kiranti languages also have direct-inverse systems (in particular Bantawa and Puma, see \citealt{doornenbal09} and \citealt{bickel07puma}), but the distribution of inverse and direct markers in these languages is much less transparent than in Gyalrong \citep{jacques14inverse}.   }

Moreover, Japhug is unique among Gyalrong languages in lacking inverse marking in the local 2\fl{}1 configuration (§\ref{sec:indexation.local}, §\ref{sec:portmanteau.prefixes.history}, \citealt{jacques18generic}),  and possibly the only known language to use the inverse marker as the sole marker of \textit{generic transitive subject}  (§\ref{sec:indexation.generic.tr}). 
  
 
\subsection{Inflectionalization} \label{sec:inflectionalization.intro}
\is{inflectionalization}
While person indexation is one of the defining properties of verbs in Japhug (§\ref{sec:indexation.intro}), a handful of expressions of nominal origin have acquired the ability to take dual \forme{-ndʑi} and plural \forme{-nɯ} indexation suffixes by analogy with imperative verb forms (§\ref{sec:non.finite.indexation}). 

For instance, the phatic expression \japhug{sɤrma}{good night} (§\ref{sec:phatic.inflectionalization}) has the dual \forme{sɤrma-ndʑi} `good night (to both of you)' when addressing two people. This unusual phenomenon has parallels in Indo-European (\citealt[113--114]{viti15wandel}).

 %\japhug{ɯ-tʰoʁ}{ground}  {sec:earth.IPN}
 %postposition \forme{zɯ}  {sec:core.locative} {ex:word.vs.clitic.postp}  
\subsection{Prefixal chain}
\is{prefix!prefixing preference}
Japhug, like other Gyalrong languages, has a large prefixal template (§\ref{sec:prefixal.chain}), allowing more than seven or eight prefixes in a row (example \ref{ex:amAGWnWtWwGznAre}, §\ref{sec:verb.overview.intro}), while the suffixal chain is much more restricted (§\ref{sec:suffixes}). It is among the rare strongly prefixal languages with strict verb-final order, alongside Ket \citep{werner97ketisch} and Athabaskan \citep{rice2000scope}. 

While the verb-final word order is relatively ancient,\footnote{Reconstructing word-order is a notoriously difficult task, but there is no positive reason to assume any other order at least for the common ancestor of Tibeto-Gyalrongic languages \citep{Sagart19ST}. }
 there is evidence that the part of the prefixal chains in Gyalrongic languages have been recently innovated, either through grammaticalization and integration of verb roots from serial-verb constructions \citep{jacques13harmonization}, or the absorption of adverbs or clause-final particles occurring immediately before the verb into the verbal word (§\ref{sec:khWti}, §\ref{sec:apprehensive.history}; see also \citealt{laiyf20betrayal} on Khroskyabs).

Japhug also has a few bipartite verbs, made from two verb stems with identical TAME and person indexation cliticized to each other and with elision of either the suffixal chain of the first verb  as in (\ref{ex:atAtWstunatAtWmbatW}), the prefixal chain of the second verb, or both (§\ref{sec:bipartite}, \citealt{jacques18bipartite}).
 
\begin{exe}
\ex \label{ex:atAtWstunatAtWmbatW}
\gll \rouge{a-tɤ-tɯ}-stu=\rouge{a-tɤ-tɯ}-mbat-\bleu{nɯ}  \\
\textsc{irr}-\textsc{pfv}-\textsc{2}-try.hard(1)=\textsc{irr}-\textsc{pfv}-\textsc{2}-try.hard(2)-\textsc{pl} \\
\glt `Do your best/try hard.'  
\end{exe} 

  
\subsection{Hybrid indirect speech} \label{sec:hybrid.intro}
\is{indirect speech!hybrid}
Reported speech in Japhug frequently presents mismatches in person indexation, with the main verb representing the point of view of the subject of the com\-ple\-ment-taking verb, and the nouns and possessive prefixes that of the current speaker.
 
For instance, in (\ref{ex:CWrtoRi.hybrid}), the verb \forme{rtoʁ} has \textsc{1pl} indexation, reflecting the point of view of the transitive subject of the com\-ple\-ment-taking verb \forme{tɯ-nɯ-sɯso-nɯ} `you think, you want' (the daughters-in-law, the addressee). On the other hand, the \textsc{2pl} possessive prefixes on \forme{nɯ-pʰama} `your parents' and the other nouns correspond to the point of view of the father-in-law (the current speaker) -- a \textsc{1pl} possessor would be expected in the reported speech clause if no shift of point of reference had taken place, as in  (\ref{ex:CWrtoRi.hybrid2}).

Three different translations of the Japhug sentence are proposed in (\ref{ex:CWrtoRi.hybrid}), the first in direct speech, the second in indirect speech (with shift toward the current speaker) and the third as a merger of the two (agrammatical in English), directly reflecting the original \textit{hybrid indirect speech} (§\ref{sec:hybrid indirect}).

\begin{exe}
\ex 
\begin{xlist}
\ex \label{ex:CWrtoRi.hybrid}
\gll  a-me ra nɯʑora kɯnɤ, (...) [\rouge{nɯ-kʰa}, \rouge{nɯ-mu} \rouge{nɯ-pʰama} ra \bleu{ɕɯ-rtoʁ-i]} tɯ-nɯ-sɯso-nɯ ɕti tɕe jɤ-nɯ-ɕe-nɯ   \\
\textsc{1sg}.\textsc{poss}-daughter \textsc{pl} \textsc{2pl} also {  } \textsc{2pl}.\textsc{poss}-house \textsc{2pl}.\textsc{poss}-mother  \textsc{2pl}.\textsc{poss}-parent \textsc{pl} \textsc{tral}-look:\textsc{fact}-\textsc{1sg} 2-\textsc{auto}-think:\textsc{fact}-\textsc{pl} be.\textsc{aff}:\textsc{fact} \textsc{lnk} \textsc{imp}-\textsc{vert}-go-\textsc{pl} \\ 
\glt  \textbf{Direct}: `My daughters in law$_i$, you$_i$ think ``\bleu{Let us$_i$ go and see our$_i$ house, our$_i$ parents},'' so go home.' 
\glt  \textbf{Indirect}:   `My daughters in law$_i$, \rouge{you$_i$ want to go (home) and see your$_i$ house, your$_i$ parents}, so go home.'
\glt  \textbf{Hybrid indirect}: `My daughters in law$_i$, \rouge{you$_i$ think}, `\bleu{Let us$_i$ go and see} \rouge{your$_i$ house, your$_i$ parents}, so go home.' (2005 tAwakWcqraR)
\ex \label{ex:CWrtoRi.hybrid2}
\gll ji-kʰa, ji-mu ji-pʰama ra ɕɯ-rtoʁ-i (ra)\\
\textsc{1pl}.\textsc{poss}-house \textsc{1pl}.\textsc{poss}-mother \textsc{1pl}.\textsc{poss}-parents \textsc{pl} \textsc{tral}-look:\textsc{fact}-\textsc{1pl} be.needed:\textsc{fact} \\
\glt `(Let us) go and see our houses, our mothers, our parents.' 
\end{xlist}
\end{exe}

Hybrid indirect speech is also found in  Tibetic languages (\citealt{tournadre08conjunct}, \citealt[327]{tournadre21tibetic}), where it has contributed to confusion around the notion of ``conjunct/disjunct'' marking. The presence of person indexation makes this phenomenon more easily identifiable in Japhug than in Tibetan.
 
\subsection{The expression of degree and comparison}
Japhug lacks comparative and superlative derivations, but has a rich array of constructions  expressing degree and comparison (chapter \ref{chap:degree}), some of which are rather uncommon at least in this part of the world.

\is{superlative}
High degree can be marked by combining a finite adjectival stative verb with a degree adverb (§\ref{sec:degree.adverbs}) as in (\ref{ex:wuma.YWCqraR}) as in most languages. However, the most common construction conveying this meaning, illustrated in (\ref{ex:WtWCqraR.YWsaXaR}), comprises a degree nominal (§\ref{sec:degree.nominal.subject}, §\ref{sec:degree.nominals}) serving as the intransitive subject of a verb of degree `be extremely'. Thus the neutral way in Japhug to say `$X$ is very intelligent' is literally `$X$'s degree of intelligence (of being intelligent) is extreme.'

\begin{exe} 
\ex 
\begin{xlist}
\ex \label{ex:wuma.YWCqraR}
\gll wuma ʑo ɲɯ-ɕqraʁ \\
really \textsc{emph} \textsc{sens}-be.intelligent \\
\glt `S/he is very intelligent.' \japhdoi{0006380\#S72}
\ex \label{ex:WtWCqraR.YWsaXaR}
\gll ɯ-tɯ-ɕqraʁ ɲɯ-saχaʁ \\
\textsc{3sg}.\textsc{poss}-\textsc{nmlz}:\textsc{deg}-be.intelligent \textsc{sens}-be.extremely \\
\glt `S/he is extremely intelligent.'  \japhdoi{0006380\#S13}
\end{xlist}
\end{exe} 

The main comparative construction in Japhug is unusual for a different reason. In many languages including Tibetic (\citealt[239]{vbrugmo03maqu}, \citealt[29]{heine-kuteva02}), the ergative or the  ablative are used to mark the standard of comparison. In Japhug, as illustrated in (\ref{ex:sAz.mpCAr}), the standard has a dedicated postposition (§\ref{sec:comparative}), whereas the ergative marker \forme{kɯ} is used to mark the \textit{comparee} instead (§\ref{sec:comparee.kW}, §\ref{sec:sAz.kW}). 


\begin{exe}
\ex \label{ex:sAz.mpCAr}
\gll  [ɯ-ʁi sɤz] [ɯ-pi nɯ kɯ] mpɕɤr  \\
\textsc{3sg}.\textsc{poss}-younger.sibling \textsc{comp} \textsc{3sg}.\textsc{poss}-elder.sibling \textsc{dem} \textsc{erg} be.beautiful:\textsc{fact} \\
\glt `The elder sibling is more beautiful than the younger sibling.' 
\end{exe}

This observation is all the more surprising given that \forme{kɯ} is most likely borrowed from Tibetan \citep{jacques16comparative}.

\subsection{Japhug morphology and Trans-Himalayan comparative linguistics} \label{sec:comparative.morphology.intro}
\is{morphology!Trans-Himalayan comparison}
The rich verbal and nominal morphology of Japhug and other Gyalrongic languages comprises both archaisms  and innovative features illustrating interesting grammaticalization pathways.

Among innovations, many voice prefixes (§\ref{sec:inner.prefixal.chain}) have been created through reanalysis of denominal derivation (§\ref{sec:voice.denominal}). The clearest case of an innovating voice marker is the antipassive \forme{rɤ-} (§\ref{sec:antipassive.rA}): irregular forms provide direct evidence  that it originated from the intransitive denominal \forme{rɤ-}  (§\ref{sec:denom.intr.rA}) applied to deverbal nouns (§\ref{sec:antipassive.irr.form},  \citealt{jacques14antipassive}). Other representative innovations include the reflexive \forme{ʑɣɤ-} from an incorporated pronoun (§\ref{sec:reflexive.origin}, \citealt{jacques10refl}), the associated motion prefixes grammaticalized from motion verbs (§\ref{sec:am.prefixes}, \citealt{jacques13harmonization}), the orientation preverbs from locational adverbs or nouns (§\ref{sec:preverbs.adverbs}) and the comitative adverbs (§\ref{sec:comitative.adverb}, \citealt{jacques17comitative}).
\is{comitative adverb}
While Japhug did innovate many affixes, it also preserves morphological archaisms which go further back than proto-Gyalrongic, some potentially even up to proto-Trans-Himalayan.

\is{indexation!antiquity}
The antiquity of person indexation in Gyalrongic (§\ref{sec:indexation.suffixes.history}) and the rest of Trans-Himalayan is a notoriously controversial topic (\citealt{bauman75, delancey89agreement, lapolla92, driem93agreement}) but in any case a paradigm comprising a second person prefix with suffixes for first person and number of third and second person should at least be reconstructed back to the common ancestor of Gyalrongic, Kiranti and probably Jinghpo (\citealt{jacques12agreement, delancey14second, jacques16th}).

In a few cases, archaic morphology only remains as lexicalized traces in Japhug, in particular the applicative \forme{-t} suffix, very prominent in Kiranti (\citealt{michailovsky85dental, jacques15derivational.khaling}) for instance, which is only attested in two verbs (§\ref{sec:applicative.t}), or the nominalization \forme{-z} (§\ref{sec:z.nmlz}) suffix, which has cognates in Tibetan and Chinese (\citealt{jacques03s.houzhui, jacques16ssuffixes}).

In other cases,  derivational processes that only exist as traces in most of the family are still productive in Japhug and other core Gyalrong languages, in particular the sigmatic denominal and causative prefixes (§\ref{sec:sigmatic.denominal}, §\ref{sec:sig.causative}, \citealt{sagart12sprefix, jacques15causative}),  velar (§\ref{sec:velar.nmlz.history}) and sigmatic (§\ref{sec:sigmatic.nmlz.history}) nominalization prefixes (\citealt{jacques14snom, konnerth16gV, jacques19fossil}), and the dental indefinite possessor prefix of inalienably possessed nouns (§\ref{sec:indef.genr.poss}, §\ref{sec:inalienably.possessed}).

Gyalrong data is particularly relevant to the debate regarding the voicing alternation in Old Chinese and other Trans-Himalayan languages
\citep{handel12valence}. Old Chinese and many other languages have pairs of verbs with a voicing contrast correlated with transitivity, in which the unvoiced verb is transitive and the voiced one intransitive. It is not obvious which one is the derived form, and the direction of derivation is still being debated (\citealt{sagart12sprefix, mei12caus}). 

Japhug and other Northern Gyalrong languages  (\citealt[411--412]{jacques04these}, \citealt[271]{gong18these}) however provide a crucial piece of evidence showing that the directionality was from the transitive verb to the intransitive one (§\ref{sec:anticausative.direction}), confirming evidence from unrelated sources \citep{sagart03prenasalized}: the Tibetan borrowing \japhug{χtɤr}{scatter} (from \tibet{གཏོར་}{gtor}{scatter}) has an intransitive counterpart \japhug{ʁndɤr}{be scattered} (§\ref{sec:anticausative.morphology}) with prenasalized onset without equivalent in Tibetan. This intransitivizing derivation can be described as anticausative (§\ref{sec:anticausative.function}). There is further evidence that the prenasalization alternation comes from a nasal prefix, very probably a lexicalization of the   autive \forme{nɯ-} prefix (§\ref{sec:autoben.historical}), whose spontaneous function (§\ref{sec:autoben.spontaneous}) is very close to that of anticausative verbs. Evidence for autive derivation only exists in Gyalrongic (\citealt[357--368]{lai17khroskyabs}, \citealt{gong18these}). However, the presence of traces of the anticausative derivation as voicing alternations in various branches of Trans-Himalayan, for instance in Kiranti \citep{jacques15derivational.khaling}, Old Chinese \citep{sagart12sprefix} or Tibetan \citep{jacques12internal}, implies that the autive derivation by extension must also be of proto-Trans-Himalayan age, and thus represents a unique archaism of Gyalrongic.
 
The richness and high productivity of morphology in Japhug and other Gyalrong languages, comparable with that of Sanskrit in Indo-European or Meskwaki in Algonquian, offer a framework to explore the fossil morphology of other Trans-Himalayan languages, in particular those belonging to the Burmo-Gyalron\-gic and Tibeto-Gyalrongic branches \citep{jacques.michaud11naish, Sagart19ST}, but also potentially for the family as a whole. 
 
 \is{morphology!irregular}
Commenting on the irregularity of correspondences between Tibetan, Old Chinese and Burmese, \citet[212]{hill2019phonology} concludes that ``the phonetic influence of defunct morphology will one day explain these complicated correspondences, but this possibility will manifest only when more languages, particularly archaic languages such as those of the Rgyalrong and Kiranti branches, are brought within purview.''  One of the aims of this grammar is precisely to provide comparativists with sufficient data on this language to make it  systematically usable in Trans-Himalayan etymological research, in the hope that this field can one day reach the degree of sophistication of Indo-European \citep{fellner19allofam}.
 
 