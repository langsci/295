\chapter{Person indexation and argument structure} \label{chap:indexation}

\section{Introduction}  
In Japhug, person indexation is the defining feature of finite verbs, as opposed to non-finite verbs (§\ref{chap:non-finite}) and other parts of speech. Japhug finite verb forms index one or two arguments, depending on the transitivity of the verb, using a combination of prefixes, suffixes and stem alternation. No verb indexes more than two arguments. The indexation system is very close to a canonical direct-inverse system (§\ref{sec:direct-inverse}).

This chapter first presents intransitive and transitive conjugations, investigates the issue of agreement mismatch, and then discusses the origin of person indexation affixes. In addition, it documents the analogical extension of person indexation suffixes to non-finite verb forms in some specific contexts.

\section{Intransitive verbs} \label{sec:intr.indexation}
Intransitive verbs comprise dynamic, stative and semi-transitive verbs. All of the verbs have in common the property of indexing one argument, the intransitive subject, which when overt is in absolutive form (§\ref{sec:absolutive.S}).

\subsection{The intransitive paradigm} \label{sec:intransitive.paradigm}
\tabref{tab:intransitive.indexation} illustrates the paradigm of intransitive verbs in Kamnyu Japhug, using the verb \japhug{ɕe}{go} in the Factual non-past\footnote{This TAM category is chosen to illustrate the paradigms due to the fact that it does not bear any orientation preverb, but at the same time presents stem alternation in the transitive paradigm.} as an example. Other Japhug dialects have slightly different indexation suffixes, a question discussed in §\ref{sec:indexation.suffixes.history} with comparative evidence from other Gyalrong languages.

There is no stem alternation related to person indexation in the intransitive paradigm in any Japhug dialect. The invariable stem is represented with the symbol \ro{} in \tabref{tab:intransitive.indexation}.\footnote{This notation follows the Kirantological tradition (for instance \citealt{driem93dumi}). }

\begin{table}[H]  
\caption{The intransitive conjugation in Japhug}\label{tab:intransitive.indexation}
\begin{tabular}{lllllllll} \lsptoprule
Person & Form & \japhug{ɕe}{go} (Factual non-past) \\
\midrule
\textsc{1sg} & \ro{}-\forme{a} & \forme{ɕe-a} \\
\textsc{1du} & \ro{}-\forme{tɕi} & \forme{ɕe-tɕi} \\
\textsc{1pl} & \ro{}-\forme{ji} & \forme{ɕe-j} \\
\hline 
\textsc{2sg} & \forme{tɯ}-\ro{} & \forme{tɯ-ɕe} \\
\textsc{2du} & \forme{tɯ}-\ro{}-\forme{ndʑi} & \forme{tɯ-ɕe-ndʑi} \\
\textsc{2pl} & \forme{tɯ}-\ro{}-\forme{nɯ} & \forme{tɯ-ɕe-nɯ} \\
\hline 
\textsc{3sg} & \ro{} & \forme{ɕe} \\
\textsc{3du} & \ro{}-\forme{ndʑi} & \forme{ɕe-ndʑi} \\
\textsc{3pl} & \ro{}-\forme{nɯ} & \forme{ɕe-nɯ} \\
\hline 
generic & \forme{kɯ}-\ro{} & \forme{kɯ-ɕe} \\
\lspbottomrule
\end{tabular}
\end{table}

With a few well-identified exceptions, the indexation suffixes agree in person and number with the intransitive subject in Japhug. There is no indexation with possessors or oblique arguments, unlike closely related languages like Khroskyabs \citep{lai15person} or Tangut \citep{jacques16th}.

In the intransitive paradigm, five suffixes and two prefixes are found. The stress is always on the last syllable of the verb stem (§\ref{sec:suffixes}, except in a handful of forms, §\ref{sec:stress}), and all person indexation suffixes, including \forme{-a}, are unstressed and sometimes are even devoiced (§\ref{sec:stress}). Unlike other languages of the Trans-Himalayan, such as Khaling (where the dual inclusive and the third dual are homophonous, see \citealt[1113]{jacques12khaling}), in Japhug all slots in the intransitive paradigm are distinct, without ambiguity.  

\subsubsection{First person} \label{sec:intr.1}
First person subjects are indexed by a set of three suffixes marking both person and number: \forme{-a}, \forme{-tɕi} and \forme{-ji} for first singular, dual and plural, respectively. As in the pronominal paradigms (§\ref{sec:pers.pronouns}), there is no inclusive/exclusive distinction in Japhug; inclusive first+second person is indexed as \textsc{1du} or \textsc{1pl}, as shown by examples such as (\ref{ex:CWnARaRtCi}) where the \textsc{1du} indexation corresponds to the sum of the \textsc{2sg} and \textsc{1sg} pronouns in the phrase \forme{nɤʑo cʰo aʑo ni}. Equivalent examples with exclusive meaning (\textsc{1sg}+\textsc{3sg}) can be found in the corpus, for instance (\ref{ex:cho.aCGAtCi}) below in §\ref{sec:intrinsically.n.sg.subject}.\footnote{In the transitive paradigm, the inclusive/exclusive distinction is not present either, but note the case of inclusive semi-reflex\-ive configurations (§\ref{sec:incl.semi.reflexive}).
}

\begin{exe}
\ex \label{ex:CWnARaRtCi}
\gll  nɤʑo cʰo aʑo ni, nɤki, rɟɤmtsʰu ɯ-taʁ nɯtɕu χsɯ-sŋi ɕɯ-nɤʁaʁ-tɕi, ɕɯ-nɤmɲole-tɕi. \\
\textsc{2sg} \textsc{comit} \textsc{1sg} \textsc{du} \textsc{filler} sea \textsc{3sg}.\textsc{poss}-on \textsc{dem}:\textsc{loc} three-day \textsc{tral}-have.a.good.time:\textsc{fact}-\textsc{1du} \textsc{tral}-do.sightseeing:\textsc{fact}-\textsc{1du} \\
\glt `You and I will go on a three day tour on the sea.' (150827 mengjiangnv-zh, 216)
\end{exe}

The \textsc{1sg} \forme{-a} suffix is the only suffix in Japhug with a vowel other than \forme{ɯ} (or \forme{i} after palatal and alveolo-palatal consonants, §\ref{sec:W.i.contrast}), and is the only indexation suffix that can be followed by another indexation suffix in the transitive paradigm (§\ref{sec:double.number.indexation}). The \forme{-a} \textsc{1sg} person index is among the suffixes revealing the underlying form of the codas: \forme{-β}, \forme{-ɣ}, \forme{-ʁ}, \forme{-z}, which become unvoiced in some contexts (§\ref{sec:codas.inventory}) are realized as voiced (see for instance in \tabref{tab:verb.stem.1sg} below; \forme{-β} is realized \phonet{-w-} in this context, since it becomes an onset, §\ref{sec:consonant.phonemes}), but the coda \forme{-t} remains unvoiced (for instance \forme{scit-a} be.happy-\textsc{1sg} `I am happy'). The codas are resyllabified; for instance \forme{scit-a} is syllabified as \forme{sci/ta}).

 Some verb stems (independently of transitivity) undergo predictable phonological alterations when followed by \forme{-a}. With verb stems whose last syllable is an open syllable,  the \forme{-a} suffix merges its vowel. With closed syllable verb stem in \forme{-ɤC} (C representing a coda), the \textsc{1sg} suffix causes vowel assimilation. These phonological rules are presented in \tabref{tab:verb.stem.1sg}.

When the verb stem ends in \forme{-a}, the \textsc{1sg} suffix merges with the stem as \phonet{a} in Kamnyu Japhug, resulting in homophony between the \textsc{1sg} and the \textsc{3sg} forms. The surface form \phonet{rga} corresponds to both \textsc{1sg} \japhug{rga-a}{I like it} and \textsc{3sg} \japhug{rga}{he likes it}. The fused and invisible suffix is systematically indicated in the orthography used in this grammar. In the Sarndzu of Japhug, a long vowel occurs in the \textsc{1sg}, which thus remains different from the \textsc{3sg}.

When the verb stem ends in vowels other than \forme{-a}, these vowels undergo synizesis with the \forme{-a} suffix (§\ref{sec:synizesis}), merging into one syllable. In addition, the mid-high vowels \forme{-e} and \forme{-o} become the corresponding high vowels \forme{-i} and \forme{-u} in this context. 

These vowel mergers are obligatory in the Kamnyu dialect. However, there are not attested in all Japhug dialects, which may favor hiatus.

With verb stem ending in \forme{-ɤt}, \forme{-ɤn}, \forme{-ɤβ}, \forme{-ɤm}, \forme{-ɤr}, \forme{-ɤl} and \forme{-ɤz},  the \textsc{1sg} suffix causes non-optional vowel assimilation \forme{-ɤC-a} $\Rightarrow$ \ipa{-aCa} (§\ref{sec:vowel.harmony}). \tabref{tab:verb.stem.1sg} provides examples for all rhymes of this type. In the orthography employed in this grammar, these forms are transcribed as \forme{aC-a} rather than the underlying \forme{ɤC-a} (\forme{jɣat-a} rather than \forme{jɣɤt-a}), to indicate the fact that \forme{ɤ} $\Rightarrow$ \forme{a} assimilation is obligatory in this context.

\begin{table}
\caption{Predictable phonological alternations on the verb stem caused by the \forme{-a} \textsc{1sg} suffix in Kamnyu Japhug} \label{tab:verb.stem.1sg}
\begin{tabular}{llllll}
\lsptoprule
Rhyme of the  & Result of  &Examples \\
last syllable & fusion with  \\
of the verb stem & the \textsc{1sg} suffix \\
\midrule
\forme{-e} & \phonet{-ia} & \forme{ɕe-a} $\Rightarrow$ \phonet{ɕia} `I will go there' \\
\forme{-o} & \phonet{-ua} & \forme{tso-a} $\Rightarrow$ \phonet{tsua} `I understand it' \\
\forme{-a} & \phonet{-a} & \forme{rga-a} $\Rightarrow$ \phonet{rga} `I like it' \\
\hline 
\forme{-ɤβ} & \phonet{-awa} & \forme{tʰɯ-rdɤβ-a} $\Rightarrow$ \phonet{tʰɯrdáwa} `I lost money' \\
\forme{-ɤm} & \phonet{-ama} & \forme{mtsʰɤm-a} $\Rightarrow$ \phonet{mtsʰáma} `I hear it' \\
\forme{-ɤt} & \phonet{-ata} & \forme{jɣɤt-a} $\Rightarrow$ \phonet{jɣáta} `I will come back' \\
\forme{-ɤn} & \phonet{-ana} & \forme{tu-nɯsmɤn-a} $\Rightarrow$ \phonet{tunɯsmána} \\
&&  `I will treat it' \\
\forme{-ɤr} & \phonet{-ara} & \forme{pɯ-atɤr-a} $\Rightarrow$ \phonet{patára} `I fell down' \\
\forme{-ɤl} & \phonet{-ala} & \forme{nɯ-nɯtɯfɕɤl-a} $\Rightarrow$ \phonet{nɯnɯtɯfɕála}\\
&& `I had diarrhea' \\
\forme{-ɤz} & \phonet{-aza} & \forme{mkʰɤz-a} $\Rightarrow$ \phonet{mkʰáza} `I am expert at it' \\
\lspbottomrule
\end{tabular}
\end{table}

The first dual \forme{-tɕi} suffix (\forme{-tsə} in some dialects of Japhug, §\ref{sec:indexation.suffixes.history}) only causes regular devoicing assimilation on the coda of the verb stem: \forme{-z}, \forme{-r}, \forme{-ɣ}, \forme{-ʁ} are realized as \forme{-s}, \forme{-ʂ}, \forme{-x}, \forme{-χ} when followed by \forme{-tɕi} (for instance \forme{mkʰɤz-tɕi} is pronounced \phonet{mkʰɛ́stɕi}). The labial coda \forme{-β} is not affected. 

The first plural \forme{-ji} has two allomorphs, \forme{-j} and \forme{-i}. The first one occurs on verb stems ending in open syllables, for instance \japhug{ɕe-j} `we (will) go', and the second follows verb stems in closed syllables, such as \forme{scit-i} `we are happy', with resyllabification of the coda (\forme{sci/ti}). Like the \forme{-a} suffix discussed above, the suffix \forme{-i} reveals the underlying form of the codas. The contrast between \forme{-ɯ} and \forme{-i} is neutralized as \phonet{i} when followed by the \textsc{1pl} suffix: for instance, the last syllable of \forme{smi tʰɯ-βlɯ-j} `we made a fire'  and \forme{lɤpɯɣ pɯ-βli-j} `we planted radish' is considered to be homophonous (the last syllable is realized as \phonet{βlij}) by Tshendzin (§\ref{sec:rhyme.inventory}).

\subsubsection{Non-first person} \label{sec:intr.23}
Second and third person forms have the same set of suffixes (zero, \forme{-ndʑi} and \forme{-nɯ} for singular, dual and plural, respectively) and only differ by the presence of a \forme{tɯ-} prefix in second person forms. Unlike in Situ (\citealt[197--208]{linxr93jiarong}), there is no second person suffix in the \textsc{2sg}.

The non-first person dual and plural suffixes \forme{-ndʑi} and \forme{-nɯ} (some Japhug dialects have \forme{-ndzə} in the dual instead, see §\ref{sec:indexation.suffixes.history}) nasalize the coda \forme{-t} to \phonet{n}, which is not audible before \forme{-ndʑi} and results in a geminate in the plural. For instance, \forme{scit-ndʑi} and \forme{scit-nɯ} are realized as \phonet{scíndʑi} and \phonet{scínnɯ}, respectively. The vowel \forme{-i} and \forme{-ɯ} is often elided, resulting in apparent \forme{-n} codas. The contrast between the codas \forme{-n} and \forme{-t} is neutralized in these forms: the last two syllables of both \forme{tu-nɤndɯt-nɯ} \textsc{ipfv}-fight-\textsc{pl} `they fight (over it)' and \forme{pjɯ-ndɯn-nɯ}  \textsc{ipfv}-read-\textsc{pl} `they read/recite it' are thus realized as \phonet{-ndɯ́nnɯ}. 

By contrast, the codas \forme{-β}, \forme{-ɣ} and \forme{-ʁ} are \textit{not} nasalized to \phonet{m}, \phonet{ŋ} and \phonet{ɴ} when followed by the suffixes \forme{-ndʑi} and \forme{-nɯ}, respectively.

The second person \forme{tɯ-} prefix fuses with the initial \forme{a-} of contracting verbs (§\ref{sec:contraction}). The result of vowel fusion is \forme{tɯ-a-} $\Rightarrow$ \phonet{ta} in the Factual Non-past (\japhug{tɯ-atɤr}{you will fall down}) or the Aorist (\japhug{jɤ-tɯ-ari}{you went there}), but \forme{tɯ-ɤ-} $\Rightarrow$ \phonet{tɤ} in Irrealis, Imperative, Imperfective or Prohibitive (\japhug{ma-tɤ-tɯ-ɤɕqʰe}{don't cough}) forms. Some irregular verbs have unpredictable second person forms (§\ref{sec:intr.person.irregular}). The generic intransitive subject prefix \forme{kɯ-} (also used for the object of transitive verbs, see §\ref{sec:indexation.generic.tr}) follows the same rules of vowel fusion as the second person prefix.


\subsection{Irregular intransitive verbs} \label{sec:intr.person.irregular}
In comparison with Zbu \citep{gong18these}, Japhug only has very few irregular verbs. Irregularities related to person marking in Japhug all involve the prefixes.

The second person forms of Sensory Evidential existential verbs \japhug{ɣɤʑu}{exist} and \japhug{maŋe}{not exist} (§\ref{sec:sensory.morphology}) are infixed rather than prefixed. The infixed forms are \forme{ɣɤtɤʑu} and \forme{mataŋe}, as in (\ref{ex:GAtAZu}) (from \citealt[91]{jacques12agreement}) and  (\ref{ex:kAmtshAm.mataNe}). These two verbs lack non-finite morphology (§\ref{sec:nmlz.defective}) and do not occur in other tenses (§\ref{sec:verbs.no.preverbs}) and can be considered to be suppletive forms of the existential verbs \japhug{tu}{exist} and \japhug{me}{not exist} (§\ref{sec:suppletive.negative}).

\begin{exe}
\ex \label{ex:GAtAZu}
\gll iɕqʰa tɯrme ra nɯ-rca ɣɤ<tɤ>ʑu \\
the.aforementioned person \textsc{pl} \textsc{3pl}.\textsc{poss}-following <2sg>exist:\textsc{sens} \\
\glt `(I saw) you among these people.' (elicited)
\end{exe}

\begin{exe}
\ex \label{ex:kAmtshAm.mataNe}
\gll kɤ-mtsʰɤm maka ma<ta>ŋe tɕe, nɤ-kɯ-mŋɤm tu ɯβrɤ-ŋu ma, mɤ-kɯ-pe tu ɯβrɤ-ŋu ma nɯra nɯ-sɯso-t-a. \\
\textsc{inf}-hear at.all <\textsc{2sg}>not.exist:\textsc{sens} \textsc{lnk} \textsc{2sg}.\textsc{poss}-\textsc{sbj}:\textsc{pcp}-hurt exist:\textsc{fact} \textsc{rh}.\textsc{q}-be:\textsc{fact} \textsc{sfp}  \textsc{neg}-\textsc{sbj}:\textsc{pcp}-be.good:\textsc{fact} exist:\textsc{fact} \textsc{rh}.\textsc{q}-be:\textsc{fact} \textsc{sfp} \textsc{dem}:\textsc{pl} \textsc{aor}-think-\textsc{pst}:\textsc{tr}-\textsc{1sg} \\
\glt `(I) have not heard at all about you (for some time), I was wondering whether you have some disease, whether something bad happened to you.' (phone conversation, 16-12-28)
\end{exe}

These are not the only infixed forms in the paradigm of these verbs: the generic person \forme{kɯ-} is also infixed (\forme{ɣɤkɤʑu}, \forme{makaŋe}) as is the autive \forme{nɯ-} (§\ref{sec:outer.prefixal.chain}).
%nə-mɐŋɛʔ-nə Tshobdun \citep[100]{jackson19tshobdun} 

The verb \japhug{zɣɯt}{reach, arrive} has in part of its paradigm forms that are identical to those of contracting verbs (§\ref{sec:contraction}). In the Aorist, it has two alternative second person forms in free variation, the regular \forme{jɤ-tɯ-zɣɯt} and the form \forme{jɤ-tɯ-azɣɯt} with an additional \forme{a\trt}, illustrated by (\ref{ex:jAtWzGWt.tCe}) and  (\ref{ex:jAtazGWt.mACtsxa}), coming from two versions of the same story by the same speaker. 

\begin{exe}
\ex \label{ex:jAtazGWt.mACtsxa}
\gll  a-rkɯ mɯ-jɤ-tɯ-azɣɯt mɤɕtʂa mɯ-pɯ-ta-mtsʰɤm tɕe \\
\textsc{1sg}.\textsc{poss}-side \textsc{neg}-\textsc{aor}-2-arrive until \textsc{neg}-\textsc{aor}-1\fl{}2-hear \textsc{lnk} \\
\glt `I did not feel your (presence) until you arrived near me.' (2012 Norbzang, 260)
\end{exe}

\begin{exe}
\ex \label{ex:jAtWzGWt.tCe}
\gll jɤ-tɯ-zɣɯt tɕe, nɤki, aʑo a-kʰa a-jɤ-tɯ-z-mɤke ma nɤj nɤ-kʰa a-mɤ-jɤ-tɯ-z-mɤke ra mɯ-tɤ-tɯ-tɯt \\
\textsc{aor}-2-arrive \textsc{lnk} \textsc{filler} \textsc{1sg} \textsc{1sg}.\textsc{poss}-house \textsc{irr}-\textsc{pfv}-2-\textsc{caus}-be.first[III] \textsc{lnk} 
\textsc{2sg} \textsc{2sg}.\textsc{poss}-house \textsc{irr}-\textsc{neg}-\textsc{pfv}-2-\textsc{caus}-be.first[III] be.needed:\textsc{fact} \textsc{neg}-\textsc{aor}-2-say[III] \\
\glt `You did not say ``When you arrive, don't go first to your house, come to my house first.'' (2005 Norbzang, 261)
\end{exe}

The paradigm of this verb otherwise includes non-optional contracting (\forme{jɤ-azɣɯt} `he arrived') and non-contracting forms (the immediate converb \forme{ju-tɯ-zɣɯt} `as soon as X arrived', §\ref{sec:immediate.converb}).

\subsection{Semi-transitive verbs} \label{sec:semi.transitive}
Semi-transitive verbs have the same paradigm as plain intransitive verbs, and lack the morphological properties of transitive verbs (§\ref{sec:transitivity.morphology}). Their intransitive subject is in absolutive form. However, they take a semi-object (§\ref{sec:semi.object}), also in absolutive form, as \japhug{paχɕi}{apple} in (\ref{ex:paXCi.ci.taroa}). These semi-objects do present some objectal properties (§\ref{sec:semi.object}, §\ref{sec:semi.tr.relativization}).
 
\begin{exe}
\ex \label{ex:paXCi.ci.taroa}
\gll  tɕe aʑo tʰam kɯki, paχɕi ci tɤ-aro-a tɕe tɕendɤre, 
[...] nɯʑora kɯnɤ ta-sɯ-ɤʁe-nɯ ra \\
\textsc{lnk} \textsc{1sg} now \textsc{dem}.\textsc{prox} apple \textsc{indef} \textsc{aor}-have-\textsc{1sg} \textsc{lnk} \textsc{lnk} { } \textsc{2pl} also 1\fl{}2-\textsc{caus}-be.needed.eat:\textsc{fact}-\textsc{pl} be.needed:\textsc{fact} \\
\glt `Now that I have (was given) this apple, I will give it to you also to eat.' (150904 zhongli-zh, 35)
\end{exe}

Unlike transitive verbs, which can index the number of the object if the subject is \textsc{1sg} (§\ref{sec:double.number.indexation}), semi-transitive verbs cannot add a person index after the \textsc{1sg} \forme{-a}. For instance, in (\ref{ex:XsWm.aroa}), although the object is plural, a form such as $\dagger$\forme{aroa-a-nɯ} with the \forme{-nɯ} plural prefix is strictly prohibited. 

\begin{exe}
\ex   \label{ex:XsWm.aroa}
 \gll aʑo tɤ-rɟit χsɯm aro-a   \\
I \textsc{indef}.\textsc{poss}-child three have:\textsc{fact}-\textsc{1sg} \\
 \glt `I have three children.' (elicited)
\end{exe} 

The subject of some semi-transitive verbs, in particular \japhug{tso}{know, understand} and \japhug{ʑɣɤpa}{pretend}, can optionally be marked with the ergative like a transitive subject (§\ref{sec:S.kW}), as \forme{tɤ-mu nɯ kɯ} in (\ref{ex:kW.mWpjAtso}) and \forme{βdaʁmu nɯ kɯ} in (\ref{ex:kW.toZGApa}). 

\begin{exe}
\ex   \label{ex:kW.mWpjAtso}
 \gll  tɕendɤre [tɤ-mu nɯ kɯ] ɕɯ ŋu nɯ maka mɯ-pjɤ-tso tɕeri \\
\textsc{lnk} \textsc{indef}.\textsc{poss}-mother \textsc{dem} \textsc{erg} who be:\textsc{fact} \textsc{dem} at.all \textsc{neg}-\textsc{ifr}.\textsc{ipfv}-know \textsc{lnk} \\
\glt `The old woman did not realize who it was.' (2002 qaCpa, 242)
\end{exe}

\begin{exe}
\ex   \label{ex:kW.toZGApa}
 \gll iɕqʰa βdaʁmu nɯ kɯ [wuma ʑo ɯ-sɯm kɯ-sna] to-ʑɣɤpa \\
 the.aforementioned lady \textsc{dem} \textsc{erg} really \textsc{emph} \textsc{3sg}.\textsc{poss}-mind \textsc{sbj}:\textsc{pcp}-be.good \textsc{ifr}-pretend \\
 \glt `The lady pretended to be a good person.' (140520 ye tiane-zh, 44)
\end{exe}


Some semi-transitive verbs can take both nominal semi-object and complement clauses. For instance, \forme{tso} (which can be translated as `know', `understand' or `realize' depending on the context) occurs with nouns referring to speech or meaning as semi-object (as in \ref{ex:apWtWtso.smWlAm}), finite relative clauses (\ref{ex:rCanW.mWkAtsoa}) and also participial clauses (\ref{ex:kWNu.kutsoa}, §\ref{sec:tso.sWXsAl}).

\begin{exe}
\ex \label{ex:apWtWtso.smWlAm}
\gll pja mɯndʐamɯχtɕɯɣ nɯ-skɤt a-pɯ-tɯ-tso smɯlɤm\\
bird all.kinds \textsc{3pl}.\textsc{poss}-speech \textsc{irr}-\textsc{ipfv}-2-understand prayer\\
\glt `May you understand the speech of all species of birds!' (2003kandZislama, 85)
\end{exe}

\begin{exe}
\ex   \label{ex:rCanW.mWkAtsoa}
 \gll ci kɤ-pa-tɕi, nɯstʰɯci tɤ-nɤrʑaʁ ri, [nɯstʰɯci nɤ-ku ʑru rcanɯ] mɯ-kɤ-tso-a \\
 one \textsc{aor}-do-\textsc{1du} so.much \textsc{aor}-pass(time) \textsc{lnk} so.much \textsc{2sg}.\textsc{poss}-head be.strong:\textsc{fact} \textsc{unexp}:\textsc{deg} \textsc{neg}-\textsc{aor}-know-\textsc{1sg} \\
\glt `So much time has passed since we have married, I did not realize that your hair was so long.' (2003 Kunbzang, 467)
\end{exe}

\begin{exe}
\ex   \label{ex:kWNu.kutsoa}
 \gll  tɕe [tɕʰi ɯ-skɤt kɯ-ŋu ra] ku-tso-a ɲɯ-ra ma tu-tɯ-ti stɯsti, mɯ́j-ɕɯftaʁ-a ɲɯ-ti \\
\textsc{lnk} what \textsc{3sg}.\textsc{poss}-speech \textsc{sbj}:\textsc{pcp}-be \textsc{pl} \textsc{ipfv}-understand-\textsc{1sg} \textsc{sens}-be.needed \textsc{lnk} \textsc{ipfv}-2-say alone \textsc{neg}:\textsc{sens}-remember-\textsc{1sg} \textsc{sens}-say \\
 \glt `He says: `I need to understand what it is about (what objects these words refer to), otherwise if you only speak (if you only explain orally) I won't remember.'' (conversation 14-05-10, 79)
\end{exe}
%\begin{exe}
%\ex   \label{ex:ŋundZi.mWkAtsoa}
% \gll pʰu ŋu ɕi, pʰu ci mu ŋu-ndʑi nɯ mɯ-kɤ-tso-a ma \\
% male be:\textsc{fact} \textsc{qu} male \textsc{indef} female be:\textsc{fact}-\textsc{du} \textsc{dem} \textsc{neg}-\textsc{aor}-know-\textsc{1sg} \\
% \glt `I did not get to know whether it is the male that is like that, or both whether both the male and the female are.' (24-ZmbrWpGa, 81)
%\end{exe} 
Among semi-transitive verbs, we find the following subclasses:

\begin{itemize}
\item Verbs of cognition and perception: \japhug{tso}{understand}, `know' \japhug{sɤŋo}{listen}
\item Verbs of evaluation: \japhug{rga}{like}, \japhug{stu}{believe}, \japhug{dɯɣ}{have enough of}
\item Modal verbs: \japhug{cʰa}{can}
\item Verbs of possession:  \japhug{aro}{own}
\item Copulas: \japhug{ŋu}{be}, \japhug{maʁ}{not be}, \japhug{apa}{become} (§\ref{sec:copula.basic})
\item Verbs of assignation: \japhug{rmi}{be called}, \japhug{artsi}{count as}, \japhug{fse}{be like}
\item Verbs requiring an argument expressing time: \japhug{acʰɤt}{have X years of difference}, \japhug{tsu}{pass X time}
\item Verbs of pretence:  \japhug{ʑɣɤpa}{pretend}
\item Verbs of obtaining (§\ref{sec:preverb.gain}): \japhug{aʁe}{have to eat/drink}, \japhug{βɟɤt}{get, obtain}
\item Some adjectival stative verbs: \japhug{mkʰɤz}{be expert}, \japhug{pʰɤn}{be efficient}
\end{itemize}

The Tshobdun verbs \forme{cʰɐ} `can, be able' and \forme{rge} `like', cognates of the Japhug semi-transitive verbs \japhug{cʰa}{can} and \japhug{rga}{like}, are fully transitive, as shown by the forms \forme{mɐ-koɣ-cʰɐ-aŋ} (\textsc{neg}-2\fl{}1-can-\textsc{1sg}) `you cannot (kill) me' \citep[634]{jackson19tshobdun} and \forme{ne-tɐ-rge}  (\textsc{ipfv}-1\fl{}2-like) `I love you' \citep[674]{jackson19tshobdun}, which occur with local portmanteau prefixes (§\ref{sec:indexation.local}).

Most semi-transitive verbs are underived bare roots. The only obviously derived verbs are \japhug{ʑɣɤpa}{pretend}, which comes from the reflexive of the verb \japhug{pa}{do} (§\ref{sec:lexicalized.refl}) and \japhug{artsi}{count as}, passive of \japhug{rtsi}{count}. The verb \japhug{aro}{own} is possibly analyzable as a denominal verb historically derived from \japhug{tɤ-ro}{excess}, `surplus', `leftover' (§\ref{sec:bare.action.nominals}, §\ref{sec:denom.a}).

Most semi-transitive verbs do not usually take a human semi-object, so that sentences with a first or second person semi-object are generally clumsy to build. For some of the verbs above, applicative forms are used when a first or second person object is needed, for instance \japhug{nɯrga}{like} and \japhug{nɤstu}{believe}(§\ref{sec:applicative.promoted}). The verbs \japhug{stu}{believe} and \japhug{nɤstu}{believe} differ in that the semi-object of the former refers to words (in general, a complement clause; `believe that X') while the object of the latter is a person (`believe him'). 

However, examples with subjects and semi-objects both either first or second person are attested. For instance, (\ref{ex:WYWfsea}) shows a very spontaneous use of a \textsc{2sg} semi-object with a \textsc{1sg} subject with the verb \japhug{fse}{be like}. Only the subject is indexed (with the suffix \forme{-a}) and the use of the transitive \forme{ta-} 1\fl{}2 portmanteau prefix (§\ref{sec:indexation.local}) here would be nonsensical. 

\begin{exe}
\ex \label{ex:WYWfsea}
\gll a-ʁi, nɤʑo ɯ-ɲɯ-fse-a? \\
\textsc{1sg}.\textsc{poss}-younger.sibling \textsc{2sg} \textsc{qu}-\textsc{sens}-be.like-\textsc{1sg} \\
\glt `Sister, do I look like you?' (2014-kWlAG, 475)
\end{exe}

The same is observed with the copulas \japhug{ŋu}{be}, \japhug{maʁ}{not be}  and also \japhug{apa}{become} as in (\ref{ex:nAZo.napaa}) and (\ref{ex:aZo.nWtapa}). (see also §\ref{sec:copula.basic}) The copulas always index the subject, never the semi-object, independently of any person hierarchy (see \ref{ex:empathy.hierarchy.japhug} in §\ref{sec:direct-inverse} below), and cannot take the \forme{ta-} 1\fl{}2  and \forme{kɯ-} 2\fl{}1 portmanteau prefixes.
 
\begin{exe}
\ex \label{ex:nAZo.napaa}
\gll nɤʑo nɯ-apa-a \\
\textsc{2sg} \textsc{aor}-become-\textsc{1sg} \\
\glt `I became you.' (elicited)
\end{exe}

\begin{exe}
\ex \label{ex:aZo.nWtapa}
\gll aʑo nɯ-tɯ-apa \\
\textsc{1sg} \textsc{aor}-2-become \\
\glt `You became me.' (elicited)
\end{exe}

Some semi-transitive verbs are labile; some have a transitive counterpart, while others have a plain intransitive one (§\ref{sec:semi.tr.labile}). The meaning of the verb also slightly changes depending on transitivity (for instance, \forme{rga} means `like' when semi-transitive, and `be happy' when stative intransitive).
  
The antipassive form of secundative verbs (§\ref{sec:secundative.theme}), such as \japhug{rɤmbi}{give to someone} have a status intermediate between semi-transitive and monotransitive verbs: they lack most transitive features (§\ref{sec:transitivity.morphology}), but can index first and second person objects (§\ref{sec:ditransitive.secundative}, §\ref{sec:antipassive.ditransitive}).

\subsection{Intransitive verbs with oblique arguments} \label{sec:intr.goal}
Semi-transitive verbs have to be distinguished from motion verbs (or perception verbs) with a goal (§\ref{absolutive.goal}), such as \japhug{ɕe}{go}, \japhug{ɣi}{come} or \japhug{ru}{look at}. These verbs are morphologically intransitive, lacking the morphological characteristics of transitive verbs (§\ref{sec:transitivity.morphology}). 
 
With these verbs, the goal can occur in absolutive form, and superficially resembles a semi-object, as \japhug{sɯŋgɯ}{forest} in (\ref{ex:sWNgW.joCendZi}). Indexation obligatorily occurs with the subject (for example, the \textsc{3du} form in \ref{ex:sWNgW.joCendZi}), never with the goal. As in the case of semi-transitive verbs, number stacking on the \textsc{1sg} \forme{-a} is not possible (§\ref{sec:semi.transitive}, example \ref{ex:XsWm.aroa}).

\begin{exe}
\ex   \label{ex:sWNgW.joCendZi}
 \gll ʁnɯz ni, [sɯŋgɯ] jo-ɕe-ndʑi. \\
two \textsc{du} forest \textsc{ifr}-go-\textsc{du} \\
\glt `Two (men) went into the forest.' (26-tAGe, 1)
\end{exe}

However, unlike semi-objects, these goals can optionally take locative postpositions, such as \forme{zɯ} in (\ref{ex:sWNgW.zW.joCe}).

\begin{exe}
\ex   \label{ex:sWNgW.zW.joCe}
 \gll tɤ-pɤtso nɯnɯ li [sɯŋgɯ zɯ] jo-ɕe. \\
 \textsc{indef}.\textsc{poss}-child \textsc{dem} again forest \textsc{loc} \textsc{ifr}-go \\
 \glt `The child went again into the forest.' (140428 yonggan de xiaocaifen-zh, 232)
\end{exe}

Dative marking on the goals is also well-attested, as in (\ref{ex:sWNgW.WCki.joCe}) -- with motion verbs, it translates as `towards X'.

\begin{exe}
\ex   \label{ex:sWNgW.WCki.joCe}
 \gll tɕhemɤpɯ nɯ kɯ ɯ-wa cʰo ɯ-pi nɯra ɲɤ-βde tɕe, sɯŋgɯ ɯ-ɕki tɕe jo-ɕe. \\
girl \textsc{dem} \textsc{erg} \textsc{3sg}.\textsc{poss}-father \textsc{comit} \textsc{3sg}.\textsc{poss}-elder.sibling \textsc{dem}:\textsc{pl} \textsc{ifr}-leave \textsc{lnk} forest \textsc{3sg}.\textsc{poss}-\textsc{dat} \textsc{loc} \textsc{ifr}-go \\
\glt `The girl left her father and her brothers, and went toward the forest.' (140506 shizi he huichang de bailingniao, 76)
\end{exe}

The subject of intransitive verbs with goals is in absolutive form, except when shared with a transitive verb in another clause, as \forme{tɕʰemɤpɯ nɯ kɯ} in (\ref{ex:sWNgW.WCki.joCe}), which owes its ergative marking to the transitive verb \forme{ɲɤ-βde} `She left them'. The verb \japhug{rpu}{bump into} (which takes as goal the surface of physical contact) however can take ergative subjects, as it is labile and can be conjugated transitively (§\ref{sec:goal.labile}). Transitive verbs of manipulation can optionally take goals that are formally similar to those of \japhug{ɕe}{go} and \japhug{ɣi}{come} (§\ref{sec:secundative.monotransitive}).

Some intransitive verbs of speech, \japhug{rɯɕmi}{speak} and \japhug{akʰu}{call}, can optionally take a dative argument, as in (\ref{ex:WCki.turWCmia}).

\begin{exe}
\ex   \label{ex:WCki.turWCmia}
 \gll tɯrme ɯ-ɕki tu-rɯɕmi-a ɕti wo \\
 person \textsc{3sg}.\textsc{poss}-\textsc{dat} \textsc{ipfv}-speak-\textsc{1sg} be.\textsc{aff}:\textsc{fact} \textsc{sfp} \\
\glt `I am talking to someone (else).' (phone conversation, 2013-12-02)
\end{exe}

Other than locative and dative, some intransitive verbs select oblique arguments with the genitive (§\ref{sec:genitive}) the relator noun \japhug{ɯ-taʁ}{on} (§\ref{sec:WtaR}) and verbs with intrinsically non-singular subjects often occur with comitative postpositional phrases in \forme{cʰo} (§\ref{sec:intrinsically.n.sg.subject}).

Some intransitive verbs which cannot take semi-objects do occur with counted nouns (in particular, \japhug{rɯɕmi}{speak}); these counted nouns however have scope over the whole sentence, and are not arguments of these verbs, as argued in §\ref{sec:CN.iterative}. 

\subsection{Semi-transitive verbs with additional oblique arguments} \label{sec:semi.transitive.dative}
Some semi-transitive verbs can have up to three arguments, and are therefore trivalent.\footnote{
This shows that trivalent verbs in Japhug are not necessarily ditransitive (§\ref{sec:ditransitive}).
} This is the case of rogative verbs (§\ref{sec:rogative.derivation}) such as \japhug{sɤmbi}{ask for} (derived from the secundative verb \japhug{mbi}{give}), which have both a dative argument and a semi-object, as shown by (\ref{ex:Wphe.piaozi}), but are yet conjugated intransitively (the generic subject is marked by \forme{kɯ\trt}, §\ref{sec:indexation.generic.tr}).

\begin{exe}
\ex   \label{ex:Wphe.piaozi}
 \gll <guojia> ɯ-pʰe <piaozi> ɲɯ-kɯ-sɤ-mbi \\
country \textsc{3sg}.\textsc{poss}-\textsc{dat} money \textsc{ipfv}-\textsc{genr}:S/O-\textsc{rog}-give \\
\glt `Ask the government for money.' (2010, 09)
\end{exe}

\subsection{Intrinsically non-singular subjects} \label{sec:intrinsically.n.sg.subject}
Some intransitive verbs have an intrinsic reciprocal meaning, and do not occur in singular form. This category includes most derived reciprocal verbs (§\ref{sec:reciprocal}), but also some historical reciprocal verbs that are synchronically non-analyzable such as \japhug{amɯmi}{be in good terms}, and denominal verbs in \forme{a-} like \japhug{anɯmqaj}{fight} or \japhug{aɕɣa}{be of the same age} (§\ref{sec:denom.a}). Example (\ref{ex:atAtAnWmqajnW}) provides some examples of verbs of this type. In the corpus, these verbs only occur in dual or plural form.

\begin{exe}
\ex   \label{ex:atAtAnWmqajnW}
 \gll tɕe a-pi a-ʁi ra kutɕu a-nɯ-tɯ-ɤnɯɣro-nɯ, ci ci a-tɤ-tɯ-ɤnɯmqaj-nɯ, ci ci a-tɤ-tɯ-ɤmɯmi-nɯ qʰe a-kɤ-tɯ-nɯ-rɤʑi-nɯ, \\
\textsc{lnk} \textsc{1sg}.\textsc{poss}-elder.sibling  \textsc{1sg}.\textsc{poss}-younger.sibling \textsc{pl}  here \textsc{irr}-\textsc{pfv}-2-<\textsc{auto}>play-\textsc{pl} once once \textsc{irr}-\textsc{pfv}-2-fight-\textsc{pl} once once \textsc{irr}-\textsc{pfv}-2-be.in.good.terms-\textsc{pl} \textsc{lnk} \textsc{irr}-\textsc{pfv}-2-\textsc{auto}-stay-\textsc{pl}\\
\glt `Brothers, stay here and play, fight from time to time, reconcile with each other from time to time.' (2003kandzwsqhaj, 43)
\end{exe}

The subject of these verbs can be a noun phrase comprising a comitative postpositional phrase in \forme{cʰo} (see §\ref{sec:comitative} and §\ref{sec:coordinator.cho}), in particular with reciprocal verbs (§\ref{sec:redp.reciprocal}); number indexation on the verb reflects the addition of the added number of all nominals in the noun phrase. For example, in (\ref{ex:cho.pjAkAmWmindZi}) and (\ref{ex:cho.aCGAtCi}), the verbs have dual indexation, referring to the total number of individuals in the subject noun phrase connected by the comitative \forme{cʰo}.

\begin{exe}
\ex   \label{ex:cho.pjAkAmWmindZi}
 \gll  <maerjina> nɯ cʰo <alibaba> ni wuma ʑo pjɤ-k-ɤmɯmi-ndʑi tɕe \\
\textsc{anthr} \textsc{dem} \textsc{comit}  \textsc{anthr} \textsc{du} really \textsc{emph} \textsc{pst}.\textsc{ifr}-\textsc{peg}-be.in.good.terms-\textsc{du} \textsc{lnk} \\
\glt `Maerjina and Alibaba were in very good terms.' (140512 alibaba-zh, 306)
\end{exe}

\begin{exe}
\ex   \label{ex:cho.aCGAtCi}
 \gll  nɤj nɤ-mu cʰo aʑo ni aɕɣa-tɕi \\
 \textsc{2sg} \textsc{2sg}.\textsc{poss}-mother \textsc{comit} \textsc{1sg} \textsc{du} be.of.the.same.age:\textsc{fact}-\textsc{1du} \\
 \glt `I have the same age as your mother.' (`You mother and I have the same age') (elicited)
\end{exe} 

The verb \japhug{acʰɤt}{have X years of difference} has an intrinsically non-singular subject, and is at the same time semi-transitive (§\ref{sec:semi.transitive}), taking as semi-object a temporal noun phrase expressing the age difference between the members of the group referred to by the subject (§\ref{sec:a.non.passive.denominal}).
 
The verb \japhug{alɯlɤt}{fight}, historically the reciprocal of \japhug{lɤt}{release} (§\ref{sec:redp.lexicalized}), almost always has non-singular indexation, either with a non-singular intransitive subject as in (\ref{ex:qro.ni.YAlWlAtndZi}), or with a comitative phrase (\ref{ex:cho.tokAlWlAtndZi}) as in examples (\ref{ex:cho.pjAkAmWmindZi}) and (\ref{ex:cho.aCGAtCi}) above.

\begin{exe}
\ex   \label{ex:qro.ni.YAlWlAtndZi}
 \gll  tɕeki qro ni ɲɯ-ɤlɯlɤt-ndʑi \\
 down ant \textsc{du} \textsc{sens}-fight-\textsc{du} \\
\glt `Down there two ants are fighting.' (conversation140501-01)
\end{exe}

\begin{exe}
	\ex   \label{ex:cho.tokAlWlAtndZi}
	\gll ʁdɯxpakɤrpu ɯ-tɕɯ cʰondɤre a-tɕɯ ni to-k-ɤlɯlɤt-ndʑi \\
	\textsc{anthr} \textsc{3sg}.\textsc{poss}-son \textsc{comit}  \textsc{3sg}.\textsc{poss}-son \textsc{du} \textsc{ifr}-\textsc{peg}-fight-\textsc{du} \\
	\glt `The son of Gdugpa Dkarpo fought with my son.' (2011-04-smanmi, 49)
\end{exe}


However, examples with singular indexation are also attested, for instance (\ref{ex:tAtalWlAt}) and (\ref{ex:tutalWlAt}) with \textsc{2sg} form. Both are from texts translated from Chinese, but were not considered infelicitous by Tshendzin. While direct Chinese influence on indexation is unlikely, the use of \japhug{alɯlɤt}{fight} with the person one fights against left unexpressed (as in \ref{ex:tAtalWlAt}) or marked in the dative (as in \ref{ex:tutalWlAt}), is not attested elsewhere.\footnote{The Chinese original sentences of examples (\ref{ex:tAtalWlAt}) and (\ref{ex:tutalWlAt}) are \ch{你是为众人的利益而战}{nǐ shì wèi zhòngrén de lìyì érzhàn}{You were fighting for the interest of the people} and \ch{你不该和舅舅动手}{nǐ bùgāi hé jiùjiù dòngshǒu}{You should not get into a fight with your uncle}.  } 

\begin{exe}
\ex   \label{ex:tAtalWlAt}
 \gll  ki ɕɯŋgɯ ki pɯpɯŋunɤ, nɤʑo kɯ iɕqʰa mkʰɤrmaŋ ɣɯ nɯ-ndʐa kɯ tɤ-tɯ-alɯlɤt pɯ-ŋu tɕe, \\
 \textsc{dem}.\textsc{prox} before \textsc{dem}.\textsc{prox} \textsc{top} \textsc{2sg} \textsc{erg} \textsc{filler} people \textsc{gen} \textsc{3pl}.\textsc{poss}-reason \textsc{erg} \textsc{aor}-2-fight \textsc{pst}.\textsc{ipfv}-be \textsc{lnk} \\
\glt `The previous time, you fought for the sake of the people.' (140512 abide he mogui-zh, 89)
 \end{exe}

\begin{exe}
\ex   \label{ex:tutalWlAt}
 \gll  tɕe nɤʑo nɤ-rpɯ ɯ-ɕki, nɤkinɯ, tu-tɯ-ɤlɯlɤt ndɤre mɤ-pe \\
 \textsc{lnk} \textsc{2sg} \textsc{2sg}.\textsc{poss}-MB \textsc{3sg}.\textsc{poss}-\textsc{dat} \textsc{filler} \textsc{ipfv}-2-fight \textsc{lnk} \textsc{neg}-be.good:\textsc{fact} \\
 \glt `It is not good for you to fight with your uncle.' (150826 baoliandeng-zh, 185)
  \end{exe}
  
The verb \japhug{naχtɕɯɣ}{be the same}, although requiring the comparison of at least two entities, most often occurs with singular indexation, as shown by examples like (\ref{ex:WGli.cho.naXtCWG}) and (\ref{ex:ndZitWwxti.YWnaXtCWG}). This may be due to the fact \japhug{naχtɕɯɣ}{be the same} is commonly used with inanimate referents (§\ref{sec:optional.indexation}), and also because it commonly takes degree nominals  (§\ref{sec:degree.nominals}) as subjects in the equative construction (§\ref{sec:nmlz.equative}). These degree nominals can take non-singular possessors (dual in \ref{ex:ndZitWwxti.YWnaXtCWG}), but remain themselves singular arguments.

\begin{exe}
\ex   \label{ex:WGli.cho.naXtCWG}
 \gll   tɕe nɯnɯ qaʑo ɯ-ɣli nɯ li tsʰɤt ɣɯ ɯ-ɣli cʰo naχtɕɯɣ ʑo tɕe, \\
\textsc{lnk} \textsc{dem} sheep \textsc{3sg}.\textsc{poss}-dung \textsc{dem} again goat \textsc{3sg}.\textsc{poss}-dung \textsc{comit} be.the.same:\textsc{fact} \textsc{emph} \textsc{lnk} \\
\glt `Sheep dung is similar to goat dung.'(05-qaZo, 97)
  \end{exe}

\begin{exe}
\ex   \label{ex:ndZitWwxti.YWnaXtCWG}
\gll  ndʑi-tɯ-wxti ɲɯ-naχtɕɯɣ \\
\textsc{3du}.\textsc{poss}-\textsc{nmlz}:\textsc{deg}-be.big \textsc{sens}-be.the.same \\
\glt `They have the same size.' (`Their size is the same.') (24-ZmbrWpGa, 24)
\end{exe}

\subsection{Invariable intransitive verbs} \label{sec:intransitive.invariable}
A non-negligible amount of intransitive verbs are only attested in \textsc{3sg} form. Four categories must be distinguished. 

First, we find intransitive auxiliary verbs taking complement clauses as intransitive subjects (§\ref{sec:ra.khW.jAG.verb}), for instance \japhug{jɤɣ}{be allowed}, which has \textsc{3sg} indexation regardless of the person of the subject or the object in the complement clause, as shown by example (\ref{ex:YWkhama.jAG}). For verbs of this type, the indexation restriction is structurally determined and independent of semantics.

\begin{exe}
\ex   \label{ex:YWkhama.jAG}
\gll aʑo a-me nɯ [nɤʑɯɣ nɤ-rʑaβ ɲɯ-kʰam-a] jɤɣ, [nɤ-rʑaβ a-kɤ-βze] jɤɣ \\
\textsc{1sg} \textsc{1sg}.\textsc{poss}-daughter \textsc{dem} \textsc{2sg}:\textsc{gen} \textsc{2sg}.\textsc{poss}-wife \textsc{sens}-give-\textsc{1sg} be.allowed:\textsc{fact} \textsc{2sg}.\textsc{poss}-wife \textsc{irr}-\textsc{pfv}-make[III] be.allowed:\textsc{fact} \\
\glt `(If you succeed), I will agree to give my daughter to you in marriage, she can become your wife (140518 huifei de muma-zh, 71)
\end{exe}

A second type of invariable verbs are those exclusively attested in noun-verb collocations where the noun is the intransitive subject. For instance, the intransitive verbs \forme{mbi} and \forme{ŋgɯ} are only found\footnote{There are homophonous verbs such as the transitive \japhug{mbi}{give} and the adjectival stative verb \japhug{ŋgɯ}{be poor} but these are synchronically and even historically unrealted (\forme{mbi} is an anticausative verb, see §\ref{sec:anticausative.collocation}).  } in collocation with \japhug{tɤ-mbrɯ}{anger} and the orphan noun \forme{tɯ-ʁo}, meaning `be angry' and `be discouraged be frustrated, lose heart', respectively. The person and number of the experiencer are marked by the possessive prefix on the nouns, as in (\ref{ex:nWRo.amAnWmbi}), and since these nouns are always singular, the verb indexation is also always in \textsc{3sg} form.

\begin{exe}
\ex   \label{ex:nWRo.amAnWmbi}
\gll  nɯ-mbrɯ a-mɤ-tɤ-ŋgɯ, a nɯ-ʁo a-mɤ-nɯ-mbi \\
\textsc{2sg}.\textsc{poss}-be.angry(1) \textsc{irr}-\textsc{neg}-\textsc{pfv}-be.angry(2) \textsc{interj} \textsc{2sg}.\textsc{poss}-be.discouraged(1) \textsc{irr}-\textsc{neg}-\textsc{pfv}-be.discouraged(2)  \\
\glt `Don't be angry, don't feel frustrated.' (2003kAndzwsqhaj2, 128)
\end{exe}

This constraint on indexation is also found in collocations (§\ref{sec:intr.light.verbs}) in cases when the verb is elsewhere attested, for instance \japhug{tɯ-ʁjiz+ɣi}{wish} (§\ref{sec:z.nmlz}, §\ref{sec:nouns.cognition.complement}). The motion verb \japhug{ɣi}{come} is compatible with all person indexation affixes, but in this collocation it is only found in \textsc{3sg}, regardless of the person of the possessive prefix on \forme{tɯ-ʁjiz}, as shown by (\ref{ex:aRjiz.Gi.Cti}).

\begin{exe}
\ex   \label{ex:aRjiz.Gi.Cti}
\gll   nɯ a-ʁjiz ɣi ɕti \\
\textsc{dem} \textsc{1sg}.\textsc{poss}-wish come:\textsc{fact} be.\textsc{aff}:\textsc{fact} \\
\glt `I want that.' (140520 xiaoyida de huar-zh, 63)
\end{exe}

A third type of verbs with restricted \textsc{3sg} indexation are verbs which pragmatically require an inanimate subject, such as \japhug{rpjɯ}{turn sour}, which can only apply with \japhug{tɤ-lu}{milk}, as in (\ref{ex:tAlu.torpjW}) and would be nonsensical in first or second person.

\begin{exe}
\ex   \label{ex:tAlu.torpjW}
\gll    tɤ-lu to-rpjɯ \\
\textsc{indef}.\textsc{poss}-milk \textsc{ifr}-turn.sour \\
\glt `The milk turned sour.' (elicited)
\end{exe}

Fourth, meteorological verbs such as \japhug{ɣɯtsʰɤdɯɣ}{hot} or \japhug{qanɯ}{be dark} occur with a dummy intransitive subject, as in (\ref{ex:qhaqhu.mAZW.YWGWtshAdWG}) and (\ref{ex:CAr.kAqanW}). In these examples, \japhug{qʰaqʰu}{behind the house} and \japhug{ɕɤr}{night} are locative and temporal absolutive adjuncts , respectively (§\ref{absolutive.locative}), not the subjects of these sentences.

\begin{exe}
\ex   \label{ex:qhaqhu.mAZW.YWGWtshAdWG}
\gll qʰaqʰu mɤʑɯ ɲɯ-ɣɯtshɤdɯɣ \\
behind.the.house even.more \textsc{sens}-be.hot \\
\glt `It feels even hotter behind the house.' (conversation 14-05-10)
\end{exe}

\begin{exe}
\ex   \label{ex:CAr.kAqanW}
\gll  tɕe ɕɤr wuma ʑo kɤ-qanɯ tɕe \\
\textsc{lnk} night really \textsc{emph} \textsc{aor}-be.dark about\textsc{lnk} \\
\glt `In the night, when it has become very dark.' (23-qapGAmtWmtW, 141)
\end{exe}

The noun \japhug{tɯ-mɯ}{sky, weather} occurs in a few cases as subject of intransitive meteorological verbs in the corpus, in examples such as (\ref{ex:tWmW.WtWGWtshAdWG}), but only in texts translated from Chinese. These sentences are rejected by Tshendzin, and the presence of \japhug{tɯ-mɯ}{sky, weather} is a calque from Chinese \ch{天气}{tiānqì}{weather}.\footnote{Example (\ref{ex:tWmW.WtWGWtshAdWG}) corresponds to \ch{由于天气太热}{yóuyú tiānqì tài rè}{Because it was too hot} in the Chinese original.}

\begin{exe}
\ex   \label{ex:tWmW.WtWGWtshAdWG}
\gll tɕendɤre tɯ-mɯ ɯ-tɯ-ɣɯtsʰɤdɯɣ pjɤ-sɤre ʑo tɕe, \\
\textsc{lnk} \textsc{indef}.\textsc{poss}-sky \textsc{3sg}.\textsc{poss}-\textsc{nmlz}:degree-be.hot \textsc{ifr}.\textsc{ipfv}-be.ridiculous \textsc{emph} \textsc{lnk} \\
\glt (140512 fushang he yaomo-zh, 6)
\end{exe}

Some verbs may appear to be invariable only because the non-\textsc{3sg} forms are rare. For instance, the auxiliary verb \japhug{ra}{have to, need} resembles \japhug{jɤɣ}{be allowed} in taking complement clauses as intransitive subject (§\ref{sec:ra.khW.jAG.verb}), and it is almost always used in \textsc{3sg} form. However, in the meaning `need, want', it can take a human as subject, and (\ref{ex:mAtWra}) provides an example of \textsc{2sg} indexation with this verb.

\begin{exe}
\ex   \label{ex:mAtWra}
\gll  tʰɯ-nɯ-ɕe ma, mɤ-tɯ-ra \\
\textsc{imp}:\textsc{downstream}-\textsc{vert}-go \textsc{lnk} \textsc{neg}-2-need:\textsc{fact} \\
\glt `Go back, (I) don't need you.' (2002 qaCpa, 31)
\end{exe}

\section{Transitive verbs} \label{sec:tr.indexation}

\subsection{The morphological marking of transitivity in Japhug} \label{sec:transitivity.morphology}
Transitivity in Japhug can be defined exclusively on the basis of verbal morphology. Transitive verbs have the following six common properties.

First, the Aorist 3\textsc{sg}\fl{}3$'$ (§\ref{sec:aor.morphology}, §\ref{sec:preverb.TAME.morphology}),\footnote{The notation 3\textsc{sg}\fl{}3$'$ means `third singular subject with another third person object', see §\ref{sec:polypersonal}. } a form that is attested on all transitive verbs except for one (§\ref{sec:irregular.transitive}) requires a C-type orientation preverb (§\ref{sec:kamnyu.preverbs}) in Kamnyu Japhug. This test can always be applied to distinguish a transitive verb from an intransitive one. For instance, the Aorist 3\textsc{sg}\fl{}3$'$ of \japhug{ndza}{eat} is \forme{ta-ndza} `he ate it' with the C-type prefix \forme{ta\trt}, while an A-type prefix \forme{tɤ-} would be expected if the verb were intransitive (the incorrect form $\dagger$\forme{tɤ-ndza}). A potential problem with this test is the fact that the result of the merger of A-type orientation preverbs with the initial \forme{a-} of contracting verbs (which are always intransitive, see §\ref{sec:contraction}) is formally identical to a C-type prefix, for instance the Aorist \textsc{3sg} \forme{tɤ-ala} `it boiled up' is realized \ipa{tala}, the first syllable \ipa{ta-} being homophonous with that of \forme{ta-ndza} `he ate it'. It is therefore necessary to take into consideration the Factual Non-Past form (§\ref{sec:fact.morphology}) to determine whether the stem has initial \forme{a-}. A contrast similar to the one between A-type and C-type orientation preverbs is also found on the apprehensive prefix \forme{ɕɯ\trt}, which has a form \forme{ɕa-} in the 3\textsc{sg}\fl{}3$'$ Past form of transitive verbs (§\ref{sec:apprehensive}).

Second, verbs with open-syllable stems allowing stem III alternation (\forme{-a}, \forme{-o}, \forme{-u} and \forme{-ɯ}, §\ref{sec:stem3.form}) have stem III in all \textsc{sg}\fl{}3 non-past finite forms  (§\ref{sec:stem3.distribution}). For instance, the Factual Non-Past 3\textsc{sg}\fl{}3$'$ of \japhug{ndza}{eat} has the stem III form \forme{ndze} `he will eat it'. If this were intransitive, the stem I \forme{ndza} would be found instead. This alternation is an effective test to determine if a verb is transitive or not, but its applicability depends on the form of the stem.

Third, transitive verbs require a bare infinitive (§\ref{sec:bare.inf}) when occurring in specific complement clauses, while intransitive verbs select dental infinitives (§\ref{sec:dental.inf}). In (\ref{ex:Wndza.tuZanW}) for example, \japhug{ndza}{eat} occurs in a complement clause governed by the phasal verb \forme{tu-ʑa-nɯ} `they start' in bare infinitive form. If it were an intransitive verb, a dental infinitive ($\dagger$\forme{tɯ-ndza tu-ʑa-nɯ}) would be found instead. This criterion is not applicable to all verbs, since some of them cannot occur in the subtype of complement clauses where bare infinitives are found for semantic reasons.

\begin{exe}
\ex   \label{ex:Wndza.tuZanW}
\gll  ɕlaʁ pjɯ-si ɕɯŋgɯ qʰe ɯ-ɕa ɯ-ndza tu-ʑa-nɯ. \\
\textsc{idph}(I):immediately \textsc{ipfv}-die before \textsc{lnk} \textsc{3sg}.\textsc{poss}-meat \textsc{3sg}.\textsc{poss}-\textsc{bare}.\textsc{inf}:eat \textsc{ipfv}-start-\textsc{pl} \\
\glt `They start eating its flesh just before it dies.' (20-sWNgi, 57)
\end{exe}

Fourth, the subject participles of transitive verbs require a possessive prefix coreferent with the object, unless another inflectional prefix (marking orientation, polarity or associated motion) precedes the participle prefix \forme{kɯ-}  (§\ref{sec:subject.participle.possessive}). The subject participle of \japhug{ndza}{eat} is thus \forme{ɯ-kɯ-ndza} `the one who eats it', and the prefix \forme{ɯ-} cannot be removed.\footnote{In the negative form \ipa{mɤ-kɯ-ndza} (\textsc{neg}-\textsc{pcp}:\textsc{sbj}-eat) `the one who does not eat it', by contrast, the possessive prefix \forme{ɯ-} is optional.}

Fifth, transitive verbs with open-syllable stems require the past tense \forme{-t} suffix (§\ref{sec:suffixes}, §\ref{sec:indexation.mixed}, §\ref{sec:other.TAME}) in Aorist and Past Imperfective \textsc{1sg}\fl{}3 and \textsc{2sg}\fl{}3 forms. For example, \forme{tɤ-ndza-t-a} `I ate it' has this past suffix; if this verb were intransitive, a form such as $\dagger$\forme{tɤ-ndza-a} would be found instead. This criterion only applies to open-syllable transitive verbs allowing first and second person subjects.

Sixth, the Progressive \forme{asɯ-} (§\ref{sec:progressive.morphology}) is only found on transitive verbs. For instance, the transitive verb \japhug{ndza}{eat} has a Progressive Sensory 3\textsc{sg}\fl{}3$'$ form \forme{ɲɯ-ɤsɯ-ndza} `He is eating it', which would not exist if this verb were intransitive.

Seventh, only transitive verbs\footnote{Antipassive secundative verbs such as \japhug{rɤmbi}{give to someone} are the only exception, see §\ref{sec:secundative.theme}, examples (\ref{ex:nWwGrAmbia}) and (\ref{ex:YAkWrAmbia}).} can take the inverse \forme{wɣ-} (§\ref{sec:allomorphy.inv}) and the portmanteau \forme{kɯ-} 2\fl{}1 and \forme{ta-} 1\fl{}2  prefixes (§\ref{sec:indexation.local}). The portmanteau prefixes are only found on verbs allowing human subjects and objects. 

These seven criteria are almost completely congruent. With the exception of a handful of defective verbs (§\ref{sec:irregular.transitive}), the dummy transitive subject construction (§\ref{sec:transitive.dummy}), and the antipassive forms of some secundative verbs (§\ref{sec:ditransitive.secundative}, §\ref{sec:antipassive.ditransitive}), if a verb satisfies one of these criteria, it will satisfies them all. This excludes cases when particular tests are not applicable, in particular criteria 2 and 5 if the verb has a closed-syllable stem or 3, 5, 6 and 7 due to semantic incompatibility. Only criteria 1 and 4 apply to all verbs. 

Intransitive verbs, including semi-transitive verbs (§\ref{sec:semi.transitive}) do not satisfy any of these seven criteria, as can be exemplified with the verb \japhug{tso}{know, understand} (\tabref{tab:tso.tests} and example \ref{ex:tWtso.kAZata} illustrating test 3).

\begin{table}[H]
	\caption{Transitivity tests with the  semi-transitive verb \japhug{tso}{know, understand}} \label{tab:tso.tests}
	\begin{tabular}{llll}
		\lsptoprule 
		&Test &Attested&Expected form  \\
		&&Form&if transitive \\
		\midrule 
		1&\textsc{3sg} Aorist (C-type preverb) & \forme{kɤ-tso} & $\dagger$\forme{ka-tso} \\
		2&Factual \textsc{3sg} (Stem III) &\forme{tso}&$\dagger$\forme{tsɤm} \\
		3&Dental infinitive&\forme{tɯ-tso (kɤ-ʑa-t-a)} (\ref{ex:tWtso.kAZata})&$\dagger$\forme{ɯ-tso (kɤ-ʑa-t-a)} \\
		4&Subject participle&\forme{kɯ-tso}& $\dagger$\forme{ɯ-kɯ-tso}\\
		5&Past tense 1/2\textsc{sg}& \forme{kɤ-tso-a} & $\dagger$\forme{kɤ-tso-t-a} \\
		6&Progressive & --&$\dagger$\forme{asɯ-tso} \\
		7&Inverse or portmanteau &--&$\dagger$\forme{ɣɯ-tso} \\
		& prefix \\
		\lspbottomrule 
	\end{tabular}
\end{table}




\begin{exe}
\ex   \label{ex:tWtso.kAZata}
\gll aʑo tɯ-tso kɤ-ʑa-t-a \\
\textsc{1sg} \textsc{inf}:II-understand \textsc{aor}-start-\textsc{pst}:\textsc{tr}-\textsc{1sg} \\
\glt `I began to understand.' (elicited)
\end{exe}

The morphosyntactic specificities of transitive verbs are not limited to inflectional verbal morphology. Some derivational processes, such as antipassivization (§\ref{sec:antipassive}) and anticausativization (§\ref{sec:anticausative.direction}) can only be applied to transitive verbs, and conversely other derivations (such as the applicative, §\ref{sec:applicative}) are only attested on intransitive verbs. Since derivational morphology however, is not as productive as inflectional morphology and can be sometimes ambiguous, these characteristics cannot be used to define transitivity in Japhug.

Another distinction between transitive and intransitive verbs is the fact that transitive subjects require ergative marking (§\ref{sec:A.kW}), whereas intransitive subjects are generally in absolutive form (§\ref{sec:absolutive.S}). This is not an absolute criterion to distinguish between transitive and intransitive verbs, however, since some transitive verbs have a dummy subject (§\ref{sec:transitive.dummy}), and since intransitive verbs with ergative marking are attested in specific contexts (§\ref{sec:S.kW}, §\ref{sec:long.distance.kW}).

\subsection{Polypersonal indexation and direction marking} \label{sec:polypersonal}
In Japhug, as in other Gyalrongic languages, transitive verbs index two arguments, the transitive subject and the object.  This section provides a detailed analysis of the transitive indexation paradigm in comparison with the intransitive conjugation.

The transitive conjugation comprises 31 different forms for each TAME category, as listed in \tabref{tab:japhug.tr} on p.\pageref{tab:japhug.tr}. \tabref{tab:mto.paradigm} exemplifies this paradigm using the Factual Non-Past of the verb \japhug{mto}{see}. These two-dimensional tables (and all other such tables in this chapter) list the subjects as rows and the objects as columns. The polypersonal configurations are represented using the notation $A$\fl{}$P$, where $A$ stands for the person and number of the transitive subject, and $P$ for that of the object; for instance \textsc{1sg}\fl{}\textsc{3du} means `first singular subject with third \textsc{du} object'. 3$'$ represents obviative third person, a concept described in more detail in §\ref{sec:indexation.non.local}. The shaded cells 1\fl{}1, 2\fl{}2, 3\fl{}3 and 3$'$\fl{}3$'$ correspond to reflexive forms, which are not transitive in Japhug, and are expressed by deriving an intransitive verb using the prefix \forme{ʑɣɤ-} (§\ref{sec:reflexive}). Generic person (§\ref{sec:indexation.generic.tr}) is not included in \tabref{tab:japhug.tr}.

One of the most fundamental features of the Japhug indexation system is the fact that the affixes found in the intransitive paradigm also occur in the transitive conjugation, and can index either the subject (intransitive S or transitive A) or the object (O) (depending on the presence or absence of direction marking, §\ref{sec:direct-inverse}) as illustrated in \tabref{tab:neutral.alignment} with the suffix \forme{-a} and the prefix \forme{tɯ-}. In other words, the indexation affixes reflect neutral alignment.

\begin{table}[H]
\caption{Neutral alignment} 
 \centering \label{tab:neutral.alignment}
\begin{tabular}{|Xlllll|} 
\lsptoprule
& S & A & O \\
\hline
\textsc{1sg} \forme{-a} & \forme{ɣi-\rouge{a}} & \forme{mtam-\rouge{a}}  & \forme{ɣɯ-mto-\rouge{a}} \\ 
&`I will come' &`I will see it'&  `he will see me' \\
\textsc{2sg} \forme{tɯ-} & \forme{\rouge{tɯ}-ɣi} & \forme{\rouge{tɯ}-mtɤm}  & \forme{\rouge{tɯ́}-wɣ-mto} \\ 
&`You will come' &`You will see it'&  `he will see you' \\
\lspbottomrule
\end{tabular}
\end{table}

When studying indexation systems of this type, it is useful to divide the space of bipersonal indexation into three domains \citep{zuniga06,jacques14inverse}: the \textsc{local} domain (when both subject and object are first or second person), the \textsc{non}-\textsc{local} domain (when both subject and object are third person) and the \textsc{mixed} domain (one argument is first or second person and the other one is third person). \tabref{tab:domain} represents these three domains in blue, red and green, respectively.

\begin{table}[H] 
\caption{The three domains of the transitive paradigm} 
 \centering \label{tab:domain}
\begin{tabular}{|l|lll|} 
\lsptoprule
&1 & 2 &3\\
\hline
1 &\grise{} &1\fl{}2\acell{} & 1\fl{}3 \bcell{} \\
2&2\fl{}1\acell{}&\grise{}&2\fl{}3 \bcell{} \\
3&3\fl{}1 \bcell{}&3\fl{}2 \bcell{}&3\fl{}3\ccell{}\\
\hline
\textsc{intr}&1&2&3\\
\lspbottomrule
\end{tabular}
\end{table}

In the following sections, we first present the forms of the transitive paradigm in each of the three domains, then discuss some specific issues (generic and double number indexation) and then analyze the structure of the Japhug indexation system in a typological perspective.


\begin{landscape}
\begin{table}[H]
\caption{Japhug transitive and intransitive paradigms}\label{tab:japhug.tr}
\resizebox{\columnwidth}{!}{
\begin{tabular}{|l|l|l|l|l|l|l|l|l|l|l|}
\textsc{} & 	\textsc{1sg} & 	  \textsc{1du} & 	\textsc{1pl} & 	\textsc{2sg} & 	\textsc{2du} & 	\textsc{2pl} & 	\textsc{3sg} & 	\textsc{3du} & 	\textsc{3pl} & 	\textsc{3$'$} \\ 	
\hline
\textsc{1sg} & \multicolumn{3}{c|}{\grise{}} &	\forme{} & 	\forme{} & 	\forme{} & 	\forme{\sigc{}-a}   & 	 \forme{\sigc{}-a-ndʑi} & 	 \forme{\sigc{}-a-nɯ} & 	\grise{} \\	
\cline{8-10}
\textsc{1du} & 	\multicolumn{3}{c|}{\grise{}} &	\forme{ta-\siga{}} & 	\forme{ta-\siga{}-ndʑi} & 	\forme{ta-\siga{}-nɯ} & 	\multicolumn{3}{c|}{ \forme{\siga{}-tɕi}}  & 	\grise{} \\	
\cline{8-10}
\textsc{1pl} & 	\multicolumn{3}{c|}{\grise{}} & 	  & 	&  & 	\multicolumn{3}{c|}{ \forme{\siga{}-ji}}  & 	\grise{} \\	
\hline
\textsc{2sg} & 	\forme{kɯ-\siga{}-a} & 	\forme{} & 	\forme{} & 	\multicolumn{3}{c|}{\grise{}}&	\multicolumn{3}{c|}{\forme{tɯ-\sigc{}}} & 	\grise{} \\	
\cline{2-2}
\cline{8-10}
\textsc{2du} & 	\forme{kɯ-\siga{}-a-ndʑi} & 	\forme{kɯ-\siga{}-tɕi} & 	\forme{kɯ-\siga{}-ji} & 	\multicolumn{3}{c|}{\grise{}} &	\multicolumn{3}{c|}{\forme{tɯ-\siga{}-ndʑi}} & 	\grise{} \\	
\cline{2-2}
\cline{8-10}
\textsc{2pl} & 	\forme{kɯ-\siga{}-a-nɯ} & 	\forme{} & 	\forme{} & 	\multicolumn{3}{c|}{\grise{}}&	\multicolumn{3}{c|}{\forme{tɯ-\siga{}-nɯ}} & 	\grise{} \\	
\hline
\textsc{3sg} &  	\forme{wɣɯ́-\siga{}-a} & 	\forme{} & 	\forme{} & 	\forme{} & 	\forme{} & 	\forme{} & \multicolumn{3}{c|}{\grise{}} &	\forme{\sigc{}} \\ 	
\cline{2-2}
\cline{11-11}
\textsc{3du} &  	\forme{wɣɯ́-\siga{}-a-ndʑi} & 	 \forme{wɣɯ́-\siga{}-tɕi} & 		\forme{wɣɯ́-\siga{}-ji} & 	\forme{tɯ́-wɣ-\siga{}} & 	\forme{tɯ́-wɣ-\siga{}-ndʑi} & 	\forme{tɯ́-wɣ-\siga{}-nɯ} & 	\multicolumn{3}{c|}{\grise{}} &	\forme{\siga{}-ndʑi} \\ 
\cline{2-2}	
\cline{11-11}
\textsc{3pl} &  	\forme{wɣɯ́-\siga{}-a-nɯ} & 	\forme{} & 	\forme{} & 	\forme{} & 	\forme{} & 	\forme{} & \multicolumn{3}{c|}{\grise{}} &	\forme{\siga{}-nɯ} \\ 	
\hline
\textsc{3$'$} & 	\multicolumn{6}{c|}{\grise{}} &	\forme{wɣɯ́-\siga{}} & 	\forme{wɣɯ́-\siga{}-ndʑi} & 	\forme{wɣɯ́-\siga{}-nɯ} & 	\grise{} \\	
	\hline	\hline
\textsc{intr}&\forme{\siga{}-a}&\forme{\siga{}-tɕi}&\forme{\siga{}-ji}&\forme{tɯ-\siga{}}&\forme{tɯ-\siga{}-ndʑi}&\forme{tɯ-\siga{}-nɯ}&\forme{\siga{}}&\forme{\siga{}-ndʑi} &\forme{\siga{}-nɯ}& 	\grise{} \\	
\hline
\end{tabular}}
\end{table}


\begin{table}[H]
\caption{The paradigm of the verb \japhug{mto}{see} in the Factual non-past}\label{tab:mto.paradigm}
\resizebox{\columnwidth}{!}{
\begin{tabular}{|l|l|l|l|l|l|l|l|l|l|l|}
\textsc{} & 	\textsc{1sg} & 	  \textsc{1du} & 	\textsc{1pl} & 	\textsc{2sg} & 	\textsc{2du} & 	\textsc{2pl} & 	\textsc{3sg} & 	\textsc{3du} & 	\textsc{3pl} & 	\textsc{3$'$} \\ 	
\hline
\textsc{1sg} & \multicolumn{3}{c|}{\grise{}} &	\forme{} & 	\forme{} & 	\forme{} & 	\forme{mtam-a}   & 	 \forme{mtam-a-ndʑi} & 	 \forme{mtam-a-nɯ} & 	\grise{} \\	
\cline{8-10}
\textsc{1du} & 	\multicolumn{3}{c|}{\grise{}} &	\forme{ta-mto} & 	\forme{ta-mto-ndʑi} & 	\forme{ta-mto-nɯ} & 	\multicolumn{3}{c|}{ \forme{mto-tɕi}}  & 	\grise{} \\	
\cline{8-10}
\textsc{1pl} & 	\multicolumn{3}{c|}{\grise{}} & 	  & 	&  & 	\multicolumn{3}{c|}{ \forme{mto-j}}  & 	\grise{} \\	
\hline
\textsc{2sg} & 	\forme{kɯ-mto-a} & 	\forme{} & 	\forme{} & 	\multicolumn{3}{c|}{\grise{}}&	\multicolumn{3}{c|}{\forme{tɯ-mtɤm}} & 	\grise{} \\	
\cline{2-2}
\cline{8-10}
\textsc{2du} & 	\forme{kɯ-mto-a-ndʑi} & 	\forme{kɯ-mto-tɕi} & 	\forme{kɯ-mto-j} & 	\multicolumn{3}{c|}{\grise{}} &	\multicolumn{3}{c|}{\forme{tɯ-mto-ndʑi}} & 	\grise{} \\	
\cline{2-2}
\cline{8-10}
\textsc{2pl} & 	\forme{kɯ-mto-a-nɯ} & 	\forme{} & 	\forme{} & 	\multicolumn{3}{c|}{\grise{}}&	\multicolumn{3}{c|}{\forme{tɯ-mto-nɯ}} & 	\grise{} \\	
\hline
\textsc{3sg} &  	\forme{wɣɯ́-mto-a} & 	\forme{} & 	\forme{} & 	\forme{} & 	\forme{} & 	\forme{} & \multicolumn{3}{c|}{\grise{}} &	\forme{mtɤm} \\ 	
\cline{2-2}
\cline{11-11}
\textsc{3du} &  	\forme{wɣɯ́-mto-a-ndʑi} & 	 \forme{wɣɯ́-mto-tɕi} & 		\forme{wɣɯ́-mto-j} & 	\forme{tɯ́-wɣ-mto} & 	\forme{tɯ́-wɣ-mto-ndʑi} & 	\forme{tɯ́-wɣ-mto-nɯ} & 	\multicolumn{3}{c|}{\grise{}} &	\forme{mto-ndʑi} \\ 
\cline{2-2}	
\cline{11-11}
\textsc{3pl} &  	\forme{wɣɯ́-mto-a-nɯ} & 	\forme{} & 	\forme{} & 	\forme{} & 	\forme{} & 	\forme{} & \multicolumn{3}{c|}{\grise{}} &	\forme{mto-nɯ} \\ 	
\hline
\textsc{3$'$} & 	\multicolumn{6}{c|}{\grise{}} &	\forme{wɣɯ́-mto} & 	\forme{wɣɯ́-mto-ndʑi} & 	\forme{wɣɯ́-mto-nɯ} & 	\grise{} \\	
\hline
\end{tabular}}
\end{table}
\end{landscape}

\subsubsection{Mixed configurations} \label{sec:indexation.mixed}
In the mixed domain, we have to distinguish between \textit{direct} configurations, where the subject is first or second person and the object is third person (1\fl{}3, 2\fl{}3) and \textit{inverse} configurations, where the opposite holds true (3\fl{}1, 3\fl{}2).\footnote{This terminology will be justified in §\ref{sec:direct-inverse}. }

\tabref{tab:mixed.direct} presents all 1\fl{}3 and 2\fl{}3 forms (except \textsc{1sg}\fl{}3\textsc{du/pl}, which are discussed in §\ref{sec:double.number.indexation}) of \japhug{mto}{see} in the Factual Non-Past and the Aorist, compared with the corresponding forms of the intransitive verb \japhug{ngo}{be ill}. The \textsc{downwards} \forme{pɯ-} and \textsc{upwards} \forme{tɤ-} preverbs are lexically selected by these verbs (§\ref{sec:preverb.perception}).

\begin{table}
\caption{Mixed domain (direct forms) compared with the intransitive paradigm} \label{tab:mixed.direct}
\begin{tabular}{|l|ll|lll|}
\lsptoprule
&\multicolumn{2}{c}{Transitive}&\multicolumn{2}{c}{Intransitive}& \\
Person&Non-Past & Past  & Non-Past & Past & \\
\hline
\textsc{1sg}(\fl{}3\textsc{sg}) & \forme{\rouge{mtam}-a} & \forme{pɯ-mto-\rouge{t}-a} &\forme{ngo-a} & \forme{tɤ-ngo-a}  &\\
\textsc{1du}(\fl{}3) & \forme{mto-tɕi} & \forme{pɯ-mto-tɕi} & \forme{ngo-tɕi} & \forme{tɤ-ngo-tɕi} & \\
\textsc{1pl}(\fl{}3) & \forme{mto-j} & \forme{pɯ-mto-j} & \forme{ngo-j} & \forme{tɤ-ngo-j}&  \\
\hline 
\textsc{2sg}(\fl{}3) &\forme{tɯ-\rouge{mtɤm}} & \forme{pɯ-tɯ-mto-\rouge{t}} & \forme{tɯ-ngo} & \forme{tɤ-tɯ-ngo}&  \\
\textsc{2du}(\fl{}3) & \forme{tɯ-mto-ndʑi} & \forme{pɯ-tɯ-mto-ndʑi} & \forme{tɯ-ngo-ndʑi} & \forme{tɤ-tɯ-ngo-ndʑi} & \\
\textsc{2pl}(\fl{}3) & \forme{tɯ-mto-nɯ} & \forme{pɯ-tɯ-mto-nɯ} & \forme{tɯ-ngo-nɯ} & \forme{tɤ-tɯ-ngo-nɯ} & \\
\lspbottomrule
\end{tabular}
\end{table}

With the exception of  \textsc{1sg}\fl{}3 forms (on which see §\ref{sec:double.number.indexation}), the number of the third person argument is never expressed in the mixed domain. Examples (\ref{ex:kAfstWni}) and (\ref{ex:ni.pjWsati}) illustrate \textsc{1pl}\fl{}\textsc{3sg} and \textsc{1pl}\fl{}\textsc{3du} forms, respectively. Whether the third person object is singular or dual, person indexation is restricted to the \textsc{1pl} \forme{-ji} suffix in both cases. The same is true of the \forme{-tɕi} suffix, and also in inverse configurations.

\begin{exe}
\ex   \label{ex:kAfstWni}
\gll tɕe iʑo ji-kʰa kɯnɤ kɤ-fstɯn-i pɯ-ra \\
\textsc{lnk} \textsc{1pl} \textsc{1pl}.\textsc{poss}-house also \textsc{aor}-serve-\textsc{1pl} \textsc{pst}.\textsc{ipfv}-be.neededo \\
\glt `We also had to take care of him in our house.' (14-siblings, 360)
\end{exe}

\begin{exe}
\ex   \label{ex:ni.pjWsati}
\gll   nɤ-wɯ cʰo nɤ-ʁi ni pjɯ-sat-i ŋu \\
\textsc{2sg}.\textsc{poss}-grandfather \textsc{comit} \textsc{2sg}.\textsc{poss}-younger.sibling \textsc{du} \textsc{ipfv}-kill-\textsc{1pl} be:\textsc{fact} \\
\glt `We will kill your grandfather and your brother.' (2011-05-nyima, 132)
\end{exe}

Although the indexation suffixes \forme{-ndʑi} and \forme{-nɯ} are the same for second and third person (§\ref{sec:intr.23}), in 2\fl{}3 and 3\fl{}2 configurations, they can only index the number of the second person argument, never that of the third person. For instance, in (\ref{ex:ra.totWGWt}), \forme{to-tɯ-ɣɯt} is \textsc{2sg}\fl{}\textsc{3pl}, and adding a plural \forme{-nɯ} here would change the meaning to `you$_{pl}$ brought people' and be incompatible with the singular pronoun \forme{nɤʑo}.

\begin{exe}
\ex   \label{ex:ra.totWGWt}
\gll    nɤʑo tɯrme ra to-tɯ-ɣɯt tɕe, \\
\textsc{2sg} people \textsc{pl} \textsc{ifr}:\textsc{up}-2-bring \textsc{lnk} \\
\glt `You(\textsc{sg}) brought people (here).' (150901 changfamei, 155)
\end{exe}

The direct forms of the transitive paradigm are nearly all identical to the corresponding intransitive paradigm. The \textsc{1sg}\fl{}3 and \textsc{2sg}{}\fl{}3 forms are the only ones that show morphological features absent from the corresponding intransitive paradigm: the Stem III alternation in the Factual Non-Past (and other tenses, §\ref{sec:stem3.distribution}, §\ref{sec:stem.TAME})\footnote{Note in addition the allomorph \forme{mtam-} of stem III when followed by the \textit{1sg} suffix \ipa{-a}, following regular vowel assimilation (§\ref{sec:intr.1}). 
} and the \forme{-t} suffix in the Past. These features are highlighted in red in \tabref{tab:mixed.direct}. Most dialects of Japhug, including those of Gdongbrgyad township, have \forme{-z} instead of \forme{-t} as Past suffix (§\ref{sec:suffixes}).

Stem III alternation and the \forme{-t} suffix mark at the same time TAME (§\ref{sec:other.TAME}, §\ref{sec:suffixes}) and person of both subject and object, and must be considered to be an integral part of the person indexation system.

Not all transitive verbs present these two features, however; in particular, verb with close syllable stems lack both of them. Stem alternation is restricted to a few open-syllable stem types (\forme{-o}, \forme{-a}, \forme{-u} and \forme{-ɯ}, §\ref{sec:stem3.form}), and the \forme{-t} suffix cannot surface in close syllable stems. For instance, the Aorist \textsc{1sg}\fl{}\textsc{3sg} of the verbs \japhug{joʁ}{raise} and \japhug{ɕlɯɣ}{drop} are \forme{tɤ-joʁ-a} `I raised it' and \forme{pɯ-ɕlɯɣ-a} `I dropped it', not the completely incorrect $\dagger$\forme{tɤ-joʁ-t-a} or $\dagger$\forme{pɯ-ɕlɯɣ-t-a} (with automatic regressive devoicing of \ipa{ʁ} and \ipa{ɣ} before \ipa{t}, §\ref{sec:codas.inventory}). These incorrect forms  would not violate Japhug phonotactics, since clusters such as \forme{-χt-} and \forme{-xt-} are well attested (§\ref{sec:xC.clusters}, §\ref{sec:XC.clusters}), showing that the rules governing the use of the \forme{-t} suffix are not purely phonological.

The direct forms of transitive verbs with closed syllable stems (such as \japhug{joʁ}{raise} and \japhug{ɕlɯɣ}{drop}) in the mixed domain are thus identical to that of intransitive verbs.

The inverse forms of the mixed domain of the verb \japhug{mto}{see} are presented in \tabref{tab:mixed.inverse}. All forms in this section of the paradigm, regardless of the TAME category, take the prefix \forme{ɣɯ́-/wɣ\trt}, whose distribution and allomorphy (\forme{ɣɯ́-} vs. \forme{wɣ-}) is discussed in more detail in §\ref{sec:allomorphy.inv}.

\begin{table}
\caption{Mixed domain (inverse forms)} \label{tab:mixed.inverse}
\begin{tabular}{|l|ll|}
\lsptoprule
Person&Non-Past & Past   \\
\hline
3\textsc{sg}\fl{}\textsc{1sg} & \forme{ɣɯ́-mto-a} & \forme{pɯ́-wɣ-mto-a}  \\
3\fl{}\textsc{1du} & \forme{ɣɯ́-mto-tɕi} & \forme{pɯ́-wɣ-mto-tɕi}  \\
3\fl{}\textsc{1pl} & \forme{ɣɯ́-mto-j} & \forme{pɯ́-wɣ-mto-j}  \\
\hline 
3\fl{}\textsc{2sg} & \forme{tɯ́-wɣ-mto} & \forme{pɯ-tɯ́-wɣ-mto}  \\
3\fl{}\textsc{2du} & \forme{tɯ́-wɣ-mto-ndʑi} & \forme{pɯ-tɯ́-wɣ-mto-ndʑi}  \\
3\fl{}\textsc{2pl} & \forme{tɯ́-wɣ-mto-nɯ} & \forme{pɯ-tɯ́-wɣ-mto-nɯ}  \\
\lspbottomrule
\end{tabular}
\end{table}

Aside from the prefix \forme{ɣɯ́-/wɣ\trt}, the inverse forms in the mixed domain present the same affixes as those of the corresponding intransitive forms (except in the case of double number indexation, treated in §\ref{sec:double.number.indexation}), and lack stem alternation (only stem I occurs).

The imperative (§\ref{sec:imp.morphology}) is only attested in the direct mixed 2\fl{}3 configurations, and is the only finite form involving a second person that neither takes the second person prefix nor a portmanteau prefix. To express 2\fl{}1 imperative, the imperfective is used instead (§\ref{sec:ipfv.hortative}).

\subsubsection{Non-local configurations} \label{sec:indexation.non.local}
\tabref{tab:non.local} presents the Non-local domain of the paradigm of \japhug{mto}{see} in the Factual Non-Past and the Aorist, compared with the intransitive paradigm as exemplified by \japhug{ngo}{be ill}.

\begin{table}
\caption{Non-local domain compared with the intransitive paradigm} \label{tab:non.local}
\begin{tabular}{|l|ll|lll|}
\lsptoprule
&\multicolumn{2}{c}{Transitive}&\multicolumn{2}{c}{Intransitive} &\\
Person&Non-Past & Past  & Non-Past & Past \\
\hline
\textsc{3sg}(\fl{}3$'$) & \forme{\rouge{mtɤm}} & \forme{\rouge{pa}-mto} &\forme{ngo} & \forme{tɤ-ngo}&  \\
\textsc{3du}(\fl{}3$'$) & \forme{mto-ndʑi} & \forme{\rouge{pa}-mto-ndʑi} & \forme{ngo-ndʑi} & \forme{tɤ-ngo-ndʑi} & \\
\textsc{3pl}(\fl{}3$'$) & \forme{mto-nɯ} & \forme{\rouge{pa}-mto-nɯ} & \forme{ngo-nɯ} & \forme{tɤ-ngo-nɯ}  &\\
\hline 
3$'$\fl{}\textsc{3sg} &\forme{ɣɯ́-mto} & \forme{pɯ́-wɣ-mto} &  \\
3$'$\fl{}\textsc{3du} & \forme{ɣɯ́-mto-ndʑi} & \forme{pɯ́-wɣ-mto-ndʑi} &   \\
3$'$\fl{}\textsc{3pl} & \forme{ɣɯ́-mto-nɯ} & \forme{pɯ́-wɣ-mto-nɯ} & \\
\lspbottomrule
\end{tabular}
\end{table}

There are two types of non-local forms: those taking the \forme{ɣɯ́-/wɣ-} prefix, the \textit{inverse} configurations (§\ref{sec:direct-inverse}), and those without it, the \textit{direct} configurations. 

Direct forms present two features distinguishing them from the corresponding third person intransitive forms. First, in non-past tenses such as the Factual Non-Past, the \textsc{3sg} subject form has Stem III (like the \textsc{1sg}\fl{}3 and \textsc{2sg}\fl{}3 forms of the mixed domain, §\ref{sec:indexation.mixed}). Second, in the Aorist, a C-type orientation preverb, with \forme{-a-} vocalism is used instead of the A-type orientation preverbs in \forme{-ɯ-} and \forme{-ɤ-} found in the mixed and local domains (\forme{pɯ-} in the case of the verb \japhug{mto}{see}). C-type orientation preverbs, only found in this section of the transitive paradigm (§\ref{sec:transitivity.morphology}, §\ref{sec:kamnyu.preverbs}), result from the fusion of A-type prefixes with another prefix which is only otherwise attested in the Apprehensive (§\ref{sec:apprehensive}). 

The \forme{-t} suffix found in some direct forms in the mixed domain (§\ref{sec:indexation.mixed}) is not attested in the non-local domain.

Inverse forms of the non-local domain only differ from the third person intransitive forms by the presence of the \forme{ɣɯ́-/wɣ-} prefix, lacking stem alternation or additional affixes, like the inverse forms of the mixed domain (§\ref{sec:indexation.mixed}).

Number indexation in the non-local domain encodes only one of the two arguments: the \textit{subject} in the direct configurations and the \textit{object} in the inverse configurations. For instance, in (\ref{ex:pjAwGnAsmandZi}), the subject of the verb \forme{pjɤ́-wɣ-nɤsma-ndʑi} is plural (\textsc{3pl}\fl{}\textsc{3du}), but plural indexation \forme{-nɯ} here instead of the dual would be incorrect, as this verb form has the \forme{ɣɯ́-/wɣ-} prefix and thus agrees in number with the object.

\begin{exe}
\ex   \label{ex:pjAwGnAsmandZi}
\gll ndʑi-jɯlco ra kɯ wuma pjɤ́-wɣ-nɤsma-ndʑi. \\
\textsc{3du}.\textsc{poss}-neighbour \textsc{pl} \textsc{erg} really \textsc{ifr}-\textsc{inv}-envy-\textsc{du} \\
\glt `Their neighbours envied the two of them.' (qajdoskAt, 14)
\end{exe}

The third person argument whose number is indexed (the \textsc{3du} argument in \ref{ex:pjAwGnAsmandZi}) is called \textit{proximate}, and the one that is not indexed on the verb (corresponding to the noun phrase \forme{ndʑi-jɯlco ra} in \ref{ex:pjAwGnAsmandZi}) is called \textit{obviative} (glossed as 3$'$), using terminology from Algonquian linguistics. While in Algonquian the term \textit{obviative} (coined by \citealt{cuoq1866}) originally refers to a category marked on both nouns and verb indexation (including intransitive verbs), in Gyalrong languages the proximate/obviative contrast is only reflected in transitive verbal morphology (see §\ref{sec:direct-inverse} and §\ref{sec:inverse.3.3.saliency})

Direct configurations are by far more common in the corpus than inverse ones. Inverse non-local forms have two functions: marking the relative saliency of the subject and the object (a question detailed in §\ref{sec:inverse.3.3.saliency}) and indexing a generic subject (§\ref{sec:indexation.generic.tr}).

\subsubsection{Local configurations} \label{sec:indexation.local}
Local configurations stand out in Japhug verbal paradigms in being the only forms involving the second person without a \forme{tɯ-} prefix. Instead, synchronically unanalyzable portmanteau prefixes are found: \forme{ta-} for 1\fl{}2, and \forme{kɯ-} for 2\fl{}1. The \forme{ta-} co-occurs with the non-first person dual and plural suffixes (\forme{-ndʑi} and \forme{-nɯ}, §\ref{sec:intr.23}), and \forme{kɯ-} with first person suffixes (\forme{-a}, \forme{-tɕi} and \forme{-ji}, §\ref{sec:intr.1}), indexing in all cases the person and number of the object. In 2\fl{}1 forms, the first person is redundantly indexed both by the suffixes and the portmanteau prefix \forme{kɯ-}. 

The presence of portmanteau prefixes in the local domain is not typologically unusual. Typologists have long noticed that languages with polypersonal indexation tend to have unanalysable affixes in 1\fl{}2 and 2\fl{}1 forms \citep{heath98skewing} in part due to pragmatic factors \citep{delancey18sociopragmatic}. The historical origin of these prefixes is discussed in \citet{jacques18generic} and §\ref{sec:portmanteau.prefixes.history}.

In 1\fl{}2 configurations, since \textsc{2sg} is exclusively indexed by the \forme{tɯ-} prefix in the intransitive paradigm, without any suffix (unlike Situ, §\ref{sec:indexation.suffixes.history}), the 1\fl{}\textsc{2sg} also lacks any indexation suffix. \tabref{tab:local.paradigm} presents all local configurations of the verb \japhug{mto}{see} in the Factual Non-Past and the Aorist, except for those with double suffixation (\textsc{2du}\fl{}\textsc{1sg} and \textsc{2pl}\fl{}\textsc{1sg}) which are treated in §\ref{sec:double.number.indexation}. 

There is no stem alternation, \forme{-t} past tense suffix or \forme{ɣɯ-/wɣ-} prefix in the local domain in Japhug. Aorist and Factual Non-Past only differ from each other by the presence of the A-type orientation preverb in the former, as can be seen in \tabref{tab:local.paradigm}.

\begin{table}
\caption{Local domain} \label{tab:local.paradigm}
\begin{tabular}{|l|lll|}
\lsptoprule
Person&Non-Past & Past&  \\
\hline
1\fl{}\textsc{2sg} &\forme{ta-mto} & \forme{pɯ-ta-mto} &  \\
1\fl{}\textsc{2du} & \forme{ta-mto-ndʑi} & \forme{pɯ-ta-mto-ndʑi} &   \\
1\fl{}\textsc{2pl} & \forme{ta-mto-nɯ} & \forme{pɯ-ta-mto-nɯ} & \\
\hline
\textsc{2sg}\fl{}\textsc{1sg} & \forme{kɯ-mto-a} & \forme{pɯ-kɯ-mto-a} & \\
2\fl{}\textsc{1du} & \forme{kɯ-mto-tɕi} & \forme{pɯ-kɯ-mto-tɕi} & \\
2\fl{}\textsc{1pl} & \forme{kɯ-mto-j} & \forme{pɯ-kɯ-mto-j} & \\
\lspbottomrule
\end{tabular}
\end{table}

In Japhug, the forms of the local domain are always different from those of the mixed domain, and the person of the subject and the object is never ambiguous. In this regard, Japhug differs from many languages of the Trans-Himalayan family, in particular those of the Kiranti branch. In Khaling, for instance, the same forms are used for 2\fl{}1 and 3\fl{}1 configurations on the one hand, and for 3\fl{}2 and \textsc{1nsg}\fl{}2 on the other hand \citep{jacques12khaling}. In that language, the only local configuration to have specific unambiguous forms is \textsc{1sg}\fl{}2. 

In Japhug, while person is unambiguously expressed in the local domain, only the number of the \textit{object} is specified, with the exception of 2\fl{}\textsc{1sg} configurations (§\ref{sec:double.number.indexation}). Examples (\ref{ex:aZo.tutafsraN}) and (\ref{ex:iZora.tutafsraN}) illustrate the same form \forme{tu-ta-fsraŋ} meaning in the first case \textsc{1sg}\fl{}\textsc{2sg} `I will save you$_{sg}$' and in the second one \textsc{1pl}\fl{}\textsc{2sg} `We will save you$_{sg}$', showing that the form remains identical regardless of the number of the subject.

\begin{exe}
\ex   \label{ex:aZo.tutafsraN}
\gll  tɕe aʑo tu-ta-fsraŋ ra tɕe, \\
\textsc{lnk} \textsc{1sg}  \textsc{ipfv}-1\fl{}2-protect be.needed:\textsc{fact} \textsc{lnk} \\
\glt `I have to save you.' (150901 changfamei-zh, 219)
\end{exe}

\begin{exe}
\ex   \label{ex:iZora.tutafsraN}
\gll  iʑora nɯnɯ koŋla ʑo tu-ta-fsraŋ, tu-ta-βri ɲɯ-sɯso-j ɕti ri, \\
\textsc{1pl} \textsc{dem} really \textsc{emph} \textsc{ipfv}-1\fl{}2-protect \textsc{ipfv}-1\fl{}2-save \textsc{ipfv}-think-\textsc{1pl} be.\textsc{aff}:\textsc{fact} \textsc{lnk} \\ 
\glt `We really want to save you.' (niulan li de lu-zh, 25)
\end{exe}

Similarly, (\ref{ex:tukWquri.ra}) and (\ref{ex:GWtukWquri}) show the form \forme{tu-kɯ-qur-i} meaning \textsc{2sg}\fl{}\textsc{1pl} `You$_{sg}$ help us' in the first example and \textsc{2pl}\fl{}\textsc{1pl} `You$_{pl}$ help us' in the second one.

\begin{exe}
\ex   \label{ex:tukWquri.ra}
\gll   wortɕʰi wojɤr ʑo tu-kɯ-qur-i ra \\
please please \textsc{emph} \textsc{ipfv}-2\fl{}1-help-\textsc{1pl} be.needed:\textsc{fact} \\
\glt `Please, help us.' (150827 taisui-zh, 22)
\end{exe}

\begin{exe}
\ex   \label{ex:GWtukWquri}
\gll  <chuhuaiwang> cʰo nɯra, kɯmaʁ nɯra nɯ-ɕki, ``ɣɯ-tu-kɯ-qur-i ɲɯ-ntsʰi" z-jo-sɯ-ti. \\
\textsc{anthr} \textsc{comit} \textsc{dem}:\textsc{pl} other \textsc{dem}:\textsc{pl} \textsc{3pl}.\textsc{poss}-\textsc{dat} \textsc{cisl}-\textsc{ipfv}-2\fl{}\textsc{1pl} \textsc{sens}-be.better \textsc{tral}-\textsc{ifr}-\textsc{caus}-say \\
\glt `He sent (someone) to the king Huai of Chu and the other ones to tell them `Come and help us'. (160721 pofuchenzhou-zh, 26)
\end{exe}

The 2\fl{}1 configuration can occur in the prohibitive with the prefix \forme{ma-} (§\ref{sec:neg.allomorphs}) as in the form \forme{ma-tʰɯ-kɯ-βlɯ-a} `don't burn me' in (\ref{ex:mathWkWBlWa}), but not in the imperative (§\ref{sec:imp.morphology}). Instead, the Imperfective is used (§\ref{sec:ipfv.hortative}), most often with a modal verb, as in (\ref{ex:tukWquri.ra}) and (\ref{ex:GWtukWquri}) and in the form \forme{cʰɯ-kɯ-rku-a} in (\ref{ex:mathWkWBlWa}). Attempts to produce 2\fl{}1 Imperative forms such as $\dagger$\forme{tɤ-kɯ-qur-i} (instead of the correct \forme{tu-kɯ-qur-i ra} `help us') are rejected by native speakers.

\begin{exe}
\ex   \label{ex:mathWkWBlWa}
\gll ma-tʰɯ-kɯ-βlɯ-a, tɕʰorzi ɯ-ŋgɯ cʰɯ-kɯ-rku-a \\
\textsc{prohib}-\textsc{imp}-2\fl{}1-burn-\textsc{1sg} wine.jar \textsc{3sg}.\textsc{poss}-inside  \textsc{ipfv}:\textsc{downstream}-2\fl{}1-put.in-\textsc{1sg} \\
\glt `Don't burn me, put me in a wine jar.' (2003 Kunbzang, 385)
\end{exe}

\subsubsection{Inclusive semi-reflexive configurations} \label{sec:incl.semi.reflexive}
Japhug lacks inclusive / exclusive contrast in both pronouns and indexation system (§\ref{sec:intr.1}), and inclusive persons are treated the same way as \textsc{1du} and \textsc{1pl} exclusive. While inclusive persons could in principle exist in the local domain, configuration of this type are problematic: if the subject or object of a transitive verb is inclusive, and the other argument strictly first or second person, the resulting configuration is partially reflexive, since the inclusive contains both first and person referents. 

In a language like Japhug where reflexivity is marked by an intransitivizing derivation (§\ref{sec:reflexive}), there is therefore a conflict between the absence of reflexive forms in the transitive paradigm and the need to express inclusive$\leftrightarrow$first/second person configurations in a way that is different from plain reflexives.

The following list provides the four theoretically possible inclusive semi-reflex\-ive configurations, displaying the referent shared by subject and object in red. This list neglects possible additional third person referents in \textsc{1du} exclusive, \textsc{1pl}, \textsc{2du} and \textsc{3pl} arguments, which would artificially increase the number of configurations.

\begin{itemize}
\item \rouge{1}+2\fl{}\rouge{1} `You and I \textbf{verb} me'
\item 1+\rouge{2}\fl{}\rouge{2} `You and I \textbf{verb} you'
\item \rouge{1}\fl{}\rouge{1}+2 `I \textbf{verb} you and me'
\item \rouge{2}\fl{}1+\rouge{2} `You \textbf{verb} you and me'
\end{itemize}

\citet{driem90hayu}, in his review of \citet{michailovsky88}, argues that Kiranti languages and Limbu in particular cannot express inclusive semi-reflexive configurations using transitive verbal morphology, and must resort to periphrases; for instance, in order to express the meaning of the \rouge{2}\fl{}1+\rouge{2} configuration `you saw both of us in the mirror', Limbu uses a complement clause containing an intransitive verb meaning `you and I appear in the mirror', object of the transitive verb `see', as in (\ref{ex:adhaapsiba}).

\begin{exe}
\ex   \label{ex:adhaapsiba}
\gll  khɛnɛʔ  anchi aina-o a-dha:p-si-ba kɛ-ni \\
\textsc{2sg} \textsc{1di} mirror-\textsc{loc} \textsc{incl}-be.visible-\textsc{du}-\textsc{nmlz} 2-see \\
\glt `You(sg) saw both of us in the mirror.’ (\citealt[277]{driem90hayu})
\end{exe}

In Japhug, a similar strategy (though with a finite clause instead of a nominalized verb form) is employed to express the \rouge{1}\fl{}\rouge{1}+2 configuration in (\ref{ex:kAntChArtCi}), with the finite complement clause \forme{χɕɤlzgoŋ ɯ-ŋgɯ kɤ-ntɕʰɤr-tɕi} `we appear in the mirror' as object of the transitive perception verb \japhug{mto}{see}.

\begin{exe}
\ex   \label{ex:kAntChArtCi}
\gll  [χɕɤlzgoŋ ɯ-ŋgɯ kɤ-ntɕʰɤr-tɕi] nɯra pɯ-mto-t-a \\
mirror \textsc{3sg}.\textsc{poss}-in \textsc{aor}-appear-\textsc{1du} \textsc{dem}:\textsc{pl} \textsc{aor}-see-\textsc{pst}:\textsc{tr}-\textsc{1sg} \\
\glt `I saw both of us in the mirror.' (elicitation, \citealt[85]{jacques12agreement})
\end{exe}

Nevertheless, there are cases in Japhug where inclusive semi-reflexive meanings can be expressed by simple verb forms of the local domain. In example (\ref{ex:tha.kWzmaqhutCi}), the verb \forme{kɯ-z-maqʰu-tɕi} presents a 2\fl{}\textsc{1du} configuration (§\ref{sec:indexation.local}); it is clear in this particular case that the object is first dual inclusive `you and I' rather than first exclusive, and that we therefore have a semi-reflexive configuration \rouge{2}\fl{}1+\rouge{2}.

\begin{exe}
\ex   \label{ex:tha.kWzmaqhutCi}
\gll   mɯ́j-tɯ-mbɣom ri tʰa kɯ-z-maqʰu-tɕi \\
\textsc{neg}:\textsc{sens}-2-be.in.a.hurry \textsc{lnk} later 2\fl{}1-\textsc{caus}-be.after:\textsc{fact}-\textsc{1du} \\
\glt `(If) you don't hurry (up), you will get us late.' (elicited)
\end{exe}

No example of this type is found in the corpus, and  such pragmatically clumsy configurations are on the borderline of the  Japhug person indexation system.

Semi-reflexive indexation also occurs with third person referents, in particular in causative constructions (see \ref{ex:toWGsWndzandZi.pjAra}, §\ref{sec:sig.caus.morphosyntax}).
 
\subsubsection{Generic indexation} \label{sec:indexation.generic.tr}
The prefix \forme{kɯ\trt}, which appears in the intransitive paradigm to express generic intransitive subject (§\ref{sec:intr.23}), is also attested in the transitive paradigm to refer to generic object. For instance, in example (\ref{ex:amAjAkWphGo.kWmto.kWndo}),\footnote{The generic person in (\ref{ex:amAjAkWphGo.kWmto.kWndo}) is translated into English by the second person `you'. } the transitive verbs \forme{kɯ-mto} `(the yeti) will see one' and \forme{kɯ-ndo} `(the yeti) will catch one' have the same \forme{kɯ-} prefix as the intransitive verb \forme{a-mɤ-jɤ-kɯ-phɣo} `one should not flee'. This example also shows that stem I is selected in the 3\fl{}\textsc{genr} form, as the stem III of \japhug{mto}{see} and \japhug{ndo}{catch} are \forme{mtɤm} and \forme{ndɤm}, respectively. Combining the generic \forme{kɯ-} prefix with stem III is impossible, and \forme{kɯ-mto} cannot be replaced by a form such as $\dagger$\forme{kɯ-mtɤm}, which would be unintelligible.

\begin{exe}
\ex   \label{ex:amAjAkWphGo.kWmto.kWndo}
\gll ɯ-qʰu-cʰu ɯ-stu ʑo a-mɤ-jɤ-kɯ-phɣo ra ma tɕe kɯ-mto tɕe kɯ-ndo ɕti tu-ti-nɯ \\
\textsc{3sg}.\textsc{poss}-behind-\textsc{approx}.\textsc{loc}  \textsc{3sg}.\textsc{poss}-direction \textsc{emph} \textsc{irr}-\textsc{neg}-\textsc{pfv}-\textsc{genr}:S/O-flee be.needed:\textsc{fact} \textsc{lnk} \textsc{lnk}
\textsc{genr}:S/O-see:\textsc{fact} \textsc{lnk} \textsc{genr}:S/O-take:\textsc{fact} be.\textsc{aff}:\textsc{fact} \textsc{ipfv}-say-\textsc{pl} \\
\glt `People say that you should not flee in the direction behind it (the yeti), as it would see you and catch you.' (140510 mYWrgAt, 20)
\end{exe}

Unlike in Tshobdun \citep{sun14generic}, generic transitive subjects in Japhug are not indexed by the same prefix as generic intransitive subjects. Apart from a handful of irregular verbs (§\ref{sec:irregular.transitive}), the inverse prefix \forme{wɣ-} occurs instead of \forme{kɯ-} to express generic transitive subject, as in the verb \forme{tú-wɣ-ndza} `one eats it' in (\ref{ex:YWkWznWtWfCAl}), whose object is definite (anaphorically referring to the noun \japhug{kʰɯrwum}{mold} in a previous clause). Note that the generic subject of \forme{tú-wɣ-ndza} is co-referent with the object of the verb  \forme{ɲɯ-kɯ-z-nɯtɯfɕɤl} `it causes one to have diarrhea' (causative of the intransitive verb \japhug{nɯtɯfɕɤl}{have diarrhea}, on which see §\ref{sec:denominalization.action.nominal}) in the following clause, indexed with the \forme{kɯ-} as in (\ref{ex:amAjAkWphGo.kWmto.kWndo}) above, and that conversely the (inanimate) object of \forme{tú-wɣ-ndza} corresponds to the subject of \forme{ɲɯ-kɯ-z-nɯtɯfɕɤl}.

\begin{exe}
\ex   \label{ex:YWkWznWtWfCAl}
\gll  ma tú-wɣ-ndza tɕe, ɲɯ-kɯ-z-nɯtɯfɕɤl ɕti\\
\textsc{lnk} \textsc{ipfv}-\textsc{inv}-eat \textsc{lnk} \textsc{ipfv}-\textsc{genr}:S/O-\textsc{caus}-have.diarrhea be.\textsc{aff}:\textsc{fact}\\
\glt `If you eat (mold), it causes you diarrhea.' (20-sWrna, 56)
\end{exe}

Generic person indexation is remarkable in Japhug morphosyntax in being one of the very few examples of ergative-absolutive alignment outside of the case marking system, since the intransitive subject and the object are marked by the same prefix \forme{kɯ\trt}, while the transitive subject is not ($S=P \ne A$).

A generic person subject or object is only compatible with a third person argument (\textsc{genr}\fl{}3 or 3\fl{}\textsc{genr}). Combinations with first or second persons are not possible. However, number indexation of the non-generic third person argument is possible when the generic argument is subject. Dual or plural suffixes in \textsc{genr}\fl{}\textsc{du} or \textsc{genr}\fl{}\textsc{pl} configurations are attested in procedural texts. For instance, (\ref{ex:YWwGznWNGAtndZi}) has dual indexation (referring to the turnip leaves and the turnip root) with generic human subject.

\begin{exe}
\ex \label{ex:YWwGznWNGAtndZi}
\gll rasti cʰo rɤjndoʁ ni, pjɯ́-wɣ-ʁndzɤr-ndʑi tɕe ɲɯ́-wɣ-z-nɯɴɢɤt-ndʑi ŋu. \\
turnip \textsc{comit} turnip.root \textsc{du} \textsc{ipfv}-\textsc{inv}-cut-\textsc{du} \textsc{lnk} \textsc{ipfv}-\textsc{inv}-\textsc{caus}-\textsc{acaus}:separate-\textsc{du} be:\textsc{fact}\\
\glt `One separates the turnip from its root by cutting them.' (150903 kAJar, 7)
\end{exe}

Number indexation occurs in particular in the case of generic subject indexation referring to a first person (§\ref{sec:1.genr}). In (\ref{ex:tuwGqurnW}), the generic subject of the verb \forme{tú-wɣ-qur-nɯ} corresponds to the first person plural, as indexed on the preceding verb \forme{rɤʑi-j}, and the plural object is overtly indexed.

\begin{exe}
\ex \label{ex:tuwGqurnW}
\gll tɯ-rdoʁ tsa rɤʑi-j tɕe tú-wɣ-qur-nɯ raʁmaʁ \\
one-piece a.little stay:\textsc{fact}-\textsc{1pl} \textsc{lnk} \textsc{ipfv}-\textsc{inv}-help-\textsc{pl} \textsc{sfp} \\
\glt `Let at least one of us stay here and help them!' (hist180503 xiyouji 12-zh, 78)
\end{exe}

On the other hand, when the generic argument is in object or intransitive subject function, no person indexation suffix can appear on the verb, even if this argument is realized as an overt noun phrase with plural marking in addition to generic indexation.

For instance, in (\ref{ex:Wku.kukWsWndo}), the verb form with generic object indexation \forme{ku-kɯ-sɯ-ndo} lacks any indexation suffix, although both the causer (`the elders') and the causee (the generic argument, \forme{tɤ-pɤtso ra} `us, the children') are plural.

\begin{exe}
\ex \label{ex:Wku.kukWsWndo}
\gll qaʑo cʰɯ-krɤɣ-nɯ tɕe tɤ-pɤtso ra kɯ nɯ qaʑo ɣɯ ɯ-ku nɯ ku-kɯ-sɯ-ndo. \\
sheep \textsc{ipfv}-shear-\textsc{pl} \textsc{lnk} \textsc{indef}.\textsc{poss}-child \textsc{pl} \textsc{erg} \textsc{dem} sheep \textsc{gen} \textsc{3sg}.\textsc{poss}-head \textsc{dem} \textsc{ipfv}-\textsc{genr}:S/O-\textsc{caus}-take \\
\glt `(Every time the adults) sheared the sheep's wool, they would ask the children (us) to grab the sheep's head.' (160712 smAG, 2)
\end{exe}

\subsubsection{Double number indexation}  \label{sec:double.number.indexation}
The transitive paradigm contains six doubly suffixed forms, two in the local domain, and four in the mixed domain, as summarized in \tabref{tab:double.indexation}. Japhug is not the only Gyalrong language with double number indexation. The same set of doubly suffixed forms is found in Tshobdun \citep{jackson02rentongdengdi} and Zbu \citep{gongxun14agreement}.

\begin{table}
\caption{Double number indexation in the transitive paradigm} \label{tab:double.indexation}
\begin{tabular}{|l|ll|}
\lsptoprule
Person&Non-Past & Past  \\
\hline
\textsc{1sg}\fl{}\textsc{3du} &\forme{mtam-a-ndʑi} & \forme{pɯ-mto-t-a-ndʑi}  \\
\textsc{1sg}\fl{}\textsc{3pl} & \forme{mtam-a-nɯ} & \forme{pɯ-mto-t-a-nɯ} \\
\hline
\textsc{3du}\fl{}\textsc{1sg} & \forme{ɣɯ-mto-a-ndʑi} & \forme{pɯ́-wɣ-mto-a-ndʑi}  \\
\textsc{3pl}\fl{}\textsc{1sg} & \forme{ɣɯ-mto-a-nɯ} & \forme{pɯ-wɣ-mto-a-nɯ}  \\
\hline
\textsc{2du}\fl{}\textsc{1sg} & \forme{kɯ-mto-a-ndʑi} & \forme{pɯ-kɯ-mto-a-ndʑi}  \\
\textsc{2pl}\fl{}\textsc{1sg} & \forme{kɯ-mto-a-nɯ} & \forme{pɯ-kɯ-mto-a-nɯ}  \\
\lspbottomrule
\end{tabular}
\end{table}

The forms corresponding to those in \tabref{tab:double.indexation} without additional \forme{-ndʑi} or \forme{-nɯ} suffix generally have a singular third or second person argument, for instance \forme{pɯ-mto-t-a} `I saw him' (§\ref{sec:indexation.mixed}) or \forme{pɯ-kɯ-mto-a} `You$_{sg}$ saw me.' (§\ref{sec:indexation.local}). There are however also cases of optional number indexation, a topic discussed in §\ref{sec:optional.indexation}. 

The \textsc{1sg}\fl{}\textsc{3du} and \textsc{1sg}\fl{}\textsc{3pl} forms have stem III alternation in non-past tenses (\forme{mto} \fl{} \forme{-mtam-} in \tabref{tab:double.indexation}) and the suffix \forme{-t} in the Aorist, like the corresponding \textsc{1sg}\fl{}\textsc{3sg} form (§\ref{sec:indexation.mixed}).

Examples (\ref{ex:tundzeandZi}), (\ref{ex:tAwGnWmGlaandZi}), (\ref{ex:tukWquranW.YWntshi})  illustrate  \textsc{1sg}\fl{}\textsc{3du} (with stem III, \forme{ndza} \fl{} \forme{ndze}), \textsc{3du}\fl{} \textsc{1sg} and \textsc{2pl}\fl{}\textsc{1sg} configurations, respectively. All these forms have double suffixation, comparable to \forme{mtam-a-nɯ}, \forme{pɯ́-wɣ-mto-a-ndʑi} and \forme{kɯ-mto-a-nɯ} in \tabref{tab:double.indexation} (with the Imperfective instead of the Factual Non-Past). 

\begin{exe}
\ex   \label{ex:tundzeandZi}
\gll tu-ndze-a-ndʑi ra \\
\textsc{ipfv}-eat[III]-\textsc{1sg}-\textsc{du} be.needed:\textsc{fact} \\
\glt `I'd like to eat them.' (lWlu2002, 69)
\end{exe}

\begin{exe}
\ex   \label{ex:tAwGnWmGlaandZi}
\gll tu-kɯ-nɯmɢla-a jɤɣ ma nɤ-pi ni kɯ tɤ́-wɣ-nɯmɢla-a-ndʑi ɕti \\
\textsc{ipfv}-2\fl{}1-step.over-\textsc{1sg} be.allowed:\textsc{fact} \textsc{lnk} \textsc{2sg}.\textsc{poss}-elder.sibling \textsc{du} \textsc{erg} \textsc{aor}-\textsc{inv}-step.over-\textsc{1sg}-\textsc{du} be.\textsc{aff}:\textsc{fact} \\
\glt `You can step over me, your two elder sister stepped over me.' (Kunbzang, 31)
\end{exe}

\begin{exe}
\ex   \label{ex:tukWquranW.YWntshi}
\gll  a-pi ra, aʑɯɣ kukutɕu a-mɤ-kɤ-cha ci ɣɤʑu tɕe [...] tu-kɯ-qur-a-nɯ ɲɯ-ntsʰi \\
\textsc{1sg}.\textsc{poss}-elder.sibling \textsc{pl} \textsc{1sg}:\textsc{gen} here \textsc{1sg}.\textsc{poss}-\textsc{neg}-\textsc{obj}:\textsc{pcp}-can \textsc{indef} exist:\textsc{sens} \textsc{lnk} { } \textsc{ipfv}-2\fl{}1-help-\textsc{1sg}-\textsc{pl} \textsc{sens}-be.better \\
\glt `Sisters, I have a problem (something that I cannot do) here, help me!' (150828 donglang, 109)
\end{exe}

Double number indexation also occurs on relativized verbs with totalitative reduplication (§\ref{sec:totalitative.redp}),  such as \forme{pɯ\redp{}pɯ-mto-t-a-nɯ} in (\ref{ex:pWpWmtotanW}). In this example, the suffix \forme{-nɯ} indexes the number of the object \japhug{tɕʰeme}{girl}, which is also the head of this head-internal relative (§\ref{sec:head-internal.relative}). It is redundant with the verb-initial reduplication, which expresses universal quantification of the object.

\begin{exe}
\ex   \label{ex:pWpWmtotanW}
\gll nɯnɯ [tɕʰeme pɯ\redp{}pɯ-mto-t-a-nɯ] ɯ-ŋgɯ nɯ kɯ-fse kɯ-mpɕɤr maŋe \\
\textsc{dem} girl \textsc{total}\redp{}\textsc{aor}-see-\textsc{pst}:\textsc{tr}-\textsc{1sg}-\textsc{pl} \textsc{3sg}.\textsc{poss}-inside \textsc{dem} \textsc{sbj}:\textsc{pcp}-be.like \textsc{sbj}:\textsc{pcp}-be.beautiful not.exist:\textsc{sens} \\
\glt `This is the most beautiful among all the girls I have ever seen.' (150818 muzhi guniang-zh, 505)
\end{exe}

%\begin{exe}
%\ex   \label{ex:pjWkWsWfCatandZi}
%\gll χpi pjɯ-fɕat-a, pjɯ-kɯ-sɯ-fɕat-a-ndʑi nɯ ʁo jɤɣ, \\
%story \textsc{ipfv}-tell-\textsc{1sg} \textsc{ipfv}-2\fl{}1-\textsc{caus}-tell-\textsc{1sg}-\textsc{du} \textsc{dem} \textsc{advers} be.\textsc{allowed}:\textsc{fact} \\
%\glt `I can tell a story, you can have me tell you a story, but' (140511 1001 yinzi, 49)
%\end{exe}


All forms with double number indexation in Japhug (\tabref{tab:double.indexation} and examples \ref{ex:tundzeandZi} to \ref{ex:tukWquranW.YWntshi} above) contain the first person \forme{-a} suffix. This includes \textsc{1sg}$\rightarrow$3, 3$\rightarrow$\textsc{1sg} and 2$\rightarrow$\textsc{1sg} configurations, but not \textsc{1sg}$\rightarrow$2: the 1$\rightarrow$2 forms, unlike other configuration involving a first person, do not take person indexation suffixes coreferent with their first person subject (§\ref{sec:indexation.local}). Two hypotheses could be proposed to account for this relationship between \forme{-a} suffix and double number indexation.

First, this constraint could be seen as a (haplological) prohibition against the presence of two identical suffixes in the same verb form. Since second and third person number markers are identical, the only way to express a form such as \textsc{2du$\rightarrow$3du} in a fully explicit way would be $\dagger$\siga{}-\forme{ndʑi-ndʑi} with two times the same suffix, a form which would be excluded by the hapological rule. Such a rule however would not account for the absence of second number marker in verb forms suffixed with the \textsc{1du} \forme{-tɕi} or the \textsc{1pl} \forme{-ji}.

Second, it could be argued to be a question of phonology: all person indexation suffixes apart from \forme{-a} have the high vowels \ipa{i} or \ipa{ɯ} (which are not contrastive in this context, §\ref{sec:W.i.contrast}), and one could suppose that the ban on double suffixation in this paradigm is due to a constraint against two unstressed suffixes with high vowels (since the stress is on the last syllable of the stem, except in a limited number of cases, §\ref{sec:stress}).

However, this hypothesis is contradicted by the fact that other verbal paradigms in Japhug do contain verb forms with two suffixes in high vowels, as in example (\ref{ex:tokAlWlAtndZici}).\footnote{See §\ref{sec:preverbs.contracting.verbs} and §\ref{sec:peg.circumfix} on the peg  circumfix \forme{kɯ-...-ci}. }

\begin{exe}
\ex \label{ex:tokAlWlAtndZici}
\gll to-k-ɤlɯlɤt-ndʑi-ci \\
\textsc{ifr}-\textsc{peg}-fight-\textsc{du}-\textsc{peg} \\
\glt `They fought each other.'
\end{exe}

Given the fact that the combination of two unstressed suffixes in high vowel are possible in Japhug, phonology cannot explain the absence of a form such as $\dagger$\forme{kɯ-}\siga{}-\forme{tɕi-ndʑi} (intended for \textsc{2du$\rightarrow$1sg}).

A third approach to explain the unique properties of the \forme{-a} suffix, involving the notion of person hierarchy, is explored in §\ref{sec:direct-inverse}.

\subsubsection{The allomorphy of the inverse prefix} \label{sec:allomorphy.inv}
The inverse prefix\footnote{The possible origins of the inverse prefix are discussed in §\ref{sec:3sg.possessive.form} and §\ref{sec:inverse.history}.} has the allomorph \forme{ɣɯ́-} when occurring in word-initial position, something which is only possible in the Factual Non-Past (as it is the only TAME category without any orientation preverb, §\ref{sec:fact.morphology}) when no other inflectional prefix is present. It surfaces as \forme{wɣ-} in all other cases, merging with the vowel of the preceding prefix, which then bears the accent; the inverse is one of the very few stress-attracting prefixes in Japhug (§\ref{sec:stress.prefixal.chain}).

The allomorph \forme{wɣ-} also occurs in Factual Non-Past forms with the following prefixes:

\begin{itemize}
	\item Second person \forme{tɯ-} (§\ref{sec:indexation.mixed}): \forme{tɯ́-wɣ-mto} (2-\textsc{inv}-see:\textsc{fact}) `he will see you'
	\item Negative \forme{mɤ-} (§\ref{sec:neg.allomorphs}): \forme{mɤ́-wɣ-mto-a} (\textsc{neg}-\textsc{inv}-see:\textsc{fact}-\textsc{1sg}) `he will not see me'
	\item Apprehensive \forme{ɕɯ-} (§\ref{sec:apprehensive}): \forme{ɕɯ́-wɣ-mtsɯɣ-a} (\textsc{appr}-\textsc{inv}-bite:\textsc{fact}-\textsc{1sg}) `(I fear) that it could bite me'
	\item Associated motion \forme{ɕɯ-/ɣɯ-} (§\ref{sec:associated.motion}): \forme{ɣɯ́-wɣ-ndza-j} (\textsc{cisl}-\textsc{inv}-eat:\textsc{fact}-\textsc{1pl}) `it will come to eat us'
	\item Proximative aspect: \forme{jɯ-} (§\ref{sec:proximative}) \forme{jɯ́-wɣ-mtsɯɣ-a} \textsc{proxm}-\textsc{inv}-bite:\textsc{fact}-\textsc{1sg} `it is about to bite me'
	\item Possible modality: \forme{ɯmɤ-} (§\ref{sec:WmA}) \forme{ɯmɤ́-wɣ-mtsɯɣ-a} (\textsc{prob}-\textsc{inv}-bite:\textsc{fact}-\textsc{1sg}) `it will perhaps bite me' (\ref{ex:WmAwGmtsWGa}, §\ref{sec:fsp.epistemic})
	\item Rhetorical Interrogative: \forme{ɯβrɤ-} (§\ref{sec:WBrA.morphology}) \forme{ɯβrɤ́-wɣ-mtsɯɣ-a} (\textsc{rh}.\textsc{q}-\textsc{inv}-bite:\textsc{fact}-\textsc{1sg}) `it will not bite me, will it?'
	\item Interrogative: \forme{ɯ-} (§\ref{sec:interrogative.W.morpho}) \forme{ɯ́-wɣ-ndza-a} (\textsc{qu}-\textsc{inv}-eat:\textsc{fact}-\textsc{1sg}) `Will it eat me?'
\end{itemize}


With the progressive prefix \forme{asɯ-} (§\ref{sec:progressive.morphology}), the inverse is \textit{infixed} rather than being prefixed, for instance in a form such as \forme{ɲɯ-tɯ-ɤ́<wɣ>sɯ-zgroʁ} (\textsc{sens}-2-\textsc{prog}<\textsc{inv}>-attach) `he is attaching you'. This question is discussed in more detail in §\ref{sec:outer.prefixal.chain}.

The merger of the \ipa{w} of the inverse prefix with the vowels \ipa{ɯ} and \ipa{ɤ} of the preceding prefixes yields \ipa{u} and \ipa{o}, respectively.  Although the transcription \forme{-wɣ-} is chosen in the present orthography, the fricative \ipa{ɣ} is most often elided, and the stress and vowel rounding are the main clues of the presence of the inverse prefix. 

Due to the vowel merger, the contrasts between several series of prefixes are neutralized when preceding the inverse prefix, causing homophony between morphologically different forms (our orthography however keeps the distinction between the rounded and unrounded vowels in this context). There are potential ambiguities in three situations.

First, since the B-type (imperfective) \textsc{upwards} preverb \forme{tu-} (§\ref{sec:kamnyu.preverbs}) and the second person \forme{tɯ-} (§\ref{sec:intr.23}) both become neutralized as \forme{tú-} before the inverse prefix, transitive verbs whose intrinsic orientation is \textsc{upwards} (§\ref{sec:lexicalized.orientation}) have the same surface form in the Imperfective 3$'$\fl{}3 (and also \textsc{3sg}\fl{}\textsc{1sg} with verbs whose stem ends in \forme{-a}, due to vowel merger, see §\ref{sec:synizesis} and §\ref{sec:intr.1}) on the one hand and the Factual Non-Past 3\fl{}\textsc{2sg} on the other hand. The two examples in (\ref{ex:tuwGndza.ambiguity}) illustrate this ambiguity with the verb \japhug{ndza}{eat} (which selects the orientation \textsc{upwards}, §\ref{sec:preverb.ingestion}) with the  Factual Non-Past 3\fl{}\textsc{2sg} \forme{tɯ́-wɣ-ndza} `it will eat you' in (\ref{ex:ma.tWwGndza}) and the Imperfective \forme{tú-wɣ-ndza-a} \textsc{3sg}\fl{}\textsc{1sg} `it eats me',\footnote{For the use of the Imperfective in subject complement clause of the verb \japhug{jɤɣ}{be allowed}, see §\ref{sec:ipfv.complement}.} both pronounced \ipa{túɣndza} in two immediately adjacent sentences in the same text.

\begin{exe}
	\ex   \label{ex:tuwGndza.ambiguity}
	\begin{xlist}
		\ex   \label{ex:ma.tWwGndza}
		\gll βdɯt a-mɤ-jɤ-zɣɯt ra ma tɯ́-wɣ-ndza  \\
		demon \textsc{irr}-\textsc{neg}-\textsc{pfv}-arrive be.needed:\textsc{fact} \textsc{lnk} 2-\textsc{inv}-eat:\textsc{fact}  \\
		\glt (She said) `(Let us hope that) the demon will not arrive, otherwise it will eat you.' (tWxtsa2003, 31)
		\ex   \label{ex:tuwGndzaa.mAjAG}
		\gll tú-wɣ-ndza-a nɯ mɤ-jɤɣ nɤ́ma nɤʑo nɤ-ndʐa pɯ-ɣe-a ɕti tɕe \\ 
		\textsc{ipfv}-\textsc{inv}-eat-\textsc{1sg} \textsc{dem} \textsc{neg}-be.allowed:\textsc{fact} \textsc{sfp} \textsc{2sg} \textsc{2sg}.\textsc{poss}-reason \textsc{aor}:\textsc{down}-come[II]-\textsc{1sg} be.\textsc{aff}:\textsc{fact} \textsc{lnk} \\
		\glt `(The demon cannot) eat me, I came for you (to save you).' (tWxtsa2003, 32)
	\end{xlist}
\end{exe}

In addition to the meaning difference, the ambiguity between the two prefixes can sometimes be resolved by the syntactic context alone. For instance, the verb \japhug{stu}{do like} often occurs with another transitive verb in a serial verb construction (§\ref{sec:svc.similative.verb}) sharing the same person and tense. If the other verb in the construction does not select the orientation \textsc{upwards}, its Imperfective 3$'$\fl{}3 (or generic subject) and Factual 3\fl{}2\textsc{sg} will not be homophonous. 

For instance, in (\ref{ex:ki.tWwGstu}), since \japhug{nɯcʰɤmda}{drink with a straw} selects the orientation \textsc{downstream} (§\ref{sec:preverb.ingestion}, example \ref{ex:chWwGnWchAmdaj}), its Imperfective 3$'$\fl{}3 would be \forme{cʰɯ́-wɣ-nɯcʰɤmda}; the form \forme{tɯ́-wɣ-nɯcʰɤmda} is thus unambiguous, and suffices to demonstrate that the surface form \ipa{tú(ɣ)stu} in this context is really a 3\fl{}2\textsc{sg} form \forme{tɯ́-wɣ-stu} and not a 3$'$\fl{}3 Imperfective \forme{tú-wɣ-stu}, even without considering the meaning of the sentence.

Conversely, in (\ref{ex:ki.tuwGstu}), the second verb \forme{ɲɯ́-wɣ-cɯ} is unambiguously a generic Imperfective form, since there is no prefix with which the rounded allomorph of the B-type \textsc{westwards} preverb \forme{ɲɯ-} could be confused; the form \ipa{tú(ɣ)stu} in this sentence must therefore necessarily be analyzed as a generic Imperfective \forme{tú-wɣ-stu}.

\begin{exe}
	\ex   \label{ex:tuwGstu.ambiguity}
	\begin{xlist}
		\ex   \label{ex:ki.tWwGstu}
		\gll tʰɯ-tɯ-rgɤz tɕe ki tɯ́-wɣ-stu tɯ́-wɣ-nɯcʰɤmda ɕti tɕe, \\
		\textsc{aor}-2-be.old \textsc{lnk} \textsc{dem}.\textsc{prox} 2-\textsc{inv}-do.like:\textsc{fact} 2-\textsc{inv}-drink.with.a.straw:\textsc{fact} be.\textsc{aff}:\textsc{fact} \textsc{lnk} \\
		\glt `When you become old, they will drink you (your blood) like this with a straw (planted on your back)' (Norbzang 2012, 67)
		\ex   \label{ex:ki.tuwGstu}
		\gll ki tú-wɣ-stu tɕe ɲɯ́-wɣ-cɯ  \\
		\textsc{dem}.\textsc{prox} \textsc{ipfv}-\textsc{inv}-do.like \textsc{lnk} \textsc{ipfv}-\textsc{inv}-open \\
		\glt `One opens it like this.' (26-tCAkWG, 19)
	\end{xlist}
\end{exe}

Second, the A-type preverbs \forme{tɤ-} \textsc{upwards}, \forme{lɤ-} \textsc{upstream}, \forme{kɤ-} \textsc{eastwards} and \forme{jɤ-} and the corresponding D-type preverbs \forme{to\trt}, \forme{lo\trt}, \forme{ko-} and \forme{jo-} (§\ref{sec:kamnyu.preverbs}) are neutralized before the inverse prefix. This implies that the Aorist (§\ref{sec:aor.morphology}) and Inferential (§\ref{sec:ifr.morphology}) of verbs selecting the orientations listed above will be homophonous in all forms bearing the inverse prefix.

In (\ref{ex:towGsWlAt}), the 3$'$\fl{}3 form \ipa{tó(ɣ)sɯlɤt} could either be Inferential \forme{tó-wɣ-sɯ-lɤt} with the D-type preverb \forme{to-} or Aorist \forme{tɤ́-wɣ-sɯ-lɤt} with the A-type preverb \forme{tɤ-}. Here, the context can help to disambiguate between the two: the former is preferred because the whole story is told in the Inferential, and using the Aorist in this particular context would imply taking the event described by the verb as a reference point (§\ref{sec:aor.temporal}), with a different translation `when the horse...'. In (\ref{ex:tAwGXtW.CimWma}) on the other hand, the presence of the postposition \japhug{ɕimɯma}{immediately after} (§\ref{sec:immediate.subsequence}) and the meaning of the sentence imply that the form \ipa{tó(ɣ)χtɯ} must be analyzed as Aorist rather than Inferential.

\begin{exe}
	\ex   \label{ex:to.tA.ambiguity}
	\begin{xlist}
		\ex   \label{ex:towGsWlAt}
		\gll χsɯ-tɤxɯr ʑo tó-wɣ-sɯ-lɤt tɕe tɤ-βɟu ɯ-taʁ tɕe pjɤ́-wɣ-βde. \\
		three-turn \textsc{emph} \textsc{ipfv}-\textsc{inv}-\textsc{caus}-release \textsc{lnk} \textsc{indef}.\textsc{poss}-mat \textsc{3sg}.\textsc{poss}-on \textsc{loc} \textsc{ifr}-\textsc{inv}-throw \\
		\glt `She made the (horse) run three laps and throw her on the mat.' (2003kAndzwsqhaj2, 88)
		\ex   \label{ex:tAwGXtW.CimWma}
		\gll  koxtɕɯn-ri nɯnɯ tɤ́-wɣ-χtɯ ɕimɯma ɕɤɣ nɯra wuma ɲɯ-mpɕɤr ri, \\
		silk-thread \textsc{dem} \textsc{aor}-\textsc{inv}-buy immediately be.new:\textsc{fact} \textsc{dem}:\textsc{pl} really \textsc{sens}-be.beautiful \textsc{lnk} \\
		\glt `Silk threads are very beautiful when one has just bought them, when they are new.' (2002thaXtsa, 167)
	\end{xlist}
\end{exe}

Third, the contrast between the Proximative \forme{jɯ-} (§\ref{sec:proximative}) and the B-type \forme{ju-} preverb is also neutralized when preceding the inverse prefix. For instance, the Proximative Factual 3\fl{}\textsc{1sg} \forme{jɯ́-wɣ-ɕaβ-a} \textsc{proxm}-\textsc{inv}-catch.up-\textsc{1sg} `it is about to catch up with me' and the Imperfective \forme{jú-wɣ-ɕaβ-a} \textsc{ipfv}-\textsc{inv}-catch.up-\textsc{1sg} are homophonous, since the verb \japhug{ɕaβ}{catch up} is compatible the indefinite orientation \forme{ju-} (§\ref{sec:proximative}).



\subsubsection{Direction marking and person hierarchies} \label{sec:direct-inverse}
The mixed and non-local domains of the Japhug indexation system present a remarkable symmetry, illustrated in \tabref{tab:symmetrical}: direct X\fl{}3 and inverse 3\fl{}X forms have the same person indexation affixes, and only differ by the presence of the prefix \forme{wɣ-} in inverse configurations, and of stem III in (Non-Past) X\textsc{sg}\fl{}3 configurations. 

\begin{table}[H]  \caption{Symmetrical indexation} \label{tab:symmetrical}
\begin{tabular}{|l|ll|}
\lsptoprule
Person & Direct (X\fl{}3)& Inverse (3\fl{}X)\\
\hline
\textsc{1sg} & \forme{mtam-a} & \forme{ɣɯ́-mto-a} \\
\textsc{1du} & \forme{mto-tɕi} & \forme{ɣɯ́-mto-tɕi} \\
\textsc{1pl} & \forme{mto-j} & \forme{ɣɯ́-mto-j} \\
\hline
\textsc{2sg} & \forme{tɯ-mtɤm} & \forme{tɯ́-wɣ-mto} \\
\textsc{2du} & \forme{tɯ-mto-ndʑi} & \forme{tɯ́-wɣ-mto-ndʑi} \\
\textsc{2pl} & \forme{tɯ-mto-nɯ} & \forme{tɯ́-wɣ-mto-nɯ} \\
\hline
\textsc{3sg} & \forme{mtɤm} & \forme{ɣɯ́-mto} \\
\textsc{3du} & \forme{mto-ndʑi} & \forme{ɣɯ́-mto-ndʑi} \\
\textsc{3pl} & \forme{mto-nɯ} & \forme{ɣɯ́-mto-nɯ} \\
\lspbottomrule
\end{tabular}
\end{table}

Although this symmetry is broken in the local domain, where neither the \forme{wɣ-} prefix nor stem III alternation occur (§\ref{sec:indexation.local}), Japhug has an indexation system very close to the canonical direct-inverse, as presented in \tabref{tab:inverse-canon} \citep{jacques14inverse}: all configurations of the lower half of the bipersonal indexation space except 2\fl{}1 take the \forme{wɣ-} prefix, while this prefix is not found in the upper half. 

\begin{table}
 \caption{The canonical direct/inverse system} \label{tab:inverse-canon}
\begin{tabular}{|l|llll|}
\lsptoprule
&1 & 2 &3&3$'$\\
\hline
1 &\grise{} &1\fl{}2 & 1\fl{}3& \\
2&2\fl{}1&\grise{}&2\fl{}3 &\\
3&3\fl{}1&3\fl{}2&\grise{}&3\fl{}3$'$\\
3$'$&&&3$'$\fl{}3&\grise{}\\
\lspbottomrule
\end{tabular}
\end{table}

As mentioned above, person indexation affixes in Japhug have neutral alignment, and can be used to index either subjects (of transitive or intransitive verbs) or objects. With transitive verbs, the \forme{wɣ-} prefix (whose allomorphy is described in §\ref{sec:allomorphy.inv}) and Stem III (§\ref{sec:stem3}) serve to disambiguate the function of the indexation affixes closest to the verb stem:\footnote{If two indexation suffixes are present, the second suffix indexes the number of the other argument (§\ref{sec:double.number.indexation}).} if the \forme{wɣ-} prefix is present, the affix(es) index the object, while if Stem III is present, they index the subject. The \forme{wɣ-} prefix and stem III mark \textit{inverse} and \textit{direct} configurations, respectively.

A way to describe the distribution of the inverse and direct markers is the notion of person or empathy hierarchy (\citealt{silverstein76, delancey81direction, jackson02rentongdengdi, lockwood12hierarchies}). A typical example of person hierarchy is (\ref{ex:empathy.hierarchy}),  on which first person ranks higher than second person, discourse participants (first of second persons) higher than third persons,\footnote{On the difference between third person on the one hand, and first and second persons on the other hand, see also \citet[253--256]{benveniste66problemes1}.} and among  animate third persons (in particular humans) higher than inanimate third persons.

\begin{exe}
\ex \label{ex:empathy.hierarchy}
\glt 1 > 2 > 3 animate > 3 inanimate
\end{exe}

This type of hierarchies,\footnote{Other hierarchies have been suggested; for instance, slot accessibility in most Algonquian languages requires to posit a hierarchy 2 > 1 > 3. } originally proposed to account for splits in pronominal systems \citep{silverstein76}, have been invoked to explain various morphosyntactic phenomena, including slot accessibility (for instance, the prefixal slot in the Independent Order paradigms of Algonquian languages, \citealt{zuniga06, lockwood12hierarchies}) and direct/inverse marking.

In this framework, when the subject and the object compete for the same morphological slot, the one that is higher on the hierarchy is indexed. In addition, \textsc{inverse} markers occur when the \textit{object} is higher on the hierarchy than the subject (2\fl{}1, 3\fl{}1, 3\fl{}2), and \textsc{direct} markers when the \textit{subject} is higher (1\fl{}2, 1\fl{}3, 2\fl{}3). 

The explanatory power of hierarchies to analyze indexation systems has been challenged \citep{zuniga18hierarchies}, in particular due to the fact that in some languages one would need to posit contradictory hierarchies (see Zúñiga's \citeyear{zuniga06} discussion of Plains Cree).

Independently of the cross-linguistic validity of the notion of person hierarchies, the Japhug indexation system is amenable to an analysis in terms of the two non-contradictory hierarchies in (\ref{ex:empathy.hierarchy.japhug}). The nature of the contrast between 3 proximate and 3$'$ obviative is discussed in §\ref{sec:inverse.3.3.saliency}.

\begin{exe}
\ex \label{ex:empathy.hierarchy.japhug}
\begin{xlist}
\ex \label{ex:hierarchy.3}
\glt 1, 2 > 3\textsc{prox} > 3$'$, \textsc{genr}
\ex \label{ex:hierarchy.1sg}
\glt \textsc{1sg} > 1\textsc{n}.\textsc{sg}, 2, 3
\end{xlist}
\end{exe}

Hierarchy (\ref{ex:hierarchy.3}) describes the distribution of the inverse \forme{wɣ-} prefix and Stem III in the mixed and non-local domains: 

\begin{itemize}
\item The inverse prefix occurs whenever the subject is lower than the object (3\fl{}1, 3\fl{}2, 3$'$\fl{}3), a rule that also accounts for its function to mark the generic transitive subject \textsc{genr}\fl{}3 (§\ref{sec:indexation.generic.tr}). 
\item Stem III is found when the subject is singular and higher than the object (\textsc{1sg}\fl{}3, \textsc{2sg}\fl{}3, \textsc{3sg}\fl{}3$'$). 
\item Neither the inverse prefix nor Stem III are found when the subject and the object are equal on (\ref{ex:hierarchy.3}), in particular in the local domain (2\fl{}1, 1\fl{}2), where portmanteau prefixes occur to indicate person configuration (§\ref{sec:indexation.local}), and the suffix closest to the verb stem indexes the object. Generic object configurations, which are marked with the \forme{kɯ-} generic prefix and lack Stem III (§\ref{sec:indexation.generic.tr}), 
are analyzed as 3$'$\fl{}\textsc{genr}, as third person obviative and generic are equal on (\ref{ex:hierarchy.3}).
\end{itemize}

Additional evidence for the existence of (\ref{ex:hierarchy.3}) is presented in §\ref{sec:ditransitive.causative}.

Hierarchy (\ref{ex:hierarchy.1sg}) accounts for the special status of the \textsc{1sg}: whenever one of the core arguments is \textsc{1sg}, the number of the other argument is indexed on the verb. In 2\fl{}\textsc{1sg}, 3\fl{}\textsc{1sg} and \textsc{1sg}\fl{}3 configurations, an additional suffix can follow the \forme{-a} \textsc{1sg} to index the number of the third or second person argument (§\ref{sec:double.number.indexation}). In the \textsc{1sg}\fl{}2 configurations, the suffix is coreferent with the number of the object like all forms in the local domain (§\ref{sec:indexation.local}). 

The hierarchies in (\ref{ex:empathy.hierarchy.japhug}) are only valid for Japhug. In other Gyalrong languages, including Situ (\citealt{delancey81direction, jackson15sastod, zhangsy19obviative}), Tshobdun \citep{jackson02rentongdengdi} and Zbu \citep{gongxun14agreement}, the inverse prefix is also found in the 2\fl{}1 configuration and does not appear in the generic subject form, suggesting a hierarchy closer to (\ref{ex:empathy.hierarchy}).

A different approach to explain the structure of the Japhug direct-inverse indexation system is to analyze it in terms of historical linguistics. Some preliminary ideas on the topic are presented in §\ref{sec:indexation.suffixes.history} and §\ref{sec:portmanteau.prefixes.history}.




\subsection{The function of the direct/inverse contrast in non-local configurations} \label{sec:inverse.3.3.saliency}
Like most Gyalrong languages, including Tshobdun \citep{jackson02rentongdengdi}, Zbu \citep{gongxun14agreement}, but excluding West Gyalrongic \citep{lai15person} and some dialects of Situ (\citealt{jackson15sastod, zhangsy19obviative}), Japhug has a contrast between inverse and direct forms in the non-local configurations.

The clearest function of the inverse prefix in this context in Japhug is marking generic subject (§\ref{sec:indexation.generic.tr}).  In non-generic forms, the choice of inverse or direct configurations is determined by several factors, including animacy, possession and saliency of the core arguments.

Japhug lacks obviative marking on nouns (§\ref{sec:possessive.prefix.obv.def}), but in the present work I use the terms `obviative' (abbreviated as 3$'$) to refer to the third person \textit{object} of a verb in direct non-local configuration or to the third person \textit{subject} of a transitive verb with inverse configuration, and `proximate' (3) for the subject of a verb in direct form or object of a verb in inverse form (the choice of these terms is discussed in §\ref{sec:indexation.non.local}).
   
\subsubsection{Animacy}  \label{sec:obviation.animacy}
The clearest factor determining the proximate or obviative status of a noun phrase in Japhug is animacy. Whenever one of the core arguments (whether a noun phrase or a clause) is inanimate and the other animate, the former will almost always be obviative, and the latter proximate. 

When the subject is animate and the object inanimate (with the exception of generic subjects §\ref{sec:indexation.generic.tr} and pseudo-passive constructions §\ref{sec:pseudo.passive}), the verb must be in direct form. Thus, indirective verbs of speech (§\ref{sec:ditransitive.indirective}) such as \japhug{ti}{say}, or transitive modal verbs such as \japhug{spa}{be able} (§\ref{sec:spa.verb}), whose object is always a complement clause (or an abstract noun for some of these verbs),\footnote{Complement clauses, being inanimate arguments, are always obviative when used as objects of complement-taking verbs. } and whose subject is necessarily human or higher animal, will always have a proximate subject and an obviative object (3\fl{}3$'$), and thus never appear in (non-generic) inverse configurations. 

Conversely, in nearly all the cases when the subject is an inanimate entity (including plants) and the object an animate one (including humans and non-human animals), the verb appears in inverse form, as illustrated by examples (\ref{ex:nWNa.chowGWt}), (\ref{ex:YAwGsWGYARnW}) and (\ref{ex:pANAxCaj}). 

Inanimate agents are usually natural forces, such as \japhug{tɯ-ci}{water} (subject of \forme{cʰɤ́-wɣ-ɣɯt} `it brings her downstream' in \ref{ex:nWNa.chowGWt}) or \forme{tɯ-mɯ} `the rain'  (subject of the verb \forme{pjɯ́-wɣ-χtɕi-nɯ} `(they) wash them away' in \ref{ex:YAwGsWGYARnW}).   
 
\begin{exe}
\ex \label{ex:nWNa.chowGWt} 
\gll tɯɲɤt tʰɯ-ɣe nɯ ɯ-rca nɯtɕu tɕe, nɯŋa pɯ-kɯ-nɯrɯ nɯ tɤrca cʰɤ́-wɣ-ɣɯt, tɯ-rdoʁ. tɕe nɯnɯ cʰɤ́-wɣ-ɣɯt tɕendɤre iɕqʰa nɯ, nɤki tɯ-ci kɯ cʰɤ́-wɣ-ɣɯt qʰe \\
landslide \textsc{aor}:\textsc{downstream}-come[II] \textsc{dem} \textsc{3sg}.\textsc{poss}-following \textsc{dem}:\textsc{loc} \textsc{loc} cow  \textsc{pst}.\textsc{ipfv}-\textsc{sbj}:\textsc{pcp}-eat.grass \textsc{dem} together \textsc{ifr}:\textsc{downstream}-\textsc{inv}-bring one-piece, \textsc{lnk} \textsc{dem} \textsc{ifr}:\textsc{downstream}-\textsc{inv}-bring \textsc{lnk} \textsc{filler} \textsc{dem} \textsc{filler} \textsc{indef}.\textsc{poss}-water \textsc{erg} \textsc{ifr}:\textsc{downstream}-\textsc{inv}-bring  \textsc{lnk} \\
\glt `There was a landslide, and together with it, a grazing cow was taken away, the water took it away.' (160715 nWNa, 5-6)
\end{exe}

Plants can also be agents of verbs with animate patients, as in (§\ref{ex:pANAxCaj}). In this example, note the inversion between the subject and object of the first verb (\forme{tu-ndze}, whose subject is the dog and object the plant) and those of the second verb (\forme{lú-wɣ-sɯ-qioʁ}, whose subject (causer) is the plant and object the dog).

\begin{exe}
\ex \label{ex:pANAxCaj} 
\gll pɤŋɤxɕaj nɤki, kʰɯna kɯ tu-ndze tɕe lú-wɣ-sɯ-qioʁ ɲɯ-ŋu. \\
plant.sp. \textsc{filler} dog \textsc{erg} \textsc{ipfv}-eat[III] \textsc{lnk} \textsc{ipfv}-\textsc{inv}-\textsc{caus}-vomit \textsc{sens}-be \\
\glt `(The plantcalled) \forme{pɤŋɤxɕaj}$_i$, when a dog$_j$ eats it$_i$, it makes it$_j$ vomit.' (140505 panaxCAj, 3)
\end{exe}

There are also cases when inert substances (such as as soot in \ref{ex:YAwGsWGYARnW}) can be agents. Indeed, in (\ref{ex:YAwGsWGYARnW}), soot is the non-overt subject of the verb  \forme{ɲɤ́-wɣ-sɯɣ-ɲaʁ-nɯ} `it caused them to become black'. In such cases the inanimate argument is always obviative, and inverse marking on the verb is required.

\begin{exe}
\ex \label{ex:YAwGsWGYARnW} 
\gll <yancong> ɯ-ŋgɯ ɲɯ-ɲaʁ rcanɯ, tɕe kumpɣɤtɕɯ ra ɲɤ́-wɣ-sɯɣ-ɲaʁ-nɯ ʑo, nɯ-kɯ-ɤkʰra ra mɯ-ɲɤ-χsɤl ʑo tɕendɤre ʑɯrɯʑɤri qale tu-βze, tɯmɯ kɯ pjɯ́-wɣ-χtɕi-nɯ tɕe ɲɯ-me ɲɯ-ŋu \\
chimney \textsc{3sg}-inside \textsc{ipfv}-be.black \textsc{unexp}:\textsc{deg} \textsc{lnk} sparrow \textsc{pl} \textsc{ifr}-\textsc{inv}-\textsc{caus}-be.black-\textsc{pl} \textsc{emph} \textsc{3pl}.\textsc{poss}-\textsc{nmlz}:S/A-be.colourful \textsc{pl} \textsc{neg}-\textsc{ifr}-be.clear \textsc{emph} \textsc{lnk} progressively wind \textsc{ipfv}-make[III] \textsc{indef}.\textsc{poss}-weather \textsc{erg} \textsc{ipfv}-\textsc{inv}-wash-\textsc{pl} \textsc{lnk} \textsc{ipfv}-not.exist \textsc{sens}-be \\
\glt `As it is black inside the chimney, the sparrows were completely blackened by it, the patterns and colours (on their feathers) were not visible any more, but progressively, the wind and the rain wash them away and (the soot on their feathers) disappears.'
 (22-kumpGatCW, 74-76)
\end{exe}

Natural forces appear to be intermediate between inanimates and animates, as counterexamples without inverse marking do exist. In (\ref{ex:BZW.thanWtsWM}) for instance, the verb \forme{tʰa-nɯ-tsɯm} `it took them downstream' has \japhug{tɯ-ci}{water} as subject and the animate noun \forme{βʑɯ ra} `the mice' as object, but a direct configuration 3\fl{}3$'$ is selected. However, only a handful of examples of this type are found in the corpus, and inverse marking would normally be expected.

\begin{exe}
\ex \label{ex:BZW.thanWtsWM}
\gll cɯrmbɯ ɯ-taʁ nɯ nɯ-z-rɤʑi ɲɤ-me ma tɯ-ci to-ɣi qʰe, tɕendɤre nɤki βʑɯ ra kɤsɯfse tʰa-nɯ-tsɯm. qʰe pɯ-si-nɯ. \\
stone.heap \textsc{3sg}.\textsc{poss}-on \textsc{dem} \textsc{3pl}.\textsc{poss}-\textsc{nmlz}:\textsc{obl}-stay \textsc{ifr}-not.exist \textsc{lnk} \textsc{indef}.\textsc{poss}-water \textsc{ifr}:\textsc{up}-come \textsc{lnk} \textsc{lnk} \textsc{filler} mouse \textsc{pl} all \textsc{aor}:\textsc{downstream}:3\fl{}3$'$-\textsc{auto}-take.away \textsc{lnk} \textsc{aor}-die-\textsc{pl} \\
\glt `The place where they$_j$ stayed on the stone heap disappeared as the water$_i$ came up, and it$_i$ took the mice$_j$ downstream. And they$_j$ died.  (150831 BZW kAnArRaR, 95)
\end{exe}

When both core arguments have third person animate referents, both direct (\ref{ex:khu.kW.thaCkWt}) and inverse (\ref{ex:pjAwGnAliAliAt}) configurations are possible, even when the subject is non-human and the object human.

\begin{exe}
\ex \label{ex:khu.kW.thaCkWt}
\gll ndʑi-sɤtɕʰa nɯnɯ ɣɯ jɯl nɯnɯ rcanɯ kʰu kɯ lonba ʑo tʰa-ɕkɯt ɲɯ-ŋu  \\
\textsc{3du}.\textsc{poss}-place \textsc{top} \textsc{gen} villager \textsc{dem} \textsc{unexp}:\textsc{foc} tiger \textsc{erg} all \textsc{emph} \textsc{aor}:\textsc{3}\fl{}3$'$-eat \textsc{sen}-be \\
 \glt `All the villagers in their land had been eaten by a tiger.'  (khu2005, 5)
\end{exe}

\begin{exe}
\ex \label{ex:pjAwGnAliAliAt}
\gll ``mtsʰoʁlaŋ ni a-kʰɤkɯm, nɤki, a-kʰɯna ʑo a-pɯ-fse-ndʑi ra" to-ti ɲɯ-ŋu tɕe mtsʰoʁlaŋ ni kɯ pjɤ́-wɣ-nɤliɤliɤt ʑo ɕti   \\
water.monster \textsc{du} \textsc{1sg}.\textsc{poss}-doorstep \textsc{filler} \textsc{1sg}.\textsc{poss}-dog \textsc{emph} \textsc{irr}-\textsc{ipfv}-be.like-\textsc{du} be.needed:\textsc{fact} \textsc{ifr}-say \textsc{sens}-be \textsc{lnk} water.monster \textsc{du} \textsc{erg} \textsc{ifr}-\textsc{inv}-greet.like.a.dog \textsc{emph} be.\textsc{aff}:\textsc{fact} \\
\glt `He said `May the water monsters be like dogs on my doorstep' and the water monsters greeted him like dogs.' (Norbzang 2012, 205-206)
\end{exe}

When both core arguments are inanimate, the verb is generally in direct form, including in the case of inanimate objects acting on body parts of animate beings, as in (\ref{ex:tAjpGom.kW.pjAndo}) or (\ref{ex:WmYaR.YAznAmbju}).\footnote{Example (\ref{ex:WmYaR.YAznAmbju}) is a particularly convoluted type of unmarked embedded clause (§\ref{sec:embedded.clause}), as the ergative phrase \forme{rɟɤlpu nɯ kɯ} `the king' is not subject of the immediately following transitive verb \forme{ɲɤ-z-nɤmbju}. }

\begin{exe}
\ex \label{ex:tAjpGom.kW.pjAndo}
\gll ɯ-mpʰɯz nɯnɯ tɤjpɣom kɯ pjɤ-ndo qʰendɤre, \\
\textsc{3sg}.\textsc{poss}-buttocks \textsc{dem} ice \textsc{erg} \textsc{ifr}-take \textsc{lnk} \\
\glt `His buttocks were stuck on the ice.' (140427 qala cho kWrtsAG, 37)
\end{exe}

\begin{exe}
\ex \label{ex:WmYaR.YAznAmbju}
\gll  rɟɤlpu nɯ kɯ [ɯ-mɲaʁ ɲɤ-z-nɤmbju] ʑo pjɤ-mto tɕe,  \\
  king \textsc{dem} \textsc{erg} \textsc{3sg}.\textsc{poss}-eye \textsc{ifr}-\textsc{caus}-be.bright \textsc{emph} \textsc{ifr}-see \textsc{lnk} \\
\glt `The king$_j$ saw it (the thread)$_i$ as it$_i$ dazzled his$_j$ eyes.'  (2012 Norbzang, 151-152)
\end{exe}

Inverse forms can occur when the object is the most salient of the two inanimate arguments (§\ref{sec:obviation.saliency}), as in (\ref{ex:YWwGzmaqhu}), where both the subject and the object of the verb \forme{ɲɯ́-wɣ-z-maqʰu} `it makes it late' are plants.

\begin{exe}
\ex \label{ex:YWwGzmaqhu}
\gll ɯ-rkɯ nɯtɕu, si kɯ-wxti a-pɯ-tu tɕe ɲɯ́-wɣ-z-maqʰu qʰe ɯʑo tu-mbro mɯ́j-cʰa.	\\
\textsc{3du}.\textsc{poss}-side \textsc{dem}:\textsc{loc} tree \textsc{sbj}:\textsc{pcp}-be.big \textsc{irr}-\textsc{ipfv}-exist \textsc{lnk} \textsc{ipfv}-\textsc{inv}-\textsc{caus}-be.after \textsc{lnk} \textsc{3sg} \textsc{ipfv}-be.big \textsc{neg}:\textsc{sens}-can \\
\glt `If there is a big tree$_j$ next to it$_i$, it$_j$ delays its$_i$ growth and it$_i$ cannot grow very big.' (14-sWNgWJu, 242)
\end{exe}
  
  
\subsubsection{Possession and obviation} \label{sec:obviation.possessor}
In Algonquian languages, there are two main constraints governing the use of obviation on nouns: on the other hand, third persons possessed by another third person are automatically obviative (\citealt[25]{wolfart73}, \citealt[625]{valentine01grammar}), and on the other hand, at most one argument can be proximate in any given clause  \citep[627]{valentine01grammar}. A consequence of these constraints is that whenever a transitive verb takes as subject a possessed noun, and as object a (proximate) noun coreferent with the possessor of the subject,\footnote{If the object noun is not co-referent with the possessor of the subject, it will be obviative too due to the second constrain. } the verb will necessarily have an inverse form, as in the Cree example in (\ref{ex:cree.kiimaakwamik}). Conversely, if the subject is a proximate noun and the object a possessed noun whose possessor is coreferent with the subject, the verb will necessarily be direct.

\begin{exe}
\ex \label{ex:cree.kiimaakwamik}
 \gll cān o-tēm-a kī-mākwam-ik  \\
  \textsc{anthr} \textsc{3sg}.\textsc{poss}-dog-\textsc{obv} \textsc{pst}-bit-\textsc{inv} \\
 \glt `John$_{prox}$'s dog$_{obv}$ bit him$_{prox}$.' (\citealt[25]{wolfart73})
\end{exe}

Though similar phenomena have been reported outside of Algonquian \citep{aissen97obviation}, not all languages with direct-inverse indexation display such relationship between possession and obviation \citep{haude16symmetrical}.
 
Japhug lacks obviation marking on nouns (§\ref{sec:possessive.prefix.obv.def}), but the presence of direct or inverse morphology can nevertheless be used to test whether possession has an effect on the obviation status of nominal arguments.

There is a very clear tendency for inverse marking to occur when the subject is a possessed noun and the object its possessor (henceforth  SPO `subject possessed by third person object'), as shown by examples such as (\ref{ex:towGnAmqe}), (\ref{ex:mWpjAwGsWXsAl}) and (\ref{ex:pjAwGnAZAmNAn}).

\begin{exe}
\ex \label{ex:towGnAmqe}
 \gll   a-wa ɯ-ɣi ra nɯ-ɕki kó-wɣ-ndʑɯ, [...] tɕendɤre  a-wa ɯ-ɣi ra kɯ tó-wɣ-nɤmqe \\
  \textsc{1sg}.\textsc{poss}-father \textsc{3sg}.\textsc{poss}-relative  \textsc{pl} \textsc{3pl}.\textsc{poss}-\textsc{dat} \textsc{ifr}-\textsc{inv}-accuse  {  } \textsc{lnk}  \textsc{1sg}.\textsc{poss}-father \textsc{3sg}.\textsc{poss}-relative \textsc{pl} \textsc{erg} \textsc{ifr}-\textsc{inv}-scold \\
 \glt `(The old monk) complained about my father$_i$ to his$_i$ parents, and my father$_i$'s parents scolded him$_i$.' (08-kWqhi, 19)
\end{exe}  
  
\begin{exe}
\ex \label{ex:mWpjAwGsWXsAl}
 \gll  tɕe ɯ-pi ʁnɯz nɯni kɯnɤ mɯ-pjɤ́-wɣ-sɯχsɤl \\
 \textsc{lnk} \textsc{3sg}.\textsc{poss}-elder.sibling two \textsc{dem}:\textsc{du} also \textsc{neg}-\textsc{ifr}-recognize \\
 \glt `Even her$_i$ two elder sisters did not recognize her$_i$.' (140504 huiguniang-zh, 120)
\end{exe}
 
 \begin{exe}
\ex \label{ex:pjAwGnAZAmNAn}
 \gll tɕendɤre rɟɤlpu ɣɯ ɯ-rʑaβ ɲɤ-k-ɤβzu-ci tɕe, tɕe ɯ-pi ni kɯ wuma ʑo, nɤkinɯ, pjɤ́-wɣ-nɤʑɤmŋɤn \\
\textsc{lnk} king \textsc{gen} \textsc{3sg}.\textsc{poss}-wife \textsc{ifr}-\textsc{peg}-become-\textsc{peg} \textsc{lnk} \textsc{lnk} \textsc{3sg}.\textsc{poss}-elder.sibling \textsc{du} \textsc{erg} really \textsc{emph} \textsc{filler} \textsc{ifr}.\textsc{ipfv}-\textsc{inv}-envy \\
\glt `She$_i$ became the queen, and her$_i$ sisters were envious of her$_i$.' (140514 huishuohua de niao-zh, 15)
\end{exe}

However, the fact that a correlation between inverse marking and SPO configurations exists does not necessarily imply that the inverse is actually triggered by SPO. 

First, there are also many examples of SPO with direct marking. For instance, in (\ref{ex:YWznAjandZi.mWtasWGendZi}), the verbs \forme{ɲɯ-z-nɤja-ndʑi} and \forme{mɯ-ta-sɯ-ɣe-ndʑi} with SPO lack inverse prefixes and have direct morphology (subject number indexation §\ref{sec:indexation.non.local} and the C-type orientation preverb \forme{ta\trt}, §\ref{sec:transitivity.morphology}). In addition, the examples (\ref{ex:prox.naBde}) and (\ref{ex:obv.nWwGBde}) in §\ref{sec:possessive.prefix.obv.def} provide a minimal of SPO configuration with direct vs. inverse configuration in exactly the same context.

  \begin{exe}
\ex \label{ex:YWznAjandZi.mWtasWGendZi}
 \gll  tɕʰeme wuma ʑo kɯ-pe ci pɯ-ŋu qʰe, ɯ-mu ɯ-wa ni kɯ ɲɯ-z-nɤja-ndʑi qʰe mɯ-ta-sɯ-ɣe-ndʑi, \\
 girl really \textsc{emph} \textsc{sbj}:\textsc{pcp}-be.good \textsc{indef} \textsc{pst}.\textsc{ipfv}-be \textsc{lnk} \textsc{3sg}.\textsc{poss}-mother \textsc{3sg}.\textsc{poss}-father \textsc{du} \textsc{erg} \textsc{sens}-\textsc{caus}-be.regrettable-\textsc{du} \textsc{lnk} \textsc{neg}-\textsc{aor}:3\fl{}3$'$-\textsc{caus}-come[II]-\textsc{du} \\
 \glt `(His wife)$_i$  was a very nice girl, her$_i$ parents were unwilling to part with her$_i$ and did not let her$_i$ come.' (14-siblings, 304)
 \end{exe}
 
 %ma ki ndɤre, ji-βdaʁmu ki, ɯ-pi ra kɯ ɲɤ-znɯqatɯkɯr-nɯ ma maka 
%/ nɯ ɕɯŋgɯ mɯ-to-fse tɕe ji-scawa

Second, the presence of inverse marking in examples (\ref{ex:towGnAmqe}), (\ref{ex:mWpjAwGsWXsAl}) and (\ref{ex:pjAwGnAZAmNAn}) above (and in most cases of SPO with inverse) can be accounted for by factors other than possession, in particular the relative saliency of the subject and the object (§\ref{sec:obviation.saliency}): in all three examples, the object corresponds to the main character of the story; notice in particular in (\ref{ex:towGnAmqe}) the presence of inverse marking on the previous verb \forme{kó-wɣ-ndʑɯ} `he complained about him', which does not have SPO. 

In stories, non-salient characters tend to be described on the basis of their relationship with a salient character (rather than by a different name), and are therefore expressed by nouns with third person possessor. Many, if not most SPO configurations in the corpus involve non-salient referents acting upon salient referents related to them; the apparent correlation between SPO and inverse marking in texts may therefore be a side-effect of argument saliency rather than a specific morphosyntactic phenomenon.


\subsubsection{Saliency}  \label{sec:obviation.saliency}
When both core arguments are third person animate, both are potentially proximate or obviative, and the verb can thus be either in direct of inverse form, and the choice between one or the other is determined, as we will discuss in this section, by pragmatics. In this section, I mainly focus on human or anthropomorphized non-human referents; animals are briefly treated at the end of the section.

Transitive verb forms where such a choice between inverse and direct forms exist are however a minority. For instance, in  (\ref{ex:towGtsWm.tonWqambWmbjom}), the verbs \forme{to-ti} `she said', \forme{to-nɯ-ŋga} `she wore it' and \forme{ko-rqoʁ} `he hugged (her neck)' are in the direct form due to animacy constraints (since the subjects of all three verbs are animates and their object inanimates, see §\ref{sec:obviation.animacy}). Note that a subject shift takes place between  \forme{to-nɯ-ŋga} `she wore it' and \forme{ko-rqoʁ} `he hugged (her neck)' without inverse marking.

The only verb in this passage taking two animate referents is \forme{tó-wɣ-tsɯm} `she took him up', whose subject is \forme{stu kɯ-xtɕi nɯ} `the smallest (of the daughters of heaven)' and whose object is a human boy.\footnote{Compare example (\ref{ex:towGtsWm.tonWqambWmbjom}), where this referent is in \textsc{3sg}, with (\ref{ex:jWfCWndzximWr}) in §\ref{sec:time.ordinals} and (\ref{ex:jWGmWr.ndAre}) in §\ref{sec:adversative.topic}, where it appears in \textsc{1sg}. }

 \begin{exe}
\ex \label{ex:towGtsWm.tonWqambWmbjom} 
\gll stu kɯ-xtɕi nɯ kɯ `...' to-ti tɕendɤre qro nɯnɯ ɯ-ŋga nɯ to-nɯ-ŋga qʰe, tɕe nɯ ɯ-mke ko-rqoʁ qʰe tɕendɤre tó-wɣ-tsɯm to-nɯqambɯmbjom qʰe \\
most \textsc{sbj}:\textsc{pcp}-be.small \textsc{dem} \textsc{erg} { } \textsc{ifr}-say \textsc{lnk} pigeon \textsc{dem} \textsc{3sg}.\textsc{poss}-clothes \textsc{top} \textsc{ifr}-\textsc{auto}-wear \textsc{lnk} \textsc{lnk} \textsc{dem} \textsc{3sg}.\textsc{poss}-neck \textsc{ifr}-hug \textsc{lnk} \textsc{lnk} \textsc{ifr}-\textsc{inv}-take.away \textsc{ifr}:\textsc{up}-fly \textsc{lnk} \\
\glt `The smallest (daughter)$_i$ said `...', she$_i$ wore the pigeon skin, he$_j$ hugged her$_i$ neck, and she$_i$ took him$_j$ and flew away (with him).' (07-deluge, 61-70)
\end{exe}

The presence of inverse marking on \forme{tó-wɣ-tsɯm} `she took him up' in (\ref{ex:towGtsWm.tonWqambWmbjom}) is neither due to semantic nor syntactic factors. Rather, animate (in particular, human/sentient) referents in a particular passage (the subsection of a story, or a complete) are classified along a scale of obviativity. The main character(s) is/are ascribed proximate status, while characters which the narrator considers to be more secondary receive obviative status. 

Example (\ref{ex:towGtsWm.tonWqambWmbjom}) is taken from a flood story whose main character is a human boy (the only survivor of the flood). This character remains proximate in the whole story, and thus by comparison, the daughter of heaven (the other character found in the passage in \ref{ex:towGtsWm.tonWqambWmbjom}) is obviative. The proximate/obviative constrast on these animate referents is not visible when they occur as subjects with inanimate objects: the inanimate referents are necessarily always more obviative that any animate referent, and therefore direct forms are found. This contrast is only revealed when a verb takes two animate referents, namely the verb \forme{tó-wɣ-tsɯm}. The apparent shift between direct and inverse forms found in (\ref{ex:towGtsWm.tonWqambWmbjom}) is not an effect of the animate referent changing their obviation status across sentences, but is rather a consequence of the fact that Gyalrong languages lack a special conjugation class used with inanimate objects, such as the VTI (transitive inanimate) verbs in Algonquian.

The scale of obviation (\ref{ex:scale.toWGtsWm}) accounts for the distribution of direct and inverse marking in (\ref{ex:towGtsWm.tonWqambWmbjom}) (and in the rest of the story): direct marking is found when the subject is higher than the object on the scale, and inverse marking when it is lower.

\begin{exe}
\ex \label{ex:scale.toWGtsWm}
\glt human boy > daughter of heaven$_i$ > inanimate referents (\forme{qro ɯ-ŋga} `pigeon clothes' ; \forme{ɯ-mke} `her$_i$ neck')
\end{exe}

The degree of obviativity of each referent, while generally stable within a particular passage, can be fluid in longer narratives. For instance, in the story `Norbzang and Padma 'Od'bar', the analysis of direct vs. inverse forms reveals several scales. 

In (\ref{ex:Padma.nWGznARaR}), the subject is the king and the object the character Padma 'Od'bar, and inverse marking indicates that the latter is proximate.

\begin{exe}
\ex \label{ex:Padma.nWGznARaR}
\gll kɯɕnɯ-sŋi kɯɕnɤ-rʑaʁ nɯ́-wɣ-z-nɤʁaʁ ɲɯ-ŋu. \\
seven-day seven-night \textsc{ipfv}-\textsc{inv}-\textsc{caus}-have.a.good.time \textsc{sens}-be \\
\glt `(The king) let (Padma 'Od'bar) have a good time for seven days and seven nights.' (Norbzang 2005, 202)
\end{exe}

Examples (\ref{ex:Padma.pasWxCAt}) and (\ref{ex:Padma.nWwGCWrNonW}) have the secundative verbs \forme{pa-sɯxɕɤt} `he taught (it) to them' (§\ref{sec:ditransitive.secundative}) and \forme{nɯ́-wɣ-ɕɯ-rŋo-nɯ} `he lent it to them' (§\ref{sec:ditransitive.causative}) in direct and inverse forms, respectively. From these examples, one could further propose the scale (\ref{ex:Padma.scale}) showing the relative obviativity of the three groups of characters (Padma 'Od'bar, the three girls, and the king).

\begin{exe}
\ex \label{ex:Padma.pasWxCAt}
\gll tɤ-tɕɯ nɯ kɯ, [...] nɯnɯ tɕʰeme χsɯm nɯ lonba pa-sɯxɕɤt ɲɯ-ŋu. \\
\textsc{indef}.\textsc{poss}-boy \textsc{dem} \textsc{erg} { } \textsc{dem} girl three dem all \textsc{aor}:3\fl{}3$'$-teach \textsc{sens}-be \\
\glt `The boy (Padma 'Od'bar) taught the whole (mantra) to the three girls.' (Norbzang 2005, 380; Padma 'Od'bar > girls)
\end{exe}

\begin{exe}
\ex \label{ex:Padma.nWwGCWrNonW}
\gll  tɕendɤre ``jɤɣ" ti nɤ χsɤrzaŋ nɯ́-wɣ-ɕɯ-rŋo-nɯ ɲɯ-ŋu. \\
\textsc{lnk} be.possible:\textsc{fact} say:\textsc{fact} \textsc{lnk} artefact \textsc{aor}-\textsc{inv}-\textsc{caus}-borrow \textsc{sens}-be \\
\glt `(The king) said `yes' and lent the Gserbzang to (the three girls).'(Norbzang 2005, 406; girls > king)
\end{exe}

\begin{exe}
\ex \label{ex:Padma.scale}
\glt Padma 'Od'bar > girls > King
\end{exe}

However, while (\ref{ex:Padma.scale}) is indeed valid in some sections of the story, we also find examples such as (\ref{ex:Padma.nanWsNom}) and (\ref{ex:Padma.katCABnW}) with a verb in direct form, despite the fact that the subject (the king) is lower on the scale (\ref{ex:Padma.scale}) than the object (Padma 'Od'bar): an inverse configuration would be expected.

\begin{exe}
\ex \label{ex:Padma.nanWsNom}
\gll tɕendɤre rɟɤlpu kɯ na-nɯsŋom ɲɯ-ŋu. \\
\textsc{lnk} king \textsc{erg} \textsc{aor}:3\fl{}3$'$-envy \textsc{sens}-be \\
\glt `The king envied (Padma 'Od'bar).' (Norbzang 2005, 374)
\end{exe}

\begin{exe}
\ex \label{ex:Padma.katCABnW}
\gll tɕendɤre tɕetu ndɤre, pɤnmawombɤr ka-tɕɤβ-nɯ ɲɯ-ŋu.  \\
\textsc{lnk} up.there \textsc{lnk}  \textsc{anthr} \textsc{aor}:3\fl{}3$'$-burn-\textsc{pl} \textsc{sens}-be \\
\glt `(The king and his servants) burned Padma 'Od'bar up there.' (Norbzang 2005, 389)
\end{exe}

In (\ref{ex:Padma.nanWsNom}) and (\ref{ex:Padma.katCABnW}), a different scale (\ref{ex:Padma.scale2}) has to be posited.

\begin{exe}
\ex \label{ex:Padma.scale2}
\glt King > Padma 'Od'bar 
\end{exe}

Thus, a particular character of this story (the king), which has obviative status in (\ref{ex:Padma.nWGznARaR}) and (\ref{ex:Padma.nWwGCWrNonW}), becomes proximate in (\ref{ex:Padma.nanWsNom}) and (\ref{ex:Padma.katCABnW}), in a passage where he temporarily becomes the `main character'.

Within each of the scenes of a narrative, the obviativity status of each referent generally remains stable. Thus, in chains of verbs sharing coreferent pairs of subject and object, all of the verbs are either in direct or in inverse form. In (\ref{ex:YowGmbi.Yowjtshi.towGCWfka}) for instance, we find four verbs in inverse form in a row; each of them takes \forme{tɤ-wɯ} `grandfather, old man' as obviative subject and the main character of the story Nyima 'Odzer as proximate object.

\begin{exe}
\ex \label{ex:YowGmbi.Yowjtshi.towGCWfka}
\gll tɕendɤre tɤ-wɯ nɯ kɯ ɲɤ́-wɣ-ngɤjtsʰi tɕe, nɤki, tɯmgo ɲɤ́-wɣ-mbi, tɯ-ci ra ɲɤ́-wɣ-j-tsʰi tó-wɣ-ɕɯ-fka tɕe tɕe \\
\textsc{lnk} \textsc{indef}.\textsc{poss}-grandfather \textsc{dem} \textsc{erg} \textsc{ifr}-\textsc{inv}-give.to.eat.and.drink \textsc{lnk} \textsc{filler} food \textsc{ifr}-\textsc{inv}-give \textsc{indef}.\textsc{poss}-water \textsc{pl} \textsc{ifr}-\textsc{inv}-\textsc{caus}-drink \textsc{ifr}-\textsc{inv}-\textsc{caus}-be.full \textsc{lnk} \textsc{lnk} \\
\glt `The old man gave him food and drinks, gave him food to eat, water to drink, gave him enough for him to eat his fill.' (2011-05-nyima, 88-89)
\end{exe}

%tɕeri lɯlu nɯ kɯ ku-rtoʁ tɕe tɕe, 
%βʑɯ nɯ kɯ maka mɯ-pjɤ́-wɣ-nɤstu tɕe tɕendɤre,

However, alternations between direct and inverse can also be found across adjacent clauses with constant subjects and objects within a single passage. In (\ref{ex:RnaRna.Zo.chAwGndzandZi}), the noun \japhug{lɯlu}{cat}\footnote{This story is about humanized animals depicted as able to think and speak, and can be considered to be similar to human referents. } is the  proximate referent of the verbs \forme{ko-mɟa} and \forme{ɲo-mɟa}, since they are in direct form, but then a shift to inverse occurs and its preys (in the dual, the mouse and the sparrow) become proximate instead. The switch between direct and inverse marking gives native speakers flexibility to change the perspective of the narration.

\begin{exe}
\ex \label{ex:RnaRna.Zo.chAwGndzandZi}
\gll tɕe nɯ jamar tɕe ci nɯni kɯ ``lɯlu kɯ ɣɯ́-ndza-tɕi" ra mɯ-ɲɤ-sɯso-ndʑi kɯ, ``aʑo ɣɤŋgi-a nɤ aʑo ɣɤŋgi-a" ɲɤ-sɯso-ndʑi, tɕendɤre lɯlu nɯ kɯ ci ko-mɟa, ci ɲo-mɟa tɕe, ndɤre ʁnaʁna ʑo cʰɤ́-wɣ-ndza-ndʑi tɕe, \\
\textsc{lnk} \textsc{dem} about \textsc{lnk} \textsc{indef} \textsc{dem}:\textsc{du} \textsc{erg} cat \textsc{erg} \textsc{inv}-eat:\textsc{fact}-\textsc{1du} \textsc{pl} \textsc{neg}-\textsc{ifr}-think-\textsc{du} \textsc{erg} \textsc{1sg} be.right:\textsc{fact}-\textsc{1sg} \textsc{add} \textsc{1sg} be.right:\textsc{fact}-\textsc{1sg} \textsc{ifr}-think-\textsc{du} \textsc{lnk} cat \textsc{dem} \textsc{erg} one \textsc{ifr}:\textsc{east}-grab one \textsc{ifr}:\textsc{west}-grab \textsc{lnk} \textsc{lnk} both \textsc{emph} \textsc{ifr}-\textsc{inv}-eat-\textsc{du} \textsc{lnk} \\
\glt `The other two were not thinking `The cat will eat us', but rather `I am right, I am right', and the cat grabbed one on his left and one on his right, and both ended up eaten by him.' (lWlu 2002, 76-79)
\end{exe} 

In other cases, alternations between direct and inverse forms without change in subjects and objects appear to be due to hesitations on the part of the speaker. For instance, in (\ref{ex:GArAtnW.pjANu}) the speaker first chooses a direct \textsc{3pl}\fl{}3$'$ verb form \forme{ɣɤrɤt-nɯ}, with \forme{rɟɤlpu ɣɯ ɯ-ʁjoʁ nɯra} `the king's servants' being the proximate referent, but then switches to the inverse 3$'$\fl{}\textsc{3sg} form \forme{ɣɯ-tsɯm}, with the character Nyima 'Odzer as proximate referent instead.

\begin{exe}
\ex \label{ex:GArAtnW.pjANu}
\gll  rɟɤlpu ɣɯ ɯ-ʁjoʁ nɯra kɯ ɲimawozɤr, nɯnɯ ɣɤrɤt-nɯ pjɤ-ŋu, ɣɯ-tsɯm pjɤ-ŋu tɕeri, \\
king \textsc{gen} \textsc{3sg}.\textsc{poss}-servants \textsc{dem}:\textsc{pl} \textsc{erg}  \textsc{anthr} \textsc{dem} throw:\textsc{fact}-\textsc{pl} \textsc{ifr}.\textsc{ipfv}-be \textsc{inv}-take.away:\textsc{fact} \textsc{ifr}.\textsc{ipfv}-be \textsc{lnk} \\
\glt `As the king's servants were about to throw him (into the lake), to take him (there).' (Nyima wodzer2002, 114-115)
\end{exe}

%ɯ-pɯ nɯra hanɯni pɯ-ɬoʁ-nɯ tɕe tɕe ɕlaʁnɤɕlaʁ kɤ-nɯqambɯmbjom mɯ́j-cha-nɯ tɕe
%lɯlu kɯ kú-wɣ-ndo-nɯ ɲɯ-ŋu
%22-kumpGatCW, 58-59

\subsubsection{Pseudo-passive} \label{sec:pseudo.passive}
In some texts translated from Chinese, inverse verb forms are used in configurations with an inanimate object and an animate subject, an exception to the animacy constraints described in §\ref{sec:obviation.animacy}, according to which a direct form would be expected. This usage is similar to the generic subject construction (§\ref{sec:indexation.generic.tr}) where the inverse also occurs with animate subjects and inanimate objects, but semantically differs in that the subject is not generic, but rather an unknown referent. 

Example (\ref{ex:jowGkio.towGcW}) illustrates this construction with the verbs \forme{jó-wɣ-kio} and \forme{tó-wɣ-cɯ}, whose object is the jar's cover and subject the girl inside the jar (whose presence was unknown to the observer from whose point of view this passage is narrated). 

\begin{exe}
\ex \label{ex:jowGkio.towGcW}
\gll ɯ-tɕʰira [...] ɣɯ ɯ-fkaβ nɯnɯ jó-wɣ-kio. tɕe nɯ tó-wɣ-cɯ tɕe nɯtɕu tɕe tɕe li, nɯɕimɯma tɕe li,  ɯ-ŋga ra kɯ-mpɕɯ\redp{}mpɕɤr ci, ɯ-ʁzɯɣ ra kɯ-βdɯ\redp{}βdi ci, tɕʰemɤpɯ ci, iɕqʰa tɕʰira ɣɯ ɯ-ŋgɯ nɯtɕu to-nɯ-ɬoʁ.\\
\textsc{3sg}.\textsc{poss}-jar { } \textsc{gen} \textsc{3sg}.\textsc{poss}-cover \textsc{dem} \textsc{ifr}-\textsc{inv}-slide \textsc{lnk} \textsc{dem} \textsc{ifr}-\textsc{inv}-open \textsc{lnk} \textsc{dem}:\textsc{loc} \textsc{lnk} \textsc{lnk} again immediately \textsc{lnk} again \textsc{3sg}.\textsc{poss}-clothes \textsc{pl} \textsc{sbj}:\textsc{pcp}-\textsc{emph}\redp{}be.beautiful \textsc{indef} \textsc{3sg}.\textsc{poss}-appearance \textsc{pl} \textsc{sbj}:\textsc{pcp}-\textsc{emph}\redp{}be.nice  \textsc{indef} girl \textsc{indef} the.aforementioned jar \textsc{gen} \textsc{3sg}.\textsc{poss}-in \textsc{dem}:\textsc{loc} \textsc{ifr}:\textsc{up}-\textsc{auto}-come.out \\
\glt `The jar's cover slid over and opened, and a girl with beautiful clothes and a very nice appearance came immediately out of it.' (150827 tianluo-zh, 102-103)
\end{exe}

%,接着,一个衣着光鲜、五官端正的年轻姑娘从水缸里跳出来
The first two clauses translate Chinese \ch{发现水缸盖被推开了}{fāxiàn shuǐgānggài bèi tuīkāi le}{He realized that the jar's cover had been pushed open}. The inverse forms correspond here to a \zh{被} \forme{bèi} `passive' construction, and nearly all of the examples of inverse expressing unknown (rather than generic) animate subjects have been found in the translation of sentences containing this construction in Chinese. This pseudo-passive construction appears to be an effect of calque from the original, although Tshendzin, when asked about these examples, did not consider them to be ungrammatical. 

In addition, there are a few cases in translated texts like (\ref{ex:pjAwGlwoR.kWme.pjAme}) with inverse marking and unknown human subject (on the verb \forme{pjɤ́-wɣ-lwoʁ} `someone spilled it') where no \zh{被} \forme{bèi} construction is found in the original: this example translates \ch{他发现水罐空了}{tā fāxiàn shuǐguàn kōng le}{He realized that his water jar was empty}. The fact that the translation in this example is somewhat removed from the original would support the idea that the pseudo-passive inverse is a native grammatical construction.

\begin{exe}
\ex \label{ex:pjAwGlwoR.kWme.pjAme}
\gll  tɕendɤre ri, ku-rtoʁ tɕe ɯ-tɯ-ci nɯra pjɤ́-wɣ-lwoʁ tɕe kɯ-tu pjɤ-me qhe,  \\
\textsc{lnk} \textsc{lnk} \textsc{ipfv}-look \textsc{lnk} \textsc{3sg}.\textsc{poss}-\textsc{indef}.\textsc{poss}-water \textsc{dem}:\textsc{pl} \textsc{ifr}-\textsc{inv}-spill \textsc{lnk} \textsc{sbj}:\textsc{pcp}-exist \textsc{ifr}.\textsc{ipfv}-not.exist \textsc{lnk} \\
\glt `He$_i$ realized that someone had spilled the water (in his$_i$ gourd), and none was left.' (140505 liuhaohan zoubian tianxia-zh, 194)
\end{exe}

Thus, it is unclear at present whether the extension of the generic subject construction to express unknown subjects (the `pseudo-passive'), is the result of translatese, or whether it reflects a genuine but rare native construction.

\subsection{Transitive irregular verbs} \label{sec:irregular.transitive}
Unlike in Zbu \citep{gongxun14agreement}, there are no transitive verbs in Japhug with irregular \textsc{1sg} forms. The only irregularities related to person indexation are either found in generic forms or defective conjugations.

The verb \japhug{ti}{say} takes the prefix \forme{kɯ-} to express generic human subject as in (\ref{ex:tukWti.Nu}) as if it were an intransitive verb (§\ref{sec:intr.23}) instead of the inverse \forme{ɣɯ-} found on regular transitive verbs (§\ref{sec:indexation.generic.tr}); a form such as $\dagger$\forme{tú-wɣ-ti} would be incorrect. It is not the only irregularity of  \japhug{ti}{say}: this is also the only underived transitive verb with a separate stem II in Japhug (§\ref{sec:stem2.form}).


\begin{exe}
\ex   \label{ex:tukWti.Nu}
 \gll   tɕe ɕkrɤz ɣɯ ɯ-mat nɯ tʰɣe tu-kɯ-ti ŋu \\
 \textsc{lnk} oak \textsc{gen} \textsc{3sg}.\textsc{poss}-fruit \textsc{dem} acorn \textsc{ipfv}-\textsc{genr}-say be:\textsc{fact} \\
 \glt `The fruit of the oak is called an acorn.' (08-CkrAz, 45)
\end{exe}

Although the prefix \forme{kɯ-} expresses generic object on other transitive verbs (§\ref{sec:indexation.generic.tr}), in the case of \japhug{ti}{say} there is no ambiguity: the object of this verb is necessarily a complement clause or a noun, and can never have a first, second or generic person referent.  The use of \forme{kɯ-} to mark the transitive subject is reminiscent of Tshobdun, where both transitive and intransitive verbs mark generic subject by the same prefix \citep{sun14generic}. 

The defective verb \japhug{mɤ-xsi}{it is not known} only occurs in Factual Non-Past negative with a generic subject. This transitive verb takes nouns or complement clauses as object, usually with either an interrogative pronoun or a polar opposition as in (\ref{ex:tukWti.Nu}), sometimes with the dubitative (§\ref{sec:dubitative}). It cannot take any additional affix (including nominalization prefixes, §\ref{sec:nmlz.defective} and orientation preverbs, §\ref{sec:verbs.no.preverbs}), except for the autive \forme{nɯ-} in the form \forme{mɤ-nɯ-xsi}, as in (\ref{ex:mAnWxsi.ri}). 

\begin{exe}
\ex   \label{ex:Nu.maR.mAxsi}
 \gll  tu-ndze ŋu maʁ mɤ-xsi. \\
 \textsc{ipfv}-eat[III] be:\textsc{fact} not.be:\textsc{fact} \textsc{neg}-\textsc{genr}:know \\
 \glt `I don't know whether (bears) eat (humans) or not.' (21-pri, 120)
\end{exe}

\begin{exe}
\ex   \label{ex:mAnWxsi.ri}
 \gll a-pa, aki nɯ staʁlupa kɤ-βde ɯ-spa nɯ mɤ-nɯ-xsi ri, \\
 \textsc{1sg}.\textsc{poss}-father down \textsc{dem} born.the.year.of.the.tiger \textsc{obj}:\textsc{pcp}-throw \textsc{3sg}.\textsc{poss}-material \textsc{dem} \textsc{neg}-\textsc{auto}-\textsc{genr}:know \textsc{lnk} \\
 \glt `Father, I don't know whether the (boy) down there is someone born in the year of the tiger, (who must) to be thrown (into the lake as a sacrifice to the lake monster), but...' (2011-05-nyima, 154)
\end{exe}

The stem \forme{-xsi} contains the same root as \japhug{sɯz}{know} but without \forme{-z} suffix and with a \forme{x-} prefix which represents a fossilized allomorph of the regular generic \forme{kɯ-}. The verb \japhug{sɯz}{know} is attested in generic form, as in (\ref{ex:pWwGsWz.me.ri}). Although such examples are very rare, they show that \forme{mɤ-xsi} cannot be considered to be a suppletive form in the paradigm of \japhug{sɯz}{know}.

\begin{exe}
\ex   \label{ex:pWwGsWz.me.ri}
 \gll nɯ ma kɯ-pe tɕi pɯ́-wɣ-sɯz me, mɤ-kɯ-pe tɕi pɯ́-wɣ-sɯz me.  \\
 \textsc{dem} apart.from \textsc{sbj}:\textsc{pcp}-be.good also \textsc{aor}-\textsc{inv}-know not.exist:\textsc{fact} \textsc{neg}-\textsc{sbj}:\textsc{pcp}-be.good also \textsc{aor}-\textsc{inv}-know not.exist:\textsc{fact}  \\
`\glt `Apart from that, I don't know about the good or the bad things (that the butterfly does).' (26-qambalWla, 55)
\end{exe}

The verb \japhug{kɤtɯpa}{tell}, which presents a unique case of incorporation (§\ref{sec:lexicalized.object.participle}, §\ref{sec:incorp.denom}), only occurs in non-prefixed forms. Its paradigm is thus restricted to the Factual Non-Past (the only tense without orientation preverb, §\ref{sec:fact.morphology}), excluding second person forms (which always take a prefix, whether in the mixed or local domain), all inverse forms (which take the \forme{wɣ-} inverse prefix) and non-finite forms (§\ref{sec:nmlz.defective}). \tabref{tab:kAtWpa} (from \citealt{jacques12incorp}) presents the defective paradigm of \japhug{kɤtɯpa}{tell}, with the regular stem III \forme{kɤtɯpe} in \textsc{1sg}\fl{}3 and \textsc{3sg}\fl{}3$'$ forms, alongside that of \japhug{ndza}{eat}.


\begin{table}[H]
\caption{Paradigm of the verb \japhug{kɤtɯpa}{tell}}\label{tab:kAtWpa}
\begin{tabular}{lllll} 
\lsptoprule
Person & `to eat' & `to tell' & \\
\midrule
1\sg{}\fl{}3 &  \forme{ndze-a}& \forme{kɤtɯpe-a} \\
1\du{}\fl{}3 &  \forme{ndza-tɕi}& \forme{kɤtɯpa-tɕi} \\
1\pl{}\fl{}3 &  \forme{ndza-j}& \forme{kɤtɯpa-j} \\
\hline
2\sg{}\fl{}3 &  \forme{tɯ-ndze}& XX \\
2\du{}\fl{}3 &  \forme{tɯ-ndza-ndʑi}& XX  \\
2\pl{}\fl{}3&  \forme{tɯ-ndza-nɯ}& XX  \\
\hline
3\sg{}\fl{}3$'$ &  \forme{ndze}& \forme{kɤtɯpe} \\
3\du{}\fl{}3$'$ &  \forme{ndza-ndʑi}& \forme{kɤtɯpa-ndʑi} \\
3\pl{}\fl{}3$'$ &  \forme{ndza-nɯ}& \forme{kɤtɯpa-nɯ} \\
\lspbottomrule
\end{tabular}
\end{table}


\subsection{Transitive verbs with dummy subjects} \label{sec:transitive.dummy}
A handful of transitive light verbs (\japhug{βzu}{make} §\ref{sec:Bzu.lv}, \japhug{lɤt}{release} §\ref{sec:lAt.lv}, \japhug{tɕɤt}{take out} §\ref{sec:tCAt.lv}, \japhug{ndo}{take} §\ref{sec:ndo.lv}, \japhug{ta}{put} §\ref{sec:ta.lv} and \japhug{tsʰoʁ}{attach}) occur in the dummy subject construction, expressing natural phenomena, in particular meteorological ones.\footnote{These dummy subjects, which cannot be realized, correspond to expletive pronouns in languages which require overt subjects. }
  
In this construction, the verb only has one overt (always singular) absolutive nominal argument. The only form of the paradigm is the direct \textsc{3sg}\fl{}3$'$ (§\ref{sec:indexation.non.local}), with Stem III alternation in non-past tenses as in (\ref{ex:kWwGrum.kute}), and C-type preverbs in Aorist as in (\ref{ex:tANe.naBzu}), showing that the verb, despite its lack of ergatively-marked subject, is unambiguously transitive (§\ref{sec:transitivity.morphology}). 
 
 
\begin{exe}
\ex \label{ex:tANe.naBzu}
\gll tɤŋe na-βzu qʰe ɲɯ-me ɕti. \\
sun \textsc{aor}:3\fl{}3$'$-make \textsc{lnk} \textsc{ipfv}-not.exist be.\textsc{aff}:\textsc{fact} \\
\glt `It disappears when the sun appears.' (25-RmArYWG, 36)
\end{exe}

\begin{exe}
\ex \label{ex:kWwGrum.kute}
\gll  ɯ-taʁ ri kɯ-wɣrum ku-te, ku-kɯ-ta ci tu qhe, nɯnɯ li kʰɯrwum ŋu \\ 
\textsc{3sg}.\textsc{poss}-on \textsc{loc} \textsc{sbj}:\textsc{pcp}-be.white \textsc{ipfv}-put[III] \textsc{ipfv}-\textsc{sbj}:\textsc{pcp}-put \textsc{indef} exist:\textsc{fact} \textsc{lnk} \textsc{dem} again mold be:\textsc{fact} \\
\glt `A white thing grows on it (meat that has been left to rot), there is something that grows on it, this is also mold.' (20-sWrna, 61)
\end{exe}

The unique absolutive argument of this construction superficially resembles an object. However, in (\ref{ex:kWwGrum.kute}), note that the subject participle \forme{ku-kɯ-ta} `(the thing) that grows (on it)' (§\ref{sec:subject.participle.subject.relative}) is used to relativize this argument instead of an object participle (§\ref{sec:dummy.subj.object.relativization}) as would have been expected (see also §\ref{sec:subject.participle.other.relative}).

Moreover, despite the unambiguous morphological transitivity of the verb, the ergative is strictly impossible on the noun,  even in the case of personified natural forces. For instance, in the retellings of Aesop's story `The North Wind and the Sun', although  \japhug{akɯcʰoʁle}{east wind}\footnote{This term was chosen to translate `Northern Wind'.} occurs with the ergative with other verbs, and although here the North Wind is anthropomorphized and is described as blowing on purpose (to compete with the sun), the ergative is not accepted when the main verb is \japhug{βzu}{make} as in (\ref{ex:akWchoRle.toBzu}).

\begin{exe}
\ex \label{ex:akWchoRle.toBzu}
\gll akɯcʰoʁle nɯ to-βzu tɕe rcanɯ, \\
east.wind \textsc{dem} \textsc{ifr}-make \textsc{lnk} \textsc{unexp}:\textsc{deg} \\
\glt `The East (North) Wind blew.' (aesop feng he taiyang-zh, 9)
\end{exe}

In addition, the light verbs in this construction lack one of the seven morphological properties of transitive verbs (§\ref{sec:transitivity.morphology}): they select dental infinitives instead of bare infinitives (§\ref{sec:dental.inf}). Compare for instance (\ref{ex:tArtsa.tWBzu.toZa}) where \japhug{βzu}{make} occurs in the bare infinitive \forme{tɯ-βzu} with (\ref{ex:WBzu.loZa}), where the bare infinitive \forme{ɯ-βzu} is found instead, as \japhug{βzu}{make} is in a plain transitive construction.

\begin{exe}
\ex \label{ex:tArtsa.tWBzu.toZa}
\gll tɤrtsa kɯ-wxti tsa tɯ-βzu to-ʑa. \\
wave \textsc{sbj}:\textsc{pcp}-be.big a.little \textsc{inf}:II-make \textsc{ifr}-start \\
\glt `There started to be big waves (on the sea).' (140430 yufu he tade qizi-zh, 100)
\end{exe}

\begin{exe}
\ex \label{ex:WBzu.loZa}
\gll tɕendɤre cʰa ɯ-skɤt ɯ-βzu lo-ʑa.  \\
\textsc{lnk} alcohol \textsc{3sg}.\textsc{poss}-speech \textsc{3sg}.\textsc{poss}-\textsc{bare}.\textsc{inf}:make \textsc{ifr}-start \\
\glt `He started drunk-talking.' (150906 qingfeng-zh, 66)
\end{exe}

The list of the verbs which allow the dummy transitive subject construction is not closed. For instance the causative of \japhug{amɲɤm}{be even, be homogeneous} occurs with the noun \japhug{tɯ-mɯ}{sky, weather} in (\ref{ex:tWmW.thasAmYAm}).

\begin{exe}
\ex \label{ex:tWmW.thasAmYAm}
\gll  jisŋi tɯ-mɯ nɯ kɯ-fse tʰa-sɯ-ɤmɲɤm ʑo, tɤŋe tɕi mɯ-na-βzu, tɯ-mɯ tɕi mɯ-ka-lɤt \\
today \textsc{indef}.\textsc{poss}-sky \textsc{dem} \textsc{sbj}:\textsc{pcp}-be.like \textsc{aor}:3\fl{}3$'$-be.homogeneous \textsc{emph} sun also \textsc{neg}-\textsc{aor}:3\fl{}3$'$-make  \textsc{indef}.\textsc{poss}-weather also \textsc{neg}-\textsc{aor}:3\fl{}3$'$-release \\ 
\glt  `Today the weather was uniform, there was neither sun nor rain.' (elicited)
\end{exe}

\section{Ditransitive verbs} \label{sec:ditransitive}
Ditransitive verbs are trivalent verbs which follow the transitive conjugation (unlike trivalent semi-transitive verbs, §\ref{sec:semi.transitive.dative}). Both indirective and secundative transitive verbs are attested \citep{malchukov10ditransitive}, and causative verbs deriving from monotransitive verbs also belong to this category.


\subsection{Indirective} \label{sec:ditransitive.indirective}
Most non-derived ditransitive verbs in Japhug are indirective. The theme is in the absolutive and is indexed as an object. The recipient either takes dative (§\ref{sec:dative}) or genitive (§\ref{sec:gen.beneficiary}) flagging, as in (\ref{ex:zYArNo}), or is encoded as a possessive prefix on the object.

\begin{exe}
	\ex \label{ex:zYArNo}
	\gll kɯ-rɤrma ra nɯ-ɕki nɯtɕu, nɤki, kuxtɕo ci z-ɲɤ-rŋo, \\
	\textsc{sbj}:\textsc{pcp}-work \textsc{pl} \textsc{3pl}.\textsc{poss}-\textsc{dat} \textsc{dem}:\textsc{loc} \textsc{filler} basket \textsc{indef} \textsc{tral}-\textsc{ifr}-borrow \\
	\glt  `(The leopard) borrowed a basket from the workers.' (2002 qalakWcqraq, 43)
\end{exe}

The indirective category can be divided into four semantic subgroups. 

First, it comprises verbs of speech, such as \japhug{ti}{say}, \japhug{tʰu}{ask}, \japhug{ndʑɯ}{complain, accuse}  and \japhug{fɕɤt}{tell}, which mark the adressee with the dative.

Second, it includes verbs expressing gift or temporary transfer of objects such as \japhug{kʰo}{give, pass} (§\ref{sec:stem3.backformation}) and \japhug{pʰɯl}{offer} and  \japhug{rŋo}{borrow} and \japhug{nɤŋgɯ}{borrow}. Th first two encode the source as their subject while their recipients takes the dative case, whereas the latter encode the recipient as their subject, and their source appear in the dative. 

The two verbs translated as `borrow' differ in that the former (\forme{rŋo}) expresses the lending of an object which can be returned in its original shape to the original owner, while the latter (\forme{nɤŋgɯ}) is used with grain or money, for which an equivalent amount is given back instead of the original object (which has presumably already been consumed).\footnote{The meaning `lend' is expressed with the corresponding causatives, see §\ref{sec:ditransitive.causative}.} In addition, verbs of manipulation such as \japhug{ɣɯt}{bring} and \japhug{tsɯm}{take away} can optionally take a dative beneficiary and convey a meaning similar to verbs of giving, as in example (\ref{ex:Wphe.latsWm}).

\begin{exe}
	\ex \label{ex:Wphe.latsWm}
	\gll  tɕendɤre rɟɤlpu ɯ-pʰe la-tsɯm ɲɯ-ŋu,  \\
	\textsc{lnk}  king \textsc{3sg}.\textsc{poss}-\textsc{dat} \textsc{aor}:3\fl{}3:\textsc{upstream}-take.away \textsc{sens}-be \\
	\glt `He brought it (the shoe) to the king.'  (tWxtsa 2003, 16)
\end{exe}

Third, verbs expressing hitting or throwing can also be indirective, for instance \japhug{βde}{throw}, and also \japhug{lɤt}{release}, which occurs in many light verb constructions (§\ref{sec:light.verb}).

Fourth, some verbs of manipulation have an obligatory indirect argument expressing the goal, in particular the verb \japhug{rku}{put in} which selects the relator noun \japhug{ɯ-ŋgɯ}{inside} (§\ref{sec:WNgW}). For most verbs of manipulation the goal is optional  (§\ref{sec:secundative.monotransitive}).

The object of indirective verbs is nearly always third person, and local and inverse configurations are almost never attested (except for inverse generic forms, §\ref{sec:indexation.generic.tr}). For instance in (§\ref{sec:tAthe.jAG}), the recipient is \textsc{1sg} (`ask me'), but the verb \japhug{tʰu}{ask} has the imperative \textsc{2sg}\fl{}3 form (§\ref{sec:imp.morphology}) because the non-overt theme corresponds to the questions that the adressee is about to ask, and using a 2\fl{}1 configuration here would be non-sensical.

\begin{exe}
	\ex \label{sec:tAthe.jAG}
	\gll  tɤ-tʰe jɤɣ \\
	\textsc{imp}-ask[III] be.allowed:\textsc{fact} \\
	\glt `Ask [me your questions]!' (conversation, several attestations)
\end{exe}

Local or inverse configurations are attested with indirective verbs of giving in the corpus to refer to asking or giving a girl in marriage, as in (\ref{ex:nWwGkhoa}), as this is the most common situation when a human can occur as theme of a verb of this type.

\begin{exe}
	\ex \label{ex:nWwGkhoa}
	\gll  a-wa kɯ, tɯrme ɣɯ ɯ-rʑaβ nɯ́-wɣ-kʰo-a ɕti nɤ \\
	\textsc{1sg}.\textsc{poss}-father \textsc{erg} person \textsc{gen} \textsc{3sg}.\textsc{poss}-wife \textsc{aor}-\textsc{inv}-give-\textsc{1sg} be.\textsc{aff}:\textsc{fact} \textsc{sfp} \\
	\glt `My father has offered me in marriage to someone.' (150828 liangshanbo zhuyingtai-zh, 166)
\end{exe}

The only indirective verb of speech which is commonly used with a human theme is \japhug{ndʑɯ}{complain, accuse}: the object and the dative argument correspond to the persons about whom and to whom one complains, respectively. For instance in (§\ref{sec:Wɕki.tandZW}), the object is \textsc{2sg}, and the dative argument a third person referrent.

\begin{exe}
	\ex \label{sec:Wɕki.tandZW}
	\gll  aʑo <laoshi> ɯ-ɕki ta-ndʑɯ \\
	\textsc{1sg} professor \textsc{3sg}.\textsc{poss}-\textsc{dat} 1\fl{}2-accuse:\textsc{fact} \\
	\glt `I will issue a complain against you to the professor.' (150826 liangshanbo zhuyingtai-zh, 84)
\end{exe}

Indirective verbs of manipulation can be used with animate or even human objects, and do occur in local configurations, as in (§\ref{sec:axtu.WNgW.chWtarku}).

\begin{exe}
	\ex \label{sec:axtu.WNgW.chWtarku}
	\gll  aʑo a-xtu ɯ-ŋgɯ cʰɯ-ta-rku \\
	\textsc{1sg} \textsc{1sg}.\textsc{poss}-belly \textsc{3sg}.\textsc{poss}-in \textsc{ipfv}:\textsc{downstream}-1\fl{}2-put.in \\
	\glt `I will put you in my belly.' (140505 xiaohaitu-zh, 83)
\end{exe}

\subsection{Secundative} \label{sec:ditransitive.secundative}
Secundative verbs have two absolutive arguments, one of which (the recipient) is indexed on the verb, and the other one (the theme) is not. The theme is a semi-object, and presents some objectal properties (§\ref{sec:object.participle.relatives}, §\ref{sec:secundative.theme.relativization}).

\subsubsection{Inventory of secundative verbs}
Most secundative verbs express temporary or permanent giving: \japhug{mbi}{give} and \japhug{ɕtʂɯ}{entrust with}. Since these verbs index the recipient, they commonly occur in local configurations (§\ref{sec:indexation.local}), as in (§\ref{sec:kAkWCtsxWa}).

\begin{exe}
	\ex \label{sec:kAkWCtsxWa}
	\gll  sɯmat kɤ-kɯ-ɕtʂɯ-a \\
	fruit \textsc{aor}-2\fl{}1-entrust.with-\textsc{1sg} \\
	\glt `You had left me (a jar of) fruit.' (140516 yiguan ganlan-zh, 77)
\end{exe}

The verb \japhug{mbi}{give} can in addition take an essive adjunct (give as $X$, §\ref{sec:essive.abs}), as \forme{nɤ-rʑaβ} `your wife' in (\ref{ex:nArZaB.YWtambi}); in this sentence the object (theme) is \forme{a-me nɯ} `my daughter'.

\begin{exe}
	\ex \label{ex:nArZaB.YWtambi}
	\gll   a-me nɯ nɤ-rʑaβ ɲɯ-ta-mbi ŋu \\
	\textsc{1sg}.\textsc{poss}-daughter \textsc{dem} \textsc{2sg}.\textsc{poss}-wife \textsc{ipfv}-1$\rightarrow$2-give be:\textsc{fact} \\
	\glt `I will give you my daughter in marriage. (=I will give her to you as your wife)' (140428 yonggan de xiaocaifeng, 175)
\end{exe}

Most verbs of speech are indirective, but \japhug{sɯxɕɤt}{teach} has secundative alignment (see example \ref{ex:pjWtasWxCAt}, §\ref{sec:antipassive.ditransitive}). This verb is perhaps historically a causative (§\ref{sec:caus.sWG}), but it is not analyzable as such synchronically. 

The similative verb \japhug{stu}{do like} can also be considered to be secundative. It takes a semi-object designating the manner (generally a demonstrative), and its object refers to the entity (human, animal or inanimate) subjected to the action (with infinitival complement clauses however, there is no raising of the object from the complement, see §\ref{sec:similative.verb.complementation}) . In (\ref{ex:tAwGstuandZi}), this verb occurs in a \textsc{3du}\fl{}\textsc{1sg} configuration (§\ref{sec:double.number.indexation}), and takes as semi-object the demonstrative \forme{nɯra} `these (things)'. In serial verb constructions, it occurs with transitive verbs, sharing the subject and the object of its counterpart, but taking an additional (generally demonstrative) manner semi-object (§\ref{sec:svc.similative.verb}). 

\begin{exe}
	\ex \label{ex:tAwGstuandZi}
	\gll   a-pi ni kɯ nɯra tɤ́-wɣ-stu-a-ndʑi ndʐa ɕti ma, \\
	\textsc{1sg}.\textsc{poss}-elder.sibling \textsc{du} \textsc{erg} \textsc{dem}:\textsc{pl} \textsc{aor}-\textsc{inv}-do.like-\textsc{1sg}-\textsc{du} reason be.\textsc{aff}:\textsc{fact} \textsc{lnk} \\
	\glt `(This is because) my two brothers treated me like that.' (qachGa2003, 175)
\end{exe}


\subsubsection{The theme of secundative verbs} \label{sec:secundative.theme}
The theme of secundative verbs cannot be indexed, unlike that of indirective verbs, and first or second person themes are not appropriate with the verb \japhug{mbi}{give}. To express meanings such as `$X$ gave me/you to $Y$', either an indirective verb has to be used (see \ref{ex:nWwGkhoa} above), or alternatively the antipassive form \japhug{rɤmbi}{give to someone}. This antipassive form is the only verb form in Japhug lacking morphological transitivity (§\ref{sec:transitivity.morphology}, §\ref{sec:antipassive.ditransitive}) that can be used with inverse  and local person indexation, as shown by (\ref{ex:nWwGrAmbia}) and (\ref{ex:YAkWrAmbia}).

\begin{exe}
	\ex 
	\begin{xlist}
		\ex \label{ex:nWwGrAmbia}
		\gll  a-wa kɯ aʑo nɯ́-wɣ-rɤ-mbi-a \\
		\textsc{1sg}.\textsc{poss}-father \textsc{erg} \textsc{1sg} \textsc{aor}-\textsc{inv}-\textsc{apass}-give-\textsc{1sg} \\
		\glt `My father gave me away.' (elicited)
		\ex \label{ex:YAkWrAmbia}
		\gll nɤʑo kɯ aʑo ɲɤ-kɯ-rɤ-mbi-a \\
		\textsc{2sg} \textsc{erg} \textsc{1sg} \textsc{ifr}-2\fl{}1-\textsc{apass}-give-\textsc{1sg} \\
		\glt `You gave me away (without me knowing).' (elicited)
	\end{xlist}
\end{exe}

\subsection{Causative of transitive verbs} \label{sec:ditransitive.causative}
The sigmatic causative is the only valency-increasing derivation in Japhug which can be applied to transitive (unlike applicative §\ref{sec:applicative} and tropative §\ref{sec:tropative}) verbs. When the base verb is transitive, its causative form is ditransitive. I use the notation $C$\fl{}$C'$\fl{}$P$ to describe causative configurations with three participants: $C$ represents the causer (corresponding to the initiator of the causation, the argument that is added by the causative derivation), $C'$ the causee (corresponding to the subject of the base verb) and $P$ the patientive (corresponding to the object of the base verb); for instance, if the base verb is `help', a \textsc{1sg}\fl{}\textsc{3pl}\fl{}\textsc{2sg} configuration can be interpreted as `I made them help you.'

Lexicalized causatives are like underived secundative verbs, and index the recipient as the object. This category includes \japhug{jtsʰi}{give to drink}, which derives from the monotransitive verb \japhug{tsʰi}{drink}, and \japhug{ɕɯrŋo}{lend}, which comes from the indirective verb \japhug{rŋo}{borrow} (see \ref{ex:YWkWCWrNoj} below for a more detailed discussion on this derivation). 

Examples (\ref{ex:YWkWjtshitCi}) and (\ref{ex:YWkWCWrNoj}) illustrate the 2\fl{}1\fl{}3 configuration with the verbs \japhug{jtsʰi}{drink} and \japhug{ɕɯrŋo}{lend}, expressed in the same way as the local 2\fl{}1 configurations of a monotransitive verb (§\ref{sec:indexation.local}).

\begin{exe}
	\ex \label{ex:YWkWjtshitCi}
	\gll a-wɯ tɯ-ci ɲɯ-kɯ-j-tsʰi-tɕi ɯ́-jɤɣ? \\
	\textsc{1sg}.\textsc{poss}-grandfather \textsc{indef}.\textsc{poss}-water \textsc{ipfv}-2\fl{}1-\textsc{caus}-drink-\textsc{1du} \textsc{qu}-be.allowed:\textsc{fact} \\
	\glt `Grandfather, could you give us water to drink?' (nyima wodzer 2002, 76)
\end{exe}

In the case of non-lexicalized sigmatic causatives, the status of the causee is more complex, since it can optionally receive ergative flagging (§\ref{sec:causee.kW}), and cannot be relativized like an object (see §\ref{sec:instrument.relativization}). The direct object of such verbs can also be relativized using constructions unavailable for the direct object of monotransitive verbs (§\ref{sec:object.causative.relativization}).

In addition, unlike secundative verbs, triactantial causatives can either index the causee or the argument corresponding to the object of the base verb, depending on their person: if the causee or object is first or second person, and the other argument is third person, the first or second person will be indexed regardless of its syntactic function. 

For instance, all three examples (\ref{ex:pWkWsWxtCia}), (\ref{ex:kWsWRndWa}) and (\ref{ex:YWkWznWntshoa}) show 2\fl{}1 indexation, but the first two has a 2\fl{}3\fl{}1 configuration, and index the patientive argument as the object, while the last one is an instance of a 2\fl{}1\fl{}3 configuration, and it indexes the causee as object (see also §\ref{sec:sig.caus.morphosyntax} and \ref{ex:pGAtCW.tAmdzu}, §\ref{sec:essive.abs} for additional examples of the same type). The prevalence of first and second persons over third persons in this context might support the person hierarchy (\ref{ex:hierarchy.3}) postulated in (§\ref{sec:direct-inverse}).

\begin{exe}
	\ex \label{ex:pWkWsWxtCia}
	\gll  aʑo tɯ-mɯ kɯ pɯ-kɯ-sɯ-χtɕi-a, tɤndʐo nɯ! \\
	\textsc{1sg} \textsc{indef}.\textsc{poss}-weather \textsc{erg} \textsc{aor}-2\fl{}1-\textsc{caus}-wash-\textsc{1sg} cold \textsc{sfp} \\
	\glt `You caused me to be drenched by the rain, it is so cold!' (2014-kWlAG, 158)
\end{exe}

\begin{exe}
	\ex \label{ex:kWsWRndWa}
	\gll  nɤʑo tɤ-ndze ma alo ma-lɤ-tɯ-tsɯm ma tʰa li kɯ-sɯ-ʁndɯ-a \\
	\textsc{2sg} \textsc{imp}-eat[III] \textsc{lnk} upstream \textsc{neg}-\textsc{imp}:\textsc{upstream}-2-take.away \textsc{lnk} later again 2\fl{}1-\textsc{caus}-hit-\textsc{1sg} \\
	\glt  `Eat it, don't take it up there, you would cause me to be beaten again.' (2003-kWBRa, 71-72)
\end{exe}

\begin{exe}
	\ex \label{ex:YWkWznWntshoa}
	\gll aʑo ɕɤrɯ ɲɯ-kɯ-z-nɯntsʰo-a \\
	\textsc{1sg} meat \textsc{ipfv}-2\fl{}1-\textsc{caus}-eat-\textsc{1sg} \\
	\glt `You make me eat the bones.' (2014-kWlAG, 211)
\end{exe}

While indexation is ambiguous in such cases, ergative flagging helps distinguish between 2\fl{}3\fl{}1  and 2\fl{}1\fl{}3 configurations: the ergative is present on the causee in (\ref{ex:pWkWsWxtCia}), and absent on the patientive argument in (\ref{ex:YWkWznWntshoa}) (see also §\ref{sec:causee.kW}).

Configurations with both causee and patientive argument being first or second person (3\fl{}2\fl{}1 and 3\fl{}1\fl{}2) are not attested in the corpus,\footnote{See however (\ref{ex:nACki.jAwGsWGWta}) below, quadrivalent configuration with a second person beneficiary in the dative and 3\textsc{sg}\fl{}1\textsc{sg} verb indexation.} and are not easy to elicit. The elicited examples (\ref{ex:WZo.kW.pWwGsWmtoa}) and (\ref{ex:WZo.kW.pWtWwGsWmto}) show that 3\fl{}1\fl{}2 and 3\fl{}2\fl{}1 configurations are expressed using mixed 3\fl{}1 and 3\fl{}2 forms, respectively. In other words, the causee is indexed as object, and the patientive argument is not indexed.

\begin{exe}
	\ex 
	\begin{xlist}
		\ex \label{ex:WZo.kW.pWwGsWmtoa}
		\gll ɯʑo kɯ nɤʑo pɯ́-wɣ-sɯ-mto-a \\
		\textsc{3sg} \textsc{erg} \textsc{2sg} \textsc{ipfv}-\textsc{inv}-\textsc{caus}-see-\textsc{1sg} \\
		\glt `He made me see you.' (3\textsc{sg}\fl{}1\textsc{sg}\fl{}2\textsc{sg}, elicited)
		\ex \label{ex:WZo.kW.pWtWwGsWmto}
		\gll ɯʑo kɯ aʑo pɯ-tɯ́-wɣ-sɯ-mto \\
		\textsc{3sg} \textsc{erg} \textsc{1sg} \textsc{aor}-2-\textsc{inv}-\textsc{caus}-see \\
		\glt `He made you see me.'  (3\textsc{sg}\fl{}2\textsc{sg}\fl{}1\textsc{sg} elicited)
	\end{xlist}
\end{exe}

The causative derivation of indirective verbs has several different effects on argument structure. In the case of the lexicalized causative \japhug{ɕɯrŋo}{lend}, the relationship between its three arguments and those of the base verb \japhug{rŋo}{borrow} can be summarized as follows (compare \ref{ex:YWkWCWrNoj} with \ref{ex:zYArNo} in §\ref{sec:ditransitive.indirective}).

\begin{itemize}
	\item The subject of \japhug{rŋo}{borrow}, the receiving person, corresponds to the object of \japhug{ɕɯrŋo}{lend}.
	\item The object of \forme{rŋo}, the thing that is borrowed, corresponds to the semi-object of \japhug{ɕɯrŋo}{lend}.
	\item The dative oblique argument of \forme{rŋo}, the giving person, corresponds to the subject of \japhug{ɕɯrŋo}{lend}, which is semantically both causer and source.
\end{itemize}

\begin{exe}
	\ex \label{ex:YWkWCWrNoj}
	\gll  wortɕhi ʑo, nɤ-χsɤrzaŋ ɲɯ-kɯ-ɕɯ-rŋo-j \\
	please \textsc{emph} \textsc{2sg}.\textsc{poss}-magical.object \textsc{ipfv}-2\fl{}1-\textsc{caus}-borrow-\textsc{1pl} \\
	\glt `Please, lend us you magical object.' (2005 Kunbzang, 395)
\end{exe}

This `inversive' causative derivation reverses the direction of the borrowing action and modifies the argument structure and the alignment (indirective to secundative) without adding any new referent, by merging the source and causer roles, interpreting \forme{ɕɯrŋo} as `$X_i$ causes $Y$ to borrow $Z$ from himself$_i$'. Other cases of this type of `inversive' causative are presented in §\ref{sec:sig.caus.inversive}.

For most indirective verbs however, recipients and beneficiaries are marked with the dative and/or with a coreferent possessive prefix. Thus, the causative derivation does turn them into absolutive patientive arguments. For instance, the beneficiary of the verb \japhug{sɯɣɯt}{send, cause to bring} is in the dative as that of its base verb \japhug{ɣɯt}{bring}, as shown by (\ref{ex:nACki.jAwGsWGWta}), where dative marking on the \textsc{2sg} is obligatory.

\begin{exe}
	\ex \label{ex:nACki.jAwGsWGWta}
	\gll   aʑo a-βdaχpu nɯ kɯ pjɤ-nɯmto tɕe tɕendɤre nɤ-ɕki jɤ́-wɣ-sɯ-ɣɯt-a ŋu tɕe, \\
	\textsc{1sg} \textsc{1sg}.\textsc{poss}-master \textsc{dem} \textsc{erg} \textsc{ifr}-find \textsc{lnk} \textsc{lnk} \textsc{2sg}.\textsc{poss}-\textsc{dat} \textsc{aor}-\textsc{inv}-\textsc{caus}-bring-\textsc{1sg} be:\textsc{fact} \textsc{lnk} \\
	\glt `My master found it and sent me to bring it to you.' (140513 qianshang he xiaotou-zh, 72)
\end{exe}


\subsection{Causative of semi-transitive verbs} \label{sec:semi.transitive.causative}
Causative verbs derived from semi-transitive ones (§\ref{sec:semi.transitive}) show secundative alignment: the causee is treated as object, and the semi-object of the base verb as an absolutive oblique, not indexed on the verb. For instance, the causative \japhug{sɤrmi}{give a name} (§\ref{sec:sig.caus.irregular.other}) of the verb \japhug{rmi}{be called} selects the entity to which a name is given as direct object, the \textsc{1pl} in example (\ref{ex:tuwGsArmij}).

\begin{exe}
	\ex \label{ex:tuwGsArmij}
	\gll   sɤrndzu cʰo tatsʰi ra kɯ sɤŋu-pɯ tu-nɯ-ti-nɯ tɕe, ʁdɯrɟɤt ra sɤŋu-pɯ tú-wɣ-sɤrmi-j ŋu. \\
	\textsc{topo} \textsc{comit}  \textsc{topo} \textsc{pl} \textsc{erg} pl.p-person.from \textsc{ipfv}-\textsc{auto}-say-\textsc{pl} \textsc{lnk} \textsc{topo} \textsc{pl} \textsc{topo}-person.from \textsc{ipfv}-\textsc{inv}-give.name-\textsc{1pl} be:\textsc{fact} \\
	\glt `Those from Gsar-rdzong and Da-tshang say `Sangu people', they name us, (those) of Gdong-brgyad, `Sangu people'.' (23-tCAphW, 12)
\end{exe}


\section{Labile verbs} \label{sec:lability}
Most Japhug verbs strictly follow either the intransitive, or the transitive conjugations, and voice alternations such as causative (§\ref{sec:sig.causative}, §\ref{sec:velar.causative}), applicative (§\ref{sec:applicative}) and antipassive (§\ref{sec:antipassive}) are needed to change their valency. A limited number of labile verbs are compatible with both conjugations; three groups can be distinguished: plain labile, labile with oblique arguments and labile with semi-object.

\subsection{Transitive-intransitive labile verbs} \label{sec:labile.tr-intr}

\subsubsection{Morphosyntactic properties} \label{sec:lability.morphosyntax}
Plain labile verbs are compatible with both the intransitive and the transitive conjugations. When conjugated intransitively, they only have one argument (no semi-object or oblique argument). When conjugated transitively (following the features in §\ref{sec:transitivity.morphology}), their subject corresponds to the intransitive subject, and the added argument is the object, except for a handful of examples (§\ref{sec:lability.pass}).

 Compare for instance the intransitive and transitive use of the verb \japhug{βlɯ}{burn} in (\ref{ex:BlW.tr.intr}). In the (\ref{ex:tABlWa}), the verb has \textsc{1sg} indexation, and since it lacks the 12\textsc{sg}\fl{}3 past \forme{-t} suffix (§\ref{sec:indexation.mixed}), cannot be a transitive verb form (§\ref{sec:transitivity.morphology}). This intransitive verb form cannot take an object, and means `make a fire'. In (\ref{ex:smi.tABlWta}), the verb has the \forme{-t} suffix and is therefore a \textsc{1sg}\fl{}3 transitive configuration (§\ref{sec:indexation.mixed}), and can take an overt object (here \japhug{smi}{fire}, but it can also occur with a noun referring to the object being burned).

\begin{exe}
\ex \label{ex:BlW.tr.intr}
\begin{xlist}
\ex  \label{ex:tABlWa}
\gll tɤ-βlɯ-a  \\
\textsc{aor}-burn-\textsc{1sg} \\
\glt  `I made a fire'. (elicited; \textsc{1sg}:\textsc{intr})
\ex \label{ex:smi.tABlWta}
\gll  smi tɤ-βlɯ-t-a \\
fire \textsc{aor}-burn-\textsc{pst}:\textsc{tr}-\textsc{1sg} \\
\glt  `I made a fire'. (elicited; \textsc{1sg}\fl{}3)
\end{xlist}
\end{exe}

\tabref{tab:labile1} gives additional examples of labile verbs with open syllable stems in transitive and intransitive conjugations.

\begin{table}[H]
\caption{Aorist \textsc{1sg}(\fl{}3) of labile verbs}\label{tab:labile1}
\begin{tabular}{lllll} 
\lsptoprule
Verb & Intransitive form &Transitive  form \\
\midrule
\japhug{βlɯ}{burn} & \forme{tɤ-βlɯ-a} & \forme{tɤ-βlɯ-t-a} \\
\japhug{ɕlu}{plough} & \forme{tɤ-ɕlu-a} & \forme{tɤ-ɕlu-t-a} \\
\japhug{fɕi}{forge} &\forme{tʰɯ-fɕi-a} & \forme{tʰɯ-fɕi-t-a} \\  
\japhug{mɯrkɯ}{steal} & \forme{tɤ-mɯrkɯ-a} & \forme{tɤ-mɯrkɯ-t-a} \\
\japhug{nɯmbrɤpɯ}{ride} & \forme{tɤ-nɯmbrɤpɯ-a} & \forme{tɤ-nɯmbrɤpɯ-t-a} \\
\japhug{nɤmɲo}{watch} & \forme{kɤ-nɤmɲo-a} & \forme{kɤ-nɤmɲo-t-a} \\
 \lspbottomrule
\end{tabular}
\end{table}

All criteria listed in §\ref{sec:transitivity.morphology} (when applicable) are congruent to distinguish between the transitive and intransitive forms of labile verbs;  \tabref{tab:labile.test} presents the application of the seven tests to the verb  \japhug{βlɯ}{burn}.

\begin{table}[H]
\caption{Transitivity tests}\label{tab:labile.test}
\begin{tabular}{lllll} 
\lsptoprule
&  & Intransitive   &Transitive form (with \\
 & &form& \japhug{smi}{fire} as object) \\
  \midrule
1&C-type preverb, \textsc{aor}:\textsc{3sg}(\fl{}3$'$)& \forme{tɤ-βlɯ} & \forme{ta-βlɯ}  \\
2&Stem III, \textsc{ipfv}:\textsc{3sg}(\fl{}3$'$)& \forme{tu-βlɯ} & \forme{tu-βli}  \\
3&Dental/bare \textsc{inf} & \forme{tɯ-βlɯ (to-ʑa)} & \forme{ɯ-βlɯ (to-ʑa)}  \\ %revoir
4&Subject participle & \forme{kɯ-βlɯ} & \forme{ɯ-kɯ-βlɯ}  \\
5&\forme{-t} suffix, \textsc{aor}:\textsc{1sg}(\fl{}3)& \forme{tɤ-βlɯ-a} & \forme{tɤ-βlɯ-t-a} \\
6&Progressive& \forme{ɲɯ-βlɯ} & \forme{ɲɯ-ɤsɯ-βlɯ}  \\
7&Generic &  \forme{tu-kɯ-βlɯ} & \forme{tú-wɣ-βlɯ}  \\
\lspbottomrule
\end{tabular}
\end{table}

Tests 2 and 5 are not operational with verbs whose stem ends in close syllable, but at least test 1 (in the case of \japhug{taʁ}{weave} in \ref{ex:taR.tr.intr}) is always applicable. Some labile verbs can even be used with the non-periphrastic Past Imperfective (§\ref{sec:pst.ifr.ipfv}), as in (\ref{ex:tCheme.pWtaR}).  The presence or absence of the ergative on third person overt subjects is also a useful confirmation of the valency of the verb; for instance, in (\ref{ex:tCheme.pWtaR}) the subject \japhug{tɕʰeme}{girl} lacks ergative marking, confirming the fact that \japhug{taʁ}{weave} is used here intransitively.
 
\begin{exe}
\ex \label{ex:taR.tr.intr}
\begin{xlist}
\ex \label{ex:tCheme.pWtaR}
\gll  tɕʰeme ci pɯ-taʁ ɲɯ-ŋu,  \\
girl \textsc{indef} \textsc{pst}.\textsc{ipfv}-weave \textsc{sens}-be \\
\glt `A girl was weaving.' (tWxsta2003, 28)
\ex \label{ex:mWntoR.thataR}
\gll  nɯnɯ mɯntoʁ nɯ tʰa-taʁ tɕe, \\
\textsc{dem} flower \textsc{dem} \textsc{aor}:3\fl{}3$'$-weave \textsc{lnk} \\
\glt `(When) he wove this pattern, ...' (150825 huluwa, 98)
\end{xlist}
\end{exe}

For verbs whose objects cannot be human for semantic reasons, test 7 (as in \ref{ex:Clu.tr.intr})  can also be used to distinguish between intransitive and transitive uses;\footnote{Otherwise, a generic \forme{kɯ-} could be generic object, see §\ref{sec:indexation.generic.tr}.} if the object can be human, then testing for local configurations (§\ref{sec:indexation.local}) is possible.

\begin{exe}
\ex \label{ex:Clu.tr.intr}
\begin{xlist}
\ex  \label{ex:tukWClu}
\gll cɯ ɲɯ́-wɣ-pʰɯt, tu-kɯ-ɕlu, \\
stone \textsc{ipfv}-\textsc{inv}-take.out \textsc{ipfv}-\textsc{inv}-plough \\
\glt  `One has to take out the stones, to plough,' (2010, 10, 11)
\ex  \label{ex:luwGClu}
\gll tɯ-ji nɯ lú-wɣ-ɕlu tɕe \\
\textsc{indef}.\textsc{poss}-field \textsc{dem} \textsc{ipfv}-\textsc{inv}-plough \textsc{lnk} \\
\glt  `One ploughs the fields, and ...' (07-tWsqar, 21)
\end{xlist}
\end{exe}

\subsubsection{Classification of labile verbs} \label{sec:lability.categories}
In addition to those in \tabref{tab:labile1}, the following verbs are labile: \japhug{ɣndʑɯr}{grind}, \japhug{lɤɣ}{herd}, \japhug{nbraʁ}{loosen the earth}, \japhug{ntʂu}{weed with a hoe}, \japhug{nɤre}{laugh}, \japhug{nɯɣɤja}{talk back, oppose}, \japhug{nɯkʰɤja}{talk back, oppose}, \japhug{nɯproʁmba}{imitate}, \japhug{rŋu}{parch},  \japhug{sɯlaʁrdɤβ}{kick (with its forelimbs)}, \japhug{sɯqartsɯ}{kick}, \japhug{sɯso}{think}, \japhug{taʁ}{weave}, \japhug{tɤβ}{thresh}. 

%\japhug{sɤro}{put on a shelf},
 
The majority of plain labile verbs denote actions modifying the substance or shape of a material (\japhug{ɣndʑɯr}{grind}, \japhug{ɕlu}{plough} etc), in particular activities related to agriculture and traditional trades. When used intransitively, these verbs do not specify any object, and generally express atelic actions. For instance in (\ref{ex:tCheme.pWtaR}), \forme{pɯ-taʁ} means `do weaving', without intrinsic endpoint and without reference to a particular piece of cloth.

Another category of labile verbs express action negatively affecting people (such as \japhug{nɯɣɤja}{talk back, oppose} etc), some with animal subjects (\japhug{sɯlaʁrdɤβ}{kick (with its  forelimbs)} etc).  These verbs take humans as objects in the transitive conjugation;\footnote{The verb \japhug{mɯrkɯ}{steal}, though adversely affecting the persons whose possession are stolen, does not belong to this category since it takes the possession as object, not the victim of the theft.} when used intransitively, they express a general propensity of the subject to do these negative actions.

%\japhug{mɯrkɯ}{steal}  
%\japhug{nɯmbrɤpɯ}{ride}  
%\japhug{nɤmɲo}{watch}  

The perception verb \japhug{mto}{see} also has an intransitive stative use, meaning `have sharp eyesight', taking the noun \japhug{tɯ-mɲaʁ}{eye} as subject, and expressing the experiencer as subject possessor). Example (\ref{ex:WmYaR.YWmto}) shows that the verb \forme{mto} is conjugated intransitively (without stem III alternation, otherwise \forme{ɲɯ-mtɤm} `he sees it' would be expected). Some verbs derived from the root \forme{mto}, such as the velar causative \japhug{ɣɤmto}{cause to recover eyesight} (§\ref{sec:velar.causative.vs.sigmatic.causative}), come from this stative intransitive use rather than the more common transitive one.

\begin{exe}
\ex \label{ex:WmYaR.YWmto}
\gll ɯ-mɲaʁ nɯ wuma ɲɯ-mto ɲɯ-ʂa ɲɯ-ŋu nɤ́ma, \\
\textsc{3sg}.\textsc{poss}-eye \textsc{dem} really \textsc{sens}-be.sharp \textsc{sens}-be.strong \textsc{sens}-be \textsc{sfp} \\
\glt `The (eagle) has a sharp eyesight.' 140522 Kamnyu zgo, 226
\end{exe}


\subsubsection{Accusative lability} \label{sec:lability.apass}
Most labile verbs in intransitive use have a meaning and syntactic function reminiscent of antipassive (§\ref{sec:antipassive}) derivations, and share with them the ability to take the non-periphrastic Past Imperfective \forme{pɯ-} (§\ref{sec:antipassive.pst.ipfv}). Nevertheless, it is unclear whether  the transitive use of labile verbs is primary; one could also consider the intransitive use to be the basic function, and compare their transitive use to the Applicative derivation (§\ref{sec:applicative}) instead. Since the alignment of the subjects between the transitive and intransitive uses of these verbs follows nominative-accusative, it is referred to as `accusative lability' in this work.\footnote{The term `agent-preserving lability' is less felicitous, since the subjects of labile verbs are not always agents.}
 
Two plain labile verbs,  \japhug{nɤre}{laugh} and  \japhug{sɯso}{think}, have an antipassive form.\footnote{This feature is shared with the semi-transitive labile verb \japhug{sɤŋo}{listen} see §\ref{sec:semi.tr.labile}. On the other hand, no labile verb is compatible with applicative derivation. } These two verbs also differ from other labile verbs in having idiosyncratic semantic differences in their intransitive and transitive uses. 


The verb \japhug{nɤre}{laugh} is a denominal verb deriving from the inalienable noun \japhug{tɤ-re}{laugh}. In its transitive use  `laugh at, mock', it encodes the stimulus as direct object. Both the intransitive meaning and the transitive one are well-attested functions of the \forme{nɯ-/nɤ-} denominal prefix: parallel examples include \japhug{nɤʁaʁ}{have a good time} (from \japhug{tɤ-ʁaʁ}{good time}, §\ref{sec:denom.intr.nW}) for the intransitive use of \forme{nɤ-} and \japhug{nɤmbrɯ}{get angry against} (from \japhug{tɤ-mbrɯ}{anger}, §\ref{sec:denom.tr.nW}) for it transitive use. Therefore, the most likely way to account for the lability of \forme{nɤre} is to posit that it represents the conflation of two denominal derivations from the same base noun.

The antipassive form \japhug{sɤnɤre}{laugh at people} is semantically connected with the meaning of the transitive use, and semantically quite different from the intransitive use of  \japhug{nɤre}{laugh}.

The intransitive use of \japhug{sɯso}{think} is restricted to the meaning `in $X$'s opinion', as in (\ref{ex:pjWsWsoa}),\footnote{The intransitivity of \japhug{sɯso}{think} in (\ref{ex:pjWsWsoa}) is shown by the absence of Stem III, which would be expected in the Imperfective \textsc{1sg}\fl{}3 (§\ref{sec:ipfv.morphology}). Note that the orientation \textsc{downwards} is selected, instead of \textsc{westwards} in the transitive use (§\ref{sec:orientation.lability}). } and has no semantic overlap with the antipassive \japhug{rɯsɯso}{think}, `ponder' (\ref{ex:pjWrWsWso}).\footnote{Note that this antipassive form is irregular, §\ref{sec:antipassive.rA}.}  

\begin{exe}
\ex \label{ex:pjWsWsoa}
\gll  tɕe nɯ aj pjɯ-sɯso-a tɕe nɯnɯ tɯ-tɯpʰu ɲɯ-ŋu-ndʑi tɕe \\
\textsc{lnk} \textsc{dem} \textsc{1sg} \textsc{ipfv}-think-\textsc{1sg} \textsc{lnk} \textsc{dem} one-species \textsc{sens}-be-\textsc{du} \textsc{lnk} \\
\glt  `In my opinion, these two (animals belong to) the same species.' (20-ldWGi, 45)
\end{exe}

\begin{exe}
\ex \label{ex:pjWrWsWso}
\gll  pjɯ-rɯ-sɯso nɤ pjɯ-rɯ-sɯso tɕe, \\
\textsc{ipfv}-\textsc{apass}-think \textsc{add} \textsc{ipfv}-\textsc{apass}-think \textsc{lnk} \\
\glt `He thought about it over and over.' (02-deluge2012, 63)
\end{exe}

In the case of \japhug{sɯso}{think},  it appears that the very restricted intransitive use is derived from the transitive one. 

\subsubsection{Ergative lability} \label{sec:lability.pass}
A handful of labile verbs have ergative (or passive-like) lability: the subject of the verb when used intransitively corresponds to its object when used transitively. 

The clearest example is provided by the denominal verb \forme{nɯʁjoʁ} from \japhug{ʁjoʁ}{servant}, which  means `give orders to' in transitive use (from `treat as a servant', §\ref{sec:denom.tr.nW}), and `work as a servant' in intransitive use. In (\ref{ex:YWnWRjoR.tuzrAme}), the transitive  \forme{nɯʁjoʁ} shares its subject and object with causativized verbs, while in (\ref{ex:zYWnWRjoR.ZYWnAme}) the intransitive subject of \forme{nɯʁjoʁ} has the same referent as the subject of \forme{z-ɲɯ-nɤme}, as shown by the presence of a translocative echo (§\ref{sec:echo.multiple.AM}) on both verbs (which cannot target the direct object, §\ref{sec:AM.argument.motion}). 

\begin{exe}
\ex \label{ex:YWnWRjoR.tuzrAme}
\gll tɕʰeme nɯ tɕe tɕe ɯ-kʰa nɯtɕu ko-z-rɤʑi tɕe, ɲɯ-nɯʁjoʁ tu-z-rɤme pjɤ-ŋu. \\
girl \textsc{dem} \textsc{lnk} \textsc{lnk} \textsc{3sg}.\textsc{poss}-house \textsc{dem}:\textsc{loc} \textsc{ifr}-\textsc{caus}-stay \textsc{lnk} \textsc{ipfv}-give.orders \textsc{ipfv}-\textsc{caus}-work[III] \textsc{ifr}.\textsc{ipfv}-be \\
\glt `The witch kept the girl in the house, and gave her orders and put her to work.' (140507 tangguowu-zh, 110-111)
\end{exe}


\begin{exe}
\ex \label{ex:zYWnWRjoR.ZYWnAme}
\gll sŋi qʰe z-ɲɯ-nɯʁjoʁ tɕe kɯβʁa ra nɯ-ma z-ɲɯ-nɤme \\
day \textsc{lnk} \textsc{tral}-\textsc{ipfv}-work.as.servant \textsc{lnk} noble \textsc{pl} \textsc{3pl}.\textsc{poss}-work \textsc{tral}-\textsc{ipfv}-do.work[III] \\
\glt `In the day, he would work as a servant, and do work for the nobles.' (150828 donglang, 10)
\end{exe}


Since the \forme{nɯ-/nɤ-} denominal prefixes occur with similar  meanings (`treat as $X$' and `become/serve as $X$) on other base nouns (see §\ref{sec:denom.intr.nW} and §\ref{sec:denom.tr.nW}), the lability of \forme{nɯʁjoʁ} can be accounted for by assuming a conflation of two homophonous denominal derivations from the same base noun, as in the case  \japhug{nɤre}{laugh} discussed above (§\ref{sec:lability.apass}). The fact that double denominal derivation yields accusative lability in one case and ergative lability in the other is a consequence of the high diversity of meanings associated with the hyper-productive \forme{nɯ-/nɤ-} prefix (§\ref{sec:denom.nW}).

The other examples of ergative lability are highly lexicalized, and the transitive vs. intransitive functions have to be treated as different lexical entries synchronically.

The intransitive \japhug{ri}{remain} `be left' (\ref{ex:tWrdoR.YAri}) (see also \ref{ex:zgo.tWrdoR}, §\ref{sec:multiple.CN}) is related to the transitive homophonous verb root \forme{ri}, which is exclusively found in collocation with \japhug{tɯ-sroʁ}{life} (§\ref{sec:orphan.verb}) in the meaning `save $X$'s live' (\ref{ex:asroR.kAtWrit}). The entity whose life is spared is encoded as subject of the intransitive \forme{ri}, and as possessor of the object \forme{tɯ-sroʁ} of transitive \forme{ri} (see also \ref{ex:ni.ndZisroR}, §\ref{sec:dual.determiners}).

\begin{exe}
\ex \label{ex:tWrdoR.YAri}
 \gll  ɯ-tɕɯ ʁnɯz nɯ pjɤ-si, tɯ-rdoʁ nɯ ɲɤ-ri qʰe \\
 \textsc{3sg}.\textsc{poss}-son two \textsc{dem} \textsc{ifr}-die one-piece \textsc{dem} \textsc{ifr}-remain \textsc{lnk} \\
\glt `Two of her sons died, and one remained (alive).' (Gesar 2003, 315)
\end{exe}

\begin{exe}
\ex \label{ex:asroR.kAtWrit}
 \gll a-sroʁ kɤ-tɯ-ri-t \\
\textsc{1sg}.\textsc{poss}-life \textsc{aor}-2-save-\textsc{pst}:\textsc{tr} \\
\glt `You saved my life.' (150906 qingfeng-zh, 121)
\end{exe}

Given the more specific meaning of transitive \forme{ri}, it is more likely that it derives from intransitive \forme{ri} rather than the other way round. A further piece of evidence in favour of this hypothesis is the irregular causative \japhug{βri}{protect}, which also appears to derive from the intransitive \japhug{ri}{remain} (§\ref{sec:causative.m}). The relationship between intransitive \forme{ri}, transitive \forme{ri} and causative \forme{βri} is purely historical. These three verbs are synchronically completely distinct, and select different orientation preverbs (\textsc{westwards},  \textsc{eastwards} and  \textsc{upwards}, respectively).
 
The verb \japhug{pa}{do} is one of the most common verbs in Japhug, with a wide range of meanings including `close (door)', `become (friend, spouses)' and `discuss' (§\ref{sec:pa.lv}). It selects not only nouns, but also infinitive complements as objects (§\ref{sec:pa.complements}).
 
Two intransitive verbs with the same root form \forme{pa}, but only attested in \textsc{3sg}, also exist: the verb \japhug{pa}{pass X years}, which selects as subject a numeral referring to a number of years (see example \ref{ex:40.topa}, §\ref{sec:pa.intr.lv} and the discussion in §\ref{sec:num.prefix.paradigm.history}), and the light verb \forme{pa} used as light verb with ideophones (§\ref{sec:idph.pa}). Both of these functions are derivable from the meaning `do' of the transitive verb (§\ref{sec:CN.verbs}, §\ref{sec:pa.lv}). The conversion to the intransitive conjugation may have occurred through third person ambiguous forms in a dummy subject construction (§\ref{sec:transitive.dummy}).

The discussion above shows that in the cases of ergative lability, either the intransitive verb (\japhug{ri}{remain}) or the transitive one (\japhug{pa}{do}) can potentially be the primary form, with transitivization or intransitivization by zero-derivation due to reanalysis in ambiguous contexts (on this topic, see also §\ref{sec:stem3.backformation}).

\subsection{Transitive-intransitive labile verbs with oblique arguments} \label{sec:goal.labile}
The verb \japhug{rpu}{bump into} is also labile, but unlike the previous verbs, it selects an argument with the relator noun \japhug{ɯ-taʁ}{on, above} (§\ref{sec:WtaR}), corresponding to the person or object that the subject knocks/bumps into, as in (\ref{ex:WtaR.korpu}).
 
 \begin{exe}
\ex \label{ex:WtaR.korpu}
\gll   ɯ-ʑmbrɯ nɯnɯ [...] rɟɤmtsʰu ɣɯ ɯ-ŋgɯ rŋgɯ tu-kɯ-nɯ-ɬoʁ nɯ ɯ-taʁ ko-rpu. \\ 
\textsc{3sg}.\textsc{poss}-boat \textsc{dem} { } sea \textsc{gen} \textsc{3sg}.\textsc{poss}-in boulder \textsc{ipfv}-\textsc{sbj}:\textsc{pcp}-\textsc{auto}-come.out \textsc{dem} \textsc{3sg}.\textsc{poss}-on \textsc{ifr}-bump \\
\glt  `His boat ran on a reef (a boulder coming out of the sea).' (150830 baihe jieme-zh, 249)
\end{exe}

When used transitively, \japhug{rpu}{bump into} takes an object corresponding to the body parts suffering the impact (\ref{ex:WtaR.kAnWrputa}, with the 12\textsc{sg}\fl{}3 past \forme{-t} suffix), while in its intransitive use as in (\ref{ex:kArpua}) no body part is specified.

 \begin{exe}
\ex \label{ex:WtaR.kAnWrputa}
\gll  a-ku kɯm ɯ-taʁ kɤ-nɯ-rpu-t-a \\
\textsc{1sg};\textsc{poss}-head door \textsc{3sg}.\textsc{poss}-on \textsc{aor}-\textsc{auto}-bump.into-\textsc{pst}:\textsc{tr}-\textsc{1sg} \\
\glt  `I bumped my head against the (top frame of the) door.' (elicited)
\end{exe}

 \begin{exe}
\ex \label{ex:kArpua}
\gll maka ʑo mɯ-kɤ-rpu-a tɕe mɯ-tɤ-nɯɣmaz-a \\
\textsc{at}.\textsc{all} \textsc{emph} \textsc{neg}-\textsc{aor}-bump.into-\textsc{1sg} \textsc{lnk} \textsc{neg}-\textsc{aor}-be.wounded-\textsc{1sg} \\
\glt  `I did not hit (the bottom) and was not injured.' (150824 kelaosi-zh, 190)
\end{exe}

When transitively conjugated, \japhug{rpu}{bump into} does not have a causative meaning `cause X to bump into Y' (the causative \forme{sɯ-rpu} is used for this meaning). Although one could be tempted to translate \japhug{rpu}{bump into} in this way in examples like (\ref{ex:WtaR.kunWrpea}) (`I caused my bracelet to knock on the tripod'), here the bracelet, which is worn on the body, is construed as an extended part of the body (note the autive, which conveys both the meaning of non-volitionality and of action affecting the subject, §\ref{sec:autoben.proper}, §\ref{sec:autoben.spontaneous}).

 \begin{exe}
\ex \label{ex:WtaR.kunWrpea}
\gll   aʑo a-zgroʁ sqʰi ɯ-taʁ ku-nɯ-rpe-a ndʐa ɕti \\
\textsc{1sg} \textsc{1sg}.\textsc{poss}-bracelet tripod \textsc{3sg}.\textsc{poss}-on \textsc{ipfv}-\textsc{auto}-bump[III]-\textsc{1sg} reason be.\textsc{aff}:\textsc{fact} \\
\glt `This is because (I accidentally made) my bracelet clang against the tripod.' (tWxtsa 2003, 52)
\end{exe}

\subsection{Semi-transitive labile verbs}\label{sec:semi.tr.labile}
The verb \japhug{sɤŋo}{listen} can be conjugated transitively or intransitively, but selects a semi-object in the second case. In addition, its meaning and the orientation preverbs it selects are different depending on its valency.

In the intransitive conjugation, \forme{sɤŋo} means `listen' and selects the orientation `towards west', as in (\ref{ex:nWsANo}) (without Stem III alternation). In (\ref{ex:YWkWsANo}), it occurs with the semi-object \forme{ɯ-skɤt}.

 \begin{exe}
\ex \label{ex:nWsANo}
\gll    nɯ-sɤŋo je  \\
 \textsc{imp}-listen \textsc{sfp} \\
\glt `Listen!' (140516 guowang halifa-zh, 76)
\end{exe}

\begin{exe}
\ex \label{ex:YWkWsANo}
\gll pɣɤɲaʁ nɯnɯ ɯ-skɤt ɲɯ-kɯ-sɤŋo tɕe saχsɤl ma kɤ-mto rkɯn   \\ 
pheasant \textsc{dem} \textsc{3sg}.\textsc{poss}-voice \textsc{ipfv}-\textsc{genr}:\textsc{S/O}-listen \textsc{lnk} be.obvious:\textsc{fact} a.part.from \textsc{obj}:\textsc{pcp}-see be.rare:\textsc{fact} \\
 \glt `The pheasant, one can recognize its presence by listening to its voice, but it is rarely seen.'   (23-pGAYaR, 39)
\end{exe} 

This verb can also take a complement clause as semi-object, and can refer to perceptions other than hearing, as in (\ref{ex:YWrtaR.nWsANo}) (see also §\ref{sec:preverb.perception}).

\begin{exe}
\ex \label{ex:YWrtaR.nWsANo}
\gll nɤ-ŋga ɲɯ-rtaʁ ɕi mɯ́j-rtaʁ kɯ nɯ-sɤŋo \\
\textsc{2sg}.\textsc{poss}-clothes \textsc{sens}-be.enough \textsc{qu} \textsc{neg}:\textsc{sens}-be.enough \textsc{sfp} \textsc{imp}-listen \\
\glt  `Make sure (literally `feel whether') you have enough clothes.' (elicited)
\end{exe} 

With transitive valency, \forme{sɤŋo} selects the orientation `towards east', and means `obey, listen to', as in (\ref{ex:kasANo}).\footnote{In example (\ref{ex:kasANo}), the C-type orientation preverb show that the verb is transitive (§\ref{sec:transitivity.morphology}).}

\begin{exe}
\ex \label{ex:kasANo}
\gll  tɕe a-mu a-wa ra ka-sɤŋo tɕe lɤ-ari tɕe, 	 \\
\textsc{lnk} \textsc{1sg}.\textsc{poss}-mother \textsc{1sg}.\textsc{poss}-father \textsc{pl} \textsc{aor}:3\fl{}3$'$-obey \textsc{lnk}  \textsc{aor}:\textsc{upstream}-go[II] \textsc{lnk}\\
 \glt `He listened to my parents and went there.' (14-siblings, 219)
\end{exe}  

The transitive \forme{sɤŋo} has an antipassive form \japhug{sɤsɤŋo}{be obedient} (from `listen to people') as in (\ref{ex:YWsAsANo}), which is not labile, unlike its base verb.

\begin{exe}
\ex \label{ex:YWsAsANo}
\gll <xiaoqian> nɯ wuma ʑo tɕʰeme ɲɯ-pe, ɲɯ-sɤ-sɤŋo,  \\
\textsc{anthr} \textsc{dem} really \textsc{emph} girl \textsc{sens}-be.good \textsc{sens}-\textsc{apass}:\textsc{hum}-listen \\
\glt `Xiaoqian is a very nice girl, she is obedient.' (150907 niexiaoqian-zh, 166)
\end{exe}  

Another type of labile semi-transitive verb is \forme{rga}, which can either be semi-transitive or intransitive stative (without semi-object). It means `like' when semi-transitive as in (\ref{ex:cha.pjArga}), with a meaning close to its own applicative derivation (§\ref{sec:applicative.promoted}), and `be happy' when stative intransitive, as in (\ref{ex:wuma.Zo.pjArgandZi}).

\begin{exe}
\ex \label{ex:cha.pjArga}
\gll iɕqʰa tɯmɲa nɯ cʰa pjɤ-rga qʰe \\
the.aforementioned arrow \textsc{dem} alcohol \textsc{ifr}.\textsc{ipfv}-like \textsc{lnk} \\
\glt `(Gesar's) arrows like (to drink) alcohol.' (Gesar, 352)
\end{exe}

\begin{exe}
\ex \label{ex:wuma.Zo.pjArgandZi}
\gll ʁzɤmi ni wuma ʑo pjɤ-rga-ndʑi. \\
husband.and.wife \textsc{du} really \textsc{emph} \textsc{ifr}.\textsc{ipfv}-be.happy-\textsc{du} \\
\glt `The husband and his wife were very happy.' (140506 woju guniang-zh, 10)
\end{exe}

 \subsection{Ditransitive-monotransitive lability} \label{sec:secundative.monotransitive}
There are two subtypes of ditransitive-monotransitive lability in Japhug: secun\-da\-tive-monotransitive and indirective-monotransitive.

Monotransitive-secundative lability is illustrated by verbs such as \japhug{ɕɣɤz}{give back}, \japhug{fsɯɣ}{repay (gratitude)} and \japhug{sɤja}{give back}: they index as object either the recipient or the theme, as shown by examples (\ref{ex:nAtsxWnlAn.YWtafsWG}) and (\ref{ex:nAtsxWnlAn.YWnWfsWGa}).\footnote{The autive is optional in (\ref{ex:nAtsxWnlAn.YWnWfsWGa}). }

\begin{exe}
\ex 
\begin{xlist}
\ex \label{ex:nAtsxWnlAn.YWtafsWG}
\gll  nɤ-tʂɯnlɤn ɲɯ-ta-fsɯɣ ra \\
\textsc{2sg}.\textsc{poss}-gratitude \textsc{ipfv}-1\fl{}2-repay be.needed:\textsc{fact} \\
\glt `I have to return the favour.' (150827 tianluo-zh, 145)
\ex \label{ex:nAtsxWnlAn.YWnWfsWGa}
\gll nɤ-tʂɯnlɤn ɲɯ-nɯ-fsɯɣ-a \\
\textsc{2sg}.\textsc{poss}-gratitude \textsc{ipfv}-\textsc{auto}-repay-\textsc{1sg} \\
\glt `I will return the favour.' (elicited, adapted from example \ref{ex:nAtsxWnlAn.YWtafsWG})
\end{xlist}
\end{exe}

Indirective-monotransitive lability is more common:  this phenomenon refers to verbs which optionally take an oblique goal argument similar to that of intransitive verbs (§\ref{sec:intr.goal}). Typical examples include allative verbs of manipulation (§\ref{sec:manipulation.verbs}) such as \japhug{ɣɯt}{bring} or \japhug{tsɯm}{take away}, which are often found with absolutive or locative postpositional phrases expressing the goal of the motion as in (\ref{ex:kha.WNgW.jowGtsWmndZi}), but also commonly occur without any locative argument (\ref{ex:jWGmWr.nArca.tukWtsWma}).
 
 \begin{exe}
\ex \label{ex:kha.WNgW.jowGtsWmndZi}
\gll rɟɤlpu nɯ kɯ tɯtɯrca kʰa ɯ-ŋgɯ jó-wɣ-tsɯm-ndʑi  \\
king \textsc{dem} \textsc{erg} together house \textsc{3sg}.\textsc{poss}-in \textsc{ifr}-\textsc{inv}-take.away-\textsc{du} \\
\glt `The king brought them together into a house.' (140505 liuhaohan zoubian tianxia-zh, 142)
\end{exe}

\begin{exe}
\ex \label{ex:jWGmWr.nArca.tukWtsWma}
\gll jɯɣmɯr ndɤre nɤ-rca tu-kɯ-tsɯm-a ra \\
this.evening \textsc{lnk} \textsc{2sg}.\textsc{poss}-following \textsc{ipfv}:\textsc{up}-2\fl {}1-take.away-\textsc{1sg} be.needed:\textsc{fact} \\
\glt `This evening, take me with you (to heaven).' (07-deluge, 48)
\end{exe}

\section{Additional questions on the  generic and number indexation}
The previous sections focused on person and number indexation in relation to  verb argument structure. This section discusses some properties of number and generic indexation that are observed on both intransitive and transitive verbs.

\subsection{Agreement mismatch}  \label{sec:agreement.mismatch}
This section investigates various types of mismatch between person indexation on the verb and nominal and pronominal elements in the clause: optional number indexation, honorific plural, partitive indexation and the interaction between first person and generic person. There are two additional types of indexation mismatch not discussed here: hybrid indirect speech (§\ref{sec:hybrid indirect}), and affixal chains in bipartite verbs (§\ref{sec:bipartite}).
 
\subsubsection{Optional number indexation} \label{sec:optional.indexation}
The default situation in Japhug is for the third person core arguments to be indexed in number on the verb in intransitive or non-local configurations (the number of the subject in direct configurations, and that of the object in inverse configurations, §\ref{sec:indexation.non.local}). This is the case in particular in series of verbs sharing the same subject, as in example (\ref{ex:tonWndzondZi.konWGindZi}) with dual (\textsc{3du}\fl{}3$'$ or \textsc{3du} intransitive) indexation on five verbs in a row, referring to the same pair of persons.

\begin{exe}
\ex \label{ex:tonWndzondZi.konWGindZi}
\gll tɕendɤre <yinliao> to-nɯ-ndo-ndʑi, kɤndza ra to-nɯ-ndo-ndʑi qʰe,
qrɤŋgɤɣ ɯ-tʰɤcu tɕe,  to-nɯna-ndʑi qhe, ɲɤ-nɤmɲole-ndʑi qʰe, tɕe ko-nɯ-ɣi-ndʑi \\
\textsc{lnk} drink \textsc{ifr}-\textsc{auto}-take-\textsc{du} food \textsc{pl} \textsc{ifr}-\textsc{auto}-take-\textsc{du} \textsc{lnk}  \textsc{topo} \textsc{3sg}.\textsc{poss}-downstream \textsc{loc} \textsc{ifr}-rest-\textsc{du} \textsc{lnk} \textsc{ifr}-do.sightseeing-\textsc{du} \textsc{lnk} \textsc{lnk}  \textsc{ifr}:\textsc{east}-\textsc{vert}-come-\textsc{du} \\
\glt `The two of them took with them drinks and food, and rested and did sightseeing further down from Qrangak, and then came back.' (conversation 140510)
\end{exe}

However, non-singular third person core arguments do not necessarily trigger number indexation in all cases. Several syntactic, semantic and discourse factors interfere with number indexation.

With existential verbs, number indexation of the subject is nearly always observed in the case of human referents in the existential construction, even in the case of collective nouns without number markers, such as the counted noun \japhug{tɯ-tɯpɯ}{one household} in (\ref{ex:tWtWpW.pjAtunW}). Dual or plural indexation is generally observed when the subject has a numeral modifier, as in (\ref{ex:XsWm.pjAtunW}). 


\begin{exe}
\ex \label{ex:tWtWpW.pjAtunW}
\gll  ɯnɯnɯtɕu tɯ-tɯpɯ pjɤ-tu-nɯ \\
\textsc{dem}:\textsc{loc} \textsc{one}-household \textsc{ifr}.\textsc{ipfv}-exist-\textsc{pl} \\
\glt `there was one household there.' (140512 yufu yu mogui-zh, 3)
\end{exe}

\begin{exe}
\ex \label{ex:XsWm.pjAtunW}
\gll kɤndʑi-xtɤɣ χsɯm pjɤ-tu-nɯ \\
\textsc{coll}-brother three \textsc{ifr}.\textsc{ipfv}-exist-\textsc{pl} \\
\glt `There were three brothers.' (31-deluge, 7)
\end{exe}

On the other hand, with inanimate referents, number indexation is rarely found; in (\ref{ex:ʁnWz.tu}) for instance, no dual indexation is found on the verb despite the numeral \japhug{ʁnɯz}{two}.

\begin{exe}
\ex \label{ex:ʁnWz.tu}
\gll  nɯnɯreri tsʰɤko ʁnɯz tu. \\
\textsc{dem}:\textsc{loc} stone.mount two exist:\textsc{fact} \\
\glt `There are two stone mounts there.' (140522 Kamnyu zgo, 138)
\end{exe}

In the possessive construction (§\ref{sec:gen.possession}, §\ref{sec:possessive.constructions}), however, the subject of existential verbs (the possessum) rarely triggers number indexation, even in the case of human referents: compare for instance in (\ref{ex:pjAtundZi.pjAtu}) the first verb form \forme{pjɤ-tu-ndʑi} (existential construction with dual indexation) vs. the second one \forme{pjɤ-tu} (possessive construction with plural possessum, no indexation).

\begin{exe}
\ex \label{ex:pjAtundZi.pjAtu}
\gll kɯɕɯŋgɯ tɕe, ʁzɤmi ci pjɤ-tu-ndʑi tɕe, ndʑi-tɕɯ χsɯm pjɤ-tu. \\
long.ago \textsc{loc} husband.and.wife \textsc{indef} \textsc{ifr}.\textsc{ipfv}-exist-\textsc{du} \textsc{lnk} \textsc{3du}.\textsc{poss}-son three \textsc{ipfv}.\textsc{ipfv}-exist \\
\glt `Long ago, there was a husband and his wife, and they had three sons.' (140430 jin e-zh, 2)
\end{exe}

With dynamic verbs, as in the existential construction, core arguments generally triggers number indexation when they have animate referents. This is in particular the case with the adverb \japhug{tɯtɯrca}{together} (§\ref{sec:secutive} ), even to describe group actions where all individuals act exactly in the same way as in (\ref{ex:tWtWrca.luzonW}).

\begin{exe}
\ex \label{ex:tWtWrca.luzonW}
\gll tɕe lɤ-zo-nɯ kɯnɤ tɯtɯrca lu-zo-nɯ, tʰɯ-nɯqambɯmbjom-nɯ kɯnɤ tɯtɯrca cʰɯ-nɯqambɯmbjom-nɯ, \\
\textsc{lnk} \textsc{aor}:\textsc{upstream}-land-\textsc{pl}  also  together  \textsc{ipfv}:\textsc{upstream}-land-\textsc{pl} \textsc{aor}:\textsc{downstream}-fly-\textsc{pl}  also together  \textsc{ipfv}:\textsc{downstream}-fly-\textsc{pl} \\
\glt `When they land they all land together, when they fly they all fly together.' (24-qro, 10)
\end{exe}

Number indexation is optional with \japhug{tɯtɯrca}{together} in the case of inanimate referents (including plants). Examples (\ref{ex:tulhoRnW.Nu}) and (\ref{ex:tulhoR.Nu}) from the same text shows verb forms with and without plural indexation in the same context (referring to a species of mushroom).

\begin{exe}
\ex 
\begin{xlist}
\ex \label{ex:tulhoRnW.Nu}
\gll tɯtɯrca kɯ-dɯ\redp{}dɤn tu-ɬoʁ-nɯ ŋu. \\
together \textsc{sbj}:\textsc{pcp}-\textsc{emph}\redp{}be.many \textsc{ipfv}-come.out-\textsc{pl} be:\textsc{fact} \\
\glt `They grow together in great numbers.' (21-jmAGni, 82)
\ex \label{ex:tulhoR.Nu}
\gll kɯ-dɯ\redp{}dɤn ʑo tɯtɯrca tu-ɬoʁ ŋu. \\
 \textsc{sbj}:\textsc{pcp}-\textsc{emph}\redp{}be.many \textsc{emph} together \textsc{ipfv}-come.out be:\textsc{fact} \\
\glt `They grow together in great numbers.'  (21-jmAGni, 84)
\end{xlist}
\end{exe}

Optional number indexation is however also attested, though uncommon, with dynamic verbs in reference to humans, as in (\ref{ex:kWtsxAG.nWsi}), where verb `die' occurs without plural indexation in its first occurrence (just after a possessive construction, where indexation on the existential verb \forme{tɤ-tu} is not found either), and with the plural suffix \forme{-nɯ} in the second occurrence.

\begin{exe}
\ex \label{ex:kWtsxAG.nWsi}
\gll ɯ-rɟit kɯngɯt tɤ-tu ri, kɯtʂɤɣ nɯ-si. kɯtʂɤɣ nɯ-si-nɯ qʰe \\
\textsc{3sg}.\textsc{poss}-child nine \textsc{aor}-exist \textsc{lnk} six \textsc{aor}-die six \textsc{aor}-die-\textsc{pl} \textsc{lnk} \\
\glt `She had nine children, six (of them) died. Six died, and...' (14-siblings, 17-18)
\end{exe}

The conditions for optional number indexation of human referents with dynamic verbs are not completely clear. Absence of number indexation is rare in intransitive and direct forms, and may be due in part to speech errors (§\ref{sec:mismatch.errors}).


With generic human subjects, the indexation of the number of the object, though attested (§\ref{sec:indexation.generic.tr}), is not common. For instance in (\ref{ex:ni.schWwGqru}), despite an overt dual object, the verb lacks number indexation.

\begin{exe}
\ex \label{ex:ni.schWwGqru}
\gll kɤndʑi-ʁi ni s-cʰɯ́-wɣ-qru ŋu \\
\textsc{coll}-sibling \textsc{du} \textsc{tral}-\textsc{ipfv}:\textsc{downstream}-\textsc{inv}-welcome be:\textsc{fact} \\
\glt `Let's invite the two brothers down here.' (Nyima wodzer2003-2, 115)
\end{exe} 
 
%"ma nɯra aʑo a-pi ŋu-nɯ wo" to-ti.
%χɕitka jɤ-ɣe tɕe, si ra tɯ-ɣɤtɕɯɣ na-ʑa tɕe
%11-qrontshom, 53-54
 
\subsubsection{Plural as honorific} \label{sec:honorific.indexation}
Plural marking on nouns (§\ref{sec:ra.honorific}) and pronouns (§\ref{sec:honorific.pronouns}) can be used to express singular honorific in Japhug. Similarly, honorific plural indexation is found with second (\ref{ex:atACime.kAnWrNgWnW}) or third (\ref{ex:rJAlpu.mArAZinW}) person referents, with or without plural \forme{ra} marking on the noun. Honorific plural is mainly attested in traditional stories, but it is still used to address lamas.

\begin{exe}
\ex \label{ex:atACime.kAnWrNgWnW}
\gll a-tɤɕime ra, ci ɲɯ-ɣɯtsʰɤdɯɣ tɕe, kɤ-nɯ-rŋgɯ-nɯ \\
\textsc{1sg}.\textsc{poss}-lady \textsc{pl} a.little \textsc{sens}-be.hot \textsc{lnk} \textsc{imp}-\textsc{auto}-lie.down-\textsc{pl} \\
\glt `My lady, it is a bit hot, lie down (to sleep).' (2014-kWlAG, 232)
\end{exe}

\begin{exe}
\ex \label{ex:rJAlpu.mArAZinW}
\gll  tɕeri tɤ-mu nɯ kɯ rɟɤlpu ra mɤ-rɤʑi-nɯ tɕe, pɣɤtɕɯ nɯ pjɤ-sɯ-sat. \\
\textsc{lnk} \textsc{indef}.\textsc{poss}-mother \textsc{dem} \textsc{erg} king \textsc{pl} \textsc{neg}-stay:\textsc{fact}-\textsc{pl} \textsc{lnk} bird \textsc{dem} \textsc{ifr}-\textsc{caus}-kill \\
\glt `The woman had the bird killed while the king was away.' (2014-kWlAG, 590)
\end{exe}

Honorific plural indexation on the noun does not always correlate with plural indexation on the verb, as in (\ref{ex:aZi.ra.YWtWnAre}), where the verb has a singular form.

\begin{exe}
\ex \label{ex:aZi.ra.YWtWnAre}
\gll wo a-ʑi ra tɕʰindʐa ɲɯ-tɯ-nɤre ŋu \\
\textsc{interj} \textsc{1sg}.\textsc{poss}-young.lady \textsc{pl} why \textsc{ipfv}-2-laugh be:\textsc{fact} \\
\glt `My lady, why are you laughing?' (2005 Kunbzang, 245)
\end{exe}



\subsubsection{Partitive indexation} \label{sec:partitive.indexation}
Number indexation with a with first person core argument, unlike that of second and third person, is compulsory. 

Apparent examples of mismatch however do exist, but are confined to a very specific partitive use of dual of plural number \citep{bickel00agreement}. With the interrogative pronoun \japhug{ɕɯ}{who} (§\ref{sec:CW.pronoun}), in particular, indexation on the verb can be non-singular with the specific partitive meaning `who among $X$', in particular in comparative constructions as in (\ref{ex:CW.kW.YWmpCArtCi}), with \textsc{1du} indexation on the verb although this sentence implies that only one of the two sisters is the most beautiful (see §\ref{sec:sAz.kW} on this comparative construction, and \citet{jacques16comparative} and §\ref{sec:comparee.kW} on the use of the ergative here).

\begin{exe}
\ex   \label{ex:CW.kW.YWmpCArtCi}
 \gll  a-ʁi, nɤki tɕetʰi tɕe, tɯ-ci ɯ-ŋgɯ ɕ-pɯ-ru tɕe, ɕɯ kɯ ɲɯ-mpɕɤr-tɕi kɯ? \\
 \textsc{1sg}.\textsc{poss}-younger.sibling \textsc{filler} downstream \textsc{loc} \textsc{indef}.\textsc{poss}-water \textsc{3sg}.\textsc{poss}-inside \textsc{tral}-\textsc{imp}:\textsc{down}-look \textsc{lnk} who \textsc{erg} \textsc{sens}-be.beautiful-\textsc{1du} \textsc{sfp} \\
 \glt `Sister, go and look down there in the water, who is the most beautiful of us?' (2014-kWlAG, 477)
\end{exe} 

%ɕɯ kɯ nɯ stu ʑo, nɤkinɯ, kɯ-sɤmtshɤr, ʑɯmkhɤm ɯ-ku kɯ-rkɯn kɯ-fse kɤ-ɣɯt kɯ-cha nɯnɯ kɯ,
%kɯki @nuoha nɯ nɯ-rʑaβ ku-tɯ-nɯ-sɯβzu-nɯ jɤɣ" to-ti

Similarly, non-singular indexation on the verb with a counted noun core argument can have a partitive meaning, for instance \japhug{tɯ-rdoʁ}{one piece} (§\ref{sec:CN.partitive}) with plural \forme{-nɯ} on the verb in (\ref{ex:tWrdoR.kW.asci.athWndonW}) can only be interpreted as meaning `one of them should take it'.

\begin{exe}
\ex   \label{ex:tWrdoR.kW.asci.athWndonW}
 \gll ``tɯ-rdoʁ kɯ a-sci a-tʰɯ-ndo-nɯ ntsʰi" ɲɯ-sɯsɤm pjɤ-ŋu \\
one-piece \textsc{erg} \textsc{1sg}.\textsc{poss}-instead \textsc{irr}-\textsc{pfv}-take-\textsc{pl} be.better:\textsc{fact} \textsc{ipfv}-think[III] \textsc{ifr}.\textsc{ipfv}-be \\
\glt `(The king) was thinking: `One of (my sons) should inherit the throne.'' (140510 sanpian yumao-zh, 7)
\end{exe} 

Another type of partitive indexation is the use of \textsc{1pl} pronouns with third person indexation, as in (\ref{ex:iZora.tutinW}), with two verbs in \textsc{3pl}\fl{}3 form and the \textsc{1pl} pronoun in topicalized position meaning `some among us'.

\begin{exe}
\ex   \label{ex:iZora.tutinW}
 \gll  iʑora tɕe ɕkɤpʰɤr tu-ti-nɯ tsuku kɯ ɕkɤjwaʁ tu-ti-nɯ ŋu ma \\
 \textsc{1pl} \textsc{lnk} wild.chives \textsc{ipfv}-say-\textsc{pl} some \textsc{erg}  wild.chives \textsc{ipfv}-say-\textsc{pl} be:\textsc{fact} \textsc{lnk} \\
\glt `Among us, some call it \forme{ɕkɤpʰɤr}, some \forme{ɕkɤjwaʁ}.' (07-Cku, 82)
\end{exe} 

\subsubsection{First person and generic person} \label{sec:1.genr}
Another type of agreement mismatch observed with first person concerns \textsc{1pl} and generic person. Before examining the examples of mismatch between pronouns and verb indexation in Japhug, it is important to note that generic person often occurs in gnomic statements applying to the speaker himself (§\ref{sec:genr.3pl}), and can be used as an indirect way to express a first person, as has been described in some Kiranti languages \citep{bickel15antipassive}. It is even found in contexts where it unambiguously refers to the first person \textit{singular}, as in (\ref{ex:mAxsi.YAnWjmWta}).

\begin{exe}
\ex   \label{ex:mAxsi.YAnWjmWta}
 \gll mɤ-xsi ko, nɯra ɲɤ-nɯ-jmɯt-a \\
\textsc{neg}-\textsc{genr}:know \textsc{sfp} \textsc{dem}:\textsc{pl} \textsc{ifr}-\textsc{auto}-forget-\textsc{1sg} \\
\glt `I don't know, I forgot about these things.' (phone conversation, 2013-12-24)
\end{exe}

Generic inverse Imperfective verb forms with modal verb such as \japhug{ntsʰi}{be better} or \japhug{ra}{be needed} (example \ref{ex:jinWNa.pjWwGnWntCha}) or even without auxiliary (\ref{ex:iZo.ci.YWwGphWt}) is a common way to express \textsc{1pl} hortative (§\ref{sec:ipfv.complement}, §\ref{sec:ipfv.hortative}), and in such contexts \textsc{1pl} pronouns or possessive prefixes can be found.

In (\ref{ex:jinWNa.pjWwGnWntCha}), generic inverse marking on the verb corresponds to the \textsc{1pl} possessive on the object \japhug{nɯŋa}{cow}. The presence of the autive \forme{nɯ-} (§\ref{sec:autobenefactive}) is a further clue to the equivalence of the generic subject and the \textsc{1pl} in this example.

\begin{exe}
\ex   \label{ex:jinWNa.pjWwGnWntCha}
 \gll skalpa ndʑɯɣ nɯ ɲɯ-ŋu tɕe, ji-nɯŋa pjɯ́-wɣ-nɯ-ntɕʰa ɲɯ-ntsʰi \\
world be.destroyed:\textsc{fact} \textsc{dem} \textsc{sens}-be \textsc{lnk} \textsc{1pl}.\textsc{poss}-cow \textsc{ipfv}-\textsc{inv}-\textsc{auto}-butcher \textsc{sens}-be.better \\
\glt `The world is about to be destroyed, let us kill our cow.' (07-deluge, 7)
\end{exe}

In (\ref{ex:iZo.ci.YWwGphWt}), the \textsc{1pl} pronoun \forme{iʑo} directly co-occurs with a verb in transitive subject generic form (§\ref{sec:indexation.generic.tr}). 

\begin{exe}
\ex   \label{ex:iZo.ci.YWwGphWt}
 \gll  iʑo kɯ-mɤku pɤjkʰu, ɯ-ɕɣa kɯ-mtɕoʁ nɯ ci ɲɯ́-wɣ-pʰɯt \\
 \textsc{1pl} \textsc{sbj}:\textsc{pcp}-be.first still \textsc{3sg}.\textsc{poss}-tooth \textsc{sbj}:\textsc{pcp}-be.sharp \textsc{dem} a.little \textsc{ipfv}-\textsc{inv}-take.out \\
\glt `Let us first take out its sharp teeth.' (150908 menglang-zh, 80)
\end{exe}

Co-occurrence of generic indexation with a \textsc{1pl} pronoun is also found in procedural texts; in such contexts the  \textsc{1pl} pronouns occur in apposition with place names (§\ref{sec:place.names}), ethnic groups or classes of people, for instance \forme{iʑo kɯrɯ ra} `we Tibetans' in (\ref{ex:iZo.luWGnWBzu}).

\begin{exe}
\ex   \label{ex:iZo.luWGnWBzu}
 \gll iʑo kɯrɯ ra, nɤkinɯ, qajɣi lú-wɣ-nɯ-βzu tɕe \\
\textsc{1pl} Tibetan \textsc{pl} \textsc{filler} bread \textsc{ipfv}-\textsc{inv}-\textsc{auto}-make \textsc{lnk} \\
\glt `We Tibetans, when we make bread,' (160706 thotsi, 1)
\end{exe}

The opposite situation, a generic pronoun in combination with \textsc{1pl} indexation, is much rarer, but also attested. For instance, in (\ref{ex:tWZAra.pWxtCij}), the adjectival stative verb \japhug{xtɕi}{be small} bears \textsc{1pl} \forme{-j} suffix, but the corresponding overt pronoun in the sentence is the generic person \japhug{tɯʑɤra}{one} (§\ref{sec:genr.pro}). The generic form \forme{pɯ-kɯ-xtɕi}, as in (\ref{ex:pWkWxtCi.tCe.pWwGmto}) would be possible in the exactly the same context, clearly including the first person.

\begin{exe}
\ex   \label{ex:tWZAra.pWxtCij}
 \gll tɯʑɤra pɯ-xtɕi-j tɕe, \\
 \textsc{genr} \textsc{pst}.\textsc{ipfv}-be.small-\textsc{1pl} \textsc{lnk} \\
 \glt `When we were young.' (17-ndZWnW, 52)
\end{exe}

\begin{exe}
\ex   \label{ex:pWkWxtCi.tCe.pWwGmto}
 \gll tɕe jinde aj pɯ-mto-t-a me ri, pɯ-kɯ-xtɕi tɕe pɯ́-wɣ-mto \\
 \textsc{lnk} now \textsc{1sg} \textsc{aor}-see-\textsc{pst}:\textsc{tr}-\textsc{1sg} not.exist:\textsc{fact} \textsc{lnk} \textsc{pst}.\textsc{ipfv}-\textsc{genr}:S/O-be.small \textsc{lnk} \textsc{aor}-\textsc{inv}-see \\
\glt `I have not seen any (wild crane) recently, but when we were young, we did see it.' (22-qomndroN, 35)
\end{exe}

Dual or plural indexation can occur with generic transitive subject indexation marking a first person, as in example (\ref{ex:tuwGqurnW}) in §\ref{sec:indexation.generic.tr}.

In addition to the generic person, the proprietive derivation is another possible strategy to indirectly refer to the first person (§\ref{sec:proprietive.generic}).
 
\subsubsection{Indexation mismatch and speech errors} \label{sec:mismatch.errors}
Speech errors are inevitable in any corpus, and are a factor to take into consideration to explain inconsistencies in person indexation. 

The clearest examples of erroneous indexation are self-corrections. In (\ref{ex:YotanW.Yota}) for instance, the speaker first chooses a plural \textsc{3pl}\fl{}3$'$ with \forme{-nɯ} suffix and then corrects herself to the appropriate form \textsc{3sg}\fl{}3$'$ \forme{ɲo-ta}. The error here is due without doubt to the presence of the overt object \forme{ʁʑɯnɯ tɕʰemɤli ra} bearing a plural marker: plural objects in non-local configuration are only indexed in the case of inverse forms (§\ref{sec:indexation.non.local}), but examples of this type show that speakers may nevertheless be tempted to index them in direct configurations too (as if the verb were intransitive).

\begin{exe}
\ex \label{ex:YotanW.Yota}
\gll  ʁʑɯnɯ tɕʰemɤli ra, ɲo-ta-nɯ tɕe, ɲo-ta tɕe, \\
young.man young.woman \textsc{pl} \textsc{ifr}-put-\textsc{pl} \textsc{lnk} \textsc{ifr}-put \textsc{lnk} \\
\glt `She left (there) the young men and women.' (2003kandzWsqhaj, 35)
\end{exe} 

In (\ref{ex:YAnArWrandZi.pl}) from a text translated from Chinese, Tshendzin realised that the number of daughters was different from what she had  remembered, and hesitated between the dual and the plural on the indexation of the intransitive verb \japhug{nɤrɯra}{look around}.

\begin{exe}
\ex \label{ex:YAnArWrandZi.pl}
\gll oma, kɯβde pjɤ-sɯ-ɣe nɤ! kɯβde pjɤ-sɯ-ɣe qʰe, nɯnɯra ɲɤ-nɤrɯra-ndʑi ri, ɲɤ-nɤrɯra-nɯ ri, \\
\textsc{interj} four \textsc{ifr}:\textsc{down}-\textsc{caus}-come \textsc{sfp} four \textsc{ifr}:\textsc{down}-\textsc{caus}-come \textsc{lnk} \textsc{dem}:\textsc{pl} \textsc{ifr}-look.around-\textsc{du} \textsc{lnk} \textsc{ifr}-look.around-\textsc{pl} \textsc{lnk} \\
\glt `He sent four (of his daughters, not three), he sent four of them, and they looked around.' (150826 baoliandeng-zh, 223)
\end{exe} 

Without self correction, indexation errors are less obvious and have to be rechecked with a native speaker. In (\ref{ex:lAwGCaBndZi}), the first verb \forme{lɤ́-wɣ-ɕaβ-ndʑi} has inverse marking (§\ref{sec:obviation.possessor}) and indexes the non-overt object (the youngest daughter and her husband). The second (intransitive) verb has dual indexation, but its subject is coreferent with that of the previous verb (\forme{ɯ-pi ra} `her elder sisters etc') and plural indexation would therefore be expected here, and Tshendzin indeed proposes to correct the verb form to \forme{kɤ-akʰu-nɯ} \textsc{aor}-call-\textsc{pl}. The confusion between dual and plural here is a combination of two factors: the presence of a verb with dual indexation just before (indexation attraction), and the ambiguity of the subject: the youngest sister has two eldest daughters (the plural here refer to her husbands and servants), and one could construe the subject of \forme{kɤ-akʰu-ndʑi} as referring only to the two elder sisters, without the additional people, hence the dual indexation.

\begin{exe}
\ex \label{ex:lAwGCaBndZi}
\gll  ndɤre ɯ-pi ra kɯ lɤ́-wɣ-ɕaβ-ndʑi nɤ kɤ-akʰu-ndʑi ɲɯ-ŋu. \\
\textsc{lnk} \textsc{3sg}.\textsc{poss}-elder.sibling \textsc{pl} \textsc{erg} \textsc{aor}:\textsc{upstream}-\textsc{inv}-catch.up-\textsc{du} \textsc{lnk} \textsc{aor}-call-\textsc{du} \textsc{sens}-be \\
\glt `Her elder sisters and the others caught up with them and called.' (2005 Kunbzang, 155)
\end{exe} 

Some apparent cases of optional number indexation (§\ref{sec:optional.indexation}) should also be analyzed as speech errors, as in (§\ref{sec:optional.indexation}), where the first verb \forme{jɤ-ɣe-ndʑi} bears correct dual indexation, but the second one lacks it, presumably due to hesitation on the part of the speaker.

\begin{exe}
\ex \label{ex:jAGendZi.du}
\gll   nɯnɯ kɯ-fse ʁnɯz jɤ-ɣe-ndʑi tɕe,  ji-kʰa zɯ jɤ-ɣe tɕe \\
\textsc{dem} \textsc{sbj}:\textsc{pcp}-be.like two \textsc{aor}-come[II]-\textsc{du} \textsc{lnk}  \textsc{1pl}.\textsc{poss}-house \textsc{loc} \textsc{aor}-come[II] \textsc{lnk} \\
\glt `Two (ghosts) like that came, came to our house.' (150902 qixian-zh, 139)
\end{exe} 

\subsection{Generic person vs. \textsc{3pl} indexation} \label{sec:genr.3pl}
In Japhug, both generic marking (§\ref{sec:indexation.generic.tr}) and third person plural indexation can be used to express generic referents. In particular, both can agree with the generic/indefinite noun \japhug{tɯrme}{person} (§\ref{sec:tWrme.generic}), as illustrated by examples (\ref{ex:tWrme.kW.tuwGndza})\footnote{The generic transitive subject marker is the inverse prefix (§\ref{sec:indexation.generic.tr}). In this section, it is however glossed as \textsc{genr}:A for clarity. } and (\ref{ex:tWrme.ra.kW.nWrganW}).


\begin{exe}
\ex   \label{ex:tWrme.kW.tuwGndza}
 \gll nɯŋa ra tɕi kɤ-ndza rga-nɯ, tɯrme kɯ tú-wɣ-ndza mɤ-sna. \\
cow \textsc{pl} \textsc{erg} \textsc{inf}-eat like:\textsc{fact}-\textsc{pl} people \textsc{erg} \textsc{ipfv}-\textsc{genr}:\textsc{A}-eat \textsc{neg}-be.good:\textsc{fact} \\
\glt `Cows like to eat it, but it is not good for people.' (11-paRzwamWntoR, 39-40)
\end{exe}

\begin{exe}
\ex   \label{ex:tWrme.ra.kW.nWrganW}
 \gll  tɕe lɯlu nɯ wuma ʑo pe tɕe, nɯnɯ, tɯrme ra kɯ nɯ nɯ-rga-nɯ tɕe \\
\textsc{lnk} cat \textsc{dem} really \textsc{emph} be.good:\textsc{fact} \textsc{lnk} \textsc{dem} people \textsc{pl} \textsc{erg} \textsc{dem} \textsc{appl}-like:\textsc{fact}-\textsc{pl} \textsc{lnk} \\
\glt `The cat is a very nice (animal), people like it.' (21-lWLU, 41)
\end{exe}

The aim of this section is to examine the semantic difference between generic person vs. \textsc{3pl} in contexts like those illustrated by the examples above.

Generic indexation (§\ref{sec:indexation.generic.tr}) is most commonly used to express general or gnomic statements applying to most humans, including the speaker. No more than one argument in a particular sentence can be generic: In particular, it is not possible to have both generic subject and object on the same verb form (§\ref{sec:indef.genr.poss} and §\ref{sec:genr.pro}), except in reflexive constructions (example \ref{ex:tWZo.kW.tWZo}, §\ref{sec:genr.pro}). 

There is obligatory agreement between all generic person markers, whether indexation on the verb, pronouns or possessive prefixes inside a clause, and across contiguous clauses. In (\ref{ex:nA.mAwGmto}) for instance, the generic intransitive subject of the verb \forme{mɯ\redp{}mɤ-pɯ-kɯ-tso} is coreferent with the transitive subject of \forme{mɤ́-wɣ-mto}.

\begin{exe}
\ex   \label{ex:nA.mAwGmto}
 \gll   tɕe wuma ʑo mɯ\redp{}mɤ-pɯ-kɯ-tso nɤ mɤ́-wɣ-mto \\
 \textsc{lnk} really \textsc{emph} \textsc{cond}\redp{}\textsc{neg}-\textsc{pst}.\textsc{ipfv}-\textsc{genr}:S/O-know \textsc{lnk} \textsc{neg}-\textsc{genr}:\textsc{A}-see:\textsc{fact} \\
\glt `If you do not know it well, you won't see it.'   (07-Cku, 59)
\end{exe}

In (\ref{ex:tCheme.tukWnAkhe}), the generic possessor prefix on \forme{tɯ-laχtɕʰa} is also coreferent with the object of \forme{ɲɯ-kɯ-nɯsɯkʰo} `they rob people of X'. Note however in this example that the generic person on \forme{tu-kɯ-nɤkʰe} is not exactly identical to the referent of the previous verb, referring to a subset of it (only women).

\begin{exe}
\ex   \label{ex:tCheme.tukWnAkhe}
 \gll  tɯrme pjɯ-sat-nɯ, tɯ-laχtɕʰa ɲɯ-kɯ-nɯsɯkʰo, tɕʰeme tɕe tu-kɯ-nɤkʰe kɯ-fse ɲɯ-ŋu \\
 people \textsc{ipfv}-kill-\textsc{pl} \textsc{genr}.\textsc{poss}-thing \textsc{ipfv}-\textsc{genr}:S/O-rob woman \textsc{lnk} \textsc{ipfv}-\textsc{genr}:S/O-bully \textsc{sbj}:\textsc{pcp}-be.like \textsc{sens}-be \\
 \glt `They did things like killing people, robbing people's things, raping women.' (17-lhazgron, 25-26)
\end{exe}

Example (\ref{ex:tWZo.tukWrWCmi}) show the agreement between the generic pronoun \japhug{tɯʑo}{one} and generic marking on the verb.

\begin{exe}
\ex   \label{ex:tWZo.tukWrWCmi}
 \gll  tɯʑo tu-kɯ-rɯɕmi nɯra ɯ-ɲɯ́-tso? \\
 \textsc{genr} \textsc{ipfv}-\textsc{genr}:S/O-speak \textsc{dem}:\textsc{pl} \textsc{qu}-\textsc{sens}-understand \\
 \glt `Does (you son) understand when people speak?' (phone conversation 15-01-13)
\end{exe}

Generic human marking can sometimes be used as a substitute for first person, both as indexation prefix (§\ref{sec:1.genr}) or as possessive prefix (§\ref{sec:generic.tW.1sg}). Example (\ref{ex:tApAtso.pWkWNu}) illustrates this function.
%{sec:genr.pro}

\begin{exe}
\ex   \label{ex:tApAtso.pWkWNu}
 \gll   tɕeri tɤ-pɤtso pɯ-kɯ-ŋu tɕe, nɯ kɤ-ndza wuma ʑo pɯ-kɯ-rga. \\
\textsc{lnk} \textsc{indef}.\textsc{poss}-child \textsc{pst}.\textsc{ipfv}-\textsc{genr}:S/O-be \textsc{lnk} \textsc{dem} \textsc{inf}-eat really \textsc{emph} \textsc{pst}.\textsc{ipfv}-\textsc{genr}:S/O-like \\
\glt `When we were children, we used to like eating it a lot.' (12-ndZiNgri, 138)
\end{exe}

The commonality between all the uses of generic person marking is that it always refers to a group including the speaker.\footnote{Example (\ref{ex:tWZo.tukWrWCmi}) above could seem to be a counterexample, but the question here is about the ability to understand speech in general, not restricted to the people in contact with my son at the moment this question was uttered. } It is particularly clear in (\ref{ex:tCheme.tukWnAkhe}), where the generic prefix \forme{kɯ-} on  \forme{tu-kɯ-nɤkʰe} is coreferent with the overt object \japhug{tɕʰeme}{woman}; this verb form is appropriate in this particular instance because the speaker is a woman, and \japhug{tɕʰeme}{woman} is used here as a generic noun.

Example (\ref{ex:Juli.chWlAtnW}) further illustrates the same phenomenon. The speaker, also a woman, includes herself among the potential Jews' harp players (though she has never played the instrument) and thus employs generic subject marking on the first verb (note also the presence of the \textsc{1pl} pronoun, §\ref{sec:1.genr}). Conversely, she excludes herself from potential flute players (traditionally, only men) and selects \textsc{3pl} marking on the second verb. 

\begin{exe}
\ex \label{ex:Juli.chWlAtnW}
\gll kɯɕɯŋgɯ tɕe, iʑora tɕʰeme kɯ ʑɴɢro ɲɯ́-wɣ-lɤt, tɤ-tɕɯ ra kɯ ɟuli cʰɯ-lɤt-nɯ. \\
 long.ago \textsc{lnk} \textsc{1pl} girl  \textsc{erg} Jews'.harp \textsc{ipfv}-\textsc{genr}:\textsc{A}-release \textsc{indef}.\textsc{poss}-boy \textsc{pl} \textsc{erg} flute \textsc{ipfv}-release-\textsc{pl}  \\
 \glt `Long ago, among us women used to play the Jews' harp, while men used to play the flute.' (150907 ZNGro, 10)
\end{exe}
 
This example is typical of procedural texts: the generic is consistently used by women speakers to refer to activities typically performed by women, and the \textsc{3pl} for men's duties (and vice-versa with men speakers). 

This contrast is also found when discussing differences between Tibetan and Chinese people, in particular the names given to animals and plants. In (\ref{ex:khWnajme}), Chinese are referred to collectively in the \textsc{3sg} (the plural would also be possible in the same context), while Tibetans (the ethnic group to which the speaker associates herself) are indexed on the verb by the generic \forme{kɯ-} prefix (§\ref{sec:irregular.transitive}).

\begin{exe}
\ex \label{ex:khWnajme}
\gll nɯnɯ kupa kɯ <gouweicao> tu-ti ŋu tɕe, kɯrɯ ra kɯ nɯ li khɯnajme tu-kɯ-ti ŋu tɕe, \\
\textsc{dem} Chinese \textsc{erg} setaria.viridis \textsc{ipfv}-say be:\textsc{fact} \textsc{lnk}  Tibetan \textsc{pl} \textsc{erg} \textsc{dem} again setaria.viridis \textsc{ipfv}-\textsc{genr}-say be:\textsc{fact}be:\textsc{fact} \textsc{lnk}  \\
\glt `The Chinese call it `gouweicao', Tibetans call it \forme{khɯnajme}.' (16-RlWmsWsi, 54)
\end{exe}

The contrast between \forme{kɯ-/wɣ-} generic person and \textsc{3pl} indexation can thus be described as \textit{speaker-inclusive} vs. \textit{speaker-exclusive} generic marking.

Another crucial difference between speaker-inclusive generic and \textsc{3pl}, is that unlike the former, the latter can refer to several different referents in the same sentence. In example (\ref{ex:amAtAndonW.ma.GWznAndAGnW}) for instance, the \textsc{3pl} on the verbs \forme{a-mɤ-tɤ-ndo-nɯ} `let them not take it' and \forme{ɣɯ-z-nɤndɤɣ-nɯ} `it will poison them'\footnote{The inverse on this verb does mark generic subject (§\ref{sec:indexation.generic.tr}), since the subject of this verb is the poisonous mushroom. Rather, it occurs here to indicate a transitive configuration with inanimate subject and animate object (§\ref{sec:inverse.3.3.saliency}).} agrees with the noun phrase \forme{tɯrme ra}, which is to be interpreted as meaning `other people' (§\ref{sec:other.pro}) rather than generic person (§\ref{sec:tWrme.generic}). Its referent is different from that of the generic \textsc{3pl} in the rest of the passage, such as the matrix verb \forme{ɲɯ-sɯso-nɯ} `they think that ...' or the preceding verb \forme{pjɯ-rɤtɕɯmtɕaʁ-nɯ} `they stamp on it'.

\begin{exe}
\ex \label{ex:amAtAndonW.ma.GWznAndAGnW}
\gll tɕe sɤndɤɣ tu-ti-nɯ ŋgrɤl. tɕe a-pɯ-sɯχsɤl-nɯ ʁo tɕe maka mɤ-pʰɯt-nɯ, na-phɯt-nɯ kɯnɤ cʰɯ-βde-nɯ ɕti. tɕe pjɯ-rɤtɕɯmtɕaʁ-nɯ ma `tɯrme ra kɯ a-mɤ-tɤ-ndo-nɯ ma ɣɯ-z-nɤndɤɣ-nɯ' ɲɯ-sɯso-nɯ. \\
\textsc{lnk} be.poisonous:\textsc{fact} \textsc{ipfv}-say-\textsc{pl} be.usually.the.case:\textsc{fact} \textsc{lnk} \textsc{irr}-\textsc{pfv}-recognize-\textsc{pl} \textsc{advers} \textsc{lnk} at.all \textsc{neg}-take.out:\textsc{fact}-\textsc{pl} \textsc{aor}:3\fl{}3$'$-take.out-\textsc{pl} even \textsc{ipfv}-throw.away-\textsc{pl} be.\textsc{aff}:\textsc{fact} \textsc{lnk} \textsc{aor}-tread-\textsc{pl} \textsc{lnk} people \textsc{pl} \textsc{erg} \textsc{irr}-\textsc{neg}-\textsc{pfv}-take-\textsc{pl} \textsc{lnk} \textsc{inv}-\textsc{caus}-be.poisoned:\textsc{fact}-\textsc{pl} \textsc{sens}-think-\textsc{pl} \\
\glt `People say that (this species of mushroom) is poisonous. If they recognize it, they don't pick it up, even if they pick it up they throw it away and stamp on it, thinking `(This way) other people won't take it and it will not poison them.'' (23-grWBgrWBftsa, 23-27)
\end{exe}

\section{Person indexation on non-finite predicative words} \label{sec:non.finite.indexation}
Despite being a highly verb-prominent language, Japhug has some non-verbal predicates (§\ref{sec:non.verbal.predicates}). There is evidence that a few nominals occurring as predicates, including nominalized verb forms and nouns, have acquired person indexation markers.

Phenomena of this type are uncommon, but clear examples do exist in Indo-European (\citealt[414]{pott1859}). For instance, in Greek, the adverbs \grec{δεῦρο} `hither' and \grec{τῆ} `here, take it' have developed the plural forms \grec{δεῦτε} and \grec{τῆτε}, with the plural present imperative suffix \grec{-τε} (and unexplainable loss of the syllable \grec{-ρο} in \grec{δεῦτε}), by contamination with imperative verb forms (\grec{φέρε}, \grec{φέρετε}), which occur in the same contexts (\citealt[113--114]{viti15wandel}).  A similar case is found in Gothic, where the adverb \forme{hiri} `hither' (which incidentally translates Greek \grec{δεῦρο}) has dual \forme{hirjats} and plural \forme{hirjiþ}, modelled on imperatives (\citealt[104]{braune53gotische}).

Inflectionalization of nominals is attested in the case of a few phatic  (§\ref{sec:phatic.inflectionalization}) and exclamative (§\ref{sec:exclamative.inflectionalization}) expressions in Japhug.

\subsection{Phatic expressions} \label{sec:phatic.inflectionalization}
The  expression \japhug{sɤrma}{good night}, used to address someone leaving one's house in the evening, transparently derives from the oblique participle \forme{sɤ-rma} `place where/time when one stays overnight' (§\ref{sec:oblique.participle}) from the verb \japhug{rma}{stay the night, live}, as illustrated by example (\ref{ex:Ckurma}).  

\begin{exe}
\ex \label{ex:Ckurma}
\gll tɕe ɯ-sɤ-rma nɯnɯ, praʁ, praʁpa tɕe ɕ-ku-rma ɲɯ-ŋu \\
\textsc{lnk} \textsc{3sg}.\textsc{poss}-\textsc{obl}:\textsc{pcp}-stay.the.night \textsc{dem} cliff cavern \textsc{lnk} \textsc{tral}-\textsc{ipfv}-stay.the.night \textsc{sens}-be \\
\glt `The place where its spends the night is the cliffs, it goes to spend the night in caverns under the cliffs.' (20-xsar, 36)
\end{exe}

This expression is probably the abbreviation of a phrase such as `go back to your resting place' (which would be \forme{nɤ-sɤrma jɤ-nɯɕe}).

Yet, when addressing more than one person, the dual form  \forme{sɤrma-ndʑi} and the plural  \forme{sɤrma-nɯ} are used, with the 2/3 dual \forme{-ndʑi} and plural \forme{-nɯ} suffixes found in verb paradigms (§\ref{sec:intr.23}). These suffixes normally only appear on finite verbs. The forms \forme{sɤrma}, \forme{sɤrma-ndʑi} and \forme{sɤrma-nɯ} could in principle be analyzed as a Factual Non-Past form (§\ref{sec:fact.morphology}), but there are three problems with this hypothesis. 

First, the meaning of the expression (`have a good night') is hardly compatible with the Factual Non-Past; an Imperative or Irrealis form would be expected instead, and a second person prefix \forme{tɯ-} would in any case be required. Second, while there are several \forme{sɤ-} verbal derivational prefix, none of them (proprietive §\ref{sec:proprietive.allomorphy}, antipassive §\ref{sec:antipassive.sA} and causative §\ref{sec:caus.sA}) has a function which could account for a derivation such as `spend the night' $\Rightarrow$ `(have a) good night'. Third, no other verb forms (including first or third person), finite or non-finite, are attested for \forme{sɤrma}. 

A more promising approach to account for these verb forms is analogy with other phatic expressions involving finite verb forms. The most probable one is the verb \japhug{astu}{be straight}, whose Imperative is used to mean `goodbye' (literally `(walk) straight') as in (\ref{ex:tAstundZi}). These verb forms optionally occur with the sentence final particle \forme{je} (§\ref{sec:fsp.imp}), another commonality with  \japhug{sɤrma}{good night}.

\begin{exe}
\ex \label{ex:tAstundZi}
\begin{xlist}
\ex 
\gll tɤ-ɤstu (je) \\
\textsc{imp}-be.straight  \textsc{sfp} \\
\glt `Goodbye' = `(walk) straight' (singular)
\ex 
\gll tɤ-ɤstu-ndʑi (je) \\
\textsc{imp}-be.straight-\textsc{du}  \textsc{sfp} \\
\glt `Goodbye' (dual)
\end{xlist}
\end{exe}

The dual and plural forms of \japhug{sɤrma}{good night} can thus be explained as trivial four-part analogy as in \tabref{tab:analogy.sArma}.

\begin{table}
\caption{The dual/plural forms of \japhug{sɤrma}{good night} as result of four-part analogy} \label{tab:analogy.sArma}
\begin{tabular}{llll}
\lsptoprule
\forme{tɤstu} `goodbye.\textsc{sg}' & {\forme{sɤrma} `good night.\textsc{sg}'} &\\
\forme{tɤstu-ndʑi} `goodbye.\textsc{du}' &X $\Rightarrow$  \forme{sɤrma-ndʑi} `good night' \\
\lspbottomrule
\end{tabular}
\end{table}

The expression \japhug{kɤnɤβdi}{take care}, used when leaving from someone's place, and which also has a dual \forme{kɤnɤβdi-ndʑi} and a plural \forme{kɤnɤβdi-nɯ}, might also be a inflectionalized form of the \forme{kɤ-} infinitive (§\ref{sec:velar.inf}) of the  tropative  \forme{nɤ-} (§\ref{sec:tropative}) of the stative verb \japhug{βdi}{be well, be good}.

However, this case is less compelling than \japhug{sɤrma}{good night}, because \forme{kɤ-nɤβdi-ndʑi} and \forme{kɤ-nɤβdi-nɯ} can alternatively be formally analyzed as imperatives with the \textsc{eastwards} \forme{kɤ-} preverb (§\ref{sec:kamnyu.preverbs}), and because \forme{nɤβdi} is also found in regular finite forms (as in \ref{ex:WkutWnABdi}).

\begin{exe}
\ex \label{ex:WkutWnABdi}
\gll  ɯ-ku-tɯ-nɤβdi? ɯ-ku-tɯ-pe? \\
 \textsc{qu}-\textsc{prs}-2-feel.well  \textsc{qu}-\textsc{prs}-2-good \\
\glt `Do you feel well, are you fine?' (conversation, 16-12-28)
\end{exe}

\subsection{Exclamative expressions} \label{sec:exclamative.inflectionalization}
The expression \japhug{dɯxpa}{poor ...} seems at first glance to be a verb, as it not only takes dual and plural suffixes (\forme{dɯxpa-nɯ} `poor them', example \ref{ex:dWxpanW}), but it can also receive first person indexation (such as \textsc{1du} in \ref{ex:dWxpatCi}). The variant form \forme{zdɯxpa} is also attested (see \ref{ex:kuZGACthWza}, §\ref{sec:ipfv.hortative}).

\begin{exe}
\ex \label{ex:dWxpanW}
\gll wo a-rɟit ra dɯxpa-nɯ ma nɯ ɯ-xtu ɯ-ŋgɯ nɯtɕu ɣɤʑu-nɯ rca  \\
\textsc{interj} \textsc{1sg}.\textsc{poss}-offspring \textsc{pl} poor-\textsc{pl} because \textsc{dem} \textsc{3sg}.\textsc{poss}-belly  \textsc{dem}:\textsc{loc}  exist:\textsc{sens}-\textsc{pl} \textsc{sfp} \\
\glt `My poor children, there are in his (the wolf's) belly' (140430 lang he qizhi xiaoshanyang-zh, 130)
\end{exe}

\begin{exe}
\ex \label{ex:dWxpatCi}
\gll tɕiʑo ndɤ dɯxpa-tɕi ɣe, nɤ tɕendɤre, kɤ-ntɕha ɯ-spa ʑo ɕti-tɕi \\
\textsc{1du} on.the.other.hand poor-\textsc{1du} \textsc{sfp} \textsc{lnk} \textsc{lnk} \textsc{obj}:\textsc{pcp}-kill \textsc{3sg}.\textsc{poss}-material \textsc{emph} be.\textsc{assert}:\textsc{fact}-\textsc{1du} \\
\glt `We, on the other hand, poor of us! We are to be butchered.' (kandZislama2003.210)
\end{exe}

Yet, \forme{dɯxpa} has a defective paradigm: it cannot take any prefix, including the second person \forme{tɯ\trt}, any TAM marker or any nominalization prefix. In addition, if \forme{dɯxpa} were to be analyzed as a verb, it would be anomalous, as it is borrowed from Tibetan \tibet{སྡུག་པ་}{sdug.pa}{suffering} (also spelled  \forme{sdug.ga}),\footnote{For an account of the alternation between \forme{p/b} and \forme{g} in the spelling, see \citet{hill11hb} and the references therein.} a nominalized form taking the \forme{-pa/-ba} suffix. 

While examples of verbs directly borrowed from Tibetan into Japhug are numerous, at least in the earliest layer (for instance \japhug{rɟɯɣ}{run}, \japhug{βzɟɯr}{change, correct} etc, see \citealt{jacques19contact}), nouns or nominalized verbs borrowed from Tibetan are never converted to verbs without denominal prefixes. Incidentally, we find in Japhug two verbs derived from the same Tibetan etymon (in the variant \forme{°zdɯxpa}, from an earlier layer of borrowing) by means of denominal prefixes, the intransitive stative verb \japhug{sɤzdɯxpa}{be pitiful} and the transitive verb \japhug{nɯzdɯxpa}{have pity for}. The word \forme{dɯxpa} cannot be analyzed either as a verb derived from a noun by zero-derivation, as zero denominal derivation does not exist in Japhug (§\ref{sec:verb.backformation}).

A better account of the defectiveness of \forme{dɯxpa} and its etymology is that it originally was an exclamative noun (§\ref{sec:exclamative.IPN}, §\ref{sec:non.verbal.predicates}), and that the third person and first person indexation suffixes were added by contamination with finite stative verbs used in exclamative sentences, as \japhug{sɤzdɯxpa}{be pitiful} in  (\ref{ex:pjAsAzdWxpanW}).

\begin{exe}
\ex \label{ex:pjAsAzdWxpanW}
\gll ndʑi-ɣi ra, nɯni mɯ-ɲɤ-k-ɤtɯɣ-nɯ-ci, pjɤ-sɤzdɯɣpa-nɯ ma \\
\textsc{3du}.\textsc{poss}-relatives \textsc{pl} \textsc{dem}:\textsc{du} \textsc{neg}-\textsc{ifr}-\textsc{peg}-meet-\textsc{pl}-\textsc{peg} \textsc{ifr}.\textsc{ipfv}-be.pityful-\textsc{pl} \textsc{lnk} \\
\glt `Their relatives did not meet them (again), poor them.' (2003zrantCWtWrme, 78)
\end{exe}

 
 %སྡུག་པ་ སྡུག་ག་
\section{Historical perspectives}
This section explores a range of hypotheses to account for the synchronic resemblance between indexation markers (indexation suffixes, the inverse prefix and the local scenario portmanteau prefixes) and other morphemes that are possibly historically related in Japhug, drawing on comparative data from other Gyalrong languages and beyond.

\subsection{Indexation suffixes, pronouns and possessive prefixes} \label{sec:indexation.suffixes.history}
The indexation suffixes (§\ref{sec:intransitive.paradigm}) are similar to the corresponding possessive prefixes (§\ref{sec:possessive.paradigm}) and pronouns (§\ref{sec:pers.pronouns}) in Japhug, as shown by the data in \tabref{tab:personal.suffixes.japhug}. In addition to the Kamnyu dialect, this table includes data on the Tatshi dialect  (from \citealt{linluo03}, \citealt{lin11direction} and personal communication) between brackets. The two dialects differ mainly in the presence of alveolo-palatal affricates in Kamnyu Japhug in the dual, while dental affricates are found in Tatshi.

\begin{table}
\caption{Indexation suffixes, pronouns and possessive prefixes in Japhug}\label{tab:personal.suffixes.japhug}
\begin{tabular}{lllllllll} \lsptoprule
 & Indexation suffix & Pronoun & Possessive Prefix & \\
\midrule
1\sg{}& \forme{-a} & \forme{aʑo} (\forme{ŋa}) &	\forme{a-}  &		 \\
2\sg{}& &\forme{nɤʑo}  &	\forme{nɤ-}  &			 \\
3\sg{}&  &\forme{ɯʑo}   (\forme{mi}) &	\forme{ɯ-}  &			 \\
\hline
1\du{}& \forme{-tɕi} (\forme{-tsə}) & \forme{tɕiʑo}  &	\forme{tɕi-}  (\forme{tsə-})  &			 \\
2\du{} & \forme{-ndʑi} (\forme{-ndzə}) & \forme{ndʑiʑo}  &	\forme{ndʑi-}  (\forme{ndzə-}) &		\\	
3\du{}& \forme{-ndʑi} (\forme{-ndzə}) & \forme{ʑɤni}  &	\forme{ndʑi-} (\forme{ndzə-}) &		 \\	
\hline
1\pl{}& \forme{-ji} & \forme{iʑo}    &	\forme{i-}  &			 \\
2\pl{}& \forme{-nɯ} (\forme{-nə}) & \forme{nɯʑo}  &	\forme{nɯ-}  (\forme{nə-}) &			 \\
3\pl{} & \forme{-nɯ} (\forme{-nə}) & \forme{ʑara}  &	\forme{nɯ-} (\forme{nə-}) &			\\
\lspbottomrule
\end{tabular}
\end{table}

In the other Gyalrong languages, very similar affixes are found, as shown by \tabref{tab:person.suff.pref.rgyalrong} (data from \citealt[139]{jackson98morphology}, \citealt[562]{jackson17tshobdun}, \citealt{gongxun14agreement, gong18these}, \citealt[168;198]{linxr93jiarong}). 


\begin{table}
\caption{Indexation suffixes and possessive prefixes in Tshobdun, Zbu and Situ} \label{tab:person.suff.pref.rgyalrong}
\begin{tabular}{Xll|ll|lll} 
\lsptoprule
 &\multicolumn{2}{c}{Tshobdun} & \multicolumn{2}{c}{Zbu} & \multicolumn{2}{c}{Situ} \\
\midrule
1\sg{}& \forme{ɐ-}  & \forme{-aŋ} &  \forme{ɐ-}  & \forme{-ŋ} & \forme{ŋa-/ŋə-}  & \forme{-ŋ} \\
2\sg{}& \forme{nɐ-}  &  & \forme{nɐ-}  & & \forme{na-/nə-}   & \forme{-n} \\
3\sg{}&  \forme{o-}  &   & \forme{və-}  & & \forme{wa-/wə-}   & (\forme{-w}) \\
\hline
1\du{}& \forme{tsə-}  & \forme{-tsə} &\forme{tɕə-}  & \forme{-tɕə} & \forme{ndʒa-/ndʒə-}   & \forme{-tʃʰ} \\
2/3\du{} &\forme{ⁿdzə-}  & \forme{-ⁿdzə} & \forme{ⁿdʑə-}  & \forme{-ⁿdʑə} & \forme{ndʒa-/ndʒə-}  & \forme{-ntʃʰ} \\
\hline
1\pl{}& \forme{jə-}  & \forme{-jə} & \forme{ⁿgə-}  & \forme{-jə} & \forme{ja-/jə-}  & \forme{-i} \\
2/3\pl{}&\forme{nə-}  & \forme{-nə} & \forme{ɲə-}  & \forme{-ɲə} & \forme{ɲa-/ɲə-}   & \forme{-ɲ} \\
\lspbottomrule
\end{tabular}
\end{table}

The main differences between the indexation systems of the four Gyalrong languages include the following observations: (i) only Situ has a \textsc{2sg} suffix (intransitive and \textsc{2sg} object)\footnote{In Situ, the second person singular is indexed by both a prefix and a suffix, while in the other languages including Japhug, it is only indexed by the prefix (§\ref{sec:intr.23}). } and a third person object suffix, (ii) the \textsc{1sg} suffix presents unique correspondences, and has many allomorphs in Zbu and Situ,  (iii) Zbu has a \textsc{1pl} possessive prefix that is unrelated to the \textsc{1pl} suffix, (iv) Situ has no distinction between \textsc{1du} and \textsc{2/3du} possessive prefixes, and (v) the \textsc{1du}, \textsc{2/3du} and \textsc{2/3pl} affixes are either (alveolo)-palatal or dental across the languages.

As shown by \tabref{tab:palatalized.dual.plural.rgyalrong}, Tshobdun and Tatshi Japhug only have dental person indices, Zbu and Situ only (alveolo-)palatal ones, and Kamnyu Japhug is intermediate between the two groups, with alveolo-palatal dual affixes \forme{-tɕi/-ndʑi} and dental plural affixes \forme{-nɯ}. It is noteworthy that Kamnyu Japhug, the dialect geographically closest to Tshobdun, is less similar to that language in this regard than the Tatshi dialect, which is not in direct contact with Tshobdun.

\begin{table}
\caption{(Alveolo-)palatal vs. dental person indices in Gyalrong languages} \label{tab:palatalized.dual.plural.rgyalrong}
\begin{tabular}{llllllll} 
\lsptoprule
&Tshobdun & Tatshi Japhug & Kamnyu Japhug & Zbu & Situ \\
\midrule
\textsc{1du} &  \forme{-tsə}  \grise{} &  \forme{-tsə}  \grise{}  &  \forme{-tɕi} & \forme{-tɕə} & \forme{-tʃʰ} \\
\textsc{2/3du} & \forme{-ⁿdzə}   \grise{} & \forme{-ndzə}   \grise{} & \forme{-ndʑi} &  \forme{-ⁿdʑə} & \forme{-ntʃʰ} \\
\textsc{3pl} &\forme{-nə}  \grise{} & \forme{-nə}  \grise{}  & \forme{-nɯ}  \grise{}  & \forme{-ɲə} & \forme{-ɲ} \\
\lspbottomrule
\end{tabular}
\end{table}

A possible explanation to account for the resemblances between the suffixes, the possessive prefixes and the pronouns is to argue that the former were grammaticalized from the latter, as has been proposed by \citet{lapolla92}. In the same line of reasoning, the second and third person dual and plural suffixes, which are similar to the dual and plural nominal markers in Situ and Tatshi Japhug, could be argued to have originated from them. In Tatshi Japhug for instance, the dual \forme{-ndzə} and plural \forme{-nə} suffixes could be analyzed as deriving from the dual \forme{ndzə} and plural \forme{nəjo} nominal clitics  (which differ from those of Kamnyu Japhug, §\ref{sec:number.determiners}).

% -nəjo -ndʒe, -ɲe
The idea of transparent grammaticalization from pronouns or number markers in indexation suffixes is however not as straightforward as it might appear at first glance. Leaving aside extra-Gyalrong comparative evidence (\citealt{jacques12agreement, delancey14second}), and the fact that most of the pronouns in Gyalrong languages are derived from possessive prefixes rather than the opposite (\citealt{jacques16th}, §\ref{sec:pers.pronouns}),  if the similarity between the three series of person markers were to be explained as resulting exclusively from a recent grammaticalization, the pattern in \tabref{tab:person.suff.pref.rgyalrong}, where the place of articulation (alveolo-palatal vs. dental) of dual and plural indexation suffixes is aligned on that of the pronouns and possessive prefixes and draws an isogloss across the dialects of Japhug (\tabref{tab:palatalized.dual.plural.rgyalrong}), would imply that the grammaticalization postdated not only the breakup of proto-Gyalrong, but even that of the common ancestor of modern Japhug dialects. 

The indexation system of Gyalrong languages cannot however be that recent, in particular because the stem alternation system, which contributes to person indexation (§\ref{sec:stem3.distribution}, §\ref{sec:indexation.mixed}) is clearly reconstructible to the common ancestor of Gyalrong languages and Tangut \citep{gong16stems} and is too complex and irregular (in particular in Zbu, see \citealt{jackson04showu} and \citealt{gong18these}) to be a recent development.

Three alternative types of explanation can be explored to account for the correspondences between affixes and pronouns: analogical simplification of allomorphy, contamination and degrammaticalization.

First, it is possible that proto-Gyalrong had phonetically conditioned allomorphs of the suffixes (the \textsc{1sg} suffix has maintained some allomorphy in Zbu and Situ, see \citealt[46]{gongxun14agreement} and \citealt[198]{linxr93jiarong}), and that the attested languages have generalized one of the allomorphs.  Analogical levelling has certainly taken place in the case of the \textsc{1sg}. Note that the form \forme{-aŋ} in Tshobdun  cannot derive from proto-Gyalrong \forme{*-aŋ} (which yields Tshobdun \forme{-i}), and the Zbu and Situ \forme{-ŋ} cannot be a phonetic reflex of \forme{*-ŋ} in most cases, otherwise for instance stems in \forme{-a} should alternate with \forme{-o} $\leftarrow$ \forme{*-aŋ} in Situ in the \textsc{1sg}.  In Situ and Zbu, the final \forme{-ŋ} in open syllables must have been restored, due to analogical spread from a particular context where the \forme{*-ŋ} was maintained.

Second, the similarity between possessive prefixes, indexation suffixes and pronouns could be the result of mutual contamination and convergence. Cases of contamination between indexation systems and possessive affixes are attested. In Hebrew and Phoenician, for instance, the first person singular perfect suffix \forme{-tî} has an unexpected vocalism, as it is commonly agreed that proto-North-West Semitic suffix was \forme{*-tu}, whose outcome should have been $\dagger$\forme{-t}. The irregular Hebrew form is explained as due to contamination from either the \textsc{1sg} pronoun \forme{ʔănî} or the \textsc{1sg} possessive suffix \forme{-î} (\citealt[122-123;132-133]{jouon06}, \citealt[227--229]{suchard16vowels}).  

Contamination could account  for the \textsc{1sg} affixes in Japhug.  The possessive prefix \forme{a-} is the regular outcome of earlier \forme{*ŋa\trt}, a proto-form which also account for the \textsc{1sg} prefixes in the other languages. The \textsc{1sg} suffix \forme{-a} however does not regularly correspond to the suffixes \forme{-aŋ} and \forme{-ŋ} found in Tshobdun, Zbu and Situ. One possibility is that \forme{*-ŋɐ} was one of the proto-Gyalrong allomorphs of the \textsc{1sg} suffix,\footnote{In Bantawa for instance, the \textsc{1sg} suffix has three allomorphs \forme{-ŋ}, \forme{-ŋ}, and \forme{-ŋa} (\citealt[155]{doornenbal09}). } and was generalized. Another explanation is that only \forme{*-ŋ} was present in proto-Gyalrong, and that the  \forme{*-ŋɐ} precursor of Japhug \forme{-a} arose by contamination of \forme{*-ŋ} with the (historically related) possessive prefix \forme{*ŋɐ-}. 


Third, the possibility of degrammaticalization from person indexation suffixes to nominal number markers in Tatshi Japhug and Situ should be taken into consideration. Kamnyu Japhug, Tshobdun and Zbu have dual and plural markers that are completely different from indexation suffixes (Kamnyu \forme{ni} / \forme{ra}, Tshobdun \forme{niʔ} /\forme{rɐʔ} and Zbu \forme{ɲi} / \forme{réʔ}, see §\ref{sec:number.determiners}, \citealt{jackson98morphology} and \citealt{gong18these}). The plural marker is cognate to Pumi \forme{=ɹə} (\citealt[135]{daudey14grammar}): Japhug \forme{-a} regularly corresponds to Pumi \forme{-ə} in the native vocabulary \citep{jacques17num}, and it is therefore unlikely to be a Northern Gyalrong innovation. 

If the Zbu, Tshobdun and Kamnyu Japhug number markers above are conservative,  the corresponding Tatshi Japhug \forme{ndzə} / \forme{nəjo} and Cogtse Situ \forme{ndʒe} / \forme{ɲe} forms must therefore be innovations, and by consequence, it cannot be argued that the 2/3 indexation suffixes derive from these number markers. The opposite scenario is possible: \citet[204--206]{norde09degrammaticalization} discusses the case of the Irish \textsc{1pl} pronoun \forme{muid} which comes from one of the allomorphs of the  \textsc{1pl} future indexation suffix. A similar type of debonding can account at least for the Tatshi Japhug dual \forme{ndzə} nominal marker, from the 2/3 dual \forme{-ndzə} indexation suffix.\footnote{The case of plural \forme{nəjo} is more complicated: the  \forme{nə-} element here probably rather derives from the distal demonstrative. } The acquisition of indexation suffixes by several non-verbal words discussed in §\ref{sec:non.finite.indexation} may have contributed to the reanalysis of indexation suffixes as number-marking enclitics.

Much remains unclear about the history of person indexation suffixes and the nature of their relationship with pronouns, possessive prefixes and number markers in Gyalrong languages. A more satisfying account of these data will only become possible when a fully explicit system of proto-Gyalrong reconstruction, and complete data on as many varieties as possible becomes available.

\subsection{The inverse prefix} \label{sec:inverse.history}
Two Gyalrong-internal scenarios could be proposed to account for the origin of the inverse prefix. 

First, it could be proposed that the inverse comes from the \textsc{3sg} possessive prefix. In Tshobdun, Zbu and Situ, the inverse prefix (\forme{o\trt}, \forme{və-} and \forme{wə\trt}, respectively) is homophonous with the \textsc{3sg} possessive prefix (see \tabref{tab:3sg.inv} in §\ref{sec:3sg.possessive.form}). This is not the case in Japhug, where the inverse \forme{ɣɯ-} / \forme{-wɣ} is clearly different from the \textsc{3sg} \forme{ɯ\trt}, but it is possible that the \textsc{3sg} possessive underwent an irregular development, as argued in §\ref{sec:3sg.possessive.form}. The 3$'$\fl{}3\textsc{sg} form would originally be a non-finite form taking a possessive prefix (for instance a predecessor of the bare infinitive, §\ref{sec:bare.inf}), and the inverse prefix would have spread from the 3$'$\fl{}3 to the mixed scenarios 3\fl{}2 and 3\fl{}1, following the pathways described in \citealt{jacques18directionality}.

Second, the partial resemblance between the Japhug allomorph \forme{ɣɯ-} and the associated motion cislocative \forme{ɣɯ-} prefix (§\ref{sec:cislocative.morpho}) could support the idea that the inverse derives from the cislocative, following a well-known grammaticalization pathway \citep{jacques14inverse}. This hypothesis is however less likely, as the cislocative itself may be an innovation in Gyalrong languages (§\ref{sec:am.prefixes}).
 
In any case, cognates of the inverse prefix are found in other Gyalrongic languages \citep{lai15person} and in Kiranti \citep{jacques12agreement}, so that whatever the ultimate origin of this prefix, it was already grammaticalized at the proto-Gyalrongic level.

\subsection{The origin of portmanteau prefixes} \label{sec:portmanteau.prefixes.history}
The presence of portmanteau prefixes in the local 1\fl{}2 and 2\fl{}1 configurations in Japhug is not unusual crosslinguistically \citep{heath98skewing}. However, the resemblance of the 1\fl{}2 prefix \forme{ta-} to the second person \forme{tɯ-} prefix on the one hand, and of the 2\fl{}1 prefix \forme{kɯ-} to various non-finite velar prefixes (§\ref{sec:velar.nmlz.history}) on the other hand, raises the question of their potential historical relatedness. 

The form of the local configuration prefixes is very similar in other Gyalrong languages, as shown by \tabref{tab:local.rgy} (data from \citealt[218]{linxr93jiarong}, \citealt{jackson02rentongdengdi} and \citealt{gongxun14agreement}). 

\begin{table}
\caption{Local scenario prefixes in Gyalrong languages} \centering \label{tab:local.rgy} 
\begin{tabular}{lllll}
\lsptoprule
& 1\fl{}2 & 2\fl{}1 \\
\midrule
Japhug &  \forme{ta-} & \forme{kɯ-} \\
Tshobdun &  \forme{tɐ-} & \forme{kə-o\trt}, \forme{tə-o-} \\
Zbu &  \forme{tɐ-} &\forme{kə-w\trt}, \forme{tə-w-} \\
Situ &  \forme{ta-} & \forme{kə-w-} \\
\lspbottomrule
\end{tabular}
\end{table}

Only two differences are found in the local domain across the Gyalrong languages: Japhug does not have the inverse \forme{wɣ-} prefix in the  2\fl{}1 form, and Zbu and Tshobdun allow an alternative form with the second person prefix \forme{tə-} and the inverse prefix. In all four languages, the verb takes suffixes that are coreferent with the object (second person in 1\fl{}2 and first person in  2\fl{}1). Situ is the only language with a suffix \forme{-n} in the 1\fl{}\textsc{2sg} form, the same as that found in intransitive \textsc{2sg} and 3\fl{}\textsc{2sg}.

\citet[420--421]{jacques18generic} proposes that the 1\fl{}2 prefix originates from the fusion of  the second person prefix \forme{tɯ-} with the agentless passive \forme{a-} (from \forme{*ŋa\trt}, \citealt{jacques07passif}), which yields the expected form in all four languages. In this view, a form like Japhug \forme{ta-mbi} 1\fl{}2-chase-\textsc{sg} `I will give it to you$_{SG}$' would have developed through the following stages: 

\begin{itemize}
\item *\forme{tə-ŋa-mbi-nə}  2-\textsc{pass}-give-\textsc{2sg} `it will be given to you' (Passive form)
\item *\forme{ta-mbi-nə}  2:\textsc{pass}-give-\textsc{2sg} (Regular phonological fusion between the person marker and the passive prefix, attested in all four Gyalrong languages, §\ref{sec:contraction})  
\item  *\forme{ta-mbi-nə}  1\fl{}2-give-\textsc{2sg} `I will give it to you' (reanalysis of the fused form as a portmanteau prefix; the unspecified agent of the passive construction is construed as being first person)
\item  \forme{ta-mbi} 1\fl{}2-give `I will give it to you' (loss of \textsc{2sg} suffix in Japhug)
\end{itemize}

This scenario is not completely straightforward; in particular, the passive derivation \japhug{ambi}{be given} of the verb \japhug{mbi}{give} takes the theme, not the recipient, as intransitive subject (§\ref{passive.ditransitive}), and passives in Japhug are only rarely attested with first or second person subjects (§\ref{sec:passive}). However, there are no major phonological or morphological obstacles against this hypothesized scenario, and  good typological parallels have been described \citep{delancey18sociopragmatic}.

In the case of 2\fl{}1 \forme{kɯ\trt}, \citet{jacques12agreement} originally proposed that this prefix might be an archaism of Gyalrong languages, based on the principle of archaic heterogeneity \citep{hetzron76two}, using the data in \tabref{tab:second.japhug.kiranti}. In  second person forms other than 1\fl{}2, some Kiranti languages have a dental stop prefix (Bantawa \forme{tɨ\trt}, \citealt{doornenbal09}), and Limbu has the velar prefix \forme{kɛ-} \citep{michailovsky02dico}, correspondance to Japhug \forme{tɯ-} in intransitive second person and 2\fl{}3, and  \forme{kɯ-} in 2\fl{}1 configurations. A possible way of interpreting this corresponding would be to suppose that a pattern similar to that found in Gyalrongic (with a velar prefix in  2\fl{}1 and a dental stop prefix in other second person forms) has to be reconstructed in proto-Kiranti: Bantawa would have generalized the \forme{tɨ-} prefix to the  2\fl{}1 slot (a type of analogical levelling attested in Zbu and Tshobdun, see \tabref{tab:local.rgy}), and Limbu the velar prefix to the 2\fl{}3 and intransitive forms.
 
\begin{table}
\caption{Comparison of second person forms in Japhug and selected Kiranti languages}  \label{tab:second.japhug.kiranti}
\begin{tabular}{llllll}
 \lsptoprule
 & Japhug & Bantawa & Limbu \\
 \midrule
2.\textsc{intr} & \bleu{\forme{tɯ}}-\ro{}&  \bleu{\forme{tɨ}}-\ro{}  &  \rouge{\forme{kɛ}}-\ro{} \\
2\fl{}3 &  \bleu{\forme{tɯ}}-\ro{} &  \bleu{\forme{tɨ}}-\ro{}-\forme{u} & \rouge{\forme{kɛ}}-\ro{}-\forme{u} \\
2\fl{}1 & \rouge{\forme{kɯ}}-\ro{}-\forme{a} & \bleu{\forme{tɨ}}-\ro{}-\forme{aŋ} &\rouge{\forme{kɛ}}-\ro{}-\forme{aŋ} \\
\hline
1\fl{}2 &  \forme{ta}-\ro{}&\ro{}-\forme{na} &\ro{}-\forme{nɛ} \\
 \lspbottomrule
\end{tabular}
\end{table}

This hypothesis raises three issues: (i) the 2\fl{}1 form is less common than second intransitive and 2\fl{}3, and is unlikely to have served as the basis for analogical levelling in Limbu, (ii) the presence of an inverse prefix in the 2\fl{}1 form in other Gyalrong languages is unexplained and (iii) the homophonies between the second person possessive prefix \forme{kɛ-} and the indexation prefix in Limbu on the one hand, and the \forme{kɯ-} 2\fl{}1 portmanteau prefix and the non-finite \forme{kɯ-} prefixes on the other hand, would be due to chance.

Since the \forme{kɛ-} second person indexation prefix in Limbu can be explained as an internal innovation 
\citep[94]{jacques12agreement}, it is necessary to explore the possibility that Japhug the 2\fl{}1 \forme{kɯ-} portmanteau is also an Gyalrong innovation, especially since no other traces of such a putative prefix are found elsewhere in the Trans-Himalayan family.

As mentioned above, Japhug 2\fl{}1 \forme{kɯ-} differs from the 2\fl{}1 portmanteau prefixes found in the three other Gyalrong languages, which co-occur with the inverse prefix (\tabref{tab:local.rgy}). An identical difference appears in the generic object form, which takes of a simple \forme{kɯ-} prefix in Japhug (§\ref{sec:indexation.generic.tr}) but has an additional inverse prefix \forme{kə-o-} in Tshobdun (\tabref{tab:genr.japhug.tshobdun}, \citealt{sun14generic}).

\begin{table}
\caption{Comparison of generic person prefix in Japhug and Tshobdun } \label{tab:genr.japhug.tshobdun}
\begin{tabular}{lllll}
\lsptoprule
 & Japhug & Tshobdun \\
\midrule
\textsc{genr}.S & \forme{kɯ-} &  \forme{kə-/kɐ-} \\
\textsc{genr}.A &  \forme{wɣ-} & \forme{kə-/kɐ-} \\
\textsc{genr}.P &  \forme{kɯ-} & \forme{kə-o-} \\          
\hline
2\fl{}1 &  \forme{kɯ-} & \forme{kə-o-} \\          
\lspbottomrule
\end{tabular}
\end{table}

The resemblance between the  2\fl{}1 portmanteau prefix, the generic person prefixes, and the nominalization prefixes in Japhug and Tshobdun is striking. However, since in these two languages velar participles (§\ref{sec:subject.participles}, §\ref{sec:object.participle}) and infinitives (§\ref{sec:velar.inf}) are compatible with neither first person indexation suffixes nor the inverse prefix, it is not possible to propose a direct path of reanalysis from participial or infinitive forms to  2\fl{}1 (which co-occur with first person suffixes, §\ref{sec:indexation.local}) or generic (which appear with inverse in Tshobdun) prefixes.

However, in Situ, unlike other Gyalrong languages, semi-finite nominalized forms combining a prefix \forme{kə-} and person indexation suffixes (\ref{ex:kESiN}) and/or inverse marking (\ref{ex:nokombEng}, in 3\fl{}\textsc{1sg} configuration) are also attested, in particular to build certain types of relative clauses \citep{jacksonlin07}.
 
\begin{exe}
\ex \label{ex:kESiN}
 \gll [ŋa kə-ʃî-ŋ]=tə teɲê tsə̂s-ŋ \\
\textsc{1sg} \textsc{sbj}:\textsc{pcp}-know[I]-\textsc{1sg}=\textsc{top} \textsc{dem}:\textsc{pl} say[I]-\textsc{1sg} \\
\glt `I will say what I know about.' (Cogtse dialect, \citealt[72]{linyj16cogtse})
\end{exe}

\begin{exe}
\ex \label{ex:nokombEng}
 \gll   [sanam-scə̄t kə no-kə-o-mbə̂-ŋ] tə, <zengguangxianwen> u-tʰiɛ̂ ˈnə-ŋɐs \\
Bsod.nams-Skyid \textsc{erg} \textsc{aor}.\textsc{inv}-\textsc{nmlz}-\textsc{inv}-give-\textsc{1sg} \textsc{det} name \textsc{3sg}.\textsc{poss}-book \textsc{sens}-be[I] \\
\glt `What Bsod nams Skyid gave me is the book `Zengguan xianwen.'' (Bragbar dialect, \citealt{zhangshuya20these})
\end{exe}

%\begin{exe}
%\ex \label{ex:kEwsEsot}
% \gll [tʰe-te kə-w-səsôt]  to-ka-tsis nə-ŋos\\
%what-\textsc{particle} \textsc{nmlz}-\textsc{3pl}:\textsc{tr}-do.action[I] \textsc{aor}-\textsc{nmlz}:\textsc{pl}-say[II] \textsc{sens}-be[I]\\
%\glt `They said:  `What will they do ?'' (\citealt[240]{linyj16cogtse}) %会是什么样子呢?”(他们)说
%\end{exe}

If one accepts the hypothesis that semi-nominalized verb forms did already exist in proto-Gyalrong, and could be used to mark \textit{indefinite} core arguments in addition to the functions they have in modern Situ, it becomes possible to account for the forms in Tables \ref{tab:local.rgy} and \ref{tab:genr.japhug.tshobdun}. 

A nominalized form combining the prefix \forme{kə\trt}, the inverse and a first person suffix like \forme{no-kə-o-mbə̂-ŋ} in example (\ref{ex:nokombEng}), originally meaning `someone gave it to me', could have become a less abrupt way of saying `you gave it to me' and have been progressively reanalyzed as a  2\fl{}1 prefix (\citealt{jacques18generic, delancey18sociopragmatic}). The absence of inverse in Japhug in the 2\fl{}1 form may be explained by supposing that in the ancestor of Japhug, semi-finite forms became incompatible with inverse marking, while still taking indexation suffixes. This idea is not outlandish, as semi-finite inverse forms are very rare in Situ (only one example is found in Lin's \citeyear{linyj16cogtse} text corpus).

The velar nominalization prefixes in semi-finite indefinite forms could also have become reanalyzed as generic person prefixes. This hypothesis implies that Tshobdun is more conservative than Japhug, and that Japhug has later innovated by replacing generic transitive subjects by inverse forms,  perhaps to avoid confusion between generic subjects and object caused by the constraint against inverse in semi-finite forms already posited above to explain the 2\fl{}1 portmanteau \forme{kɯ-}. The idea that the Japhug pattern is innovative is supported by the existence of irregular verbs whose generic transitive subjects are marked by \forme{kɯ-} prefix rather than the inverse (§\ref{sec:irregular.transitive}); these verbs would be the remnants of the stage attested in Tshobdun.
 
 
