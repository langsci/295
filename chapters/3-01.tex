\chapter{Nominal morphology} \label{chap:nominal.morphology}

This chapter focuses on possessive prefixes, compounding and noun derivations. 

It does not discuss flagging, number and quantification, as these grammatical categories are expressed by independent words or clitics, such as postpositions and noun modifiers, and are treated in chapters \ref{chap:postpositions.relators} and \ref{chap:noun.phrase}.

Nominalization (including lexicalized deverbal nouns) and denominal verbalization are treated in chapters \ref{chap:non-finite} and \ref{chap:denominal}, respectively. 

The morphology of counted nouns (quantifiers, time nominals) is discussed in §\ref{sec:counted.nouns}, and that of nouns of location in §\ref{sec:relator.location}.

\section{Possessive prefixes} \label{sec:possessive.prefixes}
Nouns in Japhug can be divided into four main subclasses, \textit{inalienably possessed} nouns, \textit{alienably possessed} nouns, \textit{unpossessible} nouns and \textit{counted} nouns, depending on the type of prefixes they can take. The present section focuses on the first two, the ones that are compatible with possessive prefixes. Unpossessible nouns are treated in §\ref{sec:unpossessible.nouns}, and counted nouns in §\ref{sec:counted.nouns} in another chapter.

\subsection{Possessive paradigm} \label{sec:possessive.paradigm}
The paradigm of possessive prefixes in Japhug is shown in \tabref{tab:possessive.prefixes}. It presents obvious commonalities with the personal pronouns (§\ref{sec:pers.pronouns}) and the indexation suffixes (§\ref{sec:indexation.suffixes.history}). \tabref{tab:possessive.prefixes} includes comparative data on the dialect of Tatshi (from \citealt{linluo03}, \citealt{lin11direction} and personal communication), which differs from Kamnyu in having non-palatalized forms in the dual prefixes. 

In this paradigm, the contrast between second and third person is neutralized in the dual and plural, while it is preserved in pronouns and person indexation.


\begin{table}[h] 
\caption{Possessive prefixes in Japhug }\label{tab:possessive.prefixes}
\begin{tabular}{llll} \lsptoprule
Person & Kamnyu dialect & Tatshi dialect & \\
\midrule
1\sg{}  &\forme{a-}  &	\forme{a-}	\\
2\sg{} &\forme{nɤ-}  &		\forme{na-}	 \\
3\sg{}& \forme{ɯ-}  &\forme{ə-}	 \\
\midrule
1\du{} &\forme{tɕi-}   & 	\forme{tsə-}	  \\
2/3\du{}&\forme{ndʑi-}  &	\forme{ndzə-}	 \\	
\midrule
1\pl{} & \forme{ji-}  &	\forme{ji-}		 \\
2/3\pl{}&\forme{nɯ-}  &		\forme{nə-}	 \\
\midrule
indefinite&\forme{tɯ-/tɤ-/ta-}  &	\forme{tə-/ta-}	  \\
generic&\forme{tɯ-}  &		\forme{tə-}	  \\
\lspbottomrule
\end{tabular}
\end{table}

The \textit{indefinite possessor} prefixes \forme{tɯ-/tɤ-/ta-} only occur on inalienably possessed nouns (§\ref{sec:inalienably.possessed.morpho}). The syntactic function of these prefixes corresponds to that of the ``absolutive'' suffix in Nahuatl (``suffixe absolu'' in \citealt[207]{launey94}). The \textit{indefinite possessor} prefixes saturate the possessive prefixal slot, just like antipassive prefixes (§\ref{sec:antipassive}) saturate the object function of transitive verbs. They do \textit{not} mark that the noun is indefinite,\footnote{On the various strategies to express definiteness in Japhug, see §\ref{sec:indefinite.markers} and §\ref{sec:definiteness}. } but rather indicate that it lacks a definite possessor, when the possessor is unknown or irrelevant.\footnote{A reviewer suggests the term ``neutral'' instead, but I find it not specific enough, and ``absolutive'' is to be avoided for obvious reasons, in a language with numerous constructions following ergative-absolutive alignment (§\ref{sec:ergativity}).  }

Possessive prefixes other than the indefinite possessor prefixes are collectively referred to as ``definite possessor prefixes''. The semantic difference between indefinite and generic possessors is discussed in §\ref{sec:indef.genr.poss}.

The vowel contrast  \ipa{ɯ} vs. \ipa{ɤ} vs. \ipa{a} on the indefinite possessor prefixes is lexically determined (§\ref{sec:inalienably.possessed.morpho}). This contrast is neutralized on definite possessor prefixes, which all have the same form regardless of the noun. This neutralization is an innovation: in Situ, the vowel contrast is present on all possessive prefixes \citep[168--169]{linxr93jiarong},\footnote{\citet[118--119]{prins16kyomkyo} analyzes the vowel as part of the nominal root.} 

Stacking of possessive prefixes is not allowed in Japhug, with the exception of the combination of a definite possessor prefix with an indefinite possessor prefix \forme{tɯ-} or \forme{tɤ-} to turn an inalienably possessed noun into an alienably possessed one (see §\ref{sec:alienabilization}).

The only irregularities in possessive morphology are found with \forme{a-} initial nouns (§\ref{sec:a.nouns}).

\subsubsection{\forme{a-} initial nouns}  \label{sec:a.nouns}

Unlike verbs (see §\ref{sec:contraction}), nouns whose stem begins with a \forme{a-} are extremely rare in Japhug. Nevertheless, as is the case with verbs, the vowel \forme{a-} merges with any prefixed element, so that nouns of this type do not have regular possessive forms. The only noun in \forme{a-} to commonly receive possessive prefixes is \japhug{araʁ}{liquor} (a loanword from \tibet{ཨ་རག་}{ʔa.rag}{liquor}). Its possessive forms are highly anomalous: \textsc{1sg} \forme{aʑɤ-raʁ}, \textsc{2sg} \forme{nɤʑɤ-nɤ-raʁ} and \textsc{2pl} \forme{nɯʑɤ-nɯ-raʁ} (as in \ref{ex:nWZAnWraR}), combining the pronoun in \textit{bound state} (§\ref{sec:status.constructus}) followed by the possessive prefix, which takes over the initial \forme{a-}.

\begin{exe}
\ex \label{ex:nWZAnWraR}
\gll nɯʑɤ-nɯ-raʁ ɯ́-ra \\
\textsc{2pl}-\textsc{2pl}.\textsc{poss}-liquor \textsc{qu}-be.needed:\textsc{fact} \\
\glt `Do you need liquor?' (elicited)
\end{exe}

To account for these forms, it is necessary to assume that the initial \forme{a-} was reanalyzed as a \textsc{1sg} possessive prefix. However, this analysis did not occur directly. The expected \textsc{1sg} possessive form \forme{*a-araʁ} (\textsc{1sg}.\textsc{poss}-liquor) would automatically yield \forme{*a-raʁ} by vowel fusion (following the rule described in §\ref{sec:contraction}). The resulting \textsc{1sg} \forme{*a-raʁ} then became the pivot for the analogical reshaping of the rest of the paradigm, from which for instance \forme{*nɤ-raʁ}  (\textsc{2sg}.\textsc{poss}-liquor) was generated (replacing putative earlier forms such as \forme{*nɤ-araʁ} *\ipa{naraʁ}). Due to homophony with the base noun \forme{araʁ}, the \textsc{1sg} pronoun \forme{aʑo} was systematically added and \forme{*aʑo a-raʁ} underwent morphological fusion to the attested \forme{aʑɤ-raʁ}, followed by the rest of the possessive paradigm.
 

\subsubsection{The expression of possession} \label{ex:prefix.expression.of.possession}
Possession cannot be expressed without a possessive prefix on the possessee, except for a handful of constructions where a pronoun occurs instead (§\ref{sec:pronouns.possessive.markers}).

Possessive prefixes can be used on nearly any noun (except the unpossessed nouns, see §\ref{sec:unpossessible.nouns}), including recent borrowings from Chinese (or quasi-code switching), as \zh{老家} \forme{lǎojiā} `native place' in (\ref{ex:aZo.GW.alaojia}). They also occur on several non-finite verbal forms, including participles (see §\ref{sec:subject.participle.possessive} and §\ref{sec:object.participle.possessive}), bare infinitives (§\ref{sec:bare.inf}) and degree nominals (§\ref{sec:degree.nominals}).
\largerpage
\begin{exe}
\ex \label{ex:aZo.GW.alaojia}
\gll aʑo ɣɯ a-<laojia> ɣɯ ɯ-lɤcu nɯre ri ku-rɤʑi-nɯ ŋu \\
\textsc{1sg} \textsc{gen} \textsc{1sg}.\textsc{poss}-old.house \textsc{gen} \textsc{3sg}.\textsc{poss}-upstream there \textsc{loc} \textsc{ipfv}-stay-\textsc{pl} be:\textsc{fact} \\
\glt `They live in a place upstream from my old house.' (14-siblings) \japhdoi{0003508\#S215}
\end{exe}

In the case of first or second person possessors, it is possible to have simply a possessive prefix on the noun, (\forme{a-ɣɲi} `my friend', \forme{a-mbro} `my horse' and \forme{a-ʁgra} `my enemy' in \ref{ex:ambro}), a personal pronoun and a possessive prefix (same person and number, as in \ref{ex:aZo.ambro}) or even a pronoun, the genitive clitic \forme{ɣɯ} and a possessive prefix as in (\ref{ex:aZo.GW.alaojia}) (§\ref{sec:gen.possession}).

 \begin{exe}
\ex \label{ex:ambro} 
\gll a-ɣɲi ci tɯ\redp{}tɯ-ŋu nɤ, a-mbro ɯ-lwa ɯ-taʁ kɤ-zo, a-ʁgra ci tɯ\redp{}tɯ-ŋu nɤ, a-mbro ɯ-jme ɯ-taʁ kɤ-zo \\
\textsc{1sg}.\textsc{poss}-friend \textsc{indef} \textsc{cond}\redp{}2-be:\textsc{fact} \textsc{lnk} \textsc{1sg}.\textsc{poss}-horse \textsc{3sg}.\textsc{poss}-mane \textsc{3sg}-on \textsc{imp}-land \textsc{1sg}.\textsc{poss}-enemy \textsc{indef} \textsc{cond}\redp{}2-be:\textsc{fact} \textsc{lnk} \textsc{1sg}.\textsc{poss}-horse \textsc{3sg}.\textsc{poss}-tail \textsc{3sg}-on \textsc{imp}-land  \\
\glt `If you are my friend, land on my horse's mane, if you are my enemy, land on my horse's tail.' (2002 qaCpa)
\end{exe}

\begin{exe}
\ex \label{ex:aZo.ambro}
\gll aʑo a-mbro nɤrwɯrɯnbotɕʰi ŋu, tɯ-sŋi χpaχtsʰɤt ci ɲɯ́-wɣ-tsɯm-a cʰa \\
\textsc{1sg} \textsc{1sg}.\textsc{poss}-horse \textsc{anthr} be:\textsc{fact} one-day yojana \textsc{indef} \textsc{ipfv}:\textsc{west}-\textsc{inv}-take.away-\textsc{1sg} can:\textsc{fact} \\
\glt `My horse is Norbu Rinpoche, he can make me cross one yojana per day.' (2003smanmi2)
\end{exe}

It is possible to have a first singular possessive preceded by a first plural pronoun, as in (\ref{ex:iZo.amu}) (see §\ref{sec:agreement.mismatch} for other examples of person mismatch involving \textsc{1pl} pronouns).

\begin{exe}
\ex \label{ex:iZo.amu}
\gll iʑo a-mu nɯ tʰamtʰam kɯrcɤsqaptɯɣ tʰɯ-azɣɯt ŋu. \\
\textsc{1pl} \textsc{1sg}.\textsc{poss}-mother \textsc{dem} now 81 \textsc{aor}-reach be:\textsc{fact} \\
\glt `My mother is now 81.' (2010-histoire09-2)
\end{exe}

In the case of a possessee shared by the speaker and the addressee, the \textsc{1sg} possessive is the preferred form. For instance, a couple of parents or grandparents talking to each other about their son or their grandchild more often use \forme{a-tɕɯ} `my son' or \forme{a-ɣe} `my grandchild' than a \textsc{1du} or a \textsc{2sg} possessor such as \forme{tɕi-ɣe} `our grandchild' or \forme{nɤ-ɣe} `your grandchild', as I have noticed by participant observation. Nevertheless, the use of other possessive prefixes than \textsc{1sg} in such contexts is not agrammatical, and systematically occurs in texts translated from Chinese (by calquing), as in (\ref{ex:tCitCW.YWsAzdWxpa}). 

\begin{exe}
\ex \label{ex:tCitCW.YWsAzdWxpa}
\gll tɕi-tɕɯ ɲɯ-sɤzdɯxpa,  \\
\textsc{1du}.\textsc{poss}-son \textsc{sens}-be.pitiful \\
\glt `Our poor son!' (150831 renshen wawa-zh) \japhdoi{0006418\#S15}
\end{exe}


\subsubsection{Definiteness and obviation} \label{sec:possessive.prefix.obv.def}
Nouns with a definite possessor in Japhug can be indefinite, unlike in most languages of Europe. They can occur with an indefinite determiner (example \ref{ex:ambro}  above). With a quantifier such as \japhug{tɯ-rdoʁ}{one piece} as in (\ref{ex:Wzda.tWrdoR}), a noun with a definite possessor is interpreted as referring to a certain number of persons out of a group (`one of his X').

 \begin{exe}
\ex \label{ex:Wzda.tWrdoR}
\gll tɤ-tɕɯ nɯ kɯ ɯ-zda tɯ-rdoʁ ɯ-pʰe to-ti, tɯ-rdoʁ nɯ kɯ li ci ɯ-pʰe tɕe ɲɤ-k-ɤ-sɯ-ɤmɯ-mtsʰɯ\redp{}mtsʰɤm-nɯ \\
\textsc{indef}.\textsc{poss}-son \textsc{dem} \textsc{erg} \textsc{3sg}.\textsc{poss}-companion one-\textsc{cl} \textsc{3sg}-\textsc{dat} \textsc{ifr}-say one-\textsc{cl}  \textsc{dem} \textsc{erg} again \textsc{indef} \textsc{3sg}-\textsc{dat} \textsc{lnk}   \textsc{ifr}-\textsc{peg}-\textsc{recip}-\textsc{caus}-\textsc{recip}-hear-\textsc{pl} \\
\glt `The boy told one of his companions, and that one another one, and [in this way] they informed each other.' (2012 Norbzang)
\japhdoi{0003768\#S70}
\end{exe}

Unlike in Algonquian languages, but like in Mapudungun \citep{haude16symmetrical}, inverse marking on the verb (§\ref{sec:indexation.non.local}) is \textit{not} required if the subject is a possessed noun whose possessor is also object of the same sentence, as shown by example (\ref{ex:prox.naBde}) where the direct form \forme{na-βde} appears. In other words, nouns with a third person possessor are not automatically \textit{obviative} (see \citealt{jacques10inverse} and §\ref{sec:obviation.possessor} for additional discussion).

\begin{exe}
\ex \label{ex:prox.naBde}
\gll ɯ-rʑaβ nɯ kɯ na-βde \\
\textsc{3sg}.\textsc{poss}-wife \textsc{dem} \textsc{erg} \textsc{aor}:3\flobv{}-throw.away \\
\glt `His$_i$ wife left him$_i$.' (14-siblings)
\end{exe}

The inverse form \forme{nɯ́-wɣ-βde} is also possible in exactly the same context -- example  (\ref{ex:obv.nWwGBde}) comes from the same text and refers to the same event. 

\begin{exe}
\ex \label{ex:obv.nWwGBde}
\gll ɯ-rʑaβ cʰo ɯ-tɕɯ nɯ ʁnaʁna kɯ nɯ́-wɣ-βde\\
\textsc{3sg}.\textsc{poss}-wife \textsc{comit} \textsc{3sg}.\textsc{poss}-son \textsc{dem} both \textsc{erg} \textsc{aor}-\textsc{inv}-throw.away\\
\glt `His$_i$ wife and his$_i$ son left him$_i$.' (14-siblings)
\japhdoi{0003508\#S266}
\end{exe}

Although not obligatory, the inverse on the verb (like in example \ref{ex:obv.nWwGBde}) is more common than a direct form (example \ref{ex:prox.naBde}) in this type of configuration (§\ref{sec:indexation.non.local}).

\subsubsection{Other uses of possessive prefixes} \label{sec:other.uses.poss.prefixes}
Possessive prefixes are also used to express beneficiaries, recipients and other oblique arguments, such as the `person needing' in the construction with the verb \japhug{ra}{be needed},  as in (\ref{ex:ambro.tARndo.kWtso}).\footnote{See §\ref{sec:gen.beneficiary} for a more detailed account of the expression of beneficiaries in Japhug.}

 \begin{exe}
\ex \label{ex:ambro.tARndo.kWtso}
\gll a-mbro taʁndo kɯ-tso ci tɕi ra \\
\textsc{1sg}.\textsc{poss}-horse speech \textsc{sbj}:\textsc{pcp}-understand one also be.needed:\textsc{fact} \\
\glt `I also need a horse who understands speech.' (2003kAndzwsqhaj2)
\end{exe}

In the case of beneficiaries and recipients, if a genitive pronoun or genitive phrase is present, the presence of a possessive prefix is possible (\ref{ex:aZWG.akWra}) but not obligatory (\ref{ex:aZWG.kWra}), in particular in the case of possessed nouns that already have a definite possessor (\ref{ex:aZWG.Wlu}).

 \begin{exe}
\ex \label{ex:aZWG.akWra}
\gll aʑɯɣ a-kɯ-ra ci tu tɕe nɯ `ɣa' tɤ-ti ra \\
\textsc{1sg}:\textsc{gen} \textsc{1sg}.\textsc{poss}-\textsc{sbj}:\textsc{pcp}-be.needed \textsc{indef} exist:\textsc{fact} \textsc{lnk} \textsc{dem} yes \textsc{imp}-say be.needed:\textsc{fact} \\
\glt `There is one thing I need, and you have to say `yes' to it.' (140429 qingwa wangzi-zh)
\japhdoi{0003890\#S47}
\end{exe}

 \begin{exe}
\ex \label{ex:aZWG.kWra}
\gll  aʑɯɣ kɯ-ra me \\
\textsc{1sg}:\textsc{gen} \textsc{sbj}:\textsc{pcp}-be.needed not.exist:\textsc{fact} \\
\glt `I don't need anything.' (2005 Norbzang)
\end{exe}

 \begin{exe}
\ex \label{ex:aZWG.Wlu}
\gll aʑɯɣ ɯ-lu ra \\
\textsc{1sg}:\textsc{gen} \textsc{3sg}.\textsc{poss}-milk be.needed:\textsc{fact} \\
\glt `I want its milk.' (02-deluge2012) \japhdoi{0003376\#S12}
\end{exe}

We also find \textsc{3sg} possessive prefixes \forme{ɯ-} indexing not a possessor or a beneficiary/recipient, but anaphorically referring to a whole clause, as in (\ref{ex:Wcha.mWm}), where \forme{ɯ-cʰa} does not mean `its/his alcohol', but `the alcohol made in the fashion described in the previous clause'.\footnote{As pointed out by a reviewer, this type of construction reminds of the diachronic evolution from third person possessive suffix to definite article in Amharic \citep{rubin10amharic}. }

 \begin{exe}
\ex \label{ex:Wcha.mWm}
\gll  kɯɕɯŋgɯ tɕe iɕqʰa ʑmbrɯβɟaj nɯ kɯ nɯnɯ cʰa nɯ tú-wɣ-sɯ-ɕmi tɕe 
ɯ-cʰa mɯm tu-ti-nɯ pɯ-ŋgrɤl \\
in.former.times \textsc{lnk} the.aforementioned boat.oar \textsc{dem} \textsc{erg} \textsc{dem} alcohol \textsc{dem} \textsc{ipfv}-\textsc{inv}-\textsc{caus}-mix \textsc{lnk} \textsc{3sg}.\textsc{poss}-alcohol be.tasty:\textsc{fact} \textsc{ipfv}-say-\textsc{pl} \textsc{pst}.\textsc{ipfv}-be.usually.the.case \\
\glt `In former times, people used to mix the alcohol with boat oars, the alcohol [made this way] is tasty, they used to say.' (cha-31)
\japhdoi{0003764\#S38}
\end{exe}


\subsubsection{The form of the \textsc{3sg} possessive prefix} \label{sec:3sg.possessive.form}
Japhug differs from other Gyalrong languages (\tabref{tab:3sg.inv}, data from \citealt{jackson02rentongdengdi}, \citealt{gongxun14agreement}) in that the third person possessive prefix is \textit{not} homophonous with the inverse prefix  (§\ref{sec:allomorphy.inv}, §\ref{sec:direct-inverse}).

\begin{table}
\caption{The form of the \textsc{3sg} possessive prefix in Gyalrong languages} \label{tab:3sg.inv} 
\begin{tabular}{lllll}
\lsptoprule
& \textsc{3sg}.\textsc{poss} & inverse \\
\midrule
Japhug &  \forme{ɯ-} & \forme{ɣɯ-}/\forme{-wɣ-} \\
Tshobdun &  \forme{o-} & \forme{o-}  \\
Zbu &   \forme{wə-} & \forme{wə-} \\
Situ &    \forme{və-} & \forme{və-} \\
\lspbottomrule
\end{tabular}
\end{table}

Independently of the question of whether these two prefixes could be historically related \citep{sanso14inverse}, it is probable that Japhug is innovative here.

In the same way as the inverse prefix \forme{ɣɯ-} has an allomorph transcribed as \forme{\trt{}wɣ\trt{}} when preceded by another prefix, realized as vowel rounding in most cases (§\ref{sec:allomorphy.inv}), there is a possible trace of a vowel rounding allomorph of the possessive prefix in the linker \japhug{núndʐa}{for this reason} (§\ref{sec:consequence}). 

This linker originates from a phrase combining the demonstrative \japhug{nɯ}{that} (on which see §\ref{sec:anaphoric.demonstrative.pro}, §\ref{sec:demonstrative.determiners} and §\ref{sec:nW.topic}) with the \textsc{3sg} possessed form of the noun \japhug{ɯ-ndʐa}{cause} (§\ref{sec:causal.clauses}). The form \forme{núndʐa} possibly reflects earlier \forme{*nɯ-w-ndʐa}, \forme{*-w-} being a frozen allomorph of the \textsc{3sg} possessive prefix in non-initial position.

There is a possible trace of the expected allomorph $\dagger$\forme{ɣɯ-} (from proto-Gyalrong \forme{*wə-}) in the noun \japhug{ɣɯfsu}{friend}, etymologically `his equal'; the inalienably possessed noun \japhug{ɯ-fsu}{of the same size}, which shares the same root, has a regular possessive prefix that is coreferent with the standard of comparison (as in \ref{ex:nWfsu} with the \textsc{3pl}; see §\ref{sec:Wfsu.equative} on this construction). The alienably possessed noun \japhug{ɣɯfsu}{friend} is thus possibly a lexicalized equivalent of \japhug{ɯ-fsu}{of the same size} (used in one of the equative constructions, §\ref{sec:Wfsu.equative}), whose \textsc{3sg} prefix was frozen before the change from \forme{*ɣɯ-} to \forme{ɯ-} occurred.

\begin{exe}
\ex \label{ex:nWfsu}
\gll tɯrme kɯ-mbro ra nɯ-fsu jamar tu-zɣɯt ma mɤ-cʰa \\
person \textsc{sbj}:\textsc{pcp}-be.high \textsc{pl}  \textsc{3pl}.\textsc{poss}-equal about \textsc{ipfv}-reach apart.from \textsc{neg}-can:\textsc{fact} \\
\glt `It can only grow about as high as a tall human.' (15-babW)
\japhdoi{0003512\#S4}
\end{exe}

Furthermore, additional evidence for the idea that the third person prefix contained \forme{*w-} comes from the etymology of the reflexive prefix \forme{ʑɣɤ-}, which is argued to originate from the third person pronoun (§\ref{sec:reflexive.origin}).

It remains unclear why the regular allomorph of the third person possessive is \forme{ɯ-} rather than expected $\dagger$\forme{ɣɯ-}. A possible explanation could be false segmentation, due to reanalysis with the genitive marker \forme{ɣɯ}, since the genitive can optionally occur between the possessor and the possessee, as in (\ref{ex:qaCpa.GW.WpW}). A pre-Japhug form such as \forme{*qaɕpa ɣɯ-pɯ} could have been misanalyzed as \forme{qaɕpa ɣɯ ɯ-pɯ} due to vowel fusion sandhi (§\ref{sec:sandhi.word}), and a new allomorph \forme{ɯ-} extracted from such constructions.\footnote{The weakness of this hypothesis is that some Japhug dialects have \forme{kɯ} rather than \forme{ɣɯ} as their genitive marker.}

\begin{exe}
\ex \label{ex:qaCpa.GW.WpW}
 \gll nɯnɯ qaɕpa ɣɯ ɯ-pɯ ŋu tɕe, \\
 \textsc{dem} frog \textsc{gen} \textsc{3sg}.\textsc{poss}-young be:\textsc{fact} \textsc{lnk} \\
 \glt `It (the tadpole) is the young of the frog.' (hist-28-kWpAz)
\japhdoi{0003714\#S203}
\end{exe} 


\subsection{Inalienably possessed nouns} \label{sec:inalienably.possessed}

\subsubsection{Morphology} \label{sec:inalienably.possessed.morpho}
Inalienably possessed nouns differ from alienably possessed ones in that they require the presence of a possessive prefix.  Unless when used with the indefinite possessor prefixes, inalienably possessed nouns are not formally distinguishable from alienably possessed ones; for instance, \forme{a-pi} `my elder sibling' and \forme{a-mbro} `my horse' both take the 1\sg{} \forme{a-} prefix and no direct clue indicates that the first noun is inalienably possessed and that the second one is alienably possessed.

The citation form however differs between inalienably and alienably possessed nouns: the former must take an indefinite possessor prefix (or in some cases a 3\sg{} \forme{ɯ-}), while the latter can occur without possessive prefix, as for instance \japhug{tɤ-pi}{elder sibling} (with the indefinite \forme{tɤ-}; the bare root $\dagger$\forme{pi} is not a correct form) vs. \japhug{mbro}{horse} (without prefix).

Inalienably possessed nouns are divided into four classes depending on their citation form. The indefinite possessor prefix has three allomorphs (\forme{tɯ\trt}, \forme{tɤ\trt}, \forme{ta-}) whose distribution is not completely predictable on the basis of phonology or semantics (though some generalizations are provided below). In addition, some inalienably possessed nouns only take definite possessor prefixes. The contrast between these four classes is neutralized when the noun takes a definite possessor prefix (unlike in Situ, see \citealt[168--169]{linxr93jiarong} and \citealt[118--119]{prins16kyomkyo}).

The most common allomorph of the indefinite possessor prefix is \forme{tɯ-}. Inalienably possessed nouns selecting this allomorph, such as \japhug{tɯ-jaʁ}{hand, arm}, have identical indefinite and generic possessor forms (see §\ref{sec:indef.genr.poss}).

The allomorph \forme{tɤ-} is also very common, in particular with kinship terms and some body parts (see §\ref{sec:body.part} and §\ref{sec:kinship}). The form \forme{ta-} is a phonological variant of \forme{tɤ\trt}, occurring mainly with nouns whose stem begins with a uvular such as \japhug{ta-ʁrɯ}{horn} or \japhug{ta-ʁi}{younger sibling}. The contrast between \ipa{ɤ} and \ipa{a} in this prefix is very difficult to perceive before uvulars with some speakers (see §\ref{sec:A.vs.a.prefixes}), and the transcription adopted in this grammar (and the online corpus and dictionary) is based on the slow syllable-by-syllable pronunciation of these words by Tshendzin. Two inalienably possessed nouns, however, \japhug{ta-ma}{work} and \japhug{ta-mar}{butter}, have the \forme{ta-} allomorph with an initial \forme{m\trt}, probably originally due to vowel assimilation (§\ref{sec:vowel.harmony}, §\ref{sec:A.vs.a.prefixes}).

The minimal pair between \japhug{tɤ-ma}{mother} and \japhug{ta-ma}{work} shows that this vowel contrast, however marginal, is distinctive, and that even if the two allomorphs \forme{tɤ-} and \forme{ta-} were originally phonologically conditioned, it is no longer the case in Kamnyu Japhug.

Some inalienably possessed nouns never occur with indefinite possessor prefixes, for instance \japhug{ɯ-tʰoʁ}{ground} is only attested with the 3\sg{} \forme{ɯ-} prefix (see §\ref{sec:earth.IPN}). In some cases, the indefinite possessor form is difficult to elicit and in case of doubt the third singular form is given in the dictionary \citet{jacques16japhug} (for instance \japhug{ɯ-mdoʁ}{colour}). Future reasearch may reveal an indefinite possessor form for some of these nouns.

When denominal verbs are derived from inalienably possessed nouns, the vocalism of the denominal prefix tends to be the same as that of the indefinite possessor prefix (for instance \japhug{tɤ-βɟu}{mattress} \fl{} \japhug{nɤβɟu}{use as a mattress}, not $\dagger$\forme{nɯβɟu}), though there are exceptions (\japhug{tɯ-rpaʁ}{shoulder} \fl{} \japhug{mɤrpaʁ}{carry on the shoulder}), as discussed in §\ref{sec:denom.mA}.

By analogy with several non-finite verb forms, in particular the subject participle of transitive verbs and the bare infinitive, which index one argument (the object) by a possessive prefix (§\ref{sec:bare.inf}), the possessors of inalienably possessed nouns are considered to be \textit{core arguments}, while those of alienably possessed nouns are treated as \textit{adjuncts}. In other words, inalienably possessed nouns have a valency of 1 like intransitive verbs, while alienably possessed nouns have a valency of 0.\footnote{On nominal valency and inalienability, see also \citet[30-31]{gutman18attributive}. } The indefinite possessor prefix can be viewed as a valency-decreasing device, the nominal equivalent of passive and antipassive derivations, especially given its use in the alienabilization of inalienably possessed nouns (see §\ref{sec:alienabilization}). Wider implications of the assumption that possessors of inalienably possessed nouns are core arguments are explored in §\ref{sec:complement.taking.nouns}.

\subsubsection{Inalienabilization} \label{sec:apn.to.ipn}
Derivation from alienably possessed to inalienably possessed nouns is not common in Japhug. An interesting case is that of \japhug{ɯ-ʁle}{reputation}, which originates from the alienably possessed \japhug{qale}{wind} with a reduced form \forme{ʁ-} of the class prefix \forme{qa\trt}, as some second members of compounds (see §\ref{sec:second.member.alternation} and §\ref{sec:class.prefixes}).

Conversion of counted nouns (§\ref{sec:counted.nouns}) to inalienably possessed nouns is a regular process (§\ref{sec:CN.IPN}).

\subsubsection{Body parts} \label{sec:body.part}
The great majority of body parts are inalienably possessed nouns with the indefinite possessor \forme{tɯ-}. These include native words, but also borrowings from Tibetan such as \japhug{tɯ-qʰoχpa}{organs, state of mind} from Tibetan \tibet{ཁོག་པ་}{kʰog.pa}{innards} (see §\ref{sec:uvular.harmony} on the phonology of this word).

Among body parts, inalienably possessed nouns selecting the prefix \forme{tɤ-} are mainly liquids from the body such as \japhug{tɤ-se}{blood}, \japhug{tɤ-spɯ}{pus} and \japhug{tɤ-lu}{milk} (though some liquids also take the prefix \forme{tɯ\trt}, for instance \japhug{tɯ-ɕtʂi}{sweat}), hair (\japhug{tɤ-rme}{hair}, \japhug{tɤ-kɤrme}{head hair} and some animal body parts (\japhug{tɤ-jme}{tail}, \japhug{tɤ-ŋkɯ}{pig skin}, \japhug{tɤ-rkʰom}{feather rachis}).

Parts of plants on the other hand mainly have the prefix \forme{tɤ\trt}, as \japhug{tɤ-jwaʁ}{leaf}, \japhug{tɤ-tsrɯ}{sprout}, \japhug{tɤ-zrɤm}{root} etc.

Alienably possessed nouns are rare among body parts. Some nouns with the \forme{qa-} class prefix (see §\ref{sec:class.prefixes}) such as \japhug{qame}{mole} and \japhug{qambɣo}{earwax} referring to physical defects or excretions from the body are alienably possessed nouns. A similar situation is observed in Koyukon Athabaskan, where nouns `denoting certain temporary or abnormal parts of the body' are also alienably possessed noun (\citealt[660]{thompson96koyukon}), though in Koyukon this subclass is considerably larger than in Japhug.

The compound \japhug{tɯciste}{amniotic sac} from \japhug{tɯ-ci}{water} and \japhug{tɤ-ste}{bladder} has a \forme{tɯ-} which is originally an indefinite possessor prefix (see §\ref{sec:earth.IPN}), but which has become frozen after being integrated into a compound (§\ref{sec:frozen.indef}), as can be shown by (\ref{ex:WtWciste}). 

\begin{exe}
\ex \label{ex:WtWciste}
\gll ɯ-tɯciste cʰɤ-ndʑɣaʁ \\
\textsc{3sg}.\textsc{poss}-amniotic.sac \textsc{ifr}-\textsc{acaus}:squeeze.out \\
\glt `Her waters have broken.' (elicited)
\end{exe}


\subsubsection{Kinship terms} \label{sec:kinship}
The great majority of kinship terms select the indefinite possessor prefix \forme{tɤ-} or \forme{ta-} (see chapter \ref{chap:kinship} for a description of the kinship system). The only kinship terms in \forme{tɯ-} are \japhug{tɯ-me}{daughter} (but this form is not attested in the text corpus) and \japhug{tɯlɤt}{second sibling}; however, the \forme{tɯ-} prefix in the latter word has become non-analyzable and this word has become an unpossessible noun (see §\ref{sec:unpossessible.nouns}). 

There are other unpossessible nouns among kinship terms, including \japhug{woɬaʁ}{(bad) stepmother}, which derives from \japhug{tɤ-ɬaʁ}{mother's sister} by replacing the possessive prefix with an unidentified element \forme{wo\trt}, and the social relation collectives (§\ref{sec:social.collective}). Being a unpossessible noun, \japhug{woɬaʁ}{(bad) stepmother} cannot take possessive prefixes, and the forms of \japhug{tɤ-ɬaʁ}{mother's sister} are used instead (\forme{a-ɬaʁ} can mean `my (bad) stepmother').

Kinship terms do not commonly occur with the indefinite possessor prefix. For those denoting spouses, forms with the indefinite prefix are found in the expression `look for a wife/husband', as in (\ref{ex:tArZaB.WkWCar}).
 
\begin{exe}
\ex \label{ex:tArZaB.WkWCar}
 \gll `ŋoj tɯ-ɕe?' to-ti, `aʑo tɤ-rʑaβ ɯ-kɯ-ɕar ɕe-a' to-ti. tɕe `ndʑiʑo ŋoj tɯ-ɕe-ndʑi?' to-ti ri, `tɕiʑo tɤ-nmaʁ ɯ-kɯ-ɕar ɕe-tɕi' to-ti. \\
 where 2-go:\textsc{fact} \textsc{ifr}-say \textsc{1sg} \textsc{indef}.\textsc{poss}-wife \textsc{3sg}.\textsc{poss}-\textsc{sbj}:\textsc{pcp}-search go:\textsc{fact}-\textsc{1sg} \textsc{ifr}-say \textsc{lnk} \textsc{2du} where 2-go:\textsc{fact}-\textsc{du} \textsc{ifr}-say \textsc{lnk} \textsc{1du} \textsc{indef}.\textsc{poss}-husband \textsc{3sg}.\textsc{poss}-\textsc{sbj}:\textsc{pcp}-search go:\textsc{fact}-\textsc{1du} \textsc{ifr}-say  \\
 \glt `She said: `Where are you going?'; He said: `I am looking for a wife. Where are you going?'; She said `We are looking for a husband.'' (2003-kWBRa)
\end{exe}

Kinship terms\footnote{The glosses used to described kinship terms are explained in chapter \ref{chap:kinship} (Table \ref{tab:kinship.abb}). } also occur with the indefinite possessor prefix to talk about family relationships in abstract terms, as in (\ref{ex:tArpW.tAftsa}) (see also \ref{ex:tArpW} below). Note that in this example the verb is in the generic transitive subject form (§\ref{sec:indexation.generic.tr}). The kinship terms in this sentence cannot take the generic possessor prefix \forme{tɯ\trt}, since only one argument in a given sentence can be generic (§\ref{sec:genr.3pl}). If the generic possessor forms (\forme{tɯ-rpɯ} `one's maternal uncle' and \forme{tɯ-ftsa} `one's sister's child') were used instead, the meaning would be different (`One's uncle cannot marry one's nephew'), as it would include the speaker as a potential possessor.

\begin{exe}
\ex \label{ex:tArpW.tAftsa}
\gll tɤ-rpɯ cʰo tɤ-ftsa ni ci kú-wɣ-pa mɤ-kɯ-kʰɯ ɲɯ-ŋu. \\
\textsc{indef}.\textsc{poss}-MB \textsc{comit} \textsc{indef}.\textsc{poss}-ZC \textsc{du} one \textsc{ipfv}-\textsc{inv}-make \textsc{neg}-\textsc{inf}:\textsc{stat}-be.possible \textsc{sens}-be \\
\glt `Maternal uncles and sister's children cannot marry each other.' (140427 kWmdza stWnmW) 
\japhdoi{0003844\#S14}
\end{exe}

Some kinship terms have an extended meaning when they take the indefinite possessor prefix: they can alternatively be used to denote a class of humans based on gender and age. The noun \japhug{tɤ-tɕɯ}{son} also commonly means `boy' or even `male human' (regardless of age). The nouns \japhug{tɤ-wa}{father} and \japhug{tɤ-mu}{mother} can denote older people without reference to their children; translations such as `old man' and `old lady' are more appropriate in these cases, for instance in (\ref{ex:tAmu.ci}). The same applies to \japhug{tɤ-wɯ}{grandfather} and \japhug{tɤ-wi}{grandmother}.


\begin{exe}
\ex \label{ex:tAmu.ci}
\gll praʁkʰaŋ zɯ tɤ-mu ci ɯ-ku tɤ-kɯ-wɣrum ci zɯŋzɯŋ pjɤ-rɤʑi tɕe, \\
cave \textsc{loc} \textsc{indef}.\textsc{poss}-mother \textsc{indef} \textsc{3sg}.\textsc{poss}-head \textsc{aor}-\textsc{sbj}:\textsc{pcp}-be.white \textsc{indef} \textsc{idph}(II):white \textsc{ifr}.\textsc{ipfv}-stay \textsc{lnk} \\
\glt `In the cave, there was an old woman whose hair was completely white.' (2003sras)
\end{exe}

\subsubsection{Relator nouns}
Relator nouns are a subset of inalienably possessed nouns which have been grammaticalized as quasi-adpositions and compensate for the relative dearth of postpositions in Japhug (§\ref{ex:postpositions}). Some are used to express basic grammatical relations (such as the dative, §\ref{sec:dative}), as well as most locative and temporal relations with noun phrases and subordinate clauses (§\ref{sec:relator.location}, §\ref{sec:temporal.reference}). A list of relator nouns and a detailed account of their functions is presented in §\ref{sec:relator.nouns}.

\subsubsection{Complement-taking nouns and relativizers} \label{sec:complement.taking.IPN}
Inalienably possessed nouns can take nominalized or finite clauses as prenominal modifiers. When the head inalienably possessed noun is at the same time an argument or an adjunct inside its modifiying clause, that clause is considered to be a prenominal relative (§\ref{sec:prenominal.relative}). The generic inalienably possessed noun \japhug{ɯ-spa}{material} is in the process of becoming a relativizer when occurring with a prenominal relative (§\ref{sec:Wspa.relative}). In other Gyalrongic languages, such as Khroskyabs (\citealt[519]{lai17khroskyabs}), former generic nouns have become fully grammaticalized as relativizers (see §\ref{sec:lexicalized.oblique.participle}, §\ref{sec:Wspa.relative}).

When the head noun is not a participant of the clause, the modifying clause is a complement clause (§\ref{sec:complement.taking.nouns}, see \citealt[239--241]{jacques16complementation}). Inalienably possessed nouns selecting complement clauses include for instance \japhug{ɯ-skɤt}{language, noise} or \japhug{ɯ-ʁjiz}{wish} (§\ref{sec:complement.taking.noun.list}). 

The inalienably possessed noun \japhug{ɯ-mdoʁ}{colour} (from Tibetan \tibet{མདོག་}{mdog}{colour}) has been further grammaticalized from a com\-ple\-ment-taking noun to a sentence-final particle marker of epistemic modality `it looks like...' (§\ref{sec:WmdoR.TAME}).

 
\subsubsection{Property nouns} \label{sec:property.nouns}
Property nouns are a subclass of inalienably possessed nouns that designate (mainly in a derogatory fashion) an entity that possesses a particular characteristic. They generally follow another noun as in (\ref{ex:penzi.WpW}) and (\ref{ex:kha.WNqra}), but not exclusively (\ref{ex:Wxso.tWrme}). In the /noun+property noun/ phrase, the latter is the syntactic head but semantically modifies the former (see §\ref{sec:attributes} on the various attributes found in the noun phrase). 

\begin{exe}
\ex \label{ex:penzi.WpW}
 \gll <penzi> ɯ-pɯ, sɤlaŋpʰɤn ɯ-pɯ jamar ɲɯ-wxti cʰa \\
 basin \textsc{3sg}.\textsc{poss}-little.one  basin \textsc{3sg}.\textsc{poss}-little.one about \textsc{ipfv}-be.big can:\textsc{fact} \\
 \glt `It can grow about as big as a little basin.' (18-NGolo)
\japhdoi{0003530\#S45}
\end{exe}

\begin{exe}
\ex \label{ex:kha.WNqra}
 \gll kʰa ɯ-ɴqra tɕe znde ɯ-mbe ma tʰam kɯ-tu me. \\
house \textsc{3sg}.\textsc{poss}-broken.one \textsc{lnk} wall \textsc{3sg}.\textsc{poss}-old.one apart.from now \textsc{sbj}:\textsc{pcp}-exist not.exist:\textsc{fact} \\ 
\glt `Now there is nothing [there], apart from some ruins and old walls. (140522 tshupa)
\japhdoi{0004053\#S58}
\end{exe}

These phrases can be turned into compounds made of the first noun and a quasi-suffix corresponding to the property noun. All diminutive and derogatory suffixes described in §\ref{sec:diminutive} and §\ref{sec:derogatory} (\tabref{tab:property.nouns}) have corresponding property nouns. In the case of \forme{sɤlaŋpʰɤn ɯ-pɯ} from example (\ref{ex:penzi.WpW}) for instance, it is possible to say \japhug{sɤlaŋpʰɤn-pɯ}{little basin} as one word. In some cases the non-final element is in bound state, as in \japhug{kʰɤɴqra}{ruin} from \japhug{kʰa}{house} and \japhug{ɯ-ɴqra}{broken one}, a form which occurs in (\ref{ex:khANqra}), in the same text as (\ref{ex:kha.WNqra}) (referring to the same house). The opposite however is not always possible; for instance, lexicalized diminutives like \japhug{staχpɯ}{pea} from \japhug{stoʁ}{broad bean} cannot be turned into a phrase with \japhug{ɯ-pɯ}{little one} as second element.

\begin{exe}
\ex \label{ex:khANqra}
 \gll tɕe nɯ tɤtsoʁsta nɯnɯ kʰɤɴqra ɕti tʰam tɕe kɯ-rɤʑi me \\
\textsc{lnk} \textsc{dem} place.name \textsc{dem} ruins be:\textsc{aff}:\textsc{fact} now \textsc{lnk} \textsc{sbj}:\textsc{pcp}-stay not.exist:\textsc{fact} \\
\glt `Now Tatsogsta (`the place of silverweed') is a house in ruins, nobody lives there.' (140522 tshupa)
\japhdoi{0004053\#S56}
\end{exe}

Property nouns are not necessarily always contiguous to the noun that they follow. In (\ref{ex:GzW.ci.WpW.ci}), the indefinite determiner \forme{ci} (§\ref{sec:indef.article}) redundantly occurs both after the constituent \forme{ɣzɯ kɯ-xtɕɯ\redp{}xtɕi} (a relative clause, §\ref{ex:attributive.participles.stative.verbs}) and the property noun \forme{ɯ-pɯ} `the little one'. The nouns \forme{ɣzɯ} and \forme{ɯ-pɯ} are thus separated by the participle \forme{kɯ-xtɕɯ\redp{}xtɕi} and the determiner \forme{ci}.

\begin{exe}
\ex \label{ex:GzW.ci.WpW.ci}
\gll wo nɯnɯ, ɣzɯ kɯ-xtɕɯ\redp{}xtɕi ci ɯ-pɯ ci ɲɯ-ɕti \\
\textsc{interj} \textsc{dem} monkey \textsc{sbj}:\textsc{pcp}-emph\redp{}be.small \textsc{indef} \textsc{3sg}.\textsc{poss}-little.one \textsc{indef} \textsc{sens}-be.\textsc{aff} \\
\glt `Oh, this is [just] a little monkey.' (18-04-28 xiyouji01-zh)
\end{exe}

\begin{table}
\caption{Property nouns and corresponding quasi-suffixes} \label{tab:property.nouns}
\begin{tabular}{l|ll}
\lsptoprule
Property Noun & Suffix& \\
\midrule
\japhug{ɯ-pɯ}{little one} & \forme{-pɯ} &diminutive \\
\japhug{ɯ-ɴqra}{broken one} & \forme{-ɴqra} &derogatory \\
\japhug{ɯ-do}{old one} & \forme{-do} & \\
\japhug{tɤ-mbe}{old thing} & \forme{-mbe} & \\
\japhug{ɯ-kʰe}{something nasty} & \\
\japhug{ɯ-rqɯ}{cold thing} & \forme{-rqɯ} & other \\
\japhug{ɯ-xso}{empty, normal} & \\
\japhug{ɯ-jlu}{something uncooked} & \\
\japhug{ɯ-maŋ}{in big groups} & \\
\japhug{ɯ-rkoz}{something special} & \\
\lspbottomrule
\end{tabular}
\end{table}
 

The property nouns \japhug{ɯ-do}{old one} and \japhug{tɤ-mbe}{old thing} differ in that the former one is used for living things (including animals and plants), while the second occurs with inanimate objects. The quasi-suffix \forme{-rqɯ} is mainly used in \japhug{tɯ-cirqɯ}{cold water}.

The noun \japhug{ɯ-jlu}{uncooked} (used in particular with \japhug{stoʁ}{broad bean}) has become grammaticalized as a restrictive focus marker (§\ref{sec:restrictive.focus}).

Property nouns are not commonly used with an indefinite possessor prefix; in attested examples, it is always \forme{tɤ-}. Their origins are diverse: \japhug{ɯ-pɯ}{little one} derives from \japhug{tɤ-pɯ}{offspring, young} (see §\ref{sec:diminutive}), while \japhug{tɤ-mbe}{old thing}, \japhug{ɯ-kʰe}{something nasty} and \japhug{ɯ-do}{old one} originate from \japhug{mbe}{be old}, \japhug{kʰe}{be stupid} and \japhug{do}{be old (of plants)} by deverbal derivation (§\ref{sec:bare.action.nominals}). The property noun \japhug{ɯ-maŋ}{in big groups} derives from \japhug{maŋ}{be many}, itself from Tibetan \tibet{མང་}{maŋ}{many}. Some \forme{tɤ-} prefixed nouns of verbal origin like \japhug{tɤkʰe}{idiot, fool} (from \japhug{kʰe}{be stupid}) may come from former property nouns.
 
The property noun \japhug{ɯ-xso}{empty, normal} is related to the verb \japhug{so}{be empty}; it originally comes from its subject participle (the regular form \japhug{kɯ-so}{empty} is still attested) with loss of vowel and fricativization of the velar participle prefix (see §\ref{sec:G.nmlz}). It had no corresponding quasi-suffix, but does appear as second element in some compounds (see for instance §\ref{sec:collective}).

The most common meaning of \forme{ɯ-xso} is `normal, usual, common', a meaning already very different from the base verb. It occurs both before and after the noun with which it is linked (compare \ref{ex:Wxso.tWrme} and \ref{ex:tWrme.Wxso}). It is also used adverbially, meaning `usually' or `without doing anything' (\ref{ex:Wxso.kurAZi}).

\begin{exe}
\ex \label{ex:Wxso.tWrme}
\gll ɯ-pa ɲɯ-kɯ-ɕe nɯ tɕe kɯmaʁ tɯrme, ɯ-xso tɯrme ra nɯ-tɕʰaʁra pjɤ-ŋu \\
\textsc{3sg}.\textsc{poss}-down \textsc{ipfv}:\textsc{west}-\textsc{sbj}:\textsc{pcp}-go \textsc{dem} \textsc{lnk} other people \textsc{3sg}.\textsc{poss}-normal people \textsc{pl} \textsc{3pl}.\textsc{poss}-toilet \textsc{ifr}.\textsc{ipfv}-be \\
\glt `The toilet for other people, for normal people (i.e., not lamas), were on the [balcony] facing west under it.' (08-kWqhi) \japhdoi{0003454\#S8}
\end{exe} 

\begin{exe}
\ex \label{ex:tWrme.Wxso}
\gll  nɤʑo tɯrme ɯ-xso tɯ-maʁ \\
\textsc{2sg} people \textsc{3sg}.\textsc{poss}-normal 2-not.be:\textsc{fact} \\
\glt `You are not a normal human.' (150829 taishan zhi zhu-zh) \japhdoi{0006350\#S39}
\end{exe} 

\begin{exe}
\ex \label{ex:Wxso.kurAZi}
\gll ɯ-xso ku-rɤʑi tɕe, ɯ-βri nɯnɯ scoʁ-pɯ pɯ-kɤ-βʁum ʑo fse \\
\textsc{3sg}.\textsc{poss}-normal \textsc{ipfv}-stay \textsc{lnk} \textsc{3sg}.\textsc{poss}-body \textsc{dem} ladle-\textsc{dim} \textsc{aor}:\textsc{down}-\textsc{obj}:\textsc{pcp}-cover \textsc{emph} be.like:\textsc{fact} \\
\glt `When [the ladybug] is resting (staying like that, without doing anything), its body looks like a little laddle put upside down.' (26-kWlAGpopo) \japhdoi{0003670\#S3}
\end{exe} 

The meaning `empty' is however also attested; in (\ref{ex:nWxso.chAnWlhoRnW}) it is used adverbially, and note that the possessive prefix is coreferent with the plural intransitive subject.

\begin{exe}
\ex \label{ex:nWxso.chAnWlhoRnW}
\gll toʁde tɕe tɕendɤre, nɯ-xso cʰɤ-nɯ-ɬoʁ-nɯ. \\
a.moment \textsc{lnk} \textsc{lnk} \textsc{3pl}.\textsc{poss}-empty \textsc{ifr}:\textsc{downstream}-\textsc{auto}-come.out-\textsc{pl} \\
\glt `A moment later, they came out empty-handed.' (140512 alibaba-zh)
\japhdoi{0003965\#S34}
\end{exe} 

The obsolete property nouns \forme{*ɯ-te} `big' is not productive, but traces of it are still attested in some compounds (§\ref{sec:augmentative}).

\subsubsection{Exclamative inalienably possessed nouns} \label{sec:exclamative.IPN}
A small class of inalienably possessed nouns in Japhug occur as exclamative verbless nominal predicates (§\ref{sec:non.verbal.predicates}), sometimes with the sentence final particle \forme{nɯ} (§\ref{sec:fsp.attitude}). This class includes degree nominals (§\ref{sec:degree.monoclausal}), as well as the non-derived \japhug{tɯ-scawa}{poor $X$} and \japhug{tɯ-kʰi}{lucky $X$}. 

The inalienably possessed noun \japhug{tɯ-scawa}{poor $X$} only occurs in the exclamative constructions, as in (\ref{ex:nWscawa}) and (\ref{ex:ascawa}). In example (\ref{ex:nWscawa}), the possessive prefix is coreferent with the entities that experience suffering (the pigs).

\begin{exe}
\ex \label{ex:nWscawa} 
\gll tsuku kɯ paʁndza ɲɯ-nɯ-pʰɯt-nɯ ɲɯ-ŋu ri, paʁ ra nɯ-scawa ma mɤ-mɯm ma ɯ-tɯ-qiaβ saχaʁ ʑo. \\
some \textsc{erg} hogwash \textsc{ipfv}-\textsc{auto}-pluck-\textsc{pl} \textsc{sens}-be \textsc{lnk} pig \textsc{pl} \textsc{3pl}.\textsc{poss}-poor \textsc{lnk} \textsc{neg}-be.tasty:\textsc{fact} \textsc{lnk} \textsc{3sg}.\textsc{poss}-\textsc{nmlz}:\textsc{deg}-be.bitter be.extremely:\textsc{fact} \textsc{emph} \\
\glt `Some people cut [\textit{Sambucus}] as hogwash, poor pigs, it is so bitter.' (12-ndZiNgri)
\japhdoi{0003488\#S28}
\end{exe}

The possessive prefix on this noun can also be coreferent not with the person suffering, but rather with another person who caused it, and expresses his apologies in this manner, as in (\ref{ex:ascawa}).

\begin{exe}
\ex \label{ex:ascawa}
\gll wo a-tɤɕime a-scawa, wo a-tɤɕime a-scawa \\
\textsc{interj} \textsc{1sg}.\textsc{poss}-lady \textsc{1sg}.\textsc{poss}-poor \textsc{interj} \textsc{1sg}.\textsc{poss}-lady \textsc{1sg}.\textsc{poss}-poor \\ 
\glt `My lady, sorry [for what] I [have done to you].' (2014-kWlAG)
\end{exe}

Alternatively, the indefinite possessive form \forme{tɯ-scawa} can occur, even if the person/entity experiencing misfortune is definite and known, as in (\ref{ex:tWscawa}).

\begin{exe}
\ex \label{ex:tWscawa}
\gll wo tɯ-scawa, ku-tɯ-tso mɯ-pɯ-ra \\
\textsc{interj} \textsc{indef}.\textsc{poss}-poor \textsc{ipfv}-2-understand \textsc{neg}-\textsc{pst}.\textsc{ipfv}-be.needed \\
\glt `Alas and woe, you should not have known that.' (2012 Norbzang) \japhdoi{0003768\#S142}
\end{exe}

The inalienably possessed noun \japhug{tɯ-kʰi}{lucky $X$} is another example of the nominal exclamative construction, as in (\ref{ex:nWkhi.Ge}), with the possessive prefix coreferent with the person experiencing good luck. This noun can also occur in the idiom \japhug{tɯ-kʰi+ŋgɯ}{be lucky}, as in (\ref{ex:akhi.YWNgW}).

\begin{exe}
\ex \label{ex:nWkhi.Ge}
\gll ɕɯ ɣɯ ŋu kɯ, nɯ-kʰi ɣe! \\
who \textsc{gen} be:\textsc{fact} \textsc{sfp} \textsc{3pl}.\textsc{poss}-how.lucky \textsc{sfp} \\
\glt `Whose are these, how lucky they are!' (2003 Kunbzang)
\end{exe}

\begin{exe}
\ex \label{ex:akhi.YWNgW}
\gll a-kʰi ɲɯ-ŋgɯ \\
\textsc{1sg}.\textsc{poss}-lucky(1) \textsc{sens}-be.lucky(2) \\
\glt `I am lucky.' (140425 shizi puluomixiusi he daxiang-zh) \japhdoi{0003798\#S40}
\end{exe}

\subsubsection{Alienabilization} \label{sec:alienabilization}
 It is possible to turn an inalienably possessed noun into an alienably possessed one by adding a definite possessor prefix before the indefinite one; this is the only case of possessive prefix stacking in Japhug. This process is very productive, and better illustrated by minimal pairs; the following examples involve the inalienably possessed nouns \japhug{tɯ-ci}{water}, \japhug{tɤ-lu}{milk} and \japhug{tɤ-muj}{feather}.
 
The noun \japhug{tɯ-ci}{water} with a definite possessor (\japhug{ɯ-ci}{its juice/water}) refers either to the juice of a plant, or to water in which a plant has been soaked as in (\ref{ex:Wci})

 \begin{exe}
\ex \label{ex:Wci}
 \gll ɯʑo tɯ-ci kɯ-sɤ-ɕke ɯ-ŋgɯ pjɯ́-wɣ-ɣɤ-la, tɕe nɯ ɣɯ ɯ-ci ɯ-ŋgɯ nɯtɕu tɯ-mi pjɯ́-wɣ-ɣɤ-la tɕe nɯnɯ, χtɕoŋ nɯ ɲɯ-pʰɤn ɲɯ-ti-nɯ ri, \\
\textsc{3sg} \textsc{indef}.\textsc{poss}-water \textsc{sbj}:\textsc{pcp}-\textsc{prop}-burn \textsc{3sg}-in \textsc{ipfv}-\textsc{inv}-\textsc{caus}-soak \textsc{lnk} \textsc{dem} \textsc{gen} \textsc{3sg}.\textsc{poss}-water \textsc{3sg}-in \textsc{dem}:\textsc{loc} \textsc{genr}.\textsc{poss}-foot \textsc{ipfv}-\textsc{inv}-\textsc{caus}-soak \textsc{lnk} \textsc{dem}, rheumatism \textsc{dem} \textsc{sens}-be.efficient \textsc{sens}-say-\textsc{pl} \textsc{lnk} \\
 \glt `One puts it in hot water, and then one puts one's feet in that water, and it is efficient against rheumatism, they say.' (20-sWrna)
\japhdoi{0003564\#S133}
 \end{exe}

The alienabilized form \japhug{ɯ-tɯ-ci}{its water}, as in (\ref{ex:WtWci}), is used to talk about water given to an animal to drink, or water absorbed by a plant.

 \begin{exe}
\ex \label{ex:WtWci}
 \gll tɕeri ɯ-tɯ-ci wuma ʑo na-ʁzi tɕe, ɯ-tɯ-ci nɯ mɯ-pjɯ-mbrɤt ɲɯ-ra. \\
 but \textsc{3sg}.\textsc{poss}-\textsc{indef}.\textsc{poss}-water really \textsc{emph} \textsc{trop}-be.necessary:\textsc{fact} \textsc{lnk} \textsc{3sg}.\textsc{poss}-\textsc{indef}.\textsc{poss}-water \textsc{dem} \textsc{neg}-\textsc{ipfv}-\textsc{acaus}:break \textsc{sens}-be.needed \\
 \glt `But it needs water a lot, it needs to have water continuously.' (07-Zmbri)
\japhdoi{0003438\#S10}
 \end{exe}
 
 When a definite possessor is present on the noun \japhug{tɤ-lu}{milk} in a form such as \japhug{ɯ-lu}{her milk}, that prefix refers to the animal producing the milk, as in (\ref{ex:Wlu}).
 
 \begin{exe}
\ex \label{ex:Wlu}
 \gll 
tɤ-pi kɯ-wxti nɯ kɯ nɯŋa ɣɯ ɯ-lu nɯ cʰondɤre ɯ-ɕa nɯ to-nɯ-ndo. \\
\textsc{indef}.\textsc{poss}-elder.sibling \textsc{sbj}:\textsc{pcp}-be.big \textsc{dem} \textsc{erg} cow \textsc{gen} \textsc{3sg}.\textsc{poss}-milk \textsc{dem} \textsc{comit} \textsc{3sg}.\textsc{poss}-meat \textsc{dem} \textsc{ifr}-\textsc{auto}-take \\
\glt `The elder brother took the cow's milk and meat.' (02-deluge2012)
\japhdoi{0003376\#S18}
 \end{exe}

The form \japhug{ɯ-tɤ-lu}{his/its milk} with alienabilization is used on the other hand when indicating the person or animal drinking the milk, as in (\ref{ex:WtAlu}).

\begin{exe}
\ex \label{ex:WtAlu}
 \gll tɕe ɯ-tɤ-lu pjɯ́-wɣ-rku tɕe nɯnɯ pjɯ-tsʰi qʰe,\\
\textsc{lnk} \textsc{3sg}.\textsc{poss}-\textsc{indef}.\textsc{poss}-milk \textsc{ipfv}:\textsc{down}-\textsc{inv}-put.in \textsc{lnk} \textsc{dem} \textsc{ipfv}:\textsc{down}-drink \textsc{lnk}\\ \\
\glt `People pour milk for it (i.e., the cat) to drink, and it drinks it.' (21-lWLU)
\japhdoi{0003576\#S43}
\end{exe}


The inalienably possessed noun \japhug{tɤ-muj}{feather} takes as its possessor a bird (or a bird body part such as `wings'), as in (\ref{ex:Wmuj}).

  \begin{exe}
\ex \label{ex:Wmuj}
 \gll  jinde tɕe ɯ-kɯ-sat koŋla maŋe tɕe, nɯ qarma ɯ-muj kɯnɤ tɯ-jaʁ mɯ́j-ɣi wo \\
 nowadays \textsc{lnk} \textsc{3sg}.\textsc{poss}-\textsc{sbj}:\textsc{pcp}-kill completely not.exist:C \textsc{lnk} \textsc{dem} crossoptilon \textsc{3sg}.\textsc{poss}-feather also \textsc{genr}.\textsc{poss}-hand \textsc{neg}:\textsc{sens}-come \textsc{sfp} \\
 \glt `Nowadays, nobody kills them, and one cannot get crossoptilon feathers.' (23-qapGAmtWmtW)
\japhdoi{0003608\#S152}
 \end{exe}

Its alienabilized form, such as \japhug{ɯ-tɤ-muj}{his feather} in (\ref{ex:WtAmuj}), is used when the feather is detached from the body of the bird, and belongs to a human.
 
\begin{exe}
\ex \label{ex:WtAmuj}
 \gll tɤtɕɯpɯ kɯ-xtɕi nɯ ɣɯ ɯ-tɤ-muj nɯ li ɯ-tʰoʁ nɯtɕu pjɤ-nɯ-jɣɤt \\
 boy:\textsc{dim} \textsc{sbj}:\textsc{pcp}-be.small \textsc{dem} \textsc{gen} \textsc{3sg}.\textsc{poss}-\textsc{indef}.\textsc{poss}-feather \textsc{dem} again \textsc{3sg}.\textsc{poss}-ground \textsc{dem}:\textsc{loc} \textsc{ifr}:\textsc{down}-\textsc{auto}-go.back \\
 \glt `The younger boy's feather fell back on the ground again.' (140510 sanpian yumao)
\japhdoi{0003947\#S65}
\end{exe}
 
As the examples above show, alienabilized inalienably possessed nouns occur to refer to disconnected or severed body parts, for instance body parts removed from an animal that are used or owned by a human or another animal on which they do not grow. They are also used for bodily fluids that have left the body, or also clothes that are not worn but held in the hand. Similar phenomena are observed in other Gyalrong languages (see \citealt[140]{jackson98morphology} on Tshobdun).

The referent marked by the possessive prefix can be beneficiary as in (\ref{ex:WtAlu}) or possessor as in (\ref{ex:WtAmuj}).
 
Alienabilization is also observed with prenominal modifiers (§\ref{sec:possessive.prefixes.prenominal}), in compounding, when the indefinite possessor prefix of an inalienably possessed noun is preserved in the final member of the compound (see §\ref{sec:possessive.prefix.second.compounds}), in comitative adverbs derived from inalienably possessed nouns (§\ref{sec:comitative.adverb}) and in conversion from inalienably possessed noun to counted noun (§\ref{sec:CN.IPN}). A related phenomenon is also the optional neutralization of possessive prefixes in relative clauses (§\ref{sec:relative.possessor.neutralization}).
 
A lexicalized way of alienabilizing nouns is by compounding with a generic possessor. For instance, \japhug{pɣɤmuj}{feather} is an alienably possessed noun built from the bound state of \japhug{pɣa}{bird} with the inalienably possessed noun \japhug{tɤ-muj}{feather}. Here the first element of the compound \forme{pɣɤ-} saturates the inalienable possessor without need to use the prefix \forme{tɤ-}.
 
\subsubsection{Frozen indefinite possessors} \label{sec:frozen.indef}
Alienably possessed nouns with a disyllabic root whose first element is \forme{tɯ-} or \forme{tɤ\trt}, with the exception of loanwords such as \japhug{tɯrsa}{grave} (from \tibet{དུར་ས་}{dur.sa}{grave}), are mainly ancient inalienably possessed nouns whose indefinite possessor prefix \forme{tɯ-} has become frozen and reanalyzed as part of the root. Comparison with other Gyalrong languages can demonstrate that such reanalysis took place in Japhug.

For instance, the noun \japhug{tɯrme}{person} is alienably possessed in Japhug (as shown by examples such as \ref{ex:atWrme}), but in the Situ language the 3\sg{} form of \forme{tə-rmî} `man' is \forme{wǝ-rmî} (\citealt[183;197]{lin09phd}), showing that \forme{tə-} is the indefinite possessor prefix, cognate of Japhug \forme{tɯ-}. 

\begin{exe}
\ex \label{ex:atWrme}
\gll ci nɯ tɤ-tɕɯ nɯ χawo, nɯnɯ aʑo a-tɯrme nɯ a-pɯ-ŋu ndɤre, nɯ ɯ-tɯ-pe nɯ\\
one \textsc{dem} \textsc{indef}.\textsc{poss}-boy \textsc{dem} \textsc{interj} \textsc{dem} \textsc{1sg} \textsc{1sg}.\textsc{poss}-man \textsc{dem} \textsc{irr}-\textsc{ipfv}-be \textsc{lnk} \textsc{dem} \textsc{3sg}-\textsc{nmlz}:\textsc{deg}-be.good \textsc{fsp} \\
\glt `That boy, if only he could be my man, it would be so nice.' (2014-kWlAG)
\end{exe}

This shift may be due to the fact that the 3\sg{} form is used in Situ in constructions where the non-possessed form is preferred in Japhug, such as in prenominal relatives \citep[190]{lin09phd}, and was therefore less prone to lexicalization. The stem \forme{-rme} of \japhug{tɯrme}{person} is still attested as second element of compounds like \japhug{tɯ-pɤrme}{one year of life} (whose first element \forme{pɤ-} is related to the stem of \japhug{tɯ-xpa}{one year}, §\ref{sec:num.prefix.paradigm.history}). 

The noun \japhug{tɤjmɤɣ}{mushroom} is alienably possessed, but the stem \forme{jmɤɣ-} appears as first element of compounds such as \japhug{jmɤɣni}{russula}, suggesting that it was formerly inalienably possessed and occurred without its indefinite possessor prefix in this compound (see §\ref{sec:loss.possessive.prefix.compounds}). The status of \japhug{tɤjmɤɣ}{mushroom} as a former inalienably possessed noun is less surprising if one takes into account the likely etymological relationship with Chinese \zh{帽} \forme{mawH} `hat' (from \forme{*mˤuk-s}; etymology suggested by L. Sagart; see also the Tibetan cognate \tibet{རྨོག་}{rmog}{helmet}, \citealt{zhangsy19cognates}). If the noun for `mushroom' in Japhug and other Gyalrongic languages comes from `hat' (cf Breton \forme{tog touseg} `toad hat' for `mushroom'), it is expected that it would become an inalienably possessed noun (like \japhug{tɤ-rte}{hat}), and for the indefinite possessor \forme{tɤ-} to become frozen after the noun ceases to be a term for head covers.
 
\subsubsection{Unusual inalienably possessed nouns in Japhug} \label{sec:earth.IPN}
While it is crosslinguistic expected that nouns of body parts or kinship terms are inalienably possessed, we also find in Japhug inalienably possessed nouns denoting natural entities such as \japhug{tɯ-ci}{water}, \japhug{tɯ-mɯ}{sky, weather} or \japhug{ɯ-tʰoʁ}{ground}, a highly unusual fact. There is no grand insight about Gyalrong \textit{Weltanschauung} to be gained from this observation however; explanations should be sought in the etymology of these words, and solved on an item per item basis. 

The noun \japhug{tɯ-ci}{water} also means `juice' or `water in which $X$ has been soaked' with a definite possessor, as was seen in §\ref{sec:alienabilization}. Cognates are found in Core Gyalrong languages, but not in West Gyalrongic (Stau \forme{ɣrə} and Wobzi Khroskyabs \forme{jdə̂}, \citealt[610]{jacques17stau}) or elsewhere in the family, and it is therefore a good candidate for a Core Gyalrong lexical innovation. 

Japhug has a transitive verb \japhug{ci}{pour completely} (of grains or liquids), from which a bare action nominal \forme{*ɯ-ci} `(liquid/grain) that has been poured out' could have been regularly derived (see §\ref{sec:bare.action.nominals}; similar to \japhug{ɯ-ndzɯ}{instruction, advice} from \japhug{ndzɯ}{educate} in example \ref{ex:WmW.mbWt} below). The meaning `water' would then be trivial narrowing of the meaning of this noun `water poured out', then replacing the older term for `water' still preserved in West Rgyalrongic. Bare action nominals being inalienably possessed nouns (§\ref{sec:bare.action.nominals}), the form of \japhug{tɯ-ci}{water} accounted for by this etymology.

The stative verb \japhug{aci}{be wet} is then derived, after the semantic narrowing, from the noun \japhug{tɯ-ci}{water} by denominal derivation (§\ref{sec:denom.a}) -- despite superficially looking like a passive of \japhug{ci}{pour completely (of grains or liquids)} (§\ref{sec:passive}), it is only indirectly derived from it. 

Concerning \japhug{tɯ-mɯ}{sky, weather}, it superficially resembles a noun with non-analyzable \forme{tɯ-} prefixal element, but the status of this element as an indefinite possessor prefix can be ascertained with rare examples such as (\ref{ex:WmW.mbWt}). In addition, note the compound \japhug{kɯndzarmɯ}{type of rain} contains the root \forme{-mɯ} as its last syllable (see a precise definition of this noun and a discussion of its etymology in \ref{ex:kWndzarmW}, §\ref{sec:lexicalized.subject.participle}).

\begin{exe}
\ex \label{ex:WmW.mbWt}
\gll ɯ-ndzɯ mɤ-kɯ-sɤŋo ɯ-mɯ mbɯt \\
\textsc{3sg}.\textsc{poss}-instruction \textsc{neg}-\textsc{nmlz}-S/A-listen \textsc{3sg}.\textsc{poss}-sky \textsc{acaus}:take.off:\textsc{fact} \\
\glt `Those who do not listen to advice from other people do not end well.' (`their sky falls') (elicited)
\end{exe}

There is no clear explanation of how this noun come have become inalienably possessed, but I propose here a tentative etymology. Cognates of \japhug{tɯ-mɯ}{sky, weather} are attested elsewhere in the Trans-Himalayan family, but mainly in languages that poorly preserve presyllables (for instance Yongning Na \forme{mv̩˥°}, \citealt[132]{michaud17yongning}). Yet, in Rawang, among the conservative languages, has a word \forme{dvmø̀} `celestial being' (\citealt[13]{lapolla01rawang}), with the same vowel correspondence to Japhug \ipa{-ɯ} as \forme{sharø} `bone' with \japhug{ɕɤrɯ}{bone}. Moreover, the name \tibt{དམུ་}{dmu}, attested in Tibetan texts to refer to a type of divinity, is probably related to the Gyalrong etymon for `sky' (\citealt[63--64]{stein61}) and provides additional support for the antiquity of a dental presyllables in this etymon.

If \forme{dvmø̀} and \tibt{དམུ་}{dmu} are indeed cognate with Japhug \japhug{tɯ-mɯ}{sky, weather},\footnote{Another potential cognate could be Rawang \forme{muq} `sky, thunder', but the final glottal stop transcribed \forme{-q} is from a former \forme{*-k}, and this word is better compared to Situ \forme{ta-rmōk} `thunder' (\citealt[73]{zhang16bragdbar})} this noun may originally have been disyllabic, and its first syllable reinterpreted as indefinite possessive; the form \japhug{ɯ-mɯ}{his sky} in (\ref{ex:WmW.mbWt}) would then be a backformation, an idea compatible with its very marginal character. 
 
The Japhug noun \japhug{ɯ-tʰoʁ}{ground} cannot take any possessive prefix other than 3\sg{} \forme{ɯ\trt}, not even the indefinite possessor prefix. It has no known cognates in other Gyalrongic languages, but it is a perfect match for a Tibetan word with the shape \forme{tʰog} (compare the other borrowed noun \japhug{tʰoʁ}{thunder} from Tibetan \tibet{ཐོག་}{tʰog}{thunder}). Two etymologies accounting for the possessive prefix on \forme{ɯ-tʰoʁ} can be proposed. On the one hand, it could be a bare action nominal (§\ref{sec:bare.action.nominals}) from the transitive verb \japhug{tʰoʁ}{stamp on}. On the other hand, it could alternatively be a borrowing from Tibetan, a hypothesis requiring a four step scenario.

First, Japhug borrowed the Tibetan relator noun \tibet{ཐོག་ཏུ་}{tʰog(tu)}{on} as \forme{ɯ-tʰoʁ} *`on' (not attested), adding a third person possessive prefix like all relator nouns (see §\ref{sec:relator.nouns} ). This relator noun was in competition with the existing native equivalent \japhug{ɯ-taʁ}{on, above}.\footnote{It is not surprising in Japhug to have several competing relator nouns for the same functional slot; the same is true of the dative \forme{ɯ-ɕki} and \forme{ɯ-pʰe}, see §\ref{sec:dative}. }
 
Second, it became restricted to the collocation \forme{*sɤtɕʰa ɯ-tʰoʁ zɯ} `on the ground' (not attested), with the native locative \forme{zɯ} and the noun of Tibetan origin \japhug{sɤtɕʰa}{earth}.
  
Third, the tautological collocation  \forme{*sɤtɕʰa ɯ-tʰoʁ zɯ}`on the ground' was reduced to \japhug{ɯ-tʰoʁ zɯ}{on the ground} (attested).
 
Fourth, the noun \japhug{ɯ-tʰoʁ}{ground} was created by backformation from the locative phrase \forme{ɯ-tʰoʁ zɯ} `on the ground'. The fact that the locative postposition \ipa{zɯ} is always optional (§\ref{sec:core.locative}) made this step less unlikely. Thus, Japhug possibly attests an example of degrammation (see \citealt[135]{norde09degrammaticalization}) from a relator noun meaning `on' (with or without motion) to a common noun meaning `ground'. 

The etymologies discussed above suggest that inalienably possessed nouns referring to natural phenomena in Japhug were created by unrelated pathways.

\subsubsection{Adverbial inalienably possessed nouns} \label{sec.IPN.adverbs}
Some inalienably possessed nouns can be used adverbially (§\ref{sec:sentential.adverbs}). In this function, the possessive prefix can be neutralized to indefinite possessor or third singular possessor, as \japhug{ɯ-stu}{truth, truly} in (\ref{ex:Wstu.Zo}), but in some cases it can also be coreferent with an argument, and different inalienably possessed nouns display different alignment patterns.

\begin{exe}
\ex \label{ex:Wstu.Zo}
\gll koŋla ɯ-stu ʑo a-pɯ-tɯ-rɤ-βzjoz, <zuoye> a-pɯ-tɯ-βze, \\
really \textsc{3sg}.\textsc{poss}-truth \textsc{emph} \textsc{irr}-\textsc{pfv}-2-\textsc{antip}-study homework \textsc{irr}-\textsc{pfv}-2-do[III] \\
\glt `Study seriously, do your homework.' (conversation 140501)
\end{exe}

The prefix of the inalienably possessed noun \japhug{ɯ-stu}{truth, truly} can be coreferent with the singular or the transitive subject (like the \textsc{2sg} prefix \forme{nɤ-} in \ref{ex:nAstu.Zo.WYWkWnWrgaa}), but never with the object, thus displaying an accusative alignment.

\begin{exe}
\ex \label{ex:nAstu.Zo.WYWkWnWrgaa}
\gll nɤʑo nɤ-stu ʑo ɯ-ɲɯ-kɯ-nɯ-rga-a nɤ, \\
\textsc{2sg} \textsc{2sg}.\textsc{poss}-really \textsc{emph} \textsc{qu}-\textsc{ipfv}-2\fl{}1-\textsc{appl}-like-\textsc{1sg} \textsc{lnk} \\
\glt `If you really love me,...' (150907 yingning-zh) \japhdoi{0006264\#S136}
\end{exe}

On the other hand, \japhug{ɯ-βra}{it is X's turn to...} presents a neutral alignment pattern: the possessive prefix can be coreferent with the intransitive subject, the object (as in \ref{ex:nAZo.nABra}), or the transitive subject (\ref{ex:nAZo.nABra.tAndze}).

%jisŋi tɕe aʑo a-βra ci zgoku tu-ɕe-a tɕe, ʑ-ɲɯ-nɤmɲole-

\begin{exe}
\ex \label{ex:nAZo.nABra}
\gll iɕqʰa nɤ-zda nɯ pɯ-sat-a ŋu tɕe, tʰam tɕe nɤʑo nɤ-βra pjɯ-ta-sat ra \\
just.before \textsc{2sg}.\textsc{poss}-companion \textsc{dem} \textsc{aor}-kill-\textsc{1sg} be:\textsc{fact} \textsc{lnk} now \textsc{lnk} \textsc{2sg} \textsc{2sg}.\textsc{poss}-turn \textsc{ipfv}-1\fl{}2-kill be.needed:\textsc{fact} \\
\glt `I just killed your companion, now it's your turn.' (elicited)
\end{exe}

\begin{exe}
\ex \label{ex:nAZo.nABra.tAndze}
\gll nɤj nɤ-βra tɤ-ndze \\
\textsc{2sg} \textsc{2sg}.\textsc{poss}-turn \textsc{imp}-eat[III] \\
\glt `It is your turn to eat.' (elicited)
\end{exe}

Thus, the alignment pattern of each adverbial inalienably possessed noun must be specified.

\subsubsection{Biactantial inalienably possessed nouns} \label{sec:biactantial.ipn}
A few inalienably possessed nouns, such as words designating speech or presents, select more than one argument, and can be considered to be the nominal equivalent of ditransitive verbs (if alienably and inalienably possessed nouns are compared to intransitive and transitive verbs, respectively, §\ref{sec:inalienably.possessed.morpho}). Since only one argument however is marked by a possessive prefix on the noun (possessive prefix stacking is not possible except for alienabilization, see §\ref{sec:alienabilization}) a choice has to be made as to which of the two arguments, the speaker/giver or the addressee/recipient, is marked on the noun.

The possessive prefix of inalienably possessed noun \japhug{tɤ-pɤro}{present} always marks the giver; the recipient of the present (which is optional) receives genitive case. 

For instance, in (\ref{ex:apAro}) the genitive pronoun \forme{nɤʑɯɣ} encodes the recipient, and the \textsc{1sg} possessive prefix on the noun is coreferent with the subject of the main verb. In (\ref{ex:nApAro}), the recipient is not overt, and the \textsc{2sg} prefix on the noun is again coreferent with the transitive subject of \japhug{ɣɯt}{bring}.

\begin{exe}
\ex \label{ex:apAro}
\gll nɤʑɯɣ a-pɤro tɕʰi ju-ɣɯt-a ra? \\
\textsc{2sg}:\textsc{gen} \textsc{1sg}.\textsc{poss}-present what \textsc{ipfv}-bring-\textsc{1sg} be.needed:\textsc{fact} \\
\glt `What present should I bring for you?' (140504 huiguniang-zh)
\japhdoi{0003909\#S27}
\end{exe}

\begin{exe}
\ex \label{ex:nApAro}
\gll nɤʑo dɯxpa pɯ-tɯ-tu ma li nɤ-pɤro jɤ-tɯ-ɣɯt! \\
\textsc{2sg} hardship \textsc{pst}.\textsc{ipfv}-2-exist \textsc{lnk} again \textsc{2sg}.\textsc{poss}-present \textsc{aor}-2-bring \\
\glt `Thank you, you brought another present [for me].' (elicited)
\end{exe}

Other biactantial nouns use possessive prefixes to indicate the recipient rather than the agent. For instance, the inalienably possessed noun \japhug{tɤ-rkuz}{parting present} (a rare example of \forme{-z} nominalization suffix in Japhug, see §\ref{sec:z.nmlz}) always marks the recipient, as in (\ref{ex:nArkuz}), never the agent -- the form \forme{nɤ-rkuz} `your parting present' with \textsc{2sg} possessive prefix can only mean `a parting present for you', not `the parting present you give to me/him'.

\begin{exe}
\ex \label{ex:nArkuz}
\gll kɯki nɤ-rkuz ŋu \\
\textsc{dem}:\textsc{prox} \textsc{2sg}.\textsc{poss}-parting.present be:\textsc{fact} \\
\glt `This is a parting present for you.' (28-smAnmi) \japhdoi{0004063\#S248}
\end{exe}

In other cases, the alignment of possessive prefixes on an inalienably possessed noun depends on the particular construction where it appears. For instance, the inalienably possessed noun \japhug{tɯ-tɕʰa}{information, news} (about someone) marks the recipient when used with the verbs \japhug{kʰo}{give} or \japhug{tu}{exist}, but has neutral alignment in other contexts. In (\ref{ex:atCha.mWnakho}), the indirective verb \japhug{kʰo}{give} (§\ref{sec:ditransitive.indirective}) does not index the recipient, whose only mark is the possessive prefix on \forme{a-tɕʰa} `news for me'. Changing the prefix to the third singular \forme{ɯ-} to refer to the subject here would be agrammatical (see however §\ref{sec:hybrid.intro} and §\ref{sec:hybrid indirect} on hybrid indirect speech).


\begin{exe}
\ex \label{ex:atCha.mWnakho}
\gll a-tɕɯ kɯ a-tɕʰa mɯ-na-kʰo. \\
\textsc{1sg}.\textsc{poss}-son \textsc{erg} \textsc{1sg}.\textsc{poss}-news \textsc{neg}-\textsc{aor}:3\flobv{}-give \\
\glt `I have not heard from my son.' (`My son did not give me any response.') (elicited)
\end{exe}

However, in other constructions, there are no such constraints on the use of possessive prefixes on \japhug{tɯ-tɕʰa}{information, news}(about someone): for instance, with the verb \japhug{ɣɯt}{bring} (§\ref{sec:nouns.speech.complement}) in (\ref{ex:WtCha.pjWGWta}), although the recipient is second person dual, \forme{ɯ-tɕʰa} takes the \textsc{3sg} prefix, coreferent with the preceding complement clause (using second person singular \forme{nɤ-tɕʰa} here would be agrammatical).

\begin{exe}
\ex \label{ex:WtCha.pjWGWta}
\gll `ma-nɯ-tɯ-ɣɤwu-ndʑi tɕe aʑo tu-ɕe-a tɕe atu ɕ-tu-tʰe-a tɕe,
ndʑi-pa ndʑi-ma ni ɯ-ɲɯ-nɯjʁo-ndʑi kɯ' ɯ-tɕʰa pjɯ-ɣɯt-a \\
\textsc{neg}-\textsc{imp}-2-cry-\textsc{du} \textsc{lnk} \textsc{1sg} \textsc{ipfv}:\textsc{up}-go-\textsc{1sg} \textsc{lnk} up \textsc{tral}-\textsc{ipfv}-ask[III]-\textsc{1sg} \textsc{lnk} \textsc{2du}.\textsc{poss}-father \textsc{2du}.\textsc{poss}-mother \textsc{du} \textsc{qu}-\textsc{ipfv}-scold-\textsc{du} \textsc{qu} \textsc{3sg}.\textsc{poss}-information \textsc{ipfv}:\textsc{down}-bring-\textsc{1sg} \\
\glt `Don't cry, I will go up there, ask whether your parents will scold you and come back to tell you.' (2003-kWBRa)
\end{exe}

Alignment effects are also found with alienably possessed nouns. For instance, the alienably possessed \japhug{skɯrma}{present} can also optionally take a possessive prefix, which is always coreferent with the recipient, not with the agent, as shown by (\ref{ex:askWrma}), where \forme{nɤ-mu ɣɯ ɯ-skɯrma} means `a present (sent) to your mother' and \forme{a-skɯrma} can only mean `a present sent to me' (from a text explaining the meaning difference between \forme{skɯrma}, \forme{tɤ-pɤro} and \forme{tɤ-rkuz}).\footnote{Japhug has three words that can be translated as `present': \forme{tɤ-pɤro} is used for presents one give to the recipient in person, \forme{tɤ-rkuz} is a parting present one gives before a person leaves a place, and \forme{skɯrma} is a present given with the help of a third party.}

\begin{exe}
\ex \label{ex:askWrma}
\gll nɤ-mu ɣɯ tʰɯci tu-rke-a tɕe ju-tɯ-tsɯm tɕe nɯnɯ nɤ-mu ɣɯ ɯ-skɯrma ju-sɯ-ɣɯt-a ŋu nɤ-mu kɯ a-tɤ-rke tɕe, a-jɤ-tɯ-ɣɯt tɕe, tɕe nɯnɯ li a-skɯrma jɤ-kɤ-sɯ-ɣɯt ŋu \\
\textsc{2sg}.\textsc{poss}-mother \textsc{gen} something \textsc{ipfv}-put.in[III]-\textsc{1sg} \textsc{lnk} \textsc{ipfv}-2-take.away \textsc{lnk} \textsc{dem} \textsc{2sg}.\textsc{poss}-mother \textsc{gen} \textsc{3sg}.\textsc{poss}-present \textsc{ipfv}-\textsc{caus}-bring-\textsc{1sg} be:\textsc{fact} \textsc{2sg}.\textsc{poss}-mother \textsc{erg} \textsc{irr}-\textsc{pfv}-put.in[III] \textsc{lnk} \textsc{irr}-\textsc{pfv}-2-bring \textsc{lnk} \textsc{lnk} \textsc{dem} again \textsc{1sg}.\textsc{poss}-present \textsc{aor}-\textsc{obj}:\textsc{pcp}-\textsc{caus}-bring be:\textsc{fact} \\
\glt `When I prepare something for your mother and you take it to her, [I can say] `I sent a present to you mother', if your mother prepares something and you bring it [to me], it is `a present sent to me.'' (def-skWrma)
\end{exe}

Other inalienably possessed nouns of this type include \japhug{tɯ-nŋa}{debt} (§\ref{sec:bare.action.nominals}, used with the verbs \japhug{tʂo}{pay}, \japhug{ɬoʁ}{come out} and \japhug{sti}{block}), which select as possessor the person owing money (rather than the one to whom one owes money), and \japhug{ɯ-pʰɯpʰɯ}{alms}, whose possessor is the person receiving alms (example \ref{ex:nAphWphW.kAGWt}, §\ref{sec:distributed.action.lexicalized}).

\subsection{Indefinite vs. generic possessor} \label{sec:indef.genr.poss}
The generic possessive prefix \forme{tɯ-} is formally identical to the indefinite possessor prefix of some inalienably possessed nouns, but must be strictly distinguished from it. Five criteria can be used to determine if a \forme{tɯ-} prefix is generic, rather than indefinite.

First, generic possessors are coreferent with the generic argument indexed on the verb, by the prefix \forme{kɯ-} for intransitive subject and object and the inverse prefix \forme{wɣɯ-} for transitive subject (§\ref{sec:indexation.generic.tr}): there can only be one generic argument in a given clause (see also \ref{ex:tWZo.kW.tWZo}, §\ref{sec:genr.pro}).

For instance, in (\ref{ex:genr.tWmtChi}), we know that the \forme{tɯ-} prefixes in \forme{tɯ-mtɕʰi} `one's mouth' and \forme{tɯ-ɕɣa} `one's teeth' are generic and not indefinite possessors because they refer to the same generic human as the transitive subject of the verbs \japhug{pʰɯt}{take out, cut} and \japhug{ndza}{eat} in the previous clause, marked by the inverse prefix (§\ref{sec:indexation.generic.tr}).

\begin{exe}
\ex \label{ex:genr.tWmtChi}
\gll tɕe ɲɯ́-wɣ-pʰɯt tɕe tú-wɣ-ndza ŋgrɤl ri, wuma ʑo tɯ-mtɕʰi cʰo tɯ-ɕɣa ra ɲɯ-sɯɣ-ɲaʁ ŋu. \\
\textsc{lnk} \textsc{ipfv}-\textsc{inv}-take.out \textsc{lnk} \textsc{ipfv}-\textsc{inv}-eat be.usually.the.case:\textsc{fact} but really \textsc{emph} \textsc{genr}.\textsc{poss}-mouth \textsc{comit} \textsc{genr}.\textsc{poss}-tooth \textsc{pl} \textsc{ipfv}-\textsc{caus}-be.black be:\textsc{fact} \\
\glt `One can pluck it and eat it, but it causes one's mouth and teeth to become black.' (11-qarGW) 
\japhdoi{0003480\#S58}
\end{exe}

Second, the generic possessor prefix appears on alienably possessed nouns, as in example (\ref{ex:tWlaXtCha}) with \forme{tɯ-kʰa} `one's house' and \forme{tɯ-laχtɕʰa} `one's things', the generic forms of \japhug{kʰa}{house} and \japhug{laχtɕʰa}{thing}.

\begin{exe}
\ex \label{ex:tWlaXtCha}
\gll tɕe aʁɤndɯndɤt ʑo ku-zo qʰe ɯ-qe ku-lɤt qʰe wuma ʑo tɯ-kʰa cʰo tɯ-laχtɕʰa ra sɯ-ɴqʰi. \\
\textsc{lnk} everywhere \textsc{emph} \textsc{ipfv}-land \textsc{lnk} \textsc{3sg}.\textsc{poss}-feces \textsc{ipfv}-throw \textsc{lnk} really \textsc{emph} \textsc{genr}.\textsc{poss}-house \textsc{comit} \textsc{genr}.\textsc{poss}-thing \textsc{pl} \textsc{caus}-be.dirty:\textsc{fact} \\
\glt `(Flies) land everywhere, shit and make one's houses and things dirty.' (25 akWzgumba)
\japhdoi{0003632\#S56}
\end{exe}

Third, the generic \forme{tɯ-} occurs on inalienably possessed nouns that normally select the \forme{tɤ-} indefinite possessor prefix, such as \forme{tɯ-rɟit} `one's child' and \forme{tɯ-rpɯ} `one's maternal uncle' in examples (\ref{ex:tWrJit}) and (\ref{ex:tWrpW}), by contrast with the citation forms \japhug{tɤ-rɟit}{child} and \japhug{tɤ-rpɯ}{mother's brother}.

\begin{exe}
\ex \label{ex:tWrJit}
\gll nɯ kɯ-fse tɕe tɯʑo tɯ-rɟit kɯnɤ ʑa mɤ-sci tu-ti-nɯ \\
\textsc{dem} \textsc{sbj}:\textsc{pcp}-be.like \textsc{lnk} \textsc{genr} \textsc{genr}.\textsc{poss}-child also early \textsc{neg}-\textsc{fact}:be.born \textsc{ipfv}-say-\textsc{pl} \\
\glt `People say that in this way, one's child will be born late.' (27 qartshaz)
\japhdoi{0003702\#S109}
\end{exe}

\begin{exe}
\ex \label{ex:tWrpW}
\gll tɯ-rpɯ ɯ-rɟit ɯ-ɕki tɕe tɕe ``a-rpɯ a-ɬaʁ" tu-kɯ-ti ŋu. \\
\textsc{genr}.\textsc{poss}-MB \textsc{3sg}.\textsc{poss}-offspring \textsc{3sg}-\textsc{dat} \textsc{lnk} \textsc{lnk} \textsc{1sg}.\textsc{poss}-MB \textsc{1sg}.\textsc{poss}-MZ \textsc{ipfv}-\textsc{genr}-say be:\textsc{fact} \\
\glt `One has to say `my maternal uncle, my maternal aunt to one's maternal uncle's sons and daughters.' (140425 kWmdza01)
\end{exe}

The use of the generic possessive \forme{tɯ-rpɯ} `one's maternal uncle' in (\ref{ex:tWrpW}) can be contrasted with the indefinite possessed form with \forme{tɤ-} in example (\ref{ex:tArpW}).

\begin{exe}
\ex \label{ex:tArpW}
\gll nɤʑo tɤ-rpɯ ɯ-rɟit a-pɯ-tɯ-ŋu, tɕe tɕe aʑo kɯ `a-rpɯ' tu-ti-a kɯ-ra. \\
\textsc{2sg} \textsc{indef}.\textsc{poss}-uncle \textsc{3sg}.\textsc{poss}-offspring \textsc{irr}-\textsc{ipfv}-2-be \textsc{lnk} \textsc{lnk} \textsc{1sg} \textsc{erg} \textsc{1sg}.\textsc{poss}-uncle \textsc{ipfv}-say-\textsc{1sg} \textsc{sbj}:\textsc{pcp}-be.needed \\
\glt `If you are the maternal uncle's son, (and I am the nephew) I have to address [you] as `my uncle'.' (140425 kWmdza)
\end{exe}

Fourth, in the case of inalienably possessed nouns whose indefinite possessive is \forme{tɯ\trt}, such as \japhug{tɯ-mtɕʰi}{mouth}, the indefinite and generic forms are homophonous, but are nevertheless distinguishable. In the case of a generic form the generic pronoun \japhug{tɯʑo}{one} (§\ref{sec:genr.pro}) can always be added as in (\ref{ex:tWZo.tWmtChi}). 

\begin{exe}
\ex \label{ex:tWZo.tWmtChi}
\gll tɯʑo sɤz kɯ-mna, kɯ-ɤʑɯχtso ra a-pɯ-ŋu nɤ, tɯʑo tɯ-mtɕʰi maŋtaʁ nɯtɕu ɲɯ-ɬoʁ ŋu. \\
\textsc{genr} \textsc{comp} \textsc{sbj}:\textsc{pcp}-be.better \textsc{sbj}:\textsc{pcp}-be.clean \textsc{pl} \textsc{irr}-\textsc{ipfv}-be \textsc{lnk} \textsc{genr} \textsc{genr}.\textsc{poss}-mouth upper.side \textsc{dem}:\textsc{pl} \textsc{ipfv}-come.out be:\textsc{fact} \\
\glt `If [one gets this disease after using the bowl of] someone who is cleaner than oneself, the [pimple] will appear on one's upper lip.' (25-khArWm)
\japhdoi{0003644\#S9}
\end{exe}

Additionally, a generic noun such as \japhug{tɯrme}{person} can occur as possessor of a noun with a generic possessive prefix as in (\ref{ex:tWrme.tWCa}). This usage is similar to that found in other generic constructions.

\begin{exe}
\ex \label{ex:tWrme.tWCa}
\gll tɯrme ɣɯ tɯ-ɕa ɯ-mdoʁ tsa asɯ-ndo kɯ-fse \\
people \textsc{gen} \textsc{genr}.\textsc{poss}-flesh \textsc{3sg}.\textsc{poss}-colour a.little \textsc{prog}-take:\textsc{fact} \textsc{sbj}:\textsc{pcp}-be.like \\
\glt `It has a little the colour of human flesh.' (14-sWNgWJu)
\japhdoi{0003506\#S91}
\end{exe}

Fifth, possessed case markers such as the dative \forme{ɯ-ɕki} (§\ref{sec:dative} ) do not have indefinite possessive forms, and therefore if prefixed in \forme{tɯ\trt}, it will always mark a generic possessor, as in (\ref{ex:tWCki}) -- such forms are often preceded by the generic pronoun \japhug{tɯʑo}{one} anyway.

\begin{exe}
\ex \label{ex:tWCki}
\gll ma tɯ-ɕki wuma ʑo ʑɣɤ-sɯ-ɤrmbat tɕe núndʐa kʰe tu-ti-nɯ ɲɯ-ŋu. \\
\textsc{lnk} \textsc{genr}-\textsc{dat} really \textsc{emph} \textsc{refl}-\textsc{caus}-be.near:\textsc{fact} \textsc{lnk} for.this.reason stupid:\textsc{fact} \textsc{ipfv}-say-\textsc{pl} \textsc{sens}-be \\
\glt `It comes near humans a lot, so people call it `stupid'.' (23-scuz) 
\japhdoi{0003612\#S59}
\end{exe}

\subsubsection{The generic possessor as a first person marker} \label{sec:generic.tW.1sg}
As in the generic verbal forms (§\ref{sec:1.genr}, §\ref{sec:genr.3pl}), the generic possessive prefixes can be used as an indirect way to express first person singular or plural. In example (\ref{ex:tWrkW.mWjrAZinW}) the generic as first person and the first person are used in two contiguous clauses, both referring to the narrator.\footnote{On the phrase \forme{tɯ-mu tɯ-wa} in (\ref{ex:tWrkW.mWjrAZinW}), see §\ref{sec:dyads}. }

\begin{exe}
\ex \label{ex:tWrkW.mWjrAZinW}
\gll tɕe tɯ-mu tɯ-wa ra tɯ-rkɯ mɯ́j-rɤʑi-nɯ tɕe, aʑo a-wi ci pɯ-tu. \\
\textsc{lnk} \textsc{genr}.\textsc{poss}-mother \textsc{genr}.\textsc{poss}-father \textsc{ra} \textsc{neg}:\textsc{sens}-stay-\textsc{pl} \textsc{lnk} \textsc{1sg} \textsc{1sg}.\textsc{poss}-grandmother \textsc{indef} \textsc{pst}.\textsc{ipfv}-exist \\ 
\glt `My parents were not by my side, but I had a grandmother [to take care of me].' (2010-09)
\end{exe}

\subsubsection{Comparative perspectives} \label{sec:indef.t.comparative}
Indefinite and generic possessive dental stop prefixes are found in all Gyalrong languages \citep{jackson98morphology}, but only indirect traces thereof exist in Khroskyabs \citep[155]{lai17khroskyabs}. 

Outside of Gyalrongic, potential cognates of these prefixes include the `relational prefix' \forme{tə-} in Ao (\citealt[84--85]{coupe07mongsen}, as first noticed by \citealt[141--142]{wolfenden29outlines}) and some \forme{d-} or \forme{g-} prefixes in body parts in Tibetan (see \citealt{jacques14snom}).
 
\subsection{Prenominal modifiers} \label{sec:possessive.prefixes.prenominal}
When prenominal modifiers occur with inalienably possessed noun, this head noun can either take a \textsc{3sg} possessive prefix, or undergo alienabilization (§\ref{sec:alienabilization}). 

Thus in (\ref{ex:XsAr.tAsno}) we find \forme{χsɤr tɤ-sno} `golden saddle' with the indefinite possessor prefix \forme{tɤ-}; \forme{χsɤr ɯ-sno} with the \textsc{3sg} possessive prefix is also attested in the same text. Neutralization of possessors of alienably possessed nouns is also attested in relative clauses (§\ref{sec:relative.possessor.neutralization}).

\begin{exe}
\ex \label{ex:XsAr.tAsno}
\gll χsɤr tɤ-sno tʰa-nɯ-ta ɲɯ-ŋu, \\
gold \textsc{indef}.\textsc{poss}-saddle \textsc{aor}:3\flobv{}-put \textsc{sens}-be \\
\glt `He harnessed the golden saddle (on the horse).' (2003qachga)
\end{exe}

However, when the whole modifier+head noun complex is possessed, the possessor is rarely marked by a possessive prefix on the head noun; rather, the prefix occurs on the leftmost noun of the phrase, as in (\ref{ex:aXsAr.tArte}), where the \textsc{1sg} prefix \forme{a-} occurs on the modifier \japhug{χsɤr}{gold}, and alienabilization of the head noun \japhug{tɤ-rte}{hat} (compare with the form \forme{a-rte} `my hat' when no prenominal modifier is present). %A noun phrase such as $\dagger$\forme{χsɤr a-rte} is incorrect.

\begin{exe}
\ex \label{ex:aXsAr.tArte}
\gll a-rte, a-χsɤr tɤ-rte ra kɯnɤ nɤʑɯɣ ɲɯ-kʰam-a jɤɣ \\
\textsc{1sg}.\textsc{poss}-hat \textsc{1sg}.\textsc{poss}-gold \textsc{indef}.\textsc{poss}-hat \textsc{pl} also \textsc{2sg}:\textsc{gen} \textsc{ipfv}-give[III]-\textsc{1sg} be.possible:\textsc{fact} \\
\glt `I will even give you my hat, my golden hat.' (140429 qingwa wangzi-zh)
\japhdoi{0003890\#S53}
\end{exe}

When the head noun is an alienably possessed noun, it is not usual either to strand the modifier and the following noun by putting a possessive prefix on the latter. The possessor is normally indicated by a possessive prefix on the leftmost word. For instance in (\ref{ex:nAXsAr.khWtsa}) the \textsc{2sg} prefix \forme{nɤ-} occurs on the modifier \japhug{χsɤr}{gold}.

\begin{exe}
\ex \label{ex:nAXsAr.khWtsa}
\gll nɤ-χsɤr kʰɯtsa nɯra ku-kɯ-sɯ-ntɕʰoz-a \\
\textsc{2sg}.\textsc{poss}-gold bowl \textsc{dem}:\textsc{pl} \textsc{ipfv}-2\fl{}1-\textsc{caus}-use-\textsc{1sg} \\
\glt `Let me use your golden bowl.' (140429 qingwa wangzi-zh) \japhdoi{0003890\#S132}
\end{exe}

However, we do find cases with a stranded NP modifier when the possessor on the head noun is first or second person, as in (\ref{ex:XsAr.akWmtChW}), where the prenominal modifier \japhug{χsɤr}{gold} does appear before the possessive prefix \forme{a-}.\footnote{The alternative form \forme{a-χsɤr kɯmtɕʰɯ} is possible to express the same meaning, and does occur in the same text. } This construction, though rarer, is considered to be acceptable by native speakers. Other examples are found with unpossessible modifiers (§\ref{sec:place.names}).

\begin{exe}
\ex \label{ex:XsAr.akWmtChW}
\gll nɯnɯ a-kɯmtɕʰɯ, χsɤr a-kɯmtɕʰɯ nɯnɯ kɤ-ɣɯt a-pɯ-tɯ-cʰa qʰe, \\
\textsc{dem} \textsc{1sg}.\textsc{poss}-toy gold \textsc{1sg}.\textsc{poss}-toy \textsc{dem} \textsc{inf}-bring \textsc{irr}-\textsc{ipfv}-2-can \textsc{lnk} \\
\glt `If you can bring my toy, my golden toy back...' (140429 qingwa wangzi-zh)
\japhdoi{0003890\#S51}
\end{exe}

Note that unlike the inalienably and alienably possessed noun modifiers discussed above, \textit{pronominal} prenominal modifiers (§\ref{sec:other.pro}) do not take possessive prefixes that have scope over the head noun. For instance, with the modifier \japhug{kɯmaʁ}{other}, the possessive prefix must appear on the following noun, as second person \forme{nɤ-} on \japhug{slama}{student} in (\ref{ex:kWmaR.nAslama}).

\begin{exe}
\ex \label{ex:kWmaR.nAslama}
 \gll kɯmaʁ nɤ-slama ci tu-tɯ-ndɤm ju-tɯ-ɣɯt ɯ́-ŋu \\
 other \textsc{2sg}.\textsc{poss}-student \textsc{indef} \textsc{ipfv}-2-take[III] \textsc{ipfv}-2-bring \textsc{qu}-be:\textsc{fact} \\
 \glt `So you are bringing other students of yours?' (conversation)
\end{exe}

 

\section{Unpossessible nouns} \label{sec:unpossessible.nouns}
In addition to inalienably and alienably possessed nouns seen in the previous sections, Japhug also has a category of unpossessible nouns, which includes names of places and ethnic groups (as in Koyukon Athabaskan, \citealt[651]{thompson96koyukon}), colour terms of Tibetan origin and some derived nouns like the `social relation collectives' (§\ref{sec:social.collective}). With the exception of colour terms (§\ref{sec:tibetan.colours}), these nouns can modify other nouns and are one of the three classes of `property words' (corresponding to the adjectives of Standard Average European), alongside adjectival stative verbs (§\ref{sec:parts.speech.introduction}) and property nouns (§\ref{sec:property.nouns}).

 
%schluecker17proper

\subsection{Place names} \label{sec:place.names}
Place names (\japhug{mbarkʰom}{Mbarkham}, \japhug{kɤmɲɯ}{Kamnyu} etc) and names of ethnic groups (such as \japhug{kɯrɯ}{Tibetan} or \japhug{kupa}{Chinese}), like personal names, cannot take possessive prefixes when used independently. They can only be used with independent pronouns as in (\ref{ex:jiphe.kAmYW}) or (\ref{ex:iZo.KamnYW}) below.
\largerpage
\begin{exe}
\ex \label{ex:jiphe.kAmYW}
\gll  iʑora ji-pʰe kɤmɲɯ nɯtɕu <xiaoxue> <yinianji> <ernianji> pɯ-ndɯn-a. \\
\textsc{1pl} \textsc{1pl}.\textsc{poss}-\textsc{dat} \textsc{topo} \textsc{dem}:\textsc{loc} primary.school first.grade second.grade \textsc{aor}-read-\textsc{1sg} \\
\glt `I studied the first and second grade of primary school at our place in Kamnyu.' (140501 tshering skyid) \japhdoi{0003902\#S12}
\end{exe}

These types of nouns can serve as strictly prenominal modifiers (as in \japhug{kɯrɯ sɤtɕʰa}{Tibetan areas}) and commonly occur as first member of nominal compounds (as in \japhug{kɯrɯɕɤmɯɣdɯ}{traditional gun}, with \japhug{ɕɤmɯɣdɯ}{gun} as second element, see §\ref{sec:determinative.n.n}). Although these nouns are unpossessible by themselves, when used as first members of a compound, or even as prenominal modifiers (on which see §\ref{sec:possessive.prefixes.prenominal}), they can take a possessive prefix which has scope over the head noun, as in examples (\ref{ex:jikWrWlAsAr}) and (\ref{ex:jikAmYWskAt}), where the \textsc{1pl} possessive prefix \forme{ji-} occurs prefixed on the name \japhug{kɯrɯ}{Tibetan} and on the place name \japhug{kɤmɲɯ}{Kamnyu}.

\begin{exe}
\ex \label{ex:jikWrWlAsAr}
 \gll nɯtɕu tɕe iʑora ji-kɯrɯ-lɤsɤr ŋu \\
 \textsc{dem}:\textsc{loc} \textsc{lnk} \textsc{1pl} \textsc{1pl}.\textsc{poss}-Tibetan-new.year be:\textsc{fact} \\
 \glt `At that time, it is our Tibetan new year.' (conversation)
\end{exe}

\begin{exe}
\ex \label{ex:jikAmYWskAt}
 \gll nɤʑo ji-kɤmɲɯ-skɤt nɯnɯ <quanshijie> ʑo ju-tɯ-sɯ-ɤzɣɯt ŋu \\
 \textsc{2sg} \textsc{1pl}.\textsc{poss}-pl.n.-language \textsc{dem} whole.world \textsc{emph} \textsc{ipfv}-2-\textsc{caus}-reach be:\textsc{fact} \\
 \glt `You are spreading our Kamnyu language to the whole world.' (conversation)
\end{exe}

The pair of examples in (\ref{ex:nAkWrWrmi}) and (\ref{ex:faguo.nArmi}) illustrates the different behaviour of unpossessible nouns as first elements of compounds on the one hand, and as noun modifiers on the other. In (\ref{ex:nAkWrWrmi}), \japhug{kɯrɯ-rmi}{Tibetan name} constitutes a single compound noun (from \japhug{kɯrɯ}{Tibetan} and \japhug{tɤ-rmi}{name}; the phrase \forme{kɯrɯ ɯ-rmi} is also possible). 

\begin{exe}
 	\ex \label{ex:nAkWrWrmi}
		\gll nɤ-kɯrɯ-rmi  \\
		\textsc{2sg}:\textsc{poss}-Tibetan-name \\
		\glt `Your Tibetan name.' 
		\ex \label{ex:nAkWrWrmi2}
		\gll $\dagger$kɯrɯ nɤ-rmi  \\
		Tibetan \textsc{2sg}:\textsc{poss}-name \\		
\end{exe}

The possessive prefix occurs before \forme{kɯrɯ}. Stranding the modifier as in (\ref{ex:nAkWrWrmi2}) is considered to be agrammatical by native speakers.
 
\begin{exe}
	\ex \label{ex:faguo.nArmi}
	\gll nɤʑɯɣ <faguo> nɤ-rmi  \\
	\textsc{2sg}:\textsc{gen} France \textsc{2sg}.\textsc{poss}-name \\
	\glt `Your French name.' 
\end{exe}
 
 
In (\ref{ex:faguo.nArmi}) however, the modifier \forme{faguo} `France, French' (from Chinese) cannot be compounded with \japhug{tɤ-rmi}{name} and cannot take possessive prefixes. This is a rare example where a noun modifier can be stranded from the stem of the head noun by a definite possessor prefix (§\ref{sec:possessive.prefixes.prenominal}).

 
Place names followed by the plural \forme{ra} designate the people living in the place (\ref{ex:iZo.KamnYW}), even without \forme{-pɯ} suffixation (§\ref{ex:gentilic.pW}). Example (\ref{ex:iZo.KamnYW}) also shows that in this usage, it is possible to use a personal pronoun in apposition as in \forme{iʑo kɤmɲɯ ra} `we Kamnyu people'.

\begin{exe}
\ex \label{ex:iZo.KamnYW}
 \gll iʑo kɤmɲɯ ra kɯ tɕʰɯχpri tu-ti-j ŋu. rcaqo ra cʰo mɤŋi ra kɯ tɕʰɯχpɯχpri tu-ti-nɯ ŋu \\
 \textsc{1pl} \textsc{topo} \textsc{pl} \textsc{erg} salamander \textsc{ipfv}-say-\textsc{1pl} be:\textsc{fact} \textsc{topo} \textsc{pl} \textsc{comit} \textsc{topo} \textsc{pl} \textsc{erg} salamander \textsc{ipfv}-say-\textsc{pl} be:\textsc{fact} \\
 \glt `We Kamnyu people call it \forme{tɕʰɯχpri}, and people from Rqakyo and Mangi call it \forme{tɕʰɯχpɯχpri}.' (25-tChWXpri)
\japhdoi{0003662\#S20}
\end{exe}

Place names can take some prenominal modifiers such as \japhug{pʰa}{whole} as in (\ref{ex:pha.RdWrJAt}), but no example of bare place names with prenominal demonstratives have been found. 
% à revérifier

\begin{exe}
\ex \label{ex:pha.RdWrJAt}
 \gll pʰa ʁdɯrɟɤt nɯ ɯ-ŋgɯ tɕe rqaco cʰo katɕa nɯ stu ɣɤndʐo \\
 whole \textsc{topo} \textsc{dem} \textsc{3sg}.\textsc{poss}-inside \textsc{lnk} \textsc{topo} \textsc{comit} \textsc{topo} \textsc{dem} most cold:\textsc{fact} \\
 \glt `In the whole of Gdongbrgyad, Rqakyo and Kacha are the coldest.' (140522 RdWrJAt)
\japhdoi{0004061\#S11}
\end{exe}

Place names and ethnic names can be used as core arguments, or nominal predicates with a copula, as in (\ref{ex:kupa.Nu}) and (\ref{ex:taRdo.Nu}).

\begin{exe}
\ex \label{ex:kupa.Nu}
\gll a-wa nɯnɯ kupa ŋu \\
\textsc{1sg}.\textsc{poss}-father \textsc{dem} Chinese be:\textsc{fact} \\
\glt `My father is Chinese.' (140501 tshering skyid) \japhdoi{0003902\#S4}
\end{exe}

\begin{exe}
\ex \label{ex:taRdo.Nu}
\gll tɕe ʁnɯ-tɯpɯ nɯnɯ taʁrdo ŋu \\
\textsc{lnk} two-household \textsc{dem} \textsc{topo} be:\textsc{fact} \\
\glt `These two households are Taqrdo.' 
\japhdoi{0004053\#S20}
\end{exe}

Like locative relator nouns (§\ref{sec:relator.location}), bare place names can be used without postposition to express motion (\ref{ex:mbarkhOm.thWwGGWta}) or static location (\ref{ex:kAmYW.GJW}), but are also found with locative postpositions, most often \forme{ri} as in (\ref{ex:tshuBdWn.ri}) but also \forme{tɕu} or \forme{zɯ} (as in \ref{ex:jiphe.kAmYW} above).

\begin{exe}
\ex \label{ex:mbarkhOm.thWwGGWta}
 \gll a-pi kɯ tɤ́-wɣ-ndo-a tɕe tɕe, mbarkʰom tʰɯ́-wɣ-ɣɯt-a, \\
 \textsc{1sg}.\textsc{poss}-elder.sibling \textsc{erg} \textsc{aor}-\textsc{inv}-take-\textsc{1sg} \textsc{lnk} \textsc{lnk} \textsc{topo}  \textsc{aor}:\textsc{downstream}-\textsc{inv}-bring-\textsc{1sg} \\
\glt `My elder brother brought me to Mbarkham.' (140501 tshering skyid) \japhdoi{0003902\#S27}
\end{exe}

\begin{exe}
\ex \label{ex:kAmYW.GJW}
 \gll kɯɕɯŋgɯ tɕe kɤmɲɯ ɣɟɯ kɯɕnɯz pjɤ-tu. \\
 former.days \textsc{lnk} \textsc{topo} watchtower seven \textsc{ifr}.\textsc{ipfv}-exist \\
 \glt `In former times, there were seven watchtowers in Kamnyu.' (140522 GJW)
\japhdoi{0004051\#S1}
\end{exe}

\begin{exe}
\ex \label{ex:tshuBdWn.ri}
 \gll tɕe alo tsʰuβdɯn ri pɯ-rɤʑi-j tɕe \\
 \textsc{lnk} upstream \textsc{topo} \textsc{loc} \textsc{pst}.\textsc{ipfv}-stay-\textsc{1sg} \textsc{lnk} \\
 \glt `We were living up there in Tshobdun.' (28-kWpAz) \japhdoi{0003714\#S164}
\end{exe}

Toponyms hardly ever occur as transitive subjects with the ergative postposition (§\ref{sec:A.kW}). The only example in the corpus is (\ref{ex:qaprANar.kW}), in the context of a mythological story associated with a cliff called \forme{qaprɤŋar} in Kamnyu village.

\begin{exe}
\ex \label{ex:qaprANar.kW}
\gll tɕe tɯ-mɯ lɤt tɤkʰa tɕe, qaprɤŋar ɯ-stu ri ɲɯ-kɯ-ru tɕe zdɯm ci tu-nɯ-ɬoʁ ŋu. tʰam kɯnɤ zdɯm tu-nɯ-ɬoʁ ŋu tɕe, nɯ maka qaprɤŋar kɯ zdɯm to-tɕɤt ra tu-ti-nɯ \\
\textsc{lnk} \textsc{indef}.\textsc{poss}-weather release:\textsc{fact} at.the.time \textsc{lnk} placename \textsc{3sg}.\textsc{poss}-direction \textsc{loc} \textsc{ipfv}:\textsc{west}-\textsc{genr}:S/O-look \textsc{lnk} cloud \textsc{indef} \textsc{ipfv}:\textsc{up}-\textsc{auto}-come.out be:\textsc{fact} now also cloud \textsc{ipfv}:\textsc{up}-\textsc{auto}-come.out be:\textsc{fact} \textsc{lnk} \textsc{dem} at.all placename \textsc{erg} cloud \textsc{ifr}-take.out \textsc{pl} \textsc{ipfv}-say-\textsc{pl} \\
\glt `And when it is about to rain, when one looks towards Qaprangar, a cloud comes out from it. Even now a cloud comes out, and [the elders] say `Qaprangar released a cloud.'' (140522 Kamnyu zgo) \japhdoi{0004059\#S317}
\end{exe}

\subsection{Colour nouns} \label{sec:tibetan.colours}
Colour names of Tibetan origin, such as \japhug{ldʑaŋkɯ}{green} from \tibet{ལྗང་གུ་}{ldʑaŋ.gu}{green}, \japhug{ʁmɤrsmɯɣ}{dark red} from \tibet{དམར་སྨུག་}{dmar.smug}{dark red} or \japhug{kʰatoʁ}{variegated} from \tibet{ཁ་དོག་}{kʰa.dog}{colour, multicolour} designate objects or animals with a particular colour. To serve as predicates, they need an existential verb (\ref{ex:khatoR.Zo.tu}), like participles of adjectival stative verbs of colour (\ref{ex:ldZaNkW}). 

\begin{exe}
\ex \label{ex:khatoR.Zo.tu}
 \gll ɯ-muj nɯra wuma ʑo mpɕɤr, kʰatoʁ ʑo tu. \\
 \textsc{3sg}.\textsc{poss}-feather \textsc{dem}:\textsc{pl} really \textsc{emph} be.beautiful variegated \textsc{emph} exist:\textsc{fact} \\
 \glt `Its feathers are very beautiful and variegated.' (24-kWmu)
\japhdoi{0003618\#S64}
\end{exe}


\begin{exe}
\ex \label{ex:ldZaNkW}
 \gll qambalɯla rcanɯ ɯ-mdoʁ ʑakastaka ʑo kɯ-ŋu tu. .... kɯ-qarŋe tu, ldʑaŋkɯ tu, kɯ-ɤrŋi tu, kɯ-ɲaʁ tu. \\
 butterfly \textsc{unexp}:\textsc{deg} \textsc{3sg}.\textsc{poss}-colour each \textsc{emph} \textsc{sbj}:\textsc{pcp}-be exist:\textsc{fact} .... \textsc{sbj}:\textsc{pcp}-be.yellow exist:\textsc{fact} blue/green exist:\textsc{fact} \textsc{sbj}:\textsc{pcp}-be.green exist:\textsc{fact} \textsc{sbj}:\textsc{pcp}-be.black exist:\textsc{fact}\\
 \glt `There are butterflies with all kinds of colours, yellow, green, blue/green, black. (26-qambalWla)
\japhdoi{0003680\#S2}
\end{exe}

These nouns are only very rarely used as postnominal modifiers (§\ref{ex:attributive.postnominal}); (\ref{ex:tAri.khatoR}) is such an example.

\begin{exe}
\ex \label{ex:tAri.khatoR}
 \gll tɤ-ri kʰatoʁ nɯ kɯ tʰɯ-kɤ-sɯ-βzu nɯ snalŋaɕtʰɤβ tu-kɯ-ti \\
 \textsc{indef}.\textsc{poss}-thread variegated \textsc{dem} \textsc{erg} \textsc{aor}-\textsc{obj}:\textsc{pcp}-\textsc{caus}-make \textsc{dem} multicolour.lace \textsc{ipfv}-\textsc{genr}-say \\
 \glt `[The laces] that are made of multicoloured thread are called \forme{snalŋaɕtʰɤβ}.' (30-rkAsnom) \japhdoi{0003754\#S32}
\end{exe}

The adjectival stative verbs in \forme{arɯ-} derived from them (for instance \japhug{arɯldʑaŋkɯ}{be green}, see §\ref{sec:denom.arW}) are as common as the colour nouns, and their participles are generally used as noun modifiers instead of the colour nouns.


\subsection{Other unpossessible nouns}  \label{sec:other.upn}
Unpossessible nouns other than proper names and colour terms include nouns occurring as postnominal modifiers like \japhug{tɯlɤt}{second sibling}\footnote{In \japhug{tɯlɤt}{second sibling} the \forme{tɯ-} element in this word is originally an indefinite possessor prefix, but has become lexicalized.} as in (\ref{ex:tWlAt}), and privative nouns in \forme{-lu} described in (§\ref{sec:privative}).

\begin{exe}
\ex \label{ex:tWlAt}
\gll nɯ-me tɯlɤt nɯ ɲɤ-mbi-nɯ \\
\textsc{3pl}.\textsc{poss}-daughter second.sibling \textsc{dem} \textsc{ifr}-give-\textsc{pl} \\
\glt `They gave him their second daughter.' (2002 qaCpa)
\end{exe} 

Numerals under 99 are also unable to take possessive prefixes and serve as postnominal modifiers (§\ref{sec:one.to.ten}), and can be considered to be a subclass of unpossible nouns.

\section{Personal names} \label{sec:personal.names}
This section focuses on three topics: the absence of vocative forms, the Tibetan origin of personal names, and their use with pronouns and possessive prefixes. 

\subsection{Vocative} \label{sec:vocative}
Unlike other Gyalrong languages, Japhug does not have specific vocative forms for personal names and kinship terms. In Tshobdun, \citet[133]{jackson98morphology} and \citet[53]{jackson05yingao} reports that personal names in the vocative have stress retraction. The same is found in Khroskyabs \citep[153]{lai17khroskyabs}. In Situ, inalienably possessed nouns have their possessive prefixes replaced by \forme{a-} in vocative forms (\citealt[471]{nagano03cogtse}, \citealt[177]{prins16kyomkyo}). 

In Japhug, due to the almost complete loss of contrastive stress (§\ref{sec:stress}) and the fact that the \textsc{1sg} possessive prefix has the form \forme{a-} (§\ref{sec:possessive.paradigm}) unlike in Tshobdun and Situ (where it is \forme{ŋa-/ŋə-}), there is no specific vocative form for either personal names or kinship terms. 

A prefix \forme{a-} does occur in the familiar form of personal names (reminding of Lin's \citeyear[162]{linxr93jiarong} description of this prefix as a \zh{爱称} `pet name' marker), but not exclusively in vocative use as in example (\ref{ex:kWlAGacAB.nW.kW}) where we see the name \forme{acɤβ} as transitive subject, familiar form of a Tibetan name with \forme{scɤβ} as second element (see §\ref{sec:names.tibet} below).

 \begin{exe}
\ex \label{ex:kWlAGacAB.nW.kW}
\gll kɯ-lɤɣ acɤβ nɯ kɯ, ɯ-pʰɯŋgɯ nɯtɕu qapɯtɯm ci na-rku ɲɯ-ŋu, \\
\textsc{sbj}:\textsc{pcp}-graze Askyabs \textsc{dem} \textsc{erg} \textsc{3sg}.\textsc{poss}-fold.of.clothes \textsc{dem}:\textsc{loc} pebble.from.flint \textsc{indef} \textsc{aor}:3\flobv{}-put.in \textsc{sens}-be \\
\glt `The shepherd Askyabs put a pebble in the folds of his clothes (to avoid forgetting what he had to told the king).' (Kunbzang 332)
\end{exe}

Since similar \forme{a-} prefixes exist in Tibetan and Chinese, and since personal names are exclusively borrowed from one of these languages (there are no clear remnants of native personal names in Japhug), it is likely that the familiar form of the names was also borrowed.

\subsection{Tibetan names} \label{sec:names.tibet}
Speakers of Japhug generally have Tibetan names (\forme{kɯrɯ ɯ-rmi} or \forme{kɯrɯ-rmi}, §\ref{sec:place.names}), and in addition a Chinese official name which may or may not be related to the Tibetan one (see \ref{ex:nAkWrWrmi} §\ref{sec:place.names} on the use of ethnic or countries names as prenominal modifiers with the inalienably possessed noun \japhug{tɤ-rmi}{name}). Buddhist or Bonpo monks are also given religious names (in Japhug \forme{χpɯn ɯ-rmi}, see \ref{ex:XpWn.Wrmi}). 
 
\begin{exe}
\ex \label{ex:XpWn.Wrmi}
\gll tɕe χpɯn ɯ-rmi nɯ, aʑo a-rmi nɯ stɤnbiɲima tɤ́-wɣ-sɤrmi-a-nɯ. \\
\textsc{lnk} monk \textsc{3sg}.\textsc{poss}-name \textsc{dem} \textsc{1sg} \textsc{1sg}.\textsc{poss}-name \textsc{dem} \textsc{anthr} \textsc{aor}-\textsc{inv}-give.name-\textsc{1sg}-\textsc{pl} \\
\glt `They gave me the name Bstanpa'i nyima as my monk name.' (160721 XpWN)
\japhdoi{0006181\#S36}
\end{exe}

In one traditional story, we find an example of person names based on Japhug words as in (\ref{ex:zrAntCW}), but it looks so strange that the narrator felt it necessary to specify that these are people's names.

\begin{exe}
\ex \label{ex:zrAntCW}
 \gll zrɤntɕɯ tɯrme ci pjɤ-tu, tɯpɕi kɯ-rmi ci pjɤ-tu, tɯrme nɯ-rmi ɲɯ-ŋu nɤ \\
mung.bean person \textsc{indef} \textsc{ifr}.\textsc{ipfv}-exist flax \textsc{sbj}:\textsc{pcp}-call \textsc{indef} \textsc{ifr}.\textsc{ipfv}-exist people \textsc{3pl}.\textsc{poss}-name \textsc{sens}-be \textsc{sfp} \\
\glt `There was [a lady] was was called `Mung bean', and [another one] called `Flax', these are names of people.' (zrAntCW)
\end{exe}

Names used by Japhug speakers are not markedly different from those found in other Tibetan areas. Lady names often include the suffixes \forme{ltɕɤm}, \forme{rcit} or \forme{mtsʰu}, (from \tibet{ལྕམ་}{ltɕam}{lady, sister}, \tibet{སྐྱིད་}{skʲid}{happy} and \tibet{མཚོ་}{mtsʰo}{lake}), and there are also non-gender specific suffixes like \forme{scɤβ} (from \tibet{སྐྱབས་}{skʲabs}{protector}, for instance \forme{tsʰɯraŋ scɤβ} from \tibet{ཚེ་རིང་སྐྱབས་}{tsʰe.riŋ.skʲabs}{p.n.}).). 

Many Tibetan names have alternative readings reflecting different reading traditions belonging to more than two layers (see §\ref{sec:historical.phono} and \citealt[83--200]{jacques04these} on the layers of Tibetan borrowings in Japhug). For instance, some people with the Tibetan name \tibet{འཕྲིན་ལས་}{ⁿpʰrin.las}{Karma} are called \forme{mpʰrɯlɤz} (with preservation of the coda), other \forme{mpʰrɯli} (with Amdo-type change to \forme{-i}). The names however tend to have non-Amdo phonological features even for people of the younger generation. For instance, the name \tibet{ཀུན་དགའ་}{kun.dga}{Ânanda} is pronounced \forme{kɯnga} without assimilation of the dental nasal to a velar nasal, and \tibet{ཀུན་བཟང་}{kun.bzaŋ}{Sarvabhadra} is \forme{kɯnɯβzaŋ} with an anaptyctic vowel (§\ref{sec:heterosyllabic.clusters}). 

%\tibet{བཀྲ་ཤིས་}{bkra.ɕis}{good fortune} appears as \forme{krɤɕiz} with preservation of the coda and part of the initial cluster, \forme{krɤɕi} with Amdo-like loss of final \forme{-s} and \forme{tʂaɕi} with cluster simplification. 

 

\subsection{Alienably possessed or unpossessible nouns?} \label{sec:personal.name.APN}
Personal names superficially look like unpossessible nouns, as they do not usually occur with possessive prefixes, even when taking placenames as modifiers, as in \forme{taʁrdo χpɤltɕin} `Dpalcan from Taqrdo' (see \ref{ex:taRrdo.dpalcan} in §\ref{sec:personal.names.modifiers})

Personal names commonly occur preceded by kinship terms which, being inalienably possessed nouns (§\ref{sec:kinship}), have a possessive prefix as in (\ref{ex:anmaR.dpalcan}). 

\begin{exe}
\ex \label{ex:anmaR.dpalcan}
\gll a-nmaʁ χpɤltɕin \\
\textsc{1sg}.\textsc{poss}-husband \textsc{anthr} \\
\glt `My husband Dpalcan.' (heard in context)
\end{exe}

It is considered impolite to address someone from an older generation than oneself without adding a kinship term -- for instance, the author of this grammar, being much younger, has to address the aforementioned Dpalcan as \forme{a-βɣo χpɤltɕin} with the \textsc{1sg} form of \japhug{tɤ-βɣo}{father's brother}.

Although personal names rarely occur with possessive prefixes, there is no grammatical constraint against it. There is one such example in the whole corpus, in a conversation where a clarification was needed. Tshendzin asks about Dpalcan, younger brother of Tshering Sgrolma, but she does not understand at once, because Tshendzin's husband is also called Dpalcan; thus Tshendzin says (\ref{ex:nWχpAltCin}) with the possessed form \forme{nɯ-χpɤltɕin} `your$_{pl}$ Dpalcan' to disambiguate between the two. 

\begin{exe}
\ex  
\begin{xlist}
\ex 
\gll χpɤltɕin kɯmaʁ kɯ-nɯhɯɲi mɯ-jo-ɕe ɯ́-ŋu. \\
\textsc{anthr} other \textsc{sbj}:\textsc{pcp}-do.work \textsc{neg}-\textsc{ifr}-go \textsc{qu}-be:\textsc{fact} \\
\\
\glt (Tshendzin): `Dpalcan did not go for another job, did he?'
\ex 
\gll ka? \\
\textsc{sfp} \\
\glt (Tshering Sgrolma): `What?'
\ex \label{ex:nWχpAltCin}
\gll χpɤltɕin, nɯʑo nɯ-χpɤltɕin nɯ \\
\textsc{anthr} \textsc{2pl} \textsc{2pl}.\textsc{poss}-\textsc{anthr} \textsc{dem} \\
\\
\glt (Tshendzin): `Dpalcan, your Dpalcan.'  (140510 tshering)
\end{xlist}
\end{exe}

Given the existence of such forms, personal names are treated as a subclass of alienably possessed nouns rather than as unpossessible nouns. Only plural forms (\forme{ji-χpɤltɕin} `our Dpalcan', \forme{ʑara nɯ-χpɤltɕin} `their Dpalcan' etc) are attested.

\subsection{Personal names and modifiers} \label{sec:personal.names.modifiers}
Proper nouns are more often than not used without demonstratives and determiners (see §\ref{sec:non.overt.definite}). However, examples of person or place names taking the postnominal distal determiners \forme{nɯ} or \forme{nɯnɯ} (§\ref{sec:demonstrative.determiners}) are not rare (\ref{ex:Yimawodzer.NW} and \ref{ex:XpAltCin.nWnW}).

\begin{exe}
\ex \label{ex:Yimawodzer.NW}
 \gll ɲimawozɤr nɯ kɯ, srɯnmɯ nɯ pjɤ-ftɯl, \\
 \textsc{anthr} \textsc{dem} \textsc{erg} râkshasî \textsc{dem} \textsc{ifr}-subdue \\
 \glt `Nyima 'Odzer subdued the râkshasî.' (2011-4-smanmi)
\end{exe}

\begin{exe}
	\ex \label{ex:XpAltCin.nWnW}
	\gll <dangshi> χpɤltɕin nɯnɯ snarndi lɤ-ari \\
	at.that.time \textsc{anthr} \textsc{dem} \textsc{topo} \textsc{aor}:\textsc{upstream}-go[II] \\
	\glt `At that time (the Wenchuan earthquake, in 2008), Dpalcan had gone to Snarndi.' (180420 waJW)
\end{exe}

It is possible to use a dual or a plural marker on a personal name to designate a group of people sharing the same name, without any associative plural meaning, as in (\ref{ex:Dpalcan.XsWm}).

\begin{exe}
\ex \label{ex:Dpalcan.XsWm}
 \gll a-pɯ-ŋu tɕe, χpɤltɕin ʁnɯz, nɯ maʁ nɤ χsɯm kɯ-fse kɯ-naχtɕɯɣ tɯtɯrca a-pɯ-rɤʑi-nɯ tɕe, `χpɤltɕin ni, χpɤltɕin ra" nɯra tu-kɯ-ti kʰɯ. \\
 \textsc{irr}-\textsc{ipfv}-be \textsc{lnk} \textsc{anthr} two \textsc{dem} not.be:\textsc{fact} \textsc{lnk} three \textsc{nmzl}:S/A-be.like \textsc{nmzl}:S/A-be.identical together \textsc{irr}-\textsc{ipfv}-stay-\textsc{pl} \textsc{lnk} \textsc{anthr} \textsc{du} \textsc{anthr} \textsc{pl} \textsc{dem}:\textsc{pl} \textsc{ipfv}-\textsc{genr}-say be.possible:\textsc{fact} \\ 
 \glt `For instance, if two or three [people called] Dpalcan live together, one can say `the two Dpalcans', `the Dpalcans'. (elicited)
\japhdoi{0006081\#S9}
\end{exe}

To distinguish between persons with the same name (a common occurrence among speakers of Japhug, given the relatively limited inventory of Tibetan names available), house names (\forme{kʰa ɯ-rmi}) are generally added as prenominal modifiers, as in (\ref{ex:taRrdo.dpalcan}).

\begin{exe}
\ex \label{ex:taRrdo.dpalcan}
\gll χpɤltɕin ɯ-kʰa nɯ taʁrdo rmi tɕe taʁrdo χpɤltɕin tu-kɯ-ti. \\
\textsc{anthr} \textsc{3sg}.\textsc{poss}-house \textsc{dem} \textsc{topo} be.called:\textsc{fact} \textsc{lnk} \textsc{topo} \textsc{anthr} \textsc{ipfv}-\textsc{genr}-say \\
\glt `Dpalcan's house is called Taqrdo, so one [can] call him `Taqrdo Dpalcan'.' (elicited)
\end{exe}

If two persons from the same household have the same name, locational modifiers (§\ref{sec:relative.location}) can be used instead, as illustrated in (\ref{ex:maNlo.dpalcan}).

\begin{exe}
\ex \label{ex:maNlo.dpalcan}
\gll nɯ maʁ nɤ, ndʑi-kʰa ɯ-rmi kɯnɤ a-pɯ-naχtɕɯɣ tɕe, kʰa kundi, lotʰi kɯ-fse nɯra tɕe, 
maŋlo χpɤltɕin, maŋtʰi χpɤltɕin, maŋkɯ χpɤltɕin, maŋndi χpɤltɕin, nɯra tu-kɯ-ti ŋgrɤl. \\
\textsc{dem} not.be:\textsc{fact} \textsc{lnk} \textsc{3du}.\textsc{poss}-house \textsc{3sg}.\textsc{poss}-name also \textsc{irr}-\textsc{ipfv}-be.identical \textsc{lnk} house east.west up.down.stream \textsc{sbj}:\textsc{pcp}-be.like \textsc{dem}:\textsc{pl} \textsc{lnk} upstream \textsc{anthr} downstream \textsc{anthr} east \textsc{anthr} west \textsc{anthr} \textsc{dem}:\textsc{pl} \textsc{ipfv}-\textsc{genr}-say be.usually.the.case:\textsc{fact} \\
\glt `Otherwise, if their house name is also the same, using the east-west or the upstream-downstream dimensions, one can say `Dpalcan from upstream, downstream, east or west.' (elicited)
\end{exe}

Like other nouns, personal names can also occur as head of non-restrictive relatives, as in (\ref{ex:mWtAkWrZaR}) and (\ref{ex:WnmaR.pWkWNu}), though such uses are rather uncommon. No examples of personal names as heads of head-internal relatives have been found.

 \begin{exe}
\ex \label{ex:mWtAkWrZaR}
\gll tɕendɤre iɕqʰa ʁlaŋsaŋtɕʰin [χsɯm ma mɯ-tɤ-kɯ-rʑaʁ] nɯ, \\
\textsc{lnk} the.aforementioned Gesar three apart.from \textsc{neg}-\textsc{aor}-\textsc{sbj}:\textsc{pcp}-pass.days \textsc{dem} \\
\glt `Gesar, who was only three days old,' (Gesar 81)
\end{exe}

\begin{exe}
\ex \label{ex:WnmaR.pWkWNu}
\gll [nɯ ɕɯŋgɯ ɯ-nmaʁ pɯ-kɯ-ŋu] tsʰɯraŋ nɯ pjɤ-mto \\
\textsc{dem} before \textsc{3sg}.\textsc{poss}-husband \textsc{pst}-\textsc{sbj}:\textsc{pcp}-be \textsc{anthr} \textsc{dem} \textsc{ifr}-see \\
\glt `See saw Tshering, who had been her husband before.' (2002qajdoskAt)
\japhdoi{0003366\#S101}
\end{exe}


 
\section{Bound state} \label{sec:status.constructus}
The term \textit{bound state} refers to the non-autonomous form of (mainly nominal, but also verbal and adverbial) roots occurring as non-final element of compounds.\footnote{The term  \textit{status constructus} or \textit{construct state} is also used in Gyalrongic linguistics (\citealt{jacques12incorp}, \citealt[163--164]{lai17khroskyabs}, \citealt{gates18harmony}) to describe to the same phenomenon, but I decided to follow the advice of a reviewer and to change it to avoid confusion with works such as \citet{creissels06hongrois}, \citet{creissels17construct} or \citet[30]{gutman18attributive} in which `construct form' refers to a specific form that is obligatory on the head noun in specific noun-modifier constructions (including with a possessive marker). }

In Gyalrongic languages including Japhug, nominal compounds generally exhibit modifier-head order. Thus, the form undergoing bound state alternation in Japhug is often the modifier noun,\footnote{In addition, in Japhug the possessed forms of nouns do not show morphological alternations (§\ref{sec:possessive.paradigm}) with the only exception of \japhug{qale}{wind} (§\ref{sec:apn.to.ipn}).} except in Noun-Verb compounds where the second element is an adjectival stative verb.

This section presents the various types of alternations attested for first or other non-final members of compounds, in particular vowel alternation (the most common type). Additionally, exceptional changes to the final members of compounds are discussed in §\ref{sec:final.compounds}.

\subsection{Vowel alternations in non-final members of compounds} \label{sec:vowel.alternations.compounds}
Regular bound state is Japhug applies to open syllables, following the correspondences in \tabref{tab:sc.regular}.

\begin{table}
\caption{Regular bound state in Japhug} \label{tab:sc.regular}
\begin{tabular}{lllll}
\lsptoprule
Base & SC & Example \\
\midrule 
\ipa{-a} &\ipa{-ɤ} & \japhug{βɣɤsni}{mill axle} from \japhug{βɣa}{mill} + \japhug{tɯ-sni}{heart} \\
\ipa{-e} &\ipa{-ɤ} & \japhug{tɕʰemɤpɯ}{little girl} from \japhug{tɕʰeme}{girl} + \japhug{ɯ-pɯ}{little one} \\
\ipa{-o} &\ipa{-ɤ} & \japhug{mbrɤsno}{horse saddle} from \japhug{mbro}{horse} + \japhug{tɤ-sno}{saddle}\\
\ipa{-u} &\ipa{-ɤ} & \japhug{tɤ-kɤrme}{head hair} from \japhug{tɯ-ku}{head} + \japhug{tɤ-rme}{hair} \\
\midrule 
\ipa{-i} &\ipa{-ɯ} & \japhug{smɯɣot}{light of the fire} from \japhug{smi}{fire}+ \japhug{ɣot}{warm light} \\
\lspbottomrule
\end{tabular}
\end{table}

\tabref{tab:sc.regular} shows that vowels other than \ipa{i} shift to \ipa{ɤ}, and \ipa{i} to \ipa{ɯ}.

In a few cases, \ipa{u} can also alternate with \ipa{ɯ}, as in \forme{ŋɤtɕɯ-} which occurs in the expression \japhug{ŋɤtɕɯkɤti+kʰɯ}{be obedient} (more details on this form are provided in §\ref{sec:interrogative.indef}), the bound state of \japhug{ŋotɕu}{where}.

The vowel \ipa{i} also alternates with \ipa{ɤ} in bound state, as in \japhug{qaprɤftsa}{centipede} from \japhug{qapri}{snake} and \japhug{tɤ-ftsa}{sister's child} or 
\japhug{tɯ-mɤmɲaʁ}{astragalus} from \japhug{tɯ-mi}{foot, leg} and \japhug{tɯ-mɲaʁ}{eye}. %\japhug{ɯ-χtɯrca}{with the others} &&from \japhug{tɯ-χti}{companion} + \japhug{tɤ-rca}{together with}

Nouns ending in \ipa{-ɯ} never have a bound state form that is different from the base form, as for instance \japhug{tɯmɯpaʁ}{slug} from \japhug{tɯ-mɯ}{sky, weather} and \japhug{paʁ}{pig}.

Vowel alternation in closed syllables is very rare, and affects only a few stems with \ipa{o} as the main vowel (\tabref{tab:sc.irregular}). The bound state \forme{ɕɤm-} of \japhug{ɕom}{iron} occurs in a few other nouns, but the form \forme{staʁ-} (with internal sandhi to \forme{staχ\trt}, cf. §\ref{sec:internal.sandhi.compounds}) from \japhug{stoʁ}{broad bean} is unique.

\begin{table}
\caption{Irregular bound state in closed syllable stems} \label{tab:sc.irregular}
\begin{tabular}{lllll}
\lsptoprule
Base & SC & Example \\
\ipa{-oʁ} &\ipa{-aʁ} & \japhug{staχpɯ}{pea} from \japhug{stoʁ}{broad bean} + \japhug{ɯ-pɯ}{little one} \\
\ipa{-om} &\ipa{-ɤm} & \japhug{ɕɤmtsʰoʁ}{iron nail} from \japhug{ɕom}{iron} + \japhug{tɤtsʰoʁ}{nail} \\
\lspbottomrule
\end{tabular}
\end{table}


\subsection{Other alternations} \label{sec.compounds.first.other.alternations}
Apart from the regular vowel changes described above, four types of alternations are observed in non-final member of compounds: internal sandhi, coda loss, reduced forms and loss of the possessive prefix.

\subsubsection{Internal sandhi in compounds} \label{sec:internal.sandhi.compounds}
First, the first element of a cluster undergoes internal sandhi (§\ref{sec:heterosyllabic.clusters}, §\ref{sec:sandhi.word}), with voicing and nasal assimilation as in \tabref{tab:sandhi.compounds}. 

\begin{table}
\caption{Internal sandhi in compounds} \label{tab:sandhi.compounds} 
\begin{tabular}{lllll}
\lsptoprule
Type & Example \\
Nasal  & \ipa{t} \fl{} \ipa{n} /\_[+nasal] & \japhug{tsʰɤnmu}{ewe} \\
assimilation&&from \japhug{tsʰɤt}{goat} + \japhug{mu}{female} \\
Voicing  & \ipa{ɣ} \fl{} \ipa{x} /\_[-voiced] & \japhug{zrɯxpɯ}{little louse} \\
assimilation&&from \japhug{zrɯɣ}{louse} + \japhug{ɯ-pɯ}{little one} \\
 & \ipa{ʁ} \fl{} \ipa{χ} /\_[-voiced] & \japhug{tɯ-jaχpa}{palm} \\
&&from \japhug{tɯ-jaʁ}{hand, arm} + \japhug{pa}{down} \\
 & \ipa{z} \fl{} \ipa{s} /\_[-voiced] & \japhug{mbrɤstsʰi}{rice gruel} \\
&&from \japhug{mbrɤz}{rice} + \japhug{tɯtsʰi}{rice gruel} \\
\lspbottomrule
\end{tabular} 
\end{table}

There are cases of irregular internal sandhi attested only in lexicalized compounds. For instance \japhug{jaŋntsɤrpa}{one-handed axe} from \japhug{tɯ-jaʁ}{hand, arm}, \japhug{ɯ-ntsi}{one of a pair} and \japhug{tɯ-rpa}{axe}, showing a rule \hbox{\ipa{ʁ} \fl{} \ipa{ŋ} /\_[+nasal]} which is not productive in the language (as shown by words such \japhug{tɯ-jaʁndzu}{finger}, also with \japhug{tɯ-jaʁ}{hand, arm} as first element).

\subsubsection{Loss of codas in compounds} \label{sec:loss.codas.compounds}
Coda loss is not a regular process in first elements of compounds. The following is a list of some of the most representative examples.\footnote{Further examples can be found in numerals (see §\ref{sec:approx.numerals} and §\ref{sec:numeral.prefixes}).}

\begin{itemize}
\item Loss of \ipa{-β}: 

\japhug{ɴqiaβ}{dark side of the mountain} + \japhug{zwɤr}{mugwort} \fl{} \japhug{ɴqiazwɤr}{\textit{Artemisia} sp.} 
\item Loss of \ipa{-t}: 

\japhug{xtɯt}{be short} + \japhug{rɲɟi}{be long} \fl{} \japhug{xtɯrɲɟi}{length (n)} 

 \japhug{tsʰɤt}{goat} + \japhug{ta-ʁrɯ}{horn} \fl{} \japhug{tsʰɤʁrɯ}{goat horn} 
\item Loss of \ipa{-z}: 

\japhug{qartsʰaz}{deer} + \japhug{tɯ-ndʐi}{skin} \fl{} \japhug{qartsʰɤndʐi}{deer hide} 
\item Loss of \ipa{-r}:

 \japhug{zwɤr}{mugwort} + \japhug{wɣrum}{be white} \fl{} \japhug{zwɤɣrum}{\textit{Artemisia} sp.} 

\japhug{ɕɤr}{night} + \japhug{ɯ-χcɤl}{center} \fl{} \japhug{ɕɤχcɤl}{middle of the night} 
\item Loss of \ipa{-ɣ}:

 \japhug{tɤjmɤɣ}{mushroom} + \japhug{--sti}{alone} \fl{} \japhug{jmɤtɤsti}{species of mushroom} 
 
\japhug{tɯ-mtʰɤɣ}{waist} + \japhug{rŋgɤβ}{attach} \fl{} \japhug{tɯ-mtʰɤrɴɢɤβ}{waistline of the trousers} (where one can tuck things in) 
\item Loss of \ipa{-ʁ}: 

\japhug{ɕoʁ}{buckwheat} + \japhug{wɣrum}{be white} \fl{} \japhug{ɕɤɣrum}{type of buckwheat} 

\japhug{paʁ}{pig} + \japhug{tɤ-qa}{paw, root} \fl{} \japhug{pɤqa}{stuffed pig feet} 
\end{itemize}

With the exception of the loss of \forme{-t}, which is relatively common, the other cases are rare and cannot be predicted by any rule based on phonology (the presence of a cluster in the following element is irrelevant, for instance). Some of them occur with other alternations in the second syllable (cf §\ref{sec:second.member.alternation}).

\subsubsection{Reduced forms} \label{sec:reduced.forms.compounds}  
A handful of nouns have reduced bound state forms when occurring as the first member of a compounds. 

The noun \japhug{nɯŋa}{cow} corresponds to the syllable \forme{ŋɤ-} in the compounds \japhug{ŋɤnɯ}{udder} (with \japhug{tɯ-nɯ}{teat} as second element), \japhug{ŋɤqe}{cow dung} (with \japhug{tɯ-qe}{shit, dung}) and \japhug{ŋɤlitɕaʁmbɯm}{dung beetle} (on which see §\ref{sec:second.member.alternation}), which would be the regular bound state from a stem \forme{ŋa-}. The apparent `loss' of a \forme{nɯ-} element is due to the fact that the noun \japhug{nɯŋa}{cow} is itself an ancient compound comprising \japhug{tɯ-nɯ}{teat} as first element (`bovid with udders').

In the case of \japhug{kʰɯna}{dog}, we find the bound state \forme{kʰɯ-} in the compounds \japhug{kʰɯndʐi}{dog skin} (with \japhug{tɯ-ndʐi}{skin} as second element), \japhug{kʰɯdo}{old dog} (see §\ref{sec:derogatory}), \japhug{kʰɯtsʰoʁ}{hunting with dogs} (probably a noun-verb compound with \japhug{tsʰoʁ}{attach}, see also the related incorporating verb in §\ref{sec:incorp.denom}) and a few plant names such as \japhug{kʰɯlu}{Euphorbia helioscopia} (a possessive compound meaning `(the plant) having dog milk' -- referring to its toxic juice, see §\ref{sec:possessive.n.n}) and \japhug{kʰɯrtsʰɤz}{Polygonum sp.} (`dog lung'; the second element is \japhug{tɯ-rtsʰɤz}{lung}). Unlike \japhug{nɯŋa}{cow}, whose reduced bound state corresponds to the second syllable, the syllable \forme{kʰɯ-} corresponds to the first syllable of \japhug{kʰɯna}{dog}, which must also be an obscured compound. The etymology of the element \forme{-na} is unclear.

\subsubsection{Loss of possessive prefix} \label{sec:loss.possessive.prefix.compounds}
Some inalienably possessed or alienabilized nouns lose their possessive prefix (or frozen indefinite possessive \forme{tɯ-/tɤ\trt}, see §\ref{sec:frozen.indef}), as for instance the noun \japhug{jmɤrtaʁ}{weevil}, which comes from \japhug{tɤ-jme}{tail} and \japhug{artaʁ}{be forked} (`forked tail').\footnote{The verb \japhug{artaʁ}{be forked} itself is denominal from \japhug{tɤ-rtaʁ}{branch} (§\ref{sec:denom.a}).} Its first element \japhug{tɤ-jme}{tail} loses the prefix \forme{tɤ-} and undergoes regular vowel alternation. 

Similar examples are particularly common with \japhug{tɯ-xtsa}{shoe}, as mainly parts of the shoes are referred to by alienably possessed noun compounds with \forme{xtsɤ-} as first element (\japhug{xtsɤɕna}{tip of the shoe}, \japhug{xtsɤrkɯ}{sides of the shoe} etc).

In some derivations that originate from compounds, such as the privative (§\ref{sec:privative}) or the derogatory (§\ref{sec:derogatory}), the indefinite possessor prefix is also removed.

\subsection{Final member of compounds} \label{sec:final.compounds}
Morphological changes affecting the last members of compounds are less common than those on the first members. The only productive morphological alternation in this context is the loss of possessive prefix when the last member is an inalienably possessed noun.

\subsubsection{Loss of possessive prefix} \label{sec:possessive.prefix.second.compounds}
In compounds with an inalienably possessed noun as final element, the indefinite possessor prefix is lost as a rule, as in for example in the plant name \textit{\japhug{kʰɯnajme}{Setaria viridis}} from \japhug{kʰɯna}{dog} and \japhug{tɤ-jme}{tail}.\footnote{The absence of the bound state \forme{kʰɯ-} is indicative in this case of a later loanword, perhaps calqued from Chinese \ch{狗尾草}{gǒuwěicǎo}{\textit{Setaria viridis}}. }

Exceptions are very few. They include compounds whose second element is itself a compound, such as \japhug{lɤndʐitɤlɤtsʰaʁ}{\textit{Delphinium} sp.} from \japhug{lɤndʐi}{ghost} and \japhug{tɤlɤtsʰaʁ}{milk filter}; the second element is from \japhug{tɤ-lu}{milk} in bound state and \japhug{tsʰaʁ}{sieve}. In \forme{lɤndʐi-tɤ-lɤ-tsʰaʁ} (ghost-\textit{indef.poss}-milk-sieve), the indefinite possessor prefix \forme{tɤ-} has become frozen when the compound \japhug{tɤlɤtsʰaʁ}{milk filter} was formed, and is therefore not subject to deletion.

Another exceptional example is \japhug{ɯ-qataʁrɯ}{hoof} from \japhug{tɤ-qa}{paw, root}, `root', `bottom' and \japhug{ta-ʁrɯ}{horn}, perhaps because the second element was perceived as being alienabilized, meaning `the horn-like thing on the foot'; in alienabilized possessive forms, definite possessor prefixes are stacked onto the indefinite possessive instead of replacing it, see §\ref{sec:alienabilization}).

\subsubsection{Alternations} \label{sec:second.member.alternation} 
Morphophonological alternations affecting last members of compounds are very rare in Japhug. 

Internal sandhi influencing the second member of a compound rather than the first occur when a root ending in \ipa{-ʁ} is followed by a cluster with a velar fricative as first element. Thus, the incorporating verb \japhug{amɲaχtsʰɯm}{be petty} is the denominal of a lost compound \forme{*mɲaχtsʰɯm} comprising \japhug{tɯ-mɲaʁ}{eye} as first element and \japhug{xtsʰɯm}{be thin}: the combination of \forme{-ʁ+xtsʰ-} yields \forme{-χtsʰ-}.

Several cases of alternations in the last member are found with animal nouns with the uvular class prefix \forme{qa\trt}, which has a variant \ipa{χ-/ʁ-} in this context in some compounds (see §\ref{sec:uvular.animal} and §\ref{sec:uvular.other}). 

Other alternations are restricted to specific lexical items, which are discussed below one by one (\japhug{rŋgɤβ}{attach}, \japhug{ɣɯrni}{be red}, \japhug{tʂu}{road} and \japhug{tɯ-ɣli}{excrement, dung}).

The inalienably possessed noun \japhug{tɯ-mtʰɤrɴɢɤβ}{waistline of the trousers} (a noun whose meaning is better explained by an example sentence like \ref{ex:WmthArNGAB}) is a compound of the noun \japhug{tɯ-mtʰɤɣ}{waist} with the transitive verb \japhug{rŋgɤβ}{attach}, which appears as a uvularized allomorph \forme{-rɴɢɤβ} not attested elsewhere: it is unclear why uvularization took place in this word (dissimilation with the coda \ipa{-ɣ} of the previous root is unlikely).

\begin{exe}
\ex \label{ex:WmthArNGAB}
\gll tsʰi tɤ-mda tɕe nɯ ɯʑo ɯ-cʰɤmdɤru nɯ pjɯ-nɯ-rʁe tɕe pjɯ-nɯ-tsʰi, mɯ-na-tsʰi tɕe tɕe li tu-nɯ-χɕoʁ tɕe ɯ-mtʰɤrɴɢɤβ cʰɯ-nɯ-rʁe \\
drink:\textsc{fact} \textsc{aor}-be.the.time \textsc{lnk} \textsc{dem} \textsc{3sg} \textsc{3sg}.\textsc{poss}-drinking.straw \textsc{dem} \textsc{ipfv}:\textsc{down}-\textsc{auto}-insert \textsc{lnk} \textsc{ipfv}:\textsc{down}-\textsc{auto}-drink \textsc{neg}-\textsc{aor}:3\flobv{}-drink \textsc{lnk} \textsc{lnk} again \textsc{ipfv}:\textsc{up}-\textsc{auto}-take.out \textsc{lnk} \textsc{3sg}.\textsc{poss}-tuck \textsc{ipfv}:\textsc{downstream}-\textsc{auto}-insert \\
\glt `When it is time to drink, he inserts his straw [into the jar] and drinks from it, and when he does not drink any more, he takes it out and tucks it back into his trousers.' (30-tChorzi)
\japhdoi{0003760\#S35}
\end{exe}
 
The noun \japhug{ftɕɤru}{path in the middle of the fields} is a compound of \japhug{ftɕar}{summer} and \japhug{tʂu}{road} (such paths are made during summer to allow workers to work in the field without damaging the crops, see a definition in \ref{ex:tusANke.Wspa} in §\ref{sec:instrumental.participle.relatives}). The first element \forme{ftɕɤ-} is the bound state of \forme{ftɕar} (with loss of final consonant) and the form \forme{-ru} for the second member of the compound is a clue that \forme{tʂu} comes from earlier \forme{*t-ro} with a dental stop+\ipa{r} cluster changing to a retroflex affricate (see §\ref{sec:Cr.clusters} and §\ref{sec:teens}) -- the \forme{*t-} element being prefixal (perhaps a fossilized indefinite possessor prefix).

The noun \japhug{jmɤɣni}{russula} clearly derives from \japhug{tɤjmɤɣ}{mushroom} and \japhug{ɣɯrni}{be red}, but while the loss of the \forme{tɤ-} prefix can be explained (see §\ref{sec:frozen.indef}), the form of the second element (without \forme{r-} preinitial) is a mystery. The form \forme{-rni} (without \forme{ɣɯ\trt}, a prefix possibly of denominal origin, §\ref{sec:denom.intr.GA}) is found in \japhug{qrorni}{red ant} with \japhug{qro}{ant} as first element (a late innovation specific to the Kamnyu dialect).
 
 The compound \japhug{ŋɤlitɕaʁmbɯm}{dung beetle}, with the irregular bound state \forme{ŋɤ-} (see §\ref{sec:reduced.forms.compounds}) of the noun \japhug{nɯŋa}{cow}, contains a syllable \forme{-li} clearly derived from the inalienably possessed noun \japhug{tɯ-ɣli}{excrement, dung}.\footnote{The second part of the noun \forme{-tɕaʁmbɯm} contains \japhug{aʁmbɯm}{be concave}. } 
 
The examples above show that most of the forms with an irregular second member also present some irregularity in the first member of the compound. 

\section{Compound nouns}
Nominal compounds in Japhug can be build by compounding nouns, but also verbs, adverbs and ideophones. In this section, compounds are first classified by the part of speech of their elements, and then by the semantic relationship between these elements.

\subsection{Noun-Noun compounds} \label{sec.n.n.compounds}
Noun-Noun compounds can be divided in three classes: Determinative, possessive and coordinative compounds.

\subsubsection{Determinative compounds} \label{sec:determinative.n.n}
In determinative (or endocentric) compounds, the two elements either have a genitival or an attributive relationship.\footnote{I use this term to encompass both the traditional notions of \textit{karmadhāraya-} and \hbox{\textit{tatpuruṣa-}:} since pre-nominal modifiers are not easily distinguishable from possessors, this distinction would not be practical. } Modifier-Head order is by far the most common, but Head-Modifier is found in some compounds based on postnominal modifiers.

While genitive phrases are followed by a noun with a third person possessive prefix (§\ref{sec:gen.possession}), in the corresponding compounds the possessive prefix is deleted (except the indefinite possessor prefix in exceptional examples, see §\ref{sec:possessive.prefix.second.compounds}).

In this type of compounds, the first element is most commonly in bound state if derived from a word ending in open syllable, both for highly lexicalized compounds \japhug{qaɕpɤrnoʁ}{wild strawberry} (`frog's brain', from \japhug{qaɕpa}{frog} and \japhug{tɯ-rnoʁ}{brain}) and more transparent ones (\japhug{jlɤndʐi}{hybrid yak hide} from \japhug{jla}{male hybrid yak} and \japhug{tɯ-ndʐi}{skin}). 

Among determinative compounds, we commonly find nouns denoting locations and places or ethnic names such as \japhug{kɯrɯ}{Tibetan} and \japhug{kupa}{Chinese} (as in \japhug{kupaŋga}{Chinese-style clothes} or \japhug{kupastaχpɯ}{soja}(with \japhug{staχpɯ}{pea}, on which see §\ref{sec:vowel.alternations.compounds}). 

Some compounds comprise elements that are themselves compounds. For instance, the first element \japhug{sɯŋgɯ}{forest} in \japhug{sɯŋgɯrmɤβja}{lophophorus} (with \japhug{rmɤβja}{peacock}) and \japhug{sɯŋgɯpɤjka}{type wild squash} (with \japhug{pɤjka}{squash})
is a compound from \japhug{si}{tree} and \japhug{ɯ-ŋgɯ}{inside}, and its original meaning was presumably `among the trees'.\footnote{The noun \japhug{sɯŋgɯ}{forest} is better translated as `wild' when occurring as prenominal modifier or first member of compounds.}

Compounds also exist with specific placenames such as \japhug{tɕʰitɕɯn}{Chuchen}, for instance in \japhug{tɕʰitɕɯn paχɕi}{pear} (with \japhug{paχɕi}{apple} as second element), a noun which can undergo denominal derivation to \japhug{nɯtɕʰitɕɯnpaχɕi}{pick pears} (§\ref{sec:denominal.vs.light.verb}), showing that the place name modifier has been integrated.\footnote{I am indebted to Gong Xun for this observation. }

In addition to nouns, participles also occur in determinative compounds. They are found both as first or second element of the compound, and both subject participles in \forme{kɯ-} (§\ref{sec:lexicalized.subject.participle}) and oblique participles in \forme{sɤ-} (§\ref{sec:lexicalized.oblique.participle}) are attested.

For instance, the compound \japhug{tʂɤsɤɴɢɤt}{crossroad} combines the bound state of \japhug{tʂu}{road} and the oblique participle \japhug{ɯ-sɤ-ɴɢɤt}{place where X part ways} from \japhug{nɯɴɢɤt}{part ways}.\footnote{This intransitive verb itself is the anticausative of \japhug{qɤt}{separate}, with an additional \forme{nɯ-} prefix, §\ref{sec:anticausative.direction}. } The obsolete noun \japhug{sɤqrɤcʰa}{alcohol offered to one's guests}, comprises the oblique participle \forme{ɯ-sɤ-qru} of the verb \japhug{qru}{greet, welcome, receive} and the noun \japhug{cʰa}{alcohol} (see other examples in §\ref{sec:lexicalized.oblique.participle}).

Compounds with the participle of a transitive verb as their second element do not necessarily derive from a genitival construction, though they might be superficially similar to compounds of this type. For instance \japhug{qalekɯtsʰi}{species of kite} comes from \japhug{qale}{wind} and the participle \japhug{ɯ-kɯ-tsʰi}{blocking (it)} (§\ref{sec:subject.participles}) of the transitive verb \japhug{tsʰi}{block}; the phrase \forme{qale ɯ-kɯ-tsʰi} (wind \textsc{3sg}.\textsc{poss}-\textsc{sbj}:\textsc{pcp}-block) `blocking the wind' is more properly a headless participial relative (§\ref{sec:headless.relative}, §\ref{sec:lexicalized.subject.participle}), and is more similar to Object-Verb compounds (§\ref{sec:object.verb.compounds}).

Only a handful of Head-Modifier determinative compounds are attested. Most of these are the lexicalized versions of nouns followed by post-nominal modifiers (§\ref{sec:unpossessible.nouns}). A good example of such compounds is provided by \japhug{ʑmbrɯkɤlu}{willow that does not grow high} from \japhug{ʑmbri}{willow} and the privative form \japhug{kɤlu}{headless} (§\ref{sec:privative}) of \japhug{tɯ-ku}{head}), a name explained in (\ref{ex:kAlu}). 

\begin{exe}
\ex \label{ex:kAlu} 
\gll ɯ-taʁ ɯ-mnɯ kɯnɤ kɯ-zri tu-ɬoʁ mɯ́j-cʰa tɕe, nɯ-kɤ-ʁndzɤr ʑo ɲɯ-fse tɕe nɯ ʑmbrɯkɤlu tu-kɯ-ti ŋu. tɕe nɯ ɯ-ku kɯ-me kɤ-ti ɲɯ-ŋu. kɤlu nɯ ɯ-ku kɯ-me kɤ-ti ɲɯ-ŋu.\\
\textsc{3sg}-on \textsc{3sg}.\textsc{poss}-new.twig also \textsc{sbj}:\textsc{pcp}-be.long \textsc{ipfv}:\textsc{up}-come.out \textsc{neg}:\textsc{sens}-can \textsc{lnk} \textsc{aor}-\textsc{obj}:\textsc{pcp}-cut \textsc{emph} \textsc{sens}-be.like \textsc{lnk} \textsc{dem} plant.name \textsc{ipfv}-\textsc{genr}-say be:\textsc{fact} \textsc{lnk} \textsc{dem} \textsc{3sg}.\textsc{poss}-head \textsc{sbj}:\textsc{pcp}-not.exist \textsc{inf}-say \textsc{sens}-be headless \textsc{dem} \textsc{3sg}.\textsc{poss}-head \textsc{sbj}:\textsc{pcp}-not.exist \textsc{inf}-say \textsc{sens}-be\\
\glt `Its new twigs cannot grow very long, and look like they have been sawed short, therefore it is called `headless willow'. `Headless' means `without head'.' (07-Zmbri)
\japhdoi{0003438\#S30}
\end{exe}

The counted noun \japhug{tɯ-pɤrme}{one year of life} attests a different type of Head-Modifier determinative compound. It comes from \japhug{tɯ-xpa}{one year} and \japhug{tɯrme}{person} (see §\ref{sec:num.prefix.paradigm.history} on the alternation between \forme{-xpa} and \forme{-pɤ-}, and §\ref{sec:frozen.indef} on the \forme{tɯ-} prefix in `man'), and originally meant `man's year (of life)'. Despite this meaning, the modifier `man' appears as the second element. Unlike `headless willows' (\ref{ex:kAlu}), in this case the modifier would not be postnominal in the corresponding noun phrase. 


\subsubsection{Possessive compounds} \label{sec:possessive.n.n}
Possessive (or exocentric) compounds (\textit{bahuvrīhi-}) are uncommon in Japhug, and tend to be synchronically obscure. All known examples appear to have Modifier-Head order, and are plant names.

The name \textit{\japhug{kʰɯlu}{Euphorbia helioscopia}} combines the reduced bound state of \japhug{kʰɯna}{dog} (§\ref{sec:reduced.forms.compounds}) with \japhug{tɤ-lu}{milk}. It presumably means `(having) dog milk', a reference to a whitish toxic liquid that comes from it (\ref{ex:khWlu}).

\begin{exe}
\ex \label{ex:khWlu} 
\gll tɕe nɯ kʰɯlu nɯnɯ sɤndɤɣ. ɯ-lu tu tɕe, tɤ-lu kɯ-fse kɯ-wɣrɯ\redp{}wɣrum ŋu. koŋla ʑo, pjɯ́-wɣ-qlɯt tɕe, nɯre ɯ-lu tu.\\
\textsc{lnk} \textsc{dem} Euphorbia.helioscopia \textsc{dem} poisonous:\textsc{fact} \textsc{3sg}.\textsc{poss}-milk exist:\textsc{fact} \textsc{lnk} \textsc{indef}.\textsc{poss}-milk \textsc{sbj}:\textsc{pcp}-be.like \textsc{sbj}:\textsc{pcp}-\textsc{emph}\redp{}be.white be:\textsc{fact} completely \textsc{emph} \textsc{ipfv}-\textsc{inv}-break \textsc{lnk} there \textsc{3sg}.\textsc{poss}-milk exist:\textsc{fact}\\
\glt `The \textit{Euphorbia helioscopia} is toxic, it has a juice white like milk, when it is broken, there is milk in [the stalk]. (19-khWlu)
\japhdoi{0003540\#S18}
\end{exe}

Another plant name, \textit{\japhug{qaprimdʑu}{Cicerbita roborowskii}}, from \japhug{qapri}{snake} and \japhug{tɯ-mdʑu}{tongue} is interpretable as a possessive compound `(having) a snake's tongue', referring to the shape of its leaves (\ref{ex:qaprimdZu}).

\begin{exe}
	\ex \label{ex:qaprimdZu} 
	\gll ɯ-jwaʁ nɯra qapri ɯ-mdʑu ɯ-tsʰɯɣa nɯ fse \\
	\textsc{3sg}.\textsc{poss}-leaf \textsc{dem}:\textsc{pl} snake \textsc{3sg}.\textsc{poss}-tongue \textsc{3sg}.\textsc{poss}-shape \textsc{dem} be.like:\textsc{fact} \\
	\glt `The shape of its leaves look like that of a snake tongue.' (xsArW) \japhdoi{0003500\#S66}
\end{exe}



The compound \textit{\japhug{kɯngɯttɤrtsɤɣ}{Leonurus}}, from the numeral \japhug{kɯngɯt}{nine} and the noun \japhug{tɤ-rtsɤɣ}{stairs}, can be analyzed as meaning `(plant having) nine stairs', referring to the nodes on the stalk of this plant. The Numeral-Noun order in this compound is remarkable, since numeral normally follow the noun (see §\ref{sec:uses.numerals}, §\ref{sec:noun.phrases.word.order}) but similar to some quantifiers (§\ref{sec:quantifiers.determiners}). The same Numeral-Noun order is found in the more complex compound \textit{\japhug{kɯngɯttɤrqʰɤɴɢaʁ}{Lonicera sp.}} discussed in §\ref{sec.n.v.compounds}.


\subsubsection{Coordinative compound} \label{sec:coordinative.n.n}
Coordinative compounds are uncommon in Japhug, and are formally indistinguishable from the previous classes. In this type of compounds, both elements are heads.

This class includes the traditional traditional \textit{dvandva-}, which are semantically intrinsically collectives, for instance \japhug{cʰɤmtʰɯm}{food and drinks} from \japhug{cʰa}{alcohol} and \japhug{tɤ-mtʰɯm}{meat}, \japhug{sŋiɕɤr}{night and day} from \japhug{tɯ-sŋi}{one day} and \japhug{ɕɤr}{night} and \japhug{χcʰoʁe}{right and left} from \japhug{χcʰa}{right} and \japhug{ʁe}{left}, the latter two being mainly used as adverbs.

An even rarer type of coordinative compound are the appositive compounds, the only clear example of which is \japhug{kɯrŋukɯɣndʑɯr}{harvestman}, a noun built from two subject participles, from the transitive verbs \japhug{rŋu}{parch} and \japhug{ɣndʑɯr}{grind}. The two elements of the compound refer to the actions supposedly performed by that type of chelicerate (`the parcher-grinder') like the participial form of a bipartite verb (§\ref{sec:bipartite}).

The compound \japhug{qajɯsmɤnba}{leech}, from \japhug{qajɯ}{bug} and \japhug{smɤnba}{doctor}, is possibly interpretable as an appositive compound `bug acting as a doctor' or `doctor who is a bug', since its meaning is clearly not `doctor treating bugs'.

\subsection{Verb-Verb compounds} \label{sec.v.v.compounds}
There are two types of Verb-Verb nominal compounds in Japhug, action nominals (involving transitive action verbs) and degree nouns (with adjectival stative verbs).

\subsubsection{Action nominals} \label{sec.v.v.compounds.action}
Action nominals built from two verb roots are not common in Japhug. Some of these action nominals are made from verbs with complementary or near-identical meanings, for instance \japhug{joʁβzɯr}{tidying up} from \japhug{joʁ}{raise} and \japhug{βzɯr}{move}. This noun occurs in a light verb construction as in (\ref{ex:joRBzWr}). The denominal compound verb \japhug{rɤjoʁβzɯr}{tidy up} has a meaning verb close to this construction (§\ref{sec:denom.compound.verbs}).

\begin{exe}
\ex \label{ex:joRBzWr}
 \gll joʁβzɯr tɤ-βzu-t-a \\
 tidying.up \textsc{aor}-do-\textsc{pst}:\textsc{tr}-\textsc{1sg} \\
 \glt `I did some tidying up.' (elicited)
\end{exe}

Another type of verb-verb action nominals are made from verbs with opposite meanings, for instance \japhug{βʁɤnŋo}{winning and losing} from \japhug{βʁa}{prevail, win} and \japhug{nŋo}{lose}, which is used with existential verbs as in (\ref{ex:BRAnNo.maNEndZi}).\footnote{A similar compound 
\forme{fqɐ́-nŋət} with identical meaning is found in Tshobdun \citep[295]{jackson19tshobdun}.

}

\begin{exe}
\ex \label{ex:BRAnNo.maNEndZi}
 \gll βʁɤnŋo maŋe-ndʑi \\
winning.and.losing not.exist:\textsc{sens}-\textsc{du} \\
\glt `One cannot decide who [of the two of them] is winning and who is losing.' 
\end{exe}

At an earlier stage, such compound action nominals may have been common, as is suggested by the existence of denominal compound verbs without a corresponding noun, such as \japhug{raχtɯtsɣe}{do business} (from \japhug{χtɯ}{buy} and \japhug{ntsɣe}{sell}, §\ref{sec:denom.tr.nW} on the \forme{-n-} element).

\subsubsection{Nouns of dimension} \label{sec.v.v.compounds.degree}
The productive way of building degree nouns in Japhug is by adding the prefix \forme{tɯ-} to an adjectival stative verb (§\ref{sec:degree.nominals}), but an alternative formation involves the compounding of two antonymic verbs or location adverbs, such as \japhug{jpumxtsʰɯm}{thickness} from \japhug{jpum}{be thick} and \japhug{xtsʰɯm}{be thin}. All known examples are listed in \tabref{tab:degree.comp} (note that whenever possible, the first member of these compounds is in bound state).

\begin{table}
\caption{Nouns of dimension} \label{tab:degree.comp}
\begin{tabular}{llll}
 \lsptoprule 
 Compound & First verb & Second verb \\
 \midrule
\japhug{jpumxtsʰɯm}{thickness} (diameter) &\japhug{jpum}{be thick} &\japhug{xtsʰɯm}{be thin} \\
\japhug{jaʁmba}{thickness} (of a sheet)&\japhug{jaʁ}{be thick} &\japhug{mba}{be thin} \\
\japhug{xtɯrɲɟi}{length} &\japhug{xtɯt}{be short} &\japhug{rɲɟi}{be long} \\
\japhug{xtɕɯxte}{size} &\japhug{xtɕi}{be small} &\japhug{wxti}{be big} \\
 \lspbottomrule
\end{tabular}
\end{table}

In the case of \japhug{xtɕɯxte}{size}, the second element \forme{xte} is a variant also found in the derived verb \japhug{mɯxte}{be the majority}, probably the relic of a former \forme{*i/e} alternation still observed in the verb \japhug{ɣi}{come} (§\ref{sec:stem2.form}).

Locational adverbs/nouns can also be compounded to express the three spatial dimensions encoded by verbal morphology and locative markers (§\ref{sec:tridimensional.preverb}, §\ref{sec:locative.adv}): \japhug{taʁki}{up and down} from \japhug{taʁ}{up} and \japhug{aki}{down}, \japhug{lotʰi}{upstream and downstream} from \japhug{lo}{upstream} and \japhug{tʰi}{downstream} and \japhug{kundi}{east and west} from \japhug{kɯ}{east} and \japhug{ndi}{west}. An example of the use of these nouns can be found in (\ref{ex:maNlo.dpalcan}), §\ref{sec:personal.names.modifiers} above.

Nouns of dimension can further derive denominal verbs in \forme{a-} meaning `of unequal $X$' (§\ref{sec:denom.a}).

\subsection{Noun-Ideophone compounds} \label{sec.n.idph.compounds}
Nominal compounds comprising an ideophone (§\ref{sec:ideo:morpho}) are rare. Three examples are attested.
 
First, the noun \japhug{mcirɯβrɯβ}{person whose saliva drips continuously} is built from the inalienably possessed \japhug{tɯ-mci}{saliva} (§\ref{ex:body.part.prefix}) and the reduplicated ideophonic root \idroot{rɯβ}, found in the pattern III form (§\ref{sec:ideo.III}) \forme{rɯβnɤrɯβ} meaning `dripping (drop by drop) continuously' (§\ref{sec:ideo.irregular}) and the deideophonic verb \japhug{ɣɤrɯβrɯβ}{drip continuously}. The collocation of \forme{tɯ-mci} with both the ideophone and its derived verb is commonly attested (see example \ref{ex:nWrqo.YWmNAm}, §\ref{sec:sensory.endopathic}).
 
Second, the bird name \forme{jaʁmɤzdoʁzdoʁ} contains the bound state form of \japhug{tɯ-jaʁmu}{thumb} and the pattern II ideophone \forme{zdoʁzdoʁ} meaning `small and active'.


Third, the name of the mushroom \japhug{salaboŋboŋ}{puffball} contains the pattern II ideophone \forme{boŋboŋ} `ovoid'. The first part of this word is obscure.

\subsection{Adverb-Verb compounds} \label{sec.adv.v.compounds}
Adverb-Verb compounds are relative marginal. Compounds with \japhug{kɯzɣa}{a long time} in bound state \forme{kɯzɣɤ-} followed by a verb are however attested, as \japhug{kɯzɣɤ-ɕar}{searching for a long time} from the verb \japhug{ɕar}{search} in (\ref{ex:kWZGACar}). These compounds are studied in more detail in §\ref{sec:action.nominal.compounds} (see also \citealt[252]{jacques16complementation}).

\begin{exe}
\ex \label{ex:kWZGACar}
\gll kɯ-xtɕi nɯ ɣɯ pjɤ-me tɕe, tɕendɤre rca kɯzɣɤ-ɕar ʑo ɲɤ-βzu-nɯ ri pjɤ-me.\\
\textsc{sbj}:\textsc{pcp}-be.small \textsc{dem} \textsc{gen} \textsc{ifr}.\textsc{ipfv}-not.exist \textsc{lnk} \textsc{lnk} \textsc{unexp}:\textsc{deg} long.time-search \textsc{emph} \textsc{ifr}-do-\textsc{pl} \textsc{lnk} \textsc{ifr}.\textsc{ipfv}-not.exist \\
\glt `The [pigeon skin] of the youngest girl was not there, there looked for it for a long time but it was not there.' (the flood 2002)
\japhdoi{0003360\#S55}
\end{exe}

\subsection{Noun-Verb compounds} \label{sec.n.v.compounds}
Noun-Verb compounds include three main types, Subject-Verb, Object-Verb and Adjunct-Verb compounds. Participles or other nominalized verbs forms are treated in sections \ref{sec.n.n.compounds}, but criteria to distinguish between ambiguous forms in cases of homophony between noun and verb are provided in §\ref{sec:object.verb.compounds}.

\subsubsection{Subject-Verb compounds} \label{sec:subject.verb.compounds}
Subject-Verb compounds occur exclusively with intransitive verbs, mainly adjectival stative verbs. The noun is generally in bound state (see \tabref{tab:subj.v.compounds} below). Based on their semantics, there three of subject-verb compounds can be distinguished: attributive, possessive and action nominals.

Attributive Subject-verb compounds are equivalent to a relative clause comprising a stative verb and its subject (§\ref{ex:attributive.participles.stative.verbs}, §\ref{sec:subject.participle.subject.relative}). If the nominal and verbal elements are represented as $N$ and $V$, an attributive $NV$ compound means `$N$ which is $V$'. They are common with stative verbs of colour such as \japhug{ɲaʁ}{be black} or \japhug{wɣrum}{be white} as in \tabref{tab:subj.v.compounds}. 
 
\begin{table}
\caption{Examples of attributive Subject-Verb compound nouns} \label{tab:subj.v.compounds}
\begin{tabular}{llllll}
\lsptoprule
 Compound& Base Noun & Verb\\
 \midrule
\japhug{tɤɕɤɲaʁ}{black barley} & \japhug{tɤɕi}{barley} & \japhug{ɲaʁ}{be black} \\
\japhug{tɤɕɤɣrum}{white barley} & & \japhug{wɣrum}{be white} \\
\japhug{mtsʰalɤɲaʁ}{black nettle} & \japhug{mtsʰalu}{nettle} & \japhug{ɲaʁ}{be black} \\
\japhug{mtsʰalɤɣrum}{white nettle} & & \japhug{wɣrum}{be white} \\
\japhug{qartsɯɲaʁ}{cold winter} & \japhug{qartsɯ}{winter} & \japhug{ɲaʁ}{be black} \\
\japhug{pɣɤɲaʁ}{Pucrasia macrolopha} & \japhug{pɣa}{bird} &  \\
\japhug{tɤmtɯɲaʁ}{deadlock} & \japhug{tɤ-mtɯ}{knot} &  \\
\lspbottomrule
\end{tabular}
\end{table}

The compounds in \tabref{tab:subj.v.compounds} are highly lexicalized; in the case for instance of \japhug{pɣɤɲaʁ}{Pucrasia macrolopha}, this bird is not even black as the speakers themselves point out (\ref{ex:pGAYaR}).

\begin{exe}
\ex \label{ex:pGAYaR}
 \gll pɣɤɲaʁ kɤ-ti ci tu tɕe, nɯnɯ ʁo lɯski li nɯ pɣa ŋu, tɕeri mɤ-ɲaʁ ma ɲɯ-mpɕɤr. ɯ-muj nɯra wuma ʑo ɲɯ-mpɕɤr qʰe kɯ-tu ra ɲɯ-nɤmbju ʑo \\
 Pucrasia.macrolopha \textsc{obj}:\textsc{pcp}-say \textsc{indef} exist:\textsc{fact} \textsc{lnk} \textsc{dem} \textsc{advers} of.course again \textsc{dem} bird be:\textsc{fact} but \textsc{neg}-be.black:\textsc{fact} \textsc{lnk} \textsc{sens}-be.beautiful \textsc{3sg}.\textsc{poss}-feather \textsc{dem}:\textsc{pl} really \textsc{emph}  \textsc{sens}-be.beautiful \textsc{lnk} \textsc{sbj}:\textsc{pcp}-exist \textsc{pl} \textsc{sens}-be.brilliant \textsc{emph} \\
 \glt `The \textit{Pucrasia macrolopha} is of course also a bird (like the previous ones we talked about), but it is not black, it is beautiful, its feathers are very beautiful and those that are there (visible) are iridescent.' (23-pGAYaR) \japhdoi{0003626\#S80}
\end{exe}

More complex $NV$ compounds of this type are found, such as \japhug{tɯ-jaʁndzumɤpa\-χcɤl}{middle finger} from \japhug{tɯ-jaʁndzu}{finger} and \japhug{mɤpaχcɤl}{be in the middle} (itself a denominal verb from \japhug{ɯ-χcɤl}{center}).

In such compounds, some stative verbs occur with a \forme{-x-} element in individual forms. This is the case of \japhug{tɤlɤxcʰi}{fresh milk} from \japhug{tɤ-lu}{milk} and \japhug{cʰi}{be sweet}. It is possible that this velar fricative represents the remnant of a participle prefix \forme{kɯ-}. This \forme{-x-} is however present in the causative (§\ref{sec:caus.sWG}) and the tropative (\japhug{nɤxcʰi}{find sweet}, §\ref{sec:tropative.allomorphy}) derivations.
 
Possessive $NV$ compounds are equivalent to a participial relative with the possessor of the subject as the relativized element (§\ref{sec:possessor.relativization}): In other words, a possessive $NV$ means `(person/animal/entity) whose $N$ is $V$'. Examples are considerably fewer than the previous ones.

\tabref{tab:compounds.Xa} illustrates a few Noun-Verb compounds based on the verb \japhug{aχa}{have a hole, have a chip}, which surfaces as \forme{-χa} with elision of the \forme{a-} prefix.\footnote{It is however alternatively possible that the verb \forme{aχa} is denominal from an unattested inalienably possessed noun \forme{*-χa} `hole, chip, notch', and that the compounds in \tabref{tab:compounds.Xa} are also from that lost noun. } Among these examples, \japhug{ɕɣɤχa}{person lacking a tooth} and  \japhug{rnɤftɕɯχa}{whose ear has ten holes}\footnote{The compound \forme{rnɤftɕɯχa} occurs exclusively as postnominal modifier (§\ref{ex:attributive.postnominal}) in the fixed expression \forme{qala rnɤftɕɯχa} `the rabbit with ten holes in his ear', a trickster character found in traditional stories. } are possessive compounds, while \japhug{tʂɤχa}{chuckhole} and \forme{tɯ-jaʁmɤχa} (\ch{虎口}{hǔkǒu}{space between thumb and index} are possessive compounds. 

\begin{table}
\caption{Possessive compound nouns in \forme{-χa}, derived from the verb \japhug{aχa}{have a hole, have a chip}} \label{tab:compounds.Xa}
\begin{tabular}{lllll}
\lsptoprule
Compound noun & First element \\
\midrule
\japhug{ɕɣɤχa}{person lacking a tooth} & \forme{ɕɣɤ-} $\leftarrow$ \japhug{tɯ-ɕɣa}{tooth} \\
\japhug{rnɤftɕɯχa}{whose ear has ten holes} & \forme{rnɤ-} $\leftarrow$ \japhug{tɯ-rna}{ear} \\
	&\forme{ftɕɯ-} $\leftarrow$ Tib. \tibet{བཅུ་}{btɕu}{ten} \\
\japhug{tʂɤχa}{chuckhole} &\forme{tʂɤ-} $\leftarrow$ \japhug{tʂu}{road} \\
\japhug{tɯ-jaʁmɤχa}{space between thumb and index} &\forme{tɯ-jaʁmɤ-} $\leftarrow$ \japhug{tɯ-jaʁmu}{thumb} \\
\lspbottomrule
\end{tabular}
\end{table}
 
A particularly interesting Noun-Verb possessive compound is the plant name \japhug{kɯngɯttɤrqʰɤɴɢaʁ}{Lonicera sp.}, which comprises three elements: the numeral \japhug{kɯngɯt}{nine}, the inalienably possessed noun \japhug{tɤ-rqʰu}{hull, skin} and the intransitive verb \japhug{ɴɢaʁ}{peel, shed skin} (anticausative of \japhug{qaʁ}{peel}, see §\ref{sec:anticausative.morphology}). This compound is to be parsed [\forme{kɯngɯt-tɤrqʰɤ-}][\forme{ɴɢaʁ}] from a morphological point of view, as its meaning is `(plant) whose nine skins shed off' as is explained in the text excerpt in (\ref{ex:kWngWttArqhANGaR}): the first element \forme{kɯngɯt-tɤrqʰɤ-}\footnote{Note that this compound has Numeral-Noun order as in other examples (see §\ref{sec:possessive.n.n}).} corresponds to the intransitive subject of \japhug{ɴɢaʁ}{peel, shed skin}. 

\begin{exe}
\ex \label{ex:kWngWttArqhANGaR}
\gll kɯngɯttɤrqʰɤɴɢaʁ ɯ-rmi kɯra nɯnɯ tɕendɤre, ɯ-rqʰu kɯ-dɯ\redp{}dɤn ʑo pjɯ-ɴɢaʁ ɲɯ-ŋu. nɯnɯ tɯ-mpɕar nɤ tɯ-mpɕar, tɯ-mpɕar nɤ tɯ-mpɕar, pɯ-ɴɢaʁ qʰe ɯ-ŋgɯ li mɤʑɯ ɲɯ-βze qʰe, \\
\textit{Lonicera} \textsc{3sg}.\textsc{poss}-name \textsc{dem}:\textsc{prox:pl} \textsc{dem} \textsc{lnk} \textsc{3sg}.\textsc{poss}-skin \textsc{sbj}:\textsc{pcp}-\textsc{emph}\redp{}be.many \textsc{emph} \textsc{ipfv}-\textsc{acaus}:peel \textsc{sens}-be \textsc{dem} one-leaf \textsc{lnk} one-leaf one-leaf \textsc{lnk} one-leaf \textsc{aor}-\textsc{acaus}:peel \textsc{lnk} \textsc{3sg}.\textsc{poss}-inside again yet \textsc{ipfv}-grow \textsc{lnk} \\
\glt `As for the name of the \textit{Lonicera} sp., [it is because] it has a lot of skins that shed off, one after the other, and after one has shed off, another one grows again inside.' (14-sWNgWJu) \japhdoi{0003506\#S66}
\end{exe}

Yet, from a phonological point of view, the form should rather be parsed as [\forme{kɯngɯt-}][\forme{tɤrqʰɤ-ɴɢaʁ}], as the phonological integration between \forme{tɤrqʰɤ-} in bound state and the following verb root is stronger than that between the numeral \japhug{kɯngɯt}{nine} and the rest, as shown by the preservation of the final \forme{-t} with a rare heterosyllabic geminate (§\ref{sec:heterosyllabic.clusters}).

Additional less obvious examples of possessive compounds include \japhug{ɕnɤsti}{person with a stuffy nose} (from \japhug{tɯ-ɕna}{nose} and \japhug{asti}{be blocked}, see the discussion in §\ref{sec:object.verb.compounds}) and \japhug{ɕnaβndʑɣi}{snotty-nosed kid} (from \japhug{tɯ-ɕnaβ}{snot} and a verbal root \forme{-ndʑɣi} attested in \japhug{nɤndʑɣi}{have (snot)}). 

Action nominals $NV$ compounds are rare with intransitive verbs. Examples include \japhug{pɣɤmbri}{bird song} from \japhug{pɣa}{bird} and the intransitive \japhug{mbri}{cry, sing} or \japhug{snɯɲaʁ}{harming people} from \japhug{tɯ-sni}{heart} and \japhug{ɲaʁ}{be black} (§\ref{sec:incorp.denom}).


\subsubsection{Object-Verb compounds} \label{sec:object.verb.compounds}
Object-Verb nominal compounds in Japhug are very productive, and can be classified into two main types: actor $OV$ compounds, and action $OV$ compounds.

Actor $OV$ compounds are common in names of trades, animals and even plants, such as \japhug{rŋɯlfɕi}{silversmith}, \japhug{βɣɤru}{miller}, \japhug{zrɯɣndza}{praying mantis} and \japhug{tɤtɕɯβraʁ}{burdock}. The first of these examples, from the Tibetan loanword \japhug{rŋɯl}{silver} and the labile verb \japhug{fɕi}{forge}, requires little explanation. Some compounds present significant morphological alterations, as \japhug{βɣɤru}{miller}, which comes from the bound state of \japhug{βɣa}{mill} and the non-reduplicated form of the verb \japhug{rɯru}{guard} (§\ref{sec:redp.voice}). In addition, some compounds of this type do not make much sense without some cultural background; as an illustration of how the Japhug corpus can be used to better understand the origin of these compounds, I discuss below the latter two nouns.

The compound \japhug{zrɯɣndza}{praying mantis} derives from \japhug{zrɯɣ}{louse} and \japhug{ndza}{eat}, and literally means `louse eater', a descriptive term based on the feeding habits of that insect, as described in (\ref{ex:zrWGndza}).

\begin{exe}
\ex \label{ex:zrWGndza}
\gll nɯ ɯ-taʁ ri zrɯɣndza kɤ-ndo-tɕi tɕe, .... tɕendɤre zrɯɣ rcanɯ lɤŋɤtʂɤɣ jamar ʑo ɯ-ɕki kɤ-ta-tɕi. tɕe kɯ-mɤku nɯra tɕe, tɕe zrɯɣ nɯ lonba ʑo cʰɯ-mqlaʁ tɕe tu-ndze ɲɯ-ŋu. tɕendɤre kɯ-maqʰu tɕe nɤki ɲɯ-ŋu, tɕendɤre ku-nɯni kɯ-fse qʰe, ɯ-ŋgɯ nɯnɯ, ɯ-se nɯ lu-nɯ-tɕɤt qʰe cʰɯ-mqlaʁ ɲɯ-ŋu. \\
\textsc{dem} \textsc{3sg}-on \textsc{loc} praying.mantis \textsc{aor}-take-\textsc{1du} \textsc{lnk} .... \textsc{lnk} louse \textsc{unexp}:\textsc{deg} five.or.six about \textsc{emph} \textsc{3sg}-\textsc{dat} \textsc{aor}:\textsc{east}-put-\textsc{1du} \textsc{lnk} \textsc{sbj}:\textsc{pcp}-be.first \textsc{dem}:\textsc{pl} \textsc{lnk} \textsc{lnk} louse \textsc{dem} \textsc{all} \textsc{emph} \textsc{ipfv}-swallow \textsc{lnk} \textsc{ipfv}-eat[III] \textsc{sens}-be \textsc{lnk} \textsc{sbj}:\textsc{pcp}-be.after \textsc{lnk} \textsc{dem}:\textsc{cataph} \textsc{sens}-be \textsc{lnk} \textsc{ipfv}-suck[III] \textsc{sbj}:\textsc{pcp}-be.like \textsc{lnk} \textsc{3sg}.\textsc{poss}-inside \textsc{dem} \textsc{3sg}.\textsc{poss}-blood \textsc{dem} \textsc{ipfv}:\textsc{upstream}-\textsc{auto}-take.out \textsc{lnk} \textsc{ipfv}-swallow \textsc{sens}-be \\
\glt `(When we were little, one of my classmate had a lot of lice, and) we took a praying mantis [and put it on his clothes], then put five or six lice near it; the first ones, it swallowed them whole, and the following ones, it did the following: it would kind of suck them, drink the blood inside them, swallow it (and then throw them away).' (26-zrWGndza)
\japhdoi{0003696\#S26}
\end{exe}

The nouns \japhug{tɤtɕɯβraʁ}{burdock} from (\japhug{tɤ-tɕɯ}{son, boy} and \japhug{βraʁ}{attach to}) and \japhug{tɕʰemeβraʁ}{little burdock} (with \japhug{tɕʰeme}{girl} as first element) literally mean `attaching boys/girls'; an explanation for these names from local folklore is provided in (\ref{ex:tAtCWBraR}).

\begin{exe}
\ex \label{ex:tAtCWBraR}
\gll tɤtɕɯβraʁ tɕe, ɕɯ kɯ pa-mto nɯnɯ tɕe tɕe nɯ ɣɯ ɯʑɤɣ maʁ nɤ, ɯ-kʰa ɣɯ maʁ nɤ, ɯ-kɯmdza kɯ-fse ra ɣɯ, nɯ-tɕɯ maʁ nɤ nɯ-me tu tu-ti-nɯ ɲɯ-ŋu. tɕe nɯ nɯ-kɯmdza kɯ-fse kɯ-ɤrɕɤt ra, nɯ-skʰrɯ mɤ-kɯ-βdi a-pɯ-tu tɕe, ``wo ... ɯ-rɟit tɤ-tɕɯ sci ma tɤtɕɯβraʁ pɯ-mto-t-a" \\
burdock \textsc{lnk} who \textsc{erg} \textsc{aor}:3\flobv{}-see \textsc{dem} \textsc{lnk} \textsc{lnk} \textsc{dem} \textsc{gen} \textsc{3sg}:\textsc{gen} not.be:\textsc{fact} \textsc{lnk} \textsc{3sg}.\textsc{poss}-house \textsc{gen} not.be:\textsc{fact} \textsc{lnk} \textsc{3sg}.\textsc{poss}-relative \textsc{sbj}:\textsc{pcp}-be.like \textsc{pl} \textsc{gen} \textsc{3pl}.\textsc{poss}-son not.be:\textsc{fact} \textsc{lnk} \textsc{3pl}.\textsc{poss}-daughter exist:\textsc{fact} \textsc{ipfv}-say-\textsc{pl} \textsc{sens}-be \textsc{lnk} \textsc{dem} \textsc{3pl}.\textsc{poss}-relative \textsc{sbj}:\textsc{pcp}-be.like \textsc{sbj}:\textsc{pcp}-be.related \textsc{pl} \textsc{3pl}.\textsc{poss}-body \textsc{neg}-\textsc{sbj}:\textsc{pcp}-be.well \textsc{irr}-\textsc{ipfv}-exist \textsc{lnk} \textsc{interj} .... \textsc{3sg}.\textsc{poss}-child \textsc{indef}.\textsc{poss}-son \textsc{lnk} be.born:\textsc{fact} burdock \textsc{aor}-see-\textsc{tr}:\textsc{pst}-\textsc{1sg} \\
\glt `The burdock, whoever saw it will have a boy or a girl, him or someone from his house or among his relatives. If someone among his relatives is pregnant, he will say `her child will be a boy, as I saw a burdock.'' (26-NalitCaRmbWm)
\japhdoi{0003676\#S110}
\end{exe}

The object-verb compound \japhug{tʰɤlwɤɕtʂat}{sparing earth} (\ref{ex:qandzxe.thAlwACtsxat}), from \japhug{tʰɤlwa}{earth} and the transitive \japhug{ɕtʂat}{spare}, occurs as postnominal modifier (§\ref{ex:attributive.postnominal}) of the noun \japhug{qandʐe}{earthworm}, semantically equivalent to a post-nominal subject relative clause \forme{tʰɤlwa ɯ-kɯ-ɕtʂat} (earth \textsc{3sg}.\textsc{poss}-\textsc{sbj}:\textsc{pcp}-spare). This is the only case of a Noun-Verb compound in which both the subject and the object of the base verb are overt.

\begin{exe}
\ex \label{ex:qandzxe.thAlwACtsxat}
\gll nɯnɯ kɯ-mɤɕi kɯnɤ kɯ-rɤ-ɕtʂat nɯ, qandʐe tʰɤlwɤ-ɕtʂat tu-ti-nɯ. \\
\textsc{dem} \textsc{sbj}:\textsc{pcp}-be.rich also \textsc{sbj}:\textsc{pcp}-\textsc{apass}-spare \textsc{dem} earthworm earth-sparing \textsc{ipfv}-say-\textsc{pl} \\
\glt `Someone who spares things (i.e., does not waste anything) even though he is rich, people call him an `earth-sparing earthworm'.' (25-akWzgumba) \japhdoi{0003632\#S125}
\end{exe}

Like actor $OV$ compounds, action nominal compounds comprise a nominal root and a transitive verbal root. The nominal root in bound state corresponds to the object of the verb, as in the compound \japhug{cʰɤtsʰi}{alcohol drinking} from \japhug{cʰa}{alcohol} and \japhug{tsʰi}{drink}. This class of nouns, which occur in light verb constructions and serve as bases for incorporating denominal verbs (§\ref{sec:incorp.denom}), is discussed in §\ref{sec:action.nominal.compounds} and §\ref{sec:compound.action.nominal.Bzu}. 

Not all compounds whose second element originates from a transitive verb are Object-Verb (or Adjunct-Verb) compounds. Two potentially ambiguous cases must be pointed out. 

First, there are Noun-Noun compounds whose second element is a bare action nominal, deriving from a transitive verb (see §\ref{sec:bare.action.nominals}), but which loses its possessive prefix as is usual in compounding (§\ref{sec:possessive.prefix.second.compounds}). In such cases the resulting Noun-Noun compound is not formally distinguishable from a Noun-Verb compound, and only the meaning can be used to differentiate between the two classes. For instance, the plant name \japhug{tʂɤɕpʰɤt}{plantain} has the bound state of \japhug{tʂu}{road} as a first element, while its second part \forme{-ɕpʰɤt} can be interpreted as either directly from the verb \japhug{ɕpʰɤt}{patch} (`road patcher') or from the derived noun \japhug{tɤ-ɕpʰɤt}{patch (n)} (a piece of fabric used to patch worn clothes) (`road patch'). In this particular case, the second interpretation is more likely, and hence \japhug{tʂɤɕpʰɤt}{plantain} is better analyzed as a Noun-Noun compound.
 
Second, when the second element of a Noun-Verb compound is a \forme{a-} passive verb (see §\ref{sec:passive}), the \forme{a-/ɤ-} prefix is absorbed by the first element of the compound and becomes invisible. In the resulting form, the second element superficially looks similar to the transitive verb. For instance, the noun \japhug{ɕnɤsti}{person with a stuffy nose} appears to derive from \japhug{tɯ-ɕna}{nose} and the transitive verb \japhug{sti}{block}. However, semantics rules out such a derivation: it is a possessive compound whose literal meaning is `whose nose is blocked' (see §\ref{sec:subject.verb.compounds}), and cannot be interpreted as `(person) blocking noses', the expected meaning of an Object-Verb compound. Since the passive \japhug{asti}{be blocked} of \japhug{sti}{block} is well-attested, as shown by (\ref{ex:pjAkAstici}), it is better to analyze \japhug{ɕnɤsti}{person with a stuffy nose} as a Subject-Verb possessive compound (§\ref{sec:subject.verb.compounds}) derived from that passive form.
 
\begin{exe}
\ex \label{ex:pjAkAstici}
\gll maka ɲɯ́-wɣ-ɕɯɣ-mu mɯ-pjɤ-cʰa ma mɯ-pjɤ-mtsʰɤm matɕi ɯ-rna pjɤ-k-ɤ-sti-ci. \\
at.all \textsc{ipfv}-\textsc{inv}-\textsc{caus}-be.afraid \textsc{neg}-\textsc{ifr}.\textsc{ipfv}-can \textsc{lnk} \textsc{neg}-\textsc{ifr}-hear because \textsc{3sg}.\textsc{poss}-ear \textsc{ifr}.\textsc{ipfv}-\textsc{peg}-\textsc{pass}-block-\textsc{peg} \\
\glt `The noise could not frighten him, as he did not hear it, because his ears were blocked.' (140514 huishuohua de niao-zh) \japhdoi{0003992\#S191}
\end{exe}

\subsubsection{Adjunct-Verb compounds} \label{sec:adjunct.verb.compounds}
Adjunct-Verb compounds are action nominals (§\ref{sec:action.nominal.compounds}). Like other action nominal compounds, they can undergo denominal derivation to become incorporating verbs (§\ref{sec:incorp.denom}). 

Adjunct-Verb compounds typically take body parts or locative nouns as first element, as \japhug{zgrɯtɕʰɯ}{nudge} and \japhug{kɤtɕʰɯ}{headbutt}, which combine the body parts \japhug{tɯ-zgrɯ}{elbow} and \japhug{tɯ-ku}{head} with with the verb \japhug{tɕʰɯ}{gore, stab} as second element. Here the body parts cannot be analyzed as objects: the object of \japhug{tɕʰɯ}{gore, stab} is the person being gored/hit, not the part of the body one uses, as shown by the \textsc{1sg} indexation on the verb in (\ref{ex:tAwGtChWa}).

\begin{exe}
\ex \label{ex:tAwGtChWa}
\gll mbala kɯ tɤ́-wɣ-tɕʰɯ-a \\
bull \textsc{erg} \textsc{aor}-\textsc{inv}-stab-\textsc{1sg} \\
\glt `The bull gored me.' (elicited)
\end{exe}

These compounds are used with the light verb \japhug{lɤt}{release} as in (\ref{ex:zgrWtChW}). Additional examples are discussed in §\ref{sec:action.nominal.compounds} and §\ref{sec:incorp.denom} and §\ref{sec:incorp.vs.other}.


\begin{exe}
\ex \label{ex:zgrWtChW}
\gll zgrɯtɕʰɯ tɤ-lat-a \\
nudge \textsc{aor}-throw-\textsc{1sg} \\
\glt `I nudged [him].' (elicited)
\end{exe}

The incorporating denominal verbs \japhug{sɯzgrɯtɕʰɯ}{give a nudge} or \japhug{nɤkɤtɕʰɯ}{give a headbutt}, `gore' are considerably more common that light verb constructions with compound action nouns such as (\ref{ex:zgrWtChW}).

The compound \japhug{mɲaʁmtsaʁ}{grasshoper} from \japhug{tɯ-mɲaʁ}{eye} and \japhug{mtsaʁ}{jump} is obscure, but unlikely to be a possessive compound `whose eyes jump', and should rather be analyzed as an adjunct compound (maybe `jumping with (big) eyes', as if from a comitative adverb like \japhug{kɤ́mɲɯmɲaʁ}{with eyes} §\ref{sec:comitative.adverb}). If this analysis is correct, \forme{mɲaʁmtsaʁ} is the only example of \textit{actor} Adjunct-Verb compound.

\subsection{Verb-Noun compounds} \label{sec.v.n.compounds}
Verb-Noun compounds are extremely rare in Japhug, as they are in general in Trans-Himalayan languages other than Chinese. 

Adjectival stative verbs nearly always occur as second element in compounds with a noun (§\ref{sec:subject.verb.compounds}), but the opposite order is attested in \japhug{sɤŋaβdi}{unpleasant smell} from \japhug{tɤ-di}{smell} and \japhug{sɤŋaβ}{be unpleasant} (on which see §\ref{sec:denom.sA.proprietive}) a noun which can occur with the intransitive verb \japhug{mnɤm}{smell} as in (\ref{ex:sANABdi}).

\begin{exe}
\ex \label{ex:sANABdi}
\gll sɤŋaβ-di ʑo ɲɯ-mnɤm \\
be.unpleasant-smell \textsc{emph} \textsc{sens}-smell \\
\glt `There is an unpleasant smell.' (elicited)
\end{exe}

A possible example of Verb-Noun compound with a transitive verb is \japhug{ndzɤpri}{brown bear}, comprising \japhug{pri}{bear} and \japhug{ndza}{eat} -- as shown by (\ref{ex:ndzApri}) from a text about bears, it is considered by some native speakers of Japhug as a man eater, though this explanation could be folk-etymology. Note that this compound is also anomalous in that when transitive verbs are used in compounds with a noun, that noun is either an object (§\ref{sec:object.verb.compounds}) or adjunct (§\ref{sec:adjunct.verb.compounds}), never the subject.

\begin{exe}
\ex \label{ex:ndzApri}
\gll tɕe ndzɤpri kɤ-ti nɯ tɕe tɯrme tu-kɯ-ndza ɲɯ-ŋgrɤl \\
\textsc{lnk} brown.bear \textsc{inf}-say \textsc{dem} \textsc{lnk} people \textsc{ipfv}-\textsc{genr}:S/O-eat \textsc{sens}-be.usually.the.case \\
\glt `The one called \forme{ndzɤpri} (`the eating bear'), it eats people.' (21-pri)
\japhdoi{0003580\#S91}
\end{exe} 

We find several examples of nominal compounds whose structure is \forme{tɤ-}+Verb +Noun, where the verb is an adjectival stative verb. This category includes \textit{\japhug{tɤqiaβjmɤɣ}{Lactarius sp.}}, literally `bitter mushroom', from the noun \japhug{tɤjmɤɣ}{mushroom} (see §\ref{sec:frozen.indef} concerning the lost of \forme{tɤ-}) and the verb \japhug{qiaβ}{be bitter}, or \japhug{tɤmbextsa}{type of shoes} from \japhug{tɯ-xtsa}{shoe} and \japhug{mbe}{be old}. These should not be analyzed as Verb-Noun compounds however, as the first element originates from a nominalized form of the verb (either a property noun deriving from a verb (§\ref{sec:bare.action.nominals}) in indefinite possessor \forme{tɤ-} form (§\ref{sec:property.nouns}) or a \forme{tɤ-} abstract noun, §\ref{sec:tA.abstract.nouns}): they are rather a subtype of Noun-Noun compounds.

The same applies to compounds whose first element comes from a participle, such as \japhug{kɤrŋijmɤɣ}{type of mushroom} from \japhug{tɤjmɤɣ}{mushroom} with the subject participle \japhug{kɯ-ɤrŋi}{green one} \ipa{kɤrŋi} from the verb \japhug{arŋi}{be green} (see additional examples in §\ref{sec:lexicalized.subject.participle}). Note that the compounding order is unexpected, as participles of adjectival stative verbs generally follow the noun (§\ref{ex:attributive.participles.stative.verbs}). 
 

\section{Noun class prefixes} \label{sec:class.prefixes}
Noun class prefixes are prefixal elements that occur in some nouns, whose root cannot occur on its own, except for a few rare exceptions (such as \japhug{qapɣɤmtɯmtɯ}{hoopoe} discussed in §\ref{sec:uvular.animal}). Uvular \forme{qa-/χ-/ʁ-} and velar \forme{kɯ-/x-/ɣ-} prefixes are attested, and occur on animal names, plant names and nouns referring to traditional objects. Additional body part class prefixes, in particular \forme{m-} are also present in Japhug. 

Dental prefixal elements such as \forme{tɤ-} or \forme{tɯ-} are very common, but are better interpreted as frozen indefinite possessor prefixes (see §\ref{sec:frozen.indef}), rather as noun class prefixes.

\subsection{Uvular animal name prefix} \label{sec:uvular.animal}
The uvular animal prefix has a basic form \forme{qa-} (\tabref{tab:animal.qa}) and a reduced allomorph \forme{χ-/ʁ\trt}, attested in a few names like \japhug{ʁmbroŋ}{wild yak}, \japhug{rtɕʰɯʁjɯ}{caterpillar} and \japhug{tɕʰɯχpri}{salamander}.

Note that \japhug{ʁmbroŋ}{wild yak} is a borrowing from Tibetan \tibet{འབྲོང་}{ⁿbroŋ}{wild yak}, a fact that possibly suggests that the \forme{χ-/ʁ-} prefix has some degree of productivity (see \citealt{jacques14snom}). 

The noun \japhug{qapɣɤmtɯmtɯ}{hoopoe} is clearly a compound containing the bound state of \japhug{pɣa}{bird} and the reduplicated form of the noun \japhug{ɯ-mtɯ}{crest}, to which the class prefix \forme{qa-} has been added. The origin of this compound is still transparent to native speakers (see \ref{ex:qapGAmtWmtW}, §\ref{sec:causal.clauses}).

The allomorph \forme{qa-} is reduced to its non-syllabic variants \forme{χ-/ʁ-} when the prefixed noun occurs as the second member of a compound. The nouns \japhug{tɕʰɯχpri}{salamander} and \japhug{rtɕʰɯʁjɯ}{caterpillar} are examples of this reduction. The former is a compound of \forme{tɕʰɯ-} (a syllable borrowed from Tibetan \tibet{ཆུ་}{tɕʰu}{water}) and \forme{-χpri}, a variant of \japhug{qapri}{snake}. The latter comprises the syllable \forme{rtɕʰɯ\trt}, bound state of the unprefixed root of \japhug{tɯrtɕʰi}{type of vegetable (\zh{酸酸菜})}, and the second \forme{-ʁjɯ} is the reduced variant of \japhug{qajɯ}{worm}.

\begin{table}
\caption{Animal name \forme{qa-} prefix} \label{tab:animal.qa}
\begin{tabular}{l|l}
 \lsptoprule 
\japhug{qacʰɣa}{fox} &	\japhug{qandʐe}{earthworm} \\
\japhug{qaɕɣi}{big fly} &	\japhug{qandʐi}{anadromous fish} \\
\japhug{qaɕpa}{frog} &	\japhug{qandʑɣi}{falcon} \\
\japhug{qajdo}{crow} &	\japhug{qaɲi}{mole} \\
\japhug{qajtʂʰa}{aegyptius monachus} &	\japhug{qapar}{dhole} \\
\japhug{qajɯ}{worm} &	\japhug{qapɣɤmtɯmtɯ}{hoopoe} \\
\japhug{qaɟy}{fish} &	\japhug{qapri}{snake} \\
\japhug{qala}{rabbit} &	\japhug{qarma}{crossoptilon} \\
\japhug{qaliaʁ}{eagle} &	\japhug{qartsʰaz}{deer} \\
\japhug{qambalɯla}{butterfly} &	\japhug{qartsʰi}{cricket} \\
\japhug{qambrɯ}{male yak} &	\japhug{qaʑo}{sheep} \\
\japhug{qamtɕɯr}{shrew} &	\\
 \lspbottomrule
\end{tabular}
\end{table}

\subsection{Velar animal name prefix} \label{sec:velar.class.prefix}
While most nouns beginning in \forme{kɯ-} are frozen participles (see §\ref{sec:lexicalized.subject.participle}), there is a residue of forms which cannot be analyzed as deverbal nouns: no corresponding verb root is attested, and moreover some of them have cognates elsewhere in the family. \tabref{tab:animal.kW} presents animal names that are not derivable from any verb root, and appear to bear a \forme{kɯ-} class prefix, which is to be distinguished from the uvular one. Among these words, \japhug{kɯrtsɤɣ}{snow leopard} has a Tibetan cognate \tibet{གཟིག་}{gzig}{leopard} with a \forme{g-} preinitial, which is possibly related to the \forme{kɯ-} prefix in Japhug.
 
\begin{table}
\caption{Animal name \forme{kɯ-} prefix} \label{tab:animal.kW}
\begin{tabular}{ll}
 \lsptoprule 
\japhug{kɯɕpaz}{marmot} \\
\japhug{kɯjka}{pyrrhocorax} \\
\japhug{kɯmu}{Tetraogallus tibetanus} \\
\japhug{kɯpɤz}{type of bug} \\
\japhug{kɯrtsɤɣ}{snow leopard} \\
\japhug{kɯrŋi}{beast} \\
\japhug{kɯrnɯ}{mite} \\
 \lspbottomrule
\end{tabular}
\end{table} 

There is a handful of nouns with reduced allomorphs \forme{ɣ\trt}, \forme{x-} or even metathesized as \forme{βɣ-} in some words, corresponding to \forme{kə-} in Situ (see the phonological discussion in \citealt[6]{jacques14antipassive}), including \japhug{xɕiri}{weasel}, \japhug{xtɯt}{wild cat}, \japhug{ɣzɯ}{monkey}, \japhug{ɣni}{flying squirrel}, \japhug{βɣɯz}{badger} and \japhug{βɣɤza}{fly}. The same allomorphy is observed between the subject participle \forme{kɯ-} (§\ref{sec:subject.participles}) and the nominalization prefixes \forme{x-}/\forme{ɣ-} (§\ref{sec:G.nmlz}).

\subsection{Uvular plant name prefix} \label{sec:uvular.plant}
Some plant names have a uvular class prefix \forme{qa\trt}, including both cultivated and wild plants (and even plant parts), such as \japhug{qaɕti}{peach}, \japhug{qaɟɤɣi}{oat}, \japhug{qampʰoʁ}{oak leaves}, \japhug{qandzi}{type of fir}, \japhug{qaʑmbri}{vine}, \japhug{qawɯz}{edelweiss} and many others.
 
\subsection{Other uses of the uvular class prefix} \label{sec:uvular.other}
In addition to animal and plants names, the class prefix \forme{qa-} appears on some tools (\japhug{qajo}{earthen pot}, \japhug{qase}{leather rope}, \japhug{qarɤt}{rake}, 
\japhug{qapi}{white stone}), names of periods of the year (\japhug{qartsɯ}{winter}, \japhug{qartsɤβ}{harvest}), materials (\japhug{qandʑi}{tin}, \japhug{qambɯt}{sand}) or natural forces like \japhug{qale}{wind}.

The reduced form \forme{ʁ-} of the class prefix occurs with the noun \japhug{qale}{wind} in some compounds such as \japhug{akɯcʰoʁle}{east wind} and the abstract inalienably possessed noun \japhug{ɯ-ʁle}{reputation} (and the verbs derived from it, such as \japhug{raʁle}{be polite}).
 
\subsection{Body part noun prefixes} \label{ex:body.part.prefix}
The identification of class prefixes in body parts mainly rests on comparative evidence. Other Trans-Himalayan languages that preserve clusters such as Tibetan have in some names for body parts cluster that do not match those found in Japhug, for instance \tibet{མཁྲིས་པ་}{mkʰris.pa}{bile} and \tibet{སྐེ་}{ske}{neck} corresponding to the Japhug inalienably possessed nouns \japhug{tɯ-ɕkrɯt}{bile} and \japhug{tɯ-mke}{neck} (see §\ref{sec:body.part}), suggesting that body part class prefixes such as \forme{ɕ-} and \forme{m-} have been added to these words in Gyalrongic and Tibetan independently.

Apart from the \forme{m-} and \forme{ɕ-/ʑ-} class prefixes, some alienably possessed body parts such as \japhug{qambɣo}{earwax} have a \forme{qa-} prefix (§\ref{sec:body.part}).

\section{Nominal derivations}
Nominal derivations pale compared to the rich verbal (§\ref{sec:inner.prefixal.chain}) and even ideophonic (§\ref{sec:ideo:morpho}) derivations in Japhug. There is little derivational prefixation in nouns (aside from the collective \forme{kɤndʑi-} prefix and derivational uses of class prefixes, as seen in §\ref{sec:class.prefixes} above), and nearly all of the suffixes or quasi-suffixes involved in these derivations are traceable to inalienably possessed nouns that are still attested in the language, and have thus nearly no antiquity. 

\subsection{Privative} \label{sec:privative}
The suffix \forme{-lu} can be combined with the bound state form of body part nouns, without possessive prefix, to derive a noun meaning `...less', `without ...' that can be used as a modifier (§\ref{sec:unpossessible.nouns}). Examples attested in the corpus are indicated in \tabref{tab:privative.lu}, but this derivation appears to be productive.

\begin{table}
\caption{Privative \forme{-lu} suffix} \label{tab:privative.lu}
\begin{tabular}{Xll}
 \lsptoprule 
 Base Noun & Privative form \\
 \midrule
\japhug{ta-ʁrɯ}{horn} &\japhug{ʁrɯlu}{hornless} \\
\japhug{tɤ-jme}{tail} &\japhug{jmɤlu}{without tail} \\
\japhug{tɯ-jaʁ}{hand, arm} &\japhug{jaʁlu}{missing a hand} \\
\japhug{tɯ-ku}{head} &\japhug{kɤlu}{headless} \\
 \lspbottomrule
\end{tabular}
\end{table}

These privative forms can modify other nouns, and are placed after the nouns and before determiners such as demonstratives or numerals, as in (\ref{ex:RrWlu}) and (\ref{ex:jmAlu}).

\begin{exe}
\ex \label{ex:RrWlu}
\gll ʑɤni ɣɯ ftsoʁ ʁrɯlu ci ta-rku-nɯ ɲɯ-ŋu \\
\textsc{3du} \textsc{gen} female.hybrid.yak hornless \textsc{indef} \textsc{aor}:3\flobv{}-put.in-\textsc{pl} \textsc{sens}-be \\
\glt `They gave them a hornless female yak (to take with them back to the husband's home.' (2005-stod)
\end{exe}

Privative nouns are systematically glossed in Japhug with possessor participial relatives in \forme{kɯ-me} `not having' (§\ref{sec:S.possessor.relativization}), as in (\ref{ex:jmAlu}) (see also example \ref{ex:kAlu} from §\ref{sec:determinative.n.n}).
 

\begin{exe}
\ex \label{ex:jmAlu}
\gll tɕe kɯju jmɤlu nɯnɯ tɯrme ɲɯ-ŋu, ɯ-jme kɯ-me nɯ tɕe, tɕe kɯju jmɤlu nɯnɯ ɲɯ-sɲu ɕti tɕe nɯ nɯ-sɲu tɕe tɕe iɕqʰa tɯ-rɣi cʰɯ-kɯ-χtɤr nɯ nɯ-kɯ-sɲu tu-sɤrmi-nɯ. \\
\textsc{lnk} animal tailless \textsc{dem} man \textsc{sens}-be \textsc{3sg}.\textsc{poss}-tail \textsc{sbj}:\textsc{pcp}-not.exist \textsc{dem} \textsc{lnk} \textsc{lnk} animal tailless \textsc{dem} \textsc{ipfv}-be.crazy be:\textsc{aff}:\textsc{fact} \textsc{lnk} \textsc{dem} \textsc{aor}-be.crazy \textsc{lnk} \textsc{lnk} the.aforementioned \textsc{indef}.\textsc{poss}-seed \textsc{ipfv}-\textsc{sbj}:\textsc{pcp}-spread \textsc{dem} \textsc{aor}-\textsc{sbj}:\textsc{pcp}-be.crazy \textsc{ipfv}-call-\textsc{pl} \\
\glt `The `tailless animal' is the man, and `he becomes crazy', [when the crow says] that [people] became crazy, it means that they are sowing seeds.' (22-qajdo)
\japhdoi{0003596\#S46}
\end{exe}

The noun \japhug{ʁejlu}{left-handed} is not a privative noun deriving from \japhug{ʁe}{left}, but rather a compound comprising the property noun \forme{ɯ-jlu} (§\ref{sec:property.nouns}) as second element, in its grammaticalized meaning as a restrictive focus marker (§\ref{sec:restrictive.focus}), literally meaning `only (with) left (hand)'.


\subsection{Relative location} \label{sec:relative.location}
The prefix \forme{maŋ-} can be used to derive nouns referring to the position of the referent on either the vertical, the riverine or the solar dimensions (§\ref{sec:tridimensional.preverb}), which are encoded in verbal morphology, postpositions, relator nouns and locative adverbs (§\ref{sec:egressive}, §\ref{sec:relator.nouns.3d}, §\ref{sec:locative.adv}). These nouns are homophonous with corresponding verbs of relative location (§\ref{sec:verbs.relative.location}).

\begin{table} \small
\caption{Nouns of relative location and corresponding locative adverbs} \label{tab:nouns.relative.location}
\begin{tabular}{Xllll}
\lsptoprule
Locative adverb & Noun of location \\
\midrule
\forme{taʁ} & \japhug{maŋtaʁ}{the one on the upper side} \\
\forme{pa} & \japhug{maŋpa}{the one on the lower side} \\
\forme{lo} & \japhug{maŋlo}{the one upstream} \\
\forme{tʰi} & \japhug{maŋtʰi}{the one downstream} \\
\forme{kɯ} & \japhug{maŋkɯ}{the one on the east side} \\
\forme{ndi} & \japhug{maŋndi}{the one on the west side} \\
\lspbottomrule
\end{tabular}
\end{table}

The nouns of relative location can either occur as postnominal attributes, but can also (less commonly) be used their own as illustrated by (\ref{ex:Cnat.maNthi}).
\largerpage
\begin{exe}
\ex \label{ex:Cnat.maNthi}
\gll iɕqʰa nɯ, tʰi nɯ, kɯki ɕnat maŋtʰi ki tɤ-joʁ-a pɯ-ŋu tɕe, tʰam tɕe maŋlo nɯra tú-wɣ-joʁ ra \\
just.before \textsc{dem} downstream de \textsc{dem}.\textsc{prox} heddle downstream.one \textsc{dem}.\textsc{prox} \textsc{aor}-lift-\textsc{1sg} \textsc{pst}.\textsc{ipfv}-be \textsc{lnk} now \textsc{lnk} upstream.one \textsc{dem}.\textsc{pl} \textsc{ipfv}-\textsc{inv}-lift be.needed:\textsc{fact} \\
\glt `Just before, I had lifted the heddle on the lower (downstream) side [of the loom], now we have to lift the ones on the upper (upstream) side.' (video 20140429090403) \japhdoi{0003776\#S68}
\end{exe}
 
\subsection{Diminutive} \label{sec:diminutive}
There are four diminutive formations in Japhug, with the quasi-suffixes \forme{-pɯ}, \forme{-tsa}, \forme{-tɕɯ} and \forme{-li}.

The most productive is the \forme{-pɯ} suffixation. This transparent suffix comes from the noun \japhug{tɤ-pɯ}{offspring, young} (from Tibetan \tibet{བུ་}{bu}{son}). A diminutive formation based on the same noun also exists in Tibetan (\citealt{uray52diminutive}, \citealt[627]{hill14derivational}); whether the diminutive formation was independently innovated, or was borrowed from Tibetan is a question that needs to be further investigated. It is also attested in Situ (\citealt{zhang16bragdbar}, \citealt[151]{lai17khroskyabs}).

Earlier diminutives are formed with the bound state of the noun, for instance \japhug{tɕʰemɤpɯ}{little girl} from \japhug{tɕʰeme}{girl}, \japhug{staχpɯ}{pea} from \japhug{stoʁ}{broad bean}, or \japhug{kʰɯzɤpɯ}{puppy} from a non-attested form \forme{*kʰɯza}, propably itself the \forme{-tsa} diminutive of \japhug{kʰɯna}{dog}, borrowed from a Situ dialect.

More recent diminutives are directly formed with the base form, such as \japhug{qapripɯ}{little serpent}. This formation is extremely productive, and applies to plants, animals and even objects as in (\ref{ex:srWnloR}).

\begin{exe}
\ex \label{ex:srWnloR}
\gll tɕe srɯnloʁ-pɯ ci ɲɤ-kʰo tɕe \\
\textsc{lnk} ring-\textsc{dim} \textsc{indef} \textsc{ifr}-give \textsc{lnk} \\
\glt `He handed him a little ring.' (2011-4-smanmi)
\end{exe}

The suffix \forme{-pɯ} is recursive: examples of doubly suffixed nouns are found in the corpus, as in (\ref{ex:lhAndzxipWpW}) for instance. 

\begin{exe}
\ex \label{ex:lhAndzxipWpW}
\gll ɬɤndʐi-pɯ-pɯ nɯra kɯ, ɯ-pʰoŋbu nɯra ko-sɤlɤɣɯ-nɯ ri, \\
demon-\textsc{dim}-\textsc{dim} \textsc{dem}:\textsc{pl} \textsc{erg} \textsc{3sg}.\textsc{poss}-body \textsc{dem}:\textsc{pl} \textsc{ifr}-link-\textsc{pl} \textsc{lnk} \\
\glt `The little demonlings put back his body together, but...' (150909 xifangping-zh) \japhdoi{0006408\#S89}
\end{exe}

Suffixation with \forme{-pɯ} is the fused variant of the property noun construction with \japhug{ɯ-pɯ}{little one} described in §\ref{sec:property.nouns}.

A diminutive that is common to all Gyalrongic languages is the suffix \forme{-tsa}/\forme{-za} (Situ \forme{-tsa} or \forme{-za} (\citealt[163]{linxr93jiarong}), Khroskyabs \forme{-ze} / \forme{-zə} / \forme{-zɑ}, \forme{-tsi} (\citealt[158]{lai17khroskyabs}), Stau \forme{-zə}), found in fossilized forms in nouns such as \japhug{kʰɯtsa}{bowl} and \japhug{βɣɤza}{fly},\footnote{The noun \japhug{βɣɤza}{fly} is cognate to Brag-dbar \forme{kəvɐ̂s}, Khroskyabs \forme{jvɑzɑ́} (\citealt{zhang16bragdbar}, \citealt[156]{lai17khroskyabs}) and originates from proto-Gyalrong \forme{*kpɔs-tsa} (\citealt[53]{jacques08}). } but still visible in diminutive forms like \japhug{paʁtsa}{piglet} (from \japhug{paʁ}{pig}). It originates from the noun `son' that is lost in Japhug but still attested in Situ and Khroskyabs (Wobzi \forme{zî} `young man'). 

In Japhug the \forme{-tsa} diminutive is not very productive; it applies to some nouns that already have a \forme{-pɯ} diminutive such as \japhug{stoʁtsa}{name of plant} from \japhug{stoʁ}{broad bean} (besides \japhug{staχpɯ}{pea}).

The third diminutive suffix \forme{-tɕɯ}, like the two preceding ones, originates from a noun meaning `offspring', \japhug{tɤ-tɕɯ}{son}, and requires bound state.

It is used for animals (\japhug{kumpɣɤtɕɯ}{sparrow} from \japhug{kumpɣa}{fowl}) or inanimate objects (\japhug{kʰɤtɕɯ}{little house} from \japhug{kʰa}{house} or \japhug{lʁɤtɕɯ}{little gunny bag} from \japhug{lʁa}{gunny bag}). It occurs in some lexicalized forms such as \japhug{mbrɯtɕɯ}{knife}.\footnote{The root of this noun is metathesized from \forme{*mbɯr}; its cognates have a \forme{-tsa} diminutive in Situ (Brag-dbar \forme{mbərtsiɛ̄}, \citealt[228]{zhang16bragdbar}) and Khroskyabs (Wobzi \forme{(bərzé}, \citealt[115]{lai17khroskyabs}).}

The suffix \forme{-li} is the least productive of all diminutive formations, and the only one that cannot be traced to an existing noun. It appears is \japhug{tɕʰemɤli}{little girl} (a synonym of \japhug{tɕʰemɤpɯ}{little girl}) and in \japhug{rgali}{young cow}.
 
\subsection{Augmentative} \label{sec:augmentative}
A handful of nouns, some of Tibetan origin, have an augmentative form in \forme{-te}, originally from a property noun \forme{*ɯ-te} `big' (related to the verb \japhug{wxti}{be big}).

Augmentatives include \japhug{tɕɣomte}{cultivated xanthoxylum} (from \japhug{tɕɣom}{xanthoxylum}), \japhug{tɯjite}{big field} (from \japhug{tɯ-ji}{field}, name of several fields in Kamnyu), \japhug{tɕʰɯte}{big river} (from \tibet{ཆུ་}{tɕʰu}{water, river}) and the possessive compound \japhug{ŋgute}{person with a big head} (with \forme{ŋgu-} from Tibetan \tibet{འགོ་}{ⁿgo}{head, top}; this stem is not attested in Japhug as an independent word).

\subsection{Derogatory} \label{sec:derogatory}
There are three derogatory quasi-suffixes in Japhug, deriving designations of old or broken things: \forme{-do} and \forme{-mbe} `old X' and \forme{-ɴqra} `broken X'. These suffixes are the fused variants of the property nouns \japhug{ɯ-ɴqra}{broken one}, \japhug{ɯ-do}{old one} and \japhug{tɤ-mbe}{old thing} (see §\ref{sec:property.nouns}). 

The suffixes \forme{-do} and \forme{-mbe}, like their corresponding property nouns, differ in that the former occurs with animals and plants (\japhug{nɯŋa-do}{old cow}, \japhug{rɟɤlpu-do}{old king}), while the latter is used for inanimate objects.

 In a few cases, the suffixed noun is in bound state (as \japhug{kʰɤɴqra}{ruin} from \japhug{kʰa}{house} and \forme{-ɴqra}, or \japhug{kʰɯdo}{old dog} (from \japhug{kʰɯna}{dog} and \forme{-do}, see §\ref{sec:reduced.forms.compounds}). When the suffixed noun is inalienably possessed, addition of a derogatory suffix does not turn it into an alienably possessed noun, as in \japhug{tɯ-rcɤmbe}{old jacket} from \japhug{tɯ-rcu}{jacket} and \forme{-mbe}.

\subsection{Gentilic} \label{ex:gentilic.pW}
The gentilic suffix \forme{-pɯ} derives from the same noun \japhug{tɤ-pɯ}{offspring, young} from which the diminutive \forme{-pɯ} ultimately originates (see §\ref{sec:diminutive}). It is used to derive nouns referring to inhabitants of a certain place, and occurs without bound state. For instance, from the village names of \forme{kɤmɲɯ} (the village whose speech is described in this grammar) and \forme{snarndi} (a village in Tshobdun), one derives \japhug{kɤmɲɯ\-pɯ}{person from Kamnyu} and \japhug{snarndi\-pɯ}{person from Snarndi} (see the text 26-tshubdWnpW in the corpus). Given the high productivity of this derivation, these nouns are not indicated in the dictionary, as it would unnecessarily inflate the number of entries.

An alternative way to refer to the inhabitants of a place is by adding the plural \forme{ra} to it (§\ref{sec:place.names}).

\subsection{Gender} \label{sec:gender}
There is no morphological expression of gender in Japhug. For animals, the nouns \japhug{pʰu}{male} and \japhug{mu}{female} (from Tibetan \tibet{ཕོ་}{pʰo}{male} and \tibet{མོ་}{mo}{female}) can be used on their own (as in \ref{ex:phu.mu}) or occur as second member of compounds, as \japhug{kumpɣapʰu}{rooster} and \japhug{kumpɣamu}{hen} from \japhug{kumpɣa}{fowl}, or \japhug{lɯlɤmu}{female cat} from \japhug{lɯlu}{cat}, with bound state of the first noun.

\begin{exe}
\ex \label{ex:phu.mu}
\gll tɤkʰe pɣɤtɕɯ ndɤre pʰu mu saχsɤl \\
stupid bird:\textsc{dim} on.the.other.hand male female be.clear:\textsc{fact} \\
\glt `The male and the female of the `stupid bird', as opposed [to the birds previously discussed], are easy to distinguish.' (23-scuz)
\japhdoi{0003612\#S42}
\end{exe}

The suffixes \forme{-pa} and \forme{-mɯ} (from Tibetan \forme{-pa} and \forme{-mo}, respectively) also occur for a handful of nouns, some of Tibetan origin (\japhug{srɯnmɯ}{râkshasî} from \tibet{སྲིན་མོ་}{srin.mo}{râkshasî}) but also some local names such as \japhug{ɴɢarpa}{male one quarter yak hybrid} vs. \japhug{ɴɢarmɯ}{female one quarter yak hybrid}.

The noun \japhug{paʁɟu}{boar} from \japhug{paʁ}{pig} has a suffix \forme{-ɟu} that is not found in any other word.

For some farm animals, a lexical distinction is made between male and female animals (see \tabref{tab:lexical.gender}).

\begin{table}
\caption{Lexical distinction of male and female animals} \label{tab:lexical.gender}
\begin{tabular}{l|l}
 \lsptoprule 
 Male & Female \\
 \midrule
\japhug{qambrɯ}{male yak} & \japhug{qra}{female yak} \\
\japhug{jla}{male hybrid yak} & \japhug{ftsoʁ}{female hybrid yak} \\
\japhug{mbala}{bull} & \japhug{nɯŋa}{cow} \\
\japhug{zraβ}{buck} & (\japhug{tsʰɤnmu}{ewe}) \\
 \lspbottomrule
\end{tabular}
\end{table}

\subsection{Collective} \label{sec:collective}
While Japhug lacks number inflection, there are several collective derivations: the social relation collective, reduplicated collectives and the \textit{dvandva} collective.

\subsubsection{Social relation collective} \label{sec:social.collective}
The first type of collective is a noun prefixed in \forme{kɤndʑi-} and built either from kinship or social relation terms (which can be either inalienably or alienably possessed nouns), designating a group of people linked to one another by a specific relation.\footnote{There is some doubt about whether this prefix should be transcribed as 	\forme{kɤndʑɯ-} or \forme{kɤndʑi-}. See §\ref{sec:W.i.compounds} on the question of the contrast between \ipa{i} and \ipa{ɯ} following palatals and alveolo-palatals in non-final syllables.}

Two types of social relation collectives should be distinguished: reciprocal and non-reciprocal collectives.

Reciprocal collectives (\tabref{tab:reciprocal.collectives}) are derived from nouns designating a relationship in which all members of the group call each other by the same term: these can be non-kinship terms like `companion' or `friend' or kinship terms like \japhug{tɤ-sqʰaj}{sister} (of a female) (§\ref{sec:siblings.gender}). 

\begin{table}
\caption{Reciprocal social relation collectives} \label{tab:reciprocal.collectives}
\begin{tabular}{lllllll}
 \lsptoprule 
 Collective & Base noun \\
\midrule
\japhug{kɤndʑiɣɯfsu}{friends} & \japhug{ɣɯfsu}{friend} \\
\japhug{kɤndʑiβzaŋsa}{friends} & \japhug{βzaŋsa}{friend} \\
\japhug{kɤndʑiɕaχpu}{friends} & \japhug{ɕaχpu}{friend} \\
\japhug{kɤndʑikɯmdza}{relatives} & \japhug{kɯmdza}{relative} \\
\japhug{kɤndʑirɣa}{neighbours} & \japhug{tɤ-rɣa}{neighbour} \\
\japhug{kɤndʑislamaχti}{classmates} & \japhug{slamaχti}{classmate} \\
\japhug{kɤndʑisqʰaj}{sisters} & \japhug{tɤ-sqʰaj}{sister} (of a female) \\
\japhug{kɤndʑimɤtsa}{mother's sister's children} & \japhug{tɤ-mɤtsa}{mother's sister's child} \\
\japhug{kɤndʑitɤtɕɯχti}{friends (between males)} & \japhug{tɤtɕɯχti}{friend (between males)} \\
\japhug{kɤndʑitɕʰemɤχti}{friends (between female)} & \japhug{tɕʰemɤχti}{friend (between female)} \\
\japhug{kɤndʑixtɤɣ}{brothers} & \japhug{tɤ-xtɤɣ}{brother} (of a male)\\
\japhug{kɤndʑiχti}{companions} & \japhug{tɯ-χti}{companion} \\
\japhug{kɤndʑizda}{companions} & \japhug{tɯ-zda}{companion} \\
 \lspbottomrule
\end{tabular}
\end{table}

Non-reciprocal collectives (\tabref{tab:non.reciprocal.collectives}) are based on nouns designating unequal relationships, in which the members designate each other by different terms, in particular kinship terms involving relatives from different generations or different gender. Aside from kinship terms, groups comprising farm animals and their owners can also be formed by the same process from the name of the animal, as \japhug{kɤndʑimbro}{horseman and his horse} and \japhug{kɤndʑiftsoʁ}{female hybrid yak and its owners} (see example \ref{ex:kAndZWftsWftsoR} below).

Non-reciprocal collectives are either formed from one of the two nouns, which can be either from the lower \hbox{(\japhug{kɤndʑiɣe}{grandparents and grandchildren})} or the higher generation (\japhug{kɤndʑiɲi}{paternal aunt and her nephews}), or by a combination of two kinship terms, the first of which, in some cases, undergoes changes to the point of being barely recognizable (\japhug{kɤndʑipɤmdɯ}{paternal uncle and his nephews}).\footnote{In the case of \japhug{kɤndʑiwɤɬaʁ}{maternal aunt and her nephews}, the origin of the element \forme{-wɤ-} is not identifiable.}

\begin{table}
\caption{Non-reciprocal social relation collectives} \label{tab:non.reciprocal.collectives}
\fittable{
\begin{tabular}{lllllll}
 \lsptoprule 
 Collective & Base noun \\
\midrule
\japhug{kɤndʑiɣe}{grandparents and grandchildren} & \japhug{tɤ-ɣe}{grandchild} \\
\japhug{kɤndʑiʁi}{siblings} & \japhug{ta-ʁi}{younger sibling} \\
\japhug{kɤndʑime}{parents and daughter} & \japhug{ɯ-me}{daughter} \\
\japhug{kɤndʑiɲi}{paternal aunt and her nephews} & \japhug{tɤ-ɲi}{father's sister} \\
\midrule
\japhug{kɤndʑimbro}{horseman and his horse} & \japhug{mbro}{horse} \\
\japhug{kɤndʑislama}{master and disciple} & \japhug{slama}{student} \\
\japhug{kɤndʑijla}{male hybrid yak and its owners} & \japhug{jla}{male hybrid yak} \\
\japhug{kɤndʑiftsoʁ}{female hybrid yak and its owners} & \japhug{ftsoʁ}{female hybrid yak} \\
\japhug{kɤndʑipaʁ}{pig and its owners} & \japhug{paʁ}{pig} \\
\japhug{kɤndʑiqaʑo}{sheep and its owners} & \japhug{qaʑo}{sheep} \\
\japhug{kɤndʑitsʰɤt}{goat and its owners} & \japhug{tsʰɤt}{goat} \\
\midrule
\japhug{kɤndʑirpɯftsa}{maternal uncle and his nephews} & \japhug{tɤ-rpɯ}{mother's brother} (1)\\
& \japhug{tɤ-ftsa}{sister's child} (2)\\
\japhug{kɤndʑiwɤɬaʁ}{maternal aunt and her nephews} & \japhug{tɤ-ɬaʁ}{mother's sister} (2) \\
\japhug{kɤndʑipɤmdɯ}{paternal uncle and his nephews} & \japhug{tɤ-βɣo}{father's brother} (1)\\
& \japhug{tɤ-mdɯ}{brother's child} (2)\\
\japhug{kɤndʑiwɤmɯsnom}{brother and sisters} & \japhug{tɤ-wɤmɯ}{brother} (of a\\
&female) (1)\\
& \japhug{tɤ-snom}{sister} (of a male) (2)\\
 \lspbottomrule
\end{tabular}}
\end{table}

The collective nouns can be used as normal nouns and take flagging, numerals and other modifiers, as in  (\ref{ex:kAndZWxtAG.XsWm}).

\begin{exe}
\ex \label{ex:kAndZWxtAG.XsWm}
\gll kɤndʑi-xtɤɣ χsɯm pjɤ-tu-nɯ \\
\textsc{coll}-brother three \textsc{ifr}.\textsc{ipfv}-exist-\textsc{pl} \\
\glt `There were three brothers.' (07-deluge) \japhdoi{0003426\#S2}
\end{exe}

Social relation collectives are also found in Situ and Tshobdun (\citealt[107]{jackson98morphology}), where they have optional reduplication; in Japhug, partial reduplication (as in §\ref{sec:redp.coll}) is used by some speakers, as \japhug{kɤndʑiftsɯftsoʁ}{female hybrid yak and its owners} in example (\ref{ex:kAndZWftsWftsoR}), from a story by Kunbzang Mtsho.

\begin{exe}
\ex \label{ex:kAndZWftsWftsoR}
 \gll kɤndʑi-ftsɯ\redp{}ftsoʁ χsɯm nɯ, tsʰɯntsʰɯn kɯ-pa kɤ-nɯ-ɬoʁ-nɯ ɲɯ-ŋu, \\
 \textsc{coll}-female.yak.hybrid three \textsc{dem} \textsc{idph}(II):in.order \textsc{inf}:\textsc{stat}-\textsc{aux} \textsc{aor}:\textsc{east}-\textsc{auto}-come.out-\textsc{pl} \textsc{sens}-be \\
\glt `[The girl, her husband] and their female hybrid yak crossed [the large river] without damage.' (2003 Kunbzang)
\end{exe}

The lists in Tables \ref{tab:reciprocal.collectives} and \ref{tab:non.reciprocal.collectives} comprise most common social relation collectives, but are by no means complete lists. For instance, next to \japhug{kɤndʑirpɯftsa}{maternal uncle and his nephews} from \japhug{tɤ-rpɯ}{mother's brother} and \japhug{tɤ-ftsa}{sister's child}, the terms \japhug{kɤndʑirpɯ}{maternal uncle and his nephews} and \japhug{kɤndʑiftsa}{nephew with his maternal uncles and aunts} are also possible though less common. However, some combinations are considered incorrect. For instance, Tshendzin considers that $\dagger$\forme{kɤndʑirʑaβ} (from \japhug{tɤ-rʑaβ}{wife}) is only found in children's language (\forme{nɯ tɤ-pɤtso ra kɯ tu-ti-nɯ ŋgrɤl} `children talk like that'), as the correct term is \japhug{ʁzɤmi}{husband and wife} from Tibetan \tibet{བཟའ་མི་}{bza.mi}{husband and wife}.

There is in addition an irregular collective \japhug{kɤtsa}{parents and children}, with the same element \forme{-tsa} found in some diminutives (see §\ref{sec:diminutive}), from an earlier word for `child'.

It can be used without any preceding noun as in (\ref{ex:kAtsa.ra}), but more commonly serves as the modifier of a kinship term, as in (\ref{ex:tAmu.kAtsa}) (note also \forme{tɤ-tɕɯ kɤtsa} `father and son' and \forme{tɕʰeme kɤtsa} `mother and daughter' from \japhug{tɤ-tɕɯ}{son, boy} and \japhug{tɕʰeme}{girl}).

\begin{exe}
\ex \label{ex:kAtsa.ra}
\gll tɕe tɤ-mu nɯ kɯ ɯ-pɯ nɯnɯ ju-ɕpʰɣɤm tɕe, ʑara kɤtsa ra stɯsti ʁɟa ʑo ɕe-nɯ ɲɯ-ra. \\
\textsc{lnk} \textsc{indef}.\textsc{poss}-mother \textsc{dem} \textsc{erg} \textsc{3sg}.\textsc{poss}-young \textsc{dem} \textsc{ipfv}-flee.with[III] \textsc{lnk} \textsc{3pl} parents.and.children \textsc{pl} alone completely \textsc{emph} go:\textsc{fact}-\textsc{pl} \textsc{sens}-be.needed \\
\glt `And the mother (i.e. lioness) flees with her cubs, and they (i.e. mother and children) have to go alone (without the father).' (20-sWNgi)
\japhdoi{0003562\#S71}
\end{exe}

\begin{exe}
\ex \label{ex:tAmu.kAtsa}
\gll tɤ-mu kɤtsa ci pjɤ-tu-ndʑi tɕe \\
\textsc{indef}.\textsc{poss}-mother parents.and.children \textsc{indef} \textsc{ifr}.\textsc{ipfv}-exist-\textsc{du} \textsc{lnk} \\
\glt `There was a mother and her son.' (tWJo 2012)
\japhdoi{0004089\#S4}
\end{exe}

Like comitative adverbs (§\ref{sec:comitative.adverb}), it is clear that social relation collectives originate from participles of denominal verbs. The only example of the verbal denominal \forme{andʑɯ-} derivation from which they originate is \japhug{andʑirɣa}{be neighbours} from the inalienably possessed noun \japhug{tɤ-rɣa}{neighbour} (§\ref{sec:denom.andZi}). The social relation collective \japhug{kɤndʑirɣa}{neighbours} can thus be analyzed as the participle of this verb \forme{kɯ-ɤndʑirɣa}. 

However, since the \forme{andʑi-} denominal derivation attested only in this single example, from a synchronic point of view it is better to consider this collective formation as a strictly nominal derivation.

\subsubsection{Reduplicated collectives} \label{sec:redp.coll}
The reduplicated collectives are built using partial reduplication. There are three different patterns.
 
First, some nouns allow standard partial reduplication with \forme{-ɯ} in the reduplicated syllable (§\ref{sec:partial.redp}) expressing a vague collective. This reduplication can apply to loanwords from Tibetan, such as \japhug{χsɯ\redp{}χsɤr}{things in gold} and \japhug{rŋɯ\redp{}rŋɯl}{things in silver} from \japhug{χsɤr}{gold} and \japhug{rŋɯl}{silver} (Tibetan \tibet{གསེར་}{gser}{gold} and \tibet{དངུལ་}{dŋul}{silver}).

\begin{exe}
\ex 
\gll a-χsɯ\redp{}χsɤr ra, a-rŋɯ\redp{}rŋɯl ra mɤ-ra kɯ ɕom rɟɤskɤt ɯ-taʁ tu-ɕe-a ŋu \\
\textsc{1sg}.\textsc{poss}-\textsc{coll}\redp{}gold \textsc{pl} \textsc{1sg}.\textsc{poss}-\textsc{coll}\redp{}silver \textsc{pl} \textsc{neg}-be.needed:\textsc{fact} \textsc{erg} iron stairs \textsc{3sg}.\textsc{poss}-on \textsc{ipfv}:\textsc{up}-go-\textsc{1sg} be:\textsc{fact} \\
\glt `I don't need things in gold or silver, I will go up the iron stairs.' (not the golden or silver stairs, 2005-Kunbzang)
\end{exe}

Some nouns, which only appear in a reduplicated form are presumably ancient collectives, like \japhug{kʰrambaχtɯχtɤm}{lies} from a possible non-reduplicated form \forme{*kʰrambaχtɤm} (from \tibet{ཁྲམ་པ་གཏམ་}{kʰram.pa.gtam}{deceiving words}).

%tɕendɤre nɯnɯ ʑɯ-ʑɯmkhɤm ʑo nɯ, nɯ-rɟɤlpu ɣɯ ɯ-me ci ʑo staʁlu pjɤ-tu nɯ ma pjɤ-k-ɤrɕo-ci, 
%nyimaowdzer2002, 86

Second, we find some reduplicated collectives with the vowel \ipa{a}, not \ipa{ɯ}, in the replicated syllable. There are only a few examples, as shown in \tabref{tab:coll.n}, but several of them are borrowings from Tibetan. In one case, \japhug{fɕafɕɤt}{words}, the base word is a transitive verb (\japhug{fɕɤt}{tell}).

\begin{table}
\caption{Collective noun derivation} \label{tab:coll.n}
\begin{tabular}{Xlll}
 \lsptoprule 
 Base form & Collective & Tibetan \\
 \midrule
\japhug{rdɯl}{dust} & \japhug{rdardɯl}{dust, dirt} & \tibet{རྡུལ་}{rdul}{dust} \\
\japhug{tɯ-ntɕʰɯr}{fragment} & \japhug{ɯ-ntɕʰantɕʰɯr}{fragments} & \\
\japhug{ɯ-zɯr}{side} & \japhug{ɯ-zarzɯr}{sides} & \tibet{ཟུར་}{zur}{side, corner} \\
\japhug{ɯ-rkɯ}{side} & \japhug{ɯ-rkarkɯ}{sides} & \\
\japhug{fɕɤt}{tell} & \japhug{fɕafɕɤt}{words} & \tibet{བཤད་}{bɕad}{explain, tell} \\
 \lspbottomrule
\end{tabular}
\end{table}

Reduplicated collective nouns in \forme{a-} can be used without a number clitic, as in (\ref{ex:WntChantChWr}), but they often appear with the \japhug{ra}{plural} as in (\ref{ex:rdardWl}).

\begin{exe}
\ex \label{ex:WntChantChWr}
\gll znɤrɣama nɯ mtʰa ɯ-kɤcu ŋu. tɕe nɯnɯtɕu tɯ-ji ɯ-ntɕʰantɕʰɯr pɯ-dɤn, jinde kʰro ɲɤ-s-qapɯ-nɯ,\\
\textsc{anthr} \textsc{dem} \textsc{anthr} \textsc{3sg}.\textsc{poss}-east be:\textsc{fact} \textsc{lnk} \textsc{dem}:\textsc{pl} \textsc{indef}.\textsc{poss}-field \textsc{3sg}.\textsc{poss}-fragment:\textsc{coll} \textsc{pst}.\textsc{ipfv}-be:many now much \textsc{ifr}-\textsc{caus}-be.fallow-\textsc{pl}\\
\glt `Znargama (`The place where one calls the rain') is on the east of Mtha, there used to be many little fragments of fields, but now people have left them become fallow.' (150903 kAmYW tWji3)
\japhdoi{0006288\#S20}
\end{exe}

\begin{exe}
\ex \label{ex:rdardWl}
\gll tɕe tɤɕi nɯ tú-wɣ-χtɕi tɕʰɣaʁtɕʰɣaʁ ʑo tɕe, rdardɯl nɯra ɲɯ́-wɣ-ɣɤ-me tɕe \\
\textsc{lnk} barley \textsc{dem} \textsc{ipfv}-\textsc{inv}-wash \textsc{idph}(II):completely.clean \textsc{emph} \textsc{lnk} dush:\textsc{coll} \textsc{dem}:\textsc{pl} \textsc{ipfv}-\textsc{inv}-\textsc{caus}-not.exist \textsc{lnk} \\
\glt `Then one washes the barley very thoroughly, one removes all the dirt.' (2002tWsqar)
\end{exe}
 
The noun \japhug{rgargɯn}{old person} has the form of a collective noun as those in \tabref{tab:coll.n}, but it is commonly used with singular or dual referents (as in \ref{ex:rgargWn}). It could be analyzed as the collective form of a loanword from Tibetan \tibet{རྒན་པོ་}{rgan.po}{old person}, though the expected form would have been $\dagger$\forme{rga-rgɤn}. 
 
\begin{exe}
\ex \label{ex:rgargWn}
\gll rgargɯn ni kɤ-fstɯn pɯ-ra \\
old.person \textsc{du} \textsc{inf}-serve \textsc{pst}.\textsc{ipfv}-be.needed \\
\glt `She had to take care of two old people.' (14-siblings) \japhdoi{0003508\#S33}
\end{exe}

A third reduplicated collective derivation is only attested by one example, the form \japhug{qajɯqaja}{all kinds of worms} (see \ref{ex:qajWqaja}) which derives from \japhug{qajɯ}{worm} by reduplicating the whole word and changing the last rhyme to \ipa{-a}, a reduplication template reminiscent of that found in Khroskyabs (see \citealt{lai13fuyin}, \citealt[22--24]{lai17khroskyabs}).

\begin{exe}
\ex \label{ex:qajWqaja}
\gll
tɯ-ci ɯ-ŋgɯ qajɯqaja tʰamtɕɤt, sɯŋgɯ ɣɯ ɯ-rɯdaʁ kɯ-xtɕi kɯ-wxti, mɤʑɯ pɣa nɯnɯra lonba ʑo kɤ-fsraŋ kɯ-ra ɲɯ-ɕti ma \\
\textsc{indef}.\textsc{poss}-water \textsc{3sg}-inside worm:\textsc{coll} all forest \textsc{gen} animal \textsc{sbj}:\textsc{pcp}-be.small \textsc{sbj}:\textsc{pcp}-be.big yet bird \textsc{dem}:\textsc{pl} all \textsc{emph} \textsc{inf}-protect \textsc{inf}:\textsc{stat}-be.needed \textsc{sens}-be:\textsc{aff} \textsc{lnk} \\
\glt `All the creatures in the water, the small and big animals of the forest, and also the birds have to be protected.' (160703 jingyu)
\japhdoi{0006169\#S43}
\end{exe}

A fourth type of collective has the vowel \forme{-e} in the reduplicated syllable. It is attested in the noun \japhug{ɯ-ʁɟoʁɟe}{all kinds of diluted drinks} from \japhug{ɯ-ʁɟo}{diluted drink} (derived from the verb \japhug{ʁɟo}{rinse}).

\subsubsection{Dvandva collective} \label{sec:dvandva.coll}
The \textit{dvandva} collective is derived from two nouns, the first one in bound state followed by the element \forme{-lɤ-} and then by the second noun stem without a possessive prefix. All known forms, some of which have Tshobdun cognates,\footnote{The nouns \japhug{tɯ-mɤlɤjaʁ}{limbs} and \japhug{ɯ-kɤlɤjme}{head upside down} correspond to \forme{o-kolɐjmɐ} `head and tail' \citep[533]{jackson19tshobdun} and \forme{o-melɐ́ja} `limbs', respectively \citep[276]{jackson19tshobdun}.} are listed in \tabref{tab:dvandva.coll.n}.

 \begin{table}
\caption{Dvandva collectives} \label{tab:dvandva.coll.n}
\begin{tabular}{Xllll}
 \lsptoprule 
Collective & First noun & Second Noun \\
 \midrule
 \japhug{tɯ-kɤlɤmɲaʁ}{facial features} & \japhug{tɯ-ku}{head} & \japhug{tɯ-mɲaʁ}{eye} \\
\japhug{tɯ-mɤlɤjaʁ}{limbs} & \japhug{tɯ-mi}{foot, leg} & \japhug{tɯ-jaʁ}{hand, arm} \\
 \japhug{ɯ-kɤlɤjme}{head upside down} & \japhug{tɯ-ku}{head} & \japhug{tɤ-jme}{tail} \\
 \japhug{kɯmɤlɤxso}{useless} & \forme{kɯ-me} `not existing' & \japhug{ɯ-xso}{empty, normal} \\
 \lspbottomrule
\end{tabular}
\end{table}

Among the examples in \tabref{tab:dvandva.coll.n}, \japhug{ɯ-kɤlɤjme}{head upside down} and  \japhug{kɯmɤlɤxso}{useless}, `in vain' are mainly used adverbially. The first one mostly occurs with verbs such as \japhug{ɕtʰɯz}{turn towards} and \japhug{ru}{look at} at, as in (\ref{ex:WkAlAjme}).\footnote{See a definition of this noun in (\ref{ex:WkAlAjme.pjWkACthWz}), §\ref{sec:inf.citation}.}

\begin{exe}
\ex \label{ex:WkAlAjme}
 \gll tɕe nɯ ɯ-sta nɯ lɤtɕʰom nɯ ɲɯ́-wɣ-ʁɟo ʑo kʰrɯŋkʰrɯŋ ʑo qʰe tɕe ɯ-kɤlɤjme pjɯ́-wɣ-ɕtʰɯz qʰe, ɯ-mŋu nɯ pa pjɯ́-wɣ-ɕtʰɯz \\
 \textsc{lnk} \textsc{dem} \textsc{3sg}.\textsc{poss}-place \textsc{dem} churning.bucket \textsc{dem} \textsc{ipfv}-\textsc{inv}-rinse \textsc{emph} \textsc{idpf}:II:completely.clean \textsc{emph} \textsc{lnk} \textsc{lnk} \textsc{3sg}.\textsc{poss}-head.upside.down \textsc{ipfv}:\textsc{down}-\textsc{inv}-turn.towards \textsc{lnk} \textsc{3sg}.\textsc{poss}-opening \textsc{dem} down  \textsc{ipfv}:\textsc{down}-\textsc{inv}-turn.towards \\
 \glt `One rinses the churning bucket very clean, and put it upside down at its place, the opening down.' (30-macha)
\japhdoi{0003746\#S62}
\end{exe}

The noun \japhug{kɯmɤlɤxso}{useless}, `superfluous', combines the subject participle of \japhug{me}{not exist} with the property noun \japhug{ɯ-xso}{empty, normal} (a lexicalized participle, whose uses and etymology are described in §\ref{sec:property.nouns}). It can be used as predicate with a copula (\ref{ex:kWmAlAxso}), but often occurs in adverbial use meaning `in vain', `for nothing' or `doing nothing' as in (\ref{ex:kWmAlAxso.kutWrAZindZi}). 

\begin{exe}
\ex \label{ex:kWmAlAxso}
 \gll nɯʑora \textbf{kɯmɤlɤxso} ɲɯ-tɯ-ɕti-nɯ ma tɯ-nɤma-nɯ maŋe! \\
 \textsc{2pl} in.vain \textsc{sens}-2-be.\textsc{aff} \textsc{lnk} 2-work:\textsc{fact}-\textsc{pl} not.exist:\textsc{sens} \\
 \glt `You are useless, you don't do any work.' (2003 Kunbzang)
\end{exe}
 
\begin{exe}
\ex \label{ex:kWmAlAxso.kutWrAZindZi}
 \gll ndʑi-<zuoye> pɯ-βzu-ndʑi ra ma \textbf{kɯmɤlɤxso} ku-tɯ-rɤʑi-ndʑi mɤ-jɤɣ \\
 \textsc{2du}.\textsc{poss}-homework \textsc{imp}-make-\textsc{du} be.needed:\textsc{fact} \textsc{lnk} in.vain \textsc{ipfv}-2-stay-\textsc{du} \textsc{neg}-be.allowed:\textsc{fact} \\
 \glt `Do your homework, don't stay there doing nothing.' (conversation, 14-05-10)
\end{exe}

The \forme{-lɤ-/-la-} element found in collective \textit{dvandva}-s is also attested in approximate numerals (§\ref{sec:approx.numerals}) and in adverbs such \japhug{tɯxpalɤskɤr}{during the whole year} (from \japhug{tɯ-xpa}{one year} and \japhug{fskɤr}{turn around}) and \japhug{rtsɯɕaŋlaŋmtɕɤt}{all the plants} (from \japhug{rtsɯɕaŋ}{plant} and \japhug{tʰamtɕɤt}{all}, respectively from Tibetan \tibet{རྩི་ཤིང༌}{rtsi.ɕiŋ}{plant} and \tibet{ཐམས་ཅད་}{tʰams.tɕad}{all}). Another possible trace of this \forme{lɤ-} element is found in the adverb \forme{lɤqʰɤtɤmbɤt} `(distance of) several mountain ranges' (\ref{ex:lAqhAtAmbAt}), which contains the noun \japhug{tɤmbɤt}{mountain} and perhaps the bound state \forme{qʰɤ-} from \japhug{ɯ-qʰu}{after, behind}.

\begin{exe}
\ex \label{ex:lAqhAtAmbAt}
 \gll ji-pɤrtʰɤβ lɤqʰɤtɤmbɤt tu \\
 \textsc{1pl}.\textsc{poss}-between several.mountains exist:\textsc{fact} \\
 \glt `There is a distance of several mountain ranges between us.' (elicited)
 \end{exe}
 
 Another linking morpheme \forme{-mɤ-} instead of \forme{-lɤ-} is found in the possessed noun \japhug{ɯ-ŋgɯmɤpɕi}{the inside and the outside} from the locative relator nouns \japhug{ɯ-ŋgɯ}{inside} and \japhug{ɯ-pɕi}{outside} (the latter from Tibetan \tibet{ཕྱི་}{pʰʲi}{outside}, §\ref{sec:other.locative.relator}).

\subsection{Superlative} \label{sec:superlative.XCWX}
While there is no adjectival superlative derivation in Japhug (the available constructions to express this meaning are described in §\ref{sec:superlative}), we find nevertheless a derivation applied to locative nouns (§\ref{sec:relator.location}), expressing the furthest location. As shown in \tabref{tab:superlative.n}, it is built by adding an element \forme{-ɕɯ-} followed by a complete copy of the root of the noun without bound state alternation or partial replication; the resulting noun is still an inalienably possessed locative noun. Example (\ref{ex:WqaCWqa}) illustrates the use of one of these forms.

\begin{table} \small
\caption{Superlative noun derivation} \label{tab:superlative.n}
\begin{tabular}{Xllll}
 \lsptoprule
\japhug{tɯ-ku}{head}, `top' & \japhug{ɯ-kuɕɯku}{the highest place} \\
\japhug{tɤ-qa}{paw, root}, `bottom' & \japhug{ɯ-qaɕɯqa}{the deepest place} \\
\japhug{ɯ-rkɯ}{side} & \japhug{ɯ-rkɯɕɯrkɯ}{the furthest place on the side} \\
\japhug{ɯ-zɯr}{side} & \japhug{ɯ-zɯrɕɯzɯr}{the furthest place on the side} \\
 \lspbottomrule
\end{tabular}
\end{table}

\largerpage
\begin{exe}
\ex \label{ex:WqaCWqa}
\gll rɟɤmtsʰu ɯ-qaɕɯqa pjɯ-ɕe tɕe, nɯnɯ ɯ-kɤ-nɤ-mɯm nɯra ɕ-tu-nɯ-tɕɤt ɲɯ-ŋu. \\
ocean \textsc{3sg}.\textsc{poss}-bottom:\textsc{super} \textsc{ipfv}:\textsc{down}-go \textsc{lnk} \textsc{dem} \textsc{3sg}.\textsc{poss}-\textsc{obj}:\textsc{pcp}-\textsc{trop}-be.tasty \textsc{dem}:\textsc{pl} \textsc{tral}-\textsc{ipfv}-\textsc{auto}-take.out \textsc{sens}-be \\
\glt `[The sperm whale] goes to the lowest depths of the ocean and catches the things it likes to eat.' (160703 jingyu) \japhdoi{0006169\#S24}
\end{exe}

\subsection{Unattested derivations} \label{sec:non.existing.derivation}

The only negative morphology possible on Japhug nouns is the privative derivation (§\ref{sec:privative}). Negative prefixes only occur verbs (§\ref{sec:negation}), and the only way to express non-privative negation on nouns is by using a relative clause (§\ref{sec:negation.noun}).

There is no nominal tense derivation corresponding to the English prefix \textit{ex}-. This meaning can only be expressed by participial relatives with the past imperfective participle of the copula \forme{pɯ-kɯ-ŋu} (\ref{ex:WnmaR.pWkWNu2}) (§\ref{sec:subject.participle.other.prefixes}).

\begin{exe}
\ex \label{ex:WnmaR.pWkWNu2}
\gll ɯ-nmaʁ pɯ-kɯ-ŋu \\
\textsc{3sg}.\textsc{poss}-husband \textsc{pst}.\textsc{ipfv}-\textsc{sbj}:\textsc{pcp}-be \\
\glt `Her ex-husband (the one who used to be her husband).' (several attestations)
 \end{exe}
 
In addition, although is is possible to derive abstract nouns from verbs (§\ref{sec:degree.nominals}, §\ref{sec:tA.abstract.nouns}, §\ref{sec:action.nominal.compounds}), there is no direct way of deriving an abstract noun from a noun. There is however an indirect way of doing it by building a denominal verb, and then to subjecting it to a nominalizing derivation. For instance, from the inalienably possessed \japhug{tɤ-mdzu}{thorn}, the proprietive verb \forme{aɣɯ-mdzu} `be thorny, have a lot of thorns' can be derived (§\ref{sec:denom.aGW}) and its (productive) degree noun \forme{ɯ-tɯ-ɤɣɯ-mdzu} `its degree of thornity' is well attested in the degree construction (§\ref{sec:degree.monoclausal}) as shown in (\ref{ex:WtAGWmdzu}).

\begin{exe}
\ex \label{ex:WtAGWmdzu}
\gll ɯ-tɯ-ɤɣɯ-mdzu saχaʁ \\
\textsc{3sg}.\textsc{poss}-\textsc{prop}:\textsc{denom}-thorn be.extremely:\textsc{fact} \\
\glt `It is very thorny (its degree of having thorns is extreme).' (18-NGolo) \japhdoi{0003530\#S70}
\end{exe}

 
\section{Denominal adverbs and postpositions} \label{sec:denominal.adverb}

\subsection{Comitative adverbs} \label{sec:comitative.adverb}
Comitative adverbs are productively derived from nouns. Their meaning is `having $X$' , `together with $X$', `including $X$' or in the case of clothes or covers `wearing $X$'.

Comitative adverbs are built by partially reduplicating the last syllable of the noun stem (following the morphophonological rules in §\ref{sec:partial.redp}) and prefixing either \forme{kɤ́-} or \forme{kɤɣɯ-}. This derivation applies to native words and loanwords from Tibetan. From instance, \japhug{χɕɤlmɯɣ}{glasses} (from Tibetan \tibet{ཤེལ་མིག་}{ɕel.mig}{glasses}) yields \forme{kɤ́-χɕɤlmɯ\tld{}lmɯɣ} or \forme{kɤɣɯ-χɕɤlmɯ\tld{}lmɯɣ} `together with glasses; wearing glasses'.\footnote{Note that reduplication applies accross morpheme boundaries, as the coda of \japhug{χɕɤl}{glass} (from \tibet{ཤེལ་}{ɕel}{glass}) is reduplicated with the following syllable. } 

No semantic difference between the comitative adverbs in \forme{kɤ́-} and those in \forme{kɤɣɯ-} has been detected: Both are fully productive and can be built from the same nouns. As argued in \citet{jacques17comitative}, the \forme{kɤɣɯ-} form is inherited (from proto-Gyalrong \forme{*kɐwə-}), while \forme{kɤ́-} is borrowed from Tshobdun \forme{ko\trt}, the exact cognate of \forme{kɤɣɯ-} \citep[107]{jackson98morphology}. The prefix \forme{kɤɣɯ-} and its Tshobdun cognate \forme{ko-} both originate from the participle \forme{kɯ-} (§\ref{sec:subject.participles}) of the proprietive \forme{aɣɯ-} denominal derivation (§\ref{sec:denom.aGW.comitative}), attesting a \textsc{proprietive} $\Rightarrow$ \textsc{comitative} grammaticalization pathway \citep{jacques17comitative}. 

When the base noun is inalienably possessed, it is possible to build a comitative adverb with the indefinite possessor prefix or with the bare stem. For instance, from \japhug{tɤ-rte}{hat} one can derive both \forme{kɤ́-rtɯ\tld{}rte} / \forme{kɤɣɯ-rtɯ\tld{}rte} `with his/her hat' and \forme{kɤ́-tɤ-rtɯ\tld{}rte} / \forme{kɤɣɯ-tɤ-rtɯ\tld{}rte} `with a/the hat' with the indefinite possessor prefix \forme{tɤ-}. These two sets of forms have different meanings: the former \forme{kɤ́-rtɯ\tld{}rte} / \forme{kɤɣɯ-rtɯ\tld{}rte} mean `wearing one's hat' (example \ref{ex:kAGWrtWrte}), while the latter \forme{kɤ́-tɤ-rtɯ\tld{}rte} / \forme{kɤɣɯ-tɤ-rtɯ\tld{}rte} imply that the subject is not wearing the hat (\ref{ex:kAGWtArtWrte}); preserving the indefinite possessor in the derived form alienabilizes the inalienably possessed noun (see §\ref{sec:alienabilization}).

\begin{exe}
\ex \label{ex:kAGWrtWrte}
\gll kɤɣɯ-rtɯ\tld{}rte ʑo kʰa ɯ-ŋgɯ lɤ-tɯ-ɣe \\
\textsc{comit}-hat \textsc{emph} house \textsc{3sg}-inside \textsc{aor}-2-come[II] \\
\glt `You came inside the house wearing your hat.' (You were expected to take it off before coming in, elicited)
\end{exe}

\begin{exe}
\ex \label{ex:kAGWtArtWrte}
\gll laχtɕʰa kɤɣɯ-tɤ-rtɯ\redp{}rte ʑo ta-ndo \\
thing \textsc{comit}-\textsc{indef}.\textsc{poss}-hat \textsc{emph} \textsc{aor}:3\flobv{}-take \\
\glt `He took the hat along with the other objects.' (Not wearing it, elicited)
\end{exe}

The alienabilized comitative adverb \japhug{kɤ́tɤlɯlu}{with milk} (from \japhug{tɤ-lu}{milk}) is used as postnominal modifier in `milk tea' (\ref{ex:tsxha.kAtAlWlu}). The inalienably possessed form \japhug{kɤ́lɯlu}{with its milk} is only compatible with the animal producing the milk (or in the case of a plant producing a milk-like juice).

\begin{exe}
\ex \label{ex:tsxha.kAtAlWlu}
\gll  tʂʰa kɤ́-tɤ-lɯ\redp{}lu \\
 tea \textsc{comit}-\textsc{indef}.\textsc{poss}-milk \\
\glt `Milk tea.' (30-Com) \japhdoi{0003736\#S92}
\end{exe}

Comitative adverbs can be used as sentential adverbs, with scope over the whole sentence (examples \ref{ex:kAGWrtWrte}, \ref{ex:kAGWtArtWrte}). 


They also occur as noun modifiers (as in \ref{ex:tsxha.kAtAlWlu} above), and either follow (\ref{ex:tsxha.kAtAlWlu}, \ref{ex:kAjWjaR}) or precede (\ref{ex:kArnWrna}, \ref{ex:kAthAlwWlwa}) the noun that they modify.


\begin{exe}
\ex \label{ex:kAjWjaR}
\gll tɤ-sno kɤ́-jɯ\redp{}jaʁ nɯ lu-ta-nɯ \\
\textsc{indef}.\textsc{poss}-saddle \textsc{comit}-hand \textsc{dem} \textsc{ipfv}-put-\textsc{pl} \\
\glt `(Then), they put the saddle with its handles.' (30-tAsno) \japhdoi{0003758\#S68}
\end{exe}
 

The noun modified by a comitative adverb can have various syntactic functions in the clauses, including object (\ref{ex:tsxha.kAtAlWlu}, \ref{ex:kAjWjaR}, \ref{ex:kAthAlwWlwa}), intransitive subject (\ref{ex:kAsnWsno}, \ref{ex:kArnWrna}) or even transitive subject (\ref{ex:kArJWrJit.kW}). This last option is not attested in the text corpus, but speakers have no trouble producing sentences of this type.


\begin{exe}
	\ex \label{ex:kAsnWsno}
	\gll kɤ́-snɯ\tld{}sno ʑo kɤ-rŋgɯ \\
	\textsc{comit}-saddle \textsc{emph} \textsc{aor}-lie.down \\
	\glt `[The horse] slept with its saddle.' (elicited)
\end{exe}


\begin{exe}
\ex \label{ex:kAthAlwWlwa}
\gll kɤ́-tʰɤlwɯ\tld{}lwa ɯ-zrɤm ra kɯnɤ cʰɯ́-wɣ-ɣɯt pjɯ́-wɣ-ji ri maka tu-ɬoʁ mɯ́j-cʰa \\ 
\textsc{comit}-earth \textsc{3sg}.\textsc{poss}-root \textsc{pl} also \textsc{ipfv}-\textsc{inv}-bring \textsc{ipfv}-\textsc{inv}-plant but at.all \textsc{ipfv}-come.out \textsc{neg}:\textsc{sens}-can \\
\glt `Even if one takes its root with earth [around it] and plant it, it cannot grow.' (15-babW)
\japhdoi{0003512\#S112}
\end{exe}


\begin{exe}
\ex \label{ex:kArJWrJit.kW}
\gll lɯlu kɤ́-rɟɯ\tld{}rɟit ra kɯ ʑo βʑɯ to-ndza-nɯ. \\
cat \textsc{comit}-offspring \textsc{pl} \textsc{erg} \textsc{emph} mouse \textsc{ifr}-eat-\textsc{pl} \\
\glt `The cat and its young ate the mouse.' (elicited)
\end{exe}

The comitative adverbs have additional meanings in certain contexts. With the verb \japhug{fse}{be like}, comitative adverbs from body parts occurring with names of animals, as in (\ref{ex:kArnWrna}) and (\ref{ex:BZW.kAmtChWmtChi}), mean `to have a body part that looks like that of the other animal'.
\largerpage
\begin{exe}
	\ex \label{ex:kArnWrna}
	\gll pɣɤkʰɯ nɯ ɯ-ku nɯnɯ lɯlu tsa ɲɯ-fse, ɯ-mtsioʁ ɣɤʑu ma kɤ́-rnɯ\redp{}rna lɯlu ɯ-tɯ-fse ɲɯ-sɤre ʑo. \\
	owl \textsc{dem} \textsc{3sg}.\textsc{poss}-head \textsc{dem} cat a.little \textsc{sens}-be.like \textsc{3sg}.\textsc{poss}-beak exist:\textsc{sens} a.part.from \textsc{comit}-ear cat \textsc{3sg}.\textsc{poss}-\textsc{nmlz}:\textsc{deg}-be.like \textsc{sens}-be.extremely/be.funny \textsc{emph} \\
	\glt `The owl's head looks a little like that of a cat, apart from the fact that it has a beak, it looks very much like a cat with its ears.' (22-pGAkhW)
\japhdoi{0003594\#S7}
\end{exe}

\begin{exe}
\ex \label{ex:BZW.kAmtChWmtChi}
\gll li βʑɯ kɤ́-mtɕʰɯ\redp{}mtɕʰi ci nɯ ɲɯ-fse \\
again mouse \textsc{comit}-mouth \textsc{indef} \textsc{dem} \textsc{sens}-be.like \\
\glt `[The bat's] mouth is like that of a mouse.' (literally `It looks like a mouse with its mouth.' 25-qarmWrwa) \japhdoi{0003648\#S11}
\end{exe}

Comitative adverbs connected to a noun can occur before the indefinite article \japhug{ci}{a} as in (\ref{ex:BZW.kAmtChWmtChi}), but this article can also be repeated on both the noun and the adverb, as in (§\ref{sec:indef.article}).

\begin{exe}
\ex \label{ex:BZW.kArWri}
\gll tɕe jlɤkrɯ ci kɤ́-rɯ\redp{}ri ci ɲɤ-rŋo, \\
\textsc{lnk} rake \textsc{indef} \textsc{comit}-thread \textsc{indef} \textsc{ifr}-borrow \\
\glt `He borrowed a rake with a thread.' (140427 qala cho kWrtsAG) \japhdoi{0003852\#S56}
\end{exe}

Nouns incorporated into comitative adverbs lose their nominal status and cannot be determined by relative clauses (including attributive adjectives), numerals or demonstratives. In a sentence such as \ref{ex:kAGWNkhWNkhor} for instance, the attributive participial relative [\forme{kɯ\tld{}kɯ-ŋɤn}] `all the ones who are evil' does not determine \forme{kɤɣɯ-ŋkʰɯ\tld{}ŋkʰor} `with his subjects', a syntactic structure which would correspond to the translation `with all his evil subjects'. Rather, it determines the head noun together with the comitative adverb \forme{rɟɤlpu} \forme{kɤɣɯ-ŋkʰɯ\tld{}ŋkʰor} `the king with his subjects', which implies the translation given below.

\begin{exe}
\ex \label{ex:kAGWNkhWNkhor}
\gll rɟɤlpu kɤɣɯ-ŋkʰɯ\tld{}ŋkʰor kɯ\tld{}kɯ-ŋɤn ʑo to-ndo tɕe, tɕendɤre kɯ-mɤku nɯ sɤtɕʰa kɯ\tld{}kɯ-sɤ-scit ʑo jo-tsɯm ɲɯ-ŋu ri kɯ-maqʰu tɕe, kɯ\tld{}kɯ-sɤɣ-mu ʑo jo-tsɯm tɕe \\
king \textsc{comit}-subjects \textsc{total}\tld{}\textsc{sbj}:\textsc{pcp}-be.bad \textsc{emph} \textsc{ifr}-take \textsc{lnk} \textsc{lnk} \textsc{sbj}:\textsc{pcp}-be.before \textsc{dem} place \textsc{total}\tld{}\textsc{sbj}:\textsc{pcp}-\textsc{prop}-be.happy \textsc{emph} \textsc{ifr}-take.away \textsc{sens}-be \textsc{lnk} \textsc{sbj}:\textsc{pcp}-be.after \textsc{lnk} \textsc{total}\tld{}\textsc{sbj}:\textsc{pcp}-\textsc{prop}-fear \textsc{emph} \textsc{ifr}-take.away \textsc{lnk} \\
\glt `She took the king and his subjects, all the evil ones, in the beginning she took them to nice places, but later she took them to fearful places.' (2012 Norbzang) \japhdoi{0003768\#S341}
\end{exe}

Other denominal adverb formations are also attested in Japhug, but are described in the sections on time nominals (§\ref{sec:time.ordinals}) and locational nouns (§\ref{sec:relator.location}) in other chapters.

 
 \subsection{Reduplicated perlative} \label{sec:perlative}
Partial reduplication of nouns, in addition to the reduplicated collectives (§\ref{sec:redp.coll}), can also derive location adverbs such as \japhug{tʂɯtʂu}{along the road} from \japhug{tʂu}{road} with a perlative meaning, as in (\ref{ex:tsxWtsxu}). A similar use of the reduplication appears in Zbu, but with an additional \forme{kə-} prefix (\citealt[114]{gong18these}).

\begin{exe}
\ex \label{ex:tsxWtsxu}
\gll cʰa ra tʂɯ\redp{}tʂu kú-wɣ-nɯ-tsʰi tɕe \\
alcohol \textsc{pl} path\redp{}\textsc{perlative} \textsc{ipfv}-\textsc{inv}-\textsc{auto}-drink \textsc{lnk} \\
\glt `One drinks alcohol along the way [back home].' (2010-histoire10)
\end{exe}

Other examples of perlative include \forme{ɯ-jrɯ\redp{}jroʁ} `following X's trace' from \japhug{tɤ-jroʁ}{trace} (see example \ref{ex:lunAkhWkhrWta}, §\ref{sec:distributed.action.lexicalized}).

\subsection{\forme{-z} suffix} \label{sec:denominal.postposition.s}
The postposition \japhug{ʁaz}{while ... still}, which is mainly used in a particular type of temporal clause (§\ref{sec:RaznA}), probably originates from the inalienably possessed noun \japhug{tɤ-ʁa}{free time} with the fossil locative suffix \forme{-z} (related to the locative postposition \forme{zɯ}, §\ref{sec:core.locative}) cognate to Situ \forme{-s} (\citealt[330--331]{linxr93jiarong}).

 \subsection{\forme{s-} prefix} \label{sec:denominal.adverb.s.prefix}
 The adverbs \japhug{stʰɯci}{so much} (§\ref{sec:sthWci.equative}) and \japhug{stʰamtɕɤt}{so much} are derived from the indefinite pronoun \japhug{tʰɯci}{something} (§\ref{sec:thWci}) and the universal quantifier \japhug{tʰamtɕɤt}{all} (§\ref{sec:universal.quant}) by what appears to be an \forme{s-} prefix. 
 
 They are attested in derived forms such as \japhug{kɯstʰɯci}{this much} and \japhug{nɯstʰɯci}{that much} (§\ref{sec:nWtshWci}) with the prefixed demonstrative stems \forme{kɯ-} and \forme{nɯ-} (§\ref{sec:demonstrative.pronouns}).
 
The standard marker \japhug{staʁ}{compared with} and its variant \forme{sɯstaʁ} (§\ref{sec:comparative}) are clearly related to the locative relator noun \japhug{ɯ-taʁ}{on, above} (§\ref{sec:WtaR}). The prefix \forme{s-} is doubled as \forme{sɯ-s-} in the form \forme{sɯstaʁ}.

 The ultimate origin of the prefix \forme{s-} in the adverbs and postpositions above is unclear, but it could be the result of the degrammaticalization of the locative \forme{-z} suffix (§\ref{sec:denominal.postposition.s}; this suffix eventually became the locative postposition \forme{zɯ}, §\ref {sec:core.locative}) and subsequent procliticization to the following host. For instance, in the case of \forme{staʁ}, the hypothesized process would be:\footnote{The symbol $X$ represents a noun phrase. The forms are presented in their Japhug orthography for convenience, but at the stage when this reanalysis happened, the actual pronunciation of the forms in question was probably different. }
 
 \begin{enumerate}
\item *$X$\forme{-z (ɯ)-taʁ} (Suffix)
\item *$X$\forme{ zə=taʁ} (Procliticization)
\item $X$ \forme{staʁ} (monosyllabicization and reanalysis as a postposition) 
\end{enumerate}
 
 
