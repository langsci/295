\chapter{Non-finite verbal morphology} \label{chap:non-finite}
The distinction between finite and non-finite verb forms is easy to draw in Japhug: the former have person indexation (see chapter §\ref{chap:indexation}), while the latter do not. The only personal markers found on non-finite verb forms are possessive prefixes, the same set as in underived nouns (§\ref{sec:possessive.prefixes}).

In this chapter, I distinguish between several sub-categories of non-finite verb forms, including participles, infinitives, degree and action nominals as well as several converbs. In addition to describing the morphology of these verb forms, I also present their functions to build various types of subordinate clauses, including relative, complement and purposive clauses.

\section{Participles} \label{sec:participles}
Japhug speakers, like Ancient Greeks, can be aptly described as \grec{φιλομέτοχοι} `participle lovers': Japhug and other Gyalrong languages have a rich system of participles, and these non-finite forms play a central role in the syntax of the language.

Participles are nominalized verb forms that keep some verbal characteristics: they can serve as predicates of subordinate clauses (relative or complement clauses), take TAM, polarity and associated motion marking, and preserve the verb's argument structure.

Participles differ from finite verbs in three ways. First, they cannot serve as the predicate of a main clause. Second, they are not compatible with the personal indexation of the intransitive and transitive conjugations (including direct/inverse marking, §\ref{sec:allomorphy.inv}), and with all inflectional suffixes without exception (§\ref{sec:suffixes}).\footnote{Japhug is identical in this regard to Tshobdun and Zbu, but crucially differs from Situ, where nominalized forms in \forme{kə-} can bear indexation suffixes \citep{jackson06guanxiju,jacksonlin07}. } Rather, like nouns, they can take a possessive prefix which can be coreferent with one of the arguments. Due to the general impossibility of stacking possessive prefixes (§\ref{sec:possessive.paradigm}), at most only one argument can be indexed this way. Third, there are restrictions on TAM marking on participles: they have at most three forms (neutral, perfective and imperfective), and completely lack Inferential (§\ref{sec:ifr}), Egophoric Present (§\ref{sec:egophoric}) or Sensory (§\ref{sec:sensory}) forms .

There are three participles in Japhug; the subject S/A participle in \forme{kɯ\trt}, the object participle in \forme{kɤ-} and the oblique participle in \forme{sɤ-}. 

Complex participial forms, including negative, associated motion or TAM prefixes are possible, as shown by example (\ref{ex:WGWjAkWqru}). However, never more than four inflectional prefixes are found; forms with all five prefixal slots filled (such as $\dagger$\forme{ɯ-ɣɯ-jɤ-kɯ-qru}) are not accepted by Tshendzin.

 \begin{exe}
\ex \label{ex:WGWjAkWqru}
\gll ɯ-ɣɯ-jɤ-kɯ-qru tɤ-tɕɯ  \\
  \textsc{3sg}-\textsc{cisl}-\textsc{aor}-\textsc{sbj}:\textsc{pcp}-meet \textsc{indef}.\textsc{poss}-boy   \\
\glt `The boy who had come to look for her' (The three sisters)
 \end{exe}

\tabref{tab:template.nmlz} summarizes the template of participial verb forms; more details are provided on possible and attested forms for each participle type in the following sections.

\begin{table}[h]
\caption{The template of participial verb forms in Japhug} \centering \label{tab:template.nmlz}
\resizebox{\columnwidth}{!}{
\begin{tabular}{lllllll}
\toprule
-6 &-5& -4&-3 &-2&-1& \ro{} \\
possessive &proximative & negative&associated   & TAM & participle prefix &enlarged  \\
prefix & &prefix &motion prefix  &orientation&&stem\\
\bottomrule
\end{tabular}}
\end{table}

Stem alternation is reduced in participle forms: stem III (§\ref{sec:stem3.distribution}) never occurs. The few verbs that have an alternation between stem I and stem II (\japhug{ɕe}{go}, \japhug{ɣi}{come}, \japhug{ti}{say} and derived forms, §\ref{sec:stem2}), however, use stem II in subject and object participles with perfective orientational prefixes (§\ref{sec:kamnyu.preverbs}), in forms like \forme{jɤ-kɯ-ɣe} \textsc{aor}-\textsc{sbj}:\textsc{pcp}-come[II] `the one who came'
or \forme{tɤ-kɤ-tɯt} \textsc{aor}-\textsc{obj}:\textsc{pcp}-say[II] `what was said'.
 
\subsection{Subject participles} \label{sec:subject.participles}
The subject participle, built by adding the prefix \forme{kɯ-} to the verb stem, designates an entity corresponding to the intransitive subject (\ref{ex:kWsi}, §\ref{sec:absolutive.S} and §\ref{sec:intr.subject.relativization} ), a possessor of the subject (§\ref{sec:S.possessor.relativization}), or the transitive subject (\ref{ex:WkWndza}, §\ref{sec:A.kW}, §\ref{sec:tr.subject.relativization}) of the base verb. 

 \begin{exe}
\ex \label{ex:kWsi}
\gll kɯ-si  \\
  \textsc{sbj}:\textsc{pcp}-die \\
 \glt  `The dead one' (many attestations)
\end{exe}

 \begin{exe} 
\ex \label{ex:WkWndza}
\gll ɯ-kɯ-ndza \\
  \textsc{3sg}-\textsc{sbj}:\textsc{pcp}-eat \\
 \glt  `The one who eats it' (many attestations)
\end{exe}

With \forme{a-} initial verbs the \forme{kɯ-} prefix regularly merge with \forme{a-} as \forme{kɤ\trt}, a form which resembles an object participle. There is almost no ambiguity since all \forme{a-} initial verbs are intransitive, §\ref{sec:contraction}). The only exception are the semi-transitive verbs in \forme{a\trt}, such as \japhug{aro}{have}, whose subject participle \forme{kɯ-ɤro} `having, the one who has' and object participle \forme{kɤ-ɤro} both surface as \ipa{kɤro}.

The subject participle \forme{kɯ-} prefix is historically related to that of object participles (§\ref{sec:object.participle}), velar infinitives (§\ref{sec:velar.inf}) and deverbal nouns in \forme{x-/ɣ-} (§\ref{sec:G.nmlz}), and has cognates elsewhere in the family (§\ref{sec:velar.nmlz.history}).

In this section, I discuss first morphological issues (possessive prefixes §\ref{sec:subject.participle.ambiguities}, other prefixes §\ref{sec:subject.participle.other.prefixes} and ambiguous forms §\ref{sec:subject.participle.ambiguities}), and then present the various functions of subject participles, including participial relatives (§\ref{sec:subject.participle.subject.relative} and §\ref{sec:subject.participle.other.relative}), complementation strategies (§\ref{sec:subject.participle.complementation}), as well as the case of lexicalized participles (§\ref{sec:lexicalized.subject.participle}).
 
Examples which could potentially be viewed as subject participles in converbial use are analyzed as \forme{kɯ-} infinitives (§\ref{sec:inf.converb}).

\subsubsection{Possessive prefixes on subject participles}  \label{sec:subject.participle.possessive}

In the case of transitive verbs, a possessive prefix coreferent with the object is obligatory when no overt object is present (\textsc{3sg} \forme{ɯ-} in \ref{ex:WkWndza}), and when no other prefix is added to the participle.

When another prefix (polarity, associated motion or orientation preverb) is present, the possessive prefix is optional, as shown by forms like \forme{mɤ-kɯ-ndza} `the one which does not eat (it)' in (\ref{ex:mAkWndza}), as opposed to \forme{ɯ-mɤ-kɯ-mto} `the one who does not see it' in (\ref{ex:WmAkWmto}) with both possessive \forme{ɯ-} and the negative prefix \forme{mɤ-}.

 \begin{exe} 
\ex \label{ex:mAkWndza}
\gll  tɤ-mtʰɯm ʁɟa ʑo ma nɯ ma, nɤki, tɯjpu mɤ-kɯ-ndza ci tu tɕe, \\
\textsc{indef}.\textsc{poss}-meat completely \textsc{emph} \textsc{lnk} \textsc{dem} apart.from \textsc{filler} flour.based.food \textsc{neg}-\textsc{sbj}:\textsc{pcp}-eat \textsc{indef} exist:\textsc{fact} \textsc{lnk} \\
\glt  `There is  [an animal like the mouse] which only eats meat, not food made from flour.' (27-spjaNkW)
(\japhdoi{0003704\#S190})
\end{exe}

 \begin{exe} 
\ex \label{ex:WmAkWmto} 
\gll  li nɯnɯ kɯnɤ ɯ-kɯ-mto ɣɤʑu, ɯ-mɤ-kɯ-mto ɣɤʑu. \\
again \textsc{dem} also \textsc{3sg}.\textsc{poss}-\textsc{sbj}:\textsc{pcp}-see exist:\textsc{sens} \textsc{3sg}.\textsc{poss}-\textsc{neg}-\textsc{sbj}:\textsc{pcp}-see exist:\textsc{sens} \\
\glt `There are [people] who see (find) it, and people who don't.' (20-sWrna)
(\japhdoi{0003564\#S21})
\end{exe}

In the case of ditransitive verbs, the possessive prefix strictly refers to the object. With indirective verbs like \japhug{tʰu}{ask}, the possessive prefix is necessarily the theme, never the recipient. The form in (\ref{ex:AkWthu}) thus cannot be interpreted as meaning `the one who asks me (about it)'; the correct construction would be (\ref{ex:ACki.kWthu}), with the recipient in the dative case.

\begin{exe}
\ex \label{ex:AkWthu}
\gll a-kɯ-tʰu  \\
\textsc{1sg}.\textsc{poss}-\textsc{sbj}:\textsc{pcp}-ask \\
\glt `The one asking for my [hand] (in marriage)' (elicited)
\ex \label{ex:ACki.kWthu}
\gll a-ɕki ɯ-kɯ-tʰu  \\
\textsc{1sg}.\textsc{poss}-\textsc{dat} \textsc{3sg}.\textsc{poss}-\textsc{sbj}:\textsc{pcp}-ask \\ 
\glt `The one who asks me about it.' 
\end{exe}

With secundative verbs (§\ref{sec:ditransitive.secundative}), the possessive prefix of the subject participle is obligatorily coreferent with the recipient, not the theme, as in (\ref{ex:nAkWmbi}).

\begin{exe}
\ex \label{ex:nAkWmbi}
\gll nɯ ma nɤ-kɯ-mbi me \\
\textsc{dem} apart.from \textsc{2sg}.\textsc{poss}-\textsc{sbj}:\textsc{pcp}-give not.exist:\textsc{fact} \\
\glt `Nobody will give you another [daughter in marriage].' (2002 qaCpa)
\end{exe}
With intransitive verbs, including adjectival stative verbs, a possessive prefix can also be added. In the case of semi-transitive verbs (§\ref{sec:semi.transitive}), the possessive can refer to the semi-object (§\ref{sec:semi.object}), as in example (\ref{ex:WkWrga.pWdAn}).

 \begin{exe} 
\ex \label{ex:WkWrga.pWdAn}
\gll  nɯ ɕɯŋgɯ tɕe, ɯ-kɯ-rga pɯ-dɤn. \\
\textsc{dem} before \textsc{lnk} \textsc{3sg}.\textsc{poss}-\textsc{sbj}:\textsc{pcp}-like \textsc{pst}.\textsc{ipfv}-be.many \\
\glt  `Before, there used to be many people who liked it.' (12-Zmbroko)
(\japhdoi{0003490\#S109})
\end{exe}

It can also refer to the beneficiary (which is normally marked with genitive or possessive prefixes, see §\ref{sec:other.uses.poss.prefixes} and §\ref{sec:gen.beneficiary}), as in (\ref{ex:tWZo.tWkWpe}) and (\ref{ex:aZo.akWra}).

 \begin{exe} 
\ex \label{ex:tWZo.tWkWpe}
\gll  kɯ-pe tú-wɣ-nɤma tɕe li tɯʑo tɯ-kɯ-pe tu \\
\textsc{sbj}:\textsc{pcp}-be.good \textsc{ipfv}-\textsc{inv}-make \textsc{lnk} again \textsc{genr} \textsc{genr}.\textsc{poss}-\textsc{sbj}:\textsc{pcp}-be.good exist:\textsc{fact} \\
\glt  `If one does good things, one will also have good things.' (140518 mao he laoshu-zh)
(\japhdoi{0004030\#S111})
\end{exe}

 \begin{exe} 
\ex \label{ex:aZo.akWra}
\gll  aʑo a-kɯ-ra nɯra a-tɤ-tɯ-ste qʰendɤre aʑo nɯnɯ, nɤki, ku-nɤtsi-a jɤɣ \\
\textsc{1sg} \textsc{1sg}.\textsc{poss}-\textsc{sbj}:\textsc{pcp}-be.needed \textsc{dem}:\textsc{pl} \textsc{irr}-\textsc{pfv}-2-do.like[III] \textsc{lnk} \textsc{1sg} \textsc{dem} \textsc{filler} \textsc{ipfv}-hide[III]-\textsc{1sg} be.possible:\textsc{fact}  \\
\glt  `If you do the things I need, I will keep it secret.'  (2014-kWlAG)
\end{exe}

Since participles are also noun-like, the possessive prefixes can be real possessive, and be preceded with a genitive phrase as in (\ref{ex:tCaXpa.ra.GW.nWkWmna}) with \forme{nɯ-kɯ-mna} `the best among them' = `their chief' (on the verb \japhug{mna}{be better}, see §\ref{sec:mna.sna}).

 \begin{exe} 
\ex \label{ex:tCaXpa.ra.GW.nWkWmna}
\gll tɕaχpa ra ɣɯ nɯ-kɯ-mna nɯ wuma ʑo pjɤ-nɯrɤŋom.\\
bandit \textsc{pl} \textsc{gen} \textsc{3pl}.\textsc{poss}-\textsc{sbj}:\textsc{pcp}-be.better \textsc{dem} really \textsc{emph} \textsc{ifr}-be.upset\\
\glt `The chief of the bandits was very upset.' (140512 alibaba-zh)
(\japhdoi{0003965\#S186})
\end{exe}

This construction is used as a type of superlative (§\ref{sec:possessed.superlative}).

\subsubsection{Associated motion, polarity and orientation preverbs on subject participles}  \label{sec:subject.participle.other.prefixes}
Of all non-finite verb forms, subject participles allow the richest possible combinations of inflectional prefixes: associated motion (§\ref{sec:associated.motion}, example \ref{ex:WCWkWphWt}) below with the translocative \forme{ɕɯ-}), polarity (§\ref{sec:negation}, see \ref{ex:WmAkWmto} above) and orientation preverbs marking TAME (§\ref{sec:kamnyu.preverbs}) all can be prefixed. 
 
\begin{exe}
\ex \label{ex:WCWkWphWt}
 \gll tɕeri nɯra ɯ-ɕɯ-kɯ-pʰɯt ra kɯ-tu me ma,   \\
 \textsc{lnk} \textsc{dem}:\textsc{pl} \textsc{3sg}.\textsc{poss}-\textsc{tral}-\textsc{sbj}:\textsc{pcp}-cut \textsc{pl} \textsc{sbj}:\textsc{pcp}-exist not.exist:\textsc{fact} \textsc{lnk} \\
 \glt `But nobody goes and picks it.'(11-paRzwamWntoR)
(\japhdoi{0003476\#S81})
\end{exe}

Two of the four series of orientation preverbs (§\ref{sec:kamnyu.preverbs}) are possible with subject participles. With A-type prefixes (\forme{tɤ-} \textsc{upwards}, \forme{pɯ-} \textsc{downwards} etc), the participle of dynamic verbs is perfective as \forme{tʰɯ-kɯ-ɣe} `the one who came' in (\ref{ex:WkWntsGe.thWkWGe}), and takes stem II (§\ref{sec:stem2}). With B-type prefixes (\forme{tu-} \textsc{upwards}, \forme{pjɯ-} \textsc{downwards}), it has a habitual imperfective meaning with dynamic verbs as \forme{ju-kɯ-ɣi} `the one who (usually) comes' in (\ref{ex:WkWndza.jukWGi}).\footnote{These two examples also illustrate the use of subject participles as purposive complements with the forms \forme{ɯ-kɯ-ntsɣe} and \forme{ɯ-kɯ-ndza} (see §\ref{sec:subject.participle.complementation}, §\ref{sec:purposive.clause.motion.verbs}).} The prefixes \forme{ɲɯ-} and \forme{ku-} do appear on subject participles, but only to express imperfective: there are no Egophoric (§\ref{sec:egophoric}) or Sensory (§\ref{sec:sensory}) subject partiples.

\begin{exe}
\ex \label{ex:WkWntsGe.thWkWGe}
\gll iɕqʰa qaʑo ɯ-kɯ-ntsɣe tʰɯ-kɯ-ɣe nɯ ɯ-pʰe \\
the.aforementioned sheep \textsc{3sg}.\textsc{poss}-\textsc{sbj}:\textsc{pcp}-sell \textsc{aor}:\textsc{downstream}-\textsc{sbj}:\textsc{pcp}-come[II] \textsc{dem} \textsc{3sg}.\textsc{poss}-\textsc{dat} \\
\glt  `[He told] the person who had come to sell the sheep.' (2003kandZislama)
\end{exe}

\begin{exe}
\ex \label{ex:WkWndza.jukWGi}
\gll ɯ-kɯ-ndza ju-kɯ-ɣi nɯ pɣa ci ɲɯ-ŋu \\
\textsc{3sg}.\textsc{poss}-\textsc{sbj}:\textsc{pcp}-eat \textsc{ipfv}-\textsc{sbj}:\textsc{pcp}-come \textsc{dem} bird \textsc{indef} \textsc{sens}-be \\
\glt   `The one who comes to eat [the fruits] is a bird.' (2012 qachGa)
(\japhdoi{0004087\#S22})
\end{exe}

The participles of stative verbs with series A and B orientation preverbs have an inchoative meaning, exactly like their finite counterpart (§\ref{sec:aor.inchoative} and §\ref{sec:ipfv.inchoative}).  In (\ref{ex:YWkWjpum}) for instance, the imperfective participle \forme{ɲɯ-kɯ-jpum} from \japhug{jpum}{be thick} means `the one which becomes thicker', as opposed to the basic participle \forme{kɯ-jpum} `the thick one'.

\begin{exe}
\ex \label{ex:YWkWjpum}
 \gll ndʑu ɯ-ku jamar ɲɯ-kɯ-jpum ɣɤʑu nɤ, kɯ-wxti.  \\
 chopsticks \textsc{3sg}.\textsc{poss}-head about \textsc{ipfv}-\textsc{sbj}:\textsc{pcp}-be.thick exist:\textsc{sens} \textsc{sfp} \textsc{sbj}:\textsc{pcp}-be.big \\
 \glt  `There are [maggots] that grow as thick as the tip of a chopstick, the big ones.' (25-akWzgumba)
(\japhdoi{0003632\#S73})
\end{exe}

Imperfective participles of stative adjectival verbs are also can also describe the gradient variation of a property across space rather than time. For instance, in (\ref{ex:YWkWjpum2}), the imperfective subject participles \forme{ku-kɯ-xtsʰɯm} and \forme{ɲɯ-kɯ-jpum} are used not to indicate a change across time, but to describe the shape of the gourd, which is progressively thinner towards the top and thicker towards the bottom (on the contrast between the \textsc{eastwards} \forme{ku-} vs. \textsc{westwards} \forme{ɲɯ-} preverbs in this context, see §\ref{sec:centripetal.centrifugal}).

\begin{exe}
\ex \label{ex:YWkWjpum2}
 \gll  tɕe ɯ-mat nɯnɯ, ɯ-taʁ ku-kɯ-xtsʰɯm, ɯ-pa ɲɯ-kɯ-jpum ci cʰɯ-βze ɲɯ-ŋu tɕe, nɯ <hulu> tu-sɤrmi-nɯ. \\
 \textsc{lnk} \textsc{3sg}.\textsc{poss}-fruit \textsc{dem} \textsc{3sg}.\textsc{poss}-up \textsc{ipfv}-\textsc{sbj}:\textsc{pcp}-be.thin \textsc{3sg}.\textsc{poss}-down \textsc{ipfv}-\textsc{sbj}:\textsc{pcp}-be.thick \textsc{indef} \textsc{ipfv}-make[III]  \textsc{sens}-be \textsc{lnk} \textsc{dem} gourd \textsc{ipfv}-call-\textsc{pl} \\
 \glt `It has a fruit that is thinner [in diameter] on the upper part, and thicker on the lower part, people call it `gourd'.' (150825 huluwa-zh)
(\japhdoi{0006346\#S3})
\end{exe}

The past imperfective of stative verbs is built using the series A prefix \forme{pɯ-} as in the corresponding finite forms (§\ref{sec:pst.ifr.ipfv.morphology}). For instance, the past imperfective participle of \japhug{ŋu}{be} is \forme{pɯ-kɯ-ŋu} `the one who used to be ....' (§\ref{sec:non.existing.derivation}), as in (\ref{ex:pWkWNu}).

\begin{exe}
\ex \label{ex:pWkWNu}
 \gll  ɯʑɤɣ nɯ ɕɯŋgɯ ɯ-nmaʁ pɯ-kɯ-ŋu tsʰɯraŋ nɯ pjɤ-mto \\
 \textsc{3sg}:\textsc{gen} \textsc{dem} before \textsc{3sg}.\textsc{poss}-husband \textsc{pst}.\textsc{ipfv}-\textsc{sbj}:\textsc{pcp}-be  \textsc{anthr} \textsc{dem} \textsc{ifr}-see \\
\glt `She saw Tshering, who used to be her husband.' (qajdoskAt2002)
\end{exe}

Most examples in the corpus have one or two prefixes, either combining a possessive prefix with another prefix (as in \ref{ex:WmAkWmto} and \ref{ex:WCWkWphWt}), or combining a negative prefix with an orientation preverb, as in (\ref{ex:mWnWkWsna}).

 \begin{exe}
\ex \label{ex:mWnWkWsna}
 \gll tɕe kʰa ɣɯ ɯ-ndzɤtsʰi ɯ-ro nɯ-kɯ-ri nɯra, mɯ-nɯ-kɯ-sna nɯra, nɯra paʁ kɯ ʁɟa tu-ndze ɲɯ-ŋu \\
 \textsc{lnk} house \textsc{gen} \textsc{3sg}.\textsc{poss}-food \textsc{3sg}.\textsc{poss}-excess \textsc{aor}-\textsc{sbj}:\textsc{pcp}-left \textsc{dem}:\textsc{pl}  \textsc{neg}-\textsc{aor}-\textsc{sbj}:\textsc{pcp}-be.good \textsc{dem}:\textsc{pl} \textsc{dem}:\textsc{pl} pig \textsc{erg} completely  \textsc{ipfv}-eat[III] \textsc{sens}-be \\
 \glt  `Leftover food from the house, or food which is not good any more, pigs eat all of it.' (05-paR)
(\japhdoi{0003400\#S32})
\end{exe}

Subject participles with three prefixes before the participle prefix \forme{kɯ-} are possible, but attestations are rare. Example (\ref{ex:WGWjAkWqru}) above shows the combination of a possessive, an associated motion and an orientation preverbs (\forme{ɯ-ɣɯ-jɤ-kɯ-qru} `the one who had come to meet/look for her'), and (\ref{ex:WmApjWkWnWfkAB}) below that of a possessive, a polarity and an orientation preverbs.

\begin{exe}
\ex \label{ex:WmApjWkWnWfkAB}
 \gll ɯ-pjɯ-kɯ-nɯ-fkaβ tu, ɯ-mɤ-pjɯ-kɯ-nɯ-fkaβ tu ri nɯ kɯ-fse tu-nɯ-ndza-nɯ ɕti. \\
 \textsc{3sg}.\textsc{poss}-\textsc{ipfv}-\textsc{sbj}:\textsc{pcp}-\textsc{auto}-cover exist:\textsc{fact}  \textsc{3sg}.\textsc{poss}-\textsc{neg}-\textsc{ipfv}-\textsc{sbj}:\textsc{pcp}-\textsc{auto}-cover exist:\textsc{fact} \textsc{lnk} \textsc{dem} \textsc{sbj}:\textsc{pcp}-be.like \textsc{ipfv}-\textsc{auto}-eat-\textsc{pl} be.\textsc{aff}:\textsc{fact} \\
 \glt `There are people who cover it (with a lid while cooking), and people who don't, they eat it as is.' (23-mbrAZim)
(\japhdoi{0003604\#S20})
\end{exe}

In addition to imperfective orientation preverbs as in (\ref{ex:WmApjWkWnWfkAB}), it is possible for subject participles to combine possessive prefixes with \textit{perfective} orientation preverbs, as in (\ref{ex:WnWkWCar}). Subject participles are the only non-finite forms attested with such a combination: the object participles do not allow combination of possessive and orientation preverbs (§\ref{sec:object.participle.possessive}) and the oblique participles cannot take perfective orientation preverbs (§\ref{sec:oblique.participle.orientation}).

 \begin{exe} 
\ex \label{ex:WnWkWCar}
\gll nɯnɯ ɯ-nɯ-kɯ-ɕar ɯ-pɯ-kɯ-mto nɯ ɣɯ ɲɯ-tʂaŋ ma nɯnɯ, nɤkinɯ, ɯ-ɕɯ-kɯ-βɟi nɯ ɣɯ mɯ́j-tʂaŋ \\
\textsc{dem} \textsc{3sg}.\textsc{poss}-\textsc{aor}-\textsc{sbj}:\textsc{pcp}-search \textsc{3sg}.\textsc{poss}-\textsc{aor}-\textsc{sbj}:\textsc{pcp}-see \textsc{dem} \textsc{gen} \textsc{sens}-be.fair \textsc{lnk} \textsc{dem} \textsc{filler} \textsc{3sg}.\textsc{poss}-\textsc{tral}-\textsc{sbj}:\textsc{pcp}-chase \textsc{dem} \textsc{gen} \textsc{neg}:\textsc{sens}-be.fair \\
\glt `It is fair that she would [be given] to the one who looked for her and found her, not to the ones chasing her.' (140517 buaishuohua-zh)
(\japhdoi{0004018\#S117})
\end{exe}

The negative prefix has the form \forme{mɯ-} when occurring with a perfective orientation preverb as \forme{mɯ-nɯ-kɯ-sna} `the one that is not good anymore' in (\ref{ex:mWnWkWsna}) and \forme{mɤ-} when no orientation preverb is present (examples \ref{ex:WmAkWmto} and \ref{ex:mAkWndza} above). With the imperfective orientation preverbs, the allomorph \forme{mɤ-} occurs when preceded by a possessive prefix (\ref{ex:WmApjWkWnWfkAB}) and \forme{mɯ-} is found when no possessive prefix is present: compare the elicited forms (\ref{ex:WmAkukWtshi}) and (\ref{ex:mWkukWtshi}). The allomorphs of the negative prefix are not in free variation: forms such as $\dagger$\forme{ɯ-mɯ-ku-kɯ-tsʰi} or $\dagger$\forme{mɤ-ku-kɯ-tsʰi} would be incorrect in Kamnyu Japhug.

 \begin{exe} 
\ex \label{ex:WmAkukWtshi}
\gll ɯ-mɤ-ku-kɯ-tsʰi \\
\textsc{3sg}.\textsc{poss}-\textsc{neg}-\textsc{ipfv}-\textsc{sbj}:\textsc{pcp}-drink \\
\ex \label{ex:mWkukWtshi}
\gll mɯ-ku-kɯ-tsʰi \\
\textsc{neg}-\textsc{ipfv}-\textsc{sbj}:\textsc{pcp}-drink \\
\glt  `The one who drinks it.' (elicited)
\end{exe}

There are no constraints on the number of derivational prefixes in participial forms. The derivational prefixes are all closer to the verb root than the participle prefix \forme{kɯ\trt}, and thus follow it as shown by (\ref{ex:WmApjWkWnWfkAB}), where the autive \forme{-nɯ\trt}, the leftmost of all derivational prefixes (§\ref{sec:inner.prefixal.chain}), is placed after \forme{kɯ-}. 

Aside from possessive, orientation, associated motion and polarity prefixes, subject participles can also receive the Proximative aspect prefix \forme{jɯ-}  (see \ref{ex:jWtukWwGrum}, §\ref{sec:proximative}).
 
Subject participles can undergo totalitative reduplication (§\ref{sec:totalitative.redp}, §\ref{sec:totalitative.relatives}), which applies to the first syllable of the word, whether it is the participle \forme{kɯ-} or an orientation preverb as in (\ref{ex:jWjAkWGe}), meaning `all of those who/that X'.

\begin{exe}
\ex \label{ex:jWjAkWGe}
\gll tɕe nɯnɯ ɯ-taʁ jɯ\redp{}jɤ-kɯ-ɣe nɯ ku-ndɤm ɲɯ-ŋu. \\
\textsc{lnk} \textsc{dem} \textsc{3sg}.\textsc{poss}-on \textsc{total}\redp{}\textsc{aor}-\textsc{sbj}:\textsc{pcp}-come[II] \textsc{dem} \textsc{ipfv}-take[III] \textsc{sens}-be \\
\glt `[The spider] catches all of the [insects] that have come onto [its web].' (26-mYaRmtsaR) (\japhdoi{0003674\#S99})
\end{exe}

\subsubsection{Ambiguities}  \label{sec:subject.participle.ambiguities}
The subject participle \forme{kɯ-} prefix is homophonous with the generic person marker for intransitive subject and object (§\ref{sec:indexation.generic.tr}; note that these two prefixes are probably historically related, §\ref{sec:portmanteau.prefixes.history}). In the case of intransitive verbs, some subject participles are therefore homophonous with generic person forms. 

For instance, the past imperfective generic \forme{pɯ-kɯ-ŋu} `one used to be' in (\ref{ex:pWkWNu.genr}) is identical to the past imperfective participle \forme{pɯ-kɯ-ŋu} `the one who used to be ....', discussed above (example \ref{ex:pWkWNu} in §\ref{sec:subject.participle.other.prefixes}). In this example, it is obvious that \forme{kɯ-} is the generic person marker because the verb \forme{pɯ-kɯ-rga} `one used to be' occurs as the main verb; outside of any context,  \forme{tɤ-pɤtso pɯ-kɯ-ŋu} could be understood as a relative clause `the one who used to be a child', but this is not the meaning of this sentence. 

\begin{exe}
\ex \label{ex:pWkWNu.genr}
 \gll tɕeri tɤ-pɤtso pɯ-kɯ-ŋu tɕe, nɯ kɤ-ndza wuma ʑo pɯ-kɯ-rga. \\
 \textsc{lnk} \textsc{indef}.\textsc{poss}-child \textsc{pst}.\textsc{ipfv}-\textsc{genr}:S/O-be \textsc{lnk} \textsc{dem} \textsc{inf}-eat really \textsc{emph} \textsc{pst}.\textsc{ipfv}-\textsc{genr}:S/O-like \\
 \glt `When [we] were children, [we] used to like eating it.' (12-ndZiNgri) (\japhdoi{0003488\#S126})
\end{exe}

More generally, the Factual, Imperfective, Past Imperfective and Aorist forms of intransitive verbs in generic person forms are homophonous with unmarked, Imperfective, Past Imperfective and Aorist participles, respectively. In the case of transitive verbs, the subject participle can be identical to the object generic form. For instance, the participle \forme{nɯ-tu-kɯ-ndza} `the one who eats them' in (\ref{ex:tukWndza.nmlz}) only differs from the generic \forme{tu-kɯ-ndza} `it eats us/people' in  (\ref{ex:tukWndza.genr}) by the possessive prefix \forme{nɯ\trt}, and that prefix being optional, there are forms that are really ambiguous between participle and generic. 

\begin{exe}
\ex \label{ex:tukWndza.nmlz}
 \gll nɯ ɯ-rkɯ jɤ-azɣɯt-nɯ tɕe, ʑara nɯ-tu-kɯ-ndza srɯnmɯ ci pjɤ-tu, \\
 \textsc{dem} \textsc{3sg}.\textsc{poss}-side \textsc{aor}-reach-\textsc{pl} \textsc{lnk} \textsc{3pl} \textsc{3pl}.\textsc{poss}-\textsc{ipfv}-\textsc{sbj}:\textsc{pcp}-eat râkshasî \textsc{indef} \textsc{ifr}.\textsc{ipfv}-be \\
\glt `There was a râkshasî who ate those who came near her.' (2012 Kunbzang) (\japhdoi{0003768\#S219})
\end{exe}

\begin{exe}
\ex \label{ex:tukWndza.genr}
 \gll tɕe ndzɤpri kɤ-ti nɯ tɕe tɯrme tu-kɯ-ndza ɲɯ-ŋgrɤl  \\
 \textsc{lnk} brown.bear \textsc{obj}:\textsc{pcp}-say \textsc{dem} \textsc{lnk} people \textsc{ipfv}-\textsc{genr}:S/O \textsc{sens}-be.usually.the.case \\
\glt `The one called `brown bear', it eats people.' (21-pri)
(\japhdoi{0003580\#S91})
\end{exe}
 
The irregular generic \forme{tu-kɯ-ti} `one says' of the verb \japhug{ti}{say} is also identical with the participle `the one who says'.

The \textsc{2sg}\fl{}\textsc{1sg} form of transitive verbs in \forme{-a}, due to the vowel fusion rule \ipa{-a-a} \fl{} \forme{-a}, are also superficially identical to subject participles. For instance \forme{tu-kɯ-ndza-a} `you eat me' is pronounced \phonet{tukɯndza} exactly like the generic and the participle \forme{tu-kɯ-ndza} in the Kamnyu dialect (in the dialects of Japhug where this vowel fusion does not occur, the forms remain distinct).

\begin{exe}
\ex \label{ex:tukWndzaa}
 \gll nɯ kóʁmɯz nɤ tu-kɯ-ndza-a \\
 \textsc{dem} only.after \textsc{lnk} \textsc{ipfv}-2\fl{}1-eat-\textsc{1sg} \\
 \glt `Eat me only after [having taken the thorn from my foot].' (140426 lang yisheng-zh)
\end{exe} 

In the case of stative verbs and some auxiliary verbs, the infinitive has in some cases the form \forme{kɯ\trt}, and there is thus ambiguity between infinitive and subject participial forms for these verbs (§\ref{sec:velar.inf}).

\subsubsection{Subject relative clauses}  \label{sec:subject.participle.subject.relative}
The most common use of subject participles is to build participial relative clauses whose head noun is the subject; it is the only way to relativize the subject in Japhug (§\ref{sec:intr.subject.relativization}). Headless relatives are most common (§\ref{sec:headless.relative}), but when the head noun is overt, the relative can be either prenominal, postnominal or head-internal. With intransitive verbs the difference between postnominal or head-internal relatives is often difficult to ascertain, and many examples are ambiguous; for instance in (\ref{ex:tCheme.RnWz.kWrWCmi}), the relative clause could be argued to be postnominal (limited to the participle \forme{kɯ-rɯɕmi} `speaking') or head-internal (including \forme{tɕʰeme ʁnɯz} `two girls', and possibly even the previous adjunct).

\begin{exe}
\ex \label{ex:tCheme.RnWz.kWrWCmi}
 \gll  kʰa ɯ-ŋgɯ nɯtɕu tɕʰeme ʁnɯz kɯ-rɯɕmi pjɤ-tu. \\
 house \textsc{3sg}.\textsc{poss}-inside \textsc{dem}:\textsc{loc} girl two \textsc{sbj}:\textsc{pcp}-speak \textsc{ifr}.\textsc{ipfv}-exist \\
\glt  `There were two girls speaking in the house.' (150909 xiaocui-zh)
(\japhdoi{0006386\#S144})
\end{exe}

Other examples such as (\ref{ex:kWm.WrkW.zW.pGa}) are unambiguously head-internal, since the locative adjunct \forme{kɯm ɯ-rkɯ zɯ} cannot belong to the matrix clause. This example additionally illustrates the necessity of using a subject relative clause with an existential verb to connect a noun with an postpositional phrase ($\dagger$\forme{kɯm ɯ-rkɯ zɯ pɣa} would not be a complete sentence).

\begin{exe}
\ex \label{ex:kWm.WrkW.zW.pGa}
 \gll  kumpɣa nɯnɯ tɕe [kɯm ɯ-rkɯ zɯ pɣa kɯ-tu] kɤ-ti ɲɯ-ŋu  \\
 hen \textsc{dem} \textsc{lnk} door \textsc{3sg}.\textsc{poss}-side \textsc{loc} bird \textsc{sbj}:\textsc{pcp}-exist \textsc{inf}-say \textsc{sens}-be \\
 \glt `The word \japhug{kumpɣa}{hen} means `the bird that is next to the door'.' (22-kumpGa). 
(\japhdoi{0003588\#S3})
\end{exe}

With transitive verbs, subject head-internal relatives can be distinguished from postnominal ones by the presence of the ergative \forme{kɯ} on the head noun (§\ref{sec:head-internal.relative}), as in (\ref{ex:WrdWrdoR.kW.thotsi.WkWta}).

\begin{exe}
\ex \label{ex:WrdWrdoR.kW.thotsi.WkWta}
 \gll [tsuku ɯ-rdɯ\redp{}rdoʁ kɯ ʑo tʰotsi ɯ-kɯ-ta] ɣɤʑu. \\
 some \textsc{3sg}.\textsc{poss}-piece \textsc{erg} \textsc{emph} decorative.stamp \textsc{3sg}.\textsc{poss}-\textsc{sbj}:\textsc{pcp}-put exist:\textsc{sens} \\
 \glt `There are people who put a decorative mark [on their bread].' (160706 thotsi) (\japhdoi{0006133\#S20})
\end{exe}

Prenominal relatives are relatively rare with intransitive verbs, but commonly occur with transitive verbs, as in (\ref{ex:tWnW.WkWtshi}). Note the presence of indefinite person possessive marking on the head noun \japhug{tɤ-pɤtso}{child} in this example; unlike in Situ \citep{jacksonlin07}, the head noun of prenominal relatives in Japhug does not take a third person singular prefix as in a possessive construction (in which case the form $\dagger$\forme{ɯ-pɤtso} would have been found).

\begin{exe}
\ex \label{ex:tWnW.WkWtshi}
 \gll  [tɯ-nɯ ɯ-kɯ-tsʰi] tɤ-pɤtso ɣɯ ɯ-kɯ-mŋɤm ɲɯ-ŋu tɕe, \\
 \textsc{indef}.\textsc{poss}-breast \textsc{3sg}.\textsc{poss}-\textsc{sbj}:\textsc{pcp}-drink \textsc{indef}.\textsc{poss}-child \textsc{gen} \textsc{3sg}.\textsc{poss}-\textsc{sbj}:\textsc{pcp}-hurt \textsc{sens}-be \textsc{lnk} \\
 \glt `It is a disease [that affects] infants who still drink their mother's milk.' (25-kACAl)
(\japhdoi{0003640\#S54})
\end{exe}

There are nevertheless prenominal genitival subject relative clauses, containing a subject participle, with the genitive \forme{ɣɯ} occurring between the relative clause and the head noun. This construction is especially common in texts translated from Chinese (due to calquing with \zh{的} <de>-relatives, §\ref{sec:genitival.relatives}), but also attested in natural speech, as in (\ref{ex:kWsAndza.GW}).

\begin{exe}
\ex \label{ex:kWsAndza.GW}
 \gll nɯnɯ kɯ-sɤ-ndza ɣɯ rɯdaʁ nɯnɯ tɕe kɯrŋi tu-kɯ-ti ŋu.  \\
 \textsc{dem} \textsc{sbj}:\textsc{pcp}-\textsc{apass}-eat \textsc{gen} animal \textsc{dem} \textsc{lnk} beast \textsc{ipfv}-\textsc{genr}-say be:\textsc{fact} \\
 \glt `Animals eating [other animals] are called `beasts'.' (150822 kWrNi)
(\japhdoi{0006260\#S6})
\end{exe}

%qambrɯ kɯ-rɤpɯ ci a-jɤ-tɯ-ɣɯt-nɯ ra, tɕe tɯmɯ nɤmkha ɯ-kɯ-luj ɣɯ raz ci a-jɤ-tɯ-ɣɯt-nɯ ra.

When subject relative clauses contain a complement clause, the main verb of the complement clause can be in subject participle form (see example \ref{ex:ndZikWsAndu}  in §\ref{sec:participial.complements.negative}).
 
\subsubsection{Other relative clauses}  \label{sec:subject.participle.other.relative}
In addition to subject relativization, the subject participle is also used in possessor relatives, when the relativized element is the possessor of the subject (§\ref{sec:S.possessor.relativization}).  The head-internal clause in (\ref{ex:kWkWtu.head.internal}) is such a possessor relative; its head noun \japhug{si}{tree}, possessor of the subject \japhug{ɯ-mat}{its fruits},  is marked with the genitive, showing that it belongs to the relative.  

\begin{exe}
\ex \label{ex:kWkWtu.head.internal}
 \gll [si ɣɯ ɯ-mat kɯ\redp{}kɯ-tu] nɯ ɯ-ku ri ɕ-ku-zo ɲɯ-ŋu tɕe. \\
 tree \textsc{gen} \textsc{3sg}.\textsc{poss}-fruit \textsc{total}\redp{}\textsc{sbj}:\textsc{pcp}-exist \textsc{dem} \textsc{3sg}.\textsc{poss}-top \textsc{loc} \textsc{tral}-\textsc{ipfv}-land \textsc{sens}-be \textsc{lnk} \\
 \glt `It lands on the top of all trees that have fruits.' (24-ZmbrWpGa)
(\japhdoi{0003628\#S41})
\end{exe}

 Headless possessor relative clauses, such as  \forme{nɯ-mtɕʰi mɤ-kɯ-pe} `those with a big mouth' in (\ref{ex:nWmtChi.mAkWpe}), are even more common.

\begin{exe}
\ex \label{ex:nWmtChi.mAkWpe}
\gll nɯ-mtɕʰi mɤ-kɯ-pe, kɤ-nɤtsɯ kɯ-ra ra kɯnɤ tu-kɯ-nɯ-ti nɯnɯra tɕaɣi tu-sɤrmi-nɯ ŋgrɤl. \\
 \textsc{3pl}.\textsc{poss}-mouth \textsc{neg}-\textsc{sbj}:\textsc{pcp}-be.good \textsc{inf}-hide \textsc{sbj}:\textsc{pcp}-be.needed \textsc{pl} also \textsc{ipfv}-\textsc{sbj}:\textsc{pcp}-\textsc{auto}-say \textsc{dem}:\textsc{pl} parrot \textsc{ipfv}-call-\textsc{pl} be.usually.the.case:\textsc{fact} \\
 \glt `People call `parrots', those with a big mouth, who tell [everything], including things that should [remain] hidden.'
 (24-qro) (\japhdoi{0003626\#S121})
\end{exe}
 
In addition, there are also participial relative clauses in \forme{kɯ-} whose relativized element is neither the subject or the possessor of the subject, in particular locative adjuncts with the relator noun \japhug{ɯ-stu}{place} (§\ref{sec:Wstu.relativization.subject}), the only argument of dummy subject verbs (§\ref{sec:dummy.subj.object.relativization}) or arguments from complement clauses embedded within the relative (§\ref{sec:relativization.complement.type}).

  \subsubsection{Purposive clauses and other complementation strategies}  \label{sec:subject.participle.complementation}
Subject participles occur in three types of complement clauses and complementation strategies (§\ref{sec:participial.clause.complementation strategies}). 

First, the three motion verbs \japhug{ɕe}{go}, \japhug{ɣi}{come} and  \japhug{ɬoʁ}{come out} (§\ref{sec:motion.verbs}) use subject participle clauses as purposive clauses (§\ref{sec:am.vs.mvc}), such as \forme{ndʑi-kɯ-qur} in (\ref{ex:ndZikWqur.chWGia}), whose (transitive or intransitive) subject is coreferent with that of the matrix verb.

\begin{exe}
\ex \label{ex:ndZikWqur.chWGia}
\gll aʑo [ndʑi-kɯ-qur] cʰɯ-ɣi-a je \\
\textsc{1sg} \textsc{2du}-\textsc{sbj}:\textsc{pcp}-help \textsc{ipfv}:\textsc{downstream}-come-\textsc{1sg} \textsc{sfp} \\
\glt `Let me come to help you.' (tWJo 2005) (\japhdoi{0003368\#S26})
\end{exe} 

\citet{sun12complementation} posits the category of \textit{supine} to refer to the cognate construction in Tshobdun. However, given the existence of object participle purposive clauses (§\ref{sec:object.participles.complement}, §\ref{sec:purposive.clause.motion.verbs}) above, I consider the supine to be only a specific use of the subject participle, rather than an independent morphology category.

Second,  subject participle relative clauses (§\ref{sec:relative.pretence}, §\ref{sec:constr.participial.clause}) are selected as objects or semi-objects by some verbs such as \japhug{nɯɕpɯz}{pretend}, `imitate', as in  (\ref{ex:kukWtshi.tonWCpWznW}).

\begin{exe}
\ex \label{ex:kukWtshi.tonWCpWznW}
 \gll  ʑara kɯ [cʰa nɯ ku-kɯ-tsʰi] to-nɯɕpɯz-nɯ, \\
\textsc{3pl} \textsc{erg} alcohol \textsc{dem} \textsc{ipfv}-\textsc{sbj}:\textsc{pcp}-drink \textsc{ifr}-pretend-\textsc{pl} \\
\glt  `They pretended to drink the alcohol (`imitated an alcohol drinker').'  (2012 Norbzang) (\japhdoi{0003768\#S77})
\end{exe}

Third, participial clauses occur as genuine complements in some constructions (§\ref{sec:participial.complements.negative}), in particular in negative existential constructions and when the matrix verb is itself in subject participle form, as in (\ref{ex:akWCWnNo}).

\begin{exe}
\ex \label{ex:akWCWnNo}
\gll  [[aʑo a-kɯ-ɕɯ-nŋo] kɯ-cʰa] me  \\
\textsc{1sg} \textsc{1sg}.\textsc{poss}-\textsc{sbj}:\textsc{pcp}-\textsc{caus}-be.defeated \textsc{sbj}:\textsc{pcp}-can not.exist:\textsc{fact} \\
 \glt `Nobody can defeat me.' (150821 edu de wangzi-zh) (\japhdoi{0006402\#S5})
 \end{exe}
 
\subsubsection{Lexicalized subject participles} \label{sec:lexicalized.subject.participle}
A certain number of subject participles have developed specialized meanings and can be considered to have been lexicalized. Some of these lexicalized participles are formally identical to the regular participle (Table  \ref{tab:lexicalized.S.nmlz}, for instance the noun \japhug{kɯcʰi}{candy} in (\ref{ex:akWchi})  as compared to the non-lexicalized participle \forme{kɯ-cʰi} `the one that is sweet' in (\ref{ex:kWchi.tu}). For such nouns, lexicalization is shown by the meaning specialization and the inability to take orientation, associated motion and polarity prefixes (but not possessive prefixes, as shown by he prefix \forme{a-} on \japhug{kɯcʰi}{candy} in \ref{ex:akWchi}).

\begin{exe}
\ex \label{ex:akWchi}
 \gll aʑo a-ŋgra a-kɯcʰi ci tɤ-χti ra \\
 \textsc{1sg} \textsc{1sg}.\textsc{poss}-salary \textsc{1sg}.\textsc{poss}-candy \textsc{indef} \textsc{imp}-buy[III] be.needed:\textsc{fact} \\
\glt `Give me a candy as a reward.' (140515 congming de wusui xiaohai-zh) (\japhdoi{0003998\#S78})
\end{exe}

\begin{exe}
\ex \label{ex:kWchi.tu}
 \gll tɕe nɯnɯ li tú-wɣ-ndza tɕe, kɯ-cʰi tu, mɤ-kɯ-cʰi tu. \\
\textsc{lnk} \textsc{dem} again \textsc{ipfv}-\textsc{inv}-eat \textsc{lnk} \textsc{sbj}:\textsc{pcp}-be.sweet exist:\textsc{fact} \textsc{neg}-\textsc{sbj}:\textsc{pcp}-be.sweet exist:\textsc{fact} \\
\glt `When one eats them, some are sweet, some are not.' (08-rasti)
(\japhdoi{0003460\#S49})
\end{exe}

\tabref{tab:lexicalized.S.nmlz} does not include the many names of profession / occupation built from the subject participles which are semantically transparent. We can distinguish two cases. 

First, labile verbs derive participial forms such as \japhug{kɯ-lɤɣ}{shepherd} or \japhug{kɯ-mɯrkɯ}{thief} (from \japhug{lɤɣ}{graze} and \japhug{mɯrkɯ}{steal}) without an obligatory possessive prefix; the absence of these prefixes cannot be attributed to lexicalization, since these verbs can also be used intransitively (§\ref{sec:lability.categories}). 

Second, plain transitive verbs have to undergo antipassive derivation (§\ref{sec:antipassive}) for their subject participles to be usable as names of professions. For instance, \japhug{kɯrɤrɤt}{writer} and \japhug{kɯrɤtʂɯβ}{tailor} are from the \forme{rɤ-} non-human antipassive forms of \japhug{rɤt}{write} and \japhug{tʂɯβ}{sew}, while  \japhug{kɯ-sɤ-sɯxɕɤt}{teacher} comes from the \forme{sɤ-} human antipassive of \japhug{sɯxɕɤt}{teach}(see \tabref{tab:kWrAverb}, §\ref{sec:antipassive.participle}). Without antipassive prefixes, the subject participles of (non-labile) transitive verbs require either an overt object or a definite and anaphorically recoverable object, and are used as names of professions. For instance, in (\ref{ex:tArmi.WkWrAt}), the participle \forme{ɯ-kɯ-rɤt} `the one writing it' is used with \japhug{tɤ-rmi}{name} as its object.

\begin{exe}
\ex \label{ex:tArmi.WkWrAt}
 \gll  [tɤ-rmi ɯ-kɯ-rɤt] tɤ-pɤtso nɯ ɯ-rkɯ ʑo, [...] pjɤ-zɣɯt tɕe, \\
 \textsc{indef}.\textsc{poss}-name \textsc{3sg}.\textsc{poss}-\textsc{sbj}:\textsc{pcp}-write  \textsc{indef}.\textsc{poss}-child \textsc{dem} \textsc{3sg}.\textsc{poss}-side \textsc{emph} { } \textsc{ifr}-reach \textsc{lnk} \\
\glt `It arrived near the boy who wrote the names (of the contestants).' (150826 shier shengxiao)
(\japhdoi{0006284\#S109})
\end{exe}

Moreover, I do not include among lexicalized participles cases like `shooting star' (\ref{ex:ZNgri.YWkWmArZaB}): although this expression is not compositional, the participle here is not frozen; the verb \japhug{mɤrʑaβ}{marry} can also occur in finite forms with the noun \japhug{ʑŋgri}{star} in the meaning `appear, fall (of a shooting star)' as in (\ref{ex:ZNgri.nWmArZaB}).

\begin{exe}
\ex \label{ex:ZNgri.YWkWmArZaB}
 \gll ʑŋgri ɲɯ-kɯ-mɤrʑaβ \\
 star \textsc{ipfv}-\textsc{sbj}:\textsc{pcp}-marry \\
 \glt `Shooting star' (`the wedding star')
 \ex \label{ex:ZNgri.nWmArZaB}
 \gll ʑŋgri nɯ-mɤrʑaβ ɯ-raŋ tɕe, tɯ-kɤrme cʰɯ́-wɣ-rɤɕi tɕe, cʰɯ-rɲɟi ŋu\\
 star \textsc{aor}-marry \textsc{3sg}.\textsc{poss}-time \textsc{lnk} \textsc{genr}.\textsc{poss}-hair \textsc{ipfv}:\textsc{downstream}-\textsc{inv}-pull \textsc{lnk} \textsc{ipfv}-be.long be:\textsc{fact} \\
 \glt `When a shooting star crosses the sky, if one pulls one's hair, it becomes longer.' (29-mWBZi) 	
 (\japhdoi{0003728\#S93})
\end{exe}
 
\begin{table}[H]
\caption{Lexicalized subject participles} \label{tab:lexicalized.S.nmlz} \centering
\begin{tabular}{llll}
\lsptoprule
Noun & Base verb \\
\midrule
\japhug{kɯβʁa}{noble} & \japhug{βʁa}{prevail, win}  \\
\japhug{kɯspoʁ}{hole} & \japhug{spoʁ}{have a hole}  \\
 \japhug{kɯcʰi}{candy} & \japhug{cʰi}{be sweet} \\
 \japhug{kɯmŋɤm}{ailment} & \japhug{mŋɤm}{hurt, feel pain} \\
 \japhug{kɯŋu}{right thing} & \japhug{ŋu}{be} \\
 \japhug{kɯmaʁ}{bad thing} & \japhug{maʁ}{not be} \\
\lspbottomrule
\end{tabular}
\end{table}

In the case of  \japhug{kɯŋu}{right thing}  and  \japhug{kɯmaʁ}{bad thing}, lexicalization is very advanced, and the meanings of the nouns are very different from those of the corresponding participles \japhug{kɯ-ŋu}{the one that is}  and  \japhug{kɯ-maʁ}{the one that is not}. Examples such as (\ref{ex:kWNu.mAtWnAme}) and (\ref{ex:kWNu.mAtWnAme}) illustrate their use in collocation with verbs like \japhug{nɤma}{work, make} and \japhug{fse}{be like}.

\begin{exe}
\ex \label{ex:kWNu.mAtWnAme}
 \gll  mɤ-ti-a ma kɯŋu mɤ-tɯ-nɤme \\
\textsc{neg}-say:\textsc{fact}-\textsc{1sg} \textsc{lnk} right.thing \textsc{neg}-2-make[III]:\textsc{fact} \\
\glt `I won't say it, because you will not do the right thing.' (2005 Kunbzang)
\end{exe}

\begin{exe}
\ex \label{ex:kWNu.mAfse}
 \gll  a-lɤ́-wɣ-ɕaβ-a tɕe tɕendɤre kɯŋu mɤ-fse \\
 \textsc{irr}-\textsc{pfv}-\textsc{inv}-catch.up-\textsc{1sg} \textsc{lnk} \textsc{lnk} right.thing \textsc{neg}-be.like:\textsc{fact} \\ 
\glt `If he catches up with me, [our enterprise] won't succeed.' (25-kAmYW-XpAltCin) 	(\japhdoi{0003642\#S36})
\end{exe}

The participle \japhug{kɯ-maʁ}{the one that is not} has been independently grammaticalized as an identity pronoun/determined \japhug{kɯmaʁ}{other} (see §\ref{sec:other.pro} and §\ref{sec:identity.modifier}).

From the nouns \japhug{kɯŋu}{right thing}  and  \japhug{kɯmaʁ}{bad thing}, the intransitive verbs \japhug{rɯkɯŋu}{do the right thing}, `take good care of one's family' and \japhug{rɯkɯŋu}{do bad things}, `happen bad things', `be clumsy' and the transitive verb \japhug{nɯkɯmaʁ}{make a mistake} have been derived by denominal derivation with \forme{rɯ-} and \forme{nɯ-} (§\ref{sec:denom.intr.rA}, §\ref{sec:denom.nW.pairing}).

The subject participle \forme{kɯ-mpɕɤr} `the beautiful one' of the verb \japhug{mpɕɤr}{be beautiful} has a derived denominal transitive verb \japhug{nɯkɯmpɕɤr}{wear (on important occasions)} with highly derived semantics, reflecting the lexicalized use of the participle in the meaning `decoration' as in (\ref{ex:WkWmpCAr.tAkABzu}).

 \begin{exe}
\ex \label{ex:WkWmpCAr.tAkABzu}
 \gll tɕe li ɯ-kɯ-mpɕɤr kɯ-fse tɤ-kɤ-βzu ɲɯ-ŋu tɕe    \\
\textsc{lnk} again \textsc{3sg}.\textsc{poss}-\textsc{sbj}:\textsc{pcp}-be.beautiful \textsc{sbj}:\textsc{pcp}-be.like \textsc{aor}-\textsc{obj}:\textsc{pcp}-make \textsc{sens}-be \textsc{lnk} \\
 \glt `[The mark on breads] is used for decoration.' (160706 WzbroN)
(\japhdoi{0006131\#S5})
\end{exe}

Several names of diseases only exist as intransitive verbs, and the disease itself or the person suffering from the disease can only be referred to by using a participial or infinitive form. In particular, the word \japhug{kɤ-kɯ-nɤndza}{leper} is the perfective subject participle of \japhug{nɤndza}{have leprosy}; this word has some degree of lexicalization (in particular, it is a common insult), but it behaves like a participle grammatically; in particular, it can undergo totalitative reduplication (§\ref{sec:totalitative.redp}, §\ref{sec:totalitative.relatives}), as in (\ref{ex:kWkAkWnAndza}).

\begin{exe}
\ex \label{ex:kWkAkWnAndza}
 \gll nɯnɯ kɯ, nɯnɯtɕu kɯ\redp{}kɯ-rɤʑi nɯ to-ɣɤ-mna. to-ɣɤ-mna ɯ-qʰu tɕe tɕendɤre <quanxian> tɕe kɯ\redp{}kɤ-kɯ-nɤndza nɯ ɲɤ-ɣɤ-me \\
 \textsc{dem} \textsc{erg} \textsc{dem}:\textsc{loc} \textsc{total}\redp{}\textsc{sbj}:\textsc{pcp}-stay \textsc{dem} \textsc{ifr}-\textsc{caus}-recover  \textsc{ifr}-\textsc{caus}-recover  \textsc{3sg}.\textsc{poss}-after \textsc{lnk} \textsc{lnk} all.the.district \textsc{loc} \textsc{total}\redp{}\textsc{aor}-\textsc{sbj}:\textsc{pcp}-have.leprosy \textsc{dem} \textsc{ifr}-\textsc{caus}-not.exist \\
\glt `He healed all those who were staying there (in the leper house). After he healed them, he had eradicated leprosy (removed all lepers) from our district.' (25-khArWm) (\japhdoi{0003644\#S71})
\end{exe}

Other disease names such as \japhug{tɤkɤzbɣaʁ}{migraine} (as in \ref{ex:tAkAzbGaR}), although clearly the perfective participle or infinitive of a verb root \forme{*azbɣaʁ}, is hardly ever attested in finite form.

\begin{exe}
\ex \label{ex:tAkAzbGaR}
 \gll tɤkɤzbɣaʁ nɯ tɤ-mŋɤm qʰe, tɕe nɯ ɯ-qʰu nɤ, ŋgɯsqɤ-rʑaʁ ʑo mɯ-tu-mna \\
 migraine \textsc{dem} \textsc{aor}-hurt \textsc{lnk} \textsc{lnk} \textsc{dem} \textsc{3sg}.\textsc{poss}-after \textsc{lnk} nine.or.ten-night \textsc{emph} \textsc{neg}-\textsc{ipfv}-recover \\
\glt `After the migraine starts, it does not recede for nine or ten days. (conversation taRrdo 2003)
\end{exe}

In addition, we find nouns in \forme{kɯ-} that can be suspected to be former lexicalized participles, such as \japhug{kɯjŋu}{oath}, which appears to contain the root of the verb   \japhug{ŋu}{be}, though the segment \forme{-j-} cannot be accounted for at the present moment,\footnote{In any case, the Tangut cognate \tangut{𗡔}{4600}{ŋwụ}{1.58}  `oath' shows that this derivation is very ancient and reflects a non-productive morphological process. } and \japhug{kɯmtɕʰɯ}{toy}, whose verbal root cannot be identified. The name \japhug{kɯsɤɣru}{mirror} (an archaic word in the process of being replaced by the Tibetan \japhug{χɕɤlzgoŋ}{mirror}) could also be a frozen subject participle  of the verb \japhug{ru}{look at}, but the nature of the prefix \forme{sɤɣ-} is unclear: it could be proprietive prefix (§\ref{sec:proprietive}), or alternatively, be analyzed as a frozen oblique participle prefix (§\ref{sec:lexicalized.oblique.participle}). In the second view, the prefix \forme{kɯ-} would not be identifiable.

Lexicalized subject participles appearing in compounds are also found. Several cases must be distinguished. First, we find subject participles of transitive verbs as the second member of a compound, with their object as the first member. This type of compounds are lexicalized headless relative clauses, like \japhug{qalekɯtsʰi}{species of kite}, which combines  \japhug{qale}{wind} and the participle \forme{ɯ-kɯ-tsʰi}  of the transitive verb \japhug{tsʰi}{block}, literally `blocking the wind'   (§\ref{sec:determinative.n.n}), a designation referring to this bird's ability to apparently remain unmoving in the sky, as described in (\ref{ex:kAnWqambWmbjom.mWjCe}).

\begin{exe}
\ex \label{ex:kAnWqambWmbjom.mWjCe}
 \gll  kɤ-nɯqambɯmbjom mɯ́j-ɕe kɯ nɯnɯre ɯ-stu ri ku-rɤʑi tɕe, [...] ɯ-ʁar nɯ tu-sɤlqɤlqɤt nɤ tu-sɤlqɤlqɤt ŋgrɤl  \\
 \textsc{inf}-fly \textsc{neg}:\textsc{sens}-go \textsc{erg} there \textsc{3sg}.\textsc{poss}-place \textsc{loc} \textsc{ipfv}-stay \textsc{lnk} { } \textsc{3sg}.\textsc{poss}-wing \textsc{dem} \textsc{ipfv}-flap.slightly \textsc{lnk}  \textsc{ipfv}-flap.slightly be.usually.the.case:\textsc{fact} \\
 \glt `It does not move [flying] but remains [in the sky] in place, slightly flapping its wings.' (23-RmWrcWftsa)
(\japhdoi{0003610\#S39})
\end{exe}

A second type involves two participles in apposition, as \japhug{kɯrŋukɯɣndʑɯr}{harvestman}, built from the subject participles of \japhug{rŋu}{parch} and  \japhug{ɣndʑɯr}{grind} (§\ref{sec:coordinative.n.n}). Both verbs being transitive, the absence of a possessive prefix \forme{ɯ-} is an additional clue that the form is fully lexicalized.

Third, there are compounds with the subject participle of transitive or intransitive verbs as first element (see also §\ref{sec.v.n.compounds}), for instance \japhug{kɯqurʑŋgri}{evening star} from \forme{ɯ-kɯ-qur} `the one helping him' (\japhug{qur}{help}) and \japhug{ʑŋgri}{star} literally `the star of the helper', for reasons explained in the following excerpt (\ref{ex:kWqur.ZNgri}).

\begin{exe}
\ex \label{ex:kWqur.ZNgri}
\gll ɯnɯnɯ kɯɕɯŋgɯ tɕe kɯ-qur ju-kɯ-ɕe tɕe nɯnɯ, mɯ-nɯ-ɬoʁ mɤɕtʂa nɯ tu-kɯ-nɯna mɯ-pjɤ-jɤɣ ɲɯ-ŋu tɕe,  tɕe núndʐa kɯqurʑŋgri tu-sɤrmi-nɯ \\
\textsc{dem} before \textsc{lnk} \textsc{sbj}:\textsc{pcp}-help \textsc{ipfv}-\textsc{genr}:S/O-go \textsc{lnk} \textsc{dem} \textsc{neg}-\textsc{aor}:\textsc{west}-come.out until \textsc{dem} \textsc{ipfv}-\textsc{genr}:S/O-rest \textsc{neg}-\textsc{ifr}.\textsc{ipfv}-be.possible \textsc{sens}-be \textsc{lnk} \textsc{lnk} for.this.reason evening.star \textsc{ipfv}-call-\textsc{pl} \\
\glt `Long ago, when one would go helping people, one was not supposed to rest until it came out, and for this reason it was called `star of the helper'.' (29-mWBZi)
(\japhdoi{0003728\#S58})
\end{exe}

An example with an intransitive verb is provided by the noun \japhug{kɯndzarmɯ}{type of rain}, compound of the participle of \japhug{ndzar}{drip dry} (§\ref{sec:causative.m}) with the noun \japhug{tɯ-mɯ}{sky, weather} (§\ref{sec:earth.IPN}). As shown by the definition provided for \forme{kɯndzarmɯ} in (\ref{ex:kWndzarmW}), the original meaning of this compound may have been `last drops of rain' -- the last rain before a relatively long period without rain.

\begin{exe}
\ex \label{ex:kWndzarmW}
\gll  iɕqʰa tɯ-mɯ nɯ sɲikuku ʑo, sɲikuku ʑo a-kɤ-lɤt tɕe tɕe, kɯ-maqʰu tɕe ci ci ku-lɤt, ci ci ɲɯ-jɯm kɯ-fse ŋu tɕe, ɯnɯnɯ ku-kɯ-lɤt, <zhenyu> kɯ-fse ku-kɯ-lɤt nɯ, nɯnɯ kɯ-fse a-kɤ-lɤt tɕe ɯ-qʰu tɕe tɯ-mɯ mɤ-lɤt tu-kɯ-ti ŋu tɕe, nɯnɯ tɯ-mɯ ku-kɯ-lɤt nɯ kɯndzarmɯ tu-kɯ-ti ŋu \\
the.aforementioned \textsc{indef}.\textsc{poss}-sky \textsc{dem} every.day \textsc{emph} every.day \textsc{emph} \textsc{irr}-\textsc{pfv}-release \textsc{lnk} \textsc{lnk} \textsc{sbj}:\textsc{pcp}-be.after \textsc{lnk} one one \textsc{ipfv}-release one one \textsc{ipfv}-be.sunny \textsc{sbj}:\textsc{pcp}-be.like be:\textsc{fact} \textsc{lnk} \textsc{dem} \textsc{ipfv}-\textsc{sbj}:\textsc{pcp}-release showery.rain \textsc{sbj}:\textsc{pcp}-be.like \textsc{ipfv}-\textsc{sbj}:\textsc{pcp}-release  \textsc{dem} \textsc{dem} \textsc{sbj}:\textsc{pcp}-be.like \textsc{irr}-\textsc{pfv}-release \textsc{lnk} \textsc{3sg}.\textsc{poss}-after \textsc{lnk} \textsc{indef}.\textsc{poss}-sky \textsc{neg}-release:\textsc{fact} \textsc{ipfv}-\textsc{genr}-say be:\textsc{fact} \textsc{lnk} \textsc{dem} \textsc{indef}.\textsc{poss}-sky \textsc{ipfv}-\textsc{sbj}:\textsc{pcp}-release \textsc{dem} type.of.rain \textsc{ipfv}-\textsc{genr}-say be:\textsc{fact} \\
\glt `When it rains everyday (continuously), followed by sporadic rain (sometimes it rains, sometimes it is sunny), when there is a showery rain and after that no more rain, this type of [showery] rain is called \forme{kɯndzarmɯ}.' (definition, 2015-04-18)
\end{exe}

Nominalizations with the \forme{x-/ɣ-} prefix (§\ref{sec:G.nmlz}) are ancient lexicalized subject participles that have undergone a syllable reduction rule (§\ref{sec:velar.class.prefix}, \citealt[6]{jacques14antipassive}) and have become completely separated from their base verbs synchronically.

There are also a few adverbs derived from verbs with a \forme{kɯ-} prefix, but these are best analyzed as lexicalized stative infinitives (§\ref{sec:velar.inf.adverb}).

\subsection{Object participles} \label{sec:object.participle}
The object participle is a nominalized form which refers to an entity corresponding to the object (§\ref{sec:absolutive.P}) or semi-object (§\ref{sec:semi.object}) of the base verb. Nearly all transitive and semi-transitive verbs (except for a handful of exceptions, §\ref{sec:nmlz.defective}) can build an object participle by adding the prefix \forme{kɤ-} (for instance \forme{kɤ-ndza} from the verb \japhug{ndza}{eat} in \ref{ex:kAndza}). This form is homophonous with, and historically related to the velar infinitive (§\ref{sec:velar.inf}, §\ref{sec:velar.nmlz.history}).

 \begin{exe} 
\ex \label{ex:kAndza}
\gll kɤ-ndza \\
   \textsc{obj}:\textsc{pcp}-eat \\
 \glt  `The one that is eaten.' (many attestations)
 \end{exe}

In the case of secundative verbs (§\ref{sec:ditransitive.secundative}), the object participle can either refer to the recipient or the theme, as in (\ref{ex:nWkAmbi}); this question is discussed in more detail in §\ref{sec:object.participle.relatives}.

  \begin{exe} 
\ex \label{ex:nWkAmbi}
\gll nɯ-kɤ-mbi \\
   \textsc{aor}-\textsc{obj}:\textsc{pcp}-give \\
 \glt  `The one that he has given it to.'
 \glt `The one that has been given to him.'  (many attestations)
 \end{exe}

   
 In this section, I first describe the morphological properties of object participles (compatibility with possessive prefixes §\ref{sec:object.participle.possessive} and other prefixes §\ref{sec:object.participle.other.prefixes}). Then, I discuss several cases of ambiguity between object participles and other \forme{kɤ-} prefixed forms in §\ref{sec:object.participle.ambiguity} (see also §\ref{sec:infinitives.participles}). The uses of object participles to build relative clauses and complement clauses are described in  §\ref{sec:object.participle.relatives} and §\ref{sec:object.participles.complement}. Finally, I present a few cases of lexicalized object participles in §\ref{sec:lexicalized.object.participle}.
 
\subsubsection{Possessive prefixes on object participles}  \label{sec:object.participle.possessive} 
Unlike subject participles, object participles never require a possessive prefix. An optional possessive prefix coreferent with the transitive subject, as in (\ref{ex:akAsWz}), can however be added.
  
  \begin{exe}
\ex \label{ex:akAsWz}
\gll a-kɤ-sɯz    \\
   \textsc{1sg}-\textsc{nmlz}:P-know \\
 \glt  `The one that I know.' (many attestations)
 \end{exe}

In the case of semi-transitive verbs, the possessive prefix is also coreferent with the subject, as in the form \forme{ɯ-kɤ-rga} `the one that he likes' in (\ref{ex:stu.WkArga}), built in the same way as the object participle of the transitive (tropative, §\ref{sec:tropative}) verb \japhug{nɤmɯm}{find tasty}.

\begin{exe}
\ex \label{ex:stu.WkArga}
\gll ri nɯnɯ stu ɯ-kɤ-rga, ɯ-kɤ-nɤ-mɯm pjɤ-ɕti. \\
\textsc{lnk} \textsc{dem}  most \textsc{3sg}.\textsc{poss}-\textsc{obj}:\textsc{pcp}-like \textsc{3sg}.\textsc{poss}-\textsc{obj}:\textsc{pcp}-\textsc{trop}-be.tasty \textsc{ifr}.\textsc{ipfv}-be.\textsc{aff} \\
\glt `But it was what he liked most, what he found most tasty.' (160703 poucet3) (\japhdoi{0006107\#S72})
\end{exe}

In addition to semi-transitive verbs, the complement-taking verb \japhug{cʰa}{can} has object participles taking possessive prefixes meaning `the one that $X$ can $Y$', $X$ being the subject (marked by the possessive prefix), and $Y$ the verb in the complement clause, which can be overt or not as in (\ref{ex:nWmAkAcha}), where \forme{nɯ-mɤ-kɤ-cʰa} stands for \forme{kɤ-ndo nɯ-mɤ-kɤ-cha} `the one(s) that they are able to catch' (see additional examples in §\ref{sec:out.complement.relativization.cha}).
 
\begin{exe}
\ex  \label{ex:nWmAkAcha}
\gll tɕe nɯ-mɤ-kɤ-cʰa nɯ kʰɯna χsɯm pɯ-tu qʰe, nɯra kɯ rcanɯ ɕlaʁ ʑo ku-ndo-nɯ ɲɯ-ɕti. \\
\textsc{lnk} \textsc{3pl}.\textsc{poss}-\textsc{neg}-\textsc{inf}-can \textsc{dem} dog three \textsc{pst}.\textsc{ipfv}-exist \textsc{lnk} \textsc{dem}:\textsc{pl} \textsc{erg} \textsc{unexp}:\textsc{deg} \textsc{ideo}.I:immediately \textsc{emph} \textsc{ipfv}-catch-\textsc{pl} \textsc{sens}-be.\textsc{aff} \\
\glt `The [rats] that [the people] had been unable to [catch], there were three dogs, and these [dogs] caught them at once.' (150831 BZW kAnArRaR)
(\japhdoi{0006378\#S41})
\end{exe}

\subsubsection{Associated motion, polarity and orientation preverbs on object participles}  \label{sec:object.participle.other.prefixes}
Object participles, like subject participles, are compatible with polarity (\ref{ex:amAkAsWz}), associated motion (\ref{ex:WCWkAnAma}) and orientation preverbs (\ref{ex:WCWkAnAma}).

\begin{exe}
\ex  \label{ex:amAkAsWz}
\gll tɕe aʑo a-mɤ-kɤ-sɯz tɤjmɤɣ nɯ kɤ-ndza mɤ-naz-a \\
\textsc{lnk} \textsc{1sg} \textsc{1sg}.\textsc{poss}-\textsc{neg}-\textsc{obj}:\textsc{pcp}-know mushroom \textsc{dem} \textsc{inf}-eat \textsc{neg}-dare:\textsc{fact}-\textsc{1sg}  \\
\glt `I do not dare to eat mushrooms that I do not recognize.' (23-mbrAZim)
(\japhdoi{0003604\#S105})
\end{exe}

\begin{exe}
\ex  \label{ex:WCWkAnAma}
\gll ɯ-pɕi tɕe ɯ-ɕɯ-kɤ-nɤma ci pjɤ-tu tɕe, \\
\textsc{3sg}.\textsc{poss}-outside \textsc{lnk} \textsc{3sg}.\textsc{poss}-\textsc{tral}-\textsc{obj}:\textsc{pcp}-work \textsc{indef} \textsc{ifr}.\textsc{ipfv}-exist \textsc{lnk} \\
\glt  `[The mouse] had something to do outside.' (140518 mao he laoshu-zh)
(\japhdoi{0004030\#S77})
\end{exe}

Associated motion and polarity prefixes on object participles co-occur with possessive prefixes, as shown by  (\ref{ex:amAkAsWz}) and  (\ref{ex:WCWkAnAma})  above, but orientation preverbs (whether perfective or imperfective) do not. This is an important difference between subject and object participles (§\ref{sec:subject.participle.other.prefixes}). Object participles only have at most two prefixes.

\begin{exe}
\ex  \label{ex:pjWKAnWji}
\gll tɕe pɤjka wuma nɯnɯ tɕe, pjɯ-kɤ-nɯ-ji ŋu tɕe, \\
\textsc{lnk} gourd really \textsc{dem} \textsc{lnk} \textsc{ipfv}-\textsc{obj}:\textsc{pcp}-\textsc{auto}-plant be:\textsc{fact} \textsc{lnk} \\
\glt `The gourd proper is cultivated (it does not grow on its own).' (16-CWrNgo) (\japhdoi{0003518\#S61})
\end{exe}

Finite relative clauses, instead of object participles, can be used to specify both TAME and the subject (§\ref{sec:finite.relatives}).

Unlike subject participles, object participles are attested with the progressive \forme{asɯ-} prefix, as in (\ref{ex:pWkASWndo}). It is the only non-finite form compatible with this prefix.  

\begin{exe}
\ex  \label{ex:pWkASWndo}
\gll  tɕʰeme nɯ kɯ iɕqʰa, ɯ-jaʁ <meihua>, mɯntoʁ pɯ-kɤ-ɤsɯ-ndo nɯ pjɤ-ɣɤrɤt.  \\
girl \textsc{dem} \textsc{erg} the.aforementioned \textsc{3sg}.\textsc{poss}-hand plum.blossom flower \textsc{pst}.\textsc{ipfv}-\textsc{obj}:\textsc{pcp}-\textsc{prog}-take \textsc{dem} \textsc{ifr}-throw \\
\glt `The girl threw down the plum blossom, the flower that she was holding in her hand.' (150907 yingning-zh)
(\japhdoi{0006264\#S28})
\end{exe}

These forms are rare and difficult to identify, as they are always ambiguous with object participles or infinitive of causativized verbs. In the case of (\ref{ex:pWkASWndo}), the context makes it clear that interpretation as the participle of a progressive form is the only possibility, as the same verb with the progressive appears a few sentences before in (\ref{ex:meihua.ci.pjAkAsWndoci}).

\begin{exe}
\ex  \label{ex:meihua.ci.pjAkAsWndoci}
\gll   ɯ-jaʁ nɯtɕu, iɕqʰa, <meihua> ci pjɤ-k-ɤsɯ-ndo-ci, \\
\textsc{3sg}.\textsc{poss}-hand \textsc{dem}:\textsc{loc} \textsc{filler} plum.blossom \textsc{indef} \textsc{ifr}.\textsc{ipfv}-\textsc{peg}-\textsc{prog}-take-\textsc{peg} \\
 \glt `She was holding a plum blossom in her hand.' (150907 yingning-zh) (\japhdoi{0006264\#S19})
\end{exe}

\subsubsection{Ambiguity} \label{sec:object.participle.ambiguity}
There is rampant ambiguity between object participles, \forme{kɤ-} infinitives and subject participles of passive verbs.  The question of the ambiguity between object participles and \forme{kɤ-} infinitives is discussed in §\ref{sec:velar.inf.ambiguity}.

The passive \forme{a-} merges with the subject participle as \ipa{kɤ}, homophonous with the infinitive and the object participle. Potentially ambiguous examples are very common. For instance, in (\ref{ex:kArku.passive}), the form \ipa{kɤrku} could be argued to be an object participle \forme{kɤ-rku} or a passive subject participle \forme{kɯ-ɤ-rku}; the second option is chosen here due to the semantics, which fits the passive \japhug{arku}{be put in, be located in} better (as this passive verb is in the process of becoming a locative existential verb, §\ref{sec:existential.basic}). In the absence of any argument in favour of the passive analysis, the ambiguous \ipa{kɤ-} forms are analyzed as object participles by default.

\begin{exe}
\ex \label{ex:kArku.passive}
 \gll  sɤtɕʰa ɯ-ŋgɯ kɯ-ɤ-rku <yangyu> cʰo lɤpɯɣ nɯra tu-ndze ŋgrɤl. \\
 earth \textsc{3sg}.\textsc{poss}-inside \textsc{sbj}:\textsc{pcp}-\textsc{pass}-put.in potato \textsc{comit} radish \textsc{dem}:\textsc{pl} \textsc{ipfv}-eat[III] be.usually.the.case:\textsc{fact} \\
 \glt `It eats the radish and the potatoes that are in the ground.' (25-akWzgumba)
(\japhdoi{0003632\#S21})
\end{exe}

Due to the fact that passive verbs in Japhug are barely attested in perfective forms (§\ref{sec:passive.stative}), participles with perfective prefixes can be considered to be object participles, especially in cases like (\ref{ex:YAXtAr.nWkAXtAr}), where the participle \forme{nɯ-kɤ-χtɤr}  `(those) that have been scattered' occurs in a sentence following the transitive form \forme{ɲɤ-χtɤr} `it scattered, it smashed'.

\begin{exe}
\ex \label{ex:YAXtAr.nWkAXtAr}
 \gll    to-ɣi tɕe nɯ-ʑmbrɯ ɲɤ-χtɤr ʑo ɲɯ-ŋu tɕe, ɯ-zda ra nɯ-pʰe, nɤki,  ``nɯnɯ ʑmbrɯ nɯ-kɤ-χtɤr nɯ ɯ-taʁ kɤ-ɴqoʁ-nɯ ra'' to-ti\\
 \textsc{ifr}:\textsc{up}-come \textsc{lnk} \textsc{3pl}.\textsc{poss}-ship \textsc{ifr}-scatter \textsc{emph} \textsc{sens}-be \textsc{lnk} \textsc{3sg}.\textsc{poss}-companion pl \textsc{3pl}.\textsc{poss}-\textsc{dat} \textsc{filler} \textsc{dem} ship \textsc{aor}-\textsc{obj}:\textsc{pcp}-scatter \textsc{dem} \textsc{3sg}.\textsc{poss}-on \textsc{imp}-hang-\textsc{pl} be.needed:\textsc{fact} \textsc{ifr}-say\\
 \glt `The [monster] came up and smashed their ship, and [Norbzang] said to his companions: ``Grab the [pieces of the] ship that have been scattered''.' (2012 Norbzang)
(\japhdoi{0003768\#S29})
\end{exe}

The same analysis as object participles, rather than passive subject participles is applied to examples of perfective \forme{kɤ-} forms also when the transitive verb is not found in finite form in a neighbouring sentence, such as (\ref{ex:pWkAprAt}).

\begin{exe}
\ex \label{ex:pWkAprAt}
 \gll fsapaʁ ɯ-ŋgo rcanɯ, pɯ-kɤ-prɤt ʑo tɤ-fse ɲɯ-ŋu. \\
 animals \textsc{3sg}.\textsc{poss}-disease \textsc{unexp}:\textsc{deg} \textsc{aor}-\textsc{obj}:\textsc{pcp}-break \textsc{emph} \textsc{aor}-be.like \textsc{sens}-be \\
 \glt  `It was like the disease of the cattle had been [suddenly] stopped.' (2003 kAndZislama)
\end{exe}

A more marginal case of homophony occurs between object participles  and velar infinitives on the one hand, and several finite forms taking the series A orientation preverb \forme{kɤ-} on the other hand (§\ref{sec:infinitives.participles}).
 
\subsubsection{Object relative clauses} \label{sec:object.participle.relatives} 
Object participles can be used to build object relative clauses, but compete in this function with finite relatives (§\ref{sec:object.relativization}). They differ in this regard from subject relatives, which are the only available construction to relativize transitive and intransitive subjects.


As was described in §\ref{sec:object.participle.other.prefixes}, object participles, unlike subject and oblique participles, cannot combine possessive and orientation preverbs.

Object participles with orientation preverbs are used in relative clauses with indefinite subjects, or with definite third person subjects as in (\ref{ex:qajGi.nWkAmbi}).  
 
\begin{exe}
\ex \label{ex:qajGi.nWkAmbi}
\gll     [ɬamu kɯ qajɣi nɯ-kɤ-mbi] nɯ tu-ndze pjɤ-ŋu \\
\textsc{anthr} \textsc{erg} bread \textsc{aor}-\textsc{sbj}:\textsc{pcp}-give \textsc{dem} \textsc{ipfv}-eat[III] \textsc{ifr}.\textsc{ipfv}-be \\
\glt `[As] he was eating the [pieces of] bread that Lhamo had given him.' (2002 qajdoskAt)
\end{exe}  

The only example of first or second person that could be interpreted as subject in a perfective object participle relative in the corpus is (\ref{ex:aZo.kW.pWkAsWrAt}), but in this example (translated from Chinese), the referent of the first person is a pen that has been used to write a poem; the ergative postpositional phrase \forme{aʑo kɯ} here can be either analyzed as an instrument (`the poem that has been written using me') or as a causee  (§\ref{sec:causee.kW}, `the poem that he has made me write'), as shown by the presence of the causative \forme{sɯ-} prefix, not a subject.

\begin{exe}
\ex \label{ex:aZo.kW.pWkAsWrAt}
\gll   [aʑo kɯ pɯ-kɤ-sɯ-rɤt] nɯnɯ pjɯ-ndɯn ɲɯ-ŋu nétɕi \\
\textsc{1sg} \textsc{erg} \textsc{aor}-\textsc{obj}:\textsc{pcp}-\textsc{caus}-write \textsc{dem} \textsc{ipfv}-read \textsc{sens}-be \textsc{sfp} \\
\glt (The pen said): [the poet is reading the poem] that has been written using me.' (150818 bi he moshuihu-zh)
(\japhdoi{0006382\#S131})
\end{exe}  

When the subject is first or second person, an object participle with a possessive prefix  is used instead. In (\ref{ex:iZo.jikArku}) for instance, we find \forme{ji-kɤ-rku} `the thing that we give' (see §\ref{sec:z.nmlz} concerning the meaning of this verb) and \forme{nɤ-kɤ-sɯso} `the thing that you think / that you want' with a first plural and a second singular possessive prefix, respectively.

\begin{exe}
\ex \label{ex:iZo.jikArku}
\gll   nɯ ma iʑo ji-kɤ-rku me,  atu spɣi tɤ-ɕe qʰe, laχtɕʰa ŋotɕu nɤ-kɤ-sɯso ʑo nɯnɯ, 
nɤ-mɲaʁ, nɤ-rna, nɤ-ɕna cʰo ra kɯ\redp{}kɯ-spoʁ nɯ ɯ-ŋgɯ tɕe a-kɤ-tɯ-rke qʰe, \\
\textsc{dem} apart.from \textsc{1pl} \textsc{1pl}.\textsc{poss}-\textsc{obj}:\textsc{pcp}-put.in not.exist:\textsc{fact} up.there granary \textsc{imp}:\textsc{up}-go \textsc{lnk} thing where \textsc{2sg}.\textsc{poss}-\textsc{obj}:\textsc{pcp}-think \textsc{emph} \textsc{dem} \textsc{2sg}.\textsc{poss}-eye  \textsc{2sg}.\textsc{poss}-ear  \textsc{2sg}.\textsc{poss}-nose \textsc{comit} \textsc{pl}  \textsc{total}\redp{}\textsc{sbj}:\textsc{pcp}-have.a.hole \textsc{dem} \textsc{3sg}.\textsc{poss}-inside \textsc{loc} \textsc{irr}-\textsc{pfv}-2-put.in[III] \textsc{lnk} \\
\glt `We don't have anything else to give you as a departing present, so go up there in the granary, and whatever you want, put it in all the holes [in your body], your eyes, your ears, your nose etc.' (31-deluge) 	(\japhdoi{0004077\#S134})
\end{exe}  

When the subject is a definite third person, it is also possible to have a third person possessive prefix on the object participle, as in (\ref{ex:WkWnWmbrApW2}) (or \ref{ex:WCWkAnAma} above).

\begin{exe}
\ex \label{ex:WkWnWmbrApW2}
\gll  lɤ-fsoʁ ɯ-jɯja nɯ pjɯ-ru tɕe [ɯ-kɤ-nɯmbrɤpɯ] nɯ kʰu pɯ-ɕti ɲɯ-ŋu,  \\
\textsc{aor}-be.clear    \textsc{3sg}-along  \textsc{dem} \textsc{ipfv}:\textsc{down}-look \textsc{lnk} \textsc{3sg}.\textsc{poss}-\textsc{obj}:\textsc{pcp}--ride \textsc{dem} tiger \textsc{pst}.\textsc{ipfv}-be.\textsc{aff}  \textsc{sens}-be \\
\glt `As the day broke, looking down, [it dawned on him that] what he was riding was a tiger.' (2005 khu)
\end{exe}

Unlike in Tshobdun (\citealt[10]{jacksonlin07}), in Japhug object participial relatives with possessive prefixes are not restricted to generic state of affairs, but can refer to particular situations as in examples such as (\ref{ex:WkWnWmbrApW2}) and (\ref{ex:nAkAti.nWra}).

\begin{exe}
\ex \label{ex:nAkAti.nWra}
\gll a-ɬaʁ, tɕe nɤ-kɯ-mŋɤm tɕʰi ɲɯ-fse ma [alo qʰaqʰu nɤ-kɤ-ti] nɯra tɤ-stu-t-a \\
\textsc{1sg}.\textsc{poss}-aunt \textsc{lnk} \textsc{2sg}.\textsc{poss}-\textsc{sbj}:\textsc{pcp}-hurt what \textsc{sens}-be.like \textsc{lnk} upstream behind.the.house \textsc{2sg}.\textsc{poss}-\textsc{obj}:\textsc{pcp}-say \textsc{dem}:\textsc{pl} \textsc{aor}-do.like-\textsc{pst}:\textsc{tr}-\textsc{1sg} \\
\glt `Stepmother, how do you feel, I did the things you said [about creating a lake] up there behind the house.' (28-smAnmi)
(\japhdoi{0004063\#S341})
\end{exe}


In object participial relatives, when the relativized element is overt, it is generally located before the participle, as in (\ref{ex:thWkAraGdWt}). 

\begin{exe}
\ex \label{ex:thWkAraGdWt}
\gll  [nɯŋa ɯ-ndʐi tʰɯ-kɤ-rɤɣdɯt], tʰɯ-kɤ-tʂɯβ nɯ ɯ-ŋgɯ nɯtɕu ko-ɕe  \\
cow \textsc{3sg}.\textsc{poss}-skin \textsc{aor}-\textsc{obj}:\textsc{pcp}-skin \textsc{aor}-\textsc{obj}:\textsc{pcp}-sew \textsc{dem} \textsc{3sg}.\textsc{poss}-inside \textsc{dem}:\textsc{loc} \textsc{evd}:\textsc{east}-go \\
\glt  `He [crawled] into the cow hide that had been skinned and sewed.'  (2-deluge2012) 	(\japhdoi{0003376\#S30})
\end{exe}  

While the relative clause in (\ref{ex:thWkAraGdWt}) can either be interpreted as head-internal or post-nominal, clear examples of head-internal object relatives are found (§\ref{sec:head-internal.relative.postnominal}), as in (\ref{ex:kha.tAkAsWBzu}), where the head noun \forme{kʰa} is located between the instrumental adjunct \forme{kɯ-cʰi kɯ} and the participle.
 
\begin{exe}
\ex \label{ex:kha.tAkAsWBzu}
\gll [kɯ-cʰi kɯ kʰa tɤ-kɤ-sɯ-βzu] ci pjɤ-mto-ndʑi\\
\textsc{sbj}:\textsc{pcp}-be.sweet \textsc{erg} house \textsc{aor}-\textsc{obj}:\textsc{pcp}-\textsc{caus}-make \textsc{indef} \textsc{ifr}-see-\textsc{du}\\
\glt `They saw a house that was made of sweets.' (140507 tangguowu-zh)
(\japhdoi{0003933\#S75})
\end{exe}

Prenominal object participial relatives are mainly attested with participles with a possessive prefix, such as \forme{a-kɤ-sɯz} in (\ref{ex:paXCi.akAsWz}). 

\begin{exe}
\ex \label{ex:paXCi.akAsWz}
\gll   nɯnɯ paχɕi [aʑo a-kɤ-sɯz] nɯ nɯra ɣɤʑu \\
dem apples \textsc{1sg} \textsc{1sg}.\textsc{poss}-\textsc{obj}:\textsc{pcp}-know allium \textsc{dem} \textsc{dem}:\textsc{pl} exist:\textsc{sens} \\
\glt `Of the apples, [the aforementioned] are the ones I know about.' (07-paXCi)
(\japhdoi{0003430\#S67})
\end{exe}  

Examples of prenominal object participial relatives with orientation preverbs are also attested, as in (\ref{ex:nWkAsWBzu.kha}) (a near minimal pair with \ref{ex:kha.tAkAsWBzu}) and (\ref{ex:khru.kW.thWkAsWlAt}). This type of relative clauses are considerably less common that the corresponding head-internal ones, especially in texts that have not been translated from Chinese as in (\ref{ex:khru.kW.thWkAsWlAt}).

\begin{exe}
\ex \label{ex:nWkAsWBzu.kha}
\gll [lonba ɕom kɯ nɯ-kɤ-sɯ-βzu] kʰa pjɤ-ŋu \\
all iron \textsc{erg} \textsc{aor}-\textsc{obj}:\textsc{pcp}-\textsc{caus}-make house \textsc{ifr}.\textsc{ipfv}-be \\
\glt `It was a house made completely of iron.' (140505 liuhaohan zoubian tianxia-zh)
(\japhdoi{0003913\#S149})
\end{exe}


\begin{exe}
\ex \label{ex:khru.kW.thWkAsWlAt}
\gll tɕe [kʰru kɯ tʰɯ-kɤ-sɯ-lɤt] laʁdɯn kɯnɤ tu ma\\
\textsc{lnk} cast.iron \textsc{erg} \textsc{aor}-\textsc{obj}:\textsc{pcp}-\textsc{caus}-release tool also exist:\textsc{fact} \textsc{lnk}\\
\glt `There are also tools that are made of cast iron.' (30-Com)
(\japhdoi{0003736\#S28})
\end{exe}

As mentioned above (example \ref{ex:nWkAmbi}), the object participles of secundative verbs can either refer to their object proper (the recipient, §\ref{sec:ditransitive.secundative}) or to the theme, which is not indexed on the verb but occurs in absolutive form (§\ref{sec:theme.ditransitive}). In fact, in the corpus examples of theme relativization with the object participle are quite common (as \ref{ex:qajGi.nWkAmbi} above and \ref{ex:WkAmbi.maNe} and \ref{ex:pWkAsWxCAt.nWra} below), but recipient relativization is rare (\ref{ex:xCiri.nWkAsWxCAt}). Examples can however be elicited without difficulty.

\begin{exe}
\ex \label{ex:WkAmbi.maNe}
\gll nɯ ma ɯ-kɤ-mbi maŋe tɕe, ``a-me ta-mbi ra" to-ti tɕe, \\
\textsc{dem} apart.from \textsc{3sg}.\textsc{poss}-\textsc{obj}:\textsc{pcp}-give not.exist:\textsc{sens} \textsc{lnk} \textsc{1sg}.\textsc{poss}-daughter 1\fl{}2-give:\textsc{fact}  be.needed:\textsc{fact} \textsc{ifr}-say \textsc{lnk} \\
\glt `He had nothing else to give him, and said `I give you my daughter'.' (2011-04-smanmi)
\end{exe}

\begin{exe}
\ex \label{ex:pWkAsWxCAt.nWra}
\gll   tɕendɤre [tɯmɯkɤrŋi kɯ pɯ-kɤ-sɯxɕɤt] ra ɲɤ-nɤxtʂɯn tɕe tɕe nɯɕimɯma ʑo pjɤ-nɯ-ɕe. \\
\textsc{lnk} heaven \textsc{erg} \textsc{aor}-\textsc{obj}:\textsc{pcp}-teach \textsc{pl} \textsc{ifr}-be.grateful \textsc{lnk} \textsc{lnk} immediately \textsc{emph} \textsc{ifr}:\textsc{down}-\textsc{vert}-go \\
\glt `[Pu'an] was thankful for the things that the god of heaven had taught him and went back [to earth] immediately.' (150827 taisui-zh)
(\japhdoi{0006390\#S134})
\end{exe}

\begin{exe}
\ex \label{ex:xCiri.nWkAsWxCAt}
\gll    iɕqʰa, [kɤntɕʰɯ-xpa ʑo, nɤki, xɕiri nɯ-kɤ-sɯxɕɤt] nɯ pjɤ-sat, \\
 the.aforementioned several-year \textsc{emph} \textsc{filler} weasel \textsc{aor}-\textsc{obj}:\textsc{pcp}-teach \textsc{dem} \textsc{ifr}-kill \\
\glt `He killed the weasel that he had trained for several years.'  (140518 xuezhe he huangshulang-zh)
(\japhdoi{0004032\#S28})
\end{exe}

With indirective verbs, the object participle can only refer to the theme, as in (\ref{ex:nAkAthu.WGAZu}), for these verbs the recipient must be relativized with the oblique participle  (§\ref{sec:other.oblique.participle.relatives}).

\begin{exe}
\ex \label{ex:nAkAthu.WGAZu}
\gll nɤ-kɤ-tʰu ɯ-ɣɤʑu nɤ, tɤ-tʰe jɤɣ \\
\textsc{2sg}.\textsc{poss}-\textsc{obj}:\textsc{pcp}-ask \textsc{qu}-exist:\textsc{sens} \textsc{lnk} \textsc{imp}-ask[III] be.allowed:\textsc{fact} \\
\glt `If you have and questions, you can ask them.' (conversation 14-11-08)
\end{exe}

The semi-object of semi-transitive verbs (§\ref{sec:semi.object}) can also be relativized with a object participial relative, as \forme{ji-kɤ-rga} `the one that we like' in (\ref{ex:stu.jikArga}).

\begin{exe}
\ex \label{ex:stu.jikArga}
\gll  iɕqʰa <macha> kɤ-ti nɯ [iʑora stu ji-kɤ-rga] ŋu \\
the.aforementioned macha.tea \textsc{obj}:\textsc{pcp}-say \textsc{dem} \textsc{1pl} most \textsc{1pl}.\textsc{poss}-\textsc{obj}:\textsc{pcp}-like be:\textsc{fact} \\
\glt `The [type of tea] called `macha' is what we like best.' (30-macha)
(\japhdoi{0003746\#S1})
\end{exe}

Secundative verbs undergoing antipassivization become semi-transitive verbs (§\ref{sec:ditransitive.secundative}, §\ref{sec:antipassive.ditransitive}) with the theme remaining the semi-object. Like other semi-transitive verbs, these antipassive verbs can build an object participle, which can then be used to relativize the theme, as in (\ref{ex:nAkArAmbi}).

\begin{exe}
\ex \label{ex:nAkArAmbi}
\gll  nɤ-kɤ-rɤ-mbi nɯ tɕʰi pɯ-ŋu? \\
\textsc{2sg}.\textsc{poss}-\textsc{obj}:\textsc{pcp}-\textsc{apass}-give \textsc{dem} what \textsc{pst}.\textsc{ipfv}-be \\
\glt `What was it that you gave (to people)?' (elicited)
\end{exe}

Object participles also occur in genitival relatives (§\ref{sec:genitival.relatives}, postnominal relative with the genitive postposition \forme{ɣɯ} occurring between the relative clause and the head noun), as in example (\ref{ex:tAkAsWBzu.GW.tWxtsa}). This type of examples is frequently found in texts translated from Chinese, but unattested in the rest of the corpus for object relativization, and is a clear case of calque (§\ref{sec:genitival.relatives}). Although speakers do accept these examples, they cannot be considered to be representative of the normal grammar of the language.

\begin{exe}
\ex \label{ex:tAkAsWBzu.GW.tWxtsa}
\gll  tɕe [<shuijing> kɯ tɤ-kɤ-sɯ-βzu] ɣɯ tɯ-xtsa nɯra jo-ɣɯt. \\
\textsc{lnk} crystal \textsc{erg} \textsc{aor}-\textsc{obj}:\textsc{pcp}-\textsc{caus}-make \textsc{gen} \textsc{indef}.\textsc{poss}-shoe \textsc{dem}:\textsc{pl} \textsc{ifr}-bring \\
\glt `[The bird] brought shoes made of crystal.' (140504 huiguniang-zh)
(\japhdoi{0003909\#S155})
\end{exe}

\subsubsection{Other relative clauses}  \label{sec:object.participle.other.relative}
Just like subject participles can relativize the possessor of subjects §\ref{sec:subject.participle.other.relative}), object participles can be used to relativize possessors of objects, as in (\ref{ex:ndZimYaR.mWtAkArAt}), where the head of the relative \forme{ndʑi-mɲaʁ mɯ-tɤ-kɤ-rɤt} is not the object `their eyes' (which would result in a non-sensical sentence `their eye which had not been drawn were still on the wall') but rather the possessors (the dragons).

\begin{exe}
\ex \label{ex:ndZimYaR.mWtAkArAt}
\gll  [ndʑi-mɲaʁ mɯ-tɤ-kɤ-rɤt] nɯni tɕetu, znde ɯ-taʁ nɯtɕu nɯnɯ pjɤ-nɯ-tu-ndʑi. \\
\textsc{3du}.\textsc{poss}-eye \textsc{neg}-\textsc{aor}-\textsc{obj}:\textsc{pcp}-draw \textsc{dem}:\textsc{du} \textsc{up} wall \textsc{3sg}.\textsc{poss}-on \textsc{dem}:\textsc{loc} \textsc{dem} \textsc{ifr}.\textsc{ipfv}-\textsc{auto}-exist-\textsc{du} \\
\glt `The two [dragons] whose eyes had not been drawn were still on the wall.' (160718 hualongdianjing-zh)
(\japhdoi{0006153\#S59})
\end{exe}

In light verb constructions with \japhug{lɤt}{release} (§\ref{sec:light.verb}), the oblique argument encoded with the relator noun \japhug{ɯ-taʁ}{on, above} (§\ref{sec:WtaR}) can be relativized with the object participle. For instance, in (\ref{ex:tWsNaR.tAkAlAt}) and (\ref{ex:WtaR.tWsNaR.tolAt}), although the noun \japhug{tɯsŋaʁ}{enchantment} is the object of the verb \japhug{lɤt}{release} in the collocation meaning `cast a spell', the participial relative \forme{tɯsŋaʁ tɤ-kɤ-lɤt} means here  `(prince) who has been enchanted', not `the spell that has been cast'.\footnote{The second interpretation is however possible, and these relatives are ambiguous. } The head of this relative is therefore not the object, but the recipient, although this oblique argument is marked with \japhug{ɯ-taʁ}{on, above}, as shown by example (\ref{ex:WtaR.tWsNaR.tolAt}).

\begin{exe}
\ex \label{ex:tWsNaR.tAkAlAt}
\gll qaɕpa nɯnɯ, iɕqʰa nɯ, rɟɤlpu ci ɣɯ ɯ-tɕɯ nɯnɯ, nɤkinɯ, tɯ-sŋaʁ tɤ-kɤ-lɤt tɕe qaɕpa nɯ-kɤ-sɤβzu pjɤ-ŋu. \\
frog \textsc{dem} \textsc{filler} \textsc{dem} king \textsc{indef} \textsc{gen} \textsc{3sg}.\textsc{poss}-son \textsc{dem} \textsc{filler} \textsc{nmlz}:\textsc{action}-enchant \textsc{aor}-\textsc{obj}:\textsc{pcp}-release \textsc{lnk} frog \textsc{aor}-\textsc{obj}:\textsc{pcp}-transform \textsc{ifr}.\textsc{ipfv}-be \\
\glt `This frog was the son of a king who had been enchanted and transformed into a frog.' (140429 qingwa wangzi-zh,  180-181) 	(\japhdoi{0003890\#S176})
\end{exe}

\begin{exe}
\ex \label{ex:WtaR.tWsNaR.tolAt}
\gll  a-tɕɯ nɯ ɯ-taʁ tɯ-sŋaʁ to-lɤt tɕe nɯ mbalɤ-pɯ ci ɲɤ-sɤβzu. \\
\textsc{1sg}.\textsc{poss}-son \textsc{dem} \textsc{3sg}.\textsc{poss}-on \textsc{nmlz}:\textsc{action}-enchant \textsc{ifr}-release \textsc{lnk} \textsc{dem} bull-\textsc{dim} \textsc{indef} \textsc{ifr}-transform \\
\glt `She cast a spell on my son and turned him into a calf.' (140512 fushang he yaomo-zh)
(\japhdoi{0003967\#S98})
\end{exe}
 
In addition, there are cases where an object participle can relativize a locative adjunct (§\ref{sec:locative.relativization.object}). The object participle of the perception verbs \japhug{mto}{see} and \japhug{mtsʰɤm}{hear} can be used to make headless locative relative clauses meaning `(a place) where $X$ can see/hear $Y$, in particular when occurring as the goal of a motion verb as in (\ref{ex:tCirna.mAkAmtshAm}). Note the optionality of the ergative on the nouns \japhug{tɕi-rna}{our ears} and \japhug{tɕi-mɲaʁ}{our eyes} in (\ref{ex:tCirna.mAkAmtshAm}). 

\begin{exe}
\ex \label{ex:tCirna.mAkAmtshAm}
\gll [tɕi-rna mɤ-kɤ-mtsʰɤm], [tɕi-mɲaʁ mɤ-kɤ-mto] a-jɤ-ɕe-ndʑi ra \\
\textsc{1du}.\textsc{poss}-ear \textsc{neg}-\textsc{obj}:\textsc{pcp}-hear \textsc{1du}.\textsc{poss}-eye \textsc{neg}-\textsc{obj}:\textsc{pcp}-see \textsc{irr}-\textsc{pfv}-go-\textsc{du} be.needed:\textsc{fact} \\
\glt  `May they go away [to a place] where our ears cannot hear them, where our eyes cannot see them.' (2003-kWBRa)
\end{exe}

It is not possible in (\ref{ex:tCirna.mAkAmtshAm}) to replace the object participle by an oblique participle \forme{sɤ-}.

\subsubsection{Purposive clauses} \label{sec:object.participles.complement}
While \forme{kɤ-} prefixed non-finite verb forms are very common in complement clauses, the near-totality of these forms are infinitives rather than object participles (§\ref{sec:inf.complementation}), since there are no restrictions on intransitive verbs (§\ref{sec:velar.inf}).

The only complementation strategy where an object participle, rather than an infinitive, has to be posited occurs in the purposive clause of  motion verbs when the verb of the purposive clause is transitive and coreference occurs between its object (rather than subject) and the subject of the matrix motion verb, as \forme{kɤ-nɤkʰu} in (\ref{ex:kAnAkhu.juGi}).

\begin{exe}
\ex \label{ex:kAnAkhu.juGi}
\gll <xingqi> raŋri ʑo tɕe nɯnɯ sɤβʑɯ ɣɯ ɯ-kʰa nɯtɕu kɤ-nɤkʰu ju-ɣi pjɤ-ŋu  \\
week each \textsc{emph} \textsc{lnk} \textsc{dem} mouse \textsc{gen} \textsc{3sg}.\textsc{poss}-house \textsc{dem}:\textsc{loc} \textsc{obj}:\textsc{pcp}-invite \textsc{ipfv}-come \textsc{ipfv}.\textsc{ifr}-be \\
\glt `He would come to the mouse's house as a guest.' (150818 muzhi guniang-zh).
(\japhdoi{0006334\#S291})
\end{exe}

 A possessive prefix coreferent with the transitive subject of \forme{nɤkʰu} can be optionally added on this object participle, as in (\ref{ex:akAnAkhu}).

\begin{exe}
\ex \label{ex:akAnAkhu}
\gll a-kɤ-nɤkʰu jɤ-ɣe  \\
 \textsc{1sg}.\textsc{poss}-\textsc{obj}:\textsc{pcp}-invite \textsc{aor}-come[II] \\
\glt `He came to my house as a guest (following my invitation).' (elicited)
\end{exe}

In purposive clauses, the rule is thus that the subject participle is used when there is subject-subject coreference (§\ref{sec:subject.participle.complementation}), and the object participle in cases of object-subject coreference  (\citealt[248]{jacques16complementation}). 
\subsubsection{Lexicalized object participles} \label{sec:lexicalized.object.participle}
While some object participles are commonly used as headless relative clauses, few can be considered to be fully lexicalized. 

The verbs related to food ingestion such as \japhug{ndza}{eat}, \japhug{tsʰi}{drink}, \japhug{ndzɤtsʰi}{eat and drink}, \japhug{moʁ}{eat powdery food} have object participles such as \japhug{kɤ-ndza}{food}, \japhug{kɤ-tsʰi}{drink (n), beverage}, \japhug{kɤ-ndzɤtsʰi}{food and drink} and \japhug{kɤmoʁ}{dry tsampa}, which commonly occur in enumerations (§\ref{sec:noun.enumeration}) with nouns not derived from verbs, as in (\ref{ex:WkAndza.WkAtshi}).  

\begin{exe}
\ex \label{ex:WkAndza.WkAtshi}
\gll  ɯ-kɤ-ndza ɯ-kɤ-tsʰi ɯ-tɯkrimgo ra to-ɣɯt qʰe, tɕendɤre, nɯra ɲɤ́-wɣ-mbi qʰe, \\
\textsc{3sg}.\textsc{poss}-\textsc{obj}:\textsc{pcp}-eat \textsc{3sg}.\textsc{poss}-\textsc{obj}:\textsc{pcp}-drink \textsc{3sg}.\textsc{poss}-butter.bread \textsc{pl} \textsc{ifr}:\textsc{up}-bring \textsc{lnk} \textsc{lnk} \textsc{dem}:\textsc{pl} \textsc{ifr}-\textsc{inv}-give \textsc{lnk} \\
 \glt  `She brought food, drinks and butter bread for her and gave them to her.' (2003-kWBRa)
\end{exe}

In these enumerations, sometimes only the first element takes a possessive prefix, as in (\ref{ex:ndZikAndza.kAtshi}), where we find \forme{ndʑi-kɤ-ndza kɤ-tsʰi} instead of the equally possible \forme{ndʑi-kɤ-ndza ndʑi-kɤ-tsʰi} (however, if the first \forme{kɤ-} participle in the enumeration has no possessive prefix, the following participle cannot take one).

\begin{exe}
\ex \label{ex:ndZikAndza.kAtshi}
\gll  ndʑi-kɤ-ndza kɤ-tsʰi mɤ-mbrɤt, ndʑi-kɯ-ndzɤtsʰi a-pɯ-me smɯlɤm \\
\textsc{2du}.\textsc{poss}-\textsc{obj}:\textsc{pcp}-eat \textsc{obj}:\textsc{pcp}-drink \textsc{neg}-\textsc{acaus}:cut  \textsc{2du}.\textsc{poss}-\textsc{sbj}:\textsc{pcp}-eat.and.drink \textsc{irr}-\textsc{ipfv}-not.exist prayer \\
\glt `May you never lack food or drink, may there nobody [coming to] eat you.' (2003kAndZWslama)
\end{exe}

In these examples, the possessive prefix always refer to the person or animal ingesting the food (not the person giving the food), and although these forms are very common, since their semantics is completely predictable from the base verb, and since the possessive prefix behaves like that of a normal oblique participle, there is no specific reason to consider that they have become nouns and constitute lexical entries that must be distinguished from the verb (except in the case of \japhug{kɤmoʁ}{dry tsampa}, whose meaning has become more specific).

The forms \forme{kɤ-pa} and \forme{kɤ-stu}, derived from the verbs from the verbs \japhug{pa}{do} and \japhug{stu}{do like}, both meaning `manner, method (to solve a problem)' (like Chinese \ch{办法}{bànfǎ}{method}), are other potential candidates to be analyzed as lexicalized object participles (or infinitives).  They are particularly commonly used with existential verbs to mean `$X$ has (no/a) way to do it' ($X$ being referred to by the possessive prefix on \forme{kɤ-pa} or \forme{kɤ-stu}), as in (\ref{ex:akApa.maNe}).

\begin{exe}
\ex \label{ex:akApa.maNe}
\gll a-kɤpa maŋe \\
\textsc{1sg}.\textsc{poss}-method not.exist:\textsc{sens} \\
\glt `I have no way of doing it.' (many attestations)
\end{exe}
 
However, collocation in texts of \forme{kɤ-pa} and \forme{kɤ-stu} with the finite forms of the verbs \japhug{pa}{do} and \japhug{stu}{do like}, as in (\ref{ex:nAkApa.tWtu}) and, suggest that these forms are still synchronically linked with these verbs, and that it may be more economical to analyze them as participles rather than derived nouns.
 
 \begin{exe}
\ex \label{ex:nAkApa.tWtu}
\gll  nɤ-kɤ-pa tɯ\redp{}tu nɤ,  tɤ-pe ma mtsʰoʁlaŋ tɤ-ɣe  \\
\textsc{2sg}.\textsc{poss}-\textsc{obj}:\textsc{pcp}-do \textsc{cond}\redp{}exist:\textsc{fact} \textsc{lnk} \textsc{imp}-do[III] \textsc{lnk} water.monster \textsc{aor}:\textsc{up}-come[II] \\
\glt `If you have some way [to protect us], use it, because the water monster has come.' (2012 Norbzang) 	(\japhdoi{0003768\#S26})
 \end{exe}
 
  \begin{exe}
\ex \label{ex:nAkAstu.WGAZu}
\gll   nɯnɯ ɯ-taʁ nɯtɕu nɤ-kɤ-stu ɯ-ɣɤʑu tɕe a-tɤ-tɯ-ste ma tɕe \\
\textsc{dem} \textsc{3sg}.\textsc{poss}-on \textsc{dem}:\textsc{loc} \textsc{2sg}.\textsc{poss}-\textsc{obj}:\textsc{pcp}-do.like \textsc{qu}-exist:\textsc{sens} \textsc{lnk} \textsc{irr}-\textsc{pfv}-2-do.like[III] \textsc{lnk} \textsc{lnk} \\
\glt `If you have a way to deal with him, use it.'   (25-kAmYW-XpAltCin)
(\japhdoi{0003642\#S35})
 \end{exe}
 
The object participle \forme{kɤ-ti} from the verb \japhug{ti}{say}, although transparently derived, has an unpredictable meaning in the existential construction. With a negative existential verb, in addition to the expected meaning `have nothing to say', it can be interpreted as `be unable to say for sure', as in (\ref{ex:akAti.maNe}).

  \begin{exe}
\ex \label{ex:akAti.maNe}
\gll    a-kɤ-ti ci maŋe \\
 \textsc{1sg}.\textsc{poss}-\textsc{obj}:\textsc{pcp}-say \textsc{indef} not.exist:\textsc{sens} \\
 \glt `I cannot say for sure.' (many examples)
  \end{exe}
  
 %kɤpupu
 
In addition, there are highly lexicalized object participles occurring as members of compounds; these cases are generally ambiguous, and alternatively analyzable as  lexicalized velar infinitives (§\ref{sec:lexicalized.velar.inf}). The incorporating verb \japhug{kɤtɯpa}{tell} is an interesting case: it combines the form \forme{kɤ-ti} (either the participle `what one says' or the infinitive `to say') in bound state \forme{kɤtɯ-} (the alternative form \forme{kɤtipa} is also attested) with the auxiliary \japhug{pa}{do} (§\ref{sec:pa.lv}).
 
\subsection{Oblique participles} \label{sec:oblique.participle}
The \forme{sɤ}-prefix (and its allomorphs \forme{sɤɣ\trt}, \forme{sɤz-} and \forme{z}-) is used for non-core argument nominalization, in particular recipients of indirective verbs (§\ref{sec:gen.beneficiary}, §\ref{sec:dative}), instruments (§\ref{sec:instr.kW}), place and time adjuncts, as in (\ref{ex:come}). It takes a possessive prefix which can be coreferent with any core argument (subject or object).

\begin{exe}
\ex \label{ex:come}
\gll ɯ-sɤ-ɣi \\
 \textsc{3sg}.\textsc{poss}-\textsc{obl}:\textsc{pcp}-come \\
\glt  `The place/moment from where/when he/it comes.' (elicited)
\end{exe}

Related forms include the gerund (§\ref{sec:gerund}) and the purposive converb (§\ref{sec:purposive.converb}); the historical relationship between these categories is discussed in §\ref{sec:sigmatic.nmlz.history}.
 
 
 \subsubsection{Allomorphy} \label{sec:oblique.participle.allomorphy}
The base form of the oblique participle is \forme{sɤ\trt}, but three additional allomorphs are also found: \forme{sɤz\trt}, \forme{z-} and \forme{sɤɣ-}.

The allomorph \forme{sɤɣ-} or \forme{sɤx-} (depending on the voicing of the next consonant) is attested with intransitive (or labile) monosyllabic verbs with an onset without velar/uvular consonant, and without consonant cluster involving a preinitial (§\ref{sec:caus.sWG}). This allomorph is to some extent lexicalized, and is not found with all verbs fulfilling these criteria. \tabref{tab:sAG.participle} presents some of the most common examples of \forme{sɤɣ-} participles in Kamnyu Japhug.

\begin{table}
\caption{Examples of oblique participles in \forme{sɤɣ-}} \label{tab:sAG.participle}
\begin{tabular}{llllll}
\lsptoprule
Base verb & Oblique participle \\
\midrule
\japhug{pʰɤn}{be efficient} & \forme{ɯ-sɤx-pʰɤn}  `advantage' \\
\japhug{me}{not exist} & \forme{ɯ-sɤɣ-me}  `place where there is no X' \\
\japhug{ɬoʁ}{come out} & \forme{ɯ-sɤɣ-ɬoʁ} `place where $X$ grows, \\
\japhug{lɤɣ}{graze} & \forme{ɯ-sɤɣ-lɤɣ} `pasture' \\
& place from which $X$ comes out' \\
\japhug{ndzoʁ}{be attached} & \forme{ɯ-sɤɣ-ndzoʁ} `place where $X$ is attached' \\
\japhug{zo}{land (of bird)} & \forme{ɯ-sɤɣ-zo} `place where $X$ lands' \\
\japhug{ɕe}{go} & \forme{ɯ-sɤx-ɕe} `direction, place where $X$ goes' \\
\lspbottomrule
\end{tabular}
\end{table} 

Some of the verbs taking the \forme{sɤɣ-} allomorph do also occur with \forme{sɤ-}. For instance, \japhug{me}{not exist} is attested with both \forme{ɯ-sɤɣ-me} as in (\ref{ex:WsAGme}) and \forme{ɯ-sɤ-me} in (\ref{ex:WsAme}). However, most verbs in \tabref{tab:sAG.participle} are only compatible with the \forme{sɤɣ-} allomorph.

\begin{exe}
\ex \label{ex:WsAGme}
\gll ʑmbɯlɯm ɯ-sɤɣ-ɬoʁ nɯra tu-ɬoʁ ŋu. ʑmbɯlɯm ɯ-sɤɣ-me ra kɯnɤ tu-ɬoʁ ɕti. \\
species.of.mushroom \textsc{3sg}.\textsc{poss}-\textsc{obl}:\textsc{pcp}-come.out \textsc{dem}:\textsc{pl} \textsc{ipfv}-come.out be:\textsc{fact} species.of.mushroom \textsc{3sg}.\textsc{poss}-\textsc{obl}:\textsc{pcp}-not.exist \ \textsc{pl} also \textsc{ipfv}-come.out be.\textsc{aff}:\textsc{fact} \\
\glt `It grows in the places where the \textit{youlaku} mushroom grows, and also in the places where there are no \textit{youlaku}.'  (22-BlamajmAG)
(\japhdoi{0003584\#S55})
\end{exe}

 \begin{exe}
\ex \label{ex:WsAme}
\gll tɤjmɤɣ ɯ-sɤ-tu ɯ-sɤ-me ɣɤʑu. \\
mushroom \textsc{3sg}.\textsc{poss}-\textsc{obl}:\textsc{pcp}-exist  \textsc{3sg}.\textsc{poss}-\textsc{obl}:\textsc{pcp}-not.exist  exist:\textsc{sens} \\
\glt `There are places where there are mushrooms, and other places where there aren't.' (20-grWBgrWB) (\japhdoi{0003554\#S45})
\end{exe}

The allomorphs \forme{sɤz-} and \forme{z-} occur in the same context, with non-monosyllabic verb stems, where the first syllable (either a productive or a frozen prefix) is sonorant-initial. These two allomorphs are completely interchangeable, without restriction on particular verbs or the function of the the relativized element (instrument, locative or temporal adjunct). For instance, the locative participle of \japhug{rɤʑi}{stay} is attested as both \forme{ɯ-sɤz-rɤʑi} and \japhug{ɯ-z-rɤʑi} `the place when he/it stays' in the corpus, as shown by examples (\ref{ex:WsAzrAZi}) and (\ref{ex:WzrAZi}), a few sentences away from each other in the same story.

\begin{exe}
\ex \label{ex:WsAzrAZi}
\gll  tɕeri nɯnɯ sɤtɕʰa nɯ li iɕqʰa qapribɯxsi ɯ-sɤz-rɤʑi pjɤ-ɕti. \\
but \textsc{dem} place \textsc{dem} again the.aforementioned python \textsc{3sg}.\textsc{poss}-\textsc{obl}:\textsc{pcp}-stay \textsc{ifr}.\textsc{ipfv}-be.\textsc{aff} \\
\glt `But that place was the abode of a python.' (140511 xinbada-zh)
(\japhdoi{0003961\#S89})
\end{exe}

\begin{exe}
\ex \label{ex:WzrAZi}
\gll tɕe <xinbaba> rcanɯ, maka nɯtɕu ɯ-z-rɤʑi ɯ-tɯ-sɤɣ-mu pjɤ-sɤre ʑo tɕe,\\
\textsc{lnk} Sinbad \textsc{unexp}:\textsc{deg} at.all \textsc{dem}:\textsc{loc} \textsc{3sg}.\textsc{poss}-\textsc{obl}:\textsc{pcp}-stay \textsc{3sg}.\textsc{poss}-\textsc{nmlz}:\textsc{deg}-\textsc{prop}-be.afraid \textsc{ifr}.\textsc{ipfv}-be.ridiculous \textsc{emph} \textsc{lnk}\\
\glt `Sinbad, the place where he stayed was extremely terrifying.' (140511 xinbada-zh)
(\japhdoi{0003961\#S95})
\end{exe}

The \forme{sɤ-} allomorph, rather than \forme{sɤz-} or \forme{z\trt}, is however found when preceding the vertitive (§\ref{sec:vertitive}) and autive (§\ref{sec:autobenefactive} ) prefixes, as in (\ref{ex:sAnWlhoR.kWme}).

\begin{exe}
\ex \label{ex:sAnWlhoR.kWme}
\gll  qʰe tú-wɣ-cɯ mɤ-kɯ-kʰɯ, sɤ-nɯ-ɬoʁ ri kɯ-me ta-βzu. \\
\textsc{lnk} \textsc{ipfv}-\textsc{inv}-open \textsc{neg}-\textsc{sbj}:\textsc{pcp}-be.possible \textsc{obl}:\textsc{pcp}-\textsc{auto}-come.out also \textsc{sbj}:\textsc{pcp}-not.exist \textsc{aor}:3\flobv{}-make \\
\glt `(He put tape on the drawers so that) they could not be opened, and there was no way to come out of them (to prevent the rats inside from escaping).' (150831 BZW kAnArRaR)
(\japhdoi{0006378\#S18})
\end{exe}

The allomorph \forme{sɤɣ-} is also (though more rarely) attested with the autive \forme{nɯ-} of verbs that take \forme{sɤɣ-} in their simplex form. For instance, next to \forme{sɤ-nɯ-ɬoʁ}, the oblique participle \forme{sɤɣ-nɯ-ɬoʁ} is found in (\ref{ex:WsAGnWlhoR}) (without autive prefix the oblique participle is \forme{ɯ-sɤɣ-ɬoʁ}, see \tabref{tab:sAG.participle}).

\begin{exe}
\ex \label{ex:WsAGnWlhoR}
\gll ɯ-sɤɣ-nɯ-ɬoʁ ɣɯ ɯ-kɯ-spoʁ pjɤ-ŋu.  \\
\textsc{3sg}.\textsc{poss}-\textsc{obl}:\textsc{pcp}-\textsc{auto}-come.out \textsc{gen} \textsc{3sg}.\textsc{poss}-\textsc{sbj}:\textsc{pcp}-have.a.hole \textsc{ifr}.\textsc{ipfv}-be \\
\glt `It was the hole from which [the animal] came out [of the cave].  (140511 xinbada-zh) (\japhdoi{0003961\#S80})
\end{exe} 

The \forme{sɤz-} allomorph is not completely impossible with the autive \forme{nɯ-} prefix, but only one example, \forme{nɯ-sɤz-nɯ-ɴɢɤt} `the place where they (had) parted ways' (\ref{ex:nWsAznWNGAt}), is found in the whole corpus (and the form \forme{sɤ-nɯ-ɴɢɤt} is also attested, see example \ref{ex:YWsAnWNGAt} in §\ref{sec:locative.participle.relatives}).

\begin{exe}
\ex \label{ex:nWsAznWNGAt}
\gll  nɯ-sɤz-nɯ-ɴɢɤt ɣɯ iɕqʰa, tʂɤsɤɴɢɤt nɯtɕu jɤ-azɣɯt-nɯ tɕe, \\
\textsc{3pl}.\textsc{poss}-\textsc{obl}:\textsc{pcp}-\textsc{auto}-\textsc{acaus}:separate \textsc{gen} \textsc{filler}  crossroads \textsc{dem}:\textsc{loc} \textsc{aor}-reach-\textsc{pl} \textsc{lnk} \\
\glt `They arrived at the crossroads where they had parted ways.' (140508 benling gaoqiang de si xiongdi)
(\japhdoi{0003935\#S106})
\end{exe} 

\subsubsection{Transitivity} \label{sec:oblique.participle.transitivity}
Like subject and object participles, oblique participle keep the verb transitivity, and transitive verbs can take an overt object as \forme{qaj ɯ-sɤ-ji} `place for planting wheat'\footnote{Note that participle \forme{ɯ-sɤ-ji} can have other interpretations, including `the period when it is planted', as in (\ref{ex:WsAji}) below. } in (\ref{ex:qaj.WsAji}).

\begin{exe}
\ex \label{ex:qaj.WsAji}
\gll  qajsta nɯnɯ kɯɕɯŋgɯ qaj ɯ-sɤ-ji pjɤ-pe tɕe tɕe núndʐa qajsta tu-ti-nɯ ɲɯ-ŋu \\
\textsc{topo} \textsc{dem} in.former.times wheat \textsc{3sg}.\textsc{poss}-\textsc{obl}:\textsc{pcp}-plant \textsc{ifr}.\textsc{ipfv}-be.good \textsc{lnk} \textsc{lnk} for.this.reason  \textsc{topo} \textsc{ipfv}-say-\textsc{pl} \textsc{sens}-be \\
\glt `Qaysta, in former times it was a wheat field which was good, and for this reason, it is called `Qaysta' `the place of the wheat'.' (140522 kAmYW tWji2) (\japhdoi{0004057\#S101})
\end{exe}

An antipassive form (§\ref{sec:antipassive}) is necessary if there is no definite object. For instance in (\ref{ex:sAZrAji}), \forme{sɤz-rɤ-ji} `place for planting, field' is based on the \forme{rɤ-} antipassive of \japhug{ji}{plant}; this form, unlike \forme{ɯ-sɤ-ji} in (\ref{ex:qaj.WsAji}), is used without (and cannot occur with) any noun specifying the crop planted in the field.

\begin{exe}
\ex \label{ex:sAZrAji}
\gll  tɕe tɯmgri sɤz-rɤ-ji nɯ koŋla ɲɤ-ɣɤ-me-nɯ ma kʰa ʁɟa ʑo to-βzu-nɯ. \\
\textsc{lnk}  \textsc{topo} \textsc{obl}:\textsc{pcp}-\textsc{apass}-plant \textsc{dem} completely \textsc{ifr}-\textsc{caus}-not.exist-\textsc{pl} \textsc{lnk} house completely \textsc{emph} \textsc{ifr}-make-\textsc{pl} \\
\glt `They removed all the fields in Temgri, and built houses there [instead].' (140522 kAmYW tWji2) 	(\japhdoi{0004057\#S17})
\end{exe}

 \subsubsection{Possessive prefixes} \label{sec:oblique.participle.possessive}
Possessive prefixes on oblique participles are optional, though their presence is preferred in careful speech.

With intransitive verbs, the possessive prefix refers to the subject, as \forme{a-sɤz-nɤri} `the place where I stay' in (\ref{ex:asAzrAZi.ri.GAZu}).

\begin{exe}
\ex \label{ex:asAzrAZi.ri.GAZu}
\gll a-kɤ-ndza ri ɣɤʑu, a-sɤz-rɤʑi ri ɣɤʑu qʰe \\
\textsc{1sg}.\textsc{poss}-\textsc{obj}:\textsc{pcp}-eat also exist:\textsc{sens} \textsc{1sg}.\textsc{poss}-\textsc{nmlz}:oblique-stay also exist:\textsc{sens} \textsc{lnk} \\
\glt `(There), I have food to eat and a place to stay.' (150831 renshen wawa) (\japhdoi{0006418\#S22})
\end{exe}

With transitive verbs, the possessive prefix can be coreferent with the object. For instance, in (\ref{ex:nWsthamtCAt.GW.nWsAtsxWB}), the plural \forme{nɯ-} on \forme{nɯ-sɤ-tʂɯβ} refers to the many types of clothes and shoes mentioned just before in the same text.

 \begin{exe}
\ex \label{ex:nWsthamtCAt.GW.nWsAtsxWB}
\gll nɯstʰamtɕɤt ɣɯ nɯ-sɤ-tʂɯβ nɯ tɯ-ŋgru tu-sɯ-βzu-nɯ.   \\
so.many \textsc{gen} \textsc{3pl}.\textsc{poss}-\textsc{nmlz}:oblique-sew \textsc{dem} \textsc{indef}.\textsc{poss}-sinew \textsc{ipfv}-\textsc{caus}-make-\textsc{pl} \\
\glt `People use sinew to sew that many [types of clothes and shoes].' (150906 tWNgru) (\japhdoi{0006304\#S17})
\end{exe} 

However, it is also possible for the possessive prefix to be coreferent with the subject. This is particularly common when the subject is first or second person, and no overt object is present, as \forme{nɤ-sɤ-ta} `the place where you put it' in (\ref{ex:nAsAta.me}).
 
 \begin{exe}
\ex \label{ex:nAsAta.me}
\gll   kɯɕte nɯtɕu nɤ-sɤ-ta me ɯ́-ŋu \\
other \textsc{dem}:\textsc{loc} \textsc{2sg}.\textsc{poss}-\textsc{obl}:\textsc{pcp}-put not.exist:\textsc{fact} \textsc{qu}-be:\textsc{fact} \\
\glt `Isn't there any other place where you put [the food]?' (meimei de gushi)
 \end{exe} 

There is no person hierarchy in slot accessibility to the possessive prefix however; in (\ref{ex:nWsAntChoz}), the possessive prefix on \forme{nɯ-sɤ-ntɕʰoz} marks the subject, although the object is first person plural. 

 \begin{exe}
\ex \label{ex:nWsAntChoz}
\gll  tɕe iʑora ɣɯ nɯ-sɤ-ntɕʰoz a-pɯ-tu tɕe ɲɯ-tʂɯn \\
\textsc{lnk} \textsc{1pl} \textsc{gen} \textsc{3pl}.\textsc{poss}-\textsc{obl}:\textsc{pcp}-use \textsc{irr}-\textsc{ipfv}-exist \textsc{lnk} \textsc{sens}-be.grateful \\
\glt `We are glad that [some] of us have an opportunity to be useful to them.' (conversation)
 \end{exe}
   
When the object is overt (§\ref{sec:oblique.participle.transitivity}), it is rare to put a first or second person possessive prefix coreferent with the subject on the participle. Rather, a possessive prefix occurs on the object, as in (\ref{ex:jikhWtsa.sArku}) and (\ref{ex:ambrAz.sArku}), as if \forme{kʰɯtsa sɤ-rku} `place where one puts the bowls' and \forme{mbrɤz sɤ-rku} `rice container' were compounds.

\begin{exe}
\ex \label{ex:jikhWtsa.sArku}
\gll    tɕe tɕe ji-kʰɯtsa sɤ-rku ɣɯ ɯ-ŋgɯ nɯ, <chouchou> ɯ-ŋgɯ pɯ-nnɯ-ŋu, nɯ kɯmaʁ nɯra pɯ-nnɯ-ŋu kɯnɤ maka, laχtɕʰa kɤ-rku me, kɤ-ndza kɤ-rku me, nɯra tu-ndze nɤ tu-ndze, \\
\textsc{lnk} \textsc{lnk} \textsc{1pl}.\textsc{poss}-bowl \textsc{obl}:\textsc{pcp}-put.in \textsc{gen} \textsc{3sg}.\textsc{poss}-inside \textsc{dem} drawer \textsc{3sg}.\textsc{poss}-inside \textsc{pst}.\textsc{ipfv}-\textsc{auto}-be \textsc{dem} other \textsc{dem}:\textsc{pl} \textsc{pst}.\textsc{ipfv}-\textsc{auto}-be  also at.all thing \textsc{obj}:\textsc{pcp}-put.in whether \textsc{obj}:\textsc{pcp}-eat \textsc{obj}:\textsc{pcp}-put.in whether \textsc{dem} \textsc{ipfv}-eat[III] \textsc{lnk}  \textsc{ipfv}-eat[III] \\
\glt `Whether it was in the cupboard in which we put [our] bowls, or in the drawers or elsewhere, whether it was food or objects in there, [the mice] ate/gnawed them again and again.' (150831 BZW kAnArRaR)
(\japhdoi{0006378\#S7})
\end{exe} 

\begin{exe}
\ex \label{ex:ambrAz.sArku}
\gll  nɯ sɤznɤ a-mbrɤz sɤ-rku a-pɯ-ŋu ɲɯ-ra \\
\textsc{dem} \textsc{comp} \textsc{1sg}.\textsc{poss}-rice \textsc{obl}:\textsc{pcp}-put.in \textsc{irr}-\textsc{ipfv}-be \textsc{sens}-be.needed \\
\glt  `Why don't I use [this basin] as a rice container?' (150831 jubaopen-zh)
(\japhdoi{0006294\#S23})
\end{exe} 

Using the possessive on the verb is never preferred, but appears to be grammatical in elicitation in negative existential constructions, thus next to (\ref{ex:jikhWtsa.sAta}),  (\ref{ex:khWtsa.jisAta}) is also possible.
\begin{exe}
\ex
\begin{xlist}
\ex \label{ex:jikhWtsa.sAta}
\gll iʑo ji-kʰɯtsa sɤ-ta me. \\
\textsc{1pl} \textsc{1pl}.\textsc{poss}-bowl  \textsc{obl}:\textsc{pcp}-put not.exist:\textsc{fact} \\ 
\ex \label{ex:khWtsa.jisAta}
\gll iʑo kʰɯtsa ji-sɤ-ta me  \\
\textsc{1pl} bowl \textsc{1pl}.\textsc{poss}-\textsc{obl}:\textsc{pcp}-put not.exist:\textsc{fact} \\ 
\glt `We  don't have any place to put the bowls.' (elicited)
\end{xlist}
\end{exe} 


If the object is an inalienably possessed noun, it can be alienabilized (§\ref{sec:alienabilization}). For instance, the \textsc{1pl} possessive form of \forme{tɯ-ŋga sɤ-χtɕi} `washing machine' can be either (\ref{ex:jitWNga.sAXtCi}) with alienabilization or (\ref{ex:jiNga.sAXtCi}) without it.

\begin{exe}
\ex
\begin{xlist}
\ex \label{ex:jitWNga.sAXtCi}
\gll ji-tɯ-ŋga sɤ-χtɕi \\
\textsc{1pl}.\textsc{poss}-\textsc{indef}.\textsc{poss}-clothes \textsc{obl}:\textsc{pcp}-wash \\
\ex \label{ex:jiNga.sAXtCi}
\gll ji-ŋga sɤ-χtɕi \\
\textsc{1pl}.\textsc{poss}-clothes \textsc{obl}:\textsc{pcp}-wash \\
\end{xlist}
\glt `Our washing machine' (\ref{ex:jitWNga.sAXtCi} heard in context, \ref{ex:jiNga.sAXtCi} elicited)
\end{exe}

\subsubsection{Polarity and orientation preverbs} \label{sec:oblique.participle.orientation}
Unlike subject and object participles, the only prefixes (other than possessive prefixes) that oblique participles can take are the polarity prefixes and series B orientation preverbs.

It is thus not possible to have perfective or past imperfective oblique participles, and alternative strategies are used to express the corresponding meanings. For instance, from the verb \japhug{sqa}{cook}, the form $\dagger$\forme{ɯ-pɯ-sɤ-sqa} (intended meaning: `the thing that has been used to cook') is incorrect, and the solution to circumvent this morphological constraint is to combine the plain oblique participle \forme{ɯ-sɤ-sqa} with \forme{pɯ-kɯ-ŋu} (the past imperfective subject participle of \japhug{ŋu}{be}) and with the phrase \forme{nɯ ɕɯŋgɯ} `before that', as in (\ref{ex:WsAsqa.pWkWNu}).

\begin{exe}
\ex \label{ex:WsAsqa.pWkWNu}
\gll  nɯ ɕɯŋgɯ ɯ-sɤ-sqa pɯ-kɯ-ŋu ɯ-ŋgɯ (tu-rku-nɯ) \\
\textsc{dem} before \textsc{3sg}.\textsc{poss}-\textsc{obl}:\textsc{pcp}-cook \textsc{pst}.\textsc{ipfv}-\textsc{sbj}:\textsc{pcp}-be \textsc{3sg}.\textsc{poss}-inside \textsc{ipfv}-put.in-\textsc{pl} \\
\glt `[They put it] in the [pan] that had been used to cook [the barley grains] previously.' (31-cha) (\japhdoi{0003764\#S58})
\end{exe}

Negative forms of the oblique participle are not very common, but examples are found in the corpus (as in \ref{ex:WmAsApe}) and there is no difficulty to elicit them.

\begin{exe}
\ex \label{ex:WmAsApe}
\gll  qaʑmbri nɯ, nɤkinɯ, ɯ-sɤ-pe ra me, ɯ-mɤ-sɤ-pe ra me, \\
vine \textsc{dem} \textsc{filler} \textsc{3sg}.\textsc{poss}-\textsc{obl}:\textsc{pcp}-be.good \textsc{pl} not.exist:\textsc{fact}  \textsc{3sg}.\textsc{poss}-\textsc{neg}-\textsc{obl}:\textsc{pcp}-be.good \textsc{pl} not.exist:\textsc{fact} \\
\glt `The vine is neither a boon nor a harm (to the plants on which it grows).' (06-qaZmbri) 	(\japhdoi{0003416\#S17})
\end{exe}


\subsubsection{Locative relative clauses} \label{sec:locative.participle.relatives}
The oblique participle can be used to build many different types of relative clauses, with various non-core arguments and adjuncts as relativized elements, including locative, temporal, instrumental adjuncts and dative arguments. The most common ones are the locative relative clauses.

With  motion verbs like \japhug{ɕe}{go} or verbs of manipulation, the relativized element can be either the goal (the endpoint of the motion, as in \ref{ex:ndZisAxCe}), or the path through which the motion event takes place: in (\ref{ex:tsxu.kusAxCe}) for instance, the head \forme{tʂu} is a locative adjunct (`the road through which one goes to $X$') different from the goal (the placename \forme{prɤɕta}).

\begin{exe}
\ex \label{ex:ndZisAxCe}
\gll ndʑi-sɤx-ɕe nɯtɕu jo-zɣɯt tɕe \\
\textsc{3du}.\textsc{poss}-\textsc{obl}:\textsc{pcp}-go \textsc{dem}:\textsc{loc} \textsc{ifr}-reach \textsc{lnk} \\
\glt  `[The ox] arrived at the place where the two of them were going.' (150826 shier shengxiao-zh) (\japhdoi{0006284\#S85})
\end{exe}

\begin{exe}
\ex \label{ex:tsxu.kusAxCe}
\gll   [prɤɕta tʂu ku-sɤx-ɕe] nɯre ri tɯ-ji ci tu tɕe, nɯ cɤŋgɤɣ rmi. \\
\textsc{topo} path \textsc{ipfv}:east-\textsc{obl}:\textsc{pcp}-go \textsc{dem}:\textsc{loc} \textsc{loc} \textsc{indef}.\textsc{poss}-field \textsc{indef} exist:\textsc{fact} \textsc{lnk} \textsc{dem}  \textsc{topo} be.called:\textsc{fact} \\
\glt  `On the road towards Prashta there is a field, it is called Kyangag.' (140522 kAmYW tWji) 	(\japhdoi{0004055\#S110})
\end{exe}

In (\ref{ex:WsAznACWCe}), the oblique participle designates the areas in which the subject (a plant) grows, without a specific goal.

\begin{exe}
\ex \label{ex:WsAznACWCe}
\gll  pɤjka wuma ʑo a-pɯ-pe, tɕe ɯ-sɤx-ɕe nɯra a-pɯ-dɤn ɯ-sɤz-nɤɕɯɕe a-pɯ-dɤn tɕe, pɤjka tɯ-pʰɯ ɯ-taʁ nɯtɕu kɯβdɤsqi jamar, ɯ-mat ku-tsʰoʁ ɲɯ-cʰa \\
pumpkin really \textsc{emph} \textsc{irr}-\textsc{ipfv}-be.good \textsc{lnk} \textsc{3sg}.\textsc{poss}-\textsc{obl}:\textsc{pcp}-go \textsc{dem}:\textsc{pl} \textsc{irr}-\textsc{ipfv}-be.many \textsc{3sg}.\textsc{poss}-\textsc{obl}:\textsc{pcp}-go.around \textsc{irr}-\textsc{ipfv}-be.many  \textsc{lnk} pumpkin \textsc{one}-tree \textsc{3sg}.\textsc{poss}-on \textsc{dem}:\textsc{loc} fourty about \textsc{3sg}.\textsc{poss}-fruit \textsc{ipfv}-attach \textsc{sens}-can \\
\glt `When the pumpkin [grows] well, and when there are a lot of [places] for it to spread, one plant can yield about fourty pumpkin.' (16-CWrNgo)
(\japhdoi{0003518\#S95})
\end{exe}

The relativized locative adjunct can also be the place of origin rather than the goal in the case of the verb \japhug{ɣi}{come} as in (\ref{ex:jisAGi}).

\begin{exe}
\ex \label{ex:jisAGi}
\gll  iʑora nɯ ji-sɤ-ɣi nɯtɕu pjɤ-ŋu tɕe. \\
\textsc{1pl} \textsc{dem} \textsc{1pl}.\textsc{poss}-\textsc{obl}:\textsc{pcp}-come \textsc{dem}:\textsc{loc} \textsc{ifr}.\textsc{ipfv}-be \textsc{lnk} \\
\glt  `The place from where we come was there.' (2010-06)
\end{exe}

With stative verbs or dynamic verbs implying no motion, the oblique participle has a static locative meaning, as in  (\ref{ex:tWsAmdzW}).
 

\begin{exe}
\ex \label{ex:tWsAmdzW}
\gll tɤ-tɕɯ tɕʰeme tɯ-sɤ-ɤmdzɯ ʑaka tu. \\
\textsc{indef}.\textsc{poss}-son girl \textsc{indef}.\textsc{poss}-\textsc{obl}:\textsc{pcp}-sit each exist:\textsc{fact} \\ 
\glt `Gentlemen and ladies each have [different] seating places.' (31-khAjmu)
(\japhdoi{0004079\#S10})
\end{exe}


Participial locative relative clauses are used to describe non-transient properties of places: directions, locations or places of origin that are unchanging characteristics of things or persons (\ref{ex:tsxu.kusAxCe} and \ref{ex:jisAGi}), places where some state of affair generally occurs due to a natural law (\ref{ex:WsAznACWCe} and \ref{ex:WsAdAn}) or places used for a specific purpose, as in (\ref{ex:tWsAmdzW}), and even more clearly in (\ref{ex:WsAta.Wrkoz}) with the property noun   \japhug{ɯ-rkoz}{special} (§\ref{sec:z.nmlz}).

\begin{exe}
\ex \label{ex:WsAta.Wrkoz}
\gll tɕe saŋdi nɯ tɕe, nɯnɯ si ɯ-sɤ-ta ɯ-rkoz ʑo pjɤ-ŋu. \\
  \textsc{lnk} lower.side.of.the.hearth \textsc{dem} \textsc{lnk} \textsc{dem} firewood \textsc{3sg}.\textsc{poss}-\textsc{obl}:\textsc{pcp}-put \textsc{3sg}.\textsc{poss}-special \textsc{emph} \textsc{ifr}.\textsc{ipfv}-be \\
\glt `The lower side of the hearth was specifically where [people] put firewood.' (2011-11)
\end{exe}


For transient properties of places, finite relatives are used instead (§\ref{sec:locative.relativization.finite}). In example (\ref{ex:nWNga.sAta}) translated from Chinese,\footnote{The original text has \ch{……走到仙女们放衣服的地方}{zǒu dào xiānnǚmen fàng yīfú de dìfāng}{...went to the place where the sky goddesses put/had put their clothes}; both interpretations are possible. }  Tshendzin hesitates between a participial relative (implying that there was a specific place where the sky goddesses put their clothes each time they came to earth) and a finite relative (suggesting that they put their clothes in some unspecific place, perhaps in a casual way as the autive \forme{-nɯ-} could indicate, §\ref{sec:autobenefactive}).

\begin{exe}
\ex \label{ex:nWNga.sAta}
\gll tɕʰemɤpɯ nɯra, [nɯ-ŋga sɤ-ta] nɯtɕu ko-ɕe, [nɯ-ŋga na-nɯ-ta-nɯ] nɯtɕu ko-ɕe matɕi, \\
girl \textsc{dem}:\textsc{pl} \textsc{3pl}.\textsc{poss}-clothes \textsc{obl}:\textsc{pcp}-put \textsc{dem}:\textsc{loc} \textsc{ifr}:\textsc{east}-go \textsc{3pl}.\textsc{poss}-clothes \textsc{aor}:3\flobv{}-\textsc{auto}-put-\textsc{pl} \textsc{dem}:\textsc{loc} \textsc{ifr}:\textsc{east}-go \textsc{lnk} \\
\glt `He went to the place where the girls put their clothes, where they had put their clothes.' (150828 niulang-zh) (\japhdoi{0006318\#S58})
\end{exe}

Locative participial relative clauses with overt head can be prenominal, in particular with genitival relatives as in (\ref{ex:sAnWCe.GW.Wtsxu}). 

\begin{exe}
\ex \label{ex:sAnWCe.GW.Wtsxu}
\gll  pjɤ-nɯkɯlu-nɯ ma sɤ-nɯ-ɕe ɣɯ ɯ-tʂu nɯ mɯ-ɲɤ-nɯ-mto-nɯ. \\
\textsc{ifr}-be.lost-\textsc{pl}  \textsc{lnk} \textsc{obl}:\textsc{pcp}-\textsc{vert}-go \textsc{gen} \textsc{3sg}.\textsc{poss}-path \textsc{dem} \textsc{neg}-\textsc{ifr}-\textsc{auto}-see-\textsc{pl} \\
\glt `They were lost, and could not find their way back home.' (160630 poucet1) (\japhdoi{0006065\#S52})
\end{exe}

However, head-internal relatives are also attested: in (\ref{ex:tsxu.kusAxCe}) above and (\ref{ex:YWsAnWNGAt}), the head of the participial relatives, the noun \japhug{tʂu}{road}, occurs between the verb in oblique participle form and a place name marking the goal. Note in addition that (\ref{ex:YWsAnWNGAt}) illustrates two locative oblique participial relative clauses embedded within another participial relative.\footnote{In (\ref{ex:YWsAnWNGAt}), the orientation preverbs reflect the basic meanings of the tridimensional system (§\ref{sec:tridimensional.preverb}). 
}

\begin{exe}
\ex \label{ex:YWsAnWNGAt}
\gll   [[rpɤŋgɯ tʂu lu-sɤx-ɕe] cʰo [prɤscʰɯ tʂu lu-sɤ-ɣi] ɲɯ-sɤ-nɯ-ɴɢɤt] nɯtɕu, \\
\textsc{topo}  path \textsc{ipfv}:\textsc{upstream}-\textsc{obl}:\textsc{pcp}-go \textsc{comit}  \textsc{topo}  path \textsc{ipfv}:\textsc{upstream}-\textsc{obl}:\textsc{pcp}-come \textsc{ipfv}:\textsc{west}-\textsc{obl}:\textsc{pcp}-\textsc{auto}-\textsc{acaus}:separate \textsc{dem}:\textsc{loc} \\
\glt `At the place where the road towards Rpangu and the road towards Praskhyu separate.' (140522 kAmYW tWji2)
(\japhdoi{0004057\#S122})
\end{exe}

\subsubsection{Instrumental relative clauses} \label{sec:instrumental.participle.relatives}
Another very productive type of oblique participial relatives are the instrumental relative clauses (§\ref{sec:instrument.relativization}). Although instruments, like transitive subjects, receive ergative case (§\ref{sec:instr.kW}), they are usually relativized with oblique participles.

There is often ambiguity between instrument relativization and locative adjunct relativization; for instance, while the participle \forme{ɯ-z-rɤ-rɤt} can mean `pen (the tool used to write)' as in (\ref{ex:WzArAt}), this form can also designate the paper on which one writes or even one's office.

\begin{exe}
\ex \label{ex:WzArAt}
\gll ɯ-slamaχti nɯ ɣɯ, [ɯ-z-rɤ-rɤt] ci to-nɯ-ndo tɕe jo-nɯ-tsɯm ɲɯ-ŋu tɕe, \\
\textsc{3sg}.\textsc{poss}-classmate \textsc{dem} \textsc{gen} \textsc{3sg}.\textsc{poss}-\textsc{obl}:\textsc{pcp}-\textsc{apass}-write \textsc{indef} \textsc{ifr}-\textsc{auto}-take \textsc{lnk} \textsc{ifr}-\textsc{vert}-take.away \textsc{sens}-be \textsc{lnk} \\
\glt `He took away the pen of a classmate.' (2014-tou dongxi de xiaohai-zh)
\end{exe}

Instrumental relative clauses built with oblique participles can occur as objects of the causativized verb \japhug{sɯ-βzu}{cause to make; use X to make}, as in (\ref{ex:tWthW.sAXtCi.tWNga.sApCiz}).

\begin{exe}
\ex \label{ex:tWthW.sAXtCi.tWNga.sApCiz}
\gll nɯnɯ [tɯtʰɯ sɤ-χtɕi], [tɯ-ŋga sɤ-pɕiz] nɯra tu-sɯ-βzu-nɯ pɯ-ŋgrɤl. \\
\textsc{dem} pan \textsc{obl}:\textsc{pcp}-wash \textsc{indef}.\textsc{poss}-clothes \textsc{obl}:\textsc{pcp}-wipe \textsc{dem}:\textsc{pl} \textsc{ipfv}-\textsc{caus}-make-\textsc{pl} \textsc{pst}.\textsc{ipfv}-be.usually.the.case \\
\glt `People used to employ [\textit{Usnea}] as tools to wash pans or wipe clothes.' (20-sWrna) (\japhdoi{0003564\#S141})
\end{exe} 

In this construction, the material used to make the tool is marked with the ergative (§\ref{sec:instr.kW}), as in (\ref{ex:WsACmi}).

\begin{exe}
\ex \label{ex:WsACmi}
\gll ɯnɯnɯ kɯ [ɯ-sɤ-ɕmi] tu-sɯ-βzu-nɯ pɯ-ŋgrɤl \\
\textsc{dem} \textsc{erg} \textsc{3sg}.\textsc{poss}-\textsc{obl}:\textsc{pcp}-mix \textsc{ipfv}-\textsc{caus}-make-\textsc{pl} \textsc{pst}.\textsc{ipfv}-be.usually.the.case \\
\glt `People used to employ it (a boat oar) to mix [the alcohol].' (31-cha) 	(\japhdoi{0003764\#S42})
\end{exe}  

Oblique participles are used to make instrumental relative clauses, used like nouns of instruments, from both transitive and intransitive verbs. If the base verb is transitive, the participle retains its transitivity: thus in (\ref{ex:WsACmi}), the absence of object in the one-word relative clause \forme{ɯ-sɤ-ɕmi} `the tool used to mix it' is the result of zero-anaphora, and implies a definite object. Instrumental relative clauses with a transitive verb more often have an overt object, as in (\ref{ex:tWthW.sAXtCi.tWNga.sApCiz}). For indefinite objects, antipassivization is necessary, as in \forme{ɯ-z-rɤ-rɤt} `the tool used to write' in (\ref{ex:WzArAt}) above (see also  §\ref{sec:oblique.participle.transitivity}). 

The relativized instrument need not be an entity, but can also be an action. For instance, in the pseudo-cleft (\ref{ex:WsAznWrmAZu}), what is referred to by the participial relative is the infinitival clause \forme{mɲɤm kɤ-pʰaʁ}.

\begin{exe}
\ex \label{ex:WsAznWrmAZu}
\gll qajdo [ɯ-rʑaβ ɯ-ɕki ɯ-sɤz-nɯrmɤʑu] nɯ mɲɤm kɤ-pʰaʁ pjɤ-ŋu  \\
crow \textsc{3sg}.\textsc{poss}-wife \textsc{3sg}.\textsc{poss}-\textsc{dat} \textsc{3sg}.\textsc{poss}-\textsc{obl}:\textsc{pcp}-show.off \textsc{dem} type.of.tree \textsc{inf}-chop \textsc{ifr}.\textsc{ipfv}-be \\
\glt  `What he was showing off with to his wife was chopping the wood of the \forme{mɲɤm} tree (which is easy to cut).' (11-mYAm)
(\japhdoi{0003474\#S22})
\end{exe}

\subsubsection{Other oblique relative clauses} \label{sec:other.oblique.participle.relatives}
In addition to goals, locative adjuncts and instruments, oblique participles are used to relativize various other types of arguments and adjuncts, though those cases are considerably less common in the corpus.

Dative arguments (in \forme{ɯ-ɕki} or \forme{ɯ-pʰe}, §\ref{sec:dative}) are relativized with an oblique participle (§\ref{sec:dative.relativization}), as is shown by (\ref{ex:WsAfCAt}),  where the verb \japhug{fɕɤt}{tell} also occurs as the main verb of the second clause with an overt recipient marked with the dative.

\begin{exe}
\ex \label{ex:WsAfCAt}
\gll [ɯ-sɤ-fɕɤt] pjɤ-me qʰe tɕe tɤ-pɤtso ɯ-ɕki nɯ tɕu nɯra tɕʰi pɯ-kɯ-fse nɯra pjɤ-fɕɤt. \\
\textsc{3sg}.\textsc{poss}-\textsc{obl}:\textsc{pcp}-tell \textsc{ipfv}.\textsc{ifr}-not.exist \textsc{lnk} \textsc{lnk} \textsc{indef}.\textsc{poss}-child \textsc{3sg}-\textsc{dat} \textsc{dem} \textsc{loc} \textsc{dem}:\textsc{pl} what \textsc{pst}-\textsc{sbj}:\textsc{pcp}-be.like  \textsc{dem}:\textsc{pl} \textsc{ifr}-tell \\
\glt `She had no one [else] to tell, so she told the boy everything that had happened.' (140515 congming de wusui xiaohai-zh) 	(\japhdoi{0003998\#S74})
\end{exe} 

Likewise, comitative phrases in \forme{cʰo} (§\ref{sec:comitative}) and occurring with verbs with intrinsically non-singular subjects (§\ref{sec:intrinsically.n.sg.subject}), are relativized with an oblique participle. For instance, the participle \forme{ɯ-sɤ-ɤmɯmi} in (\ref{ex:WsAmWmi}) is a headless relative meaning `those with whom it is on good terms with'.

\begin{exe}
   \ex \label{ex:WsAmWmi}
 \gll  tɕe ɯʑo [ɯ-sɤ-ɤmɯmi] nɯ dɤn ma ca kɯ-fse qaʑo kɯ-fse, tsʰɤt kɯ-fse,  ɯʑo cʰo kɯ-naχtɕɯɣ sɯjno, xɕaj ma mɤ-kɯ-ndza nɯ ra cʰo nɯ amɯmi-nɯ tɕe, \\
\textsc{lnk} it \textsc{3sg}.\textsc{poss}-\textsc{obl}:\textsc{pcp}-be.on.good.terms \textsc{dem} be.many:\textsc{fact} because musk.deer \textsc{obl}:\textsc{pcp}-be.like sheep \textsc{obl}:\textsc{pcp}-be.like goat  \textsc{obl}:\textsc{pcp}-be.like it with  \textsc{obl}:\textsc{pcp}-be.identical herbs grass apart.from \textsc{neg}-\textsc{obl}:\textsc{pcp}-eat \textsc{dem} \textsc{pl} with \textsc{dem} be.in.good.term:\textsc{fact}-\textsc{pl} \textsc{lnk} \\
\glt `The [animals] that are on good terms with the rabbit are many, it is on good terms with those that only eat grass, like musk deer, sheep or goats.' (04 qala1)
(\japhdoi{0003392\#S30})
\end{exe}

Time adjuncts are also possibly relativized using oblique participles, as \forme{ɯ-sɤ-ji}, which means `the period when it is planted' in (\ref{ex:WsAji}). However, finite relative clauses are the preferred way of relativizing time adjuncts (§\ref{sec:time.relativization}). As in the case of locative relative clauses (§\ref{sec:locative.participle.relatives}), participial relative clauses are only used to refer to specific dates and time periods that are intrinsic properties of the event.

\begin{exe}
   \ex \label{ex:WsAji}
   \gll    tɕe nɯnɯ ʑaka [ɯ-sɤ-ji] ɲɯ-ŋu tɕe \\
   \textsc{lnk} \textsc{dem} each \textsc{3sg}.\textsc{poss}-\textsc{obl}:\textsc{pcp}-plant \textsc{sens}-be \textsc{lnk}\\
\glt `These are the [periods] when people plant each of these [crops].' (15 tChWma)
(\japhdoi{0003514\#S18})
\end{exe}

Temporal relative clauses are generally headless, but (\ref{ex:WsAsna}) shows an example of head-internal (or postnominal) relative clause, with \japhug{skɤrma}{minute, date} as its head noun. In addition, this participial relative has here a superlative interpretation (§\ref{sec:possessed.superlative}).

\begin{exe}
   \ex \label{ex:WsAsna}
   \gll    lɤsɤr χsɯm ɯ-raŋ tɕe, tɯxpalɤskɤr ɣɯ, nɤkinɯ, [skɤrma ɯ-sɤ-sna] ŋu tu-kɯ-ti ŋu. \\
   new.year three \textsc{3sg}.\textsc{poss}-time \textsc{loc} whole.year \textsc{gen} \textsc{filler} date \textsc{3sg}.\textsc{poss}-\textsc{obl}:\textsc{pcp}-be.good be:\textsc{fact} \textsc{ipfv}-\textsc{genr}-say be:\textsc{fact} \\
\glt `We say that the third day of the year is the [most] auspicious day in the whole year.' (2010-10)
\end{exe}

A further derived meaning of the oblique participle is that of `opportunity to do X', as in (\ref{ex:sAznWNgra.GAZu}) and (\ref{ex:nWsAntChoz}) above.

\begin{exe}
\ex \label{ex:sAznWNgra.GAZu}
\gll tɕe [sɤz-nɯŋgra] ɣɤʑu ri, li sɤɣʑɯr \\
\textsc{lnk} \textsc{obl}:\textsc{pcp}-earn.wages exist:\textsc{sens} \textsc{lnk} again be.dangerous:\textsc{fact} \\
\glt `Although it [provides] an opportunity to earn wages, it is also dangerous.' (conversation 140510)
\end{exe}


Even in context, the exact meaning of a particular oblique relative clause may allow some leeway in interpretation. For instance, the participle \forme{ji-sɤx-ɕe} in the negative existential construction in (\ref{ex:jisAxCe.maNe}) could be understood as a locative relative clause `(we had no) place to go' but also alternatively as `(we had no) opportunity to go (anywhere)' in this particular context.

\begin{exe}
\ex \label{ex:jisAxCe.maNe}
\gll  iʑora tɕe kʰa ɯ-ŋgɯ kɤ-kɤ-ja ʑo ɲɯ-fse-j ku-rɤʑit-i ma [ji-sɤx-ɕe] maŋe \\
\textsc{1pl} \textsc{lnk} house \textsc{3sg}.\textsc{poss}-inside \textsc{aor}-\textsc{obj}:\textsc{pcp}-close \textsc{emph} \textsc{sens}-be.like-\textsc{1pl} \textsc{ipfv}-stay-\textsc{1pl} \textsc{lnk} \textsc{1pl}.\textsc{poss}-\textsc{obl}:\textsc{pcp}-go not.exist:\textsc{sens} \\
\glt  `It was like we were locked in the house, with nowhere to go.' (140501 tshering skyid)
(\japhdoi{0003902\#S115})
\end{exe}

\subsubsection{Causative} \label{sec:oblique.participle.causative}
The oblique participles can occur as object of the verb \japhug{βzu}{make}, with a purposive (\ref{ex:WmAsAxCe.tuBzunW}) or causative (\ref{ex:WsARjWRjit}) interpretation. These meanings derive from the instrumental nominalizing uses of the oblique participle. 

\begin{exe}
\ex \label{ex:WmAsAxCe.tuBzunW}
\gll tɯ-ji ɯ-rkɯ ra tɤ-ɣur pjɯ-ta-nɯ tɕe, fsapaʁ ɯ-mɤ-sɤx-ɕe tu-βzu-nɯ ŋgrɤl \\
\textsc{indef}.\textsc{poss}-field \textsc{3sg}.\textsc{poss}-side \textsc{pl} \textsc{indef}.\textsc{poss}-fence \textsc{ipfv}-put-\textsc{pl} \textsc{lnk} animals \textsc{3sg}.\textsc{poss}-\textsc{neg}-\textsc{obl}:\textsc{pcp}-go \textsc{ipfv}-make-\textsc{pl} be.usually.the.case:\textsc{fact} \\
\glt `They put a fence around the fields, so as to prevent farm animals from going there.' (140427 qamtsWrmdzu) 	(\japhdoi{0003854\#S13})
\end{exe}

In (\ref{ex:WmAsAxCe.tuBzunW}), the phrase \forme{fsapaʁ ɯ-mɤ-sɤx-ɕe} can be interpreted as `(something made so) that animals do not go (there)', while in (\ref{ex:WsARjWRjit}) \forme{rgargɯn nɯnɯ ɯ-sɤ-ʁjɯ\redp{}ʁjit} means `(something used to) remind the old man'. 

\begin{exe}
\ex \label{ex:WsARjWRjit}
\gll tɕʰeme kɯ-ŋɤn nɯ kɯ iɕqʰa nɯ, rgargɯn nɯnɯ ɯ-sɤ-ʁjɯ\redp{}ʁjit to-βzu tɕe, \\
woman \textsc{sbj}:\textsc{pcp}-be.evil \textsc{dem} \textsc{erg} \textsc{filler} \textsc{dem} old.man \textsc{dem} \textsc{3sg}.\textsc{poss}-\textsc{obl}:\textsc{pcp}-\textsc{emph}\redp{}remember \textsc{ifr}-make \textsc{lnk} \\
\glt `The evil woman reminded the old man.' (140515 jiesu de laoren-zh)
(\japhdoi{0004004\#S118})
\end{exe}

%tu-kɤ-sɯ-ʁjit ftɕaka to-βzu

\subsubsection{Ambiguity} \label{sec:oblique.participle.ambiguity}
The various allomorphs of the oblique participle do resemble other prefixes found in Japhug. The \forme{sɤ-} and \forme{sɤɣ-} allomorphs are also found with the proprietive derivation (§\ref{sec:proprietive}), and \forme{sɤ-} is also similar to the human antipassive, or the sigmatic causative of \forme{a-} initial verbs (§\ref{sec:contraction}). Finally, the \forme{z-} allomorph of the oblique participle can resemble the causative (§\ref{sec:caus.z}) or one allomorph of the translocative prefix (§\ref{sec:translocative.morpho}).

However, unlike subject (§\ref{sec:subject.participle.ambiguities}) and object (§\ref{sec:object.participle.ambiguity}) participles, these surface ambiguities are only very superficial, as the forms with which the oblique participle could potentially be confused are all finite, and hardly ever occur in the same syntactic context as the oblique participle (and except for their bare infinitive form, in the case of transitive verb, never occur with a possessive prefix). 

In the case of the \forme{z-} allomorph of translocative prefix, note that it only occurs before a few orientation preverbs (\forme{ɲɯ\trt}, \forme{ɲɤ\trt}, \forme{ju\trt}, \forme{jɤ\trt}, \forme{jo\trt}, \forme{ja-}), whereas the oblique participle \forme{z-} can only follow an orientation preverb, as in \forme{ɯ-cʰɯ-z-raʁrɯz} in (\ref{ex:WchWzraRWz}), so that the two forms can never be confused.

\begin{exe}
\ex \label{ex:WchWzraRWz}
\gll  nɯnɯ ɣɯ ɯ-cʰɯ-z-raʁrɯz nɯ ɯ-sɤ-pɕiz rmi \\ 
\textsc{dem} \textsc{gen} \textsc{3sg}.\textsc{poss}-\textsc{ipfv}:\textsc{downstream}-\textsc{obl}:\textsc{pcp}-sweep \textsc{dem}  \textsc{3sg}.\textsc{poss}-\textsc{obl}:\textsc{pcp}-wipe be.called:\textsc{fact} \\
\glt `[The tool used to] sweep [the flour] is called a ``wiper''.' (06-BGa) 	(\japhdoi{0003408\#S149})
\end{exe}

\subsubsection{Lexicalized oblique participles} \label{sec:lexicalized.oblique.participle}
Nouns of instruments and of location, including placenames, are often made from oblique participles. 

The noun \japhug{sɤcɯ}{key}, although transparently originating from the instrumental use of the oblique participle of \japhug{cɯ}{open}, is lexicalized as shown by the fact that it cannot take orientation preverbs, and that it occurs in collocation with the auxiliary \forme{lɤt} to mean `lock (the door)' as in (\ref{ex:sAcW.malAt2}).

\begin{exe}
\ex \label{ex:sAcW.malAt2}
\gll   ɯ-ŋgɯ lɤ-ɣi jɤɣ ma sɤcɯ mɤ-a-lɤt \\
\textsc{3sg}.poss-inside \textsc{imp}:\textsc{upstream}-come be.allowed:\textsc{fact} \textsc{lnk} key \textsc{neg}-\textsc{pass}-throw \\
\glt `Come in, the door is not locked.' (140428  xiaohongmao-zh) (\japhdoi{0003884\#S73})
\end{exe}

Place names built from oblique participle include \forme{Znɤrɣɤma}, from the locative participle \forme{z-nɤrɣɤma} of the verb \japhug{nɤrɣɤma}{pray for rain} (probably a denominal verb from a compound *\forme{rɣɤma} based on \japhug{tɯ-rɣi}{seed} and \japhug{ta-ma}{work}, §\ref{sec:denom.intr.nW}), as it was the place where people used to perform this activity in Kamnyu in the traditional society, as explained in §\ref{sec:antipassive.avoidance}.
 
Another example is the uninhabited place called \forme{kɯlɤɣsɤmdzɯ}, a transparent combination  \japhug{kɯ-lɤɣ}{shepherd} (§\ref{sec:lexicalized.subject.participle}) and \japhug{ɯ-sɤ-ɤmdzɯ}{sitting place} reflecting the use of this place (as described in \ref{ex:kWlAG.sAmdzW}).

\begin{exe}
\ex \label{ex:kWlAG.sAmdzW}
\gll    kɯ-xtɕɯ\redp{}xtɕi ci ʑo antɤm, tɕe nɯ kɯ-lɤɣ ra nɯtɕu ku-rɤʑi-nɯ pjɤ-ŋgrɤl \\
\textsc{inf}:\textsc{stat}-\textsc{emph}\redp{}be.small a.little \textsc{emph} be.flat:\textsc{fact}  \textsc{lnk} \textsc{dem} \textsc{sbj}:\textsc{pcp}-graze \textsc{pl} \textsc{dem}:\textsc{loc} \textsc{ipfv}-stay-\textsc{pl} \textsc{ifr}.\textsc{ipfv}-be.usually.the.case \\
\glt `[The place called \forme{kɯlɤɣsɤmdzɯ}] is a bit flat, and shepherds used to stay there.' (140522 Kamnyu zgo) (\japhdoi{0004059\#S269})
 \end{exe}
 
There are also case of nouns of instruments in \forme{sɤ-} whose base verb is not identifiable. For instance, the noun \japhug{sɤɕtɕɯɣ}{strap to carry children on the back}, which is glossed using an oblique participle as in (\ref{ex:tApAtso.WsAzbWwa}), is most certainly a frozen oblique participle, but there is no verb \forme{*ɕtɕɯɣ} in Japhug.

\begin{exe}
\ex \label{ex:tApAtso.WsAzbWwa}
\gll   tɤ-pɤtso ɯ-sɤz-bɯwa \\
\textsc{indef}.\textsc{poss}-child \textsc{3sg}.\textsc{poss}-\textsc{obl}:\textsc{pcp}-carry.on.the.back \\
\glt `Something used to carry children on the back' (definition given for the noun \forme{sɤɕtɕɯɣ})
\end{exe}

Some nouns originating from lexicalized participles have an irregular \forme{s-} allomorph (\tabref{tab:spa.sta.stu}). Their antiquity is shown by the existence of exact cognates in Tangut (\citealt[49;299]{jacques14esquisse}) and Khroskyabs (\citealt[514; 580]{lai17khroskyabs}).  All three nouns are used as relators in relative clauses (§\ref{sec:Wspa.relative}, §\ref{sec:Wstu.relativization.subject}), and \japhug{ɯ-spa}{material} additionally occurs to build a type of purposive clauses (§\ref{sec:purposive.clauses}).


\begin{table}
\caption{Examples of fossilized oblique participles in \forme{s-} } \label{tab:spa.sta.stu}
\begin{tabular}{lllll}
\lsptoprule
Noun & Base verb & Tangut & Khroskyabs \\
\midrule
\japhug{ɯ-spa}{material} & \japhug{pa}{do} & &\forme{=spi} \\
\japhug{ɯ-stu}{place} & \japhug{tu}{exist} &\tangut{𗯩}{5165}{twụ}{1.58} &\\
\japhug{ɯ-sta}{place} & \japhug{ta}{put} & \tangut{𘎪}{5645}{tjị}{2.60}&  \\
\lspbottomrule
\end{tabular}
\end{table}

In addition from the general meaning of `place', the nominal stem \forme{-sta} has three highly specific meanings: \forme{tɯ-sta} `bed', \forme{tɤ-sta} `designated place (for burying a dead person)' and \forme{ɯ-sta} `habit, state', a complement-taking noun (§\ref{sec:nouns.manner.complement}) always selecting a third singular possessor, used in particular in a collocation meaning `return to one's previous state, recover (from a disease)' with \japhug{fse}{be like}, as in (\ref{ex:Wsta.pWnWfse}).

\begin{exe}
\ex \label{ex:Wsta.pWnWfse}
\gll  ki ɯ-ʁɤri kɯ-fse nɯ ɯ-sta ʑo pɯ-nɯ-fse nɤ \\
\textsc{dem}.\textsc{prox} \textsc{3sg}.\textsc{poss}-before \textsc{sbj}:\textsc{pcp}-be.like \textsc{dem} \textsc{3sg}.\textsc{poss}-state \textsc{emph} \textsc{pst}.\textsc{ipfv}-\textsc{auto}-be.like \textsc{add} \\
\glt  `She had become again like she was before.' (2003 Kunbzang)
 \end{exe}

More speculatively, it is possible that other allomorphs of the oblique nominalization are preserved in some nouns, even when the base is lost. For example, \japhug{ʑmbrɯ}{boat} might be the lexicalized participle of a verb \forme{*mbrɯ} `float' (cognate of \zh{浮} \textit{bjuw} $\leftarrow$ \forme{*m.b(r)u} `float', \citealt{zhangsy19cognates}) with a fronted \forme{ʑ-} allomorph (like that of the sigmatic causative, §\ref{sec:caus.Z}).

An even more pronounced type of lexicalization occurs when an oblique participle becomes member of a compound with \textit{status contructus} vowel alternation. In these cases, it is probable that the compound is genitival, rather than a reduced relative clause.

As first element of compound, we find the oblique participle \forme{ɯ-sɤ-qru} (from the verb \japhug{qru}{greet, welcome, receive}) in bound state combined with the noun \japhug{cʰa}{alcohol} into 
\japhug{sɤqrɤcʰa}{alcohol to treat the guests} (§\ref{sec:determinative.n.n}). There are also cases of undetectable bound state, as in \japhug{sɤrŋgɯŋga}{bed cover} from the oblique participle of \japhug{rŋgɯ}{lie down} and the inalienably possessed noun \japhug{tɯ-ŋga}{clothes}: since the first element of this compound \forme{sɤ-rŋgɯ} ends in \forme{-ɯ}, it would not have a different bound form.

Oblique participles are also attested as second element of compounds from both transitive and intransitive verbs. 

As an example of lexicalized participle from an intransitive verb, the compound \japhug{tʂɤsɤɴɢɤt}{crossroad} combines the participle \japhug{ɯ-sɤ-ɴɢɤt}{place where X part ways} from the anticausative verb \japhug{ɴɢɤt}{part ways, part company}) with the bound state of the noun \japhug{tʂu}{path}, from an earlier locative participial relative \forme{*tʂu ɯ-sɤ-ɴɢɤt} `the place where roads separate'. 

With a transitive verb, we find the noun \japhug{βɣɤsɤprɤt}{watermill valve}, from an earlier instrumental relative \forme{*βɣa ɯ-sɤ-prɤt} `the tool used to stop (water) in the mill' from the bound state \forme{βɣɤ-} of \japhug{βɣa}{watermill} combined with the oblique participle \forme{ɯ-sɤ-prɤt} of the verb \japhug{prɤt}{break, stop}. With undetectable bound state, we have for instance \japhug{pʰɯsɤti}{belows} from the  onomatopoeia \forme{pʰɯ} and the oblique participle of \japhug{ti}{saying}, literally `the tool used to make `pff' sound'. 

The noun \japhug{tɤresɤtɕɯtɕɤt}{joke} is an example of compound without bound state on the first element. It comes from the reduplicated oblique participle of the verb \japhug{tɕɤt}{take out}, which occurs in collocation with the inalienably possessed noun \japhug{tɤ-re}{laugh (n)} to mean `mock', as in (\ref{ex:are.matWtCAt}). The original meaning of this compound presumably was `something/someone to make fun of'.

\begin{exe}
\ex \label{ex:are.matWtCAt}
\gll a-re ma-tɯ-tɕɤt \\
\textsc{1sg}.\textsc{poss}-laugh \textsc{neg}:\textsc{imp}-2-take.out \\
\glt `Don't laugh at me!' (elicited)
\end{exe}

In the lexicalized compound \forme{tɤresɤtɕɯtɕɤt}, \forme{tɤ-re} is alienabilized (§\ref{sec:alienabilization}); this noun occurs in collocation with the verb \japhug{βzu}{make} or the causative \japhug{sɯβzu}{cause to make} in the sense of `joke'. It can specifically mean `mock', in which case the compound \forme{tɤresɤtɕɯtɕɤt} takes a possessive prefix coreferent with the \textit{subject} of \forme{βzu} (the person making the joke, not the object of mockery), as shown by the presence of the third dual possessive prefix \forme{ndʑi-} in (\ref{ex:ndZitAresAtCWtCAt}) and the second singular \forme{nɤ-} in (\ref{ex:nAtAresAtCWtCAt}).

\begin{exe}
\ex \label{ex:ndZitAresAtCWtCAt}
\gll ɯ-pi ʁnɯz ni kɯnɤ tsʰɯrɟɯn ʑo, nɤkinɯ, pjɯ́-wɣ-ɣɤ-kʰe tɕe ndʑi-tɤresɤtɕɯtɕɤt ra tu-βzu-ndʑi pjɤ-ŋu. \\
\textsc{3sg}.\textsc{poss}-elder.sibling two \textsc{du} also often \textsc{emph} \textsc{filler} \textsc{ipfv}-\textsc{inv}-\textsc{caus}-be.stupid \textsc{lnk} \textsc{3du}.\textsc{poss}-joke \textsc{pl} \textsc{ipfv}-make-\textsc{du} \textsc{ifr}.\textsc{ipfv}-be \\
\glt `His two elder brothers often called him stupid and mocked him.' (140430 jin e-zh)
(\japhdoi{0003893\#S9})
\end{exe}

\begin{exe}
\ex \label{ex:nAtAresAtCWtCAt}
\gll nɤ-tɤresɤtɕɯtɕɤt ma-tɤ-kɯ-sɯ-βzu-a  \\
\textsc{2sg}.\textsc{poss}-joke \textsc{neg}:\textsc{imp}-\textsc{imp}-2\fl{}1-\textsc{caus}-make-\textsc{1sg} \\
\glt `Don't mock me!' (elicited)
\end{exe}

In these cases, the noun occurring as first element of the compound corresponds to the object of the verb.

The expression \japhug{sɤrma}{good night} derives from the oblique participle of the verb \japhug{rma}{stay the night, live}, and presents unusual morphological properties (§\ref{sec:phatic.inflectionalization}).

 

\section{Infinitives} \label{sec:inf}

 
\subsection{Velar infinitives} \label{sec:velar.inf}
The most common infinitives in Japhug are the velar infinitives, built from the stem I of the verb and prefixed either with \forme{kɤ-} or or \forme{kɯ-}; they are homophonous with participles (and historically related to them, §\ref{sec:velar.nmlz.history}) and not always easily distinguishable from them (§\ref{sec:infinitives.participles}). They are in particular the preferred citation form of the verbs (§\ref{sec:inf.citation}), though not with all speakers.
 
The \forme{kɯ-} infinitives are found with stative verbs (including adjectives and existential verbs), impersonal modal verbs and some anticausative verbs; other verbs take the \forme{kɤ-} infinitives. In Tshobdun, \citet[235]{sun14generic} reports that the \forme{kə-} and \forme{kɐ-} infinitives (corresponding to \forme{kɯ-} and \forme{kɤ-} in Japhug) occur with non-human and human arguments, respectively (in the case of dynamic verbs). It seems that this criterion is not applicable to Japhug.

Stative verbs in \forme{a-} have regular fusion of \forme{kɯ-} and \forme{a-} as \ipa{kɤ\trt}, and thus superficially appear to have \forme{kɤ-} infinitives (for instance, the infinitive of \japhug{arŋi}{be green} is \forme{kɯ-ɤrŋi} \ipa{kɤrŋi}).

\subsubsection{Infinitives vs. participles} \label{sec:infinitives.participles}
It is not immediately obvious that a category of `velar infinitives' needs to be distinguished from participles in Japhug, as both are non-finite verbal categories prefixed in \forme{kɤ-} or \forme{kɯ-} (§\ref{sec:object.participle.ambiguity} and §\ref{sec:subject.participle.ambiguities}). 

The necessity to set \forme{kɤ-} infinitives apart from object participles stems from the fact that the latter can only be built from transitive or semi-transitive verbs, while the former also occurs with strictly intransitive verbs. Thus, if one were to argue that all \forme{kɤ-}prefixed non-finite forms of transitive verbs are object participles, including in the case of complement clauses (for instance \forme{kɤ-ndza} in \ref{ex:kAndza.mWpWrYota}), one would not be able to account for the \forme{kɤ-}prefixed forms of intransitive verbs occurring in the same context such as \forme{kɤ-ɕe} in (\ref{ex:kACe.mWpWrYota}) -- even though \japhug{ɕe}{go} could be considered to be a kind of semi-transitive verb (since it can take a goal, which can be relativized with a finite relative clause, §\ref{sec:locative.relativization.finite}), it is not possible to build a participial relative clause by prefixing \forme{kɤ-} on this verb (the oblique participle \forme{sɤ-} must be used instead, §\ref{sec:oblique.participle}).

\begin{exe}
\ex \label{ex:kAndza.mWpWrYota}
\gll aʑo kɤ-ndza mɯ-pɯ-rɲo-t-a \\
\textsc{1sg} ???-eat \textsc{neg}-\textsc{aor}-experience-\textsc{pst}:\textsc{tr}-\textsc{1sg} \\
\glt `I never ate that.' (many attestations)
\end{exe}

\begin{exe}
\ex \label{ex:kACe.mWpWrYota}
\gll  aj kɤ-ɕe mɯ-pɯ-rɲo-t-a \\
\textsc{1sg} \textsc{inf}-go \textsc{neg}-\textsc{aor}-experience-\textsc{pst}:\textsc{tr}-\textsc{1sg}  \\
\glt `I never went there.' (150820 ZNGWloR) (\japhdoi{0006292\#S3})
\end{exe}

I therefore adopt the following criteria to distinguish between a \forme{kɤ-} infinitive and an object participle:  \forme{kɤ-} non-finite forms of (non-semi-transitive) intransitive verbs are infinitives; \forme{kɤ-} non-finite forms of transitive and semi-transitive verbs occurring in the same contexts as the infinitives of intransitive verbs are infinitives.

By systematically applying these criteria, we can identify three contexts where infinitives are attested: citation form (§\ref{sec:inf.citation}), complementation (§\ref{sec:inf.complementation}; there are however a few cases of object participles used in complement clauses, §\ref{sec:object.participles.complement}), and manner converbs (§\ref{sec:inf.converb}). 

Distinguishing between subject participles and \forme{kɯ-} infinitives in Japhug is less straightforward, unlike in other Gyalrong languages such as Tshobdun for instance, \citealt{jackson14morpho}, since even stative verbs take the \forme{kɤ-} infinitive in complement clauses (§\ref{sec:inf.complementation}). The only clear contexts where \forme{kɯ-} infinitives do occur is that of citation forms (§\ref{sec:inf.citation}) and complement clauses containing impersonal modal verbs (§\ref{sec:inf.complementation}). Converbs in \forme{kɯ-} are analyzed as infinitives rather than subject participle because they are only attested with stative verbs or other verbs taking the \forme{kɯ-} infinitives (§\ref{sec:inf.converb}).

\subsubsection{Associated motion, polarity and orientation preverbs on infinitives}  \label{sec:infinitives.other.prefixes}
With the exception of the construction in §\ref{sec:inf.exist} and some converbial uses (§\ref{sec:inf.converb}), \forme{kɤ-} infinitives do not take possessive prefixes. However, like participles, they are compatible with associated motion (\ref{ex:CWkAXtW}), negative prefixes (\ref{ex:mAkACe.mAkhW}) (with double negation, §\ref{sec:double.negation}) and B-type orientation preverbs (\ref{mWpjWkAlhoR.ftCaka}), with combinations of two prefixes.

\begin{exe}
\ex \label{ex:CWkAXtW}
\gll a-mgɯr ɲɯ-mŋɤm tɕe ɕɯ-kɤ-χtɯ mɯ́j-cʰa-a \\
\textsc{1sg}.\textsc{poss}-back \textsc{sens}-hurt \textsc{lnk} \textsc{tral}-\textsc{inf}-buy \textsc{neg}:\textsc{sens}-can-\textsc{1sg} \\
\glt `My back hurts and I cannot go to  buy [apples].' (conversation, 30-04-2018)
\end{exe}
 
\begin{exe}
\ex \label{ex:mAkACe.mAkhW}
\gll   rɟɤlpu fka ɕti tɕe, mɤ-kɤ-ɕe mɤ-kʰɯ \\
king order be.\textsc{aff}:\textsc{fact} \textsc{lnk} \textsc{neg}-\textsc{inf}-go \textsc{neg}-be.possible:\textsc{fact} \\
\glt `This is the king's order, [I] have no choice but to go.' (2005 Norbzang)
(\japhdoi{0003768\#S146})
\end{exe}
 
\begin{exe}
\ex \label{mWpjWkAlhoR.ftCaka}
 \gll  tɤ-se mɯ-pjɯ-kɤ-ɬoʁ ftɕaka tu-βze-a tu-mdzoz-a pɯ-ŋu ma, \\
 \textsc{indef}.\textsc{poss}-blood \textsc{neg}-\textsc{ipfv}-\textsc{inf}-come.out manner \textsc{ipfv}-make[III]-\textsc{1sg} \textsc{ipfv}-avoid-\textsc{1sg} \textsc{pst}.\textsc{ipfv}-be \textsc{lnk} \\
\glt `I avoided by all means to let the blood come out.' (of a skin disease, of which it is said that if the blood comes out and gets into contact with other parts of the skin, it will spread) (24-pGArtsAG) (\japhdoi{0003624\#S56})
 \end{exe}
 
Negative infinitives take the allomorph \forme{mɤ-} (as in \ref{ex:mAkACe.mAkhW}), unless an imperfective orientation preverb is present, in which case the negative is \forme{mɯ-} as in (\ref{mWpjWkAlhoR.ftCaka}).

Impersonal and stative infinitives in \forme{kɯ-} are only attested with the negative prefix \forme{mɤ-}.

\subsubsection{Ambiguity}  \label{sec:velar.inf.ambiguity}
Aside from the homophony between infinitives and participles discussed in §\ref{sec:infinitives.participles}, another type of ambiguity occurs with verbs selecting the orientation preverb \textsc{eastwards}, whose A form is \forme{kɤ-} (§\ref{sec:kamnyu.preverbs}). Intransitive verbs in imperative singular and perfective third singular forms (for instance \forme{kɤ-rŋgɯ} `he laid down') and transitive verbs without stem alternation in imperative singular (\forme{kɤ-tsʰi} `drink!', §\ref{sec:imp.morphology}) have forms that are homophonous with the corresponding infinitives (\forme{kɤ-rŋgɯ} `to lie down', \forme{kɤ-tsʰi} `to drink'), but these cases are never really ambiguous, as it is trivial to distinguish between a finite verb form and a non-finite one, for instance by changing from singular to dual or plural. 

For instance, in a particular context, if a form such as \forme{kɤ-rŋgɯ} can be changed to the corresponding plural \forme{kɤ-rŋgɯ-nɯ} (which can be either perfective \textsc{aor}-lie.down-\textsc{pl} `they laid down' or imperative `lie down!'), it is possible to conclude that this \forme{kɤ-rŋgɯ} is necessarily finite (since non-finite verb forms in Japhug never take indexation affixes, §\ref{sec:indexation.finiteness.intro}) and cannot be an object participle or an infinitive.

\subsubsection{Citation form} \label{sec:inf.citation}
The infinitive is the preferred form to refer to a verb in metalinguistic discourse as a citation form. In this context stative and impersonal verbs consistently take the \forme{kɯ-} prefix as in (\ref{ex:mAkWBdi}), and the rest of verbs the \forme{kɤ-} prefix, as in (\ref{ex:kAnARarphAB}) and (\ref{ex:WkAlAjme.pjWkACthWz}). Even in citation form, the infinitive verb can take orientation (\ref{ex:WkAlAjme.pjWkACthWz}) and polarity prefixes (\ref{ex:mAkWBdi}).  

\begin{exe}
\ex  \label{ex:mAkWBdi}
 \gll ɯnɯnɯ tɕe tɕe [ɯ-tɯ-tʂɯβ mɤ-kɯ-βdi] tu-kɯ-ti ŋu \\ 
 \textsc{dem} \textsc{lnk} \textsc{lnk} \textsc{3sg}.\textsc{poss}-\textsc{nmlz}:\textsc{action}-sew \textsc{neg}-\textsc{inf}:\textsc{stat}-be.good  \textsc{ipfv}-\textsc{genr}:A-say be:\textsc{fact}  \\
\glt `People call this `badly sewn'.'  (12-kAtsxWb-zh)
(\japhdoi{0003486\#S28})
\end{exe}

\begin{exe}
\ex \label{ex:kAnARarphAB}
 \gll  pjɯ-sɯ-ʁndi tɕe pjɯ-sɯ-sat tɕe nɯ kóʁmɯz nɤ cʰɯ-nɯtsɯm ɲɯ-ra tɕe nɯnɯ [kɤ-nɤʁarpʰɤβ] tu-kɯ-ti ŋu  \\
 \textsc{ipfv}-\textsc{caus}-hit[III]  \textsc{lnk} \textsc{ipfv}-\textsc{caus}-kill \textsc{lnk} \textsc{dem} only.after \textsc{lnk} \textsc{ipfv}:\textsc{downstream}-take.away \textsc{sens}-be.needed \textsc{lnk} \textsc{dem} \textsc{inf}-strike.with.wings \textsc{ipfv}-\textsc{genr}:A-say be:\textsc{fact}  \\
 \glt `It strikes it and kills it [with its wings] and only then takes it away. This is called \japhug{kɤ-nɤʁarphɤβ}{strike with one's wings}.' (150819 RarphAB-zh)
(\japhdoi{0006356\#S9})
\end{exe}


The infinitive is commonly used in metalinguistic discussions about collocations, and in those cases can appear together with intransitive subjects (\ref{ex:mAkWBdi}) or objects (\ref{ex:WkAlAjme.pjWkACthWz}). In the latter, the focus is on the noun \japhug{ɯ-kɤlɤjme}{head upside down} (§\ref{sec:dvandva.coll}), the verb \japhug{ɕtʰɯz}{turn towards} being present only because it is selected by \forme{ɯ-kɤlɤjme}.

\begin{exe}
\ex \label{ex:WkAlAjme.pjWkACthWz}
 \gll  ɯ-mŋu nɯ pa pjɯ́-wɣ-ɕtʰɯz tɕe nɯ [ɯ-kɤlɤjme pjɯ-kɤ-ɕtʰɯz] tu-kɯ-ti ŋu. \\
\textsc{3sg}.\textsc{poss}-mouth \textsc{dem} down \textsc{ipfv}:\textsc{down}-\textsc{inv}-turn.towards \textsc{lnk} \textsc{dem}  \textsc{3sg}.\textsc{poss}-head.upside.down   \textsc{ipfv}:\textsc{down}-\textsc{inf}-turn.toward \textsc{ipfv}-\textsc{genr}:A-say be:\textsc{fact}  \\
\glt `One turns the mouth [of the container] downwards, it is called `to turn upside down'.' (30-macha)
(\japhdoi{0003746\#S64})
\end{exe}

The infinitive is not the only possible choice as a citation form; some speakers sometimes cite a generic form (especially with the imperfective, as \forme{pjɯ́-wɣ-ɕtʰɯz} in \ref{ex:WkAlAjme.pjWkACthWz}) or other finite forms (even imperatives).

Outside of metalinguistic discourse, the stative/impersonal infinitive is also used as subject of adjectival stative verbs such as \japhug{pe}{be good} as in (\ref{ex:kArAt.kWkhW}), expressing the meaning `the fact of ... is good'.

\begin{exe}
\ex \label{ex:kArAt.kWkhW}
 \gll tɕe [[kɤ-rɤt] kɯ-kʰɯ] nɯ tɕe tɕe ɲɯ-pe ma, tɕe nɯ tɤ-scoz nɯ pjɯ-kɯ-ru ɲɯ-kʰɯ. \\
 \textsc{lnk} \textsc{inf}-write \textsc{inf}:\textsc{stat}-be.possible \textsc{dem} \textsc{lnk} \textsc{lnk} \textsc{sens}-be.good \textsc{lnk} \textsc{lnk} \textsc{dem} \textsc{indef}.\textsc{poss}-writing \textsc{dem} \textsc{ipfv}-\textsc{genr}:S/O-look \textsc{sens}-be.possible \\
 \glt `It is good to have the possibility to write [a language], because then one can look at the writing (to learn that language; otherwise, one has to learn just by listening). (150901 tshuBdWnskAt)
(\japhdoi{0006242\#S44})
\end{exe}

In the verb doubling construction, the infinitive of stative verbs can be neutralized to the \forme{kɤ-} form, as in (\ref{ex:kArZi}) with an adjectival stative verb and (\ref{ex:kAtu.nW.tu}) with the existential verb \japhug{tu}{exist}.

\begin{exe}
\ex \label{ex:kArZi}
 \gll kɤ-rʑi ri pjɤ-rʑi, 	  \\
 \textsc{inf}-be.heavy also \textsc{ifr}.\textsc{ipfv}-be.heavy \\
 \glt `As for being heavy, [the old man] was heavy.'  (140511 xinbada-zh) (\japhdoi{0003961\#S179})
\end{exe}

\begin{exe}
\ex \label{ex:kAtu.nW.tu}
 \gll ɯʑo rkɯn, ri kɤ-tu nɯ tu  \\
 \textsc{3sg} be.rare:\textsc{fact} \textsc{lnk} \textsc{inf}-exist \textsc{dem} exist:\textsc{fact} \\
 \glt `It is rare, but as for existing, it does exist.' (140511 qamtsWrmdzu)
(\japhdoi{0003957\#S17})
\end{exe}

\subsubsection{Complementation}    \label{sec:inf.complementation}
The most common function of the velar infinitive is to build complement clauses (§\ref{sec:velar.infinitives.complement.clauses}). Apart from a handful of well-identified cases (§\ref{sec:subject.participle.complementation}), all \forme{kɤ-} prefixed verb forms in complement clauses are infinitive rather than object participles (using the criteria in §\ref{sec:infinitives.participles}). 

Velar infinitives occur with a great variety of auxiliaries and other complement-taking verbs (§\ref{sec:velar.infinitives.complement.clauses}) as well as a few complement-taking nouns (§\ref{sec:complement.taking.nouns}). Few verbs however require the infinite in complement clauses. Some complement-taking verbs like \japhug{cʰa}{can} occur with either infinitival complement clauses as in (\ref{ex:kAnWCe.YWCti}) or finite complement clauses (§\ref{sec:finite.complement}), and other verbs like \japhug{rɲo}{experience} are compatible with both velar infinitives and bare infinitives (§\ref{sec:dental.inf}). The constraints on co-reference between the subject of the matrix verb and the participants of the complement clause is treated in §\ref{sec:velar.inf.coreference}.

\begin{exe}
\ex \label{ex:kAnWCe.YWCti}
 \gll  ɯ-kɤχcɤl ɯ-ʁrɯ nɯ a-nɯ-pʰɯt tɕe kɤ-nɯ-ɕe cʰa ɲɯ-ɕti \\
 \textsc{3sg}.\textsc{poss}-top.of.the.head \textsc{3sg}.\textsc{poss}-horn \textsc{dem} \textsc{irr}-\textsc{pfv}-take.out \textsc{lnk} \textsc{inf}-\textsc{vert}-go can:\textsc{fact} \textsc{sens}-be.\textsc{aff} \\
\glt `If one takes out the horn on his head, he will be able to go back [to the heavens].' (divination 2005)
\end{exe}
 
 Complex velar infinitive forms with polarity, orientation and associated motion prefixes are attested in complement clauses, as in (\ref{ex:CWkACar.pWrYota}) (see also for instance \ref{ex:CWkAXtW} and \ref{ex:mAkACe.mAkhW} in §\ref{sec:infinitives.other.prefixes}).
 
\begin{exe}
\ex \label{ex:CWkACar.pWrYota}
 \gll aʑo [ɕɯ-kɤ-ɕar] pɯ-rɲo-t-a. \\
 \textsc{1sg} \textsc{tral}-\textsc{inf}-search \textsc{aor}-experience-\textsc{tr}:\textsc{pst}-\textsc{1sg} \\
 \glt `I did go to search [for \textit{Amanita caesarea}].'  (22-BlamajmAG) (\japhdoi{0003584\#S30})
\end{exe}

Stative verbs, when occurring in a complement clause, generally take the \forme{kɤ-} infinitive, as in example  (\ref{ex:rYo}) and (\ref{ex:kAscit}). The main verb of the complement clauses in these examples have the \forme{kɤ-} infinitive, even though both \japhug{tu}{exist} and \japhug{scit}{be happy} are stative verbs and have a citation form with the \forme{kɯ-} prefix.

\begin{exe}
\ex \label{ex:rYo}
\gll   [a-rŋɯl kɤ-tu] pɯ-rɲo-t-a \\
\textsc{1sg}.\textsc{poss}-money \textsc{inf}-exist \textsc{pst}:\textsc{ipfv}-experience-\textsc{pst}:\textsc{tr}-\textsc{1sg} \\
\glt `I used to have money'. (elicited)
\end{exe}

\begin{exe}
\ex \label{ex:kAscit}
 \gll  [kɤ-scit] pjɤ-ŋgrɯ ɲɯ-ŋu  \\
 \textsc{inf}-be.happy \textsc{ifr}-succeed \textsc{sens}-be \\
 \glt `She succeeded in being happy.' (150818 muzhi guniang-zh) (\japhdoi{0006334\#S6})
 \end{exe} 
 
The conversion to \forme{kɤ-} infinitive only applies to stative verbs, not to impersonal modal verbs such as \japhug{ra}{have to, need}. When the latter occur in a complement clause, as in example (\ref{ex:kAndza.kWra}), they always have the \forme{kɯ-} prefix.

\begin{exe}
\ex \label{ex:kAndza.kWra}
\gll  [[smɤn kɤ-ndza] kɯ-ra] pɯ-rɲo-t-a  \\ 
medicine \textsc{inf}-eat \textsc{inf}:\textsc{impers}-be.needed  \textsc{pst}:\textsc{ipfv}-experience-\textsc{pst}:\textsc{tr}-\textsc{1sg} \\
\glt `I used to have to take medicine.' (elicited)
\end{exe} 
 
The velar infinitive is also found in some adnominal complement clauses, for instance in the collocation comprising the nouns \japhug{ftɕaka}{manner} or \japhug{kowa}{manner} with the transitive verb \japhug{βzu}{make} (§\ref{sec:nouns.manner.complement}), as (\ref{ex:kAndza.kowa.tWwGBzu}) below (see also  \ref{mWpjWkAlhoR.ftCaka} above). 

In infinitive complement clauses, the complement verb lacks person/number indexation. However, in the case of the verbs that require coreference between the core arguments of the matrix clause and those in the complement clause, the person and number of the subject (and sometimes also the object) of the complement clauses are reflected on the indexation of the verb in the matrix clause (§\ref{sec:velar.inf.coreference}). For instance, in (\ref{ex:kAti.mWjspea}), both the infinitive \forme{kɤ-ti} `to say' and the matrix verb \forme{mɯ́j-spe-a} `I am not able' share the same \textsc{1sg} subject and \textsc{3sg} object (\forme{ɯ-mdoʁ} `its colour').  

\begin{exe}
\ex \label{ex:kAti.mWjspea}
\gll nɯ ɯ-mdoʁ nɯ aj [kɤ-ti] mɯ́j-spe-a \\
\textsc{dem} \textsc{3sq}.\textsc{poss}-colour \textsc{dem} \textsc{1sg} \textsc{inf}-say \textsc{neg}:\textsc{sens}-be.able[III]-\textsc{1sg} \\
\glt `I don't know how to say (describe) its colour.' (06-qaZmbri)
(\japhdoi{0003416\#S50})
  \end{exe} 
  
 This is also the case with noun+verb collocations taking complement clauses: in (\ref{ex:kAndza.kowa.tWwGBzu}), the verb form \forme{tɯ́-wɣ-βzu} reflects the \textsc{3pl}\fl{}\textsc{2sg} configuration of the infinitive \forme{kɤ-ndza} `to eat' in the complement clause. 
  
  \begin{exe}
\ex \label{ex:kAndza.kowa.tWwGBzu}
\gll   a-rɟit ra nɯ-ɣi-nɯ ɕti tɕetʰa, kɤ-ndza kowa tɯ́-wɣ-βzu ɕti \\
 \textsc{1sg}.\textsc{poss}-children \textsc{pl} \textsc{vert}-come:\textsc{fact}-\textsc{pl} be.\textsc{aff}:\textsc{fact} in.a.moment \textsc{inf}-eat manner 2-\textsc{inv}-make:\textsc{fact}  be.\textsc{aff}:\textsc{fact} \\
 \glt `My children are coming back home soon, and they will try to eat you.' (2012 Norbzang) 
 (\japhdoi{0003768\#S258})
 \end{exe} 
 
With velar infinitive complements, some auxiliary verbs take the orientation preverb selected by the verb in the complement clause (§\ref{sec:orientation.raising}).
 

\subsubsection{Doubly prefixed velar infinitives with negative existential verbs} \label{sec:inf.exist}
The infinitive in \forme{kɤ-} can take two prefixes in a construction combining the negative existential verb \japhug{me}{not exist} (§\ref{sec:existential.basic}) with a verb in the infinitive prefixed with a B-type orientation preverb and a possessive prefix coreferent with the subject,\footnote{The assertion in \citet[228]{jacques16complementation} that infinitives cannot take possessive prefixes is thus wrong.} meaning `have no way to $X$, be completely unable to $X$', as in (\ref{ex:ndZijukACe}) and (\ref{ex:WpjWkAnWZWB}). Note that since in both of these examples, the verbs are intransitive and lack an object participle, the \forme{kɤ-} form can only be analyzed as an infinitive here. This construction is also possible with transitive verbs, in which case the possessive prefix corresponds to the transitive subject.

\begin{exe}
\ex \label{ex:ndZijukACe}
\gll tɕe ndʑi-ju-kɤ-ɕe pjɤ-me \\
\textsc{lnk} \textsc{3du}.\textsc{poss}-\textsc{ipfv}-\textsc{inf}-go \textsc{ifr}.\textsc{ipfv}-not.exist \\
\glt `They could not go.' (150908 menglang-zh) (\japhdoi{0006320\#S41})
\end{exe} 

\begin{exe}
\ex \label{ex:WpjWkAnWZWB}
\gll tɯ-rʑaʁ nɯ ɯ-pjɯ-kɤ-nɯʑɯβ pjɤ-me matɕi, \\
one-night \textsc{dem} \textsc{3sg}.\textsc{poss}-\textsc{ipfv}-\textsc{inf}-sleep \textsc{ifr}.\textsc{ipfv}-not.exist \textsc{lnk} \\
\glt `He could not sleep the whole night, because...' (150831 BZW kAnArRaR)
(\japhdoi{0006378\#S12})
\end{exe}

A derived construction involves the causative \japhug{ɣɤme}{cause not to exist, suppress} with doubly prefixed infinitives to `make it impossible for $X$ to $Y$' as in (\ref{ex:apjWkAnWZWB.naGAme}).

\begin{exe}
\ex \label{ex:apjWkAnWZWB.naGAme}
\gll a-pjɯ-kɤ-nɯʑɯβ na-ɣɤ-me \\
\textsc{1sg}.\textsc{poss}-\textsc{ipfv}-\textsc{inf}-sleep \textsc{aor}:3\flobv{}-\textsc{caus}-not.exist \\
\glt `It kept me from sleeping.' (elicited)
\end{exe}

%cʰa a-ku-kɤ-tsʰi na-ɣɤme
There is a variant of this construction with imperfective subject participles in \forme{kɯ-} instead of infinitives (§\ref{sec:subject.participle.complementation}). 

\subsubsection{Converbial function}    \label{sec:inf.converb}
Velar infinitives in \forme{kɤ-} can also be used as converbs, in subordinate clauses that are neither relatives nor complement clauses, with a variety of meanings.

The most common function of converbial infinitives is similar to the gerund (§\ref{sec:gerund.clauses}), expressing either the manner in which an action takes place (as in \ref{ex:kANke.lAGea}, \ref{ex:tChi.kAcha.Zo.tokAndzWtci} and \ref{ex:kANke.RJa.Zo}) or describing an additional action occurring at the same time as that referred to by the main verb (as \forme{laχtɕʰa kɤ-fkur}, with an overt object, in \ref{ex:kAfkur.tutWNke}) (additional examples are presented in §\ref{sec:manner.clauses}).

\begin{exe}
\ex \label{ex:kANke.lAGea}
\gll  kɤ-ŋke lɤ-ɣe-a \\
\textsc{inf}-walk  \textsc{aor}:\textsc{upstream}-come[II]-\textsc{1sg} \\
\glt `I came on foot.' (17-lhazgron)
\end{exe}

\begin{exe}
\ex \label{ex:kAfkur.tutWNke}
\gll nɤʑo [laχtɕʰa kɤ-fkur] tu-tɯ-ŋke ɲɯ-ŋu tɕe \\
\textsc{2sg} thing \textsc{inf}-carry.on.the.back \textsc{ipfv}-2-walk \textsc{sens}-be \textsc{lnk} \\
\glt `You are walking carrying things on your back.' (150909 hua pi-zh)
(\japhdoi{0006278\#S12})
\end{exe}

Converbial infinitives also indicate the degree to which an action is undertaken, as in the common expression \forme{tɕʰi kɤ-cʰa ʑo} `do whatever $X$ can to $Y$' in (\ref{ex:tChi.kAcha.Zo.tokAndzWtci}).

\begin{exe}
\ex \label{ex:tChi.kAcha.Zo.tokAndzWtci}
\gll   [tɕʰi kɤ-cʰa] to-k-ɤndzɯt-ci ri, maka ʑo kɯ-pʰɤn pjɤ-me \\
what \textsc{inf}-can  \textsc{ifr}-\textsc{peg}-bark-\textsc{peg}  \textsc{lnk} completely \textsc{emph} \textsc{sbj}:\textsc{pcp}-be.efficient \textsc{ipfv}.\textsc{ifr}-not.exist \\
\glt `[The dog] barked as much as it could, but it was all for nothing.'  (140426 gou he qingwa-zh)
(\japhdoi{0003802\#S20})
\end{exe}

The converbs can also be followed by adverbs of quantification like \japhug{ʁɟa}{completely} (§\ref{sec:universal.quantification.adverbs}), as illustrated by (\ref{ex:kANke.RJa.Zo}).

\begin{exe}
\ex \label{ex:kANke.RJa.Zo}
\gll <kaihui> kɯ-fse tɤ-ra tɕe, kɤ-ŋke ʁɟa ʑo ju-kɯ-ɕe pɯ-ra. \\
meeting \textsc{sbj}:\textsc{pcp}-be.like \textsc{aor}-be.needed \textsc{lnk} \textsc{inf}-walk completely \textsc{emph} \textsc{ipfv}-\textsc{genr}:S/O-go \textsc{pst}.\textsc{ipfv}-be.needed \\
\glt `When one had to [take part in] a meeting for instance, one had to go on foot.' (12-BzaNsa)
(\japhdoi{0003484\#S20})
\end{exe}

Infinitives in these functions can optionally be followed by the ergative \forme{kɯ}, like the gerunds (§\ref{sec:gerund.clauses}), as shown by (\ref{ex:kArJWG.kW.Zo}). 

\begin{exe}
\ex \label{ex:kArJWG.kW.Zo}
\gll maka to-nɯmbrɤpɯ-nɯ tɕe li kɤ-rɟɯɣ kɯ ʑo jo-ɕe-nɯ. \\
completely \textsc{ifr}-ride-\textsc{pl} \textsc{lnk} again \textsc{inf}-run \textsc{erg} \textsc{emph} \textsc{ifr}-go-\textsc{pl} \\
\glt `They mounted their horses and galloped away.' (140512 alibaba-zh)
(\japhdoi{0003965\#S39})
\end{exe}

Unlike gerunds (§\ref{sec:gerund.clauses}), infinitive converbs do not necessarily imply that two different actions take place at the same time. In (\ref{ex:kANke.RJa.Zo}) and (\ref{ex:kArJWG.kW.Zo}) for instance, the converbs \forme{kɤ-ŋke} and \forme{kɤ-rɟɯɣ} do not express a motion event distinct from that described by the main verb; rather, the verb \japhug{ɕe}{go} indicates the direction and deixis of the motion, while the converbs specify the speed and manner of realization of the action. Note that in this particular example, using the gerund is not possible.

The converbial infinitives are attested with the negative prefix \forme{mɤ\trt}, meaning `without $X$ing', and generally refer to the way in which the action is performed as in (\ref{ex:mAkArWsWso.kW.Zo}). 

\begin{exe}
\ex \label{ex:mAkArWsWso.kW.Zo}
\gll   maka mɤ-kɤ-rɯsɯso kɯ ʑo `jɤɣ' to-ti, to-nɤla. \\
at.all \textsc{neg}-\textsc{inf}-think \textsc{erg} \textsc{emph} be.allowed:\textsc{fact} \textsc{ifr}-say \textsc{ifr}-agree \\
\glt `She said `yes', she agreed without thinking at all.' (140429 qingwa wangzi-zh)
(\japhdoi{0003890\#S65})
\end{exe}

Negative infinitive converbs can have an adversative interpretation (§\ref{sec:concessive.clauses}). In (\ref{ex:mAkAtWG.kuwWGsWjGatandZi}),  \forme{mɤ-kɤ-ɤtɯɣ} `not meeting' expresses the non-realization of a telic event that was the original purpose of the previous actions of the main referent. 

\begin{exe}
\ex \label{ex:mAkAtWG.kuwWGsWjGatandZi}
\gll [a-mu a-wi ni mɤ-kɤ-ɤtɯɣ] kú-wɣ-sɯ-jɣat-a-ndʑi. \\
\textsc{3sg}.\textsc{poss}-mother \textsc{3sg}.\textsc{poss}-grand.mother \textsc{du} \textsc{neg}-\textsc{inf}-meet \textsc{ipfv}:\textsc{east}-\textsc{inv}-\textsc{caus}-go.back-\textsc{1sg}-\textsc{du} \\
\glt `[My uncles] forced me to go back [to school] without having seen my mother and my grandmother (even though I had not met them).' (2010-Dpalcan-09)
\end{exe}

Converbial infinitival clauses can contain an object or a semi-object which does not belong to the main clause, as \forme{a-mu a-wi ni} `my mother and my grandmother' in (\ref{ex:mAkAtWG.kuwWGsWjGatandZi}). The dual on \forme{kú-wɣ-sɯ-jɣat-a-ndʑi} refers to the subject (`the uncles', mentioned in the previous clause), not the mother and the grandmother.

In nearly all examples, there is subject (S/A) coreference between the matrix clause and the converbial clause, but examples like (\ref{ex:mAkAtWG.kuwWGsWjGatandZi}), where we rather observe coreference between the \textit{object} of the main clause and the subject of the subordinate clause, shows that there is no syntactic constraint on subject coreference in this construction.

Infinitives in \forme{kɯ-} (impersonal or stative) also occur as converbs, though these forms could in principle also be analyzed as subject participles (§\ref{sec:subject.participles}). For instance, in (\ref{ex:mAkWmbrAt.YWrAma}) \forme{mɤ-kɯ-mbrɤt} `without stop' is considered to be an impersonal infinitive serving as a manner converb, but it could be possible to propose an alternative analysis as a \forme{kɯ-} subject participle `the one which does not stop' used adverbially.  The analysis as infinitives however better accounts for the fact that only the verbs whose infinitive is in \forme{kɯ-} have converbial forms in \forme{kɯ-}.

\begin{exe}
\ex \label{ex:mAkWmbrAt.YWrAma}
 \gll nɯ maka mɤ-kɯ-mbrɤt ʑo ɲɯ-rɤma ɲɯ-ɕti tɕe,  \\
 \textsc{dem} at.all  \textsc{neg}-\textsc{inf}:\textsc{impers}:S/A-\textsc{acaus}:break \textsc{emph} \textsc{ipfv}-work \textsc{sens}-be.\textsc{aff} \textsc{lnk} \\
 \glt `It works without stopping at all.' (26-GZo)
(\japhdoi{0003668\#S62})
\end{exe}

The existential verbs also occur in converbial use. For instance, \japhug{tu}{exist} in infinitive form \forme{kɯ-tu} following a noun or a pronoun can mean `in the presence of...' as in (\ref{ex:Zara.kWtu.Zo}).

\begin{exe}
\ex \label{ex:Zara.kWtu.Zo}
\gll [ʑara kɯ-tu] ʑo to-sɤrɯru tɕe, \\
\textsc{3pl} \textsc{inf}:\textsc{stat}-exist \textsc{emph} \textsc{ifr}-compare \textsc{lnk} \\
\glt `He compared [his testimony with theirs] in their presence.' (150909 xifangping-zh) (\japhdoi{0006408\#S147})
\end{exe}

Similarly, the negative existential verb \japhug{me}{not exist} in infinitive converbial form means `without...', as in example  (\ref{ex:nAzo.kWme}).

\begin{exe}
\ex \label{ex:nAzo.kWme}
\gll tɕe nɤj a-pi [nɤʑo kɯ-me] aʑo kɤ-rɤʑi mɤ-cʰa-a \\
\textsc{lnk} \textsc{2sg} \textsc{1sg}.\textsc{poss}-elder.sibling \textsc{2sg} \textsc{inf}:\textsc{stat}-not.exist \textsc{1sg} \textsc{inf}-stay \textsc{neg}-can:\textsc{fact}-\textsc{1sg} \\
\glt `Brother, without you I cannot stay [here].' (2011-05-nyima)
\end{exe}

\subsubsection{Velar infinitives as adverbs}    \label{sec:velar.inf.adverb}
The \forme{kɯ-} infinitive of stative verbs can be used to create adverbs. The most common example is the degree adverb  \japhug{kɯxtɕɯxtɕi}{a little} from \japhug{xtɕi}{be small} in sentences such as (\ref{ex:kWxtCWxtCi.wxti}). In this example, the co-occurrence of an adverb derived from \japhug{xtɕi}{be small} with the verb \japhug{wxti}{be big} shows that this adverb is already fully grammaticalized, otherwise such a sentence would be self-contradictory.

\begin{exe}
\ex \label{ex:kWxtCWxtCi.wxti}
 \gll βʑɯ sɤz kɯ-xtɕɯ\redp{}xtɕi wxti. \\
 mouse \textsc{comp} \textsc{inf}:\textsc{stat}-\textsc{emph}\redp{}be.small be.big:\textsc{fact} \\
 \glt `It is a little bigger than a mouse.' (21-GzWLa) (\japhdoi{0003570\#S4})
\end{exe}

An even more lexicalized example is \japhug{mɤkɯftsʰi}{forcibly}, which occurs in particular with causative verbs to express coercive causation (§\ref{sec:sig.caus.coercitive}), as in (\ref{ex:mAkWfsthi.chAsWrku}).

\begin{exe}
\ex \label{ex:mAkWfsthi.chAsWrku}
 \gll  tɕendɤre mɤkɯftsʰi ʑo, nɤki ɯ-me ɣɯ ɯ-mi nɯ iɕqʰa tɯ-xtsa ɯ-ŋgɯ nɯtɕu cʰɤ-sɯ-rku  \\
 \textsc{lnk} forcibly \textsc{emph} \textsc{filler} \textsc{3sg}.\textsc{poss}-daughter \textsc{gen}  \textsc{3sg}.\textsc{poss}-foot \textsc{dem} the.aforementioned \textsc{indef}.\textsc{poss}-shoe \textsc{3sg}.\textsc{poss}-in \textsc{dem}:\textsc{loc} \textsc{ifr}:\textsc{downstream}-\textsc{caus}-put.in \\
\glt `She forced her daughter to put her foot into that shoe.' (140504 huiguniang-zh)
(\japhdoi{0003909\#S219})
\end{exe}

This adverb is formally the negative stative infinitive of the verb \japhug{ftsʰi}{feel better} (of a disease) (§\ref{sec:sig.caus.lexicalized}).

These adverbs are probably lexicalized from infinitival converbs (§\ref{sec:inf.converb}), but differ from them in lacking any argument structure.
 
\subsubsection{Lexicalized velar infinitives}    \label{sec:lexicalized.velar.inf}
Lexicalized velar infinitives in \forme{kɤ-} found in some compounds, though some cases could alternatively be analyzed as lexicalized object participles, and are treated in §\ref{sec:lexicalized.object.participle}.

The delocutive expression \japhug{ŋɤtɕɯkɤti,kʰɯ}{obey to everything} provides an unambiguous example of lexicalized velar infinitive. The compound \forme{ŋɤtɕɯkɤti} combines the pronoun \japhug{ŋotɕu}{where} in bound state form \forme{ŋɤtɕɯ-} with the  infinitive \forme{kɤ-ti} of the verb \japhug{ti}{say}, and is exclusively used in collocation with \japhug{kʰɯ}{be possible, agree}, as in (\ref{ex:NAtCWkAti}).\footnote{The causative \japhug{ŋɤtɕɯkɤti,sɯkʰɯ}{cause to obey to everything} also exists.} 
This expression originates presumably from a phrase such as `agree (\forme{kʰɯ}) to whatever (\forme{ŋotɕu}) $X$ says (\forme{kɤ-ti})'. However, it should be noted that the pronoun \japhug{tɕʰi}{what}, not \japhug{ŋotɕu}{where} is used in Japhug in the free-choice indefinite construction meaning `whatever' as in examples (\ref{ex:tChi.pWnWNWNu}) to (\ref{ex:tChi.kWstWstua}) in §\ref{sec:interrogative.indef}). The form \forme{kɤ-ti} in any case was originally the complement of the verb \japhug{kʰɯ}{be possible, agree}, which takes infinitival complements (§\ref{sec:ra.khW.jAG.verb}), and thus is not analyzable as a former object participle.

 \begin{exe}
\ex \label{ex:NAtCWkAti}
\gll  ɯ-tɕɯ kɯβde nɯra wuma ʑo ŋɤtɕɯkɤti pjɤ-kʰɯ-nɯ  \\
\textsc{3sg}.\textsc{poss}-son four \textsc{dem}:\textsc{pl} really \textsc{emph} obey.to.everything(1) \textsc{ifr}.\textsc{ipfv}-obey.to.everything(2)-\textsc{pl} \\
\glt `His four sons were very obedient.' (140508 benling gaoqiang de si xiongdi-zh)
(\japhdoi{0003935\#S15})
\end{exe} 

\subsection{Bare infinitives} \label{sec:bare.inf}
Bare infinitives are formed by combining the stem I of the verb with a possessive prefix coreferential with the object of the complement clause, as in example (\ref{ex:Wmto.mWpWrYota}). Bare infinitives are not attested with orientation, polarity or associated motion prefixes. They historically derive from bare action nominal (§\ref{sec:bare.action.nominals}), which have become a very restricted subclass.

\begin{exe} 
\ex \label{ex:Wmto.mWpWrYota}
\gll nɤʑo kɯ-fse a-ŋkʰor nɯ ɯ-mto mɯ-pɯ-rɲo-t-a \\
you \textsc{nmlz}:\textsc{stat}-be.like \textsc{1sg}.\textsc{poss}-subject \textsc{top} \textsc{3sg}.\textsc{poss}-\textsc{bare}.\textsc{inf}:see \textsc{neg}-\textsc{aor}-experience-\textsc{pst}:\textsc{tr}-\textsc{1sg} \\
\glt `I never saw anyone like you among my subjects.' (28-smAnmi,  393) (\japhdoi{0004063\#S373})
\end{exe} 

With \japhug{ʁzɤβ}{be careful} as matrix verb however, the possessive prefix can also index the transitive subject as in (\ref{ex:ndZipa.YWRzAB}) (expressing an indefinite object).

\begin{exe}
\ex \label{ex:ndZipa.YWRzAB}
\gll kɤtsa ni ndʑi-pa ɲɯ-ʁzɤβ rca \\
parents.and.children \textsc{du} \textsc{3du}.\textsc{poss}-\textsc{bare}.\textsc{inf}:do \textsc{sens}-be.careful \textsc{sfp} \\
\glt `The two of them do things very carefully.' (14-05-10)
\end{exe}

Intransitive verbs do not have bare infinitives. Complement-taking verbs selecting bare infinitives for transitive verbs either take dental \forme{tɯ-} infinitives (§\ref{sec:dental.inf}) or velar infinitives (§\ref{sec:velar.inf}) when occurring with intransitive verbs.

\subsubsection{Complement clauses} \label{sec:bare.inf.complement} 
Bare infinitives only occur in complement clauses (§\ref{sec:bare.dental.inf}). Apart from the aspectual verb \japhug{rɲo}{experience, have already} mentioned above in (\ref{ex:Wmto.mWpWrYota}), bare infinitives are found with two categories of complement-taking verbs.

First, they are compatible with some phasal verbs (§\ref{sec:phasal.complements}) such as \japhug{ʑa}{begin}, \japhug{sɤʑa}{begin}, \japhug{stʰɯt}{finish} and \japhug{jɤɣ}{finish} as in (\ref{ex:WtsxWB.thWjAG}).

\begin{exe} 
\ex \label{ex:WtsxWB.thWjAG}
\gll  tɕe nɯnɯ tɯ-ŋga nɯ ɯ-tʂɯβ tʰɯ-jɤɣ \\
\textsc{lnk} \textsc{dem} \textsc{indef}.\textsc{poss}-clothes \textsc{dem} \textsc{3sg}.\textsc{poss}-\textsc{bare}.\textsc{inf}:sew \textsc{aor}-finish   \\
\glt `When one has finished sewing the clothes, ...' (30-tWNga)
(\japhdoi{0004069\#S29})
\end{exe} 

Second, they are found with some adjectives such as \japhug{βdi}{be well, be good} and derived sigmatic or velar causative verbs such as \forme{ɣɤ-βdi} `repair, cause to be good, do $X$ well' as in (\ref{ex:WnApWpa.tuGABdinW}) (see §\ref{sec:velar.caus.complement}, §\ref{sec:causative.manner.complement}).

\begin{exe} 
\ex \label{ex:WnApWpa.tuGABdinW}
\gll lu-ji-nɯ qʰe ɯ-nɤpɯpa tu-ɣɤ-βdi-nɯ, ɯ-ɣli ra ku-sɤpe-nɯ qʰe, cʰɯ-do ɲɯ-cʰa. \\
\textsc{ipfv}-plant-\textsc{pl} \textsc{lnk} \textsc{3sg}.\textsc{poss}-\textsc{bare}.\textsc{inf}:take.care \textsc{ipfv}-\textsc{caus}-be.good-\textsc{pl} \textsc{3sg}.\textsc{poss}-dung \textsc{pl} \textsc{ipfv}-do.well-\textsc{pl} \textsc{lnk} \textsc{ipfv}-be.fibrous \textsc{sens}-can \\
\glt `[Now people] plant [pumpkin also in higher areas], [if] they take good care of it and add enough fertilizer, it becomes fibrous (so that it can be sowed for the next year).' (140522 kAmYW tWji)
(\japhdoi{0004055\#S46})
\end{exe} 

The same set of verbs take complement clauses with dental infinitives when the verb of the complement is intransitive (§\ref{sec:dental.inf.complement}).

The bare infinitives also occur in adnominal complement clauses with the noun \japhug{ɯ-tsʰɯɣa}{shape, manner}, as in (\ref{ex:bare.inf.noun}).

\begin{exe}
\ex \label{ex:bare.inf.noun}
\gll ndʑi-mi ɯ-tsʰoʁ ɯ-tsʰɯɣa nɯra wuma ʑo naχtɕɯɣ-ndʑi.   \\
\textsc{3du}.\textsc{poss}-foot \textsc{3sg}-\textsc{bare}.\textsc{inf:}attach.to \textsc{3sg}.\textsc{poss}-form \textsc{dem}:\textsc{pl} very \textsc{emph} be.the.same:\textsc{fact}-\textsc{du}  \\
\glt `The way their legs [of fleas and crickets] touch the ground is very similar.' (26-mYaRmtsaR)
(\japhdoi{0003674\#S17})
\end{exe}

No verb requires a bare infinitive: all complement-taking verbs selecting it are either alternatively compatible with velar infinitives (§\ref{sec:inf.complementation}) or a finite complement (§\ref{sec:finite.complement}).

\subsubsection{Bare verb stem in negative existential construction} \label{sec:bare.inf.negative} 
A non-finite form resembling bare infinitives without possessive prefix is found in an unusual construction with the negative existential verbs  \japhug{me}{not exist} and \japhug{maŋe}{not exist}(§\ref{sec:suppletive.negative}) with alternative concessive meaning `whether or not $X$, it amounts to the same' (§\ref{sec:alt.concessive.conditional}), in which the bare verb stem occurs in affirmative and in negative form with the negative prefix \forme{mɤ-}. Unlike bare infinitives proper, this form exists with both transitive (\ref{ndza.mea}) and intransitive verbs (\ref{mACe.me}).

\begin{exe}
\ex \label{ndza.mea}
\gll ndza mɤ-ndza me-a \\
\textsc{bare}.\textsc{inf}:eat \textsc{neg}-\textsc{bare}.\textsc{inf}:eat not.exist:\textsc{fact}-\textsc{1sg} \\
\glt `Whether [you] eat me or not, it amounts to the same.' (sentence obtained as the correction of a sentence I produced to translate a story in Japhug)
\end{exe}

\begin{exe}
\ex \label{mACe.me}
\gll tɕe ɯ-qiɯ ɲɯ-mtsʰam-a, ɯ-qiɯ mɯ́j-mtsʰam-a qʰe, ɕe mɤ-ɕe maŋe   \\
\textsc{lnk} \textsc{3sg}.\textsc{poss}-half \textsc{sens}-hear-\textsc{1sg} \textsc{3sg}.\textsc{poss}-half \textsc{neg}:\textsc{sens}-hear-\textsc{1sg} \textsc{lnk} \textsc{bare}.\textsc{inf}:go \textsc{neg}-\textsc{bare}.\textsc{inf}:go not.exist:\textsc{sens} \\
\glt `I can hear half of it, can't hear the other half, whether or not [I] go, it amounts to the same.' (conversation 140510)
\end{exe}

In this construction, the negative auxiliaries can take person marking, and are obligatorily coreferential with the object if the verb in the complement clause is transitive, as in (\ref{ndza.mea}). With intransitive verbs, no person marking appears on the negative verb as in (\ref{mACe.me}).\footnote{The fact that only the object is indexed on the auxiliary verb suggests here that this construction displays nominative-accusative alignment.  }
In this construction, the transitive subject (whether of transitive or intransitive verbs) cannot be overt.

Although the non-finite form \forme{ɕe} in  (\ref{mACe.me}) superficially resembles a \textsc{3sg} Factual Non-Past (§\ref{sec:factual}), the absence of stem III alternation (§\ref{sec:stem3}) on \forme{ndza} in (\ref{ndza.mea}) shows that this cannot be a finite form (otherwise \forme{ndze} `s/he/it will eat/eats it' would be expected).

\subsection{Dental infinitives} \label{sec:dental.inf}
Dental infinitives (glossed as `second infinitives' \textsc{inf}:II) are built by prefixing \forme{tɯ-} with the verb stem. Dental infinitives occur with intransitive verbs, including dynamic (\ref{ex:tWNke.taZa.tCe}) and stative verbs (\ref{ex:tArNi.YoZa}), including semi-transitive verbs, but are not attested with transitive verbs, which take bare infinitives or velar infinitives instead.

\begin{exe} 
\ex \label{ex:tWNke.taZa.tCe}
\gll tɯ-ŋke ta-ʑa tɕe \\
\textsc{inf}:\textsc{II}-walk \textsc{aor}:3\flobv{}-start  \textsc{lnk} \\
\glt `When it starts moving...' (26-NalitCaRmbWm) (\japhdoi{0003676\#S76})
\end{exe} 

 
With contracting verbs (§\ref{sec:contraction}), regular vowel fusion between the \forme{tɯ-} prefix and stem-initial  \forme{a-} occurs, resulting in the surface form \ipa{tɤ-} as in (\ref{ex:tArNi.YoZa}).

\begin{exe} 
\ex \label{ex:tArNi.YoZa}
\gll si nɯ daltsɯtsa nɯ tɯ-ɤrŋi ɲo-ʑa tɕe \\
tree \textsc{dem} slowly \textsc{dem} \textsc{inf}:II-be.green \textsc{ifr}-start \textsc{lnk} \\
\glt `The tree slowly started to become green.' (divination 2003)
\end{exe} 

However, a few morphologically transitive verbs with dummy subjects (§\ref{sec:transitive.dummy}), in particular \japhug{lɤt}{throw} and \japhug{βzu}{make, do} do take \forme{tɯ-} infinitives as in (\ref{ex:tWlAt.pjAZa}).  Note that in these complex predicates, the light verbs \japhug{lɤt}{throw} and \japhug{βzu}{make, do}, although transitively conjugated, cannot take an overt subject marked with the ergative, and only have one argument.
 
\begin{exe}
\ex  \label{ex:tWlAt.pjAZa}
\gll tɯ-mɯ kɯ-wxtɯ\redp{}wxti ʑo tɯ-lɤt pjɤ-ʑa \\
\textsc{indef}.\textsc{poss}-sky \textsc{sbj}:\textsc{pcp}-\textsc{emph}\redp{}be.big \textsc{emph} \textsc{inf}-throw \textsc{ifr}-start \\
\glt `A heavy rain started.' (150819 haidenver-zh) (\japhdoi{0006314\#S101})
\end{exe}

It is possible that dental infinitives are historically related to degree nominals (§\ref{sec:degree.nominals}) and action nominals (§\ref{sec:action.nominals}), which however do not display the same transitivity restrictions. Several hypotheses accounting for the origin of dental infinitives are presented in §\ref{sec:dental.nmlz.history}.

\subsubsection{Polarity prefixes} \label{sec:dental.inf.polarity}
Dental infinitives are only compatible with polarity prefixes (as in example \ref{ex:mAtWrga}), and cannot take orientation or associated motion. Possessive prefixes on dental infinitives only occur in the simultaneous construction with the verb \japhug{sɯpa}{cause to do} (§\ref{sec:bare.dental.inf.sWpa}).

\begin{exe}
\ex  \label{ex:mAtWrga}
\gll qaɟy ɯ-me nɯnɯ, tɕendɤre kʰro mɤ-tɯ-rga to-ʑa\\
fish \textsc{3sg}.\textsc{poss}-daughter \textsc{dem} \textsc{lnk} a.lot \textsc{neg}-\textsc{inf}:\textsc{II}-like \textsc{ifr}-start \\
\glt `He started not liking the mermaid that much [anymore].' (150819 haidenver-zh)
(\japhdoi{0006314\#S149})
\end{exe}

Degree nominals present the same constraints on orientation and associated motion prefixes (§\ref{sec:degree.nominal.prefixes}).

\subsubsection{Complement clauses} \label{sec:dental.inf.complement}
Dental \forme{tɯ-} infinitives are only attested in complement clauses (§\ref{sec:bare.dental.inf}), and are only found with the verbs that select a bare infinitive (§\ref{sec:bare.inf.complement}) when the verb in the complement clause is transitive. This complementary distribution suggests that bare infinitive and dental infinitives could be treated as two variants of the same grammatical category.

Dental infinitives are more often attested with phasal verbs, in particular \japhug{ʑa}{start} and \japhug{sɤʑa}{start} as in (\ref{ex:tWskAm.pjAsAZa}).

\begin{exe}
\ex \label{ex:tWskAm.pjAsAZa}
\gll mtsʰu nɯ cʰɯmcʰɯm ʑo, tɕe, tɯ-skɤm pjɤ-sɤʑa \\
lake \textsc{dem} \textsc{idph}(II):slowly.retreating \textsc{emph} \textsc{lnk} \textsc{inf}:II-be.dry \textsc{ifr}-start \\
\glt `The [water of the] lake started to retreat slowly.' (nyima wodzer 2003)
\end{exe}

Non-phasal verbs selecting bare infinitive such as \japhug{rɲo}{experience} never occur with the dental infinitive in the corpus, but such forms can be elicited, as in (\ref{ex:tWGi.pWrYota}). In the corpus, intransitive complements of the verb \japhug{rɲo}{experience} rather velar infinitives; it is also possible  in (\ref{ex:tWGi.pWrYota}) to replace the dental infinitive \forme{tɯ-ɣi} with a velar infinitive \forme{kɤ-ɣi}.

\begin{exe}
\ex \label{ex:tWGi.pWrYota}
\gll  mbarkʰom tɯ-ɣi pɯ-rɲo-t-a  \\
\textsc{topo} \textsc{inf}:II-come \textsc{aor}-experience-\textsc{pst}:\textsc{tr}-\textsc{1sg} \\
\glt `I came to Mbarkham before.' (elicited)
\end{exe}

Like other intransitive verbs, antipassivized transitive verbs can take a dental infinitive (as in \ref{ex:tWrArAt.paZa}), unlike the base verb from which they are derived (as in \ref{ex:WrAt.paZa}, where a bare infinitive is used instead).

\begin{exe}
\ex \label{ex:tWrArAt.paZa}
\gll tɯ-rɤ-rɤt pa-ʑa \\
\textsc{inf}:II-\textsc{apass}-write \textsc{aor}:3\flobv{}-start \\
\glt `He started writing.' (elicited)
\end{exe}

\begin{exe}
\ex \label{ex:WrAt.paZa}
\gll tɤscoz ɯ-rɤt pa-ʑa \\
letter \textsc{3sg}.\textsc{poss}-\textsc{bare}.\textsc{inf:}write \textsc{aor}:3\flobv{}-start \\
\glt `He started writing the/a letter.' (elicited)
\end{exe}

In addition, dental infinitives also occur in complements of causativized verbs (§\ref{sec:bare.dental.inf.sWpa}, §\ref{sec:causative.manner.complement}), and in the construction expressing simultaneous actions with \japhug{sɯpa}{cause to do} described in §\ref{sec:bare.inf.complement}.

\section{Degree nominals} \label{sec:degree.nominals}

Degree nominals are built by combining the verb stem with a nominalizing \forme{tɯ-} prefix and a possessive prefix coreferent with the subject, as the dual \forme{ndʑi-} in (\ref{ex:ndZitAmWmi}). The form \forme{ndʑi-tɯ-ɤmɯmi} also shows that the nominalization \forme{tɯ-} prefix undergoes regular vowel fusion with stem-initial \forme{a-} to \ipa{tɤ}. 

 \begin{exe}
\ex \label{ex:ndZitAmWmi}
\gll  tɕendɤre ndʑi-tɯ-ɤmɯmi ndʑi-tɯ-scit pɯ-saχaʁ ʑo ɲɯ-ŋu \\
 \textsc{lnk} \textsc{3du}.\textsc{poss}-\textsc{nmlz}:\textsc{deg}-be.on.good.terms \textsc{3du}.\textsc{poss}-\textsc{nmlz}:\textsc{deg}-be.happy \textsc{pst}.\textsc{ipfv}-be.extremely \textsc{emph} \textsc{sens}-be \\
 \glt `They were very happy together.' (2005 Lobzang)
(\japhdoi{0003370\#S14})
\end{exe}

All gradable stative verbs can form degree nominals. They most typically occur in constructions expressing degree as in (\ref{ex:ndZitAmWmi}) (§\ref{sec:degree.nominal.construction}, §\ref{sec:degree.nominal.subject}), but can also indicate manner as in (\ref{ex:WtAjRu}).

 \begin{exe}
	\ex \label{ex:WtAjRu}
	\gll nɯ-mi ɯ-tɯ-ɤjʁu kɯnɤ mɯ́j-naχtɕɯɣ. \\
	\textsc{3pl}.\textsc{poss}-leg \textsc{3sg}.\textsc{poss}-\textsc{nmlz}:\textsc{deg}-be.curved also \textsc{neg}:\textsc{sens}-be.the.same \\
	\glt `[People with clubfoot] also differ in the way that their feet are curved (some have both legs curved, some only one, some have the legs rotated inwards, others rotated outwards) (160719 kAmARu) 
\end{exe}

With dynamic verbs, degree nominals express either the intensity or the frequency of an action as in (\ref{ex:WtWnAru}), or the manner of the action (\ref{ex:tWtAmdzW.kWBdi}). Such examples are however uncommon in the corpus.

\begin{exe}
\ex \label{ex:WtWnAru}
\gll  japa tɕe pʰaʁrgot ɯ-tɯ-nɤru ɲɯ-saχaʁ ʑo tɕe \\
last.year \textsc{loc} boar  \textsc{3sg}.\textsc{poss}-\textsc{nmlz}:\textsc{deg}-eat.crops \textsc{sens}-be.extremely \textsc{emph} \textsc{lnk} \\
\glt `Last year, a boar was causing a lot of damages to the crops.' (150829 phaRrgot)
(\japhdoi{0006414\#S1})
\end{exe}

\begin{exe}
\ex \label{ex:tWtAmdzW.kWBdi}
\gll tɯ-tɯ-ɤmdzɯ kɯ-βdi mɤ-kɯ-βdi kɯnɤ ʑo cʰɯ-sɤfɕɤra-nɯ pɯ-ɕti.  \\
\textsc{genr}:\textsc{poss}-\textsc{nmlz}:\textsc{deg}-sit \textsc{inf}:\textsc{stat}-be.well \textsc{neg}-\textsc{inf}:\textsc{stat}-be.well also \textsc{emph} \textsc{ipfv}-discuss-\textsc{pl} \textsc{pst}.\textsc{ipfv}-be.\textsc{aff}:\textsc{fact} \\
\glt `People would discuss whether one sat properly or not.' (31-khAjmu) 	(\japhdoi{0004079\#S28})
\end{exe}

Degree nominals from transitive verbs are extremely rare, but do occur in particular for tropative verbs (§\ref{sec:tropative}) as in (\ref{ex:WtWnAmpCAr}). It is possible to elicitate degree nominals for most transitive verbs, even if such forms are not attested in the corpus.
 
\begin{exe}
\ex \label{ex:WtWnAmpCAr}
\gll  maka ɯ-tɯ-nɤ-mpɕɤr kɯ pjɤ-nɤscɤr ʑo. \\
completely  \textsc{3sg}.\textsc{poss}-\textsc{nmlz}:\textsc{deg}-\textsc{trop}-be.beautiful \textsc{erg} \textsc{ifr}-be.startled \textsc{emph} \\
\glt `He found her so beautiful that he was startled.' (140429 jiedi-zh)
\end{exe} 
 
Although similar in form to dental infinitives (§\ref{sec:dental.inf}) and action nominals (§\ref{sec:action.nominals}), they differ from both categories in requiring the presence of a possessive prefix (see §\ref{sec:dental.nmlz.history} concerning the historical relationship between these forms). In addition, unlike dental infinitives, degree nominals are compatible with both intransitive and transitive verbs. Their semantics is also fully predictable, unlike action nominals which tend to be lexicalized (§\ref{sec:lexicalized.action.nominals}).

Degree nominal occur in exclamative nominal predicates (§\ref{sec:degree.nominal.predicates}), degree constructions (§\ref{sec:degree.nominal.construction}) and also several types of complement clauses (§\ref{sec:degree.nominal.complement}).


Another type of degree nominal is built by compounding the stems of two antonyms, for instance \japhug{jaʁmba}{thickness} (of a sheet) from \japhug{jaʁ}{be thick} and \japhug{mba}{be thin} (§\ref{sec.v.v.compounds.degree}). The corresponding regular degree nouns also exist (\forme{ɯ-tɯ-jaʁ} and \forme{ɯ-tɯ-mba}).

\subsection{Polarity prefixes} \label{sec:degree.nominal.prefixes}
Like dental infinitives (§\ref{sec:dental.inf.polarity}), degree nominals cannot be used with orientation or associated motion prefixes, but can be found with the negative prefix \forme{mɤ\trt}, as in (\ref{ex:WmAtWnWGWNke2}).

 \begin{exe}
\ex \label{ex:WmAtWnWGWNke2}
\gll  maka ɯ-mɤ-tɯ-nɯɣɯ-ŋke pjɤ-saχaʁ ʑo \\
at.all \textsc{3sg}.\textsc{poss}-\textsc{neg}-\textsc{nmlz}:\textsc{deg}-\textsc{facil}-walk \textsc{ifr}.\textsc{ipfv}-be.extremely \textsc{emph} \\
\glt `It was extremely inconvenient to walk [on the soft earth].' (2014-kWlAG)
\end{exe}


\subsection{Argument structure} \label{sec:degree.nominal.arguments}
Unlike action nominals, degree nominals still keep their argument structure intact. They can take complements exactly in the same way as finite verb forms. For instance, \japhug{βdi}{be well, be good} takes the bare infinitive \forme{ɯ-taʁ} in (\ref{ex:WtaR.WtWBdi}).\footnote{Note that the bare infinitive \forme{ɯ-taʁ} from the verb \japhug{taʁ}{weave} is homophonous with the relator noun \japhug{ɯ-taʁ}{on, above} (§\ref{sec:WtaR}) and that this construction is potentially ambiguous. }

\begin{exe}
\ex \label{ex:WtaR.WtWBdi}
\gll maka nɯ raz rcanɯ, ɯ-tɯ-pe ɲɯ-saχaʁ ʑo, ɯ-taʁ ɯ-tɯ-βdi, ɯ-tɯ-mpɕɤr ɲɯ-saχaʁ \\
at.all \textsc{dem} cloth \textsc{unexp}:\textsc{deg}  \textsc{3sg}.\textsc{poss}-\textsc{nmlz}:\textsc{deg}-be.good \textsc{sens}-be.extremely \textsc{emph} \textsc{3sg}.\textsc{poss}-\textsc{bare}.\textsc{inf}:weave  \textsc{3sg}.\textsc{poss}-\textsc{nmlz}:\textsc{deg}-be.well \textsc{3sg}.\textsc{poss}-\textsc{nmlz}:\textsc{deg}-be.beautiful \textsc{sens}-be.extremely \\
\glt `This piece of cloth is very nice, it is extremely well woven, it is extremely beautiful.' (140521 huangdi de xinzhuang-zh)
(\japhdoi{0004047\#S77})
\end{exe}

In (\ref{ex:chWwxti.WtWmbat}), the verb \japhug{mbat}{be easy} selects a finite complement clause comprising the verb \forme{cʰɯ-wxti} `it grows bigger' in imperfective third singular form.

\begin{exe}
\ex \label{ex:chWwxti.WtWmbat}
\gll cʰɯ-wxti ɯ-tɯ-mbat ɲɯ-sɤre ʑo \\
\textsc{ipfv}-be.big  \textsc{3sg}.\textsc{poss}-\textsc{nmlz}:\textsc{deg}-be.easy \textsc{sens}-be.ridiculous \textsc{emph} \\ 
\glt `It grows very easily.' (25-akWzgumba) 
(\japhdoi{0003632\#S70})
\end{exe}

Degree nominals can also be used with oblique arguments, such as the first person marked by the relator noun \forme{a-taʁ} (unrelated to the bare infinitive \forme{ɯ-taʁ} from the previous example) in (\ref{ex:ataR.WtWGWtsxWn}).


\begin{exe}
\ex \label{ex:ataR.WtWGWtsxWn}
\gll a-wi a-taʁ ɯ-tɯ-ɤɣɯtʂɯn ɯ-grɤl me \\
\textsc{1sg}.\textsc{poss}-grandmother \textsc{1sg}-on \textsc{3sg}.\textsc{poss}-\textsc{nmlz}:\textsc{deg}-be.kind \textsc{3sg}.\textsc{poss}-order not.exist:\textsc{fact} \\
\glt `My grandmother was extremely kind to me (so that I have to repay her).' (2005 Kunbzang)
\end{exe}


\subsection{Nominal predicates} \label{sec:degree.nominal.predicates}
Degree nominals commonly occur as nominal predicates (§\ref{sec:non.verbal.predicates}), either with the sentence final particle \forme{nɯ} as in (\ref{ex:nWtWmtsWrCpaR}), or as a bare noun phrase as in (\ref{ex:WtWBdi.WtWmpCAr}). In this predicative use, degree nominals express an exclamation, possibly including surprise as in (\ref{ex:mtshAri.ataR.WtWpe}) and (\ref{ex:WtWBdi.WtWmpCAr}).

\begin{exe}
\ex \label{ex:nWtWmtsWrCpaR}
\gll lɤ-ɣi-nɯ wo tɤ-rɯndzɤtsʰi-nɯ ma, nɯ-tɯ-mtsɯr-ɕpaʁ nɯ, nɯ-tɯ-ɲat nɯ\\
\textsc{imp}:\textsc{upstream}-come-\textsc{pl} \textsc{sfp} \textsc{imp}-have.a.meal-\textsc{pl} C \textsc{2pl}.\textsc{poss}-\textsc{nmlz}:\textsc{deg}-be.hungry-be.thirsty \textsc{sfp} \textsc{2pl}.\textsc{poss}-\textsc{nmlz}:\textsc{deg}-be.tired \textsc{sfp} \\
\glt `Come in and have a meal, you [must be] so hungry, thirsty and tired!' (160701 poucet2)	(\japhdoi{0006155\#S39})
\end{exe}

Example (\ref{ex:WtaR.WtWBdi}) can be compared with (\ref{ex:WtWBdi.WtWmpCAr}) with the verb of degree \japhug{saχaʁ}{be extremely} in sensory form. It is possible that the use of degree nominal as nominal predicates historically results from the ellipsis of the degree predicate. The high frequency of the predicative use of degree nominals in the corpus however indicates that this usage has been constructionalized.

\begin{exe}
\ex \label{ex:WtWBdi.WtWmpCAr}
\gll ɯ-taʁ ɯ-tɯ-βdi, ɯ-tɯ-mpɕɤr! \\
\textsc{3sg}.\textsc{poss}-\textsc{bare}.\textsc{inf}:weave  \textsc{3sg}.\textsc{poss}-\textsc{nmlz}:\textsc{deg}-be.well \textsc{3sg}.\textsc{poss}-\textsc{nmlz}:\textsc{deg}-be.beautiful  \\
\glt `It is woven so well, it is so beautiful!' (140521 huangdi de xinzhuang-zh) 	(\japhdoi{0004047\#S162})
\end{exe}

\subsection{Degree construction} \label{sec:degree.nominal.construction}
The most common use of the degree nominals is in the degree construction, where they occur as intransitive subjects of degree verbs like \japhug{saχaʁ}{be extremely}, which are always in \textsc{3sg} form as in (\ref{ex:WtWjpum.pjAsaXaR}). The possessive prefix on the degree noun is coreferent with the referent having the property described by the nominalized verb, often \textsc{3sg} as in  (\ref{ex:WtWjpum.pjAsaXaR}), but not exclusively, for instance with a \textsc{3du} in (\ref{ex:ndZitAmWmi}) above.


\begin{exe}
\ex \label{ex:WtWjpum.pjAsaXaR}
\gll nɯnɯ si nɯ ɯ-tɯ-jpum pjɤ-saχaʁ ʑo tɕe,\\
\textsc{dem} tree \textsc{dem} \textsc{3sg}.\textsc{poss}-\textsc{nmlz}:\textsc{deg}-be.thick \textsc{ifr}.\textsc{ipfv}-be.extremely \textsc{emph} \textsc{lnk} \\
\glt `That tree was extremely thick.' (150902 luban-zh)
(\japhdoi{0006268\#S90})
\end{exe}

This construction is described in more details in (§\ref{sec:degree.nominal.subject}).



\subsection{Complementation} \label{sec:degree.nominal.complement}
Degree nominals are marginally attested in a complementation strategy, as subject clauses of the modal verb \forme{ra} `need, have to' in negative form, meaning `$X$ should not be so $Y$', where $X$ is the subject indexed by the possessive prefix preceding the \forme{-tɯ-} nominalization prefix, and $Y$ the verb in degree nominal form. For instance, the common expression in (\ref{ex:nAtWsAre.nW.mAra}) with the degree nominal \forme{nɤ-tɯ-sɤre} literally means `You should not be so ridiculous'.  

\begin{exe}
\ex \label{ex:nAtWsAre.nW.mAra}
\gll  nɤ-tɯ-sɤre nɯ mɤ-ra \\
\textsc{3sg}.\textsc{poss}-\textsc{nmlz}:\textsc{deg}-be.ridiculous \textsc{dem} \textsc{neg}-be.needed \\
\glt `What you [did/said)] is outrageous.' (several examples, \japhdoi{0004087\#S76})
\end{exe}

This construction also occurs with the antipassive verb \japhug{sɤnɤkʰe}{bully people}, as in (\ref{ex:nAtWsAnAkhe.nW.mAra}).\footnote{The meaning of this sentence is close to Chinese \ch{你太欺负人了}{nǐ tài qīfù rén le}{you are out of line}.
}

\begin{exe}
\ex \label{ex:nAtWsAnAkhe.nW.mAra}
\gll nɤ-tɯ-sɤ-nɤkʰe nɯ mɤ-ra \\
\textsc{3sg}.\textsc{poss}-\textsc{nmlz}:\textsc{deg}-\textsc{apass}:\textsc{hum}-bully \textsc{dem} \textsc{neg}-be.needed \\
\glt `You are out of line (`you should not bully people like that').'  (28-qajdo) 
(\japhdoi{0003718\#S14})
\end{exe}

The degree nominals also seem to occur as adnominal complement of \japhug{ɯ-tsʰɯɣa}{shape, method} as in (\ref{ex:WtWrRom.WtshWGa}). Note however that since bare infinitives are attested in complement clauses of this noun (see example \ref{ex:bare.inf.noun}, §\ref{sec:bare.inf.complement}), it is also conceivable that the forms \forme{ɯ-tɯ-tsʰu} and \forme{ɯ-tɯ-rʁom} could be alternatively analyzed as dental infinitives (§\ref{sec:dental.inf}).

\begin{exe}
\ex \label{ex:WtWrRom.WtshWGa}
\gll  paʁ nɯ ɯ-βri, nɤki, ɯ-rme kɯ-me ʑo ɯ-tsʰɯɣa ɲɯ-fse ma ɯ-tɯ-tsʰu cʰondɤre ɯ-tɯ-rʁom ɯ-tsʰɯɣa nɯ ʑo ɲɯ-fse \\
pig \textsc{dem} \textsc{3sg}.\textsc{poss}-body \textsc{filler} \textsc{3sg}.\textsc{poss}-hair \textsc{inf}:\textsc{stat}-not.exist \textsc{emph} \textsc{3sg}.\textsc{poss}-shape \textsc{sens}-be.like \textsc{lnk} \textsc{3sg}.\textsc{poss}-\textsc{nmlz}:\textsc{deg}-be.fat \textsc{comit} \textsc{3sg}.\textsc{poss}-\textsc{nmlz}:\textsc{deg}-be.rough \textsc{3sg}.\textsc{poss}-shape \textsc{dem} \textsc{emph} \textsc{sens}-be.like \\
\glt `[The body of the elephant] resembles that of the pig in that it has no hair, in that it is fat and [its skin] is rough.' (19-RloNbutChi)
(\japhdoi{0003550\#S44})
\end{exe}

% tu-mbro nɯ, ɯ-tɯ-mbro nɯ tu-orɕo tsa tɕe tɕe,
\section{Action nominals and abstract noun} \label{sec:action.nominals}

\subsection{\forme{tɯ-} action nominals} \label{sec:tW.action.nominal}
There are two types of action nominals in Japhug: action nominals in \forme{tɯ\trt}, a very productive formation which is the main topic of this section, and the bare action nominals, treated in §\ref{sec:bare.action.nominals}.

Action nominals in \forme{tɯ-} can be built from both intransitive and transitive verbs. They differ from both participles and infinitives in that the argument structure of the verb is lost and the transitivity contrast neutralized, and cannot take objects or oblique arguments other than possessors like normal alienably possessed nouns.


The action nominal has three potential meanings.  First, it can refer to the action itself, for instance \japhug{tɯji}{planting and sowing}\footnote{This action noun should not be confused with the related inalienably possessed noun \japhug{tɯ-ji}{field}, which is a different nominalized form. } from the verb \japhug{ji}{plant}, \japhug{tɯɣɟaβ}{action of churning} from \japhug{ɣɟaβ}{churn}, \japhug{tɯrɟaʁ}{dance (n)} from \japhug{rɟaʁ}{dance} (intransitive verb), \japhug{tɯsi}{death} from \japhug{si}{die} or \japhug{tɯmu}{fear} from \japhug{mu}{fear}. In this function, it can be used in a collocation with \japhug{βzu}{make} (§\ref{sec:action.nominal.Bzu}).

Second, it can mean an object affected by, or resulting from the action. This is also the case with some intransitive verbs, for instance \forme{tɯqioʁ} from the intransitive \japhug{qioʁ}{vomit} which can either mean `the action of vomiting' or `vomitus' or \forme{tɯɕkʰo} from \japhug{ɕkʰo}{dry in the sun} which can either be `action of drying in the sun' or `grain that are dried in the sun' (see \ref{ex:tWCkho.chWBze} in §\ref{sec:action.nominal.Bzu}).

Third, it can also refer to the way an action is performed, for instance \forme{tɯ-rɤt} which means `style of writing, way of writing' as in (\ref{ex:er.WtWrAt}). This function is probably the direct historical origin of the degree nominals (§\ref{sec:degree.nominals}).

\begin{exe}
\ex \label{ex:er.WtWrAt}
\gll  tɕe <er> ɯ-tɯ-rɤt tsa ɲɯ-fse ri, nɯ stʰɯci mɯ-ɲɯ-ŋgɤɣ \\
\textsc{lnk} two \textsc{3sg}.\textsc{poss}-\textsc{nmlz}:\textsc{action}-write a.little \textsc{sens}-be.like \textsc{lnk} \textsc{dem} so.much \textsc{neg}-\textsc{sens}-\textsc{acaus}:warp \\
\glt `[The constellation] looks a little bit like the way `two' is written, but not as curved.' (29-LAntshAm)
(\japhdoi{0003726\#S50})
\end{exe} 
 
Stative verbs can also have \forme{tɯ-} nominals, expressing abstract nouns, for instance \japhug{tɯtʂaŋ}{justice} from \japhug{tʂaŋ}{be fair}.
%tɯkon
%tɯkrɤz

When the base verb has a stem in \forme{a\trt}, the action nominal prefix \forme{tɯ-} undergoes regular vowel fusion  to \forme{tɤ-} as in \japhug{tɤɣro}{game} from the intransitive verb \japhug{aɣro}{play} (however, this verb is also potentially analyzable as a \forme{a-} denominal verb, see §\ref{sec:denom.a} and §\ref{sec:bare.action.nominals} below).

The direction of derivation between noun and verb is not always trivial to determine. For instance, the noun \japhug{tɯtsɣe}{commerce} could seem to be the action nominal of the transitive verb \japhug{ntsɣe}{sell}. However, in this case it is better to analyze the noun as the base form, and the verb as denominal (the \forme{n-} element being an irregular allomorph of the \forme{nɯ-} denominal prefix, §\ref{sec:antipassive.irr.form}).

Action nominals can occur as prenominal modifiers, as in \japhug{tɯtaʁ mtʰɯxtɕɤr}{weaving belt} (the belt used to attach  the back-tension loom), comprising the action nominal \japhug{tɯtaʁ}{weaving} (from \japhug{taʁ}{weave}) and the compound noun \japhug{mtʰɯxtɕɤr}{belt}.

\subsection{\forme{tɤ-} abstract nouns} \label{sec:tA.abstract.nouns}
Abstract nouns in \forme{tɤ-} can be exclusively derived from adjectival stative verbs. The derivation is generally regular but there are exceptions like \japhug{tɤɣɲat}{tiredness} from \japhug{ɲat}{be tired} with an additional \forme{-ɣ-} or  \japhug{tɤŋɤm}{pain} from \japhug{mŋɤm}{hurt} with a missing \forme{-m-}.\footnote{In the case of \japhug{tɤŋɤm}{pain} it is possible that the verb \japhug{mŋɤm}{hurt} derives from the noun with the irregular allomorph of a denominal prefix, which could be either \forme{ɣɤ-} (§\ref{sec:denom.GW}) or \forme{mɤ-} (§\ref{sec:denom.mA}). Note also the existence of the compound noun \japhug{tɯxtɤŋɤm}{dysentery} from the bound state of \japhug{tɯ-xtu}{belly} with the root \forme{-ŋɤm}. }

Abstract nouns mainly occur with the ergative \forme{kɯ}. These postpositional phrases express either the manner in which an action takes place (§\ref{sec:manner.nominal.kW}) as in (\ref{ex:tArga.kW.Zo}), or the cause of the action describe by the main verb due to a high degree as in (\ref{ex:tAGYat.tAmtsWr.kW.Zo}) (see also §\ref{sec:degree.monoclausal}). 

%tɤsɤɕke kɯ

\begin{exe}
\ex \label{ex:tArga.kW.Zo}
\gll  tɤ-rga kɯ ʑo tʰɯ-ari ɲɯ-ŋu. \\
\textsc{nmlz}:\textsc{abstract}-be.happy \textsc{erg} \textsc{emph} \textsc{aor}:\textsc{downstream}-go[II] \textsc{sens}-be \\
\glt  `She went there happily.' (2005 Kunbzang)
\end{exe}

Example (\ref{ex:tAGYat.tAmtsWr.kW.Zo}) also illustrates that one ergative postposition can follow several abstract nouns linked with the comitative \forme{cʰo} or in bare coordination (§\ref{sec:bare.coordination}). 

\begin{exe}
\ex \label{ex:tAGYat.tAmtsWr.kW.Zo}
\gll  a-ʁi, tɤ-ɣɲat cʰo tɤ-mtsɯr tɤ-ɕpaʁ kɯ mɯ́j-sɤ-cʰa tɕe, nɯna-j je \\
1sg.poss-younger.sibling \textsc{nmlz}:\textsc{abstract}-be.tired \textsc{comit} \textsc{nmlz}:\textsc{abstract}-be.hungry \textsc{nmlz}:\textsc{abstract}-be.thirsty \textsc{erg} \textsc{neg}:\textsc{sens}-\textsc{prop}-can \textsc{lnk} rest:\textsc{fact}-\textsc{1pl} \textsc{hort} \\
\glt `Brother, we are so tired, hungry and thirsty that we can't [go any further], let us rest!' (qachGa 2012) (\japhdoi{0004087\#S138})
\end{exe}

The meaning of the abstract nouns in (\ref{ex:tAGYat.tAmtsWr.kW.Zo}) is very similar to that of degree nominals (§\ref{sec:degree.nominal.construction}) combined with the ergative in examples such as (\ref{ex:WtWmtsWr.kW}).
 
\begin{exe}
\ex \label{ex:WtWmtsWr.kW}
\gll  maka ɯ-tɯ-mtsɯr kɯ kɤ-ŋke mɯ-ɲɤ-cʰa \\
at.all \textsc{3sg}.\textsc{poss}-\textsc{nmlz}:\textsc{deg}-hungry \textsc{erg} \textsc{inf}-go \textsc{neg}-\textsc{ifr}-can \\
\glt `It was so hungry that it could not walk anymore.'  (140515 huli he yelv-zh)
(\japhdoi{0004002\#S37})
\end{exe}

%tɤɕtɯɕte tɯxtɤŋɤm tɤɕtɯɕte tu tu-ti-nɯ ŋgrɤl.
Abstract nouns in \forme{tɤ-} are also attested without ergative to express the state or condition described by the adjective, for instance \japhug{tɤ-mtsɯr}{hunger} from \japhug{mtsɯr}{be hungry}, as in (\ref{ex:atAmtsWr.anWsWGZi}) and (\ref{ex:tWtAmtsWr.YWGAphAn}). Note in both example the possibility of adding a possessive prefix on the noun.

\begin{exe}
\ex \label{ex:atAmtsWr.anWsWGZi}
\gll  a-tɤ-mtsɯr nɯra ci a-nɯ-sɯɣ-ʑi ɲɯ-ra \\
\textsc{1sg}.\textsc{poss}-\textsc{nmlz}:\textsc{abstract}-be.hungry \textsc{dem}:\textsc{pl} a.little \textsc{irr}-\textsc{pfv}-\textsc{caus}-ease \textsc{sens}-be.needed \\
\glt `Let's [catch the mouse] to ease my hunger.' (140518 mao he laoshu-zh) 	(\japhdoi{0004030\#S17})
\end{exe}

\begin{exe}
\ex \label{ex:tWtAmtsWr.YWGAphAn}
\gll   nɯnɯ ɕoʁɕoʁ ɯ-taʁ pɯ-kɤ-rɤt qajɣi nɯ kɯ tɯ-tɤ-mtsɯr ɲɯ-ɣɤ-pʰɤn mɤ-cʰa \\
\textsc{dem} paper \textsc{3sg}.\textsc{poss}-on \textsc{aor}-\textsc{obj}:\textsc{pcp}-write bread \textsc{dem} \textsc{erg} \textsc{genr}.\textsc{poss}-\textsc{nmlz}:\textsc{abstract}-be.hungry \textsc{ipfv}-\textsc{caus}-be.efficient \textsc{neg}-can:\textsc{fact} \\
\glt `Bread drawn on a piece of paper cannot ease one's hunger.' (160718 huabingchongji-zh)
(\japhdoi{0006087\#S35})
\end{exe}

In addition, we find abstract alienably possessed nouns in \forme{tɤ-} expressing an abstract state, but which are not synchronically derived from a verb, and whose corresponding verbs are denominal. \tabref{tab:abstract.nouns.denominal} presents some examples.  

\begin{table}
\caption{Abstract nouns not derived from verbs} \label{tab:abstract.nouns.denominal}
 \begin{tabular}{lll}
\lsptoprule
Abstract noun & Denominal Verb&  \\ 
\midrule
\japhug{tɤndʐo}{cold} & \japhug{ɣɤndʐo}{be cold} & vs. \\
\japhug{tɤscɤr}{being startled} &\japhug{nɤscɤr}{be startled} & vi.\\
\japhug{tɤzraʁ}{shame} &\japhug{nɤzraʁ}{feel shame, be embarrassed} & vi. \\
\japhug{tɤmqe}{scolding} & \japhug{nɤmqe}{scold} & vt.\\
\japhug{tɤndɯt}{quarrel, dispute (n)} & \japhug{nɤndɯt}{dispute} & vt.\\
\lspbottomrule
\end{tabular}
\end{table}

The relator noun \japhug{ɯ-tɤjɯ}{addition}, which is essentially attested as an incremental addition linker (`in addition to $X$', §\ref{sec:incremental.addition}) is a trace of the abstract noun from which the denominal verb \japhug{ɣɤjɯ}{add} was built (§\ref{sec:denom.tr.GA}).

These nouns possibly derive from base verbs which disappeared and were replaced by the corresponding denominal verbs.

The noun \japhug{tɤkʰe}{idiot, fool} deriving from \japhug{kʰe}{be stupid} formally resembles an abstract noun, but semantically differs from the other nouns in this category; it is possibly an alienabilized form of the property noun \japhug{ɯ-kʰe}{nasty} (§\ref{sec:property.nouns}, §\ref{sec:bare.action.nominals}).

  

\subsection{Simultaneous} \label{sec:simultaneous.action.nominal}
The simultaneous action nominal is built by prefixing an additional \forme{tɯ-} to the base form of the action nominal, resulting in a double \forme{tɯ-tɯ-} prefixed form. It is found in collocation with the verb \japhug{βzu}{make} (§\ref{sec:tr.light.verbs}), and optionally with the comitative \forme{cʰo} (§\ref{sec:comitative}), linking two noun phrases referring to the entities undergoing the action together as in (\ref{ex:tWtWrqoR.koBzu}).


\begin{exe}
	\ex \label{ex:tWtWrqoR.koBzu}
	\gll tɤ-wa nɯ kɯ ɯ-tɕɯ cʰo ɯ-me ni ʁnaʁna ʑo tɯ-tɯ-rqoʁ ko-βzu. \\
	\textsc{indef}.\textsc{poss}-father \textsc{dem} \textsc{erg} \textsc{3sg}.\textsc{poss}-son \textsc{comit} \textsc{3sg}.\textsc{poss}-daughter du both \textsc{emph} \textsc{simult}-\textsc{nmlz}:\textsc{action}-hug \textsc{ifr}-make \\
	\glt `The father hugged both his son and his daughter at the same time.' (140427 xiong he mei-zh)
(\japhdoi{0003862\#S21})
 \end{exe}

In this construction, the orientation preverb on \japhug{βzu}{make} is the one that is lexically selected by the verb in simultaneous action nominal form (\textsc{eastwards} in \ref{ex:tWtWrqoR.koBzu}, reflecting the centripetal function of this orientation, §\ref{sec:centripetal.centrifugal}). With orientable verbs (§\ref{sec:orientable.verbs}), the unspecified orientation \forme{ja-}  can be selected (\ref{ex:tWtWCe.jaBzundZi}). 
 
\begin{exe}
\ex \label{ex:tWtWCe.jaBzundZi}
\gll ʑɤni tɯ-tɯ-ɕe ja-βzu-ndʑi \\
\textsc{3du} \textsc{simult}-\textsc{nmlz}:\textsc{action}-go \textsc{aor}:3\flobv{}-make-\textsc{du} \\
\glt `They went at the same time.' (elicited)
 \end{exe}
 
 Example (\ref{ex:tWtWCe.jaBzundZi}) also illustrates the fact that, when the verb in simultaneous action nominal form is intransitive, the shared subject (here \forme{ʑɤni}) is in absolutive form, even though \forme{βzu} is transitive. This type of construction is described in more detail in §\ref{sec:simult.action.nominal.Bzu}.

The first \forme{tɯ-} prefix in the simultaneous construction is probably from the numeral \forme{tɯ-} `one' prefix (§\ref{sec:num.prefixes.1.10}), added to a \forme{tɯ-} action nominal. With intransitive verbs, the dental infinitives in \forme{tɯ-} (with a possessive prefix coreferent with the subject) also occur to express simultaneous actions in collocation with the light verb \japhug{sɯpa}{cause to do}, as shown by (\ref{ex:Wti.cho.WtWCe}) in §\ref{sec:bare.inf.complement}. The action nominal simultaneous construction differs however from the dental/bare infinitive simultaneous construction in that in the former the action of the same verb applies to different subject/objects, while in the latter the same subject performs actions expressed by different verbs.

\subsection{Denominalization of action nominals}  \label{sec:denominalization.action.nominal}
Some \forme{tɯ-} action nominals can serve as base for denominal derivation in \forme{nɯ-} or \forme{rɯ-} (§\ref{sec:denom.rA.pairing}), resulting in a verb with a double derivation \forme{nɯ-tɯ-} or \forme{rɯ-tɯ}. This category is highly heterogeneous, and unlike voice derivations originating from denominal prefixes, has not developed a consistent meaning and grammatical function.

In some cases, the meaning of the (doubly) derived verb is predictable from the base verb, but the two verbs differ in their argument structure. For instance, in \japhug{fɕɤl}{have diarrhea} $\rightarrow$ \japhug{tɯfɕɤl}{diarrhea} $\rightarrow$ \japhug{nɯtɯfɕɤl}{have diarrhea}. In this particular case, the base verb (from Tibetan \tibet{}{bɕal}{diarrhea}) and the doubly derived verb \japhug{nɯtɯfɕɤl}{have diarrhea} are both intransitive verbs, but differ in that the former select a body part as subject (the person being indicated by a possessive prefix), while the latter selects the person suffering the disease.

\begin{exe}
\ex \label{ex:axtu.YWfCAl}
\gll a-xtu ɲɯ-fɕɤl \\
\textsc{1sg}.\textsc{poss}-belly \textsc{sens}-have.diarrhea \\
\ex \label{ex:YWnWtWfCala}
\gll ɲɯ-nɯtɯfɕal-a \\
\textsc{sens}-have.diarrhea-\textsc{1sg} \\
\glt `I have diarrhea.' (elicited)
\end{exe}

Note that a collocation with \japhug{βzu}{make} is also attested with the action nominal \japhug{tɯfɕɤl}{diarrhea} (as in §\ref{sec:action.nominal.Bzu}), but the experiencer is encoded as a possessive prefix on that noun, as \forme{nɯ-tɯ-fɕɤl} in (\ref{ex:nWtWfCAl.YWBze}). Given the parallelism between denominal derivations and noun-light verb collocations (§\ref{sec:Bzu.lv}), is possible that the denominal verb \japhug{nɯtɯfɕɤl}{have diarrhea} came into existence as the verbalized form of this collocation.

\begin{exe}
\ex \label{ex:nWtWfCAl.YWBze}
\gll nɤrŋi nɯra, nɯ-tɯ-fɕɤl ci ɲɯ-βze ŋgrɤl tɕe \\
infant \textsc{dem}:\textsc{pl} \textsc{3pl}.\textsc{poss}-\textsc{nmlz}:\textsc{action}-have.diarrhea \textsc{indef} \textsc{sens}-make[III] be.usually.the.case:\textsc{fact} \textsc{lnk} \\
\glt `Infants, they often suffer from diarrhea.' (17-xCAj) 	(\japhdoi{0003528\#S107})
\end{exe}

In other cases, the doubly derived verb may have a much more restricted meaning than the base verb, especially when the action nominal on which it is based is more lexicalized (§\ref{sec:lexicalized.action.nominals}). 

For instance, the noun \japhug{tɯsqa}{wheat gruel} derived from the transitive verb \japhug{sqa}{cook} is denominalized as the intransitive \japhug{rɯtɯsqa}{have wheat gruel} (§\ref{sec:denom.intr.rA}), whose semantic relationship with the base verb is more remote.

This category is difficult to distinguish from denominal verbs from bare action nominals (§\ref{sec:bare.action.nominals}) taking an indefinite possessor prefix \forme{tɯ-}. For instance, the transitive verb \japhug{tɕʰɯ}{gore, stab, pierce} can be used in a light verb construction in \japhug{lɤt}{throw, release} in bare infinitive form, as in (\ref{ex:WtChW.tolAt}).

\begin{exe}
\ex \label{ex:WtChW.tolAt}
\gll  ɯ-tɕʰɯ to-lɤt \\
\textsc{3sg}.\textsc{poss}-\textsc{bare}.\textsc{inf}:gore \textsc{ifr}-throw \\
\glt `He stabbed him.' (elicited)
\end{exe}

The denominal verb \japhug{nɯtɯtɕʰɯ}{stab} (attested in \ref{ex:tonWtWtChW.pjAsat}) is most probably derived from the indefinite form \forme{tɯ-tɕʰɯ} of the bare infinitive in the construction in (\ref{ex:WtChW.tolAt}), though it is also conceivable that this form comes from the action nominal.

\begin{exe}
\ex \label{ex:tonWtWtChW.pjAsat}
\gll  to-nɯtɯtɕʰɯ tɕe pjɤ-sat \\
\textsc{ifr}-stab \textsc{lnk} \textsc{ifr}-kill \\
\glt `She stabbed and killed him.' (140512 alibaba) (\japhdoi{0003965\#S285})
\end{exe}

\subsection{Lexicalized action nominals}  \label{sec:lexicalized.action.nominals}
Action nominals in \forme{tɯ-} are prone to lexicalization, developing meanings that are impredictable from the base verb.

Action nominals from verbs related to preparation or ingestion of food can become names of specific types of food. For instance, \japhug{tɯsqa}{wheat gruel} and \japhug{tɯtsʰi}{rice gruel} originate from the transitive verbs \japhug{sqa}{cook} and \japhug{tsʰi}{drink}, though the synchronic link between these nouns and the base verbs has ceased to be completely obvious. Other cases of unpredictable meanings are found with \japhug{tɯpu}{moxibustion} (with the verb \japhug{ta}{put} §\ref{sec:ta.lv}, as in \ref{ex:tWpu.kutanW}), which derives from \japhug{pu}{cook} (especially of potatoes in hot ashes, but is not used in reference to moxibustion).

\begin{exe}
\ex \label{ex:tWpu.kutanW}
\gll tɯpu ku-ta-nɯ tɕe, tɤ-pɤtso ɣɯ ɯ-laz ci ku-ta-nɯ, \\
moxibustion \textsc{ipfv}-put-\textsc{pl} \textsc{lnk} \textsc{indef}.\textsc{poss}-child \textsc{gen} \textsc{3sg}.\textsc{poss}-forehead one \textsc{ipfv}-put-\textsc{pl} \\
\glt `(To treat this disease), they apply moxibustion, they apply it to the forehead of the child.' (25-kACAl)	(\japhdoi{0003640\#S64})
\end{exe}

In some cases, the action noun has a highly restricted meaning in comparison with the base verb, limited to one particular sub-meaning. For instance, the noun \japhug{tɯpɣaʁ}{land clearing} (\ref{ex:tWpGaR.lotCAtndZi}) (used in collocation with \japhug{tɕɤt}{take out}, §\ref{sec:tCAt.lv}) derives from the transitive verb \japhug{pɣaʁ}{turn over}, which has the meaning `plough' when occurring with the orientation preverb \textsc{upstream} (see example \ref{ex:nWji.lopGaRnW} in §\ref{sec:antipassive.lexicalized}). This noun lacks the basic meaning of the base verb and its additional extended meanings (such as `go across (a mountain)', like Chinese \ch{翻山}{fānshān}{cross a mountain}). The antipassive \japhug{rɤpɣaʁ}{clear fields} has the same meaning restriction as the action noun, an observation whose significance is developed in §\ref{sec:antipassive.irr.semantic}.

\begin{exe}
\ex \label{ex:tWpGaR.lotCAtndZi}
\gll tɯpɣaʁ lo-tɕɤt-ndʑi \\
field.clearing \textsc{ifr}:\textsc{upstream}-take.out-\textsc{du} \\
\glt `They cleared fields.' (07-deluge) (\japhdoi{0003426\#S109})
\end{exe} 

The divergence in meaning between the action nominal and the base verb can also be due to semantic innovation in the verb. For instance, the verb \forme{rma} means in Japhug `to stay at someone else's place (for a few nights), as in (\ref{ex:CpWnWrmaa}), but the action nominal \japhug{tɯrma}{household} (\ref{ex:tWrma.kondondZi}), suggesting that the original meaning of the verb used to be `live' and became more restricted semantically, while the derived noun preserved its original meaning.

\begin{exe}
\ex \label{ex:CpWnWrmaa}
\gll sɤndzɯn <laoshi> ɯ-ɕki ri kɤ-ari-a tɕe, tɯ-rʑaʁ ɕ-pɯ-nɯ-rma-a \\
\textsc{anthr} teacher \textsc{3sg}.\textsc{poss}-\textsc{dat} \textsc{loc} \textsc{aor}-go[II]-\textsc{1sg} \textsc{lnk} \textsc{one}-night \textsc{tral}-\textsc{aor}-\textsc{auto}-stay.at-\textsc{1sg} \\
\glt `I went to Sandzin's [house], and stayed there for one night.' (conversation 160811)
\end{exe}

\begin{exe}
\ex \label{ex:tWrma.kondondZi}
\gll kʰa ra cʰɤ-fkaβ-ndʑi qʰe tɯrma ko-ndo-ndʑi \\
house \textsc{pl} \textsc{ifr}-cover-\textsc{du} \textsc{lnk} household \textsc{ifr}-take-\textsc{du} \\
\glt `They built a house and established a family.' (02-deluge 2012)
(\japhdoi{0003376\#S121})
\end{exe}

Note the barely translatable use of the action nominals \forme{tɯrma} and \forme{tɯβlɯ} (the latter from the transitive verb \japhug{βlɯ}{burn}) in the expression used as the conclusion of most traditional stories in (\ref{ex:tWrma.tWBlW}).

\begin{exe}
\ex \label{ex:tWrma.tWBlW}
\gll  tɯrma tɯ-βlɯ cʰɤ-nɯ-sɤɲcɣɤɲcɣɤt-nɯ kɤ-ti ɲɯ-ŋu \\
household \textsc{nmlz}:\textsc{action}-burn \textsc{ifr}-\textsc{auto}-cause.to.be.prosperous-\textsc{pl} \textsc{inf}-say \textsc{sens}-be \\
\glt `They lead a prosperous life = they live happily ever after' (many examples)
\end{exe}


There are cases where an alienably possessed noun in \forme{tɯ-} lacks a corresponding base verb; for instance, no verb \forme{*qartsɯ} `kick' is attested besides the noun \japhug{tɯqartsɯ}{kick (n)}, which is used with \japhug{lɤt}{throw, release} as in (\ref{ex:tWqartsW.thalAt}) (note that the \forme{tɯ-} prefix here is neither the indefinite possessor nor the numeral `one' prefixes). The only related verb is the denominal labile verb \japhug{sɯqartsɯ}{kick} (used only for animals, kicking with the rear limbs), which comes from the action nominal \japhug{tɯqartsɯ}{kick} (n) (§\ref{sec:denom.sW.caus.instr}).

\begin{exe}
\ex \label{ex:tWqartsW.thalAt}
\gll  ɯ-rqo nɯtɕu tɯqartsɯ tʰa-lɤt ɲɯ-ŋu \\
\textsc{3sg}.\textsc{poss}-throat \textsc{dem}:\textsc{loc} kick \textsc{aor}:3\flobv{}-throw \textsc{sens}-be \\
\glt `He kicked her in the throat.' (2012 Norbzang) 	(\japhdoi{0003768\#S253})
\end{exe}

In this case, it is most likely that the base verb \forme{*qartsɯ} `kick' did exist at some stage but was lost, only leaving derived words.

\subsection{Inalienably possessed bare action nominals} \label{sec:bare.action.nominals}
Bare action nominals lack any nominalization affix. They are inalienably possessed (§\ref{sec:inalienably.possessed}), and take the indefinite possessor prefixes \forme{tɯ-} or \forme{tɤ\trt}, very similar in function to the \forme{tɯ-} action nominals and in form to the bare infinitives (§\ref{sec:bare.inf}).

They can refer to the action itself, as in \japhug{tɯ-sɯso}{thought} from the transitive verb \japhug{sɯso}{think} or the way an action is performed, as \japhug{ɯ-ti}{way of saying}, `wording', `expression' from \japhug{ti}{say}. They can also be concrete nouns (\tabref{tab:concrete.action.IPN.verbs}). In the case of transitive verb, these nouns refer to an instrument used to perform the action, or resulting from the action (\japhug{tɤ-tsʰoʁ}{nail} from \japhug{tsʰoʁ}{attach}, on which see §\ref{sec:anticausative.dummy}). In the case of intransitive verbs, bare action nominals can refer to an object having a property described by the verb (for instance \japhug{tɤ-ro}{surplus, leftover} from \japhug{ro}{be in surplus, be protruding}). Note the presence of a borrowing from Tibetan among these verbs (\japhug{fkaβ}{cover} from \tibet{བཀབ་}{bkab}{cover (past)}), showing the productivity of this type of derivation.


\begin{table}
\caption{Bare action nominals designating concrete objects} \label{tab:concrete.action.IPN.verbs}
\begin{tabular}{Xlll}
\lsptoprule
Noun & Base verb& \\
\midrule
\japhug{tɤ-ro}{surplus, leftover} & \japhug{ro}{be in surplus, be protruding} \\
\midrule
\japhug{tɤ-fkaβ}{lid} & \japhug{fkaβ}{cover}  \\
\japhug{tɤ-ɕpʰɤt}{patch} (n) & \japhug{ɕpʰɤt}{patch}(vt)\\
\japhug{tɤ-tsʰoʁ}{nail} & \japhug{tsʰoʁ}{attach} (or `plant')  \\
\lspbottomrule
\end{tabular}
\end{table}

For some examples, the semantic relationship between the bare action nominal and the base verb is not transparent anymore; for instance, \japhug{tɯ-ɲɟoʁ}{helper} apparently derives from \japhug{ɲɟoʁ}{glue, paste} (possibly through the sense `(person) attached to oneself').

Some property nouns, which are derived from adjectival stative verbs (§\ref{sec:property.nouns}) without nominalization \forme{x-/ɣ-} prefix (§\ref{sec:G.nmlz}), are also bare action nominals. \tabref{tab:property.nouns.verbs} presents a list of these property nouns and their respective base verbs (among which \japhug{maŋ}{be many} is borrowed from \tibet{མང་}{maŋ}{be many}).

\begin{table}
\caption{Property nouns derived from stative verbs} \label{tab:property.nouns.verbs}
\begin{tabular}{Xlll}
\lsptoprule
Property Noun & Base verb& \\
\midrule
\japhug{tɤ-mbe}{old thing} &  \japhug{mbe}{be old} \\
\japhug{ɯ-do}{old thing} & \japhug{do}{be old (of plants)} \\
\japhug{ɯ-kʰe}{nasty} &  \japhug{kʰe}{be stupid} \\
\japhug{ɯ-maŋ}{in big groups} & \japhug{maŋ}{be many} \\
\lspbottomrule
\end{tabular}
\end{table}

Intransitive verbs in \forme{a-} are correlated with inalienably possessed action nominals with the \forme{tɤ-} indefinite possessor prefix, for instance \japhug{aɕqʰe}{cough} (vi) and \japhug{tɤ-ɕqʰe}{cough} (n). Two hypotheses can be proposed to account for such pairs. 

First, one can argue that the base forms are the verbs, and that the inalienably possessed noun are derived from them; in this view the \forme{a-} element is absorbed the possessive prefixes, and leaves no trace in the possessive paradigm: the \textsc{2sg} and \textsc{2pl} of \japhug{tɤ-ɕqʰe}{cough} are \forme{nɤ-ɕqʰe} `your$_{SG}$ cough' (\ref{ex:nACqhe.WBRAGAZu}) \forme{nɯ-ɕqʰe} `your$_{PL}$ cough', whereas one would have expected these two forms to have become homophonous due to vowel fusion (\textsc{2pl} $\dagger$\forme{nɯ-ɤɕqʰe} realized as \ipa{nɤɕqʰe} like the \textsc{2sg}, §\ref{sec:a.nouns}, §\ref{sec:contraction}).

Second, it is also possible that \japhug{aɕqʰe}{cough} (vi) (and other verbs of the same type) derives from  \japhug{tɤ-ɕqʰe}{cough} by the \forme{a-} denominal prefix (§\ref{sec:denom.a}). 

\begin{exe}
\ex \label{ex:nACqhe.WBRAGAZu}
\gll nɤ-ɕqʰe ɯβrɤ-ɣɤʑu? \\
\textsc{2sg}.\textsc{poss}-cough \textsc{rh}.\textsc{q}-exist:\textsc{sens} \\
\glt `You don't have a cough, do you?' (conversation, 2013-11-12)
\end{exe}

The functional proximity between bare action nominals and \forme{tɯ-} action nominals is illustrated by example (\ref{ex:Wti.WtWsAZirja}), where both types of nominals (from different verbs) appear in parallel contexts.\footnote{This sentence is a metalinguistic comment on the phenomenon described in §\ref{sec:dyads}, examples (\ref{ex:apa.ama}) and (\ref{ex:amu.awa.ni.GW}). }

\begin{exe}
\ex \label{ex:Wti.WtWsAZirja}
\gll  ɯ-ti tɕi mɯ-pjɤ-naχtɕɯɣ, ɯ-tɯ-sɯ-ɤʑirja kɯnɤ mɯ-pjɤ-naχtɕɯɣ ma, \\
\textsc{3sg}.\textsc{poss}-expression also \textsc{neg}-\textsc{pst}.\textsc{ipfv}-be.the.same \textsc{3sg}.\textsc{poss}-\textsc{nmlz}:\textsc{action}-\textsc{caus}-be.aligned also \textsc{neg}-\textsc{pst}.\textsc{ipfv}-be.the.same \textsc{lnk} \\
\glt `The words for [father and mother] were different [between common and honorific register], and the respective orders [in which `mother' and `father' appear] were also different.'  (160706 apa ama)	(\japhdoi{0006211\#S8})
\end{exe}

Given the fact that the only formal difference between the verb and the noun in this formation is the presence of a possessive prefix on the noun, there are cases where the historical relationship between the noun and the verb is ambiguous. The inalienably possessed nouns \japhug{tɤ-rmi}{noun} and \japhug{tɤ-rʑaʁ}{time} synchronically look like bare action nominals derived from the semi-transitive verbs \japhug{rmi}{be called} and \japhug{rʑaʁ}{spend a night}. However, there may be comparative evidence that these verbs are back-formations from the nouns (§\ref{sec:verb.backformation}).

Formally irregular bare action nominals are rare. The inalienably possessed noun \japhug{tɯ-nŋa}{debt} derives from the transitive verb \japhug{ŋa}{buy on credit, owe}, which takes as object the amount of money owed (§\ref{sec:antipassive.history}). The stem of the noun has an additional \forme{n-} prefix, which may reflect a reduced allomorph \forme{*t-} of the \forme{tɯ-} action nominal prefix, with further automatic nasalization to \forme{n-} before nasal consonant \citep{jacques14antipassive}.

Bare action nominals are restricted to a few verbs. However, it is likely that they were more common at an earlier stage, as suggested by the existence of bare infinitives (§\ref{sec:bare.inf}) and by the development of voice derivations from the reanalysis of denominal prefixes (§\ref{sec:voice.denominal}).

In some cases, the direction of derivation between inalienably possessed noun and verb deserves a more detailed discussion. The noun \japhug{tɯ-ŋgo}{disease} could be analyzed as deriving from the verb \japhug{ngo}{be ill}, but it is preferable to suppose that the noun is primary, and that the verb takes an irregular allomorph of the denominal \forme{nɯ-} prefix (§\ref{sec:denom.intr.nW}).


We also find inalienably possessed nouns without a corresponding base verb, but which may originate from bare action nouns of lost base verbs. For instance \japhug{tɤ-re}{laugh (n)} is a noun expressing an action, but whose \forme{tɤ-} is an indefinite possessor prefix and not the action or abstract noun prefix, as shown by (\ref{ex:Wre.CmWG.YAClWG}) where it is replaced by the \textsc{3sg} possessive \forme{ɯ-} prefix. 

\begin{exe}
\ex \label{ex:Wre.CmWG.YAClWG}
\gll  ɯ-re ci ɕmɯɣ ɲɤ-ɕlɯɣ \\
 \textsc{3sg}.\textsc{poss}-laugh \textsc{indef} \textsc{idph}(I):laugh.suddenly \textsc{ifr}-drop \\
\glt  `She giggled despite herself.' (2002 qaCpa)
\end{exe}

The verbs related to  \japhug{tɤ-re}{laugh (n)}, the adjectival stative verb \japhug{sɤre}{be ridiculous} and the labile verb \japhug{nɤre}{laugh} (§\ref{sec:labile.tr-intr}), are both denominal and derive from it. However, comparative evidence suggests that this noun itself derives at an earlier stage from a verb, which was replaced by denominal verbs in Japhug.


\subsection{Action nominal compounds} \label{sec:action.nominal.compounds}
Action nominal compounds, like bare action nominals (§\ref{sec:bare.action.nominals}), lack any nominalization affixes. Unlike other noun-verb compounds such as actor nominal compounds, which are rare and sporadic (§\ref{sec:object.verb.compounds}), action nominal compounds are a well-identified grammatical category. As with other compound nouns, the first element (the nominal root) is nearly\footnote{Vowel alternation does not take place in the case of some very productive constructions, such as that with \japhug{kʰramba}{lie} treated below and in §\ref{sec:compound.action.nominal.Bzu}.} always in bound state form (§\ref{sec:status.constructus}). For instance, the action nominal \japhug{cʰɤtsʰi}{alcohol drinking} is built from the noun \japhug{cʰa}{alcohol} (with regular vowel alternation to \forme{cʰɤ-}) and the transitive verb \japhug{tsʰi}{drink}.  When the incorporated noun is originally inalienably possessed, the indefinite possessor prefix (§\ref{sec:inalienably.possessed.morpho}) is removed, as in \japhug{ɣlɯtɕɤt}{removing dung out of the stable} (to be used as fertilized) from \japhug{tɯ-ɣli}{dung} and \japhug{tɕɤt}{take out}.
 
The nominal element generally corresponds to the object of the verb as in \forme{cʰɤtsʰi} or \forme{ɣlɯtɕɤt} (§\ref{sec:object.verb.compounds}). There are however also cases of goal or adjunct being incorporated, as in \japhug{qʰaru}{look back} from \japhug{ɯ-qʰu}{after, back} (bound state \forme{qʰa-}) with the intransitive verb \japhug{ru}{look at} (§\ref{sec:intr.goal}) and \japhug{kɤtɕʰɯ}{headbutt} from the inalienably possessed noun \japhug{tɯ-ku}{head} (bound state \forme{kɤ-}) and the verb \japhug{tɕʰɯ}{gore} (§\ref{sec:adjunct.verb.compounds}).

These nouns mainly occur with light verbs such as \japhug{lɤt}{release} (§\ref{sec:lAt.lv}) or  \japhug{βzu}{make} (§\ref{sec:Bzu.lv})  as in  (\ref{ex:qharu}), but are also found in other constructions as in (\ref{ex:chAtshi.koGAtChom}), where a free object \japhug{cʰa}{alcohol} with the bare infinitive \forme{ɯ-tsʰi} can also be used (see §\ref{sec:velar.caus.complement}). It is thus likely that the action nominal compounds originate (at least in part) from the coalescence of nouns with bare infinitives.

\begin{exe}
\ex \label{ex:qharu}
\gll ɯʑo nɯ  tatpa ta-ta ma qʰaru mucin ʑo mɯ-pa-lɤt nɤ tɤ-ari ɲɯ-ŋu. \\
\textsc{3sg} \textsc{dem} faith \textsc{aor}:3\flobv{}-put \textsc{lnk} look.back at.all \textsc{emph} \textsc{neg}-\textsc{aor}:3\flobv{}:\textsc{down}-release \textsc{lnk} \textsc{aor}:\textsc{up}-go[II] \textsc{sens}-be \\
\glt `He had faith, did not look back (downwards) at all and [succeeded in] going up to [the abode of the gods]. (Norbzang)
\end{exe}

\begin{exe}
\ex \label{ex:chAtshi.koGAtChom}
 \gll cʰɤtsʰi ko-ɣɤ-tɕʰom tɕe  \\
 alcohol.drinking \textsc{ifr}-\textsc{caus}-be.too.much \textsc{lnk} \\
\glt `He had drunk too much alcohol.' (150829 jidian-zh) 
(\japhdoi{0006338\#S16})
\end{exe}

Compounds with \japhug{rpu}{bump into} or \japhug{tɕʰɯ}{gore} as second element can be built productively with the meaning `hit with $X$' and `stab with $X$' in collocation with \japhug{lɤt}{release}, $X$ corresponding to the first element of the compound. For instance, in (\ref{ex:RzAnrpu}) we find the nonce formation \forme{ʁzɤn-rpu} `hitting with a monastic robe', whose first element \japhug{ʁzɤn}{monastic robe} comes from Tibetan \tibet{གཟན་}{gzan}{monastic robe}, a rather incongruous action which cannot possibly have been lexicalized.
 
\begin{exe}
\ex \label{ex:RzAnrpu}
 \gll ɯ-βɣo nɯ pɯ-ari nɤ, ʁzɤn-rpu ʁɟa ʑo ɕ-ta-lɤt ɲɯ-ŋu. \\
 \textsc{3sg}.\textsc{poss}-FB \textsc{dem} \textsc{aor}:\textsc{down}-go[II] \textsc{add} robe-bump completely \textsc{emph} \textsc{tral}-\textsc{aor}:3\flobv{} \textsc{sens}-be \\
\glt `(As he$_i$ fell down the throat of the giant snake$_j$) his lama$_k$ went down [to the place where the snake was] and hit it$_j$ repeatedly with his$_k$  monastic robe [to force it$_j$ to spit him$_i$ out].' (2003 kandZislama)
\end{exe}

The patient of the hitting action, if overt, is marked with the relator  \japhug{ɯ-taʁ}{on, above} (§\ref{sec:WtaR}) as in (\ref{ex:RrWrpu}), with the compound \japhug{ʁrɯrpu}{hitting with horns} (sideways, not goring) from \japhug{ta-ʁrɯ}{horn}.

\begin{exe}
\ex \label{ex:RrWrpu}
 \gll jla kɯ a-taʁ ʁrɯ-rpu ta-lɤt \\
 hybrid.yak \textsc{erg} \textsc{1sg}-on  horn-bump \textsc{aor}:3\flobv{}-throw \\
 \glt `The hybrid yak hit me with his horn.' (elicited)
\end{exe}
 
In addition, we find action nominal compounds whose incorporated noun expresses the manner of the action, rather than the object, the goal or the instrument as in the previous cases. 

First, the noun \japhug{kʰramba}{lie} can be compounded (without vowel alternation) with transitive or intransive verbs, for instance with the verb \japhug{tsʰi}{drink} as \forme{kʰramba-tsʰi} in (\ref{ex:khramba.tshi}). These nouns, in collocation with \forme{βzu} (§\ref{sec:Bzu.lv}, have the meaning `pretend to $X$'. Note that the compounding with \forme{kʰramba} does not saturate the object position, and has no antipassivization effect: the object \japhug{cʰa}{alcohol} is overt in (\ref{ex:khramba.tshi}).

\begin{exe}
\ex \label{ex:khramba.tshi}
\gll ʑara kɯ [cʰa nɯ kʰramba-tsʰi] ka-βzu-nɯ \\
\textsc{3pl} \textsc{erg} alcohol \textsc{dem} lie-drink \textsc{aor}:3\flobv{}-make-\textsc{pl} \\
\glt `They pretended to drink alcohol.' (2005 Norbzang)
\end{exe}

Second, the lexicalized participle \japhug{kɯzɣa}{a long time} can be compounded with a verb root as \forme{kɯzɣɤ-} (here with vowel alternation). These compounds also occur with light verb \forme{βzu}, and the collocation means `do $X$ for a long time' as in (\ref{ex:kWzGACar}).

\begin{exe}
\ex \label{ex:kWzGACar}
\gll kɯzɣɤ-ɕar ʑo ɲɤ-βzu-nɯ \\
long.time-search \textsc{emph} \textsc{ifr}-make-\textsc{pl} \\
\glt `They searched for it for a long time.' (elicited)
\end{exe}

In these constructions, the verb \japhug{βzu}{make} takes the person indexation of subject and object (§\ref{sec:orientation.raising}), as well as the orientation preverbs selected by the verb in the compound, for instance \textsc{eastwards} like \japhug{tsʰi}{drink} in (\ref{ex:khramba.tshi}) and \textsc{westwards} like \japhug{ɕar}{search} in (\ref{ex:kWzGACar}).  Additional examples of these constructions are discussed in §\ref{sec:compound.action.nominal.Bzu} (see also \citealt[252]{jacques16complementation}).
 
Some action nominal compounds can serve as basis for incorporating denominal verbs. The incorporated object sometimes saturates the object function, and the resulting incorporating verbs are intransitive, as in the case of \japhug{ɣɯcʰɤtsʰi}{drink alcohol} or \japhug{ɣɯɣlɯtɕɤt}{remove dung out of stable} from  \japhug{cʰɤtsʰi}{alcohol drinking} and \japhug{ɣlɯtɕɤt}{removing dung out of the stable}, respectively. In the case of \japhug{ʁrɯrpu}{hitting with horns} however, the resulting verb \japhug{nɯʁrɯrpu}{hit with horns} is transitive, the oblique argument marked with \japhug{ɯ-taʁ}{on, above} in the construction in (\ref{ex:RrWrpu}) being promoted to direct object status.  It is not possible to derive a denominal verb from all of the Noun+\forme{rpu} or Noun+\forme{tɕʰɯ} compounds.
 
A complete list of incorporating denominal verbs and the corresponding action nominal compounds is provided in §\ref{sec:incorporation}. There are also action nominals built by compounding two verb roots (§\ref{sec.v.v.compounds.action}), which serve as the basis for compound verbs by way of denominal derivation (§\ref{sec:denom.compound.verbs}).

\section{Other deverbal nouns} \label{sec:fossil.nmlz}
This section present vestigial nominalization affixes, which despite of their rarity are however important for historical linguistics: the \forme{-z} suffix and the \forme{x/ɣ-} prefix.

\subsection{Nominalization \forme{-z} suffix} \label{sec:z.nmlz}
Japhug has five inalienably possessed nouns derived from verbs by means of a nominalizing \forme{-z}\footnote{The \forme{-z} nominalizing suffix, though rare in Japhug, is of Sino-Tibetan origin. In Situ, the corresponding nominalizing \forme{-s} suffix is much more common (\citealt{jacques03s.houzhui}), and Tibetan and Chinese have traces of a cognate suffix \citep{jacques16ssuffixes}. 
} (\tabref{tab:z.nmlz}), three of which take the indefinite possessor prefix \forme{tɤ-} (§\ref{sec:inalienably.possessed.morpho}).\footnote{Nouns in \forme{-z} such as \japhug{tɤ-rtsɯz}{number} cannot be counted as a Japhug-internal derivation from \japhug{rtsi}{count}: both the noun and the verb come from Tibetan, respectively from \tibet{རྩིས་}{rtsis}{calculation} and  \tibet{རྩི་}{rtsi}{calculate} \citep{hill14derivational}. }
 
\begin{table}
\caption{Traces of the nominalization \forme{-z} suffix in Japhug} \label{tab:z.nmlz}
\begin{tabular}{llll}
\lsptoprule
Noun & Base verb \\
\midrule
\japhug{tɤ-rkuz}{parting present}  & \japhug{rku}{put in} \\
\japhug{tɤ-scoz}{letter, writing} &  \japhug{sco}{see off, accompany} \\
\japhug{ɯ-mɲoz}{preparation} & \japhug{mɲo}{prepare} \\
 \japhug{ɯ-ʁjiz}{wish} & \japhug{ʁjit}{think of} \\
  \japhug{tɤ-rkoz}{special}, `on purpose' & \japhug{rko}{be hard} \\
\lspbottomrule
\end{tabular}
\end{table}
 
The first two nouns \japhug{tɤ-rkuz}{parting present} and \japhug{tɤ-scoz}{letter, writing}\footnote{
This noun is possibly borrowed from Situ (\citealt{jacques03s.houzhui}). }  are object nominalizations. The former  \japhug{tɤ-rkuz}{parting present} is biactantial possessed noun, whose possessor corresponds to the recipient (§\ref{sec:biactantial.ipn}). The etymological relationship  between \forme{tɤ-rkuz} and \japhug{rku}{put in} is obvious when the use of  this verb in the sense of `give as a present to take away' (put in someone's luggage) is considered, as in (\ref{ex:kWki.nArkuz.Nu}). 

\begin{exe}
\ex \label{ex:kWki.nArkuz.Nu}
\gll tɕe tó-wɣ-z-rɤŋgat tɕe, tɕendɤre nɯnɯ kɯ, iɕqʰa nɯ,  tɯ-ci tɯ-tɤ-ste to-rku. tɕe `kɯki nɤ-rkuz ŋu' to-ti. \\
\textsc{lnk} \textsc{ifr}-\textsc{inv}-\textsc{caus}-prepare.to.leave \textsc{lnk} \textsc{lnk} \textsc{dem} \textsc{erg} \textsc{filler} \textsc{dem} \textsc{indef}.\textsc{poss}-water \textsc{one}-\textsc{indef}.\textsc{poss}-bladder \textsc{ifr}-put.in \textsc{lnk} \textsc{dem}.\textsc{prox} \textsc{2sg}.\textsc{poss}-present be:\textsc{fact} \textsc{ifr}-say \\
\glt `He prepared his departure, and gave him a bladder full of water to take with him, and said `this is your departing present'.' (28-smAnmi.txt)
(\japhdoi{0004063\#S247})
\end{exe}


The verb \japhug{rku}{put in} can even occur with its derived noun \japhug{tɤ-rkuz}{parting present} in the \textit{figura etymologica} construction in (\ref{ex:arkuz.tarku}) (the verb \japhug{βzu}{make} can alternatively be used instead of \japhug{rku}{put in}).

\begin{exe}
\ex \label{ex:arkuz.tarku}
\gll a-me kɯ a-rkuz rŋɯl ta-rku \\
\textsc{1sg}.\textsc{poss}-daughter \textsc{erg} \textsc{1sg}.\textsc{poss}-present money \textsc{aor}:3\flobv{}-put.in \\
\glt `My daughter gave me some money (as a present for my departure) (elicited)
\end{exe}

The other nominalizations in \forme{-z} are all abstract nouns. The form \forme{ɯ-ʁjiz}, which derives from the transitive verb \japhug{ʁjit}{think of}, `miss', `remember', results from the simplification of a complex coda \forme{*-ts} to \forme{-z}. This noun only occurs in collocation with motion verbs, and is preceded by a finite or infinitive complement clause (§\ref{sec:nouns.cognition.complement}). The whole construction has the meaning `want to X', where X refers to the content of the complement clause; the experiencer is encoded by the possessive prefix on \forme{ɯ-ʁjiz}, as illustrated by (\ref{ex:aRjiz.mWjGi}).

\begin{exe}
\ex \label{ex:aRjiz.mWjGi}
\gll  tɕe aʑo nɯtɕu kɤ-ɕe a-ʁjiz mɯ́j-ɣi tɕe \\
\textsc{lnk} \textsc{1sg} \textsc{dem}:\textsc{loc} \textsc{inf}-go \textsc{1sg}.\textsc{poss}-want \textsc{neg}:\textsc{sens}-come \textsc{lnk} \\
\glt `I don't want to go there.' (150909 hua pi-zh)
(\japhdoi{0006278\#S20})
\end{exe}

The noun \japhug{ɯ-mɲoz}{preparation} is also only attested in a collocation  (with the verb \japhug{βzu}{make} as in \ref{ex:WmYoz.tuwGBzu}). The transitive verb \japhug{mɲo}{prepare} from which it derives is itself the irregular causative of \japhug{ɲo}{be prepared} (§\ref{sec:causative.m}). A \forme{-z}-less bare action nominal \japhug{ɯ-mɲo}{preparation} (§\ref{sec:bare.action.nominals}) is also attested.

\begin{exe}
\ex \label{ex:WmYoz.tuwGBzu}
\gll  pjɯ-ɲɟo ɕɯŋgɯ tɕe ɯ-mɲoz tú-wɣ-βzu ra \\
\textsc{ipfv}-be.damaged before \textsc{lnk} \textsc{3sg}.\textsc{poss}-preparation \textsc{ipfv}-\textsc{inv}-make be.needed:\textsc{fact} \\
\glt `One has to make preparations before it gets damaged.' (elicited)
\end{exe}
 
The property noun \japhug{ɯ-rkoz}{special} (\ref{ex:Wrkoz.me}), probably derived from \japhug{rko}{hard},\footnote{This etymology is however uncertain, as it would be the only noun of this type from an intransitive verb, and besides the semantic relationship is not entirely transparent.  }  is used adverbially (with the \textsc{3sg} or the indefinite possessor prefix) as  \japhug{tɤ-rkoz}{specially}, `on purpose'. This meaning is however frequently expressed by the borrowing \ch{专门}{zhuānmén}{specially, on purpose}).

\begin{exe}
\ex \label{ex:Wrkoz.me}
\gll ɯ-rmi ɯ-rkoz me. \\
\textsc{3sg}.\textsc{poss}-name \textsc{3sg}.\textsc{poss}-special not.exist:\textsc{fact} \\
\glt `There is no specific name (for this type of kinship relationship).' (140425 kWmdza06)
\end{exe}




\subsection{Nominalization \forme{ɣ-/x-} prefix} \label{sec:G.nmlz}
A handful of nouns, most of them inalienably possessed, are derived from intransitive verbs by means of a velar prefix  \forme{ɣ-} or \forme{x\trt}, harmonizing in voicing with the initial consonant of the stem since the voicing contrast is neutralized in preinitial position (§\ref{sec:xC.clusters}). These nouns are lexicalized ancient subject participles (§\ref{sec:subject.participles}, §\ref{sec:velar.nmlz.history}) which underwent the same phonological change as that observed with the velar animal class prefix (§\ref{sec:velar.class.prefix}), that has a syllabic allomorph \forme{kɯ-} and reduced allomorphs \forme{ɣ-} or \forme{x}-. 

The reduced \forme{ɣ-} / \forme{x-} prefix only derives nouns from intransitive verbs with monosyllabic stems, without consonant clusters. Some of the nouns in \tabref{tab:irregular.nmlz} have cognates in other Rgyalrong languages with reduced prefixes. For instance, \forme{ɣndʑɤβ} has an exact cognate in Tshobdun: \forme{ɣⁿdʒov} `fire' \citep[214]{jackson19tshobdun}. The noun \forme{ɯ-ɣɲɟɯ} has two corresponding forms in Tshobdun: \forme{-ʁⁿɟúʔ} `window'  \citep[609]{jackson19tshobdun} with a reduced uvularized prefix, and \forme{kəⁿɟuʔ} `hole'  \citep[374]{jackson19tshobdun} with a non-reduced prefix.

\begin{table}[H]
\caption{Irregular subject nominalizations in \forme{ɣ-} and \forme{x-} } \label{tab:irregular.nmlz} 
\begin{tabular}{llll}
\lsptoprule
Noun & Base verb & Reference \\
\midrule
\japhug{ɣndʑɤβ}{disastrous fire} & \japhug{ndʑɤβ}{burn} \\
\japhug{ɯ-ɣɲaʁ}{disaster}& \japhug{ɲaʁ}{be black} \\
\japhug{ɯ-ɣɲɟɯ}{orifice} & \japhug{ɲɟɯ}{be opened} \\
\japhug{ɯ-xso}{empty, normal} &\japhug{so}{be empty} &  §\ref{sec:property.nouns} \\
\japhug{ɯ-ɣrom}{dried thing} & \japhug{rom}{be dry} \\
\lspbottomrule
\end{tabular}
\end{table}

The noun \japhug{tɯ-xpa}{one year}, although derived from the verb \japhug{pa}{pass X years} and having an additional \forme{x-} element, does not belong to this category, see  §\ref{sec:num.prefix.paradigm.history} and §\ref{sec:CN.verbs}.

The noun \japhug{tɯ-ɣɲi}{friend, ally} also belongs to this category, though the base verb does not exist in Japhug. It is a near-exact cognate of Tibetan \tibet{གཉེན་}{gɲen}{friend, relative}, a noun derived from the adjective \tibet{ཉེ་}{ɲe}{near}.

\section{Converbs} \label{sec:converbs}
This section discusses several non-finite verb forms which exclusively occur in subordinate clauses other than relative and complement clauses. Other verb forms which might be labeled as converbs, in particular some uses of the velar infinitive, are treated in previous sections (§\ref{sec:inf.converb}).

\subsection{Gerund} \label{sec:gerund}
The gerund is built by prefixing \forme{sɤ-} or \forme{sɤz-} to the verb stem with partial reduplication of the final syllable. It clearly derives from the oblique participle (§\ref{sec:oblique.participle}, §\ref{sec:sigmatic.nmlz.history}), but differs from the latter by the impossibility of adding orientation or possessive prefixes, by the obligatory reduplication, and by the absence of the allomorphs \forme{z-} and \forme{sɤɣ-}. 

Given the fact that the gerund is marked by both the prefix \forme{sɤ-/sɤz-} and the reduplication, I only gloss \textsc{ger} under the prefix, and leave the reduplication unglossed; for instance in (\ref{ex:sArgWrga.jonWCe}) \forme{sɤ-rgɯ\redp{}rga} is glossed \textsc{ger}-be.happy instead of the more explicit but burdensome \textsc{ger}-\textsc{ger}\redp{}be.happy.

\begin{exe}
\ex \label{ex:sArgWrga.jonWCe}
\gll  [sɤ-rgɯ\redp{}rga] kɯ jo-nɯ-ɕe \\
\textsc{ger}-be.happy \textsc{erg} \textsc{ifr}-\textsc{vert}-go \\
\glt `He went back home happy.' (140516 yiguan ganlan-zh) (\japhdoi{0004014\#S134})
\end{exe}

This section reviews the morphology of the gerund (§\ref{sec:gerund.allomorphs}), §\ref{sec:gerund.neg}), then discusses the syntactic properties of gerundive clauses and their meaning (§\ref{sec:gerund.clauses}), and finally presents some cases of lexicalized gerunds (§\ref{sec:gerund.lexicalized}).


\subsubsection{Allomorphy} \label{sec:gerund.allomorphs}
The distribution of the \forme{sɤ-} and \forme{sɤz-} allomorphs is illustrated in \tabref{tab:gerund}. The \forme{sɤz-} allomorph of the gerund prefix is found when the verb stem contains a sonorant initial syllabic prefix (like \forme{nɯ/ɤ\trt}, \forme{rɯ/rɤ\trt}, \forme{ɣɯ-/ɣɤ-} etc), while the \forme{sɤ-} allomorph appears in all other contexts, in particular with monosyllabic stems. Verbs with a stem in \forme{a-} undergo vowel merger \forme{sɤ-ɤ-} to \ipa{sɤ\trt}, as in \forme{sɤmdzɯdzɯ} `sitting' from \japhug{amdzɯ}. The \forme{sɤ-} allomorph can also optionally be used in all contexts.

\begin{table}
\caption{Examples of gerunds} \label{tab:gerund}
\begin{tabular}{llll}
\lsptoprule
Base verb & Gerund \\
\midrule
\japhug{tu}{exist} & \forme{sɤ-tɯ\redp{}tu} \\
\japhug{mu}{fear} & \forme{sɤ-mɯ\redp{}mu} \\
\japhug{rŋgɯ}{lie down} & \forme{sɤ-rŋgɯ\redp{}rŋgɯ} \\
\japhug{amdzɯ}{sit} & \forme{sɤ-ɤmdzɯ\redp{}mdzɯ} \\
\japhug{nɤre}{laugh} & \forme{sɤz-nɤrɯ\redp{}re} \\
\japhug{nɤrte}{wear (head cover)} & \forme{sɤz-nɤrtɯ\redp{}rte} \\
\japhug{ɣɤwu}{cry} & \forme{sɤz-ɣɤwɯ\redp{}wu} \\
\lspbottomrule
\end{tabular}
\end{table}

Verbs whose stem already contains a reduplication are not triplicated. For instance, the gerund of \forme{nɯqambɯmbjom} is \forme{sɤ(z)-nɯqambɯmbjom}, not the impossible form $\dagger$\forme{sɤ(z)-nɯqambɯmbɯmbjom}.

As in other reduplicated forms (§\ref{sec:partial.redp}), partial reduplication in the gerund disregards morpheme boundaries. In (\ref{ex:sAznWGmWGmu.kuXse}), the allomorph \forme{nɯɣ-} of the applicative has a coda \ipa{-ɣ-} (§\ref{sec:caus.sWG}, §\ref{sec:allomorphy.applicative}) which resyllabifies and becomes the preinitial of the next syllable of the verb stem; as such it undergoes partial reduplication, resulting in \forme{sɤznɯ\-ɣmɯ\-ɣmu} rather than the incorrect form $\dagger$\forme{sɤznɯɣmɯmu}.


\begin{exe}
\ex \label{ex:sAznWGmWGmu.kuXse}
\gll tɕe [sɤz-nɯɣ-mɯ\redp{}ɣmu] ʑo ku-χse ɲɯ-ra. \\
 \textsc{lnk} \textsc{ger}-\textsc{appl}-fear \textsc{emph} \textsc{ipfv}-feed[III]  \textsc{sens}-be.needed \\
\glt `It$_i$ has to feed it$_i$  while being afraid of it.$_i$ '  (24-ZmbrWpGa) (\japhdoi{0003628\#S101})
\end{exe}

Some lexicalized gerunds have slightly irregular forms (§\ref{sec:gerund.lexicalized}), but otherwise gerund formation is very regular. However, there are traces of the \forme{sɤɣ/x-} allomorph remaining as alternative gerund forms of some verbs. For instance, the intransitive motion verb \japhug{ɕe}{go}, next to the  regular gerund \forme{sɤɕɯɕe}, also has the form \forme{sɤxɕɯxɕe}. Like the oblique participle \forme{ɯ-sɤx-ɕe} (§\ref{sec:oblique.participle.allomorphy}), the gerund \forme{sɤxɕɯxɕe} was derived with the \forme{sɤx-} allomorph, with resyllabification of the final \forme{-x-} and its inclusion in the partial reduplication of the last syllable, like \forme{sɤznɯ\-ɣmɯ\-ɣmu} above. Since in the case of oblique participles the \forme{sɤɣ-} allomorph is only found with monosyllabic verbs, in the corresponding gerund form the \forme{-ɣ/x-} excrescent element necessarily resyllabifies and undergoes partial reduplication. 

\subsubsection{Polarity prefixes} \label{sec:gerund.neg}
Gerunds cannot bear possessive or orientation preverb, but are attested with the negative prefix \forme{mɤ\trt}, as in (\ref{ex:ndZisWm.mAsACWCe}) with the negative gerund \forme{mɤ-sɤ-ɕɯ\redp{}ɕe} `not going' in collocation with the noun \japhug{tɯ-sɯm}{mind}, here meaning `not willing, unwillingly, reluctantly'.

\begin{exe}
\ex \label{ex:ndZisWm.mAsACWCe}
\gll tɕendɤre tɤ-pi ni kɯ li [ndʑi-sɯm mɤ-sɤ-ɕɯ\redp{}ɕe] ʑo ɲɤ-ta-ndʑi \\
\textsc{lnk} \textsc{indef}.\textsc{poss}-elder.sibling \textsc{du} \textsc{erg} again \textsc{3du}.\textsc{poss}-mind \textsc{neg}-\textsc{ger}-go \textsc{emph} \textsc{ifr}-put-\textsc{du} \\
\glt `The two elder brothers reluctantly left [the ducks] alone.' (140510 fengwang-zh)
(\japhdoi{0003939\#S40})
\end{exe}

The form of the negative prefix is \forme{mɤ-} with all verbs, except for the lexicalized \japhug{masɤrɯrju}{quietly, in secret} (§\ref{sec:gerund.lexicalized}); the form \forme{ma-} is perhaps due to the fact that the base verb \japhug{arju}{speak} has an initial \forme{a-} vowel; however, other verbs in \forme{a-} have the \forme{mɤ-} allomorph, for instance \japhug{mɤ-sɤ-ɤɕqʰɯ\redp{}ɕqʰe}{without coughing} from \japhug{aɕqʰe}{cough}.

 
\subsubsection{Gerundive clauses}  \label{sec:gerund.clauses}
Gerunds are non-finite forms but preserve the verb's argument structure, and gerundive clauses can contain overt intransitive subjects (\ref{ex:Wqom.sAlhWlhoR} and \ref{ex:ndZisWm.mAsACWCe} above) or, very rarely, objects (\ref{ex:Wrte.sAznWtWta}).

\begin{exe}
\ex \label{ex:Wqom.sAlhWlhoR}
\gll tɤɕime nɯnɯ kɯ [ɯ-qom sɤ-ɬɯ\redp{}ɬoʁ] kɯ nɯra tʰɯtʰɤci pɯ-kɯ-fse ra lonba ʑo pjɤ-fɕɤt ɲɯ-ŋu. \\
young.lady \textsc{dem} \textsc{erg} \textsc{3sg}.\textsc{poss}-tear \textsc{ger}-come.out \textsc{erg} \textsc{dem}:\textsc{pl} something \textsc{aor}-\textsc{sbj}:\textsc{pcp}-be.like \textsc{pl} all \textsc{emph} \textsc{ifr}-tell \textsc{sens}-be \\
\glt `The young lady told him everything that had happened while shedding tears.' (140428 mu e guniang-zh)
(\japhdoi{0003880\#S198})
\end{exe}

\begin{exe}
\ex \label{ex:Wrte.sAznWtWta}
\gll [ɯ-rte sɤz-nɯ-tɯ\redp{}ta] jɤ-ari \\
\textsc{3sg}.\textsc{poss}-hat \textsc{ger}-\textsc{auto}-put \textsc{aor}-go[II] \\
\glt `He went away wearing his hat.' (elicited)
\end{exe}

Gerundive clauses are always subordinate to a main finite clause, and express a background action or state occurring at the same time as that referred to by the main verb; they can nearly always be translated either by a gerund in English or by a `while' clause (§\ref{sec:simultaneity}, §\ref{sec:manner.clauses}).

Gerundive clauses, like converbial infinitival clauses (§\ref{sec:inf.converb}), can be followed by the ergative \forme{kɯ} (example \ref{ex:Wqom.sAlhWlhoR} above), the emphatic \forme{ʑo} (\ref{ex:sAznWGmWGmu.kuXse} and \ref{ex:ndZisWm.mAsACWCe}, §\ref{sec:emphatic.Zo}) or both (as in \ref{ex:sAmbWmbGom.YABdenW} below). Bare gerundive clauses are also common, as in (\ref{ex:Wrte.sAznWtWta} and \ref{ex:sAmdzWmdzW.kuzrAZinW} below). No semantic difference between bare gerundive clauses and gerundive clauses followed by \forme{kɯ} or \forme{ʑo} can be brought to light. 

The focus marker \forme{kɯnɤ} can also follow a gerundive clause with the meaning `even $X$ing' as in (\ref{ex:sArNgWrNgW.tWkArnoR}) below.

The (intransitive or transitive) subject of the gerundive clauses is often coreferent with that of the main clause, as in (\ref{ex:Wrte.sAznWtWta}) above and (\ref{ex:sAmbWmbGom.YABdenW}) below.

\begin{exe}
\ex \label{ex:sAmbWmbGom.YABdenW}
\gll tɤɕime ra rca sɤ-mbɯ\redp{}mbɣom ʑo kɯ, iɕqʰa nɤki, rɟɤlpu ɯ-tɕɯ nɯra ɲɤ-βde-nɯ tɕe jo-nɯ-ɕe-nɯ \\
young.lady \textsc{pl} \textsc{unexp}:\textsc{deg} \textsc{ger}-be.in.a.hurry \textsc{emph} \textsc{erg} the.aforementioned \textsc{filler} king \textsc{3sg}.\textsc{poss}-son \textsc{dem}:\textsc{pl} \textsc{ifr}-leave-\textsc{pl} \textsc{lnk} \textsc{ifr}-\textsc{vert}-go-\textsc{pl} \\
\glt `The princesses left the princes in a hurry and went back home.' (140508 shier ge tiaowu de gongzhu-zh) (\japhdoi{0003937\#S156})
\end{exe}

There is however no strict syntactic constraint on coreference between the subject of the gerundive clause and that of the main clauses. Other types of configurations are attested. In (\ref{ex:sAmdzWmdzW.kuzrAZinW}), the intransitive subject of the gerundive clause \forme{tɤ-pɤtso nɯ} `child(ren)' is the object, not the subject, of the main clause verb \forme{ku-z-rɤʑi-nɯ}.\footnote{In this example I assume that \forme{tɤ-pɤtso nɯ} belongs to the gerundive clause, with zero anaphora in the main clause; the opposite analysis could also be considered.}

\begin{exe}
\ex \label{ex:sAmdzWmdzW.kuzrAZinW}
\gll  [tɤ-pɤtso nɯ sɤ-ɤmdzɯ\redp{}mdzɯ] ku-z-rɤʑi-nɯ \\
\textsc{indef}.\textsc{poss}-child \textsc{dem} \textsc{ger}-sit \textsc{ipfv}-\textsc{caus}-stay-\textsc{pl} \\
\glt `They [used to] put the children in sitting position (after having covered them in cloth).' (140426 tApAtso kAnWBdaR 2)
\end{exe}

Coreference between the possessor of the subject in the gerundive clause and the object of the main clause as in (\ref{ex:Wqom.sAlhWlhoR}) and (\ref{ex:ndZisWm.mAsACWCe}) above is also attested. It is particularly common in inalienably possessed noun+intransitive verb collocations where the experiencer is marked as the possessor on the inalienably possessed noun (§\ref{sec:light.verb}), such as \japhug{tɯ-sɯm,ɕe}{want}, \japhug{tɯ-ʁjiz,ɣi}{wish} (§\ref{sec:motion.light.verbs}) or \japhug{tɤ-mbrɯ,ŋgɯ}{get angry} as in (\ref{ex:WmbrW.sANgWNgW}) (§\ref{sec:other.collocation.intr}).

\begin{exe}
\ex \label{ex:WmbrW.sANgWNgW}
\gll ɯ-mbrɯ sɤ-ŋgɯ\redp{}ŋgɯ kɯ ʑo jo-nɯ-ɕe. \\
\textsc{3sg}.\textsc{poss}-anger \textsc{ger}-get.angry \textsc{erg} \textsc{emph} \textsc{ifr}-\textsc{vert}-go \\
\glt `He went back home angry.' (150826 baoliandeng-zh) (\japhdoi{0006370\#S89})
\end{exe}

The opposite configuration, with the possessor in the main clause coreferent with the subject of the gerundive clause, is also attested, as in (\ref{ex:sArNgWrNgW.tWkArnoR}).

\begin{exe}
\ex \label{ex:sArNgWrNgW.tWkArnoR}
\gll  sɤ-rŋgɯ\redp{}rŋgɯ kɯnɤ tɯ-kɤrnoʁ ɲɯ-mtɕɯr tɕe, \\
\textsc{ger}-lie.down also \textsc{genr}.\textsc{poss}-brain \textsc{ipfv}-turn \textsc{lnk} \\
\glt `Even lying down, one feels dizzy [one's head is turning].' (29-tAmtshAzkAkWndo) 
(\japhdoi{0004065\#S55})
\end{exe}

The only example of gerund without apparent coreference between any participant of the gerundive clause and of the main clause in the corpus is (\ref{ex:sArkWrkWn.chWwGtCAt}). However, even here one can interpret the gerundive clause as having a non-overt subject whose possessor would be coreferent with the subject or the object of the main clause.

\begin{exe}
\ex \label{ex:sArkWrkWn.chWwGtCAt}
\gll nɯ sɤ-rkɯ\redp{}rkɯn ʑo tɤ-pɤtso cʰɯ́-wɣ-tɕɤt pjɤ-ra tɕe, \\
\textsc{dem} \textsc{ger}-be.few \textsc{emph} \textsc{indef}.\textsc{poss}-child \textsc{ipfv}-\textsc{inv}-take.out \textsc{ifr}.\textsc{ipfv}-be.needed \textsc{lnk} \\ 
\glt `People had to raise children with few [resources].' (140426 tApAtso kAnWBdaR)
\end{exe}

While gerunds can be built for motion verbs, in the corpus the velar infinitive converbs \forme{kɤ-} (§\ref{sec:inf.converb}) are more common than gerunds to express meanings such as `running' (\forme{kɤ-rɟɯɣ (kɯ ʑo)}), `walking' (\forme{kɤ-ŋke (kɯ ʑo)}) when occurring in a main clause with another motion verb (for instance with \forme{jo-nɯ-ɕe-nɯ} `they went away' in \ref{ex:kArJWG.kW.Zo}). Motion verbs are only attested in non-motional collocations (as in \ref{ex:ndZisWm.mAsACWCe}) above). Gerunds of motion verbs are however possible if the motion is different from the action of the main verb, as in (\ref{ex:sANkWNke.YAsWndza}).

\begin{exe}
\ex \label{ex:sANkWNke.YAsWndza}
\gll sɤ-ŋkɯ\redp{}ŋke ʑo ɲɯ-ɤsɯ-ndza \\
\textsc{ger}-walk \textsc{emph} \textsc{sens}-\textsc{prog}-eat \\
\glt `He is eating it while walking.' (elicited)
\end{exe}

\subsubsection{Lexicalized gerunds}  \label{sec:gerund.lexicalized}
There are a few examples of adverbs from lexicalized gerunds, whose form and meaning is not completely predictable from the base verb.

The gerund \forme{sɤ-xtɕɯ-xtɕi}, from the stative verb \japhug{xtɕi}{be small} means `in childhood, when $X$ was young, since childhood', as in (\ref{ex:sAxtCWtCi.YAme}). Since the verb \japhug{xtɕi}{be small} includes `be young' among its range of meanings, the use of this gerund to refer to young age is not unexpected (`while being young'), but the additional meaning `since  childhood' does not correspond to the usual function of the gerund, which expresses an action or state taking place simultaneously with the action of the main verb.

\begin{exe}
\ex \label{ex:sAxtCWtCi.YAme}
\gll sɤ-xtɕɯ\redp{}xtɕi ʑo ɯ-mu ɯ-wa ɲɤ-me. \\
\textsc{ger}-be.small \textsc{emph} \textsc{3sg}.\textsc{poss}-mother  \textsc{3sg}.\textsc{poss}-father \textsc{ifr}-not.exist \\ 
\glt `He lost his parents when he was young.' (150827 tianluo) (\japhdoi{0006250\#S4})
\end{exe}

The adverb \japhug{masɤrɯrju}{quietly, in secret}  from \japhug{arju}{speak} is formally an ancient negative gerund, with the negative prefix \forme{ma-} rather than \forme{mɤ-}). It originally meant `without speaking', but its meaning has become `in secret, without someone knowing' as in (\ref{ex:masArWrju.pjAlwoR}), and it can even be applied to acts involving speech as in (\ref{ex:masArWrju.khAndWn}), showing that it is not semantically linked to its base verb anymore.

\begin{exe}
\ex \label{ex:masArWrju.pjAlwoR}
\gll  tɕʰemɤpɯ nɯra, nɤkinɯ, mɯ-tɤ-rɯndzaŋspa-nɯ jamar tɕe tɕe,
ɯʑo kɯ masɤrɯrju iɕqʰa cʰa nɯ pjɤ-lwoʁ. \\
girl \textsc{dem}:\textsc{pl} \textsc{filler} \textsc{neg}-\textsc{aor}-pay.attention-\textsc{pl} about \textsc{loc} \textsc{lnk} \textsc{3sg} \textsc{erg} quietly the.aforementioned alcohol \textsc{dem} \textsc{ifr}-spill \\
\glt `While the girls were not paying attention, he poured the alcohol in secret.' (140508 shier ge tiaowu de gongzhu-zh)
(\japhdoi{0003937\#S75})
\end{exe}

\begin{exe}
\ex \label{ex:masArWrju.khAndWn}
\gll ɯʑo kɯ masɤrɯrju kʰɤndɯn nɯ ɲɤ-ndɯn \\
\textsc{3sg} \textsc{erg} quietly mantra \textsc{dem} \textsc{ifr}-recite \\
\glt `He recited the mantra in secret.' (2012 Norbzang)
(\japhdoi{0003768\#S170})
\end{exe}

The adverb \japhug{mɤsɤmdɤla}{in advance} (example \ref{ex:masAmdAla.toNke}), related to the verb \japhug{mda}{arrive (time)}, might also be an ancient negative gerund, but its morphological structure is not completely clear, in particular the element \forme{-la} and the absence of reduplication. 

\begin{exe}
\ex \label{ex:masAmdAla.toNke}
\gll nɤki tɤ-rɟit nɯ mɤsɤmdɤla ʑo to-ŋke \\
\textsc{dem} \textsc{indef}.\textsc{poss}-child \textsc{dem} in.advance \textsc{emph} \textsc{ifr}-walk \\
\glt `This child started walking early.' (elicited)
\end{exe}

\subsubsection{\forme{stɤ-} Gerund} \label{sec:stArJWG}
The Tibetan loan verb \japhug{rɟɯɣ}{run} has a regular gerund \forme{sɤ-rɟɯ\redp{}rɟɯɣ} `running'; however, the adverb \japhug{stɤrɟɯɣ}{running} can also be derived from this verb, with a meaning identical to that of the gerund, as in (\ref{ex:stArJWG}).\footnote{In addition, the velar infinitive converb \forme{kɤ-rɟɯɣ} can be used in the same contexts (\ref{ex:kArJWG.kW.Zo}, §\ref{sec:inf.converb}). } It is the only verb with the prefix \forme{stɤ-}; an exact cognate \forme{stɐrɟəɣʔ} is found in Tshobdun \citep[610]{jackson19tshobdun}.
%some ideophones stɤrjɤt

\begin{exe}
\ex \label{ex:stArJWG}
\gll stɤrɟɯɣ nɤ stɤrɟɯɣ ʑo jo-nɯ-pʰɣo \\
running add running \textsc{emph} \textsc{ifr}-\textsc{vert}-flee \\
\glt `She fled back home running.' (140504 huiguniang-zh)
\end{exe}

 
\subsection{Purposive} \label{sec:purposive.converb}
The purposive converb is used in clauses meaning `in order to',  `for $X$ to $Y$', `so that $X$ does $Y$'. This converb originates from the oblique participle (§\ref{sec:sigmatic.nmlz.history}). It combines the \forme{sɤ-}/\forme{sɤz-} with a reduplicated verb stem like the gerund, but in addition takes a B type orientation preverb preceded by a possessive prefix coreferent with a core argument, as for instance \forme{a-ɲɯ-sɤ-stɯ\redp{}stu} `in order for me to believe in it' from the semi-transitive \japhug{stu}{believe} in (\ref{ex:aYWsAstWstu}). 

\begin{exe}
\ex \label{ex:aYWsAstWstu}
\gll  tɕe nɯnɯ a-ɲɯ-sɤ-stɯ\redp{}stu nɯra tu-nɤme pjɤ-ŋu \\
\textsc{lnk} \textsc{dem} \textsc{1sg}-\textsc{ipfv}-\textsc{purp}-believe \textsc{dem}:\textsc{pl} \textsc{ipfv}-make[III] \textsc{ifr}.\textsc{ipfv}-be \\
\glt `He was doing these things so that I would believe [his predictions].' (150904 yaoshu-zh) (\japhdoi{0006394\#S96})
\end{exe}

Purposive converbs without reduplication are attested, for instance \forme{ɯ-mɤ-pjɯ-sɤ-sɯ-spoʁ} from \japhug{sɯspoʁ}{pierce} (the form with reduplication \forme{ɯ-mɤ-pjɯ-sɤ-sɯ-spɯ-spoʁ} `so that it would not pierce it' is also possible) in \ref{ex:WmApjWsAsWspoR}) or without orientation preverb such as \forme{ɯ-mɤ-sɤ-jmɯ\redp{}jmɯt} `so that he would not forget it' from \japhug{jmɯt}{forget} in (\ref{ex:WmAsAjmWjmWt}) (the complete form with orientation preverb is found in another version of the same story, for instance in \ref{ex:amAYWsAjmWjmWt.nWrkuta}).


\begin{exe}
\ex \label{ex:WmApjWsAsWspoR}
\gll  tɕe nɯ ɯ-pa nɯnɯ li kʰɤxtu nɯnɯ, tɯ-ci, tɯftsaʁ kɯ pjɯ-sɯ-spoʁ ŋgrɤl tɕe, tɕe ɯ-mɤ-pjɯ-sɤ-sɯ-spoʁ, nɯnɯtɕu tɤrɤm kɯ-fse ɲɯ́-wɣ-ta nɯmaʁnɤ cupa kɯ-fse ɲɯ́-wɣ-ta tɕe, \\
\textsc{lnk} \textsc{dem} \textsc{3sg}.\textsc{poss}-under \textsc{dem} again platform \textsc{dem} \textsc{indef}.\textsc{poss}-water leaking.water \textsc{erg} \textsc{ipfv}-\textsc{caus}-have.a.hole be.usually.the.case:\textsc{fact} \textsc{lnk} \textsc{lnk} \textsc{3sg}-\textsc{neg}-\textsc{ipfv}-\textsc{conv}:\textsc{purp}-\textsc{caus}-have.a.hole \textsc{dem}:\textsc{loc}  \textsc{topo} \textsc{sbj}:\textsc{pcp}-be.like \textsc{ipfv}-\textsc{inv}-put otherwise  flat.stone \textsc{sbj}:\textsc{pcp}-be.like \textsc{ipfv}-\textsc{inv}-put \textsc{lnk} \\
\glt `Under the top platform, the water, the leaking water can leak through [the roof], and in order to prevent it from leaking through, people put planks or flat stones there.' (26-tChWra)
(\japhdoi{0003690\#S11})
\end{exe}

\begin{exe}
\ex \label{ex:WmAsAjmWjmWt}
\gll [kɯ-lɤɣ acɤβ nɯ kɯ ɯ-mɤ-sɤ-jmɯ\redp{}jmɯt], ɯ-pʰɯŋgɯ nɯ tɕu rdɤstaʁ-pɯpɯ tɕʰirdu ci ɲɤ-rku, \\
 \textsc{sbj}:\textsc{pcp}-herd Askyabs \textsc{dem} \textsc{erg}  \textsc{3sg}-\textsc{neg}-\textsc{purp}:\textsc{conv}-forget \textsc{3sg}.\textsc{poss}-inside.clothes \textsc{dem} \textsc{loc} stone-little pebble \textsc{indef}
 \textsc{ifr}-put.in\\
\glt `The shepherd Askyabs put a little pebble inside his clothes so that he would not forget [to tell it].' (2002 qaCpa)
\end{exe}

Purposive converbs are most commonly found with the negative prefix \forme{mɤ\trt}, as (\ref{ex:WmApjWsAsWspoR}) and (\ref{ex:WmAsAjmWjmWt}) above. Non-negative purposive converbs, as in (\ref{ex:aYWsAstWstu}), are comparatively much rarer.

With transitive verbs, the possessive prefix can refer either to the subject (with \textsc{1sg} possessive \forme{a-} in \ref{ex:amAYWsAjmWjmWt.nWrkuta}) or the object (\textsc{3sg} possessive \forme{ɯ-} in \ref{ex:WmAtusArpWrpu.pWphaBa}).

\begin{exe}
\ex \label{ex:amAYWsAjmWjmWt.nWrkuta}
\gll jisŋi tɕe tɕendɤre a-mɤ-ɲɯ-sɤ-jmɯ\redp{}jmɯt nɯ nɯ-rku-t-a ŋu \\
today \textsc{lnk} \textsc{lnk} \textsc{1sg}-\textsc{neg}-\textsc{purp}:\textsc{conv}-forget \textsc{dem} \textsc{aor}-put-in-\textsc{pst}:\textsc{tr}-\textsc{1sg} be:\textsc{fact} \\
\glt `Today I put [the pebble in my clothes] so that I would not forget [to tell you].' (2014-kWlAG)
\end{exe}

In example (\ref{ex:WmAtusArpWrpu.pWphaBa}), the purposive form \forme{a-mɤ-tu-sɤ-rpɯ\redp{}rpu} with subject indexation is also possible, without meaning difference.

\begin{exe}
\ex \label{ex:WmAtusArpWrpu.pWphaBa}
\gll kɯm ɲɯ-mbɤr tɕe, a-ku ɯ-mɤ-tu-sɤ-rpɯ\redp{}rpu pɯ-pʰaβ-a \\
door \textsc{sens}-low \textsc{lnk} \textsc{1sg}.\textsc{poss}-head \textsc{3sg}-\textsc{neg}-\textsc{ipfv}-\textsc{conv}:\textsc{purp}-bump \textsc{aor}-lower-\textsc{1sg}\\
\glt `As the door is low, I lowered my head so as not to bump on it.' (elicited)
\end{exe}

Although most examples of purposive converbs have the same subject as the main clause, this is not a syntactic constraint. In (\ref{ex:amAkusAmtsWmstWG}), it is the object of the main clause \japhug{kʰɯna}{dog} that corresponds to the subject of the purposive clause. Furthermore, in (\ref{ex:WmApjWsAsWspoR}) above, the subject and object of the purposive clause are not even arguments of the main clause.

\begin{exe}
\ex \label{ex:amAkusAmtsWmstWG}
\gll  a-mɤ-ku-sɤ-mtsɯ\redp{}mtsɯɣ ɯʑo kɯ kʰɯna ka-βraʁ \\
\textsc{1sg}-\textsc{neg}-\textsc{ipfv}-\textsc{conv}:\textsc{purp}-bite \textsc{3sg} \textsc{erg} \textsc{dog} \textsc{aor}:3\flobv{}-attach \\
\glt `He tied up the dog so that it would not bite me.' (elicited)
\end{exe}

Purposive converbs, although they can be generated for most verbs without difficulty, are very rare in the corpus, and several alternative constructions are preferred to build purposive clauses (§\ref{sec:purposive.clauses}).

\subsection{Immediate} \label{sec:immediate.converb}
The immediate perfective converb expresses that the action in the converbial clause is immediately followed by that in the main clause (`as soon as'). It is built by adding the B type orientation preverb and a \forme{tɯ-} prefix (which may be historically related to the homophonous prefix of action nominals, see §\ref{sec:dental.nmlz.history}) and the verb stem I. It is the only verb form with a type B orientation preverb (which in all other cases occurs with imperfective TAM categories) that has a perfective value.

Since there is a homophonous prefix \forme{tɯ-} for second person (§\ref{sec:intr.23}), the immediate converb is formally identical to the second person singular imperfective form\footnote{More precisely, the \textsc{2sg} form of intransitive verbs and the \textsc{2sg$\rightarrow$3} form of transitive ones. } for all verbs whose stem I and stem III are identical (including all intransitive verbs and some transitive ones, §\ref{sec:stem3.form}); these forms are however easily distinguished for transitive verbs with stem III alternation, as illustrated by \tabref{tab:imm.converb}.


\begin{table}
\caption{Examples of the immediate perfective converb \ipa{tɯ}-} \label{tab:imm.converb}
\begin{tabular}{lllll}
\lsptoprule
  &stem & meaning &\textsc{2sg}(\fl{}3) \textsc{ipfv} & \textsc{imm}  \\
\midrule
intransitive &\forme{sci} & to be born  & \forme{cʰɯ-tɯ-sci} &  \forme{cʰɯ-tɯ-sci}\\
&\forme{ɕe} & to go  & \forme{ju-tɯ-ɕe} & \forme{ju-tɯ-ɕe} \\
\midrule
&\forme{tsʰi} & to drink  & \forme{ku-tɯ-tsʰi} & \forme{ku-tɯ-tsʰi}  \\
transitive &\forme{ndza} & to eat  & \forme{tu-tɯ-ndze} &  \forme{tu-tɯ-ndza}\\
&\forme{mto} & to see & \forme{pjɯ-tɯ-mtɤm} &  \forme{pjɯ-tɯ-mto}\\
\lspbottomrule
\end{tabular}
\end{table}

The immediate converb cannot take any additional prefix, even possessive or negative prefixes.

The converbial clause can contain overt arguments, including  absolutive arguments as in (\ref{ex:YWtWRaR}) or transitive subjects marked with the ergative as in (\ref{ex:pjWtWmto}). 

\begin{exe}
\ex \label{ex:YWtWRaR} 
\gll  ɯ-pɯ ɲɯ-tɯ-ʁaʁ nɤ kumpɣɤtɕɯ jamar ma me	\\
\textsc{3sg}.\textsc{poss}-child \textsc{ipfv}-\textsc{conv}:\textsc{imm}-hatch.out \textsc{lnk} sparrow about apart.from not.exist:\textsc{fact} \\
\glt `Just after its chick has hatched, it is just [as big as] a sparrow.' (24-kWmu) (\japhdoi{0003618\#S85})
\end{exe}

\begin{exe}
\ex \label{ex:pjWtWmto}
\gll tɯrme ra kɯ pjɯ-tɯ-mto ʑo sat-nɯ ɕti.    \\
people \textsc{pl} \textsc{erg} \textsc{ipfv}-\textsc{conv}:\textsc{imm}-see \textsc{emph} kill:\textsc{fact}-\textsc{pl} be.\textsc{aff}:\textsc{fact} \\
\glt  `People kill it as soon as they see it.' (28-qapar)
(\japhdoi{0003720\#S15})
\end{exe}


\begin{exe}
\ex \label{ex:pjWtWBde} 
\gll nɯ pjɯ-tɯ-βde ʑo tɯrme nɯ pjɯ-kɯ-si pjɤ-ŋgrɤl. \\
\textsc{dem} \textsc{ipfv}-\textsc{conv}:\textsc{imm}-throw \textsc{emph} person \textsc{dem} \textsc{ipfv}-\textsc{genr}:S/O-die \textsc{ifr}.\textsc{ipfv}-be.usually.the.case \\
\glt `As soon as one was thrown in there, one would die.' (28-smAnmi) (\japhdoi{0004063\#S154})
\end{exe}

There is often coreference between the arguments of the converbial clause and those of the main clause. In (\ref{ex:YWtWRaR}) and (\ref{ex:pjWtWBde}), the subjects of the main clauses correspond to the intransitive subject and the object of the converbial clause, respectively. In (\ref{ex:pjWtWmto}) above and (\ref{ex:kutWndo}) below, both the subject and the object of the subordinate clauses are coreferent with those of the main clauses. Example (\ref{ex:kutWndo}) also shows that immediate converbs can have non-third person subjects, though this is rare in the corpus.

\begin{exe}
\ex \label{ex:kutWndo} 
\gll tɯtʰɯ ɲɯ-sɤ-ɕke tɕe, aʑo a-jaʁ ku-tɯ-ndo ʑo pɯ-nɯ-ɕlɯɣ-a \\
pot \textsc{sens}-\textsc{prop}-burn \textsc{lnk} \textsc{1sg} \textsc{1sg}.\textsc{poss}-hand \textsc{ipfv}-\textsc{conv}:\textsc{imm}-take \textsc{emph} \textsc{aor}-\textsc{auto}-drop-\textsc{1sg} \\
\glt `As the pot was burning hot, I dropped it as soon as I had grabbed it.' (elicited)
\end{exe}

There is however no strict syntactic constraint on subject or object coreference, as we also find examples where the subject of the converbial clause is not a participant of the main clause, such as (\ref{ex:lutWfsoR}). 

\begin{exe}
\ex \label{ex:lutWfsoR} 
\gll  lu-tɯ-fsoʁ ʑo qʰe tɯ-rɤma tu-ʑe ɲɯ-ŋu. \\
\textsc{ipfv}-\textsc{conv}:\textsc{imm}-be.clear \textsc{emph} \textsc{lnk} \textsc{inf}:\textsc{II}-work \textsc{ipfv}-begin[III] \textsc{sens}-be \\
\glt `It starts working as soon as the day breaks.' (26-GZo)
(\japhdoi{0003668\#S64})
\end{exe}

The converbial clause nearly always followed either by the emphatic \forme{ʑo} (§\ref{sec:emphatic.Zo}) and/or by a linker such as \forme{tɕe}, \forme{qʰe} (\ref{ex:lutWfsoR}) or \forme{nɤ} (\ref{ex:YWtWRaR}).

The immediate converb commonly occurs in the corpus, but it is not the only way to express immediate succession. Several constructions have a very close meaning (§\ref{sec:immediate.subsequence}), involving in particular the postposition \japhug{ɕimɯma}{as soon as} (§\ref{sec:temporal.postpositions}).

\subsection{Adverb from finite verb} \label{sec:nWfse}
The adverb \japhug{nɯfse}{just like that} does not derive from a non-finite verb form, but rather has been lexicalized from the demonstrative \forme{nɯ} (§\ref{sec:anaphoric.demonstrative.pro}) with the \textsc{3sg} form of the Factual Non-Past of \japhug{fse}{be like} in manner clause from a serial verb construction (§\ref{sec:svc.similative.verb}). The range of meanings of \forme{nɯfse} is however not predictable from those of the base verb.

The main meaning of \forme{nɯfse} is `just like that, for no particular reason' as in (\ref{ex:nWfse.pjAsi}). 
%\textsc{1pl} \textsc{lnk} \textsc{aor}-feed-\textsc{1pl} \textsc{dem} 
\begin{exe}
\ex \label{ex:nWfse.pjAsi}
\gll  tɯ-rdoʁ tɕʰi mɯ-pɯ-nnɯ-pe mɤ-xsi ma nɯnɯ nɯfse pjɤ-si \\
one-piece what \textsc{neg}-\textsc{pst}.\textsc{ipfv}-\textsc{auto}-be.good \textsc{neg}-\textsc{genr}:know \textsc{lnk} \textsc{dem} like.that \textsc{ifr}-die \\
\glt `One [of them] died just like that, I don't know what went wrong.' (140510 wugui) 
(\japhdoi{0003951\#S51})
\end{exe}  

It can also mean `for nothing, in vain', in particular in combination with a verb prefixed with the autive prefix (§\ref{sec:autoben.spontaneous}),  as in (\ref{ex:nWfse.pjWnWNgra}).

\begin{exe}
\ex \label{ex:nWfse.pjWnWNgra}
\gll nɯnɯ ɯ-mat nɯfse pjɯ-nɯ-ŋgra ɲɯ-ɕti ma ɯ-rɣi mɯ́j-nɤjtsʰɯ \\
\textsc{dem} \textsc{3sg}.\textsc{poss}-fruit like.that \textsc{ipfv}-\textsc{auto}-\textsc{acaus}:cause.to.fall \textsc{sens}-be.\textsc{aff} \textsc{lnk} \textsc{3sg}.\textsc{poss}-seed \textsc{neg}:\textsc{sens}-be.useful \\
\glt `Its fruits fall [to the ground] just like that, it is not useful as seed.' (08-CkrAz)
(\japhdoi{0003444\#S47})
\end{exe}  

%nɯfse, antonym of tɤrkoz
%aʑo nɯ fse tʰɯ-βde-t-a ɕti ma tɤrkoz pɯ-maʁ
%\japhug{tɤ-rkoz}{specially}, `on purpose'{sec:z.nmlz}
%140512_fushang_he_yaomo, 19

%nɯ fse taqaβ nɯ maʁ tɕe, 
\section{Defective verbs} \label{sec:nmlz.defective}
Nearly all Japhug verbs have participles, infinitives and the other non-finite forms described in this chapter. There is however a handful of defective verbs lacking non-finite categories. 

The sensory existential verbs \japhug{ɣɤʑu}{exist} and \japhug{maŋe}{not exist}, which present other types of irregularities (§\ref{sec:intr.person.irregular}), lack all non-finite forms; the nominalized forms of the existential verbs \japhug{tu}{exist} and \japhug{me}{not exist} are used instead.

The transitive verb \japhug{kɤtɯpa}{tell} cannot be prefixed (§\ref{sec:irregular.transitive}), and therefore lacks all non-finite forms.

The verb \japhug{mɤ-xsi}{one does not know} is only found in the generic negative Factual Non-Past (§\ref{sec:irregular.transitive}), and the corresponding non-finite forms are provided by the verb \japhug{sɯz}{know}.

\section{Historical perspectives} \label{sec:nmlz.historical.perspectives}
The great majority of non-finite verb forms studied in this chapter take either a velar, a sigmatic or a dental prefix. While it is necessary to distinguish many sub-categories from a synchronic point, it is equally obvious that a diachronic relationship exists between some of these non-finite forms. While a complete account of the history of nominalized forms in Japhug will have to wait a proper reconstruction of proto-Gyalrongic, it is nevertheless possible to offer some preliminary thoughts on the relationship between these morphological categories.

\subsection{Velar non-finite prefixes} \label{sec:velar.nmlz.history}
Many Trans-Himalayan languages, including Karbi and Kiranti, have productive velar nominalization prefixes \citep{konnerth16gV}, and traces of such prefixes can also be found in other languages such as Tibetan \citep{jacques14snom}. These forms are very probably historically related to the velar nominalization prefixes found in Gyalrong languages, but the present chapter focuses on Gyalrong-internal evidence.

In Japhug, non-finite verb forms with velar prefixes include the following ones:

\begin{itemize}
\item \forme{kɯ-} subject participles (§\ref{sec:subject.participles})
\item \forme{kɤ-} object participles (§\ref{sec:object.participle})
\item \forme{kɯ-} and \forme{kɤ-} velar infinitives (§\ref{sec:velar.inf})
\item Infinitival converbs (§\ref{sec:inf.converb})
\item \forme{x-/ɣ-} deverbal nouns (§\ref{sec:G.nmlz})
\end{itemize}

In addition, three finite prefixes are likely to be related to these forms: the 2\fl{}1 \forme{kɯ-} prefix (§\ref{sec:indexation.local}, §\ref{sec:portmanteau.prefixes.history}), the generic S/O \forme{kɯ-} prefix (§\ref{sec:indexation.generic.tr}), and the circumfix \forme{kɯ-...-ci} occurring in several morphological contexts (§\ref{sec:peg.circumfix}).

Given the difficulty of distinguishing infinitives from participles even synchronically (§\ref{sec:infinitives.participles}), it is quite obvious that these forms are ultimately related, but accounting for the precise distribution of the forms is not trivial.

In addition to the categories mentioned above, shared by all core Gyalrong languages, Situ has a type of semi-finite participles in \forme{kə\trt}, used in various types of relative clauses \citep{jacksonlin07}. In (\ref{ex:nEkEmaSentS}) for instance, the form \forme{nǝ-kǝ-maʃe-ntʃ} has both a participle \forme{kə-} and a dual suffix \forme{-ntʃ}; in Japhug, such combination is impossible: indexation suffixes (and the second person prefix \forme{tɯ-}) are incompatible with all non-finite forms studied in this chapter. 

\begin{exe}
\ex \label{ex:nEkEmaSentS} 
\gll  ndzok kǝ-nǝwɐtjô=ndʒês=tǝ ptʂêrǝ dʒɐspɐ̂ nǝ-kǝ-maʃe-ntʃ nǝ-ŋos \\
slightly \textsc{nmlz}-be.hardworking[II]=\textsc{du}=\textsc{top} then quite \textsc{aor}-\textsc{nmlz}-be.rich[II]-\textsc{du} \textsc{sens}-be[I] \\
\glt `The two hardworking ones then became rather rich.' \citep[193--194]{lin09phd}
\end{exe}

In the following, I assume that in proto-Gyalrong both semi-finite participles of the type exemplified in (\ref{ex:nEkEmaSentS}) and subject participles as in Japhug (§\ref{sec:subject.participles}) did exist, and were marked by the ancestor of the \forme{kɯ-} prefix.

The object participle in \forme{kɤ\trt}, which is synchronically still homophonous with the subject participle of a passivized transitive verb (§\ref{sec:object.participle.ambiguity}), likely originates from the fusion of the \forme{*kə-} participle with the passive \forme{*ŋa-} prefix (see also \citealt{jackson06guanxiju, jacksonlin07} for a suggestion in the same lines).

Since only transitive verbs can have a passive form, the use of the object for semi-transitive verbs (for instance example \ref{ex:stu.jikArga} in §\ref{sec:object.participle.relatives}) must have been an analogical extension occurring after the merger of the participle prefix and the passive was complete. 

The \forme{kɤ-} infinitive possible derives from the object participles. The pivot construction where such a reanalysis could have taken place is the object of verbs of perception such as \japhug{mto}{see}. When an object participial clause occurs as the object of a verb of perception, while in some case there is no ambiguity (as in \ref{ex:pWkAsAt.nW.kAmto}), in examples such as (\ref{ex:kAntsGe.nW.mWpWmtota}) it is possible to analyze the \forme{kɤ-} prefixed form either as an object participle (entailing a translation `grains that are sold like that') or as an infinitive (`selling grains like that', referring to the whole action rather than the object).

\begin{exe}
\ex \label{ex:pWkAsAt.nW.kAmto} 
\gll ma [pɯ-kɤ-sat] nɯ kɤ-mto nɯ pɯ-rɲo-t-a. \\
\textsc{lnk} \textsc{aor}-\textsc{obj}:\textsc{pcp}-kill \textsc{dem} \textsc{inf}-see \textsc{dem} \textsc{aor}-experience-\textsc{tr}:\textsc{pst}-\textsc{1sg} \\
\glt `I have seen killed ones.' (22-pGAkhW)
(\japhdoi{0003594\#S20})
\end{exe}

\begin{exe}
\ex \label{ex:kAntsGe.nW.mWpWmtota} 
\gll tɕe [ɯ-rdoʁ nɯ kɯ-fse kɤ-ntsɣe] nɯ mɯ-pɯ-mto-t-a \\
\textsc{lnk} \textsc{3sg}.\textsc{poss}-grain \textsc{dem} \textsc{sbj}:\textsc{pcp}-be.like \textsc{obj}:\textsc{pcp}/\textsc{inf}-sell \textsc{dem} \textsc{neg}-\textsc{aor}-see-\textsc{tr}:\textsc{pst}-\textsc{1sg} \\
\glt `I have not seen its grains sold like that (i.e. unprocessed).' (09-mi) (\japhdoi{0003466\#S51})
\end{exe}

After such reanalysis took, the \forme{kɤ-} infinitive, originally restricted to transitive verbs, was extended to dynamic intransitive verbs. 

The nominalization \forme{x/ɣ-} prefix is probably the result of the application of a sound law of presyllable reduction on monosyllables without initial cluster (§\ref{sec:velar.class.prefix}, §\ref{sec:G.nmlz}) to the subject participle prefix. The regular \forme{kɯ-} subject participle on monosyllabic clusterless verb stems is due to the analogical generalization of the \forme{kɯ-} allomorph to all verb forms.

The generic \forme{kɯ-} and the 2\fl{}1 \forme{kɯ-} prefixes, which occur in finite verb forms in Japhug, are less likely to come from subject participles. Instead, as argued in \citet{jacques18generic} and §\ref{sec:portmanteau.prefixes.history}, they are traces of the semi-finite participles in Japhug and other Northern Gyalrong languages.

\figref{fig:velar.nmlz.history} summarizes the pathways of reanalysis proposed in this section and in §\ref{sec:portmanteau.prefixes.history}. The semi-finite participle and subject participle forms are without doubt historically related, but both have to be reconstructed at the proto-Gyalrong level and the function of the original prefix from which they derive is unclear -- it is conceivable that the original prefix was completely non-finite like the Japhug subject participle, and that the semi-finite participle is an innovation (on the addition of person indexation markers on non-verbal predicative words, see §\ref{sec:portmanteau.prefixes.history}), but the opposite is equally possible.

   \begin{figure}[H]
   \caption{Development of velar non-finite forms from proto-Gyalrong to Japhug} \label{fig:velar.nmlz.history}  
  \begin{tikzpicture}
  \node (X) at (2,2) {???};  
  \node (A) at (4,0) {Semi-finite participle \forme{*kə-}};
  \node (B) at (4,-2) {Indefinite?};  
  \node (C) at (6,-4) {2\fl{}1 \forme{kɯ-}};
  \node (D) at (2,-4) {generic S/O \forme{kɯ-}};  
  \node (A') at (0,0) {Subject participle \forme{*kə-}};
  \node (B') at (-1,-2) {Object participle \forme{kɤ-}};  
  \node (C') at (-1,-4) {Infinitive \forme{kɤ-}};  
  \node (D') at (-5,-4) {\forme{x/ɣ-} nominalization};  
\tikzstyle{peutetre}=[->,dotted,very thick,>=latex]
\tikzstyle{sur}=[->,very thick,>=latex]
\draw[peutetre] (X)--(A);
\draw[peutetre] (X)--(A');
\draw[peutetre] (A)--(B);
\draw[peutetre] (B)--(C);
\draw[peutetre] (B)--(D);
\draw[peutetre] (A') to node [left] {+ Passive \forme{a-}} (B');
\draw[peutetre] (B') -- (C');
\draw[peutetre] (A') to[bend right] (D');
\end{tikzpicture}
\end{figure}


\subsection{Sigmatic non-finite prefixes} \label{sec:sigmatic.nmlz.history}
Nominalized forms involving a coronal fricative prefix, or a trace thereof, have been described in Old Chinese (\citealt[73]{sagart99roc}, \citealt[56]{bs14oc}), Tibetan (\citealt{jacques18oc-nmlz}) and Jinghpo (\citealt[3--4]{dai92yufa}). 


In Japhug, non-finite verb forms with dental fricative prefixes are very widespread, and include the following ones:

\begin{itemize}
\item Oblique participle (§\ref{sec:oblique.participle})
\item Gerund (§\ref{sec:gerund})
\item Purposive (§\ref{sec:purposive.converb})
\end{itemize}

It is quite obvious that both gerund and purposive converbs derive from the oblique participle. The reduplication found in these forms (optional in the case of the purposive, see \ref{ex:WmApjWsAsWspoR} in §\ref{sec:purposive.converb}) presumably reflects the emphatic reduplication (§\ref{sec:emph.redp}) that can be applied to nearly all verb forms.


While derivation of gerunds from oblique participle is not a problem from a formal point of view, the exact pathway of reanalysis deserves some discussion. Given the fact that oblique participles are used to build temporal relatives `the time when...' (§\ref{sec:other.oblique.participle.relatives}), it is tempting to suppose that the gerund derives from this function, in absolutive locative form (§\ref{absolutive.locative}). In this view, a gerund like \forme{sɤ-ɤmdzɯ\redp{}mdzɯ} `sitting' would come from an original construction meaning `at the time when $X$ was sitting'. However, this hypothesis is difficult, because of the very restricted nature of participial temporal relatives, which are only found for very specific time periods that belong to common knowledge (for instance \forme{ɯ-sɤ-ji} `the period when it is planted' in example \ref{ex:WsAji} in §\ref{sec:other.oblique.participle.relatives}), and are not compatible with most types of temporal clauses.

Other possibilities to explain the origin of the gerund are the locative (§\ref{sec:locative.participle.relatives}) and instrumental (§\ref{sec:instrumental.participle.relatives}) uses of the oblique participles. Locative participles could account for the use of gerund with psychological verbs; for instance \forme{sɤ-mɯ\redp{}mu} `fearing' from \japhug{mu}{fear} could originate from a metaphorical locative similar to English `in fear'. However, the hypothesis that gerunds derive from instrumental oblique participles is more probable due to the optional presence of the ergative \forme{kɯ} with the gerunds (§\ref{sec:gerund.clauses}) while locative postpositions are never found in this context. The gerunds thus probably originated from subordinate clauses expressing reason or cause, from which they came to express manner or simultaneous action (`by sitting' $\Rightarrow$ `sitting').

The purposive converbs probably also come from the instrumental use of oblique participles, but through a different pathway. As suggested in \citet[272]{jacques14linking}, examples such as (\ref{ex:tWthW.sAXtCi}) may constitute the pivot construction between instrumental participial relatives and purposive converbs.


\begin{exe}
\ex \label{ex:tWthW.sAXtCi}
\gll ɣzɯtʰɯz nɯ kɯɕɯŋgɯ tɕe [tɯtʰɯ sɤ-χtɕi] ɲɯ́-wɣ-nɯ-pʰɯt pɯ-ŋgrɤl \\
Selaginella \textsc{dem} in.the.past \textsc{lnk} pan \textsc{obl}:\textsc{pcp}-wash \textsc{ipfv}-\textsc{inv}-\textsc{auto}-uproot \textsc{pst}.\textsc{ipfv}-be.usually.the.case \\
\glt `In the past, people would dig up \textit{Selaginella} [to use as] a pan cleaner.' (16-RlWmsWsi)
(\japhdoi{0003520\#S102})
\end{exe}

In (\ref{ex:tWthW.sAXtCi}), the participial relative clause \forme{tɯtʰɯ sɤ-χtɕi} `pan cleaner' is an essive adjunct (in absolutive form, §\ref{sec:essive.abs}) which can be translated as `(to use) as a pan cleaner'. This meaning is very close to that of a purposive clause `in order to clean pans'. An instrumental participial relative clause in essive function could thus easily be reanalyzed as a purposive clause, and hence the oblique participle as a purposive converbal.


\subsection{Dental non-finite prefixes} \label{sec:dental.nmlz.history}
Non-finite verbal forms taking a dental stop prefix \forme{tɯ-} or \forme{tɤ-} in Japhug include the following ones:

\begin{itemize}
\item Dental infinitive (§\ref{sec:dental.inf})
\item Degree nominals (§\ref{sec:degree.nominals})
\item \forme{tɯ-} action nominals (§\ref{sec:action.nominals})
\item \forme{tɤ-} abstract nouns (§\ref{sec:tA.abstract.nouns})
\item Simultaneous action nominal (§\ref{sec:simultaneous.action.nominal})
\item Immediate perfective converb (§\ref{sec:immediate.converb})
\end{itemize}


It is possible that all of these forms are historically related. However, given the fact that some of the non-finite verb forms in Japhug are inalienably possessed nouns (§\ref{sec:inalienably.possessed}) derived by adding a possessive prefix to the bare verb stem (bare infinitives §\ref{sec:bare.inf} and bare action nominals §\ref{sec:bare.action.nominals}), it is likely that at least some of the categories listed above originate from the indefinite possessive form of a bare infinitive or a bare action nominal, as suggested in \citet[236]{jacques16complementation} concerning the bare infinitives. 

Since bare infinitives and dental infinitives are in complementary distribution, the former being used with transitive verbs and the latter with intransitive ones (see §\ref{sec:bare.inf.complement} and §\ref{sec:dental.inf.complement}), it is legitimate to consider the possibility that both forms go back to a single category.  The bare infinitive takes a possessive prefix that is coreferent with the object of the verb; intransitive verbs lack an object (semi-transitive verbs have a semi-object §\ref{sec:semi.object}, but its morphosyntactic properties are different from those of canonical objects), and thus if a bare infinitive were built from an intransitive verb, one would only have three choices: (i) index the subject, (ii) use the bare stem with any possessive prefix or (iii) use a `dummy' possessive prefix indicating the absence of object. It is possible to argue that dental infinitives correspond to solution (iii), and that the \forme{tɯ-} prefix in this form is the indefinite possessor \forme{tɯ-} (§\ref{sec:indef.genr.poss}). In this hypothesis, dental infinitives taking a possessive prefix (see example \ref{ex:tWtAmdzW.kWBdi} in §\ref{sec:dental.inf.complement}) are later creations, made after the etymological origin of the \forme{tɯ-} had become obscured.

Proposing that \forme{tɯ-} action nominals and \forme{tɤ-} abstract nouns come from alienabilized bare action nominals is not to be excluded, but appears to be less compelling, since there is no complementary distribution between the former and the latter, unlike in the case of bare vs. dental infinitives.

It is also conceivable that dental infinitives come from \forme{tɯ-} action nominals, and bare infinitives from bare action nominals, respectively. In the absence of evidence from languages other than Japhug, I leave this issue unresolved.

Regardless of the origin of action nominals, it is clear that degree nominals on the one hand, and simultaneous action nominals on the other hand, derive from them by prefixing possessive prefixes (in degree nominals) and the numeral \forme{tɯ-} prefix, respectively (in simultaneous action nominal, §\ref{sec:simultaneous.action.nominal}).

The origin of the immediate perfective converbs is quite puzzling. While immediate converb take an obligatory type B orientation preverb, no other \forme{tɯ-} non-finite form is compatible with orientation preverbs, and moreover type B prefixes normally occur with imperfective TAME categories (§\ref{sec:kamnyu.preverbs}, §\ref{sec:preverb.TAME}). It is conceivable that these converbs ultimately originate from action nominals, but the pathway of morphological evolution that has lead to their creation is unclear.

\figref{fig:dental.nmlz.history} summarizes the hypotheses presented in this section; the dotted arrows represent uncertain derivations.

   \begin{figure}[H]
   \caption{Several hypotheses to account for the historical origin of dental non-finite forms in Japhug} \label{fig:dental.nmlz.history}  
  \begin{tikzpicture}
  \node (X) at (0,2) {Bare action nominal};
  \node (A) at (-4,0) {Action nominal \forme{tɯ-}};
  \node (A') at (0,0) {Abstract noun \forme{tɤ-}};
  \node (B) at (4,0) {Bare infinitive};
  \node (C) at (4,-2) {Dental infinitive \forme{tɯ-}};
  \node (D) at (0,-3) {Degree nominal};
  \node (E) at (-4,-3) {Simultaneous action nominal};
  \node (F) at (4,-4) {?Immediate perfective converb};
\tikzstyle{peutetre}=[->,dotted,very thick,>=latex]
\tikzstyle{sur}=[->,very thick,>=latex]
\draw[peutetre] (X)--(A);
\draw[peutetre] (X)--(A');
\draw[sur] (X)--(B);
\draw[peutetre] (B) --(C);
\draw[peutetre] (A) -- (C);
\draw[sur] (A) to node [right] {+ Possessive} (D);
\draw[sur] (A) to node [left] {+ Numeral \forme{tɯ-}} (E);
\end{tikzpicture}
\end{figure}

