\chapter{Numerals and counted nouns}  \label{chap:numerals}

\section{Plain numerals} \label{sec:plain.numerals}
This section describes the morphology of cardinal and ordinal numerals in Japhug. Unlike some other languages of the Sino-Tibetan family, which have vigesimal features or substractive numerals \citep{mazaudon02nombre}, Japhug has a strict decimal system.

In addition to plain numerals, Japhug has a system of numeral prefixes (§\ref{sec:numeral.prefixes}) which occur on a specific type of nouns, \textit{counted nouns}, discussed in §\ref{sec:counted.nouns}.

\subsection{Numerals 1-10}  \label{sec:one.to.ten}
The basic numerals from one to ten are indicated in \tabref{tab:numerals.under.10}. The corresponding numeral prefixes are discussed in §\ref{sec:num.prefixes.1.10}.

Some dialects of Japhug other than the Kamnyu variety use \japhug{tɤɣ}{one} instead. In calculations (see §\ref{sec:arithmetic}), the generic counted noun \japhug{tɯ-rdoʁ}{one piece} (§\ref{sec:CN.quantifier}) with the numeral prefix `one' (§\ref{sec:numeral.prefixes}) 
is used instead of \japhug{ci}{one} to express the number `one'.

Apart from \japhug{ci}{one} and \japhug{sqi}{ten}, these numerals have clear cognates in languages outside of the Gyalrongic group, even in Tibetan and Chinese; \tabref{tab:numerals.under.10} includes the Tibetan equivalent of these numerals (the numerals that are \textit{not} cognate with their Japhug equivalents are indicated between brackets) and the Japhug pronunciation of these Tibetan words in borrowed words (see §\ref{sec:tibetan.numerals}, in particular \tabref{tab:tibetan.ordinals}).

The numerals from 2 to 9 have a prefix, uvular \forme{χ-/ʁ-} in `two' and `three' and velar \forme{kɯ-} from `four' to `nine'. These prefixes do not appear in some derived forms such as the numbers 11-19 (\tabref{sec:teens} below) or approximate numerals (§\ref{sec:approx.numerals}).

The numeral \japhug{ʁnɯz}{two} is etymologically related to the dual postnominal determiner \forme{ni} (§\ref{sec:dual.determiners}). The latter lacks the uvular prefix and the sibilant suffix, but the vowel difference is expected, as after adding the suffix \ipa{*-s}, the proto-Gyalrong rhyme \forme{*-is} regularly yields Japhug \forme{-ɯz}. The adverb \japhug{ʁnaʁna}{both} (§\ref{sec:dual.determiners}) is also probably related, though its morphological relationship with \japhug{ʁnɯz}{two} does not fit any known pattern.

The superficial resemblance between \japhug{ʁnɯz}{two} and \japhug{kɯɕnɯz}{seven} could suggest the existence of a former quinary system. However, the fact that these two numerals have different vowels in Situ (Cogtse \forme{kəniɛ̄s} `two' and \forme{kəɕnə̄s} `seven', from \citealt{huangsun02}) makes this assumption less likely; it is in any case irrelevant to the synchronic grammar of Japhug.

Unlike the other numerals in \tabref{tab:numerals.under.10},  the free numeral \japhug{χsɯm}{three} is identical to the form found in compound loans such as  \japhug{kɯmtɕʰoχsɯm}{triratna} from \tibet{དཀོན་མཆོག་གསུམ་}{dkon.mtɕʰog.gsum}{triratna}. Therefore, one cannot exclude the possibility that it is a borrowing from Tibetan \tibet{གསུམ་}{gsum}{three} that replaced the native numeral `three' (which was cognate to the Tibetan form, and phonetically similar to it). In this hypothesis, the alternative forms \forme{-fsum} and \forme{fsɯ-} for `three' found in the numerals 11-19 (§\ref{sec:teens}) and tens  (§\ref{sec:decades}) could be remnants of the native numeral. Alternatively, it is possible that the native word and the borrowing are true cognates, and happen to have the same form by coincidence.

The numeral \japhug{kɯngɯt}{nine} has a coda \forme{-t} which is not found in the cognates of this numeral in Situ and languages outside of Gyalrongic, suggesting analogical spreading of the coda from \japhug{kɯrcat}{eight}.
This innovation is shared by all Northern Gyalrong languages: Tshobdun has \forme{kə́nⁿgət} \citep{jackson19tshobdun} and Zbu \forme{kənⁿɡə́t} \citep[130]{gong18these}. The same analogy independently occurred in the Siyuewu dialect of Khroskyabs, where  `nine' is \forme{ŋgə́d} (\citealt[174]{lai17khroskyabs}).

\begin{table}
\caption{Basic numerals in Japhug and Tibetan}  \label{tab:numerals.under.10} \centering
\begin{tabular}{lllllll}
\lsptoprule
& Native Japhug & Tibetan &Tibetan loanwords in Japhug  \\
\midrule
1	&	\forme{ci} or \forme{tɤɣ} & \tibet{གཅིག་}{gtɕig}{one} & \forme{χtɕɯɣ} \\
2	&	\forme{ʁnɯz}  & \tibet{གཉིས་}{gɲis}{two} & \forme{ʁɲiz} \\
3	&	\forme{χsɯm}  & \tibet{གསུམ་}{gsum}{three} & \forme{χsɯm} \\
4	&	\forme{kɯβde} & \tibet{བཞི་}{bʑi}{four} & \forme{βʑi} \\ 
5	&	\forme{kɯmŋu}  & \tibet{ལྔ་}{lŋa}{five} & \forme{rŋa} \\
6	&	\forme{kɯtʂɤɣ}  & \tibet{དྲུག་}{drug}{six} & \forme{tʂɯɣ} \\
7	&	\forme{kɯɕnɯz} & (\tibet{བདུན་}{bdun}{seven}) & \forme{βdɯn} \\
8	&	\forme{kɯrcat}  & \tibet{བརྒྱད་}{brgʲad}{eight} & \forme{βɟɤt} \\
9	&	\forme{kɯngɯt}  & \tibet{དགུ་}{dgu}{nine}& \forme{rgɯ} \\
10	&	\forme{sqi}  & (\tibet{བཅུ་}{btɕu}{ten})& \forme{ftɕɯ} \\
\lspbottomrule
\end{tabular}
\end{table}		

Numerals from 1 to 99 are a subclass of unpossessible nouns (§\ref{sec:unpossessible.nouns}), and cannot take possessive prefixes; they differ in this regard from the higher numerals (§\ref{sec.hundred.plus}, §\ref{sec:approx.numerals}).

\subsection{Tens} \label{sec:decades}
The numerals for tens (\tabref{tab:decades}) are relatively straightforward. With the exception of \japhug{ɣnɤsqi}{twenty} and \japhug{fsɯsqi}{thirty}, they are predictable by combining \japhug{sqi}{ten} with the corresponding numeral prefix (§\ref{sec:numeral.prefixes}). The numeral \japhug{kɯngɯsqi}{ninety} is ambiguous, as the same form can also mean `nine or ten' (see \tabref{tab:approx.num.1to10}).

The element \forme{ɣnɤ-} in \japhug{ɣnɤsqi}{twenty} is related to the numeral  \japhug{ʁnɯz}{two}, but has a velar \forme{ɣ-} prefix instead of the uvular \forme{ʁ\trt}, and has a different vowel. The adverb \japhug{ʁnaʁna}{both} is also relatable, but the alternations are not explainable from a synchronic point of view.

\begin{table}
	\caption{Tens}  \label{tab:decades} \centering
	\begin{tabular}{lllllll}
		\lsptoprule
		10	&	\forme{sqi} \\			
		20	&	\forme{ɣnɤ-sqi} \\		
		30	&	\forme{fsɯ-sqi}  \\		
		40	&	\forme{kɯβdɤ-sqi}  \\	
		50	&	\forme{kɯmŋɤ-sqi}  \\	
		60	&	\forme{kɯtʂɤ-sqi}  \\	
		70	&	\forme{kɯɕnɤ-sqi}  \\	
		80	&	\forme{kɯrcɤ-sqi}  \\	
		90	&	\forme{kɯngɯ-sqi}  \\	
		\lspbottomrule
	\end{tabular}
\end{table}		

Other numerals under one hundred are built by combining the tens in \tabref{tab:decades} (removing the \forme{-sqi} element) with the units in \tabref{tab:teens}. For instance, 37 can be obtained by putting together \japhug{fsɯsqi}{thirty} and \japhug{sqaɕnɯz}{seventeen} as \forme{fsɯ-sqa-ɕnɯz}.

\subsection{Numerals 11-19 and units} \label{sec:teens}
The numerals 11-19, listed in \tabref{tab:teens}, serve as the basis for indicating units in all following numerals between 21 and 99, by replacing the \forme{-sqi} element of the tens (\tabref{tab:decades}) by the appropriate form. As an example of how to build numerals above 19 with a unit 1-9, \tabref{tab:teens}  illustrates the formation of the numerals 21 to 29 from \japhug{ɣnɤsqi} {twenty}. 

\begin{table}
\caption{Numerals 11-19 and 21-29}  \label{tab:teens} \centering
\begin{tabular}{lllllll}
\lsptoprule
10 & \forme{sqi} &	20	&	\forme{ɣnɤsqi}  \\	
\midrule
11 & \forme{sqa-p-tɯɣ} &	21	&	\forme{ɣnɤ-sqa-p-tɯɣ}  \\	
12 & \forme{sqa-m-nɯz} &	22	&	\forme{ɣnɤ-sqa-m-nɯz}  \\	
13 & \forme{sqa-f-sum} &	23	&	\forme{ɣnɤ-sqa-f-sum}  \\	
14 & \forme{sqa-βde} &	24	&	\forme{ɣnɤ-sqa-βde}  \\	
15 & \forme{sqa-mŋu} &	25	&	\forme{ɣnɤ-sqa-mŋu}  \\	
16 & \forme{sqa-p-rɤɣ} &	26	&	\forme{ɣnɤ-sqa-p-rɤɣ}  \\	
17 & \forme{sqa-ɕnɯz} &	27	&	\forme{ɣnɤ-sqa-ɕnɯz}  \\	
18 & \forme{sqa-rcat} &	28	&	\forme{ɣnɤ-sqa-rcat}  \\	
19 & \forme{sqa-ngɯt} &	29	&	\forme{ɣnɤ-sqa-ngɯt}  \\	
\lspbottomrule
\end{tabular}
\end{table}		
 
The numerals 11-19 present three morphological changes in comparison with the basic numerals 1-9.

First, the form \japhug{sqi}{ten} alternates with \forme{sqa-}. The origin of this Ablaut is unknown, though it could be a type of bound state (§\ref{sec:status.constructus}); some Gyalrongic languages, such as Khroskyabs have a similar alternation (\citealt[175--176]{lai17khroskyabs}). 

Second, the velar \forme{kɯ-} and uvular \forme{χ-/ʁ-} prefixes found in the base numerals are lost in all numerals 11-19.

Third, a labial element \ipa{p} (\japhug{sqaptɯɣ}{eleven}, \japhug{sqaprɤɣ}{sixteen}), \ipa{m} (\japhug{sqamnɯz}{twelve}), or \ipa{w} (\japhug{sqafsum}{thirteen}) is inserted between the \forme{sqa-} and the following numeral root. It does not occur in seventeen, eighteen and nineteen (which already have a cluster), fourteen and fifteen (which have a cluster with a labial as first element).

The form \japhug{sqaptɯɣ}{eleven} contains an ablauted form of \japhug{tɤɣ}{one} as second element. The cluster \forme{-pt-} in this word is the only case in the language of a \ipa{p} followed by an obstruent (§\ref{sec:heterosyllabic.clusters}).

In \japhug{sqamnɯz}{twelve}, the labial linker is nasalized by the following \forme{n}. This is not a synchronic rule: for instance, a noun \japhug{ɕnaβndʑɣi}{snotty-nosed kid} has \forme{β} allomorph of \ipa{w} before a prenasalized obstruent (§\ref{sec:subject.verb.compounds}). However, there are other cases of nasalization of labial consonants to \ipa{m} before nasal or prenasalized consonants in Japhug (see §\ref{sec:causative.m}).

In \japhug{sqaprɤɣ}{sixteen}, not only the prefix \forme{kɯ-} is lost, the \forme{tʂ} affricate of the base form 	\japhug{kɯtʂɤɣ}{six} is replaced by \ipa{r}, preceded by the linking element \forme{-p-}. This \ipa{tʂ} \tld{} \ipa{r} alternation is evidence for a sound change \forme{*tr-} \fl{} \ipa{tʂ} (§\ref{sec:Cr.clusters}).  The numeral \japhug{kɯtʂɤɣ}{six} contains two etymological prefixes, \forme{kɯ-} and a prefix \forme{*t-} that has fused with the root as \forme{-tʂɤɣ}. This \forme{*t-} prefix is possibly related to the \forme{d-} of its Tibetan cognate  \tibet{དྲུག་}{drug}{six} .

The numeral prefixes corresponding to the numerals between 11 and 99 are discussed in §\ref{sec:num.prefixes.11.99}.

\subsection{Hundred and above} \label{sec.hundred.plus}
There are two ways of expressing numbers above 99 in Japhug. First, the noun-like numeral \japhug{ɣurʑa}{one hundred} can occur on its own or be followed by another numeral to express a number between 101 and 199, as in (\ref{ex:hundred.and.eight}).

\begin{exe}
\ex \label{ex:hundred.and.eight}
\gll aʑo kɯ-fse kɯ-ɤcʰɯcʰa ʑo ʁʑɯnɯ ɣurʑa kɯrcat ra\\
\textsc{1sg} \textsc{sbj}:\textsc{pcp}-be.like  \textsc{sbj}:\textsc{pcp}-be.capable \textsc{emph} young.man hundred eight need:\textsc{fact} \\
\glt `I need one hundred and eight able young men like me.' (Norbzang, 16)
(\japhdoi{0003768\#S16})
\end{exe}

The numeral \japhug{ɣurʑa}{one hundred} cannot be combined with single digit numerals to express numbers between 200 and 900. The counted noun\footnote{Counted nouns are nouns that take an obligatory numeral prefix (§\ref{sec:counted.nouns}). } \japhug{tɯ-ri}{one hundred} is used for this purpose, as in \ref{ex:three.hundreds}. The two suppletive roots for hundreds are shared with Pumi (the numeral \forme{ɕí} `hundred' vs. the counted noun \forme{-ɻɛj}, see \citealt[101]{daudey14grammar}; evidence for cognacy with \forme{ɣurʑa} and \forme{tɯ-ri} is presented in \citealt{jacques17num}).

\begin{exe}
\ex \label{ex:three.hundreds}
\gll χsɯ-ri jamar ndɤre tu-nɯ ko, tɯ-tɯpʰu nɯ \\
three-hundred about \textsc{lnk} exist:\textsc{fact}-\textsc{pl} \textsc{sfp} one-hive \textsc{dem} \\
\glt `There are about three hundred of them in one hive.' (26-GZo, 53)
(\japhdoi{0003668\#S48})
\end{exe}
 
Numerals above the hundreds are all borrowed from Tibetan: \japhug{stoŋtsu}{thousand}, \japhug{kʰrɯtsu} {ten thousands}, \japhug{mbɯmχtɤr}{hundred thousands} originate from \tibet{སྟོང་ཚོ་}{stoŋ.tsʰo}{thousand}, \tibet{ཁྲི་ཚོ་}{kʰri.tsʰo}{ten thousands} and \tibet{འབུམ་ཐེར་}{ⁿbum.tʰer}{hundred thousands}, respectively.  Like other numerals, they are postnominal, as shown by (\ref{ex:stoNtsu}).


\begin{exe}
\ex \label{ex:stoNtsu}
\gll ɴɢoɕna ɣɯ nɯ, ɯ-pɯ stoŋtsu tu tu-ti-nɯ ɲɯ-ŋu. qaɟɯ ɣɯ nɯ, kʰrɯtsu tu tu-ti-nɯ ɲɯ-ŋu. \\
spider \textsc{gen} \textsc{dem} \textsc{3sg}.\textsc{poss}-young thousand exist:\textsc{fact} \textsc{ipfv}-say-\textsc{pl} \textsc{sens}-be fish \textsc{gen} \textsc{dem} ten.thousands exist:\textsc{fact} \textsc{ipfv}-say-\textsc{pl} \textsc{sens}-be  \\
\glt `People say that spiders have a thousand offspring, and fishes ten thousands.' (26-mYaRmtsaR)
(\japhdoi{0003674\#S111})
\end{exe}

The numerals thousand and above can take other numerals as multiplicative modifiers, as in (\ref{ex:stoNtsu.kWtsxAG}), where \japhug{kɯtʂɤɣ}{six} follows  \japhug{stoŋtsu}{thousand} to express `six thousand'. This use illustrates the difference between the numerals \japhug{stoŋtsu}{thousand} and above with \japhug{ɣurʑa}{hundred}, as when the latter is followed by a numeral, as \japhug{ɣurʑa kɯrcat}{one hundred and eight}, it can only be in additive, not multiplicative, relation to it (see example \ref{ex:hundred.and.eight} above). The multiplicative numeral modifier to \japhug{stoŋtsu}{thousand} cannot be preposed (Tshendzin says of a combination like $\dagger$\forme{kɯtʂɤɣ stoŋtsu} that \forme{nɯ mɤ-sɤtso} `it does not make sense').

\begin{exe}
\ex  \label{ex:stoNtsu.kWtsxAG}
 \gll   rŋɯl tɯ-xpa tɕe stoŋtsu kɯtʂɤɣ jarma ɲɯ-fsoʁ ɲɯ-cʰa \\
 money one-year \textsc{lnk} thousand six about \textsc{ipfv}-earn \textsc{sens}-can \\
 \glt `He can earn about six thousand [renminbi] per year.' (14-siblings, 178) (\japhdoi{0003508\#S159})
\end{exe}
  
 To have an additive interpretation, the comitative \japhug{cʰo}{with} (or \japhug{cʰondɤre}{with}) is used as in (\ref{ex:stoNtsu.1600}) and (\ref{ex:stoNtsu.1006}).
 
\begin{exe}
\ex  \label{ex:stoNtsu.1600}
\gll  stoŋtsu ci cʰo kɯtʂɤɣ-ri \\
 thousand one \textsc{comit} six-hundred \\
\glt `One thousand six hundred (1600).' (elicited)
\end{exe}
  
\begin{exe}
\ex  \label{ex:stoNtsu.1006}
\gll  stoŋtsu ci cʰondɤre kɯtʂɤɣ \\
 thousand one \textsc{comit} six-hundred \\
\glt `One thousand and six (1006).' (elicited)
\end{exe}

In (\ref{ex:khrWtsu.sqi}), a multiplicative modifier \japhug{sqi}{ten} following \japhug{kʰrɯtsu}{ten thousands} is used instead of \japhug{mbɯmχtɤr}{hundred thousands}. The latter numeral, although known to native speakers, is hardly ever used in speech.

\begin{exe}
\ex  \label{ex:khrWtsu.sqi}
\gll ɯ-kɤ-nɯkon nɯnɯ, rɟɤlkʰɤβ kʰrɯtsu sqi pjɤ-tu. \\
 \textsc{3sg}.\textsc{poss}-\textsc{obj}:\textsc{pcp}-rule \textsc{dem} country ten.thousands ten \textsc{ifr}.\textsc{ipfv}-exist \\
\glt `There were a hundred thousand countries under his rule.' (140518 jinyin chengbao-zh, 13) (\japhdoi{0004028\#S12})
\end{exe} 


 
Unlike numerals under 100, \japhug{ɣurʑa}{one hundred} and above are alienably possessed nouns and can take a third person possessive prefix \forme{ɯ-} to express an approximate number (§\ref{sec:approx.numerals}).

 \subsection{Ordinals} \label{sec:ordinals}
There are no native ordinal numbers in Japhug, even for `first', which is expressed by combining the superlative \japhug{stu}{most} (§\ref{sec:stu.superlative}) with the subject participle (§\ref{sec:subject.participles}) of the verb \japhug{mɤku}{be first} (§\ref{sec:denom.mA}). This minimal participial relative \forme{stu kɯ-mɤku} (§\ref{sec:intr.subject.relativization}) can be used as prenominal modifier as in (\ref{ex:stu.kWmAku}), but most commonly occurs adverbially as in (\ref{ex:stu.kWmAku2}).   
 
\begin{exe}
\ex  \label{ex:stu.kWmAku}
\gll stu kɯ-mɤku tɯ-xpa nɯ ɲɯ-rɯmɯntoʁ ma, nɯ ma ɯ-mat me \\
most \textsc{sbj}:\textsc{pcp}-be.first one-year \textsc{dem} \textsc{ipfv}-have.flower \textsc{lnk} \textsc{dem} apart.from \textsc{3sg}.\textsc{poss}-fruit not.exist:\textsc{fact} \\
\glt `The first year, it has flowers, but no fruits.' (08-qaCti, 35)
(\japhdoi{0003456\#S33})
\end{exe}
  
\begin{exe}
\ex  \label{ex:stu.kWmAku2}
\gll stu kɯ-mɤku nɯ a-pi kɯ pɯ́-wɣ-sat-a,  nɯ ɯ-qʰu tɕe, pɣɤtɕɯ tɤ-sci-a, \\
most \textsc{sbj}:\textsc{pcp}-be.first \textsc{dem} \textsc{1sg}.\textsc{poss}-elder.sibling \textsc{erg} \textsc{aor}-\textsc{inv}-kill-\textsc{1sg} \textsc{dem} \textsc{3sg}.\textsc{poss}-after \textsc{lnk} bird \textsc{aor}-be.born-\textsc{1sg} \\
\glt `First, my elder sister killed me, and then I was reborn as a bird.' (2005-stod-kunbzang, 396)
\end{exe}

With counted nouns (§\ref{sec:counted.nouns}), it is strictly prenominal, and the counted noun can be converted to an inalienably possessed noun (§\ref{sec:CN.IPN}); for instance, the noun phrase (\ref{ex:stu.kWmAku.Wxpa}) can be used instead of \forme{stu kɯ-mɤku tɯ-xpa} in (\ref{ex:stu.kWmAku}). The phrase \forme{stu kɯ-mɤku} does not occur postnominally.

\begin{exe}
\ex  \label{ex:stu.kWmAku.Wxpa}
\gll  stu kɯ-mɤku ɯ-xpa \\
 most \textsc{sbj}:\textsc{pcp}-be.first \textsc{3sg}.\textsc{poss}-year  \\
\glt  `The first year' (elicitation based on \ref{ex:stu.kWmAku})
\end{exe}

There is no specific word meaning `last' in Japhug either, and the participial relative \forme{stu kɯ-maqʰu}  with the subject participle (§\ref{sec:subject.participles}) of \japhug{maqʰu}{be after} is used instead; as in the case of \forme{stu kɯ-mɤku}, counted noun to inalienably possessed noun conversion is possible, and for instance both (\ref{ex:stu.kWmaqhu.tWsNi}) and (\ref{ex:stu.kWmaqhu.WsNi}) exist.

\begin{exe}
\ex \label{ex:stu.kWmaqhu.tWsNi}
\gll  stu kɯ-maqʰu tɯ-sŋi  \\
 most \textsc{sbj}:\textsc{pcp}-be.after one-day \\
 \ex \label{ex:stu.kWmaqhu.WsNi}
\gll  stu kɯ-maqʰu ɯ-sŋi  \\
 most \textsc{sbj}:\textsc{pcp}-be.after \textsc{3sg}.\textsc{poss}-day \\
\glt `The last day' (elicited)
\end{exe}

Given the absence of native ordinals,  Chinese ordinals are used  (§\ref{sec:chinese.numerals}). Tibetan ordinals are only attested for names of months (§\ref{sec:tibetan.numerals}) and even this use is disappearing, as they are being replaced by their Chinese equivalents.

Unlike many languages of the Sino-Tibetan family, the system of birth-order ordinals is very rudimentary. The eldest son is simply called \forme{stu kɯ-wxti} `the one who is the biggest' as in (\ref{ex:stu.kWwxti}).

\begin{exe}
\ex  \label{ex:stu.kWwxti}
\gll ɯ-rɟit stu kɯ-wxti nɯ rɯntɕʰɯn tʂoma rmi\\
\textsc{3sg}.\textsc{poss}-child most \textsc{sbj}:\textsc{pcp}-be.big \textsc{dem}  \textsc{anthr} \textsc{anthr} be.called:\textsc{fact}\\
\glt `Her eldest child is called Rinchen Sgrolma.' (12-BzaNsa, 62)
(\japhdoi{0003484\#S59})
\end{exe}

The only dedicated birth-order ordinal is \japhug{tɯlɤt}{the second sibling} (see §\ref{sec:kinship} and §\ref{sec:other.upn}). The inalienably possessed noun \japhug{ɯ-pa}{the one under} can be used to designate any following sibling, and the youngest sibling is designated by the superlative form of the participle of \japhug{xtɕi}{be small} (see example \ref{ex:Wpa.nW}, which illustrates all these terms).

\begin{exe}
\ex  \label{ex:Wpa.nW}
\gll stu kɯ-wxti cʰondɤre nɯ ɯ-pa nɯ tɯlɤt ni wuma ʑo pjɤ-ɕqraʁ-ndʑi. tɕe nɯ ndʑi-tɕɯ stu kɯ-xtɕi nɯnɯ, dɯxpa ma wuma ʑo pjɤ-kʰe. \\
most \textsc{sbj}:\textsc{pcp}-be.big \textsc{comit} \textsc{dem} \textsc{3sg}.\textsc{poss}-under \textsc{dem} second.sibling \textsc{du} really \textsc{emph} \textsc{ifr}.\textsc{ipfv}-be.intelligent-\textsc{du} \textsc{lnk} \textsc{dem} \textsc{3du}.\textsc{poss}-son most \textsc{sbj}:\textsc{pcp}-be.small \textsc{dem} poor \textsc{lnk} really \textsc{emph} \textsc{ifr}.\textsc{ipfv}-be.stupid \\
\glt `The eldest son and the one under him, the second one, were very intelligent. Their smallest son, poor him, was very stupid' (140430 jin e-zh) (\japhdoi{0003893\#S5})
\end{exe}
 
\subsection{Tibetan and Chinese numerals}  
In addition to the native numerals presented above, Tibetan and Chinese numerals commonly occur in Japhug stories and conversations.

 \subsubsection{Tibetan numerals} \label{sec:tibetan.numerals}

In Japhug, Tibetan numerals (see \tabref{tab:numerals.under.10} above) are mainly used in fixed expressions.\footnote{Only a minority of Japhug speakers are fluent in Amdo or other Tibetic languages, so that Tibetan loanwords in Japhug, and in particular numerals, cannot be considered to be code-switching. 
} Some of these expressions exclusively comprise Tibetan words, for instance \japhug{luskɤr χtɕɯʁɲiz}{twelve year cycle} from \tibet{ལོ་སྐར་}{lo.skar}{year of the cycle} and \tibet{བཅུ་གཉིས་}{btɕu.gɲis}{twelve}.  Other compounds combine Tibetan numerals with Japhug roots, for instance the possessive compound\japhug{rnaftɕɯχa}{having ears with ten holes} from \japhug{tɯ-rna}{ear} (or Tibetan \tibet{རྣ་}{rna}{ear}, the form is ambiguous between cognate and borrowing), the numeral \forme{ftɕɯ-} (from  \tibet{བཅུ་}{btɕu}{ten}) and the verb \japhug{aχa}{lacking a piece}, which appears as postnominal modifier in the name of a trickster character \japhug{qala rnaftɕɯχa}{the rabbit with ten holes in his ears}.
 
The Japhug form \forme{-χtɕɯɣ-} of the Tibetan numeral  \tibet{གཅིག་}{gtɕig}{one} occurs in compounds, but is not attested as an independent word; however, it is possible that  it used to exist in Japhug at an earlier stage of the language (though perhaps not as a numeral), as the stative verb \japhug{naχtɕɯɣ}{be the same} is a denominal derivation based on this numeral (§\ref{sec:identity.modifier}, §\ref{sec:denom.intr.nW}). 

The Tibetan ordinal numbers up to ten (and the cardinal numbers for `eleven' and `twelve'), indicated in \tabref{tab:tibetan.ordinals}, can be used for months names. This usage is not attested in conversations, but does occur in texts when  speakers consciously try to avoid using Chinese words. In example (\ref{ex:XsWmba}), Tshendzin first used the Chinese \ch{三月份}{sānyuèfèn}{third month} and then corrected it to its Tibetan equivalent. 

\begin{exe}
\ex \label{ex:XsWmba}
\gll <sanyuefen> jamar zlawa χsɯmba jamar tɕe ci cʰɯ-rɤpɯ, \\
 third.month about month third about \textsc{lnk} one \textsc{ipfv}-have.young  \\
\glt `It bears young once in the third month.' (21-lWLU, 64)
(\japhdoi{0003576\#S61})
\end{exe}

\begin{table}[H]
\caption{Tibetan ordinals used in Japhug}  \label{tab:tibetan.ordinals} \centering  
\begin{tabular}{lllllll}
\lsptoprule
& Japhug & Tibetan  \\
\midrule
1	&	\forme{taŋbu} & \tibet{དང་པོ་}{daŋ.po}{first} \\
2	&	\forme{ʁɲispa}  & \tibet{གསུམ་པ་}{gɲis.pa}{second} \\
3	&	\forme{χsɯmba}  & \tibet{གསུམ་པ་}{gsum.pa}{third} \\
4	&	\forme{βʑɯpa} & \tibet{བཞི་པ་}{bʑi.pa}{fourth} \\
5	&	\forme{rŋapa}  & \tibet{ལྔ་པ་}{lŋa.pa}{fifth} \\
6	&	\forme{tʂɯxpa}  & \tibet{དྲུག་པ་}{drug.pa}{sixth} \\
7	&	\forme{βdɯnpa} & \tibet{བདུན་པ་}{bdun.pa}{seventh} \\
8	&	\forme{βɟɤtpa}  & \tibet{བརྒྱད་པ་}{brgʲad.pa}{eighth} \\
9	&	\forme{rgɯpa}  & \tibet{དགུ་པ་}{dgu.pa}{ninth} \\
10	&	\forme{ftɕɯpa}  & \tibet{བཅུ་པ་}{btɕu.pa}{tenth} \\
11	&	\forme{ftɕɯχtɕɯɣ}  & \tibet{བཅུ་གཅིག་}{btɕu.gtɕig}{eleventh} \\
12	&	\forme{ftɕɯʁɲiz}  & \tibet{བཅུ་གཉིས་}{btɕu.gɲis}{twelfth} \\
\lspbottomrule
\end{tabular}
\end{table}		

The ordinals can be used with the Tibetan word \japhug{zlawa}{month} (from \tibet{ཟླ་བ་}{zla.ba}{moon, month}) as in (\ref{ex:XsWmba}), or without it as in (\ref{ex:rgWpa}) from the same text.

\begin{exe}
\ex \label{ex:rgWpa}
\gll  rgɯpa jamar tɕe ci cʰɯ-rɤpɯ ɲɯ-ŋu. \\
ninth about \textsc{lnk} one \textsc{ipfv}-have.young  \textsc{sens}-be \\
\glt `It bears young once in the ninth month.' (21-lWLU, 66)
(\japhdoi{0003576\#S63})
\end{exe}

In traditional stories of Tibetan origin, complete noun phrases in Tibetan including numerals can be found. For instance, in (\ref{ex:gsum.brgya.drug.bcu}), the numeral \forme{χsɯmrɟatʂɯɣftɕɯ} from \tibet{གསུམ་བརྒྱ་དྲུག་བཅུ་}{gsum.brgʲa.drug.btɕu}{three hundred sixty} occurs with the noun \forme{pja}  (from Tibetan \tibet{བྱ་}{bʲa}{bird}; not normally used in Japhug except in compounds) and the adverb \japhug{mɯndʐamɯχtɕɯɣ}{all kinds} (from \tibet{མི་འདྲ་མི་གཅིག་}{mi.ⁿdra.mi.gtɕig}{diverse, different}, §\ref{sec:quantifiers.other}).

\begin{exe}
\ex \label{ex:gsum.brgya.drug.bcu}
\gll pja mɯndʐamɯχtɕɯɣ, χsɯmrɟatʂɯɣftɕɯ kɯ pɣɤmbri ɲcɣɤnɤɲcɣɤt ʑo to-lɤt-nɯ, \\
 bird all.kinds 360 \textsc{erg} bird.song \textsc{idph}.III:very.noisy \textsc{emph} \textsc{ifr}-throw-\textsc{pl} \\
\glt `Three hundred sixty birds of all kinds sung bird songs.' (2003smanmi, 139)
\end{exe}

\subsubsection{Chinese numerals}  \label{sec:chinese.numerals}
Chinese numerals are ubiquitous in Japhug, and many younger people (born after 1990) have trouble counting above twenty, even if they otherwise speak Japhug fairly fluently. Even the most fluent speakers use Chinese numerals (with their appropriate Chinese classifiers) when speaking normally. 

In examples (\ref{ex:quanxiang}) and (\ref{ex:liangge}), from a text describing the hamlets in the township of Gdong-brgyad, the Chinese expressions \ch{全乡}{quánxiāng}{the whole township}, \ch{三千多人}{sānqiānduōrén}{three thousand people and a little more} and \ch{两个小队}{liǎnggèxiǎoduì}{two groups} (with the generic classifier \ch{个}{gè}{\textsc{cl}}) are used instead of Japhug native numerals and quantifiers to count the number of people in the area. Note the use of the Cultural Revolution term \ch{小队}{xiǎoduì}{team, group} as a counting unit. The use of Chinese here is due to the fact that administration-related counting, even in villages, is done in Chinese.

\begin{exe}
\ex \label{ex:quanxiang}
\gll nɯnɯ ʁdɯrɟɤt <quanxiang> nɯnɯ <sanqianduoren> ma maŋe. \\
 \textsc{dem}  \textsc{anthr} the.whole.township \textsc{dem} three.thousand.people apart.from not:exist:\textsc{sens} \\
\glt  `There are only a little more than three thousand people in the township of Gdong-brgyad.' (140522 RdWrJAt, 116)
(\japhdoi{0004061\#S109})
\end{exe}
 
\begin{exe}
\ex \label{ex:liangge}
\gll  kɤmɲɯ nɯ <lianggexiaodui> ma maŋe \\
 Kamnyu \textsc{dem} two.groups apart.from not.exist:\textsc{sens} \\
\glt `In Kamnyu there are only two \textit{xiaodui}.' (140522 RdWrJAt, 119)
\end{exe} 


Given the absence of ordinal numerals in Japhug (§\ref{sec:ordinals}), Chinese expressions with the ordinal marker \zh{第} \ipa{dì} are used instead. In example (\ref{ex:stu.kWmAku3}), the Chinese phrase \ch{第一名}{dìyīmíng}{the first (in a competition)} occurs, directly followed by the participial clause \forme{stu kɯ-mɤku} `the first one'.

\begin{exe}
	\ex  \label{ex:stu.kWmAku3}
	\gll tɕe βʑɯ nɯ kɯ <diyiming> tɕe stu kɯ-mɤku pjɤ-mɟa. \\
	\textsc{lnk} mouse \textsc{dem} \textsc{erg} first \textsc{lnk} most \textsc{sbj}:\textsc{pcp}-be.first \textsc{ifr}-take \\
	\glt `The mouse obtained the first place.' (150826 shier shengxiao-zh, 97)
(\japhdoi{0006284\#S96})
\end{exe}

 \subsection{Use of the numerals}  \label{sec:uses.numerals}
 Japhug numerals can occur on their own when counting (\forme{ci}, \forme{ʁnɯz}, \forme{χsɯm}, \forme{kɯβde}...) or be used as postnominal attributive modifiers (§\ref{sec:noun.phrases.word.order}). The noun can be elided when the context is clear, especially when the same referent occurs in the previous proposition as in (\ref{ex:WrJit.kWngWt})  with \japhug{ɯ-rɟit}{her children} and (\ref{ex:kWBde.kW}) with \ch{菜}{cài}{dish}. In this case the numeral constitutes the head of the noun phrase.

\begin{exe}
\ex \label{ex:WrJit.kWngWt} 
\gll ɯ-rɟit kɯngɯt tɤ-tu ri, kɯtʂɤɣ nɯ-si \\
\textsc{3sg}.\textsc{poss}-child nine \textsc{aor}-exist \textsc{lnk} six \textsc{aor}-die \\
\glt `She had nine children, but six of them died.' (14-tApi taRi, 17)
\end{exe}

\begin{exe}
\ex \label{ex:kWBde.kW} 
\gll <cai> χsɯm tu-sɯ-lɤt-i tɕe tɕe  kɯβde nɯ kɯ χsɯm tu-ndza-j kɯ-fse. \\
dish three \textsc{ipfv}-\textsc{caus}-throw-\textsc{1pl} \textsc{lnk} \textsc{lnk} four \textsc{dem} \textsc{erg} three \textsc{ipfv}-eat-\textsc{1pl} \textsc{sbj}:\textsc{pcp}-be.like \\
\glt `We used to order three dishes, and the four of us would eat [the] three [of them].' (140501 tshering skyid)
(\japhdoi{0003902\#S90})
\end{exe}		

Numerals can also be modifiers of dual and plural pronouns as in (\ref{ex:iZo.kWBde}). Since pronouns are never obligatory in Japhug (§\ref{sec:overt.non.overt}), it is also possible to use a bare numeral in core argument function with first or second person indexation on the verb, as in (\ref{ex:kWBde.kW}), where the verb \forme{tu-ndza-j} `we eat' of the second proposition has the \textsc{1pl} \forme{-j} suffix indexing the transitive subject, coreferent with the ergatively-marked phrase \forme{kɯβde nɯ kɯ} `the four' (standing for \forme{iʑo kɯβde nɯ kɯ} `the four of us').

\begin{exe}
\ex \label{ex:iZo.kWBde} 
\gll tɕe iʑo kɯβde nɯ tɯtɯrca ku-rɤʑi-j tɕe, \\
\textsc{lnk} \textsc{1pl} four \textsc{dem}  together \textsc{ipfv}-stay-\textsc{1pl} \textsc{lnk} \\
\glt `The four of us were living together.' (140501  tshering skyid, 85)
\end{exe}		

Numerals can be used in collocations with two light verbs: the intransitive verb  \japhug{pa}{pass $X$ years} (§\ref{sec:pa.intr.lv}) and the transitive verb \japhug{ndo}{take} (§\ref{sec:ndo.lv}).
 
The numeral \japhug{ci}{one} occurs in many additional functions: as an identity pronoun `the other one' (§\ref{sec:other.pro}), as a partitive pronoun `one of them' (§\ref{sec:partitive.pronouns}), as an indefinite article `a, one' (postnominal, §\ref{sec:indefinite.markers}), as an identity modifier `the other X' (prenominal, §\ref{sec:identity.modifier}) and  as an adverb meaning `once, one time, a little' (§\ref{sec:tense.aspect.adverbs}).
 
\section{Approximate numerals} \label{sec:approx.numerals}
To express an approximate number, it is possible in Japhug to use the adverb \japhug{jamar}{about} (from Tibetan \tibet{ཡར་མར་}{jar.mar}{about, up and down} and/or to combine adjacent numerals in a row as in (\ref{ex:RnWz.XsWm.kWBde}).

\begin{exe}
\ex \label{ex:RnWz.XsWm.kWBde}
\gll tɯ-kʰɤl nɯtɕu ʁnɯz, χsɯm kɯβde jamar ku-ndzoʁ. \\
one-place \textsc{dem}:\textsc{loc} two three four about \textsc{ipfv}-\textsc{anticaus}:attach \\
\glt `In each place [the flowers] are attached in [groups] of about two, three or four.' (16-RlWmsWsi, 9)
(\japhdoi{0003520\#S9})
\end{exe}

However, Japhug also has five morphological devices to build approximate numerals (Tables \ref{tab:approx.num.1to10} and \ref{tab:approx.decades} ).

First, for numerals under seven (\tabref{tab:approx.num.1to10}), one can build approximate numerals by prefixing a \forme{la-} or \forme{lɤ-} element to one (or two) numeral root(s). Not all possibilities are attested (for instance there is no such approximate numeral $\dagger$\forme{lɤtʂɤɣ} derived from only \japhug{kɯtʂɤɣ}{six}).  Prefixation of \forme{la-} / \forme{lɤ-} occurs with other morphological changes: (i) loss of the velar \forme{kɯ-} prefix (but not the uvular one in `two' and `three', §\ref{sec:one.to.ten}) (ii) loss of the \forme{m-} preinitial in  \japhug{lɤŋu}{about five}, but not of the \forme{*t-} prefix of \forme{kɯtʂɤɣ} (§\ref{sec:teens}) in \japhug{lɤŋɤtʂɤɣ}{five or six} (otherwise $\dagger$\forme{lɤŋɤrɤɣ} would be have been found). The \forme{la-} / \forme{lɤ-} prefix is probably historically related to the \forme{-lɤ-} element found in \textit{dvandva} collectives (§\ref{sec:dvandva.coll}).  


Above `seven', the \forme{lɤ-} prefix seem to have some productivity, as in example (\ref{ex:lAkhrWtsusqi}) from a story translated from Chinese, it is found in the form \forme{lɤ-kʰrɯtsu-sqi}, which translates \ch{数十万}{shùshíwàn}{several hundred thousands}. The normal way to express this meaning in contemporary Japhug would be direct borrowing from Chinese (§\ref{sec:chinese.numerals}), and this \textit{Augenblicksbildung} was introduced to avoid using a Chinese word.

\begin{exe}
\ex \label{ex:lAkhrWtsusqi} 
\gll <zhanghan> kɤ-ti nɯnɯ kɯ ʁmaʁmi lɤ-kʰrɯtsu-sqi jamar ʑo to-ndo. \\
\textsc{anthr} \textsc{obj}:\textsc{pcp}-say \textsc{dem} \textsc{erg} soldier about-10000-ten about \textsc{emph} \textsc{ifr}-take \\
\glt `The [general] called Zhang Han took several hundred thousand soldiers.' (hist160721 pofuchenzhou-zh, 18)
\end{exe}

Second, some approximate numerals are built by compounding two numeral roots (in some cases with the  \forme{la-} / \forme{lɤ-} prefix, \tabref{tab:approx.num.1to10}). The first numeral undergoes bound state vowel change (§\ref{sec:status.constructus}), with loss of the codas \forme{-z} and \forme{-t} (§\ref{sec:loss.codas.compounds}). In the case of \japhug{ɕnɤcat}{seven or eight} (illustrated by example \ref{ex:CnAcat.ci}), the form \forme{ɕnɤ-} is irregular ($\dagger$\forme{ɕnɯ-} would be expected instead). Note that in this list \japhug{kɯngɯsqi}{nine or ten} is ambiguous: this form can also mean `ninety' (see \tabref{tab:decades}).

\begin{exe}
\ex \label{ex:CnAcat.ci} 
\gll ɯʑo nɯnɯ ɕnɤcat ci tɤ-kɤ-sɯpa jamar ʑo qarma wxti ri, \\
\textsc{3sg} \textsc{dem} seven.or.eight one \textsc{aor}-\textsc{obj}:\textsc{pcp}-\textsc{caus}-do about \textsc{emph} crossoptilon be.big:\textsc{fact} but \\
\glt `Although the crossoptilon is as big as about seven or eight of them (weasels) put together.' (27-spjaNkW, 56)
(\japhdoi{0003704\#S54})
\end{exe}
 
\begin{table}
\caption{Approximate numerals in Japhug (one to ten)} \label{tab:approx.num.1to10} \centering
\begin{tabular}{llllll}
\lsptoprule
Approximate Numeral & Base Numerals \\
\midrule
\japhug{laʁnɯz}{a few} & \japhug{ʁnɯz}{two} \\
\japhug{laʁnɯχsɯm}{two or three}  & 	\japhug{ʁnɯz}{two} \\
&\japhug{χsɯm}{three} \\
\japhug{lɤβdelɤŋu}{four or five}  & 		\japhug{kɯβde}{four} \\
 & 		\japhug{kɯmŋu}{five} \\
 \japhug{lɤŋu}{about five}   & 		\japhug{kɯmŋu}{five} \\
\japhug{lɤŋɤtʂɤɣ}{five or six}  & 	\japhug{kɯmŋu}{five} \\
&\japhug{kɯtʂɤɣ}{six} \\
\japhug{ɕnɤcat}{seven or eight}  & 	\japhug{kɯɕnɯz}{seven} \\
 & 	\japhug{kɯrcat}{eight} \\
\japhug{kɯngɯsqi}{nine or ten}  & 	\japhug{kɯngɯt}{nine} \\
& 	\japhug{sqi}{ten} \\
\lspbottomrule
\end{tabular}
\end{table}


Third, for tens (\tabref{tab:approx.decades}), approximate forms can be formed using the same rule as tens from 40 to 90 (§\ref{sec:decades}), by combining the bound state of the unit numeral with the root \japhug{sqi}{ten}, for instance  \japhug{lɤŋɤsqi}{about fifty} from \japhug{lɤŋu}{about five}  like \japhug{kɯmŋɤsqi}{fifty} from \japhug{kɯmŋu}{five}. These approximate numerals are rare and not attested in the non-elicited corpus.

Fourth, an alternative way of producing approximate tens is to add the numeral prefix \forme{tɯ-}  (§\ref{sec:numeral.prefixes}) to a ten, as for instance \japhug{tɯɣnɤsqi}{about twenty} from \japhug{ɣnɤsqi}{twenty} (see example \ref{ex:tWGnAsqi}).

\begin{exe}
\ex \label{ex:tWGnAsqi}
\gll tɯɣnɤsqi jamar tɯtɯrca ju-ɣi-nɯ ŋgrɤl \\
 about.twenty about together \textsc{ipfv}-come-\textsc{pl} be.usually.the.case:\textsc{fact} \\
\glt  `They come in groups of about twenty individuals.' (23-qapGAmtWmtW, 105)
\end{exe}

 
\begin{table}
\caption{Approximate numerals in Japhug (tens)} \label{tab:approx.decades} \centering
\begin{tabular}{llllll}
\lsptoprule
Approximate Numeral & Base Form \\
\midrule
\japhug{tɯɣnɤsqi}{about twenty} & \japhug{ɣnɤsqi}{twenty} \\
\japhug{tɯfsɯsqi}{about thirty}  & 	\japhug{fsɯsqi}{thirty} \\
\japhug{tɯkɯβdɤsqi}{about forty} 	&	\japhug{kɯβdɤsqi}{forty}  \\	
\japhug{tɯkɯmŋɤsqi}{about fifty} 	&	\japhug{kɯmŋɤsqi}{fifty}  \\	
\japhug{tɯkɯtʂɤsqi}{about sixty} 	&	\japhug{kɯtʂɤsqi}{sixty}  \\	
\japhug{tɯkɯɕnɤsqi}{about seventy} 	&	\japhug{kɯɕnɤsqi}{seventy}  \\	
\japhug{tɯkɯrcɤsqi}{about eighty} 	&	\japhug{kɯrcɤsqi}{eighty}  \\	
\japhug{tɯkɯngɯsqi}{about ninety} 	&	\japhug{kɯngɯsqi}{ninety}  \\	
\midrule
 \japhug{lɤŋɤsqi}{about fifty}   & 		\japhug{lɤŋu}{about five} \\
\japhug{lɤŋɤtʂɤsqi}{fifty or sixty}  & 	\japhug{lɤŋɤtʂɤɣ}{five or six}  \\
\japhug{ɕnɤcɤsqi}{seventy or eighty}  & 	\japhug{ɕnɤcat}{seven or eight} \\
\lspbottomrule
\end{tabular}
\end{table}

Fifth, in the case of numerals above 99, approximate numerals are built by prefixing a third singular possessive \forme{ɯ-} prefix, as \japhug{ɯ-ɣurʑa}{several hundreds} (\ref{ex:WGurZa}), \japhug{ɯ-stoŋtsu}{several thousands} (\ref{ex:WstoNtsu}) and higher numerals.
 
\begin{exe}
\ex \label{ex:WGurZa}
\gll  ɯ-ɣurʑa, χsɯ-ri jamar ndɤre tu-nɯ ko, tɯ-tɯpʰu nɯ  \\
\textsc{3sg}.\textsc{poss}-hundred three-hundred about \textsc{top}.\textsc{advers} exist:\textsc{fact}-\textsc{pl} \textsc{sfp} one-hive \textsc{dem} \\
\glt `There are about several hundreds, about three hundred of them in one hive.' (26-GZo)
(\japhdoi{0003668\#S48})
\end{exe}

\begin{exe}
\ex \label{ex:WstoNtsu}
\gll  tɯ-ŋga tɯ-rdoʁ nɯ ɯ-stoŋtsu ɯ-pʰɯ kɯ-fse ŋu ma \\
\textsc{indef}.\textsc{poss}-clothes one-piece \textsc{dem} \textsc{3sg}.\textsc{poss}-thousand \textsc{3sg}.\textsc{poss}-price \textsc{sbj}:\textsc{pcp}-be.like be:\textsc{fact} \textsc{lnk} \\ 
\glt `One piece of clothes [made from it], its price is several thousand [renminbi].' (05-qaZo, 81)
\end{exe}

\section{Counted nouns} \label{sec:counted.nouns}
The term \textit{counted nouns} designates a subclass of nouns that differs from inalienably possessed (§\ref{sec:inalienably.possessed}), alienably possessed and unpossessible nouns (§\ref{sec:unpossessible.nouns}) in that they require a numeral prefix, whose paradigm is described in (§\ref{sec:numeral.prefixes}). No other part of speech is compatible with these prefixes.

For instance, the counted noun \forme{-xpa} `year' occurs in (\ref{ex:XsW.xpa}) with the prefix \forme{χsɯ-} `three'. It is not possible to express the same meaning by combining the bare stem of the noun \forme{-(x)pa} with the corresponding numeral numeral \japhug{χsɯm}{three} (something  like  $\dagger$\forme{xpa χsɯm} or $\dagger$\forme{pa χsɯm} would be agrammatical).

\begin{exe}
\ex \label{ex:XsW.xpa}
\gll χsɯ-xpa \\
three-year \\
\glt `Three years'
\end{exe}

In this grammar, counted noun are cited using the form with the numeral prefix `one' (\forme{tɯ-xpa} in the case of `year').\footnote{One reviewer pointed out that this choice may be infelicitous, but in my opinion it is preferable in order to avoid confusion with other subclasses of nouns, in particular in the case of conversion from inalienable and alienable nouns to counted nouns (§\ref{sec:CN.parts.of.speech}). }

Numeral prefixes are bound forms historically derived from free numerals (§\ref{sec:num.prefix.paradigm.history}). These bound forms strictly precede the nominal stem, unlike free numerals, which generally \textit{follow} the nouns they modify  (§\ref{sec:uses.numerals}).\footnote{However, the numeral+noun order is attested in some compounds (§\ref{sec:determinative.n.n}, §\ref{sec:possessive.n.n}). }

Only a highly restricted number of nominal stems can take numeral prefixes. For instance, the alienably possessed \japhug{mbro}{horse} or the inalienably possessed \japhug{tɤ-pi}{elder sibling} are not compatible with numerals prefixes. To express the meanings `three horses' or `three brothers', one cannot add the numeral prefix \forme{χsɯ-}: forms such as $\dagger$\forme{χsɯ-mbro}, $\dagger$\forme{χsɯ-tɤ-pi} or $\dagger$\forme{χsɯ-pi} would be unintelligible. Instead, one must use the free numeral \japhug{χsɯm}{three} (§\ref{sec:uses.numerals}) as in (\ref{ex:mbro.XsWm}) 

\begin{exe}
\ex \label{ex:mbro.XsWm}
\gll mbro χsɯm \\
horse three \\
\glt `Three horses'
%\begin{xlist}
%\ex \label{ex:Wpi.XsWm}
%\gll ɯ-pi χsɯm \\
%\textsc{3sg}.\textsc{poss}-elder.sibling three \\
%\glt `His/her three elder brothers/sisters'
%\ex \label{ex:tApi.XsWm}
%\gll tɤ-pi χsɯm \\
%\textsc{3sg}.\textsc{poss}-elder.sibling three \\
%\glt `The three elder brothers/sisters'
%\end{xlist}
\end{exe}

A handful of alienably and inalienably possessed nouns can be \textit{converted} to counted nouns and thus take numeral prefixes, but always with a semantic narrowing (§\ref{sec:CN.IPN}, §\ref{sec:CN.APN}).

The class of `counted nouns' in this grammar is similar to what have elsewhere in the literature been described as `classifiers' (in Chinese \zh{量词} \forme{liàngcí} `measure words'). The morphology-based term `counted noun', independent of meaning and function, is preferred over `classifier' because it is highly contestable that classification is indeed the main function of this class of words (see for instance  \citealt{francois99classificateurs}, and §\ref{sec:CN.classification}). 

\subsection{Numeral prefixes} \label{sec:numeral.prefixes}
In this section, numeral prefixes are described following several categories (1-10, 11-99, approximate numerals and prefixes derived from nouns) and irregular forms are discussed in a separated subsection (§\ref{sec:irregular.numeral.prefixes}). A final subsection presents historical hypotheses to account for the numeral prefixal paradigm and its relationship to that of other Gyalrongic languages.



\subsubsection{Numeral prefixes between 1 and 10} \label{sec:num.prefixes.1.10}
The paradigm of regular numeral prefixes from 1 to 10 in Kamnyu Japhug is indicated in \tabref{tab:num.prefix.1.to.10}. The prefixes are derived from the corresponding numeral by bound state (§\ref{sec:status.constructus}), with loss of the coda and vowel alternation. Vowel alternation is optional for \japhug{kɯβde}{four} and  \japhug{kɯmŋu}{five}. 

%In the corpus,  the coda \forme{-z} of the numeral \japhug{kɯɕnɯz}{seven} seems to be preserved in one example (\ref{ex:kWCnWztWphu}), but this is an incorrect form, with a slight pause of hesitation between the numeral of the following noun.
%
%\begin{exe}
%\ex \label{ex:kWCnWztWphu}
%\gll kɯɕnɯz... -tɯpʰu ʑo pjɤ-tu \\
% seven -types \textsc{emph} \textsc{ifr}.\textsc{ipfv}-exist \\
%\glt `There were five types [of meals on the table].' (140504 baixuegongzhu-zh, 57)
%\end{exe}


\begin{table}
\caption{1-10 regular numeral prefixes in Japhug}  \label{tab:num.prefix.1.to.10} 
\begin{tabular}{lllllll}
\lsptoprule
Numeral & Free form &  \forme{-sŋi} `day'   \\
\midrule
 1	&	\forme{tɤɣ}  &	\forme{tɯ-sŋi}  &	\\
2	&	\forme{ʁnɯz}  &	\forme{ʁnɯ-sŋi}  &	\\
3	&	\forme{χsɯm}  &	\forme{χsɯ-sŋi}  &	\\
4	&	\forme{kɯβde}  &	\forme{kɯβde-sŋi}, \forme{kɯβdɤ-sŋi}  &	\\
5	&	\forme{kɯmŋu}  &	\forme{kɯmŋu-sŋi}, \forme{kɯmŋɤ-sŋi}  &	\\
6	&	\forme{kɯtʂɤɣ}  &	\forme{kɯtʂɤ-sŋi}  &	\\
7	&	\forme{kɯɕnɯz}  &	\forme{kɯɕnɯ-sŋi}  &	\\
8	&	\forme{kɯrcat}  &	\forme{kɯrcɤ-sŋi}  &	\\
9	&	\forme{kɯngɯt}  &	\forme{kɯngɯ-sŋi}  &	\\
10	&	\forme{sqi}  &	\forme{sqɯ-sŋi}  &\\
\lspbottomrule
\end{tabular}
\end{table}

Note that the forms of the numeral prefixes differ in some cases from the corresponding numeral prefixes in tens (cf. \tabref{tab:decades}), compare \japhug{kɯɕnɯ-sŋi}{seven days} with \japhug{kɯɕnɤ-sqi}{seventy}.


\subsubsection{Numeral prefixes between 11 and 99} \label{sec:num.prefixes.11.99}
Above ten, numeral prefixes present some variation.  \tabref{tab:num.prefix.11.to.20} shows the most common forms of the numeral prefixes between 11 and 20 (from which all numerals between 11 and 99 can be generated following the rules described in §\ref{sec:decades}), but many cases without vowel alternation or with preservation of the codas \forme{-z} (\ref{ex:GnAsqamnWzpArme}) or \forme{-ɣ} (\ref{ex:GnAsqaptWWGrZaR}) are attested, as in (\ref{ex:GnAsqamnWzpArme}) for instance

\begin{exe}
	\ex \label{ex:GnAsqamnWzpArme}
	\gll  ma ɯ-me kɯnɤ ɣnɤsqamnɯz-pɤrme tʰɯ-azɣɯt. \\
	\textsc{lnk} \textsc{3sg}.\textsc{poss}-daughter also twenty.two-year.old \textsc{aor}-reach \\
	\glt `Even his daughter is now twenty-two.' (14-siblings, 317)
(\japhdoi{0003508\#S284})
\end{exe}

 \begin{table}
\caption{11-20 numeral prefixes in Japhug}  \label{tab:num.prefix.11.to.20} 
\begin{tabular}{lllllll}
\lsptoprule
Numeral & Free form &  \forme{-sŋi} `day'   \\
\midrule
11	&	\forme{sqaptɯɣ}  &	\forme{sqaptɯ-sŋi}  &	\\
12	&	\forme{sqamnɯz}  &	\forme{sqamnɯ-sŋi}  &	\\
13	&	\forme{sqafsum}  &	\forme{sqafsum-sŋi}  &	\\
14	&	\forme{sqaβde}  &	\forme{sqaβde-sŋi}  &	\\
15	&	\forme{sqamŋu}  &	\forme{sqamŋu-sŋi}  &	\\
16	&	\forme{sqaprɤɣ}  &	\forme{sqaprɤ-sŋi}  &	\\
17	&	\forme{sqaɕnɯz}  &	\forme{sqaɕnɯ-sŋi}  &	\\
18	&	\forme{sqarcat}  &	\forme{sqarcɤ-sŋi}  &	\\
19	&	\forme{sqangɯt}  &	\forme{sqangɯ-sŋi}  &	\\
20	&	\forme{ɣnɤsqi}  &	\forme{ɣnɤsqɯ-sŋi}   &	\\
\lspbottomrule
\end{tabular}
\end{table}

In example (\ref{ex:fsWsqildZa}), we observe the two alternative forms \forme{-sqɯ-} and \forme{-sqi-} for the tens in the same sentence. While for the numeral ten only the prefix \forme{sqɯ-} (or its variant \forme{sqɤ\trt}, see \tabref{tab:num.prefix.tArZaR} below) is found, for tens between 20 and 90 vowel alternation is optional and there is free variation between the two forms. In the corpus, we find 15 examples of \forme{-sqi-} and 14 of  \forme{-sqɯ\trt}, suggesting that both are about equally common.

\begin{exe}
\ex \label{ex:fsWsqildZa}
\gll tɯ-pʰɯ nɯ tɕe rcanɯ, li, fsɯsqi-ldʑa jamar, kɯβdɤsqɯ-ldʑa jamar tu. \\
one-tree \textsc{dem} \textsc{lnk} \textsc{unexp}:\textsc{deg} again thirty-long.object about forty-long.object about exist:\textsc{fact} \\
\glt `On one tree, there are about thirty or forty [branches].'   (14-sWNgWJu, 200)
\end{exe}

Example (\ref{ex:GnAsqaptWWGrZaR}) illustrates three alternative forms with the counted noun \japhug{tɤ-rʑaʁ}{one night}: \forme{ɣnɤsqaptɯ-rʑaʁ} with regular loss of coda, \forme{ɣnɤsqaptɯɣ-rʑaʁ} with preservation of the coda,   and \forme{ɣnɤsqamnɯz} \forme{tɤ-rʑaʁ} as two words, the numeral being a kind of prenominal modifier. The first form is regular, while the other ones each are \textit{hapax legomena}.

\begin{exe}
\ex \label{ex:GnAsqaptWWGrZaR}
\gll  tɕe nɯ ɣnɤsqaptɯɣ-rʑaʁ tu-tsu ɲɯ-ra. ``tɕe ɣnɤsqaptɯ-rʑaʁ tu-tsu tɕe ɲɯ-ʁaʁ ŋu" ɲɯ-ti-nɯ ri, aʑɯɣ nɯ ɣnɤsqamnɯz tɤ-rʑaʁ mɤɕtʂa mɯ-nɯ-ʁaʁ. \\
\textsc{lnk} \textsc{dem} twenty.one-night \textsc{ipfv}-pass \textsc{sens}-be.needed  
\textsc{lnk}  twenty.one-night \textsc{ipfv}-pass  \textsc{lnk} \textsc{ipfv}-hatch be:\textsc{fact} \textsc{sens}-say-\textsc{pl} \textsc{lnk} \textsc{1sg}:\textsc{gen} \textsc{dem} twenty.two one-night until \textsc{neg}-\textsc{aor}-hatch \\
\glt `[Eggs] need twenty-two days to hatch; people say `They hatch in twenty-two days' but mine only hatch after twenty two days.' (150819 kumpGa)
(\japhdoi{0006388\#S32})
\end{exe}


\subsubsection{Approximate numeral prefixes} \label{sec:approximate.numeral.prefixes}
Approximate numerals also have corresponding prefixal forms. \tabref{tab:approx.num.prefixes} presents the forms attested in the corpus.

 \begin{table}
\caption{Approximate numeral prefixes in Japhug} \label{tab:approx.num.prefixes} 
\begin{tabular}{llllll}
\lsptoprule
Approximate Numeral & Approximate Numeral Prefix \\
\midrule
\japhug{laʁnɯz}{a few} & \forme{laʁnɯ-} \\
\japhug{lɤβdelɤŋu}{four or five}  & 		\forme{lɤβdelɤŋu-}  \\
 \japhug{lɤŋu}{about five}   & 		\forme{lɤŋu-}  \\
\japhug{lɤŋɤtʂɤɣ}{five or six}  & 	\forme{lɤŋɤtʂɤ\trt}, \forme{lɤŋɤtʂɤɣ-} \\
\japhug{ɕnɤcat}{seven or eight}  & 	\forme{ɕnɤcɤ-} \\
\lspbottomrule
\end{tabular}
\end{table}

By far the most commonly used approximate numeral prefix is \forme{laʁnɯ-X}, whose meaning is not  `one or two' as could have been expected from the use of the \forme{lɤ-/la-} prefix in the other examples, but  `a few' (see example \ref{ex:laʁnWxpa} -- life expectancy of goats and sheep is much above two years).

\begin{exe}
\ex \label{ex:laʁnWxpa}
\gll tsʰɤt qaʑo nɯnɯ tɕe laʁnɯ-xpa ma cʰɯ-mdɯ mɯ́j-ŋgrɤl ma tɕe cʰɯ-rgɤz ɕti \\
goat sheep \textsc{dem} \textsc{lnk} a.few-year apart.from \textsc{ipfv}-live.up.to \textsc{neg}:\textsc{sens}-be.usually.the.case \textsc{lnk} \textsc{lnk} \textsc{ipfv}-be.old be.\textsc{aff}:\textsc{fact} \\
\glt `Goats and sheep only live for a few years, and then become old.' (05-qaZo, 144)
\end{exe}

Just like approximate numbers can also be expressed by juxtaposition of numerals  (§\ref{sec:approx.numerals}), it is possible to juxtapose counted nouns (the same counted noun with contiguous numeral prefixes, as in \ref{ex:XsWzWm.kWBdezWm}), or to combine a numeral with a counted noun whose numeral prefix is contiguous to it, as in (\ref{ex:GnAsqi.fsWqifkur}).

\begin{exe}
\ex \label{ex:XsWzWm.kWBdezWm}
\gll tɕe nɯnɯ χsɯ-zɯm kɯβde-zɯm jamar kɯ-xtɕʰɯt tu. \\
 \textsc{lnk} \textsc{dem} three-bucket four-bucket about \textsc{sbj}:\textsc{pcp}-contain exist:\textsc{fact} \\
\glt `There are [jars] that can contain about three or four buckets [of water].' (26-tChWra, 4)
(\japhdoi{0003690\#S4})
\end{exe}

\begin{exe}
\ex \label{ex:GnAsqi.fsWqifkur}
\gll <tuolaji> tɯ-ɣjɤn ju-ɣɯt nɯ rcanɯ, ɣnɤsqi fsɯsqi-fkur jamar ju-sɯ-ɤzɣɯt cʰa.  \\
 tractor one-time \textsc{ipfv}-bring \textsc{dem} \textsc{unexp}:\textsc{deg} twenty thirty-burden about \textsc{ipfv}-\textsc{caus}-reach can:\textsc{fact} \\
\glt `Each time the tractor brings [firewood], it can move about twenty or thirty loads (the size a person can carry on his back).' (140430 tWfkur)
(\japhdoi{0003898\#S22})
\end{exe}

Additionally, the paucal meaning can be expressed by converting the counted noun into an inalienably possessed noun with a third person possessive prefix and reduplication of the noun stem (see §\ref{sec:CN.IPN}).

\subsubsection{Other numeral prefixes} \label{sec:other.numeral.prefixes}
In addition to the numerals mentioned above, all higher and compound numerals, an interrogative pronoun and a participle form can appear as prefixes of counted nouns.

The numerals above 99 (§\ref{sec.hundred.plus}) occur as numeral prefixes without vowel alternation, as \japhug{ɣurʑa-xpa}{a hundred years} and \japhug{stoŋtsu-xpa}{a thousand years} (from \japhug{ɣurʑa}{hundred} and \japhug{stoŋtsu}{thousand}).\footnote{Even in the absence of vowel alternation, the prefixal status of these numerals is shown by three pieces of evidence: (i) the fact that they combine with non-free elements like the stem \forme{-xpa} `year', which require a numeral prefix, (ii) the absence of stress and (iii) the inability to pause or insert any element in between, except in the case of speech errors. } 

The prefixal form of hundreds based on the counted noun \japhug{tɯ-ri}{one hundred} have double numeral prefixes, such as \japhug{χsɯ-ri-xpa}{three hundred years} in (\ref{ex:XsWrixpa}).

\begin{exe}
\ex \label{ex:XsWrixpa}
\gll tɕendɤre χsɯ-ri-xpa tɤ-tsu tɕe, li kɯjŋu pɯ-ta-t-a tɕe, \\
 \textsc{lnk} three-hundred-year \textsc{aor}-pass \textsc{lnk} again oath \textsc{aor}-put-\textsc{tr}:\textsc{pst}-\textsc{1sg} \textsc{lnk} \\
\glt `Three hundred years passed, and I made another oath.' (140512 yufu yu mogui-zh, 92)
(\japhdoi{0003973\#S87})
\end{exe}

In the case of complex numerals, such as the very common \japhug{ɣurʑa kɯrcat}{one hundred and eight}, only the last one undergoes bound state, as in (\ref{ex:GurZa.kWrCAJom}) and (\ref{ex:GurZa.kWrCAGdAt}), where we find \forme{ɣurʑa} as a separate phonological word followed by the counted noun with the numeral prefix \forme{kɯrcɤ-} (from \japhug{kɯrcat}{eight}).

\begin{exe}
\ex \label{ex:GurZa.kWrCAJom}
\gll ɕommbri ɣurʑa kɯrcɤ-ɟom, [...] ɕomtsʰoʁ ɣurʑa kɯrcɤ-ldʑa ra \\
iron.chain hundred eight-length.of.two.outstretched.arms ... iron.nails hundred eight-long.object be.needed:\textsc{fact} \\
\glt  `[I] need a chain of one hundred and eight fathoms, and one hundred and eight nails.' (2003 tWxtsa, 21)
\end{exe}

\begin{exe}
\ex \label{ex:GurZa.kWrCAGdAt}
\gll  ɣurʑa kɯrcɤ-ɣdɤt qapri nɯ a-tɤ-ɕe ra \\
 hundred eight-section snake \textsc{dem} \textsc{irr}-\textsc{pfv}:\textsc{up}-go be.needed:\textsc{fact} \\
\glt `May the snake be cut into hundred and eight sections!' (2012 Norbzang,  280)
(\japhdoi{0003768\#S243})
\end{exe}

The interrogative pronoun \japhug{tʰɤstɯɣ}{how many} has the prefixal form \forme{tʰɤstɯ-} with counted nouns (§ §\ref{sec:thAstWG}). 

The subject participle \forme{kɯ-ɤntɕʰɯ}  of the stative verb \japhug{antɕʰɯ}{be many},  has the prefixal form \forme{kɤntɕʰɯ\trt}, as in (\ref{ex:kAntChWtWpW}). This prefix is very common in the corpus.

\begin{exe}
\ex \label{ex:kAntChWtWpW}
\gll  tɕe nɯnɯ kɤntɕʰɯ-tɯpɯ ɣɯ nɯ-tɯrsa ɯ-sta ɲɯ-ŋu ma \\
 \textsc{lnk} dem many-household \textsc{dem} \textsc{3pl}.\textsc{poss}-grave \textsc{3sg}.\textsc{poss}-place \textsc{sens}-be \textsc{lnk} \\
\glt `It is the grave-place of many families.' (140522 kAmYW tWji, 96)
(\japhdoi{0004055\#S91})
\end{exe}

It is however not possible to convert any noun, pronoun or quantifier into a numeral prefix; with other words, it is necessary to convert the counted noun to an inalienably possessed noun, with a third singular \forme{ɯ-} prefix instead of the numeral prefixes, see §\ref{sec:CN.IPN}.  

The nouns \japhug{tɯ-sla}{one month} and \japhug{tɯ-xpa}{one year} can be used with a numeral prefix \forme{kɤrɤ-} not attested with other counted nouns (see §\ref{sec:CN.time}).

\subsubsection{Irregular forms} \label{sec:irregular.numeral.prefixes}
A handful of counted nouns, in particular \japhug{tɤ-rʑaʁ}{one night}, have an alternative paradigm with \ipa{ɤ} instead of \forme{ɯ} in the prefixes, as shown by the forms \forme{ʁnɤ-rʑaʁ} `two nights' and \forme{χsɤ-rʑaʁ} `three nights' in (\ref{ex:RnA.XsArZaR}).

\begin{exe}
	\ex \label{ex:RnA.XsArZaR}
	\gll ʁnɤ-rʑaʁ jamar, χsɤ-rʑaʁ jamar to-tsu tɕe tɕendɤre cʰa ɯ-di tu-mnɤm ɲɯ-ŋu. \\
	two-night about three-night about \textsc{ifr}-pass \textsc{lnk} \textsc{lnk} alcohol \textsc{3sg}.\textsc{poss}-smell \textsc{ipfv}-have.a.smell \textsc{sens}-be \\
	\glt `After two or three nights, one can smell the smell of alcohol.' (160703 araR, 40)
(\japhdoi{0006101\#S39})
\end{exe}

The complete paradigm of this noun is presented in  
\tabref{tab:num.prefix.tArZaR}. 

 \begin{table}
	\caption{Irregular numeral prefixes in Japhug}  \label{tab:num.prefix.tArZaR} \centering
	\begin{tabular}{lllllll}
		\lsptoprule
		Numeral & Free form  &  \forme{-rʑaʁ} `night' \\
		\midrule
		1	&	\forme{tɤɣ}  &		\forme{tɤ-rʑaʁ}  &	\\
		2	&	\forme{ʁnɯz}  &		\forme{ʁnɤ-rʑaʁ}  &	\\
		3	&	\forme{χsɯm}  &		\forme{χsɤ-rʑaʁ}  &	\\
		4	&	\forme{kɯβde}  &		\forme{kɯβdɤ-rʑaʁ}  &	\\
		5	&	\forme{kɯmŋu}  &		\forme{kɯmŋɤ-rʑaʁ}  &	\\
		6	&	\forme{kɯtʂɤɣ}  &		\forme{kɯtʂɤ-rʑaʁ}  &	\\
		7	&	\forme{kɯɕnɯz}  &		\forme{kɯɕnɤ-rʑaʁ}  &	\\
		8	&	\forme{kɯrcat}  &		\forme{kɯrcɤ-rʑaʁ}  &	\\
		9	&	\forme{kɯngɯt}  &		\forme{kɯngɤ-rʑaʁ}  &	\\
		10	&	\forme{sqi}  &	\forme{sqɤ-rʑaʁ}  &	\\
		\lspbottomrule
	\end{tabular}
\end{table}
 

There is however some degree of variation, and using the regular paradigm is not considered erroneous. In the corpus, the numeral `one' form \japhug{tɯ-rʑaʁ}{one night} is more common than \forme{tɤ-rʑaʁ}, possibly because of the homophony with the inalienably possessed noun \japhug{tɤ-rʑaʁ}{time} found in examples such as (\ref{ex:tArZaR.tArYJi}). For other numerals, the forms in Table (\ref{tab:num.prefix.tArZaR}) are considerably more common than the regular ones. In addition, the \forme{ɤ} vocalism is also found with other numeral prefixes such as the interrogative (\japhug{tʰɤstɤ-rʑaʁ}{how many nights}).

\begin{exe}
\ex \label{ex:tWrZaR}
\gll tɕe qarma nɯ, tɯ-rʑaʁ tɕe kɯβde kɯmŋu jamar pjɯ-sat-nɯ, tɯ-rdoʁ, tɯrme tɯ-rdoʁ kɯ \\
\textsc{lnk} crossoptilon \textsc{dem} one-night \textsc{lnk} four five about \textsc{ipfv}-kill-\textsc{pl}, one-piece person one-piece \textsc{erg} \\
\glt `Crossoptilons, in one night, each of [the hunters] can kill four or five of them.' (23-qapGAmtWmtW, 163)
\end{exe}


\begin{exe}
\ex \label{ex:tArZaR.tArYJi}
\gll tɤ-rʑaʁ tɤ-rɲɟi tɕe, nɯ-ji ra kɯ-dɯ\redp{}dɤn kɯ-jɯ\redp{}jom lo-pɣaʁ-nɯ, ɕoŋtɕa kɯ-dɯ\redp{}dɤn pjɤ-pʰɯt-nɯ \\
\textsc{indef}.\textsc{poss}-time \textsc{aor}-be.long \textsc{lnk} \textsc{3pl}.\textsc{poss}-field \textsc{pl} \textsc{sbj}:\textsc{pcp}-\textsc{emph}\redp{}be.many \textsc{sbj}:\textsc{pcp}-\textsc{emph}\redp{}be.broad \textsc{ifr}-turn.over-\textsc{pl} timber \textsc{sbj}:\textsc{pcp}-\textsc{emph}\redp{}be.many \textsc{ifr}-remove-\textsc{pl} \\
\glt `After some time/as time went on, [those people] had ploughed many broad fields for them, and chopped a lot of timber.' (2002qajdoskAt, 90)
\end{exe}

Apart from  \japhug{tɤ-rʑaʁ}{one night}, counted nouns following the paradigm in Table (\ref{tab:num.prefix.tArZaR}) are very rare. The counted nouns \japhug{tɯ-tɣa}{one span} and \japhug{tɯ-rtsɤɣ}{one storey} have the irregular forms \japhug{χsɤ-tɣa}{three spans} and \japhug{χsɤ-rtsɤɣ}{three storeys} (competing with regular \forme{χsɯ-tɣa} and \forme{χsɯ-rtsɤɣ}) as in (\ref{ex:XsAtGa}), but not for other numeral prefixes. 

\begin{exe}
\ex \label{ex:XsAtGa}
\gll nɯ χsɤ-tɣa kɯβde-tɣa jamar tu-rɲɟi cʰa. \\
\textsc{dem} three-span four-span about \textsc{ipfv}-be.long can:\textsc{fact} \\
\glt `It can grow three or four spans long.' (14-sWNgWJu, 194)
\end{exe}

Another unrelated irregularity concerns the counted noun \japhug{tɯ-ɣjɤn}{one time}: free variation between \forme{-jɤn} and \forme{-ɣjɤn} is observed for the numerals `two' and `three' (both \japhug{χsɯ-ɣjɤn}{three times} and \forme{χsɯ-jɤn} are attested).

A similar case is observed with counted noun \japhug{tɯ-xpa}{one year}; the form \forme{-xpa} is obligatory for one to three (\japhug{ʁnɯ-xpa}{two years}, \japhug{χsɯ-xpa}{three years}), but for `four' on, both stems \forme{-xpa} and \forme{-pa} are attested (both \japhug{sqɯ-xpa}{ten years} and \japhug{sqɯ-pa}{ten years} are possible, though the former is more common), including for approximate numerals (\japhug{ɕnɤcɤ-pa}{seven or eight years}).  

The counted noun \japhug{tɯ-zloʁ}{one time} (see §\ref{sec:arithmetic} on its use) is attested in (\ref{ex:RYizloR}) with the special form \japhug{ʁɲɯ-zloʁ}{two times} whose first element is a numeral prefix form \forme{ʁɲɯ-} influenced by the Tibetan numeral \tibet{གཉིས་}{gɲis}{two} (in Japhug pronunciation \forme{ʁɲiz-}). The regular form \japhug{ʁnɯ-zloʁ}{two times} also exists.
 
\begin{exe}
\ex \label{ex:RYizloR}
\gll a-zda ɣɯ kɯ-fse ʁɲɯ-zloʁ nɯ tu-ndze-a ɲɯ-tʂaŋ \\
 \textsc{1sg}.\textsc{poss}-companion \textsc{gen} \textsc{sbj}:\textsc{pcp}-be.like two-times \textsc{dem} \textsc{ipfv}-eat[III]-\textsc{1sg} \textsc{sens}-be.fair \\
\glt `It would be fair if I had two times as much to eat as the other one.' (140426 lv he luozi-zh, 9)
(\japhdoi{0003816\#S9})
\end{exe}

The stem \forme{-zloʁ} looks like a blend from the two Tibetan words \tibet{ཟློག་}{zlog}{turn around} and \tibet{ཟློ}{zlo}{repeat}. Tibetan numerals are generally post-nominal, but prenominal numerals also exist as in \tibet{ཉིས་ལྡབ་}{ɲis.ldab}{two times}.

\subsubsection{Distributed numeral prefixes} \label{sec:numeral.prefixes.distributed}
The distributed form of counted nouns is built by reduplicating the numeral `one' prefix \forme{tɯ\trt}, as \forme{tɯ-tɯ-rdoʁ} from the generic counted noun \japhug{tɯ-rdoʁ}{one piece} (see §\ref{sec:CN.classification}) in example (\ref{ex:tWtWrdoR}).

\begin{exe}
\ex  \label{ex:tWtWrdoR}
\gll ʑɯrɯʑɤri qʰe, tɕe tɯ-tɯ-rdoʁ nɯ ɲɯ-ʑɣɤ-qɤr-nɯ qʰe \\
 progressively \textsc{lnk} \textsc{lnk} one-one-piece \textsc{dem} \textsc{ipfv}-get.separated-\textsc{pl} \textsc{lnk} \\
\glt `Progressively, some of them get separated [from the herd].' (20-RmbroN, 59)
\end{exe}

Example (\ref{ex:tWtWxpa}) with \japhug{tɯ-tɯ-xpa}{some years} from \japhug{tɯ-xpa}{one year} clearly illustrates the functional difference with approximate numerals (§\ref{sec:approx.numerals} and §\ref{sec:approximate.numeral.prefixes}), as the meaning of the distributed counted noun cannot be translated here as `a few years' -- the years when the income is good are not necessarily contiguous in time.

\begin{exe}
\ex  \label{ex:tWtWxpa}
\gll   tɯ-tɯ-xpa tɕe a-pɯ-pe tɕe, kʰrɯtsu ɯ-ro jamar ɲɯ-fsoʁ ɲɯ-cʰa \\
one-one-year  \textsc{lnk} \textsc{irr}-\textsc{ipfv}-be.good \textsc{lnk} ten.thousand \textsc{3sg}.\textsc{poss}-excess about \textsc{ipfv}-earn \textsc{sens}-can \\ 
\glt `Some years if [his income] is good, he can earn more than ten thousands [renminbi].' (14-siblings, 180)
(\japhdoi{0003508\#S161})
\end{exe}

An alternative distributed form involves partial reduplication of the stem of the counted noun and replacing the numeral prefix by a third singular possessive \forme{ɯ\trt}, as in the conversion from counted noun to inalienably possessed noun (§\ref{sec:CN.IPN}). In (§\ref{sec:WphWphW}), the counted noun \japhug{tɯ-pʰɯ}{one tree} is changed to \japhug{ɯ-pʰɯ\redp{}pʰɯ}{some trees}. The other distributed form \forme{tɯ-tɯ-pʰɯ} could also be used in the same context without meaning difference. 

\begin{exe}
\ex \label{sec:WphWphW}
\gll zgoku kɯ-mbro tɕe, ɕkrɤz kɯ-wxti ra nɯ ɯ-rcʰɤβ ri ɯ-pʰɯ\redp{}pʰɯ ʑo tu tɕe \\
mountain \textsc{nmzl}:S/A-be.tall \textsc{lnk} oak \textsc{nmzl}:S/A-be.big \textsc{pl} \textsc{dem} \textsc{3sg}.\textsc{poss}-between \textsc{loc} \textsc{3sg}.\textsc{poss}-\textsc{part}\redp{}tree \textsc{emph} exist:\textsc{fact} \textsc{lnk} \\
\glt `On high mountains, among big oaks, there are some [of these little trees].' (16-CWrNgo, 177)
(\japhdoi{0003518\#S168})
\end{exe}   

This distributed form puts emphasis on the non-contiguousness and spread over distribution of the entities designated by the reduplicated counted noun: in (§\ref{sec:WphWphW}) for instance, its presence implies that the little trees are not clustered, but rather scattered among the oaks.

Distributed counted nouns can also be repeated to put even more emphasis on the scattered distribution (see §\ref{sec:CN.repetition}).

\subsubsection{Historical perspectives on the numeral prefixal paradigm} \label{sec:num.prefix.paradigm.history}
Most non-Tibetan languages of Western Sichuan/Northern Yunnan have counted nouns (generally called `classifiers', see §\ref{sec:CN.classification}) with numeral-counted noun order (see for instance \citealt{zhang14classifiers}, \citealt[163--194]{michaud17yongning}).  It is striking that among the quasi-isolating languages (Lolo-Burmese, Naish), numeral prefix paradigms are commonly a pocket of irregular morphology (\citealt{bradley05numerals}, \citealt{michaud11cl}); this is also true in some Hmong-Mien  languages (see \citealt{gerner10classifier.isolating}).

By contrast, while Japhug and the other Gyalrong languages have a richer morphology in general, the numeral prefixal paradigms are, with only few exceptions (§\ref{sec:irregular.numeral.prefixes}), suspiciously regular. The historical interpretation of this observation is not completely straightforward \citep{jacques17num}, but in any case counted nouns are morphologically like determinative compounds (§\ref{sec:determinative.n.n}) with a numeral as first element, undergoing bound state in the case of more integrated numerals (some of the numerals under 100, see §\ref{sec:num.prefixes.1.10},  §\ref{sec:num.prefixes.11.99} and §\ref{sec:approximate.numeral.prefixes}) and immune from it in the case of higher numerals (§\ref{sec:other.numeral.prefixes}).

One morphological alternation common to all numeral prefixes under 10 is the loss of the codas, otherwise a rare phenomenon in noun compounds (see §\ref{sec:loss.codas.compounds}). For numerals above 10, there is some degree of free variation in the preservation of the codas (see for instance example \ref{ex:GnAsqaptWWGrZaR} p.\pageref{ex:GnAsqaptWWGrZaR}); the forms preserving codas are rarer, and may be ongoing analogical levelling.

Even in the case of lower numerals, there are indirect traces of the former existence of codas, in particular in irregularities (§\ref{sec:irregular.numeral.prefixes}). A few counted nouns like \japhug{tɯ-xpa}{one year} and \japhug{tɯ-ɣjɤn}{one time} have velar fricative preinitials \forme{x-/ɣ-} which are lacking in some forms of the paradigm. 

In the case of \japhug{tɯ-xpa}{one year}, note the alternative stem \forme{-pa} (\japhug{sqɯ-xpa}{ten years} vs. \japhug{sqɯ-pa}{ten years}), the time ordinals (§\ref{sec:time.ordinals}), the adverb \japhug{pakuku}{every year} and the verb  \japhug{pa}{pass $X$ years} (see example \ref{ex:40.topa}, §\ref{sec:pa.intr.lv}) from which the counted noun \japhug{tɯ-xpa}{one year} is historically derived. Cognates of this root \forme{-pa} appear in other Burmo-Gyalrongic languages such as Naish (\citealt{jacques.michaud11naish}) but are restricted to time ordinals, and do not occur as counted nouns. Even in the closely related Stau language,  time ordinals such as \stau{javə}{last year} and \stau{pəvə}{this year} have a root \forme{-və} cognate to Japhug \forme{-pa}, but the corresponding counted noun \stau{e-fku}{one year} has a different root which left no trace in the Gyalrong languages. The generalization of the \forme{-pa} root to the counted paradigm is probably a common Gyalrong innovation (see §\ref{sec:CN.verbs}). 

A possible explanation for this \forme{x-/ɣ-} element is that it originates from the coda of the numeral `one' (although replaced by \japhug{ci}{one}, this former numeral is still found in the element \forme{-tɯɣ} in \japhug{sqaptɯɣ}{eleven}) through false segmentation (\forme{*tɯk-pa} $\rightarrow$ \forme{*tɯ-kpa} $\rightarrow$ \forme{*tɯ-xpa} `one year') and subsequent generalization to the whole paradigm.\footnote{Note that since proto-Gyalrong \forme{*kp-} regularly yields \forme{βɣ-} with metathesis (\citealt[272]{jacques04these}), this false segmentation must have occurred \textit{after} the \forme{*kp-} $\rightarrow$\forme{βɣ-} sound change, which is not shared with other Gyalrong languages. } It is not an example of \forme{x-/ɣ-} nominalization (§\ref{sec:G.nmlz}).


\subsection{Counted nouns in quantifying function} \label{sec:CN.quantifier}
This section discusses the use of counted nouns in  quantifying function, as noun modifiers, head of a noun phrase or a sentential quantifier.

The main functions of counted nouns include partitive, distributive, restrictive and iterative meanings.

\subsubsection{Partitive} \label{sec:CN.partitive}
When occurring as postnominal modifiers, most counted nouns are essentially \textit{partitive} in meaning, referring to a certain number of individuals from a group, and are not used to express indefiniteness (§\ref{sec:CN.definiteness}): the polyfunctional determiner \japhug{ci}{one}  (§\ref{sec:indef.article}) occurs instead in this meaning.

Thus, a phrase such as  \forme{tɯrme tɯ-rdoʁ} combining the noun \japhug{tɯrme}{person} with the generic counted noun \japhug{tɯ-rdoʁ}{one piece} (see § §\ref{sec:CN.classification} on counted noun selection) is generally either to be translated as a partitive `one of them'  as in (\ref{ex:tWrme.tWrdoR.RnWz}).

\begin{exe}
	\ex \label{ex:tWrme.tWrdoR.RnWz}
	\gll tɯrme ʁnɯz pjɤ-tu tɕe, tɯrme tɯ-rdoʁ nɯ rcanɯ, ɯ-stɤrju ʁɟa ʑo tu-βze tɕe, \\
	people two \textsc{ifr}.\textsc{ipfv}-exist \textsc{lnk} people one-piece \textsc{dem} \textsc{unexp}:\textsc{deg} \textsc{3sg}.\textsc{poss}-truth completely \textsc{emph} \textsc{ipfv}-do[III] \textsc{lnk} \\
	\glt `There were two persons, one of them always told the truth (and the other one was a liar).' (140427 yuanhou-zh)
(\japhdoi{0003870\#S3})
\end{exe} 

The partitive meaning is found even when the individual counted noun occurs on its own without overt head noun as in (\ref{ex:tWrdoR.cinA}).

\begin{exe}
	\ex \label{ex:tWrdoR.cinA}
	\gll ɯ-tɕɯ kɯβde pɯ-tu ri, ɯʑo kɯ-fse kɯ-ɕqraʁ tɯ-rdoʁ cinɤ pɯ-me ɲɯ-ŋu 	\\
	\textsc{3sg}.\textsc{poss}-son four \textsc{pst}.\textsc{ipfv}-exist but \textsc{3sg} \textsc{sbj}:\textsc{pcp}-be.like \textsc{sbj}:\textsc{pcp}-be.intelligent one-piece even \textsc{pst}.\textsc{ipfv}-not.exist \textsc{sens}-be \\
	\glt `He had four sons, but not even one of them was smart like him.' (2005tAwakWcqraR, 3)
\end{exe} 


The partitive meaning is found in particular when counted nouns modify mass nouns, such as \japhug{paʁɕa}{pork} in (\ref{ex:paʁɕa.tWrdoR}).  

\begin{exe}
	\ex \label{ex:paʁɕa.tWrdoR}
	\gll pakuku ʑo paʁɕa tɯ-rdoʁ, qʰe tɕe tɕe stoʁ cʰɯ-ɣɯt, \\
	every.year \textsc{emph} pork one-piece \textsc{lnk} \textsc{lnk} \textsc{lnk} broad.bean \textsc{ipfv}:\textsc{downstream}-bring \\
	\glt `Every year, he would bring a piece of pork and broad beans.' (140501 tshering scid, 35)
(\japhdoi{0003902\#S34})
\end{exe} 

\subsubsection{Distributive} \label{sec:CN.distributive}
Counted nouns can also have a distributive meaning  `each of them' when another numeral or counted noun occurs in the same clause, indicating that each of the members of the group performs the same action, the verb can either receive plural (as in \ref{ex:tWrme.tWrdoR.kW} and \ref{ex:tWrZaR}) or singular indexation (example \ref{ex:tWrme.tWrdoR2}). This is a particular case of fluid number indexation (§\ref{sec:agreement.mismatch}).

\begin{exe}
	\ex \label{ex:tWrme.tWrdoR.kW}
	\gll  tɕe tɕe tɯrme tɯ-rdoʁ kɯ kʰɯna ʁnɯz χsɯm jamar tu-ndo-nɯ \\
	\textsc{lnk} \textsc{lnk}  person one-piece \textsc{erg} dog two three about \textsc{ipfv}-take-\textsc{pl} \\
	\glt `Each [of the hunters] takes two or three dogs.' (150829 KAGWcAno, 25)
(\japhdoi{0006420\#S24})
\end{exe} 

\begin{exe}
	\ex \label{ex:tWrme.tWrdoR2}
	\gll tɯrme tɯ-rdoʁ kɯ cʰɤmdɤru tɯ-ldʑa tu-nɯ-ndɤm  \\
	people one-piece \textsc{erg} drinking.straw one-\textsc{cl} \textsc{ipfv}-\textsc{auto}-take[III] \\
	\glt `Each person takes one straw.' (30-tChorzi, 40)
(\japhdoi{0003760\#S33})
\end{exe}
 

Counted noun with a collective meaning (§\ref{sec:collective.counted.noun}) such as \japhug{tɯ-tɯpɯ}{one household} also frequently have a partitive meaning, especially when combined with a numeral.

\begin{exe}
	\ex \label{ex:tWtWpW.tCe}
	\gll tɕe tɯ-tɯpɯ tɕe, ɯ-qaʑo kɯβdɤsqi, kɯmŋɤsqi jamar pɯ-tu. \\
	\textsc{lnk} one-household \textsc{lnk} \textsc{3sg}.\textsc{poss}-sheep forty fifty about \textsc{pst}.\textsc{ipfv}-exist \\
	\glt `Each household used to have about forty or fifty sheep.' (160712 smAG, 24)
(\japhdoi{0006073\#S23})
\end{exe} 


Temporal counted nouns such as \japhug{tɯ-sŋi}{one day} (§\ref{sec:CN.time}) are generally repeated when used in distributive function in the meaning `every (day)' (§\ref{sec:CN.repetition}). However, in combination with other counted nouns, they can mean `per ...' as in (\ref{ex:tWkhWtsa.tWrdoR}), even without repetition.

\begin{exe}
	\ex \label{ex:tWkhWtsa.tWrdoR}
	\gll tɯ-sŋi tɕe tɯ-kʰɯtsa jamar tɯ-rdoʁ kɯ pjɯ-tsʰi ɲɯ-cʰa. \\
	one-day \textsc{lnk} one-bowl about one-\textsc{cl} \textsc{erg} \textsc{ipfv}-drink \textsc{sens}-can \\
	\glt `One [cat] can drink about one bowl of milk per day.' (21-lWLU, 48)
(\japhdoi{0003576\#S44})
\end{exe}

When no other numeral or counted noun is present, the modifier of Tibetan origin \japhug{raŋri}{each} (from \tibet{རང་རེ་}{raŋ.re}{each}; see also §\ref{sec:raNri}) can be used to specify the distributive meaning as in (\ref{ex:tWtWpW.rANri}).

\begin{exe}
	\ex \label{ex:tWtWpW.rANri}
	\gll tɯ-tɯpɯ raŋri ɣɯ, nɯ-mbro pjɤ-tu... \\
	one-household  each \textsc{gen} \textsc{3pl}.\textsc{poss}-horse \textsc{ifr}.\textsc{ipfv}-exist \\
	\glt `Each household used to have horses etc.' (150820 kAnWCkat, 2)
\end{exe} 

Counted nouns can have a combined distributive and partitive meaning `each one of their...', such as \japhug{tɯ-ntsi}{one of a pair} in (\ref{ex:nWXpWm.tWntsi}).

\begin{exe}
	\ex \label{ex:nWXpWm.tWntsi}
	\gll tɕe tɕʰeme ra kɯ nɯ-χpɯm tɯ-ntsi ka-ta-nɯ, tɤ-tɕɯ ra kɯ nɯ-χpɯm ʁnɯz ka-ta-nɯ ri,  \\
	\textsc{lnk} girl \textsc{pl} \textsc{erg} \textsc{3pl}.\textsc{poss}-knee one-of.a.pair \textsc{aor}:3\flobv{}-put-\textsc{pl} \textsc{indef}.\textsc{poss}-son \textsc{pl} \textsc{erg}  \textsc{3pl}.\textsc{poss}-knee two \textsc{aor}:3\flobv{}-put-\textsc{pl} \textsc{lnk} \\
	\glt `[Each of] the girls (in their group) put one of their knees, the boys put their two knees [as a support for the tea kettle].' (2005-stod-kunbzang, 179)
\end{exe}


\subsubsection{Repetition of counted nouns} \label{sec:CN.repetition}
Repeating a counted nouns with the same numeral prefix has either a distributive or a distributed meaning.\footnote{If the numeral prefixes are different however, an approximate numeral interpretation results, see §\ref{sec:approximate.numeral.prefixes}.}

Example (\ref{ex:tWldZa.tWldZa}) illustrates the distributive meaning (`each', `each single', `one by one') of counted noun repetition with an individual counted noun (\japhug{tɯ-ldʑa}{one long object}) and also a partitive counted noun  (\japhug{tɯ-spra}{one handful}). Repeated counted nouns of measure can be used to refer to the unit in which a whole mass of elements (here the hemp stalks) is divided into (`into bundles', `bundle by bundle').

\begin{exe}
	\ex \label{ex:tWldZa.tWldZa}
	\gll  tɤsɤmu nɯ tɕe tɯɲcɣa tú-wɣ-lɤt tɕe, [...] tɯ-ldʑa tɯ-ldʑa ɲɯ́-wɣ-pʰɯt tɕe tɕe nɯnɯ li tɯ-spra tɯ-spra tú-wɣ-xtɕɤr. \\
	hemp \textsc{dem} \textsc{lnk} sickle \textsc{ipfv}-\textsc{inv}-throw \textsc{lnk} { } one-long.object one-long.object \textsc{ipfv}-\textsc{inv}-pluck \textsc{lnk} \textsc{lnk} \textsc{dem} again one-handful one-handful \textsc{ipfv}-\textsc{inv}-tie  \\
	\glt `As for hemp, one uses a sickle and cuts [the stalks] one by one and then ties them into bundles.'  (14-tasa)
(\japhdoi{0003510\#S50})
\end{exe}

The distributive meaning also occurs with repeated temporal counted nouns, as in (\ref{ex:tWxpa.tWxpa}).

\begin{exe}
	\ex \label{ex:tWxpa.tWxpa}
	\gll tɕe tɯ-xpa tɯ-xpa tu-ɬoʁ qʰe tɕe qartsɯ tɕe pjɯ-kʰrɯ ɕti. \\
	\textsc{lnk} one-year one-year \textsc{ipfv}-come.out \textsc{lnk} \textsc{lnk} winter \textsc{lnk}  \textsc{ipfv}-be.dry be.\textsc{aff}:\textsc{fact} \\
	\glt  `It grows every year (i.e. it is an annual plant), and dies in winter.' (140512 tAzraj, 10)
(\japhdoi{0003971\#S10})
\end{exe}

Alternatively, rather than juxtaposing counted nouns, coordinating them with the additive \forme{nɤ} (§\ref{sec:additive.nA}) as in (\ref{ex:tWCha.nA.tWCha}) also results in a distributive meaning.

\begin{exe}
	\ex \label{ex:tWrdoʁ.nA.tWrdoʁ}
	\gll  tɕe nɯŋa ndɤre ɯ-pɯ nɯ tɯ-rdoʁ nɤ tɯ-rdoʁ ma me tɯ-xpa tɯ-ɣjɤn ma ɲɯ-rɤpɯ mɤ-cʰa \\
	\textsc{lnk} cow \textsc{top}.\textsc{advers} \textsc{3sg}.\textsc{poss}-offspring \textsc{dem} one-piece \textsc{add} one-piece apart.from not.exist:\textsc{fact}  one-year one-time apart.from \textsc{ipfv}-bear.young \textsc{neg}-can:\textsc{fact} \\
	\glt `As for cows, they [have] their young only one by one, and can only bear young once a year.'
(\japhdoi{0003404\#S123})
\end{exe}

\begin{exe}
	\ex \label{ex:tWCha.nA.tWCha}
	\gll nɯnɯ mɯrmɯmbju nɯ tɯ-tɕʰa nɤ tɯ-tɕʰa ntsɯ tɯtɯrca ntsɯ ku-rɤʑi-nɯ ŋu tɕe \\
	\textsc{dem} swallow \textsc{dem} one-pair \textsc{add}  one-pair  always together always \textsc{ipfv}-stay-\textsc{pl} be:\textsc{fact} \textsc{lnk}  \\
	\glt `Swallows are always in pairs.' (03-mWrmWmbjW, 56)
\end{exe}

Repeated counted nouns also express a distributed meaning as in (\ref{ex:tWrdoR.tWrdoR}). Like the distributed numeral prefixes (§\ref{sec:numeral.prefixes.distributed}), this construction indicates that the entities referred to by the counted nouns are scattered more or less homogeneously.

\begin{exe}
	\ex \label{ex:tWrdoR.tWrdoR}
	\gll  tɯ-rdoʁ tɯ-rdoʁ kɯ-fse tu-ɬoʁ ŋu ma mɤ-arɤkʰɯmkʰɤl. \\
	one-piece one-piece \textsc{sbj}:\textsc{pcp}-be.like \textsc{ipfv}-come.out  be:\textsc{fact} \textsc{lnk} \textsc{neg}-be.heterogeneously.distributed:\textsc{fact} \\
	\glt `It grows in scattered fashion, not in clusters here and there.' (22-BlamajmAG, 132)
(\japhdoi{0003584\#S124})
\end{exe}

It is also possible to repeat counted nouns with distributed numeral prefixes to emphasize even more the scattered distribution as in (\ref{ex:tWtWrdoR.tWtWrdoR}).

\begin{exe}
	\ex \label{ex:tWtWrdoR.tWtWrdoR}
	\gll 
	tɕeri tɯ-tɯ-rdoʁ tɯ-tɯ-rdoʁ ɲɯ-ŋu ma kɯ-ɤndʑɯrɣa kɯ-fse kɯ-ɤrɤkhɯmkʰɤl kɯ-fse maŋe.  \\
	\textsc{lnk} one-one-piece one-one-piece \textsc{sens}-be \textsc{lnk}  \textsc{sbj}:\textsc{pcp}-be.neighbours \textsc{sbj}:\textsc{pcp}-be.like \textsc{sbj}:\textsc{pcp}-be.heterogeneously.distributed \textsc{sbj}:\textsc{pcp}-be.like not.exist:\textsc{sens} \\
	\glt  `They are scattered one by one, and are not together in clusters.' (24-zwArqhAjmAG, 81)
\end{exe}


\subsubsection{Restrictive} \label{sec:CN.restrictive}

The only context where the phrase \forme{tɯrme tɯ-rdoʁ} consistently means `one person' is in restrictive constructions (`only one person') with the exceptive postposition \japhug{ma}{apart from} (§\ref{sec:exceptive}), as in (\ref{ex:tWrme.tWrdoR.ma.me}).  In negative restrictive constructions with \japhug{cinɤ}{(not) even one}, individual counted nouns with the numeral \forme{tɯ-} `one' also occur as in (\ref{ex:Wjwaʁ.tWmpCar.cinA}).

\begin{exe}
	\ex \label{ex:tWrme.tWrdoR.ma.me}
	\gll tʰam tɯrme tɯ-rdoʁ ma me \\
	now people one-piece apart.from not.exist:\textsc{fact} \\
	\glt `Now there is only one person [in that place].' (140522 tshupa, 35)
(\japhdoi{0004053\#S35})
\end{exe} 

\begin{exe}
	\ex \label{ex:Wjwaʁ.tWmpCar.cinA}
	\gll   qartsɯ tɕe ɯ-jwaʁ tɯ-mpɕar cinɤ ɯ-ku kɯ-ndzoʁ me.  \\
	winter \textsc{lnk} \textsc{3sg}.\textsc{poss}-leaf one-leaf not.even \textsc{3sg}.\textsc{poss}-head \textsc{sbj}:\textsc{pcp}-\textsc{anticaus}:attach exist:\textsc{fact} \\ 
	\glt `In winter, not even one leaf [remains] on it.' (11-mYAm, 37)
(\japhdoi{0003474\#S33})
\end{exe} 

As illustrated by the following pair of examples (from a similar episode in two traditional stories), in this construction both noun+individual counted noun  (\ref{ex:tWrme.tWrdoR.ma.maNetCi})\footnote{The \textsc{1du} suffix on the existential verb \forme{maŋe-tɕi} is a case of partitive indexation (§\ref{sec:partitive.indexation}). }  or plain numeral (\ref{ex:RnWz.ma.maNetCi}) can occur.

\begin{exe}
	\ex \label{ex:tWrme.tWrdoR.ma.maNetCi}
	\gll tɯrme ʁnɯ-rdoʁ ma maŋe-tɕi tɕe, kɤ-ɤnɯndzɤqɯqɤr mɤ-nɯ-cʰa-tɕi, \\
	people two-piece apart.from not.exist:\textsc{sens}-\textsc{1du} \textsc{lnk} \textsc{inf}-\textsc{recip}:eat.on.one's.own \textsc{neg}-\textsc{auto}-can:\textsc{fact}-\textsc{1du} \\
	\glt `There are only two of us, we cannot eat on our own [without sharing with each other].' (2003kunbzang, 100)
\end{exe} 

\begin{exe}
	\ex \label{ex:RnWz.ma.maNetCi}
	\gll  tɕiʑo ʁnɯz ma maŋe-tɕi tɕe, ʑaka kɤ-nɯ-βzu mɤ-rtaʁ-tɕi \\
	\textsc{1du} two apart.from not.exist:\textsc{sens}-\textsc{1du} \textsc{lnk} each \textsc{inf}-\textsc{auto}-make \textsc{neg}-be.enough:\textsc{fact}-\textsc{1du} \\
	\glt  `There are only two of us, there are not enough of us to each act on our own.' (2002 qaCpa, 220)
\end{exe} 

In examples (\ref{ex:tWrme.tWrdoR.Wndzxa}) and (\ref{ex:tAYi.tWldZa}), also with noun+individual counted noun, there is no specific restrictive construction, but there is an implicit restrictive meaning (`because of (just) one person', `only one staff'). 

\begin{exe}
	\ex \label{ex:tWrme.tWrdoR.Wndzxa}
	\gll tɯrme tɯ-rdoʁ ɯ-ndʐa nɯ ʑo to-stu-nɯ ɕti ri, \\
	people one-piece  \textsc{3sg}.\textsc{poss}-reason \textsc{dem} \textsc{emph} \textsc{ifr}-do.like-\textsc{pl} be.\textsc{aff}:\textsc{fact} \textsc{lnk} \\
	\glt `They did all that because of one person.' (2003smanmi-tamu, 101)
\end{exe} 

\begin{exe}
	\ex \label{ex:tAYi.tWldZa}
	\gll nɯʑora ɣɯ nɯ-ɕɤmɯɣdɯ cʰo kɯ-fse nɯ ɯ-tsʰɤt nɯ, tɕiʑo ɣɯ tɕi-tɤɲi tɯ-ldʑa pɯ-tu tɕe, nɯ kɤ-nɯ-tʰɯ-tɕi ɕti wo \\
	\textsc{2pl} \textsc{gen} \textsc{2pl}.\textsc{poss}-gun \textsc{comit} \textsc{sbj}:\textsc{pcp}-like \textsc{dem} \textsc{3sg}.\textsc{poss}-instead \textsc{dem}  \textsc{1du} \textsc{gen} \textsc{1du}.\textsc{poss}-staff one-long.object \textsc{pst}.\textsc{ipfv}-exit \textsc{lnk} \textsc{dem} \textsc{aor}-\textsc{auto}-spread-\textsc{1du} be.\textsc{aff}:\textsc{fact} \textsc{fsp} \\
	\glt `Instead of guns like you have, we (only) had one staff, and we laid it over [the river to walk on it as a bridge].' (2003 Kunbzang, 188)
\end{exe} 

Individual counted nouns can also express a combination of partitive and restrictive meaning, `just $X$ of them', as in (\ref{ex:tWrdoR.mapWwGsat}).

\begin{exe}
	\ex \label{ex:tWrdoR.mapWwGsat}
	\gll 
	tɯ-rdoʁ ma-pɯ́-wɣ-sat ra ma tɕe rcanɯ, kɤrɤxpa ʑo cʰɯ-ɕar tɕe nɯ ɯ-zda nɯ kɯ ɲɯ-ɕar nɤ ɲɯ-ɕar tɕe \\
	one-piece \textsc{neg}.\textsc{imp}-\textsc{pfv}-\textsc{inv}-kill be.needed:\textsc{fact} \textsc{lnk} \textsc{lnk} \textsc{unexp}:\textsc{deg} several.years \textsc{emph} \textsc{ipfv}-search \textsc{lnk} \textsc{dem} \textsc{3sg}.\textsc{poss}-companion \textsc{dem} \textsc{erg} \textsc{ipfv}-search \textsc{lnk} \textsc{ipfv}-search \textsc{lnk} \\
	\glt `One should not just kill one of them, otherwise it will look for it for years, its mate will search and search for it.' (22-qomndroN)
\end{exe}

Another type of restricted meaning is, in the case of mass nouns such as \japhug{rdɤstaʁ}{stone}, the meaning `in one piece' as in example (\ref{ex:rdAstaR.tWldZa}), with the individual counted nouns \japhug{tɯ-ldʑa}{one long object} or \japhug{tɯ-rdoʁ}{one piece} (the storyteller hesitates between the two) occurring here not as noun modifier, but as nominal predicates with the participle of the copula \japhug{ŋu}{be} (§\ref{sec:copula.basic}). In this example, \forme{tɯ-rdoʁ} is in this analysis a noun phrase on its own.

\begin{exe}
	\ex \label{ex:rdAstaR.tWldZa}
	\gll rdɤstaʁ tɯ-ldʑa, tɯ-rdoʁ kɯ-ŋu kɯ ɣɟɯ χsɯ-ldʑa rɟɤlsa ɯ-ʁɤri kutɕu a-pɯ-tu, \\
	stone one-long.object one-piece \textsc{sbj}:\textsc{pcp}-be \textsc{erg} watchtower three-long.object palace \textsc{3sg}.\textsc{poss}-front here \textsc{irr}-\textsc{ipfv}-exist \\
	\glt `If from one monolith (a stone which is in one piece), three  towers were made here in front of the palace, (I would feel much better).'  (2003smanmi-tamu, 120)
\end{exe} 

Unlike languages such as Chinese, where the meaning  `alone' can be expressed by the numeral `one' with a quantifier (\ch{一个人}{yīgèrén}{one person, alone}), this meaning is not normally conveyed by counted nouns in Japhug, but rather by the adverbial root \forme{-sti} `alone' (see §\ref{sec:stWsti}).

\subsubsection{Sentential quantifying function } \label{sec:CN.iterative}
Some counted nouns, rather than being used as postnominal quantifiers, can have scope over the whole sentence. Two constructions can be distinguished: the iterative/semelfactive construction, in which the counted nouns designate the number of time that an action takes place, and the pseudo-object construction.

Only a minority of counted nouns can be used in the iterative/semelfactive construction. This group includes temporal counted nouns like \japhug{tɯ-ɣjɤn}{one time} (§\ref{sec:CN.time}),  counted nouns derived from verbs (§\ref{sec:CN.verbs})  such as \japhug{tɯ-tɤtɕʰɯ}{hitting with the hoe one time} (from \japhug{tɕʰɯ}{hit, gore}) and \japhug{tɯ-tɤfskɤr}{one turn} (from \japhug{fskɤr}{turn around}) or from ideophones  (§\ref{sec:CN.ideophones}) such as \japhug{tɯ-tɤxɯr}{one turn}.

Semelfactive counted nouns can either be used as objects of an auxiliary verb, mainly \japhug{lɤt}{release} (§\ref{sec:lAt.lv}), as in (\ref{ex:kAntChWtAtChW}) and (\ref{ex:RnWtAxWr}), or as sentential quantifiers (§\ref{sec:CN.iterative}), as in (\ref{ex:laRnWtAxWr}) with the intransitive verb \japhug{mtɕɯr}{turn}.

\begin{exe}
	\ex \label{ex:kAntChWtAtChW}
	\gll qaʁ kɤntɕʰɯ-tɤtɕʰɯ to-lɤt \\
	hoe several-hitting.with.the.hoe \textsc{ifr}-release \\
	\glt `He hit several times with the hoe.' (elicited)
\end{exe}

\begin{exe}
	\ex \label{ex:RnWtAxWr}
	\gll ʁnɯ-tɤxɯr to-lɤt ɲɯ-ŋu \\
	two-turn \textsc{ifr}-throw \textsc{sens}-be \\
	\glt `He ran two laps.' (2003sras, 210)
\end{exe}

\begin{exe}
	\ex \label{ex:laRnWtAxWr}
	\gll χcʰa pɕoʁ laʁnɯ-tɤxɯr ku-mtɕɯr, ʁe pɕoʁ laʁnɯ-tɤxɯr ku-mtɕɯr ŋu. \\
	right side a.few-turn \textsc{ipfv}-turn left side a.few-turn \textsc{ipfv}-turn be:\textsc{fact} \\
	\glt `It turns several times on the right, and several times one the left.' (150826 qro kWnWkhABGa)
(\japhdoi{0006362\#S5})
\end{exe}


In the pseudo-object construction, counted nouns semantically related to the \textit{patient} of the main verb (even in intransitive constructions lacking an object) occur to express the meaning `not even one ...' (in a negative construction) or `a few...'. This construction is most clearly illustrated with intransitive verbs. In examples (\ref{ex:tWtWkWr.cinA}) and (\ref{ex:laRnWNka.rWCmitCi}), the intransitive verbs with \japhug{rɯndzɤtsʰi}{have a meal} and \japhug{rɯɕmi}{speak} cannot take overt object noun phrases (they are not semi-transitive, see §\ref{sec:intr.goal}), but occur here with counted noun quantifiers referring to the non-overt patients (the food in \ref{ex:tWtWkWr.cinA} and the words in \ref{ex:laRnWNka.rWCmitCi}), which would be the object of the corresponding transitive verbs \japhug{ndza}{eat} and  \japhug{ti}{say}.


\begin{exe}
	\ex \label{ex:tWtWkWr.cinA}
	\gll tɤ-pɤtso ra tɯ-mu ɯ-xɕɤt kɯ tɯ-tɯ-kɯr cinɤ ʑo kɤ-rɯndzɤtsʰi mɯ-pjɤ-cʰa-nɯ. \\
	\textsc{indef}.\textsc{poss}-child \textsc{pl} \textsc{nmlz}:action-be.afraid \textsc{3sg}.\textsc{poss}-strength \textsc{erg} one-\textsc{indef}.\textsc{poss}-mouth even \textsc{emph} \textsc{inf}-eat \textsc{neg}-\textsc{ifr}.\textsc{ipfv}-can-\textsc{pl} \\
	\glt `The children were so afraid that they could not even eat one mouthful.' (160704 poucet4-v2, 52)
(\japhdoi{0006097\#S52})
\end{exe}

\begin{exe}
	\ex \label{ex:laRnWNka.rWCmitCi}
	\gll laʁnɯ-ŋka rɯɕmi-tɕi \\
	few-word speak:\textsc{fact}-\textsc{1du} \\
	\glt `Let us speak a few words.' (elicited)
\end{exe}

In these examples, adding an overt noun would be agrammatical, and the quantifiers \forme{tɯ-tɯ-kɯr cinɤ ʑo} `not even one mouthful' and \japhug{laʁnɯ-ŋka}{a few words} cannot be analyzed as objects. Rather, they are sentential adverbs, whose grammatical status is comparable to that of temporal counted nouns (§\ref{sec:CN.time}).


\subsubsection{Other}
We also find examples of counted nouns without partitive, distributive, restricted or iterative meaning.

This the case in particular for counted nouns with a collective meaning, denoting a group of entities whose quantity can be precise (\japhug{tɯ-tɕʰa}{one pair}) or unspecified (\japhug{tɯ-boʁ}{one group}, \japhug{tɯ-tɯpɯ}{one household} or \japhug{tɯ-tɯpʰu}{one hive}). In (\ref{ex:taRrdp.RnWtWpW}) for instance, \forme{ʁnɯ-tɯpɯ} means `two households', not `two of the households' or `each of the households'. Note also the plural (rather than dual) indexation on the verb \forme{tu-nɯ}, showing the inherent collective meaning of the counted noun \japhug{tɯ-tɯpɯ}{one household}. Plural indexation here is optional (§\ref{sec:optional.indexation}). In this example, \forme{ʁnɯ-tɯpɯ} is \textit{not} a modifier of the placename \forme{taʁrdo}: this placename is an absolutive locative adjunct (§\ref{absolutive.locative}).

\begin{exe}
	\ex \label{ex:taRrdp.RnWtWpW}
	\gll tʰam taʁrdo ʁnɯ-tɯpɯ tu-nɯ \\
	now placename two-household exist-\textsc{pl} \\
	\glt `Now there are two households in Tarrdo.' (140522 tshupa, 19)
\end{exe} 


This number quantification meaning is also attested with non-collective counted nouns. For instance, in (\ref{ex:xCAndZu.XsWldZa}), \forme{χsɯ-ldʑa} `three (long objects)' occurs in the same context as the numerals \japhug{ci}{one} and \japhug{χsɯm}{three}. In (\ref{ex:si.tWphW.pjAtu}) likewise, it is clear from the context that neither a partitive, distributive, nor restrictive interpretation is possible.

\begin{exe}
	\ex \label{ex:xCAndZu.XsWldZa}
	\gll tɕendɤre tɤ-tɕɯ nɯ kɯ tɤɲi ci na-ɕar xɕɤndʑu χsɯ-ldʑa, qapi χsɯm kʰɤzɟi ɯ-ŋgɯ pa-rku ɲɯ-ŋu. \\
	\textsc{lnk} \textsc{indef}.\textsc{poss}-son dem \textsc{erg} staff one \textsc{aor}:3\flobv{}-search twig three-long.object white.stone three feedbag \textsc{3sg}.\textsc{poss}-inside \textsc{aor}:3\flobv{}-put.in \textsc{sens}-be \\
	\glt `The boy looked for a staff, and put three twigs and three stones in the (horse's) feedbag.' (2005-stod-kunbzang, 148)
\end{exe} 

\begin{exe}
	\ex \label{ex:si.tWphW.pjAtu}
	\gll tɕe nɯtɕu tɕe tʂu ɯ-rkɯ zɯ si tɯ-pʰɯ pjɤ-tu,  \\
	\textsc{lnk} \textsc{dem}:\textsc{loc} \textsc{lnk} path \textsc{3sg}.\textsc{poss}-side \textsc{loc} tree one-tree \textsc{ifr}.\textsc{ipfv}-exist \\
	\glt `There, on the side of the road, there was a tree.' (The divination 2002, 10)
(\japhdoi{0003364\#S10})
\end{exe} 

However, this function is not very widespread in native texts.

In texts translated from Chinese or elicited material however, this usage is very common, as speakers will easily calque the Chinese noun+classifier construction. In example (\ref{ex:qaJY.tWldZa}) for instance \forme{qaɟy tɯ-ldʑa} `one fish' is very probably calqued from Chinese \zh{一条鱼} \forme{yī-tiáo yú} (one-\textsc{cl} fish).\footnote{The original text from which (\ref{ex:qaJY.tWldZa}) was translated is \zh{不久,老三又看见了一条鱼和一只狐狸}. The counted noun \japhug{tɯ-ldʑa}{one long object} however is really applied to fishes, snakes and worms in Japhug even in non-translated texts (§\ref{sec:CN.classification}).  } 

\begin{exe}
	\ex \label{ex:qaJY.tWldZa}
	\gll qaɟy tɯ-ldʑa cʰondɤre qacʰɣa ci pjɤ-mto ɲɯ-ŋu \\
	fish one-long.object \textsc{comit} fox one \textsc{ifr}-see \textsc{sens}-be \\
	\glt `He saw a fish and a fox.' (140505 xiaohaitu-zh, 41)
(\japhdoi{0003921\#S40})
\end{exe} 

Future studies on the use of counted noun as quantifiers should be therefore exclusively based on texts not translated from Chinese and conversation, not on translations and elicitation.


\subsubsection{Definiteness} \label{sec:CN.definiteness}
In Japhug, counted nouns are not specifically used to mark indefiniteness; they can even occur with demonstratives such as \japhug{nɯ}{that} in noun phrases with a definite referent. In  (\ref{ex:tWkhWtsa.nW}) and (\ref{ex:tWphW.nW}) for instance, the bowl of oil and the tree in question were mentioned earlier in the  the story and are clearly definite.

\begin{exe}
	\ex \label{ex:tWkhWtsa.nW}
	\gll kʰa cʰɤ-zɣɯt tɕe tɯ-kri tɯ-kʰɯtsa nɯ ko-ɕkɯt.  \\
	house \textsc{ifr}:\textsc{downstream}-reach \textsc{lnk} \textsc{indef}.\textsc{poss}-oil one-bowl \textsc{dem} \textsc{ifr}-drink.completely \\
	\glt `He arrived at the house and drank the bowl of oil.' (140501 mdzadi, 23)
\end{exe}

\begin{exe}
	\ex \label{ex:tWphW.nW}
	\gll si tɯ-pʰɯ nɯ ɯ-pʰaʁ ɯ-ntsi nɯ pjɤ-rom ʑo, ɯ-pʰaʁ ɯ-ntsi nɯ pjɤ-k-ɤrŋi-ci ʑo, \\
	tree one-tree \textsc{dem} \textsc{3sg}.\textsc{poss}-half \textsc{3sg}.\textsc{poss}-one.of.a.pair \textsc{dem} \textsc{ifr}.\textsc{ipfv}-be.dry \textsc{emph} \textsc{3sg}.\textsc{poss}-half \textsc{3sg}.\textsc{poss}-one.of.a.pair \textsc{dem} \textsc{ifr}.\textsc{ipfv}-\textsc{peg}-be.green-\textsc{peg} \textsc{emph} \\
	\glt `One half of that tree was dry and the other half was green.' (The divination 2002, 11)
\end{exe}


\subsection{Counted nouns and semantic classes} \label{sec:CN.classification}

This section presents the semantic restrictions on the use of particular counted nouns. It also discusses the cases of nouns that are compatible with several counted nouns.

\subsubsection{Non-collective counted noun}
The counted noun \japhug{tɯ-rdoʁ}{one piece} can occur as postnominal modifier with a considerable variety of nouns, designating people, animals, inanimate objects, including mass nouns; it is the counted noun by default.

A minority of nouns select other individual counted nouns referring to specific shapes: \japhug{tɯ-ldʑa}{one long object}, \japhug{tɯ-mpɕar}{one leaf} and \japhug{tɯ-pʰɯ}{one tree}.

 
The counted noun \japhug{tɯ-ldʑa}{one long object}, like Chinese \ch{一条}{yitiáo}{one long object} or \ch{一根}{yigēn}{one long object}, occurs in the corpus with nouns belonging to the following semantic categories:

\begin{itemize}
\item Stick-like objects:  \japhug{ɯ-ru}{its stalk}, \japhug{tɤ-zrɤm}{root}, \japhug{tɤɲi}{staff}, \japhug{taqaβ}{neddle},  \japhug{tɯmɲa}{arrow}, \japhug{dɤrʁɯ}{fern} or \japhug{ndʑu}{stick}, `chopsticks' (ex. \ref{ex:ndZu.XsWldZa})
\item Limbs, hair and other protruding body-parts: \japhug{tɯ-mi}{leg, foot}, \japhug{tɯ-kɤrme}{hair},  \japhug{tɤ-muj}{feather}, \japhug{ta-ʁrɯ}{horn}.
\item Limbless animals:  \japhug{qapri}{snake}, \japhug{qaɟy}{fish} 
\item Towers: \japhug{ɣɟɯ}{watchtower}
\end{itemize} 

\begin{exe}
\ex \label{ex:ndZu.XsWldZa}
\gll  tɕi-xɕɤndʑu χsɯ-ldʑa pɯ-tu tɕe \\
\textsc{1du}.\textsc{poss}-twig three-long.object  \textsc{pst}.\textsc{ipfv}-exist \textsc{lnk} \\
\glt `We had three twigs.'  (2003kubzang, 204)
\end{exe}

The counted noun \japhug{tɯ-mpɕar}{one leaf} occurs with flat objects, including tree leaves (with \japhug{tɤ-jwaʁ}{leaf} as in \ref{ex:RnWmpCar}), sheets of cloth (with \japhug{raz}{cloth}) and snowflakes (with \japhug{tɤjpa}{snow}).  

\begin{exe}
\ex \label{ex:RnWmpCar}
\gll ɯ-jwaʁ nɯ ʁnɯ-mpɕar ma me tɕe \\
\textsc{3sg}.\textsc{poss}-leaf \textsc{dem} two-leaf apart.from not.exist:\textsc{fact} \textsc{lnk} \\
\glt `It only has two leaves.'  (16-CWrNgo, 188)
(\japhdoi{0003518\#S176})
\end{exe}

The counted noun \japhug{tɯ-mpɕar}{one leaf} can also refer to money (in present-day China, bank notes are by far more common than coins even for small amounts of money), meaning `one renminbi'.

The counted noun \japhug{tɯ-phɯ}{one tree} occurs with the generic noun \japhug{si}{tree} or names of particular species.

\subsubsection{Collective counted noun} \label{sec:collective.counted.noun}
Most nouns referring to humans and animals can occur with the counted noun \japhug{tɯ-boʁ}{one group}. However, the collective counted noun \japhug{tɯ-ɟɯɣ}{one pack} appears to be exclusively used with horses. Its only attestations in the corpus (for instance \ref{ex:tWJWG}) are found in a translated text.

\begin{exe}
\ex \label{ex:tWJWG}
\gll li nɯ jamar ki a-jɤ-tɯ-ɕe tɕe, tɕendɤre nɯtɕu tɕe, mɤʑɯ mbro tɯ-ɟɯɣ tu tɕe,  \\
again \textsc{dem} about \textsc{dem}:\textsc{prox} \textsc{irr}-\textsc{pfv}-2-go \textsc{lnk} \textsc{lnk} \textsc{dem}:\textsc{loc} \textsc{lnk} more horse one-herd exist:\textsc{fact} \textsc{lnk} \\
\glt `Continue going again for about that distance, and there will be a pack of horse.' (150824 kelaosi-zh, 205)
(\japhdoi{0006276\#S193})
\end{exe}

\subsubsection{Multiple counted nouns} \label{sec:multiple.CN}
Some nouns are compatible with more than one counted noun postnominal modifier, with different connotations. For instance \japhug{zgo}{mountain} can be used with the generic counted noun \japhug{tɯ-rdoʁ}{one piece} as in (\ref{ex:zgo.tWrdoR}), but also with  \japhug{tɯ-ldʑa}{one long object}  as in (\ref{ex:zgo.RnWldZa}) in the meaning `mountain range' and also with \japhug{tɯ-tɤɣmbaj}{one side} to mean `mountain face'.

\begin{exe}
\ex \label{ex:zgo.tWrdoR}
\gll nɯ-kʰa ɯ-rkɯ zgo tɯ-rdoʁ pɯ-ri \\
\textsc{dem} house \textsc{3sg}.\textsc{poss}-side mountain one-piece \textsc{aor}-be.left  \\
\glt `When there was only one mountain left [on his way back] home, ...' (Lobzang03, 65)
\end{exe}

\begin{exe}
\ex \label{ex:zgo.RnWldZa}
\gll rŋgɯkɤta nɯ li zgo ʁnɯ-ldʑa tu tɕe, \\
\textsc{topo} \textsc{dem} again mountain two-long.object exist:\textsc{fact} \textsc{lnk} \\
\glt `[As for the placename] Rngukata, there are also two mountain ranges.' (140522 Kamnyu zgo, 64)
(\japhdoi{0004059\#S62})
\end{exe}

In other cases, the change of counted noun does not entail a radical semantic contrast. For instance, the noun \japhug{si}{tree}, while generally used with \japhug{tɯ-pʰɯ}{one tree}, is also attested with the counted noun \japhug{tɯ-ldʑa}{one long object}, as in (\ref{ex:si.tWldZa}), without clear semantic difference.

\begin{exe}
\ex \label{ex:si.tWldZa}
\gll maka rdɤstaʁ cʰo si rcanɯ tɯ-ldʑa cinɤ ʑo kɯ-me scʰiz kɤ-azɣɯt-ndʑi ɲɯ-ŋu, \\
completely stone \textsc{comit} tree \textsc{unexp}:\textsc{deg} one-long.object even.one \textsc{emph} \textsc{sbj}:\textsc{pcp}-not.exist \textsc{indef}.\textsc{loc} \textsc{aor}:\textsc{east}-reach-\textsc{du} \textsc{sens}-be  \\
\glt `They arrived at a place where there was no stones and not even one tree.' (2003 Kunbzang, 195)
\end{exe}




\subsection{Counted nouns and other parts of speech} \label{sec:CN.parts.of.speech}

This section presents the derivations from other parts of speech (including other nominal classes, such as inalienably and alienably possessed nouns) into counted noun, and from counted noun into other classes.

\subsubsection{Counted nouns and inalienably possessed nouns}   \label{sec:CN.IPN}
Counted nouns and inalienably possessed nouns stand out among other nouns in having an obligatory prefix. Since the citation form of both classes of nouns -- the numeral `one' prefix \forme{tɯ-} and the indefinite possessor prefixes \forme{tɯ-/tɤ-} -- are homophonous, it is not unexpected that conversion occurs between the two classes. 

Given the fact that numeral prefixes are a closed class (see §\ref{sec:numeral.prefixes} and in particular §\ref{sec:other.numeral.prefixes}), when one needs to use a quantifier without numeral prefix equivalent, it is necessary to convert the counted noun into an inalienably possessed noun in third person singular form, with the quantifier before it. In example (\ref{ex:nW.thamtCAt.WtWphu}), the quantifier \japhug{nɯ tʰamtɕɤt}{that many} cannot be converted to a prefix, and therefore the counted noun \japhug{tɯ-tɯpʰu}{one type} is converted to an inalienably possessed noun in \textsc{3sg} possessive form \forme{ɯ-tɯpʰu}.

\begin{exe}
\ex \label{ex:nW.thamtCAt.WtWphu}
\gll tɕe paʁ tɯ-ɣjɤn pjɯ́-wɣ-ntɕʰa nɯ nɯ tʰamtɕɤt ɯ-tɯpʰu ɲɯ-ɬoʁ ra \\ 
 \textsc{lnk} pig one-time \textsc{ipfv}-\textsc{inv}-butcher \textsc{dem} \textsc{dem} all \textsc{3sg}.\textsc{poss}-type \textsc{ipfv}-come.out be.needed:\textsc{fact} \\
\glt `Each time one kills a pig, one will get that many types [of foodstuff from it].' (05-paR, 93)
(\japhdoi{0003400\#S89})
\end{exe}

Similarly, in (\ref{ex:GurZa.Wro.WtWpW}), the more complex phrase \forme{ɣurʑa ɯ-ro} `more than one hundred' with the emphatic \forme{ʑo} occurs with the \textsc{3sg} prefix.

\begin{exe}
\ex \label{ex:GurZa.Wro.WtWpW}
\gll ɣurʑa ɯ-ro ʑo ɯ-tɯpɯ tu-j \\
 hundred \textsc{3sg}.\textsc{poss}-excess \textsc{emph} \textsc{3sg}.\textsc{poss}-household exist:\textsc{fact}-\textsc{1pl} \\
\glt `There are more than one hundred households of us.' (22-kumpGatCW, 32)
(\japhdoi{0003590\#S29})
\end{exe}

Example (\ref{ex:WtWpW.pjAkAntChWci}) illustrates a third case of conversion from counted noun to inalienably possessed noun: a third singular possessive on the converted counted noun indicates here indefinite number, which makes it possible to specify the quantity as the predicate (\forme{pjɤ-k-ɤntɕʰɯ-ci} `they used to be many') instead of the numeral prefix (\japhug{kɤntɕʰɯ-tɯpɯ}{many households}, see §\ref{sec:other.numeral.prefixes}).

\begin{exe}
\ex \label{ex:WtWpW.pjAkAntChWci}
\gll kɯɕɯŋgɯ nɯ, ɯ-tɯpɯ pjɤ-k-ɤntɕʰɯ-ci nɤ, \\
 former.time \textsc{dem} \textsc{3sg}.\textsc{poss}-household \textsc{ifr}.\textsc{ipfv}-\textsc{peg}-be.many-\textsc{peg} \textsc{sfp} \\
\glt `In former times, the households [there] were many.' (140522 tshupa, 67)
(\japhdoi{0004053\#S67})
\end{exe}

Conversion from counted noun to inalienably possessed noun is also observed when a prenominal demonstrative or adnominal clause is present, as in (\ref{ex:nW.Wxpa.nW}) and (\ref{ex:arCo.pWNu.WsNi}). Conversion is however not obligatory, as shown by (\ref{ex:nW.tWsNi}) (see additional examples in §\ref{sec:ordinals}).

\begin{exe}
\ex \label{ex:nW.Wxpa.nW}
\gll nɯ ɯ-xpa nɯ taχpa wuma pjɤ-pe \\
\textsc{dem} \textsc{3sg}.\textsc{poss}-year \textsc{dem} harvest really \textsc{ifr}.\textsc{ipfv}-be.good \\
\glt `On that year, the harvest was really good.' (02-montagnes-kamnyu, 72)
(\japhdoi{0003378\#S60})
\end{exe}

\begin{exe}
\ex \label{ex:arCo.pWNu.WsNi}
\gll [arɕo pɯ-ŋu] ɯ-sŋi nɯtɕu tɕe \\
be.finished.up:\textsc{fact} \textsc{pst}.\textsc{ipfv}-be \textsc{3sg}.\textsc{poss}-day \textsc{dem}:\textsc{loc} \textsc{loc} \\
\glt `The day when [the appointed time] was about to be finished,' (2003 sras, 31)
 \end{exe}
 
\begin{exe}
\ex \label{ex:nW.tWsNi}
\gll tɕendɤre nɯ tɯ-sŋi nɯnɯ mɯ-pjɤ-ko \\
\textsc{lnk} \textsc{dem} one-day \textsc{dem} \textsc{neg}-\textsc{ifr}-defeat \\
\glt `On that day, he failed in his attempt [to force her to take him with her].' (02-deluge2012, 73)
\end{exe}

Finally, in the case of counted nouns expressing body-based units of length, conversion to an inalienably possessed noun has a very specific meaning.  These units have a value that depends on the person of reference (few pairs of people have exactly the same handspan). The possessive prefix serves to indicate the person whose body part serves as the reference as in \forme{a-tɣa} `my handspan' in (\ref{ex:aZo.atGa}). It is the only case that a counted noun converted to inalienably possessed noun can take a possessive prefix other than \textsc{3sg}.

\begin{exe}
\ex \label{ex:aZo.atGa}
\gll kɯ-zri nɯra, tɯ-tɣa ma kɯki ɕaŋtaʁ mɤ-zri, aʑo a-tɣa jamar ci ma me. \\
 \textsc{sbj}:\textsc{pcp}-be.long \textsc{dem}:\textsc{pl} one-span apart.from \textsc{dem}.\textsc{prox} up.from \textsc{neg}-be.long:\textsc{fact} \textsc{1sg} \textsc{1sg}.\textsc{poss}-span about one apart.from not.exist:\textsc{fact} \\
\glt `As for its length, it is at most one handspan, only the length of my handspan.' (28-tshAwAre, 53)
(\japhdoi{0003722\#S50})
\end{exe}

Conversion from inalienably possessed noun to counted noun also exists, but is very marginal in Japhug. When inalienably possessed nouns are converted to counted nouns referring to a quantity, they are alienabilized (§\ref{sec:alienabilization}) and the numeral prefixes are added to the noun stem with its indefinite possessor prefix. For instance, the counted noun \japhug{tɯ-tɯ-kɯr}{one mouthful} (example \ref{ex:tWtWkWr.cinA} p.\pageref{ex:tWtWkWr.cinA}) or \japhug{tɯ-tɤ-ste}{one bladder of} (\ref{ex:tWci.tWtAste}) are derived from the body part inalienably possessed nouns \japhug{tɯ-kɯr}{mouth} and \japhug{tɤ-ste}{bladder}. 

\begin{exe}
\ex \label{ex:tWci.tWtAste}
\gll  tɕe ʁdɯxpanaχpu ɣɯ ɯ-me ɣɯ nɯ, tɯ-ci tɯ-tɤ-ste, tɤrcoʁ sɯrna ci to-rku-nɯ, \\
 \textsc{lnk}  \textsc{anthr} \textsc{gen} \textsc{3sg}.\textsc{poss}-daughter \textsc{gen} \textsc{dem} \textsc{indef}.\textsc{poss}-water one-\textsc{indef}.\textsc{poss}-bladder clay figurine \textsc{indef} \textsc{ifr}-put.in-\textsc{pl} \\
\glt `They gave to Gdugpa Nagpo's daughter a bladder full of water and a clay figurine (to take with her on the road, as her dowry).' (2003smanmi-tamu, 115)
\end{exe}

Direct conversion from inalienably possessed noun to counted noun without alienabilization is rarer, and it is not always obvious whether the inalienably possessed noun, or the counted noun is primary. For instance, the counted noun \japhug{tɯ-qiɯ}{one half} (see §\ref{sec:fractions}) is likely to have been derived from the inalienably possessed noun \japhug{ɯ-qiɯ}{half}, but the other directionality cannot be excluded. 

Clearer cases is provided by the inalienably possessed \japhug{ɯ-mdoʁ}{colour}  (from Tibetan \tibet{མདོག་}{mdog}{colour}), which derives a counted noun with the numeral prefix \forme{kɤntɕʰɯ-} `many' (on which see §\ref{sec:other.numeral.prefixes}) in (\ref{ex:kAntChWmdoR}), and the native inalienably possessed noun \japhug{tɤ-ʁar}{wing} from which the counted noun \japhug{tɯ-ʁar}{the length of one arm} originates (§\ref{sec:measures}).

\begin{exe}
\ex \label{ex:kAntChWmdoR}
\gll  tɕeri kɤntɕʰɯ-tɯpʰu, kɤntɕʰɯ-mdoʁ ɣɤʑu. \\
 but many-types many-colour exist:\textsc{sens} \\
\glt  `There are many types [of the mushrooms called \forme{tɯqejmɤɣ}], and with many colours.' (24-zwArqhAjmAG, 50)
\end{exe}

\subsubsection{Counted nouns and alienably possessed nouns}   \label{sec:CN.APN}
Alienably possessed nouns designating containers can be converted to a partitive counted nouns by adding the numeral prefixes to the noun stem. For instance, the noun \japhug{kʰɯtsa}{bowl} has a corresponding counted noun  \japhug{tɯ-kʰɯtsa}{one bowl} as in (\ref{ex:tWkhWtsa}).

 \begin{exe}
\ex \label{ex:tWkhWtsa}
\gll tɯ-kri tɯ-kʰɯtsa pjɤ-tu \\
\textsc{indef}.\textsc{poss}-oil  one-bowl \textsc{ifr}.\textsc{ipfv}-exist \\
\glt `They had one bowl [full] of oil.' (140501 mdzadi, 6)
\end{exe}

The same is true of nouns borrowed from Tibetan and Chinese such as \japhug{pʰoŋ}{bottle} (from \ch{瓶}{píng}{bottle}). As an alienably possessed noun, \forme{pʰoŋ} designates the bottle itself (not its content), and takes postnominal numerals (as in \ref{ex:phoN}).

\begin{exe}
\ex \label{ex:phoN}
\gll  pʰoŋ kɯβde ɣɤʑu \\
bottle four exist:\textsc{sens} \\
\glt `There are four bottles.' (elicited; can refer for instance to empty bottles)
\end{exe}

Converted to a counted noun \japhug{tɯ-pʰoŋ}{one bottle} it refers to the quantity of liquid contained in a bottle, and typically follows a mass noun as in (\ref{ex:kWBdephoN}).

\begin{exe}
\ex \label{ex:kWBdephoN}
\gll cʰa kɯβde-pʰoŋ pjɤ-k-ɤ-ta-ci. \\
alcohol four-bottle \textsc{ifr}.\textsc{ipfv}-\textsc{peg}-\textsc{pass}-put-\textsc{peg} \\
\glt  `There were four bottles [full] of alcohol.' (140510 sanpian sheye-zh, 51)
\end{exe}
 %tɯ-tɯkro tɯ-tɯrpa

The derivation process is quite productive, and potentially new counted nouns meaning `a ... full of' can be derived from any alienably possessed noun if a meaning can be made out of it, as in \forme{tɯ-co} from \japhug{co}{valley} in (\ref{ex:mbro.tWco}), which means in this particular context `an entire valley full of ...'.
 
 \begin{exe}
\ex \label{ex:mbro.tWco}
 \gll  mbro tɯ-co kɯ-fse, qaʑo tɯ-co kɯ-fse nɯ, fsapaʁ rmɯrmi ʑo nɯ tɯ-co nɤ tɯ-co ʑo pɯ-tu ɲɯ-ŋu. \\
 horse one-valley \textsc{sbj}:\textsc{pcp}-be.like  sheep one-valley \textsc{sbj}:\textsc{pcp}-be.like \textsc{dem} animal all.kinds \textsc{emph} \textsc{dem}  one-valley \textsc{lnk}  one-valley \textsc{emph} \textsc{pst}.\textsc{ipfv}-exist \textsc{sens}-be \\
 \glt `There was one valley entirely for each species of animals, like one valley full of horses, one valley full of sheep.' (2005 Kunbzang, 189)
 \end{exe}
 
\subsubsection{Counted nouns and verbs}   \label{sec:CN.verbs}
The derivation of verbs into counted nouns and that of counted nouns into verbs are both productive processes in Japhug. 

Counted nouns of verbal origin are either built by adding a numeral prefix to the verb stem (as \japhug{tɯ-fkur}{one load} from \japhug{fkur}{carry on the back}) or to the verb stem with an additional prefix \forme{tɤ-} (\japhug{tɯ-tɤrmbɯ}{one heap} from the transitive verb \japhug{rmbɯ}{heap up}, §\ref{sec:action.nominals}). Note also the simultaneous action nominal with two prefixes \forme{tɯ-tɯ\trt}, which can be analyzed as involving a counted noun derived from a verb (§\ref{sec:simultaneous.action.nominal}). 
%   tɯ-tɤfskɤr tɯ-tɤtɕʰɯ tɯ-tɤjŋgɤɣ

Most deverbal counted nouns are from transitive verbs, but examples from intransitive verbs are also found. For instance, in (\ref{ex:tWtshoz}) the counted noun \japhug{tɯ-tsʰoz}{one complete set} (from the stative intransitive verb \japhug{tsʰoz}{be complete}) translates Chinese \ch{一整套}{yīzhěngtào}{ one complete set}.

\begin{exe}
\ex \label{ex:tWtshoz}
\gll nɯnɯ kɯ iɕqʰa rŋɯl kɯ tɯ-ŋga tʰɯ-kɤ-βzu ci tɯ-tsʰoz pjɤ-βde. \\
\textsc{dem} \textsc{erg} the.aforementioned silver \textsc{erg} \textsc{indef}.\textsc{poss}-clothes \textsc{aor}-\textsc{obj}:\textsc{pcp}-make \textsc{indef} one-complete.set \textsc{ifr}-throw.down \\
\glt `[The bird] threw her a complete set of clothes that had been made in silver.'  (140504 huiguniang-zh, 111)
(\japhdoi{0003909\#S106})
\end{exe}

Deverbal counted nouns are either partitive counted nouns (§\ref{sec:CN.partitive}), designating a quantity of objects resulting from the action of the verb (for instance \japhug{tɯ-fkur}{one load} or \japhug{tɯ-tɤrtsɯɣ}{one pile} from \japhug{rtsɯɣ}{pile up} as in \ref{ex:RnWtArtsWG}) or semelfactive counted nouns (§\ref{sec:CN.iterative}), referring to the number of time an iterative/semelfactive action takes place (such as \japhug{tɯ-tɤtɕʰɯ}{hitting with the hoe one time} from \japhug{tɕʰɯ}{hit, gore}).

\begin{exe}
\ex \label{ex:RnWtArtsWG}
\gll ɕɤrɯ nɯ ʁnɯ-tɤrtsɯɣ to-βzu-ndʑi tɕe \\
bone \textsc{dem} two-pile \textsc{ifr}-make-\textsc{du} \textsc{lnk} \\
\glt `They had make two piles from the bones.' (2002nyimavodzer, 126)
\end{exe}

\begin{exe}
\ex \label{ex:tWtAtChW}
\gll qaʁ tɯ-tɤtɕʰɯ ta-lɤt  \\
hoe one-hitting.with.hoe \textsc{aor}:3\flobv{}-throw \\
\glt `He used the hoe one time.' (2003qachga, 166)
\end{exe}

There also are more lexicalized and synchronically less obvious examples of counted nouns derived from verbs. For instance the temporal counted noun \japhug{tɯ-xpa}{one year} originates from the intransitive verb \japhug{pa}{pass $X$ years}  illustrated in (\ref{ex:12.topa}) (see also example \ref{ex:40.topa}  in §\ref{sec:pa.intr.lv} and the account of the \forme{-x-} element in \forme{tɯ-\textbf{x}pa} proposed in in §\ref{sec:num.prefix.paradigm.history}). 

\begin{exe}
\ex \label{ex:12.topa} 
\gll kʰa na-βde nɯ sqamnɯz to-pa tɕe \\ 
house \textsc{aor}:3\flobv{}-leave \textsc{dem} twelve  \textsc{ifr}-pass.X.years \textsc{lnk} \\
\glt `Twelve years had passed since she had left home.' (150828 huamulan-zh, 126)
(\japhdoi{0006396\#S127})
\end{exe}

Since Situ, Zbu and Tshobdun all have cognates of \japhug{tɯ-xpa}{one year}, this non-trivial derivation must have occurred at the time of their common ancestor, but not earlier. Other Burmo-Gyalrongic languages, even Horpa and Khroskyabs \citep{jacques17stau} have a root related to Japhug \japhug{pa}{pass $X$ years} for time ordinals (§\ref{sec:time.ordinals}), but not for the corresponding counted noun. 
  
Derivation of transitive verbs from counted nouns with the denominal prefix \forme{rɤ-} is well-attested, for instance \japhug{rɤtɣa}{measure by handspan} from \japhug{tɯ-tɣa}{one span}. A complete list of these verbs and the various meanings of this derivation is presented in  §\ref{sec:denom.tr.rA}.
 
\subsubsection{Counted nouns and ideophones}   \label{sec:CN.ideophones}
Deideophonic counted nouns are quite common, and belong to all major functional categories described above. In the following discussion, ideophones (§\ref{sec:idph}) are by default cited in pattern II (§\ref{sec:ideo.II}), though other patterns are mentioned in some cases where appropriate.

The collective counted nouns \japhug{tɯ-boʁ}{one group} and \japhug{tɯ-ɟɯɣ}{one pack} originate from the ideophonic roots found in \japhug{boʁboʁ}{in group, in order} and \japhug{ɟɯɣɟɯɣ}{in great number (of long objects)}, respectively. The presence of plain voiced initial stops in these roots is a clue to their ideophonic origin (§\ref{sec:idph.onsets}).

The semelfactive/iterative counted noun \japhug{tɯ-tɤxɯr}{one turn} comes from the ideophonic root of \japhug{xɯrxɯr}{round} (pattern III \japhug{xɯrnɤxɯr}{turning around}) with an additional prefix \forme{tɤ-}. This prefix is possibly the trace of an intermediate stage as an inalienably possessed noun, with a two-step derivation \textsc{ideophone} $\rightarrow$ inalienably possessed noun $\rightarrow$ counted noun.

Direct derivation from ideophone to semelfactive counted noun is also attested, for instance \japhug{tɯ-ɬɯm}{one period of sleep} (\ref{ex:tWlhWm.pjAnWZWB}) from the root in \japhug{ɬɯmɬɯm}{have a nice sleep} without the \forme{tɤ-} prefix.

\begin{exe}
\ex \label{ex:tWlhWm.pjAnWZWB} 
\gll li tɯfsɤkʰa tɕe tɯ-ɬɯm pjɤ-nɯʑɯβ tɕe li ɯ-jmŋo ko-ntɕʰɤr tɕe \\
again dawn \textsc{lnk} one-sleep \textsc{ifr}-sleep \textsc{lnk} again  \textsc{3sg}.\textsc{poss}-dream \textsc{ifr}-appear \textsc{lnk} \\
\glt `(The mother was so worried she could not sleep all night.) At dawn, she had a little sleep and had again a dream.' (2012 Norbzang, 186)
\end{exe}

\section{Measures} \label{sec:measures}
Measures of size, weight and volume are mainly expressed by counted nouns, though some unpossessible nouns are also found.\footnote{Measures of time are discussed in §\ref{sec:time}. } Most or these words are falling out of use, and being replaced by Chinese words, or calques from Chinese.

\tabref{tab:length.cn} presents the counted nouns used in Japhug for measures of lengths. The obsolete forms are indicated in brackets. The counted nouns \japhug{tɯ-tɣa}{one span} and \japhug{tɯ-ɟom}{the length of two outstretched arms} are very commonly used in narratives and conversations (their Chinese equivalents are  \ch{一拃}{yīzhǎ}{one span} and   \ch{一庹}{yītuǒ}{the length of two outstretched arms}, respectively). There are two ways to measure the handspan, \japhug{ndzoʁtɣa}{length between the thumb and the forefinger} and \japhug{ɲaʁtɣi}{length between the thumb and the middle finger}; these nouns are followed by the \forme{tɯ-tɣa} when one wants to specify which of the measures one chooses, as in \forme{ɲaʁtɣi tɯ-tɣa}. The counted noun \japhug{tɯ-ʁar}{the length of one arm} (related to \japhug{tɤ-ʁar}{wing}) is less common that the other ones in \tabref{tab:length.cn} but attested for instance in (\ref{ex:tWRar}).

\begin{exe}
\ex \label{ex:tWRar} 
\gll  tsuku tɕe tɕe tɯ-ɟom kɯnɤ tu-zri mɯ́j-cʰa,  tɯ-ʁar jamar, tɯ-ʁar tɕe tɕe kɯki jamar, kɯki jamar tu-zri ɲɯ-cʰa. \\
some \textsc{lnk} \textsc{lnk} one-length.of.two.outstretched.arms even \textsc{ipfv}-be.long \textsc{neg}:\textsc{sens}-can one-arm.length about  one-arm.length \textsc{lnk} \textsc{lnk} \textsc{dem}:\textsc{prox}  about \textsc{dem}:\textsc{prox}  about \textsc{ipfv}-be.long \textsc{sens}-can \\
\glt `Some of them cannot even grow up to the length of one fathom, they can only grow up to the length of one arm, this much.' (16-RlWmsWsi)
(\japhdoi{0003520\#S115})
\end{exe}

\begin{table}
\caption{Units of length} \label{tab:length.cn}
\begin{tabular}{lll}
\lsptoprule
\japhug{tɯ-tɣa}{one span}    \\
(\japhug{tɯ-kʰa}{one foot})   \\
\japhug{tɯ-ʁar}{the length of one arm}   \\
\japhug{tɯ-ɟom}{the length of two outstretched arms}  \\
(\japhug{tɯ-tɯnɯna}{one mile})   \\
\lspbottomrule
\end{tabular}
\end{table}

The counted noun \japhug{tɯ-tɯnɯna}{one mile} derives from the \forme{tɯ-} actional nominal (§\ref{sec:tW.action.nominal}) of the verb \japhug{nɯna}{rest}, designating a milestone on the road indicating travellers' resting places (at regular intervals). Distances are counted now however only in kilometers using Chinese (including Chinese numerals), as for instance the expression \ch{三公里}{sāngōnglǐ}{three kilometers}  in (\ref{ex:sangongli}).

\begin{exe}
\ex \label{ex:sangongli} 
\gll a-pi tɕʰeme nɯnɯ mbarkhom ɯ-rkɯ <xiaoshuigou> <sangongli> nɯtɕu tʰɯ-ɣe. \\
\textsc{1sg}.\textsc{poss}-elder.sibling girl \textsc{dem} Mbarkham \textsc{3sg}.\textsc{poss}-side  \textsc{anthr} three.kilometers \textsc{dem}:\textsc{loc} \textsc{aor}:\textsc{downstream}-come[II] \\
\glt  `My elder sister came to Xiaoshuigou, three kilometers from Mbarkham.' (140501 tshering scid, 40)
(\japhdoi{0003902\#S39})
\end{exe}

Not all nouns of measure are counted nouns.  In traditional stories, the unpossessible noun \japhug{χpaχtsʰɤt}{yojana} from Tibetan \tibet{དཔག་ཚད་}{dpag.tsʰad}{yojana} occurs to designate a mythical measure of distance taken from Indian sources. Numerals are indicated as postnominal numeral modifiers as in (\ref{ex:dpag.tshad}).

\begin{exe}
\ex \label{ex:dpag.tshad}
\gll  tɯ-sŋi χpaχtsʰɤt kɯngɯt ɲɯ-tɯ́-wɣ-tsɯm cʰa \\
one-day yojana nine \textsc{ipfv}:\textsc{west}-2-\textsc{inv}-take.away can:\textsc{fact} \\
\glt `[This horse] can carry you (west) nine \textit{yojana}s in one day.'  (2003smanmi, 57)
\end{exe}

For measuring the surface of fields, the term \forme{tɯ-rkoŋɕɤl}, illustrated by example (\ref{ex:tWrkoNCAl}), is still in use.

\begin{exe}
\ex \label{ex:tWrkoNCAl}
\gll mɤʑɯ tɯ-rkoŋɕɤl jamar tɕe kɤ-ɕlu lu-jɤɣ ɲɯ-ŋu \\
more one-unit.of.field.surface about \textsc{lnk} \textsc{inf}-plow \textsc{ipfv}:\textsc{upstream}-finish \textsc{sens}-be \\
\glt `One more unit and the plowing will be finished.' (elicited)
\end{exe}

In the traditional society, cereals were more often measured by volume (using containers of various size) than by weight, an action called \japhug{ɕtʂo}{measure by scooping}. \tabref{tab:volume.cn} presents the known units of volume.\footnote{Among these units, \japhug{tɯ-po}{one dou} apparently comes from Tibetan \tibet{འབོ་}{ⁿbo}{unit of measure} corresponding to Chinese \ch{一斗}{yīdǒu}{one dou} (about ten liter in the metric system). The other units are native words.
} Among them, \japhug{χtsiɯ}{bushel} and \japhug{ɕpɣo}{ten bushels}  occurs as either alienably possessed or counted nouns. The relationship between the units of volume is described in (\ref{ex:sqWXtsiW}).


\begin{table}
\caption{Units of volume} \label{tab:volume.cn}
\begin{tabular}{lll}
\lsptoprule
\japhug{tɯ-χtsiɯ}{one bushel}    \\
\japhug{tɯ-ɕpɣo}{ten bushels}    \\
\japhug{tɯ-ɣna}{thirty bushels}    \\
\japhug{tɯ-po}{one dou}    \\
\lspbottomrule
\end{tabular}
\end{table}

\begin{exe}
\ex \label{ex:sqWXtsiW}
\gll sqɯ-χtsiɯ tɕe tɕe nɯ ɕpɣo ɲɯ-ŋu. tɕe tɯ-ɕpɣo ɲɯ-ŋu. ɕpɣo nɯnɯ tɕe tɕe, mɤʑɯ fsɯsqɯ-χtsiɯ a-tɤ-ɤpɯpa tɕe tɕe nɯnɯ tɯ-ɣna. tɕe nɯ tɯ-ɣna pjɤ-ŋu tɕe tɕe nɯ ɕaŋtaʁ ɯ-sɤ-ɕtʂo pjɤ-me tɕe, \\
ten-bushel \textsc{lnk} \textsc{lnk} \textsc{dem} ten.bushels \textsc{sens}-be \textsc{sens}-be ten.bushels \textsc{sens}-be ten.bushels  \textsc{dem} \textsc{lnk} \textsc{lnk} even.more thirty-bushel \textsc{irr}-\textsc{pfv}-accumulate \textsc{lnk} \textsc{lnk} \textsc{dem} one-thirty.bushels  \textsc{lnk} dem one-thirty.bushels \textsc{ifr}.\textsc{ipfv}-be \textsc{lnk} \textsc{lnk} \textsc{dem} up.from \textsc{3sg}.\textsc{poss}-\textsc{nmlz}:oblique-measure \textsc{ifr}.\textsc{ipfv}-not.exist \textsc{lnk} \\
\glt  `Ten bushels was a \forme{ɕpɣo}, after the \forme{ɕpɣo}, thirty bushels put together was a  \forme{tɯ-ɣna}. After that, there was no container used to measure [grains].'  (140515 rJama)
(\japhdoi{0004006\#S24})
\end{exe}

The units of weight include \japhug{tɯ-sraŋ}{one ounce} (from Tibetan \tibet{སྲང་}{sraŋ}{ounce}),  \japhug{rɟɤpɕɤt}{half pound} (from the first syllable of \tibet{རྒྱ་མ་}{rgʲa.ma}{scales} with \tibet{ཕྱེད་}{pʰʲed}{half}) and the counted noun \japhug{tɯ-tɯrpa}{one pound} derived from \japhug{tɯrpa}{axe} (see §\ref{sec:CN.APN}).\footnote{This may be an ancient calque from \ch{斤}{jīn}{pound}, which also meant `axe' in Old Chinese. Apart from \japhug{tɯ-tɯrpa}{one pound}, all technical terms related to weighing are from Tibetan (including \japhug{rɟama}{scales} and \japhug{skɤr}{weigh}). } 

\begin{exe}
\ex \label{ex:sqaprAsrang}
\gll kɯɕɯŋgɯ rɟama ɯ-taʁ tɕe, kɯrcɤ-sraŋ tɕe rɟɤpɕɤt,  sqaprɤ-sraŋ tɕe tɯ-tɯrpa pjɤ-ŋu. \\
in.former.times weighing.scales \textsc{3sg}.\textsc{poss}-on \textsc{lnk} eight-pound \textsc{lnk} half.pound sixteen-ounce \textsc{lnk} one-pound \textsc{ifr}.\textsc{ipfv}-be \\
\glt  `In former times, on the scales, one half pound was eight ounces, and one pound was sixteen ounces.' (140515 rJama)
(\japhdoi{0004006\#S2})
\end{exe}
 

\section{Counting time} \label{sec:time}
\subsection{Temporal counted nouns} \label{sec:CN.time}
Japhug has native counted nouns for time durations related to solar and lunar cycles: \japhug{tɯ-xpa}{one year}, \japhug{tɯ-sla}{one month}, \japhug{tɯ-sŋi}{one day}, \japhug{tɤ-rʑaʁ}{one night} (also used to express 24 hours). There are no native concepts for `weeks' or `ten days'; the expression of hours is presented in §\ref{sec:hours}. Other temporal counted nouns include \japhug{tɯ-mɲɯtsi}{one lifetime} (from \tibet{མི་ཚེ་}{mi.tsʰe}{human life}), \japhug{tɯ-tɯpɕɯr\-tɕʰaʁ}{one generation} and \japhug{tɯ-rzɯɣ}{one section, one instant}, and \japhug{tɯ-skɤrma}{one minute} (from Tibetan \tibet{སྐར་མ་}{skar.ma}{star, minute})


The temporal counted nouns  \japhug{tɯ-sŋi}{one day} and \japhug{tɤ-rʑaʁ}{one night} are commonly used in apposition with the same numeral prefix to express the meaning `X days and X nights', as in (\ref{ex:XsWsNi.XsArZaR}).

\begin{exe}
\ex \label{ex:XsWsNi.XsArZaR}
\gll χsɯ-sŋi χsɤ-rʑaʁ ʑo pjɤ-rɤʑi \\
three-day three-night \textsc{emph} \textsc{ifr}.\textsc{ipfv}-stay \\
\glt `He stayed there for three days and three nights.' (2011-13-qala, 16)
\end{exe}  

To express the meanings corresponding to English `after' or `later' with a time span, the idiomatic way in Japhug is to use verbs such as \japhug{tsu}{pass (of time)}, as shown by example (\ref{ex:tWsNi.tAtsu}) and (\ref{ex:tAtsua}). Note that \japhug{tsu}{pass (of time)} is a semi-transitive verb (§\ref{sec:semi.transitive}) whose semi-object is the time period expressed by the counted noun, and whose subject is the person affected by the passing of time (see for instance \ref{ex:tAtsua} with \textsc{1sg} indexation).

\begin{exe}
\ex \label{ex:tWsNi.tAtsu}
\gll ɯ-tɯ-ɣɤ-wxti pjɤ-saχaʁ ʑo tɯ-sŋi tɤ-tsu tɕe, χsɯ-sŋi tɤ-kɯ-tsu to-fse. χsɯ-sŋi tɤ-tsu tɕe, tɯ-sla tɤ-kɯ-tsu to-fse. tɯ-sla tɤ-tsu tɕe, tɯ-xpa tɤ-kɯ-tsu to-fse. \\
\textsc{3sg}.\textsc{poss}-\textsc{nmlz}:\textsc{deg}-\textsc{facil}-be.big \textsc{ifr}.\textsc{ipfv}-be.extremely \textsc{emph} one-day \textsc{aor}-pass \textsc{lnk} three-days \textsc{aor}-\textsc{sbj}:\textsc{pcp}-pass \textsc{ifr}-be.like three-day \textsc{aor}-pass \textsc{lnk} one-month \textsc{aor}-\textsc{sbj}:\textsc{pcp}-pass \textsc{ifr}-be.like one-month \textsc{aor}-pass \textsc{lnk} one-year \textsc{aor}-\textsc{sbj}:\textsc{pcp}-pass \textsc{ifr}-be.like  \\
\glt `He grew extremely fast, one day after [he was born] he looked like [an infant] who was three days old, after three days he looked like a [baby] who was one month old, after one month he looked like a [toddler] who was one year old.' (2012 Norbzang)
(\japhdoi{0003768\#S103})
\end{exe}


\begin{exe}
\ex \label{ex:tAtsua}
\gll  tɕʰorzi ɯ-ŋgɯ cʰɯ-kɯ-rku-a, χsɤ-rʑaʁ tɤ-tsu-a tɕe ci a-tɤ-kɯ-rtoʁ-a, \\
jar \textsc{3sg}.\textsc{poss}-inside \textsc{ipfv}:\textsc{downstream}-2\fl{}1-put.in-\textsc{1sg} three-days \textsc{aor}-pass-\textsc{1sg} \textsc{lnk} once \textsc{irr}-\textsc{pfv}-2\fl{}1-look-\textsc{1sg} \\
\glt `Put me in a jar, and three days later have a look at me.' (2003 Kunbzang, 385)
\end{exe} 

The postposition \japhug{ɕɯŋgɯ}{before} (§\ref{sec:temporal.postpositions}) occurs after temporal counted nouns in examples such as (\ref{ex:ngWsqWxpa.CWNgW}) (see also \ref{ex:sqamNusNi.CWNgW}  below); the relator noun \japhug{ɯ-qʰu}{after} (§\ref{sec:relator.temporal}) is not used in this way except in texts translated from Chinese (where it is likely calqued) as in (\ref{ex:kWBdexpa.Wqhu}).

\begin{exe}
\ex \label{ex:ngWsqWxpa.CWNgW}
 \gll kɯngɯsqɯ-xpa ɕɯŋgɯ nɯra tɕʰeme ra wuma ʑo kɯ-taʁ pɯ-dɤn \\
nine.ten-years before \textsc{dem}:\textsc{pl} girl \textsc{pl} really \textsc{emph} \textsc{sbj}:\textsc{pcp}-weave \textsc{pst}.\textsc{ipfv}-be.many \\
 \glt `Nine or ten years ago, there were many weavers among women.' (thaXtsa2002, 100)
\end{exe}

\begin{exe}
\ex \label{ex:kWBdexpa.Wqhu}
\gll  kɯβdɤ-xpa ɯ-qʰu tɕe kukutɕu tu-kɤ-ɤwɯwum to-nɯ-pa-nɯ. \\
four-year \textsc{3sg}.\textsc{poss}-after \textsc{lnk} here \textsc{ipfv}-\textsc{inf}-\textsc{recip}:gather \textsc{ifr}-\textsc{auto}-make-\textsc{pl} \\
\glt `They agreed to meet again at this place in four years.' (140508 benling gaoqiang de si xiongdi-zh)
(\japhdoi{0003935\#S23})
\end{exe} 

In order to express meanings such as `beginning' or `end' of a time period indicated by a counted noun, verbs such as \japhug{arɕo}{be finished} are used as in (\ref{ex:tWxpa.YArCo}) instead of nouns or participles like \japhug{ɯ-sɤɣjɤɣ}{its end}.

\begin{exe}
\ex \label{ex:tWxpa.YArCo}
\gll ɣɯjpa tɯ-xpa ɲɯ-ɤrɕo ɕɯŋgɯ tɕe, <lunwen> kɤ-rɤt pjɯ-jɤɣ ɲɯ-ra. \\
this.year one-year \textsc{ipfv}-be.finished  before \textsc{lnk} dissertation \textsc{inf}-write \textsc{ipfv}-finish \textsc{sens}-be.needed \\
\glt `He has to finish writing his dissertation before the end of this year.' (elicitation)
\end{exe} 

Temporal counted nouns are generally used on their own without head noun. The counted noun \japhug{tɯ-xpa}{one year} however does occur with the noun \japhug{lu}{year} (from Tibetan \tibet{ལོ་}{lo}{year}) in some traditional stories as in (\ref{ex:lu.XsWxpa}).

\begin{exe}
\ex \label{ex:lu.XsWxpa}
\gll lu χsɯ-xpa pɯ-ŋke-j pɯ-ra ri, \\
year three-year \textsc{pst}.\textsc{ipfv}-walk-\textsc{1sg} \textsc{pst}.\textsc{ipfv}-be.needed \textsc{lnk} \\
\glt `We had to walk for three years.' (sras2003, 58)
\end{exe} 

The temporal counted nouns \japhug{tɯ-sla}{one month} and \japhug{tɯ-xpa}{one year} have the special forms \japhug{kɤrɤ-sla}{several months} and \japhug{kɤrɤ-xpa}{several years} as in (\ref{ex:kArAxpa}) and (\ref{ex:kArAsla}) with what appears to be a numeral prefix \forme{kɤrɤ-} `several'.\footnote{These two counted nouns are also compatible with the approximate numeral prefix \forme{laʁnɯ-} `a few'as in \forme{laʁnɯ-sla} `a few months' and \forme{laʁnɯ-xpa} `a few years' (§\ref{sec:approx.numerals}).
} This prefix cannot however be used with any other counted nouns, even \japhug{tɯ-sŋi}{one day}. 


\begin{exe}
\ex \label{ex:kArAxpa}
\gll tɯ-ji ɯ-ŋgɯ kɤrɤ-xpa ʑo pɯ-a-nɯ-rku kɯnɤ, pjɯ-tsɣi mɤ-cʰa. \\
\textsc{indef}.\textsc{poss}-field \textsc{3sg}.\textsc{poss}-inside several-years \textsc{emph} \textsc{pst}.\textsc{ipfv}-\textsc{pass}-\textsc{auto}-put.in also \textsc{ipfv}-rot \textsc{neg}-can:\textsc{fact} \\
\glt `Even if it remains in [the ground of] the field for several years, it does not rot.' (08-qaJAGi, 37)
\end{exe}

\begin{exe}
\ex \label{ex:kArAsla}
\gll kɤrɤ-sla ʑo tɯ-ŋga ra ma-nɯ́-wɣ-χtɕi qʰe tɕe zrɯɣ ku-βze ɕti.  \\
several-months \textsc{emph} \textsc{indef}.\textsc{poss}-clothes \textsc{pl} \textsc{neg}:\textsc{irr}-\textsc{pfv}-\textsc{inv}-wash \textsc{lnk} \textsc{lnk} louse \textsc{ipfv}-grow be.\textsc{aff}:\textsc{fact} \\
\glt `If one does not wash clothes for several months, lice will grow in it.' (21-mdzadi, 51)
(\japhdoi{0003578\#S51})
\end{exe}


\subsection{Time ordinals} \label{sec:time.ordinals}
Japhug has two series of time ordinals, one for days (\tabref{tab:day.ordinals}) and another one for years (\tabref{tab:year.ordinals}).  In the following discussion, I adopt Michailovsky's (\citeyear{michailovsky03ordinals}) notations: $D^{-\alpha}$ `$\alpha$ day(s) ago', $D^{+\beta}$ `in $\beta$ day(s)', $Y^{+ \gamma}$ `$\gamma$ year(s) ago', $Y^{+ \delta}$  `in $\delta$ year(s)'.


\begin{table}
	\caption{Japhug day ordinals} \label{tab:day.ordinals} \centering
	\begin{tabular}{llllll}
		\lsptoprule
		Day & Time ordinal expression \\
		\midrule
		-2 & \japhug{jɯfɕɯndʐi}{two days ago}, `the other day' \\
		-1 & \japhug{jɯfɕɯr}{yesterday} \\
		0 & \japhug{jisŋi}{today} \\
		+1 & \japhug{fso}{tomorrow} \\
		+2 & \japhug{fsɤndi}{the day after tomorrow} \\
		+3 & \japhug{qʰɤndi}{in three days} \\
		+4 & \japhug{ɲɤndi}{in four days} \\
		+5 & \japhug{βʑindi}{in five days} \\
		+6 & \japhug{pɤtsɤndi}{in six days} \\
		\lspbottomrule
	\end{tabular}
\end{table}


The Japhug system has relatively few ordinals for past days and years (some Kiranti languages have terms for up to $D^{-6}$ and $Y^{-4}$, as shown by \citealt{michailovsky03ordinals}), but stands out by having six ordinals for future years (no Kiranti languages has more than $D^{+6}$ and $Y^{+4}$ in Michailovsky's \citeyear{michailovsky03ordinals} data). However, the $D^{+5}$ \japhug{βʑindi}{in five days}, $D^{+6}$  \japhug{pɤtsɤndi}{in six days} and the corresponding year ordinals  $Y^{+5}$  and  $Y^{+6}$  are not used any more in day-to-day speech even by the eldest speakers.
 

The $D^{0}$ to $D^{-2}$ and $Y^{0}$ to $Y^{-2}$ordinals have a prefix \forme{ji-/jɯ-/ja\trt}, which is probably of demonstrative origin. This prefix also occurs in the derived ordinals \japhug{jɯɣmɯr}{this evening} and \japhug{jɯxɕo}{this morning} (see Tables \ref{tab:morning.ordinals} and \ref{tab:evening.ordinals} below), as well as in some frozen adverbs, such as \japhug{jɤxtsʰi}{this time} (from \japhug{tɯ-xtsʰi}{one time}) and \japhug{jinde}{these days}. Cognates of this prefix are found in other time ordinals elsewhere in Gyalrongic. In the case of \japhug{ɣɯjpa}{this year}, this prefix surfaces as a preinitial  \forme{-j-} to the root \forme{-pa} `year' (see §\ref{sec:num.prefix.paradigm.history} on the etymology of this noun). The \forme{ɣɯ-} (from \forme{*wə-}) prefix may be cognate to the prefix \forme{pə-} in \stau{pəvə}{this year} and \stau{pəsɲə}{today} in Stau.

\begin{table}
	\caption{Japhug year ordinals} \label{tab:year.ordinals} \centering
	\begin{tabular}{llllll}
		\lsptoprule
		Year & Time ordinal expression \\
		\midrule
		-2 & \japhug{japandʐi}{two years ago}, `a few years ago' \\
		-1 & \japhug{japa}{last year} \\
		0 & \japhug{ɣɯjpa}{this year} \\
		+1 & \japhug{fsaqʰe}{next year} \\
		+2 & \japhug{fsɤndɤpa}{in two years} \\
		+3 & \japhug{qʰɤndɤpa}{in three years} \\
		+4 & \japhug{ɲɤndɤpa}{in four years} \\
		+5 & \japhug{βʑindɤpa}{in five years} \\
		+6 & \japhug{pɤtsɤndɤpa}{in six years} \\
		\lspbottomrule
	\end{tabular}
\end{table}


 The expressions \japhug{jɯfɕɯndʐi}{two days ago}  and \japhug{japandʐi}{two years ago}, which are based on the $D^{-1}$ and $Y^{-1}$ forms with a suffix \forme{-ndʐi}, can be analyzed as time ordinals ($D^{-2}$ and $Y^{-2}$, respectively), as in (\ref{ex:japandzxi}).\footnote{This suffix \forme{-ndʐi} is not attested elsewhere, but is probably related to the syllable \forme{-ri} of the relator noun \japhug{ɯ-ʁɤri}{in front of; before}, from an earlier \forme{*-n-ri} with epenthesis (\forme{*-ndri}) and fricativization of the resulting \forme{*dr} cluster. }
 
\begin{exe}
\ex \label{ex:japandzxi}
\gll   nɯ ma kɯ-dɤn japa mɯ-tɤ-χtɯ-t-a. japandʐi alo, ɬaltɕɤm kɯ z-ɲɤ-ɕar tɕendɤre \\
 \textsc{dem} apart.from \textsc{sbj}:\textsc{pcp}-be.many last.year \textsc{neg}-\textsc{aor}-buy-\textsc{tr}:\textsc{pst}-\textsc{1sg}  two.years.ago upstream  \textsc{anthr} \textsc{erg} \textsc{tral}-\textsc{ifr}-search \textsc{lnk}   \\
\glt `Last year, it did not buy a lot [of edible fern]. The year before, up there (in Kamnyu), Lhalcam collected some.' (conversation, 14.05.10)
\end{exe}

However, in many cases, they have less specific meanings such as `a few days ago' and `a few years ago',  as shown by (\ref{ex:jWfCWndzxi}), referring to an event that had occurred more than one week before.\footnote{This use reminds of the English expression `a couple of days ago', which can mean up to a week ago. }

\begin{exe}
\ex \label{ex:jWfCWndzxi}
\gll jɯfɕɯndʐi pɯ-tɯ-χɕu-ndʑi ma, a-ŋga lɤ-tɯ-sɯ-ɣɯt-ndʑi nɯ, a-xtsa nɯ wuma ɲɯ-pe tɕe nɯ ku-nɯ-ŋge-a,  a-tsa wuma ɲɯ-βze \\ 
last.days \textsc{aor}-2-be.strong-\textsc{du} \textsc{lnk} \textsc{1sg}.\textsc{poss}-clothes \textsc{aor}-2-\textsc{caus}-bring-\textsc{du} \textsc{dem} \textsc{1sg}.\textsc{poss}-shoe \textsc{dem} really \textsc{sens}-be.good \textsc{lnk} \textsc{dem} \textsc{prs}-\textsc{auto}-wear[III]-\textsc{1sg}  \textsc{1sg}.\textsc{poss}-adapted really \textsc{sens}-make[III] \\
\glt `Thank you both for the other day, the clothes that you have sent me, the shoes are very nice, I wear them, they are my size.' (conversation, 15.04.18)
\end{exe}

Even the $Y^{-1}$ ordinal \japhug{japa}{last year} can have a less specific meaning if combined with \japhug{dal}{early} and the locative \forme{ri} as \japhug{japa dal ri}{a few years ago} as in (\ref{ex:japa.dal.ri}). No other time ordinal can be used with \japhug{dal}{early} or other modifiers.

\begin{exe}
\ex \label{ex:japa.dal.ri}
\gll japa dal ri, ftɕar ri  ʁmbɣɯzɯn ɣɤʑu ɲɯ-ti-nɯ ma \\
 last.year early \textsc{loc} summer \textsc{loc} solar.eclipse exist:\textsc{sens} \textsc{sens}-say-\textsc{pl} \textsc{sfp}\\
\glt `A few years ago, they said there was a solar eclipse during summer.' (29-RmGWzWn2, 27)
\end{exe}

The day ordinals from $D^{+2}$ to $D^{+5}$ contain a suffix \forme{-ndi}, which attaches to the bound state forms of \japhug{fso}{tomorrow} in $D^{+2}$ \forme{fsɤndi} and \japhug{ɯ-qʰu}{after} in $D^{+3}$ \forme{qʰɤndi}. The $D^{+5}$ form \forme{βʑindi} contains the bound state \forme{βʑɯ-} of the Japhug pronunciation \forme{βʑi} of the Tibetan numeral \tibet{བཞི་}{bʑi}{four}  (see §\ref{sec:one.to.ten}), presumably meaning `the fourth day after tomorrow'.

Year ordinals from $Y^{+2}$ to $Y^{+5}$ (\tabref{tab:year.ordinals}) are built by simply adding the root \forme{-pa} `year' (see §\ref{sec:num.prefix.paradigm.history} and § §\ref{sec:CN.verbs}) to the bound state of the corresponding day ordinals $D^{+2}$ to $D^{+5}$. The $Y^{+1}$ time ordinal \japhug{fsaqʰe}{next year} comes from the bound state of \japhug{fso}{tomorrow} , and the second element may be from the counted noun \japhug{ɯ-qʰu}{after} followed by the locative suffix \forme{*-j} with regular vowel fusion (§\ref{sec:locative.j}).

In addition, there are morning ordinals (\tabref{tab:morning.ordinals}) which derive from the corresponding day ordinals by adding the noun \japhug{soz}{morning} (with the exception of \japhug{jɯxɕo}{this morning}), and the evening and night ordinals (\tabref{tab:evening.ordinals}) which are built by adding the bound form \forme{-mɯr} (found in \japhug{tɯ-ɣmɯr}{one evening} and \japhug{mɯrkɯrku}{every evening}). These ordinals are not attested after $D^{+3}$, though the forms could be built easily; note the absence of bound state, except for $D^{-1}$, where \japhug{jɯfɕɯr}{yesterday} loses its coda, as in \japhug{jɯfɕɯsoz}{yesterday morning}.

Note the isolated \japhug{qʰuj}{this afternoon} \tabref{tab:evening.ordinals}), which can refer to the time period between noon and the night (including the early evening). 

\begin{table}
\caption{Morning ordinals} \label{tab:morning.ordinals} \centering
\begin{tabular}{lllll}
\lsptoprule
Day & Morning ordinal \\
\midrule
-2 & \japhug{jɯfɕɯndʐisoz}{the morning of two days ago} \\
-1 & \japhug{jɯfɕɯsoz}{yesterday morning} \\  
0 & \japhug{jɯxɕo}{this morning} \\
+1& \japhug{fsosoz}{tomorrow morning} \\
+2&  \japhug{fsɤndisoz}{in two days in the morning} \\
+3&\japhug{qʰɤndisoz}{in three days in the morning} \\
\lspbottomrule
\end{tabular}
\end{table}

 As shown by (\ref{ex:jWfCWndzximWr}), it is also possible to combine day ordinals with other time nouns such as \japhug{tɯrmɯ}{dusk} to indicate particular moments of previous or future days.

\begin{exe}
\ex \label{ex:jWfCWndzximWr}
\gll jɯfɕɯndʐi-mɯr nɤ-pi kɯ-wxti ɯ-taʁ ko-ɴqoʁ-a ri mɯ-tɤ́-wɣ-tsɯm-a,
jɯfɕɯr tɯrmɯ tɕe nɤ-pi tɯlɤt nɯ ɯ-taʁ ko-ɴqoʁ-a ri mɯ-tɤ́-wɣ-tsɯm-a tɕe,
jɯɣmɯr ndɤre nɤʑo tu-kɯ-tsɯm-a ra  \\
two.days.ago-evening \textsc{2sg}.\textsc{poss}-elder.sibling \textsc{sbj}:\textsc{pcp}-be.big \textsc{3sg}.\textsc{poss}-on \textsc{ifr}-hang-\textsc{1sg} \textsc{lnk} \textsc{neg}-\textsc{aor}:\textsc{up}-\textsc{inv}-take.away-\textsc{1sg}  yesterday dusk \textsc{lnk} \textsc{2sg}.\textsc{poss}-elder.sibling  second.sibling \textsc{dem} \textsc{3sg}.\textsc{poss}-on \textsc{ifr}-hang-\textsc{1sg} \textsc{lnk} \textsc{neg}-\textsc{aor}:\textsc{up}-\textsc{inv}-take.away-\textsc{1sg} \textsc{lnk} this.evening \textsc{top}.\textsc{advers}  \textsc{2sg} \textsc{ipfv}:\textsc{up}-2$\rightarrow$1-take.away-\textsc{1sg} be.needed:\textsc{fact}  \\
\glt  `Two days ago, during the evening, I clung onto your eldest sister  but she did not take me away (to heaven), yesterday at dusk I clung onto your second eldest sister but she did not take me away, this evening take me away.' (07-deluge, 55)
\end{exe}

\begin{table}
\caption{Evening ordinals} \label{tab:evening.ordinals} \centering
\begin{tabular}{lllll}
\lsptoprule
Day &  Evening & Night \\
\midrule
-2 &   \japhug{jɯfɕɯndʐimɯr}{the evening of two days ago}  \\
-1 &  \japhug{jɯfɕɯmɯr}{yesterday evening} & \japhug{jɯfɕɯɕɤr}{yesterday night}\\
0 &  \japhug{qʰuj}{this afternoon}, \japhug{jɯɣmɯr}{this evening} &\\
+1&   \japhug{fsomɯr}{tomorrow evening} &\\
+2&  \japhug{fsɤndimɯr}{in two days in the evening} \\
+3&\japhug{qʰɤndimɯr}{in three days in the evening} \\
\lspbottomrule
\end{tabular}
\end{table}


Time ordinals can be used either on their own as (\ref{ex:japandzxi}), (\ref{ex:jWfCWndzxi}) or (\ref{ex:jWfCWndzximWr}) above, or be followed by the linker \forme{tɕe} as in (\ref{ex:fsAndi}). They are never used with the locative postpositions \forme{ri}, \forme{zɯ},  or \forme{tɕu} (which are used to mark some temporal adjuncts, §\ref{sec:core.locative}), except for the $D^{-1}$,  $S^{-2}$, $Y^{-1}$ and $Y^{-2}$ ordinals which can occur with \forme{ri}, as in (\ref{ex:japa.ri}).

\begin{exe}
\ex \label{ex:fsAndi}
\gll fsɤndi tɕe li kɯmaʁ ji-kɤ-nɤma ɣɤʑu \\
 in.two.days \textsc{lnk} again other \textsc{1pl}.\textsc{poss}-\textsc{obj}:\textsc{pcp}-work exist:\textsc{sens} \\
\glt `After tomorrow we have something else to do.' (conversation, 2012.12)
\end{exe}
 
\begin{exe}
\ex \label{ex:japa.ri}
\gll  japa ri tɕe, <hongyuan> <caizhengju> ri pɯ-cʰa, <kaoshi> pɯ-cʰa  \\
 last.year \textsc{loc} \textsc{lnk}  \textsc{anthr} finance.office \textsc{loc} \textsc{aor}-can exam \textsc{aor}-can \\
\glt `Last year, he passed the exam for the finance office in Hongyuan.' (12-BzaNsa)
(\japhdoi{0003484\#S69})
\end{exe}
 
  
Time ordinals can be converted to inalienably possessed nouns with the third singular prefix \forme{ɯ-} to change the reference time from the present to a point of time in the past or a to hypothetical time reference. For instance,  from \japhug{fso}{tomorrow} one can build \japhug{ɯ-fso}{the next day} or \japhug{ɯ-fsosŋi}{the next day} (by adding the counted noun \japhug{tɯ-sŋi}{one day}), as in (\ref{ex:WfsosNi}), (\ref{ex:Wfsosoz}) or (\ref{ex:Wfsqaqhe}).

\begin{exe}
\ex \label{ex:WfsosNi}
\gll  ɯ-fso-sŋi tɕe tɕe li pjɤ-nɤtsoʁ-nɯ tɕe \\ 
 \textsc{3sg}.\textsc{poss}-tomorrow-day \textsc{lnk} \textsc{lnk} again \textsc{ifr}-dig.up.silverweed-\textsc{pl}  \textsc{lnk}\\
\glt `The next day, they dug up again silverweed roots.' (07-deluge) (\japhdoi{0003426\#S46})
\end{exe}
  
\begin{exe}
\ex \label{ex:Wfsosoz}
\gll nɯ tɯ-rʑaʁ nɯ a-pɯ-nɯ-fse tɕe, ɯ-fso-soz tɕe ɯ-ci ʑo ɲɯ-ɬoʁ ɲɯ-ŋu  \\
 \textsc{dem} one-night \textsc{dem} \textsc{irr}-\textsc{ipfv}-\textsc{auto}-be.like \textsc{lnk} \textsc{3sg}.\textsc{poss}-tomorrow-morning \textsc{lnk} \textsc{3sg}.\textsc{poss}-water \textsc{emph} \textsc{ipfv}-come.out \textsc{sens}-be \\
\glt `If you leave it one night, the next day in the morning the juice comes out.' (conversation 14.05.10)
\end{exe}
   
\begin{exe}
\ex \label{ex:Wfsqaqhe}
\gll   tɕe tɯ-ji ɯ-ŋgɯ pjɯ-nɯ-ɕe tɕe, ɯ-fsaqʰe tɕe li tu-ɬoʁ ɲɯ-ŋu tɕe.   \\
 \textsc{lnk} \textsc{indef}.\textsc{poss}-field \textsc{3sg}.\textsc{poss}-inside \textsc{ipfv}:\textsc{down}-\textsc{auto}-go  \textsc{lnk} \textsc{3sg}.\textsc{poss}-next.year \textsc{lnk} again \textsc{ipfv}:\textsc{up}-come.out \textsc{sens}-be \textsc{lnk}  \\
\glt `[Its seed] goes into [the soil of] the field, and it grows again the next year.' (13-NanWkWmtsWG)
(\japhdoi{0003492\#S105})
\end{exe}

 \subsection{Other derived time adverbs} \label{sec:time.adv}
Japhug has a series of distributive adverbs meaning `every/each $X$' (\tabref{tab:every.time}). These adverbs are derived from time nominals (mainly counted nouns, but also alienably possessed nouns like \japhug{soz}{morning}) by adding a suffix \forme{-ku} and then applying partial reduplication (§\ref{sec:partial.redp}). If the nominal root has a coda such as that of \forme{-mɯr} `evening' (as in \japhug{tɯ-ɣmɯr}{one evening}), this coda is resyllabified and undergoes reduplication together with the suffix, as in the form \japhug{mɯrkɯrku}{every evening} (not $\dagger$\forme{mɯrkɯku}). 

The \ipa{ɯ} of the preceding syllable tends to become [\forme{u}] due to vowel assimilation (§\ref{sec:vowel.harmony}); since vowel assimilation always occurs in the forms without a reduplicated cluster, the transcriptions \japhug{sɲikuku}{every day}  and  \japhug{pakuku}{every year}  are used in this grammar (\tabref{tab:every.time}) instead of the more phonological representations \ipa{sɲikɯku} and \ipa{pakɯku}.

These adverbs are commonly used with the emphatic marker \forme{ʑo}, as in (\ref{ex:mWrkWrku}).


\begin{exe}
\ex \label{ex:mWrkWrku}
\gll  mɯrkɯrku ʑo kɤntɕʰaʁ a-pi ɯ-pʰe kɯ-nɤpɤri ɲɯ-ɣi-a, ... pɯ-ŋu. \\
every.evening \textsc{emph} {street} \textsc{1sg}.\textsc{poss}-elder.sibling \textsc{3sg}.\textsc{poss}-\textsc{dat} \textsc{sbj}:\textsc{pcp}-have.supper ... \textsc{pst}.\textsc{ipfv}-be \\
\glt `Every evening, if would go in the town at my elder brother's home to have supper and...' (140501 tshering skyid)
(\japhdoi{0003902\#S110})
\end{exe}

This morphological derivation is not productive, and cannot be used with any other temporal counted noun, even \japhug{tɯ-sla}{one month} -- the distributive modifiers \japhug{raŋri}{each} or \japhug{rɯri}{each} are used instead (§\ref{sec:raNri}).

\begin{table}[H]
\caption{Distributive time adverbs} \label{tab:every.time}
\begin{tabular}{lll}
\lsptoprule
Time nominal & Adverb \\
\midrule
\japhug{soz}{morning} & \japhug{soskɯsku}{every morning} \\
\japhug{tɯ-ɣmɯr}{one evening} & \japhug{mɯrkɯrku}{every evening} \\
\japhug{tɯ-sŋi}{one day} & \japhug{sɲikuku}{every day} \\
%\japhug{tɯ-sla}{one month} & \japhug{slɤrɯri}{every months} \\
\japhug{tɯ-xpa}{one year} & \japhug{pakuku}{every year} \\
\lspbottomrule
\end{tabular}
\end{table}

 \subsection{Clock time} \label{sec:hours}
Japhug has a series of words borrowed from Tibetan that can be used to refer to hours and minutes.  The alienably possessed \japhug{tɯtsʰot}{time, hour, clock} from Tibetan \tibet{དུས་ཚོད་}{dus.tsʰod}{time, hour} (note that \forme{tɯ-} here is not a prefix, but represents the Tibetan syllable  \tibet{དུས་}{dus}{time}), if directly followed by a numeral (or the generic counted noun \japhug{tɯ-rdoʁ}{one piece}) refers to hours, as in examples (\ref{ex:tWtshot.sqamnWz}) and (\ref{ex:tWtshot.tWrdoR}).

\begin{exe}
\ex \label{ex:tWtshot.sqamnWz}
\gll ɕɤr tɯtsʰot sqamnɯz ʑo tɕe a-jɤ-tɯ-z-nɤtɯɣ tɕe, \\
 night hour twelve \textsc{emph} \textsc{lnk}  \textsc{irr}-\textsc{pfv}-2-\textsc{caus}-happen.to.be.at \textsc{lnk} \\
\glt `You will have to make sure to be there at midnight.' (2003qachga)
\end{exe}
 
\begin{exe}
\ex \label{ex:tWtshot.tWrdoR}
\gll  tɕe nɯnɯ ci ta-ɕɯ-mŋɤm nɯnɯ tɯtsʰot tɯ-rdoʁ jamar mɤɕtʂa mɤ-ʑi. \\
 \textsc{lnk} \textsc{dem} one \textsc{aor}:3\flobv{}-caus-hurt \textsc{dem} hour one-piece about until \textsc{neg}-subside:\textsc{fact} \\
\glt `When [nettles] start to cause pain, it won't stop for about one hour.' (11-mtshalu) (\japhdoi{0003472\#S7})
\end{exe}

To express a length of time in minutes, the noun   \japhug{tɯtsʰot}{time, hour, clock} is followed by the counted noun \japhug{tɯ-skɤrma}{one minute}, as in (\ref{ex:tWtshot.qamNuskArma}).
  
\begin{exe}
\ex \label{ex:tWtshot.qamNuskArma}
\gll  tɯtsʰot sqamŋu-skɤrma, ɣnɤsqi-skɤrma jamar tɤ-tsu tɕe ku-smi ɕti. \\
 time fifteen-minute twenty-minute about \textsc{aor}-pass \textsc{lnk} \textsc{ipfv}-be.cooked be.\textsc{aff}:\textsc{fact} \\
\glt `After fifteen or twenty minutes, it is cooked.' (160706 thotsi)
(\japhdoi{0006133\#S50})
\end{exe}

To ask about clock time, the interrogative pronoun \japhug{tʰɤstɯɣ}{how many, how much} occurs with the verb \japhug{zɣɯt}{reach} as in (\ref{ex:tWtshot.thAstWG}) (see also §\ref{sec:thAstWG}).

\begin{exe}
\ex \label{ex:tWtshot.thAstWG}
\gll     nɤki nɯ tɯtsʰot tʰɤstɯɣ ko-zɣɯt? \\
 \textsc{dem}:\textsc{distal} \textsc{dem} hour how.many \textsc{ifr}-reach \\
\glt  `At your place, what time is it?' (conversation, 14.12.24 -- the question refers to the time lag between Paris and Mbarkham)
\end{exe}
    
These expressions, although used in the everyday language in Japhug, are calqued from Chinese since they follow modern time counting units and are all recent. Another calque from Chinese includes \japhug{tɯ-tɕɯlɤβ}{one unit of tobacco} as in  (\ref{ex:thamaka.tWtCWlAB}), from the expression \ch{一斗烟的功夫}{yī dǒu yān de gōngfū}{time of one unit of tobacco}.
 

 %gram140505_math.wav
\begin{exe}
\ex \label{ex:thamaka.tWtCWlAB}
\gll    tʰamaka tɯ-tɕɯlɤβ kɤ-sko ɯ-raŋ jamar ɲɯ-ra \\
tobacco  one-tobacco.unit \textsc{inf}-smoke \textsc{3sg}.\textsc{poss}-time about \textsc{sens}-be.needed \\
\glt `One needs about the time of a tobacco smoke.' (elicited)
\end{exe}
    
\section{Basic arithmetic operations} \label{sec:arithmetic}
%gram140505_math.wav

Although Japhug lacks an elaborate mathematical vocabulary, is it possible to express at least the basic arithmetic operations without recourse to Chinese. 

The counted noun \japhug{tɯ-rdoʁ}{one piece} is used instead of the numeral \japhug{ci}{one} is calculations (see \ref{ex:tWrdoR.pjWwGta} below).

Additions are expressed by the construction in (\ref{ex:tWrdoR.pjWwGta}) and (\ref{ex:kWrCat.pjWwGta}),\footnote{The form \forme{kɯɕnɯ-kɯrcat} (with the numeral prefix \forme{kɯɕnɯ-} on the numeral \japhug{kɯrcat}{eight}) in example (\ref{ex:kWrCat.pjWwGta}) is part of a riddle in the story, and is not a normal way to express additions in Japhug. } with the verb \japhug{ta}{put} (or alternatively, \japhug{ɣɤjɯ}{add}) and the locative noun \japhug{ɯ-taʁ}{on top of}, literally  `If one puts Y on the top of X, it makes Z' corresponding to $X+Y=Z$.  

\begin{exe}
\ex \label{ex:tWrdoR.pjWwGta}
\gll kɯmŋu ɯ-taʁ tɯ-rdoʁ pjɯ́-wɣ-ta tɕe kɯtʂɤɣ tu-βze ŋu.  \\
 five \textsc{3sg}-on one-piece \textsc{ipfv}:\textsc{down}-\textsc{inv}-put \textsc{lnk} six \textsc{ipfv}-do[III] be:\textsc{fact} \\
\glt `Five plus one equals six.' (gram140505 math, elicitation)
\end{exe}

 

\begin{exe}
\ex \label{ex:kWrCat.pjWwGta}
\gll  nɤʑo kɯɕnɯ-kɯrcat nɯ pɯpɯŋu nɤ, kɯɕnɯz ɯ-taʁ kɯrcat pjɯ́-wɣ-ta tɕe sqamŋu ɲɯ-ŋu sqamŋu tɕe ju-tɯ-nɯ-ɕe ɲɯ-ŋu tɕe, \\
 \textsc{2sg} seven-eight \textsc{dem} as.for \textsc{lnk} seven  \textsc{3sg}-on eight \textsc{ipfv}:\textsc{down}-\textsc{inv}-put \textsc{lnk}  fifteen   \textsc{sens}-be fifteen \textsc{lnk} \textsc{ipfv}-2-\textsc{vert}-go \textsc{sens}-be \textsc{lnk} \\
\glt `As for the expression `seven eight' [that your father in law has told] you, seven plus eight equals fifteen, [it means that] you go back [to your husband's home] in fifteen days.' (2005tAwakWcqraR)
\end{exe}
  
 
In the case of subtractions, the  main verb is  \japhug{tɕɤt}{take out} combined with the downstream orientation preverb  (the preverb \forme{cʰɯ-} in \ref{ex:tWrdoR.chWwGtCAt}, §\ref{sec:illative.elative}).  The minuend (\forme{tɯ-rdoʁ} in \ref{ex:tWrdoR.chWwGtCAt}) is the object of \japhug{tɕɤt}{take out}, and the subtrahend  (\forme{kɯmŋu} in \ref{ex:tWrdoR.chWwGtCAt}) is encoded with the relator noun \japhug{ɯ-ŋgɯ}{inside} (§\ref{sec:WNgW}). Alternatively, the verb \japhug{sɯxtɕʰaʁ}{make diminish}, causative of \japhug{tɕʰaʁ}{diminish} (§\ref{sec:caus.sWG}) can also occur (selecting the \textsc{downwards} orientation). The construction is exactly the same as in (\ref{ex:tWrdoR.chWwGtCAt}), replacing  \forme{cʰɯ́-wɣ-tɕɤt} by  \forme{pjɯ́-wɣ-sɯxtɕʰaʁ}.

\begin{exe}
\ex \label{ex:tWrdoR.chWwGtCAt}
\gll kɯmŋu ɯ-ŋgɯ tɯ-rdoʁ cʰɯ́-wɣ-tɕɤt tɕe kɯβde ma ɲɯ-me ŋu \\
 five \textsc{3sg}-inside one-piece \textsc{ipfv}-\textsc{inv}-take.out \textsc{lnk} four apart.from \textsc{ipfv}-not.exist be:\textsc{fact} \\
\glt `If one takes out one from five, only four remain = five minus one equals four' (gram140505 math, elicitation)
\end{exe}

Japhug is rarely used for multiplication. The counted noun \japhug{tɯ-zloʁ}{one time} can convey a multiplicative meaning  as in example (\ref{ex:XsW.zloR}). Note that this counted noun of Tibetan origin has two alternative forms for `two times', the regular  \forme{ʁnɯ-zloʁ} and the hybrid form \forme{ʁɲɯ-zloʁ} with an irregular numeral prefix that seems influenced by the Tibetan numeral \tibet{གཉིས་}{gɲis}{two} (see §\ref{sec:tibetan.numerals}, §\ref{sec:irregular.numeral.prefixes}). Divisions are treated in the section on fractions (§\ref{sec:fractions}).

\begin{exe}
\ex \label{ex:XsW.zloR}
\gll  kɯngɯt nɯ χsɯm ɣɯ χsɯ-zloʁ ŋu \\
 nine \textsc{dem} three \textsc{gen} three-time be:\textsc{fact} \\
\glt `Nine is three times three.' (gram140505 math, elicitation)
\end{exe}
 
 
\subsection{Fractions} \label{sec:fractions}
Japhug does not have specific names for fractions other than \japhug{ɯ-qiɯ}{half}.  This inalienably possessed noun has no prefixal form, and to express the meaning `half a $X$', the only available construction is the one illustrated in (\ref{ex:tWJom.WqiW}) and (\ref{ex:tWxpa.WqiW}), combining a counted noun with the numeral prefix `one', the genitive \forme{ɣɯ} and \japhug{ɯ-qiɯ}{half}.


\begin{exe}
\ex \label{ex:tWJom.WqiW}
\gll tɯ-ɟom ɣɯ ɯ-qiɯ jamar ɲɯ-rɲɟi \\
one-length.of.two.outstretched.arms \textsc{gen} \textsc{3sg}.\textsc{poss}-half about \textsc{sens}-be.long \\
\glt `[Its tail] is about half a fathom (i.e. one arm) long.' (24-ZmbrWpGa)
(\japhdoi{0003628\#S63})
\end{exe}

\begin{exe}
\ex \label{ex:tWxpa.WqiW}
\gll nɯnɯ tɯ-xpa ɣɯ ɯ-qiɯ nɯ ɲɤ-ɕqʰlɤt. \\
\textsc{dem} one-year \textsc{gen} \textsc{3sg}.\textsc{poss}-half \textsc{dem} \textsc{ifr}-disappear \\
\glt  `Half a year passed.' (150907 laoshandaoshi-zh)
(\japhdoi{0006398\#S97})
\end{exe}

The denominal verb \japhug{nɤqiɯ}{pass half of} (§\ref{sec:denom.intr.nW}) can be used with temporal counted nouns to express the same meaning (\ref{ex:YAnAqiW}).

\begin{exe}
	\ex \label{ex:YAnAqiW}
	\gll tɯ-sla ɲɤ-nɤqiɯ \\
	one-month \textsc{ifr}-pass.half \\
	\glt `Half a month passed.' (elicited)
\end{exe}

The additive interpretation  `and a half' can be expressed by coordinating the counted noun and the inalienably possessed \japhug{ɯ-qiɯ}{half} using the comitative \japhug{cʰo}{with}, as in (\ref{ex:tWJom.cho.WqiW}).

\begin{exe}
\ex \label{ex:tWJom.cho.WqiW}
\gll ki kɯ-fse tɯ-ɟom cʰo ɯ-qiɯ jamar kɯ-zri ra \\
 \textsc{dem}.\textsc{prox} \textsc{sbj}:\textsc{pcp}-be.like one-length.of.two.outstretched.arms \textsc{comit} \textsc{3sg}.\textsc{poss}-half about \textsc{sbj}:\textsc{pcp}-be.long be.needed:\textsc{fact} \\
\glt `They need [a piece of leather] long like one fathom and a half.' (24-mbGo)
\end{exe}

Alternatively, the additive meaning `and a half' can be expressed with the counted noun \japhug{tɯ-qiɯ}{one half}, which derives from  \japhug{ɯ-qiɯ}{half} (§\ref{sec:CN.IPN}). 

\begin{exe}
\ex \label{ex:tWqiW}
\gll tɕe ɯ-pɤrtʰɤβ tɕe, tɯ-pɤrme tɯ-qiɯ jamar, ʁnɯ-pɤrme jamar ma kɯ-me ʁɟa. \\
 \textsc{lnk} \textsc{3sg}.\textsc{poss}-between \textsc{lnk} one-year one-half about, two-year about apart.from \textsc{sbj}:\textsc{pcp}-not.exist completely  \\
\glt  `(Some women had thirteen or fifteen children) Between each of them, there was only one year and a half or two years.' (140426 tApAtso kAnWBdaR, 89; see the preceding sentence in example \ref{ex:tsuku.tWrme.tWrdoR} in §\ref{sec:tsuku})
\end{exe}

For more complex fractions, the counted nouns \japhug{tɯ-tɯcɯr}{one part}, \japhug{tɯ-tɤsɯm}{one part} or \japhug{tɯ-tɯkro}{one part} are used. Two related constructions are possible; in the following discussion, $X$ represents the numerator, $Y$ the denominator (the fraction $\frac{X}{Y}$). 

The first one is a noun phrase with the structure $Y$-\forme{tɯcɯr} \forme{ɯ-ŋgɯ} $X$-\forme{tɯcɯr}, literally `$X$ parts among $Y$ parts', illustrated by example (\ref{ex:kWmNutWcWr}). This may be a calque of the Chinese construction $Y$\zh{分之}$X$ 
($Y$-fēn zhī-$X$) literally `$X$ of $Y$ parts', a way of expressing fractions attested from Han dynasty documents \citep{anicotte15fractions} to standard Mandarin.
 
\begin{exe}
\ex \label{ex:kWmNutWcWr}
\gll kɯmŋu-tɯcɯr ɯ-ŋgɯ χsɯ-tɯcɯr  \\
 five-part \textsc{3sg}-inside three-part \\
\glt $\dfrac{3}{5}$ = `Three parts among five parts.' (elicited)
\end{exe}

An alternative possibility is to use the auxiliary verb \japhug{lɤt}{release} as in (\ref{ex:XsWtWcWr}), but such construction is biclausal (the second clause lacks a verbal predicate, §\ref{sec:non.verbal.predicates}).

\begin{exe}
\ex \label{ex:XsWtWcWr}
\gll χsɯ-tɯcɯr tú-wɣ-lɤt tɕe tɯ-tɯcɯr   \\
 three-part \textsc{ipfv}-\textsc{inv}-release \textsc{lnk} one-part \\
\glt $\dfrac{1}{3}$; literally `Making three parts, one part.' (elicited)
\end{exe}

Divisions can be expressed using the constructions in  (\ref{ex:division}) and (\ref{ex:division2}).

\begin{exe}
\ex
\begin{xlist}
\ex \label{ex:division}
\gll sqi nɯ kɯmŋu-tɯcɯr tú-wɣ-lɤt tɕe, ʁnɯz nɤ ʁnɯz ɲɯ́-wɣ-βzu kʰɯ \\
 ten \textsc{dem} five-part \textsc{ipfv}-\textsc{inv}-release \textsc{lnk} two \textsc{lnk} two \textsc{ipfv}-\textsc{inv}-make be.possible:\textsc{fact} \\
 \ex \label{ex:division2}
\gll sqi nɯ kɯmŋu-tɯkro ɲɯ́-wɣ-lɤt tɕe, ʁnɯz nɤ ʁnɯz ɲɯ́-wɣ-sɤβzu kʰɯ \\
 ten \textsc{dem} five-part \textsc{ipfv}-\textsc{inv}-throw \textsc{lnk} two \textsc{lnk} two \textsc{ipfv}-\textsc{inv}-transform be.possible:\textsc{fact} \\
 \end{xlist}
\glt $\dfrac{10}{5}=2$; literally `Making 5 parts out of 10, one can make them in sets of two'.
\end{exe}

In these examples, the numerator is a left-dislocated numeral followed by the determiner \forme{nɯ}, the denominator a counted noun serving as object of \japhug{lɤt}{release} as in the fraction in (\ref{ex:XsWtWcWr}), and the result of the computation is indicated by a distributive repetition of the numeral with the additive postposition \forme{nɤ} (§\ref{sec:CN.repetition}).
 
