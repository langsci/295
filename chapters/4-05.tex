\chapter{Valency-increasing derivations} \label{chap:valency.increasing.derivation}

\section{Introduction} 
Japhug has much fewer valency-increasing derivations than valency-decreasing ones: leaving aside fossil derivations (§\ref{sec:marginal.derivations}), only four clearly identified prefixes are found: sigmatic causative (§\ref{sec:sig.causative}), velar causative (§\ref{sec:velar.causative}), applicative (§\ref{sec:applicative}) and tropative (§\ref{sec:tropative}). The sigmatic causative however is probably the most productive of all derivational processes in Japhug,  and can be combined with the other three (§\ref{sec:inner.prefixal.chain}, §\ref{sec:sig.caus.other.derivations}, §\ref{sec:appl.other.derivations}, §\ref{sec:velar.caus.other}).
 

\section{Sigmatic causative} \label{sec:sig.causative}
\is{causative!sigmatic}
Despite the existence of periphrastic causative constructions (§\ref{sec:sWpa.sABzu}), the main morphosyntactic device to express causation in Japhug is the causative verbal derivation by dental or alveolo-palatal fricative prefixes, referred to as `sigmatic' causative in this work.  

\subsection{Regular allomorphy} \label{sec:sig.caus.allomorphs}
\is{causative!allomorphy}\is{morphology!allomorphy} \is{allomorphy!causative}
Although not as complex as the causative derivations in Stodsde \citep{jackson07shangzhai} or in Khroskyabs \citep{lai16caus}, the sigmatic causative prefix is the derivation with the greatest number of allomorphs in Japhug.

It has five regular allomorphs \forme{sɯ\trt}, \forme{sɯɣ\trt}, \forme{z\trt}, \forme{s-} and \forme{sɤ-} depending on the following element, and a number of irregular ones (§\ref{sec:sig.caus.irregular}). The allomorph \forme{sɯ-} occurs in most environments, and can be considered to be the default form.

\subsubsection{\forme{z-} allomorph} \label{sec:caus.z}
The \forme{z-} allomorph appears in non-monosyllabic verb bases, when the first syllable (generally a derivational prefix, or a synchronically non-analysable prefixal element  belonging to the verb root) has a sonorant initial (in practice only \forme{mV\trt}, \forme{nV\trt}, \forme{ɣV-} or \forme{rV-}). \tabref{tab:causative.z} illustrates some examples of this allomorph.

\begin{table}
\caption{Examples of the \forme{z-} allomorph of the causative prefix}\label{tab:causative.z} 
\begin{tabular}{lllllllll} 
\lsptoprule
Nature of the  & Base verb &Derived verb \\
prefixal element && \\
\midrule
non-analyzable &  \japhug{nɯna}{rest} & \japhug{znɯna}{stop} \\
 &  \japhug{ɣɯrni}{be red} & \japhug{zɣɯrni}{redden} \\
 \midrule
 §\ref{sec:volitional.mW} &  \japhug{mɯnmu}{move} & \japhug{zmɯnmu}{cause to move} \\
 \midrule
denominal &  \japhug{nɤma}{do} & \japhug{znɤma}{make/let do} \\
 &  \japhug{mɤku}{be first} & \japhug{zmɤku}{make/do first} \\
  \midrule
antipassive  &  \japhug{rɤrɤt}{write/draw things} & \forme{zrɤrɤt} `cause to write/draw\\
&& things', `draw/write with' \\
\lspbottomrule
\end{tabular}
\end{table}

On the other hand, monosyllabic verbs with single sonorant initials, whether nasals, rhotics or semi-vowels, never select the \forme{z-} allomorph. Monosyllabic bases with these initials take the allomorph \forme{sɯ-} (§\ref{sec:caus.sWG}):  \japhug{no}{drive} and  \japhug{mar}{smear} have the causative forms \japhug{sɯno}{make/let drive}, `drive with' and \japhug{sɯmar}{make/let smear}, not $\dagger$\forme{zno} and $\dagger$\forme{zmar}. The same constraint on monosyllabicity is observed with the regular \forme{s-} allomorph (§\ref{sec:caus.s.allomorph}), but does not apply to the irregular vowel-less allomorphs of the causative (\forme{ɕ-} §\ref{sec:caus.C}, \forme{ʑ-} §\ref{sec:caus.Z}, and \forme{j-} §\ref{sec:caus.j}). 

Intransitive bases with nasal (except \forme{ŋ-}), rhotic and semi-vowel initial consonants select the \forme{sɯɣ-} allomorph, as illustrated by \japhug{jɤɣ}{finish}, \japhug{ru}{look at} and \japhug{ɲo}{be prepared}, whose sigmatic causative forms are \japhug{sɯɣjɤɣ}{finish}, \japhug{sɯɣru}{make/let look at} and \japhug{sɯɣɲo}{prepare}, respectively. 
 
In the case of intransitive verbs with the velar sonorant \forme{ɣ\trt}, the allomorph \forme{sɯ-} occurs. For instance, \japhug{ɣi}{come} has the causative form \japhug{sɯɣe}{invite} (with irregular ablaut, §\ref{sec:sig.caus.irregular.other}).

Causative verbs with the \forme{z-} allomorph in irregular contexts are discussed in §\ref{sec:sig.caus.irregular.other}.
\subsubsection{\forme{s-} allomorph} \label{sec:caus.s.allomorph}
The \forme{s-} allomorph is only found with  base verbs whose first syllable is \forme{qa-} (\tabref{tab:causative.sqa}). All of these verbs are intransitive; it is unclear whether this \forme{qa-} element is analyzable as a prefix historically.

Monosyllabic verbs with initial \forme{q-} always select the \forme{sɯ-} allomorph. For example, the causative of \japhug{qaʁ}{peel} is \japhug{sɯqaʁ}{make peel}, not $\dagger$\forme{sqaʁ}.

\begin{table}
\caption{Examples of the \forme{s-} allomorph of the causative prefix}\label{tab:causative.sqa} 
\begin{tabular}{lllllllll} \lsptoprule
Base verb &Derived verb \\
\midrule
\japhug{qanɯ}{be dark}   &  \japhug{sqanɯ}{put in darkness}  \\ 
\japhug{qapɯ}{be fallow} (of a field)   &  \japhug{sqapɯ}{leave fallow}  \\ 
\japhug{qarndɯm}{be murky}   &  \japhug{sqarndɯm}{make murky}  \\ 
\lspbottomrule
\end{tabular}
\end{table}

\subsubsection{Vowel fusion} \label{sec:caus.sA}
With verbs whose stem begins in \forme{a-} (contracting verbs, §\ref{sec:contraction}), the sigmatic causative prefix merges with this vowel as \forme{sɤ\trt}, as shown in \tabref{tab:causative.sA}. In the glosses, this vowel merger is represented as \forme{sɯ-ɤ\trt}, following the orthographic rules in (§\ref{sec:contraction}). Only one contracting verb has an irregular causative with intrusive \forme{-ɣ-} (§\ref{sec:caus.sAG}).
 

\begin{table}
\caption{The \forme{sɤ-} allomorph of the causative prefix}\label{tab:causative.sA} 
\begin{tabular}{lll} 
\lsptoprule
Base verb &Derived verb \\
 \midrule
\japhug{aɕqʰe}{cough}  & \japhug{sɤɕqʰe}{cause to cough} \\
\japhug{ajtɯ}{accumulate} (vi)  & \japhug{sɤjtɯ}{accumulate} (vt) \\
\japhug{amɲɤm}{be homogeneous} & \japhug{sɤmɲɤm}{do homogeneously} \\
\lspbottomrule
\end{tabular}
\end{table}

There are two irregular causative verbs in \forme{sɤ\trt}, whose base verb is not a contracting verb (§\ref{sec:sig.caus.irregular.other}).

\subsubsection{\forme{sɯɣ-} allomorph} \label{sec:caus.sWG}
For all other types of verb stem, the choice between the \forme{sɯ-} and \forme{sɯɣ-/sɯx-} allomorphs depends on both phonology and morphology. The \forme{sɯɣ-/sɯx-} allomorphs occur when the base verb is intransitive, monosyllabic, has no initial cluster and no velar or uvular initial consonant, while \forme{sɯ-} appears in all other cases, in particular in all bases with consonant clusters. 

With the intransitive verb \japhug{ʁaʁ}{hatch} (the only verb with the single \forme{ʁ-} onset), some speakers (such as \iai{Tshendzin}) select the \forme{sɯɣ-} allomorph and use the causative form \japhug{sɯɣʁaʁ}{cause to hatch}  (\ref{ex:nWnWsWGRaRa}) with an internal \ipa{-ɣʁ-} cluster (§\ref{sec:heterosyllabic.clusters}), while other speakers select the \forme{sɯ-} allomorph.

\begin{exe}
\ex \label{ex:nWnWsWGRaRa}
\gll tɕendɤre nɯnɯ kɤ-ɣɯt-a tɕe tɕe nɯ-nɯ-sɯɣ-ʁaʁ-a    \\
\textsc{lnk} \textsc{dem} \textsc{aor}:\textsc{east}-bring-\textsc{1sg} \textsc{lnk} \textsc{lnk} \textsc{aor}-\textsc{auto}-\textsc{caus}-hatch-\textsc{1sg}   \\
\glt `I brought [the eggs] and made them hatch by myself (without a hen).' (150819 kumpGa)
\japhdoi{0006388\#S42}
\end{exe}

\tabref{tab:causative.sW} illustrates the correlation between the \forme{sɯ-} / \forme{sɯɣ-/sɯx-} contrast and transitivity. The intrusive \forme{-ɣ-} element undergoes regressive voice assimilation to \forme{-x-} when the initial of the verb root is unvoiced.   
 

\begin{table}
\caption{The \forme{sɯ-} and \forme{sɯɣ-/sɯx-} allomorphs of the sigmatic causative prefix} \label{tab:causative.sW}
\begin{tabular}{lllllllll} 
\lsptoprule
Transitivity & Base verb & Derived verb & \\
\midrule
intr. & \japhug{mbuz}{overflow} &\japhug{sɯɣmbuz}{let overflow}  \\
tr. & \japhug{mbi}{give} & \japhug{sɯmbi}{make/let give}, `give with' \\ 
\midrule
intr. & \japhug{ɕe}{go} & \japhug{sɯxɕe}{send} \\
tr. & \japhug{ɕɯm}{brood} & \japhug{sɯɕɯm}{make/let brood} \\
\midrule
intr. & \japhug{tso}{know, understand} & \japhug{sɯxtso}{make understand} \\
tr. & \japhug{tsɯm}{take away} & \japhug{sɯtsɯm}{send with} \\ 
\midrule
intr. & \japhug{ndzur}{stand} & \japhug{sɯɣndzur}{make/let stand up} \\
tr. & \japhug{ndza}{eat} & \japhug{sɯndza}{make/let eat} \\ 
\midrule
intr. & \japhug{nɤz}{dare} & \japhug{sɯɣnɤz}{cause to dare} \\
tr. & \japhug{no}{drive} & \japhug{sɯno}{make/let drive}, `drive with' \\
\lspbottomrule
\end{tabular}
\end{table}

The intrusive \forme{-ɣ-} also appears in one of the irregular allomorphs of the sigmatic causative (§\ref{sec:caus.CWG}).

Other derivational prefixes, including the velar causative  \forme{ɣɤ-} (§\ref{sec:allomorphy.applicative}), the applicative \forme{nɯ-} (§\ref{sec:allomorphy.applicative}), the tropative \forme{nɤ-}  (§\ref{sec:tropative.allomorphy}), the proprietive \forme{sɤ-}  (§\ref{sec:proprietive.allomorphy}) and some denominal derivations (§\ref{sec:denom.sW.caus.instr}), present an allomorphy involving the insertion of the \forme{-ɣ-} element, originally in the same context as that of the \forme{sɯɣ-} allomorph of the sigmatic causative. However, in the case of the proprietive and tropative derivation, this \forme{-ɣ-} insertion has ceased to be productive.


\subsection{Irregular allomorphs} \label{sec:sig.caus.irregular}
\is{causative!allomorphy}\is{morphology!allomorphy}
In addition to the regular allomorphs described in the previous section, the sigmatic causative has five irregular allomorphs with alveolo-palatal or palatal consonants instead of alveolar fricatives: \forme{ɕɯ\trt}, \forme{ɕɯɣ\trt}, \forme{ɕ\trt}, \forme{ʑ-} and \forme{j-}. All known examples are presented in \tabref{tab:causative.irregular}.


\begin{table}
\caption{The irregular allomorphs of the causative prefix}\label{tab:causative.irregular} 
\begin{tabular}{llll} 
\lsptoprule
Base verb & Derived verb \\
 \midrule
\japhug{fka}{be full} &  \japhug{ɕɯfka}{cause to be full}  \\ 
\japhug{fkaβ}{cover} &  \japhug{ɕɯfkaβ}{cover with} \\ 
\japhug{mbɣom}{be in a hurry} &  \japhug{ɕɯmbɣom}{cause to be in a hurry} \\ 
\japhug{mnɤm}{smell} &  \japhug{ɕɯmnɤm}{cause to have a smell} \\ 
\japhug{mŋɤm}{feel pain}  (of a body part) &  \japhug{ɕɯmŋɤm}{cause pain} (vt) \\ 
\japhug{ntaβ}{be stable}  &  \japhug{ɕɯntaβ}{leave} (there)      \\ 
\japhug{ngo}{be ill}   &  \japhug{ɕɯngo}{make sick}  \\ 
\japhug{nŋo}{lose}   &  \japhug{ɕɯnŋo}{win}   \\ 
\japhug{ɴqoʁ}{hang} (vi) &  \japhug{ɕɯɴqoʁ}{hang} (vt)  \\ 
\japhug{rŋo}{borrow}  &  \japhug{ɕɯrŋo}{lend}     \\ 
\japhug{tɤ-mbrɯ+ŋgɯ}{get angry}  &  \japhug{tɤ-mbrɯ+ɕɯŋgɯ}{anger} (vt) \\ 
\japhug{rŋgɯ}{lie down}  &  \japhug{ɕɯrŋgɯ}{make/let lie down} \\
%&&&ferment (alcohol)   \\ 
\japhug{rga}{be happy}  &  \japhug{ɕɯrga}{please} (vt)  \\ 
\midrule
\japhug{mu}{be afraid}  &  \japhug{ɕɯɣmu}{frighten}   \\ 
\midrule
\japhug{pʰɣo}{flee}  &  \japhug{ɕpʰɣo}{flee with}      \\ 
\japhug{lɯɣ}{get loose}  &  \japhug{ɕlɯɣ}{drop}   \\ 
\midrule
\japhug{ɴqoʁ}{hang} (vi) &  \japhug{ʑɴɢoʁ}{hang} (on a hook)  \\ 
\japhug{ŋga}{wear}  &  \japhug{ʑŋga}{help wearing}  \\ 
\japhug{mbri}{cry, sing}    &  \japhug{ʑmbri}{play} (an instrument) \\ 
\midrule
\japhug{tsʰi}{drink}   &  \japhug{jtsʰi}{give to drink}  \\ 
 \lspbottomrule
\end{tabular}
\end{table}

\subsubsection{\forme{ɕɯ-} allomorph} \label{sec:caus.CW}
The \forme{ɕɯ-} allomorph is the most common of all alveolo-palatal allomorphs. It occurs on verb bases with initial clusters with nasal, \forme{f-} or \forme{r-} preinitials, contexts where the regular \forme{sɯ-} allomorph would be expected (§\ref{sec:caus.sWG}). The only clusterless verb root which selects the  \forme{ɕɯ-} allomorph is the orphan verb \forme{ŋgɯ} in the complex predicate \japhug{tɤ-mbrɯ+ŋgɯ}{get angry} (§\ref{sec:orphan.verb}).

The \forme{ɕɯ-} allomorph  derives causatives mainly from intransitive verbs, except for \japhug{rŋo}{borrow}  (with the inversive causative \japhug{ɕɯrŋo}{lend}, §\ref{sec:sig.caus.inversive}) and \japhug{fkaβ}{cover} (which has the instrumental causative  \japhug{ɕɯfkaβ}{cover with}, §\ref{sec:sig.caus.instrumental}).

The base verb \japhug{ɴqoʁ}{hang} has three causative forms, \forme{ɕɯ-ɴqoʁ} (for the most common meaning corresponding to transitive `hang'), \forme{ʑ-ɴɢoʁ}
 (with a more restricted meaning, §\ref{sec:caus.Z}) and the regular \forme{sɯ-ɴqoʁ} for instrumental (`hang with') or indirect causation.
 
\subsubsection{\forme{ɕɯɣ-} allomorph} \label{sec:caus.CWG}
The verb  \japhug{mu}{be afraid} has the causative \japhug{ɕɯɣmu}{frighten}, the only example of the   \forme{ɕɯɣ-} allomorph of the sigmatic causative. The intrusive  \forme{-ɣ-} is expected since the base verb is intransitive and has a labial initial consonant with no cluster (§\ref{sec:caus.sWG}), and appears in other derivations such as applicative (\japhug{nɯɣmu}{be afraid of}, §\ref{sec:allomorphy.applicative}) and proprietive (\japhug{sɤɣmu}{be frightening}, §\ref{sec:proprietive.allomorphy}).
 
\subsubsection{\forme{ɕ-} allomorph} \label{sec:caus.C}
The \forme{ɕ-} allomorph is one of the three irregular vowel-less allomorphs of the causative. It occurs on two monosyllabic verbs, \japhug{ɕpʰɣo}{flee with}\footnote{An alternative form \forme{ɕɯpʰɣo} with the \forme{ɕɯ-} allomorph is also attested. } (from \japhug{pʰɣo}{flee}) and \japhug{ɕlɯɣ}{drop} (from \japhug{lɯɣ}{get loose}), unlike the vowel-less regular allomorphs \forme{z-} and \forme{s-} which are never found on monosyllabic bases (§\ref{sec:caus.z}, §\ref{sec:caus.s.allomorph}). The contrast between \forme{ɕ-} and \forme{ʑ-} is not determined by the voicing of the initial consonant of the base verb (since \japhug{lɯɣ}{get loose} has a sonorant initial \forme{l-}). Rather, \forme{ʑ-} exclusively occurs with voiced prenasalized obstruents (§\ref{sec:caus.Z}), and \forme{j-} derives from \forme{ɕ-} by a recent sound change (§\ref{sec:caus.j}). 

Like other irregular allomorphs, \forme{ɕ-} is not restricted to one particular sub-func\-tion of the sigmatic causative. The verb \japhug{ɕpʰɣo}{flee with}, `help $X$ flee with oneself' reflects the adjutative function of causative (§\ref{sec:sig.caus.adjutative}), indexing as direct object the entity helped by the subject, as shown by the \textsc{3sg}\fl{}\textsc{2pl} configuration in (\ref{ex:ajAtWwGCphGonW}).

\begin{exe}
\ex \label{ex:ajAtWwGCphGonW}
\gll nɯnɯ ɯ-pʰe ``wortɕʰi" tɤ-ti-nɯ tɕe nɯnɯ kɯ a-jɤ-tɯ́-wɣ-ɕ-pʰɣo-nɯ ma, \\
\textsc{dem} \textsc{3sg}.\textsc{poss}-\textsc{dat} please \textsc{imp}-say-\textsc{pl} \textsc{lnk} \textsc{dem} \textsc{erg} \textsc{irr}-\textsc{pfv}-2-\textsc{inv}-\textsc{caus}-flee-\textsc{pl} \textsc{lnk} \\
\glt `Say `please' to him, and he will help you flee [from here].' (2012 Norbzang)
\japhdoi{0003768\#S65}
\end{exe}

On the other hand, \japhug{ɕlɯɣ}{drop} generally expresses non-volitional causation, especially when used with the autive prefix as in (\ref{ex:qandzxe.pjAnWClWG}) (see also \ref{ex:kW.pjAsWGWtnW} in §\ref{sec:sig.caus.morphosyntax}) but even without it (\ref{ex:toCLWGa}).

\begin{exe}
\ex \label{ex:qandzxe.pjAnWClWG}
\gll ɯ-xɕɤt tɯ\redp{}tu ʑo to-ɣɤɕqali ri tɕe ɯ-kɯr ɯ-ŋgɯ qandʐe nɯ pjɤ-nɯ-ɕ-lɯɣ tɕe ɯ-zda ra kɯ jo-nɯ-tsɯm-nɯ \\
\textsc{3sg}.\textsc{poss}-strength \textsc{total}\redp{}exist:\textsc{fact} \textsc{emph} \textsc{ifr}-shout \textsc{lnk} \textsc{lnk} \textsc{3sg}.\textsc{poss}-mouth \textsc{3sg}.\textsc{poss}-in earthworm \textsc{dem} \textsc{ifr}:\textsc{down}-\textsc{auto}-\textsc{caus}-get.loose \textsc{lnk} \textsc{3sg}.\textsc{poss}-companion \textsc{pl} \textsc{erg} \textsc{ifr}-\textsc{auto}-take.away-\textsc{pl} \\
\glt `[The crow] shouted with all its strength, dropped the earthworm that was in its mouth, and its companion took it.' (2011-10-qajdo)
\end{exe}

\begin{exe}
\ex \label{ex:toCLWGa}
\gll kupa-skɤt to-ɕ-lɯɣ-a \\
Chinese-language \textsc{ifr}-\textsc{caus}-get.loose-\textsc{1sg} \\
\glt `I spoke Chinese by mistake.' (`I should have spoken Japhug') (heard in context)
\end{exe}

The regular causative of \japhug{lɯɣ}{get loose} is \forme{sɯɣ-lɯɣ} (§\ref{sec:caus.sWG}), and is used to express a volitional causation `untie, detach, take off', as in (\ref{ex:WCombri.YAsWGlWG}). The \japhug{lɯɣ}{get loose} itself cannot express a volitional action; this meaning is provided by the reflexive-causative \forme{ʑɣɤ-sɯɣ-lɯɣ} `detach oneself' (§\ref{sec:refl.caus.volitional}).

\begin{exe}
\ex \label{ex:WCombri.YAsWGlWG}
\gll tɤrʁaʁkɕi nɯ kɯ nɯra ɯ-ɕombri ra ɲɤ-sɯɣ-lɯɣ \\
hunting.dog \textsc{dem} \textsc{erg} \textsc{dem}.\textsc{pl} \textsc{3sg}.\textsc{poss}-chain \textsc{pl} \textsc{ifr}-\textsc{caus}-get.loose \\
\glt `The hunting dog took off its chains.' (140426 liegou he zhonggo-zh)
\japhdoi{0003812\#S10}
\end{exe}


\subsubsection{\forme{ʑ-} allomorph} \label{sec:caus.Z}
The \forme{ʑ-} allomorph of the causative is only attested with verb bases having prenasalized onsets. In each of the three verbs, the meaning of the prefix is slightly different.
 
The causative \japhug{ʑmbri}{play} (an instrument), `make noise with' from the intransitive \japhug{mbri}{cry, sing}, `make noise', is a plain instrumental causative (§\ref{sec:sig.caus.instrumental}), which selects as object the musical instrument (see for instance \ref{ex:taZmbri}, §\ref{sec:absolutive.P}), while \japhug{ʑŋga}{help wearing}, `make/force to wear'  (from the transitive \japhug{ŋga}{wear}) reflects the  adjutative  (§\ref{sec:sig.caus.adjutative}) function rather than the instrumental one, since it selects as object the person wearing the clothes (the \textsc{3du} object indexation in \ref{ex:tuwWGZngandZi}), while the clothes are a semi-object (\forme{tɯ-ŋga nɯra ɯ-mbe tʰɯ-kɯ-ɴɢraʁ nɯra} `old and torn clothes', without ergative marking).

\begin{exe}
\ex \label{ex:tuwWGZngandZi}
\gll [tɯ-ŋga nɯra ɯ-mbe tʰɯ-kɯ-ɴɢraʁ nɯra] tú-wɣ-ʑ-ŋga-ndʑi, \\
\textsc{indef}.\textsc{poss}-clothes \textsc{dem}:\textsc{pl} \textsc{3sg}.\textsc{poss}-old.one \textsc{aor}-\textsc{sbj}:\textsc{pcp}-\textsc{acaus}:damage \textsc{dem}:\textsc{pl} \textsc{ipfv}-\textsc{inv}-\textsc{caus}-wear-\textsc{du} \\
\glt `[Their$_i$ stepmother was evil], and made the two of them$_i$ wear old and torn clothes.' (140429 jiedi-zh)
\end{exe}

The causative \forme{ʑɴɢoʁ} from \japhug{ɴqoʁ}{hang} (vi) presents an irregular voicing of the onset, possibly the effect of a phonotactic constraint against clusters comprising a preinitial fricative with a unvoiced prenasalized obstruent such as \forme{*ɕɴq\trt}, since only \textit{voiced} prenasalized obstruents are monophonemic (§\ref{sec:consonant.phonemes}).

The meaning of \forme{ʑɴɢoʁ} is not completely predictable from that of the base verb \japhug{ɴqoʁ}{hang}. Although it can be translated as transitive `hang' like the other irregular causative  \japhug{ɕɯɴqoʁ}{hang} (§\ref{sec:caus.CW}), its more common meaning is `pull threads (that are coiled around one's fingers) apart (as part of the weaving process)' (in Chinese \ch{牵线}{qiānxiàn}{pull the threads}), as in (\ref{ex:atAri.kAZNGoR}).

\begin{exe} 
\ex \label{ex:atAri.kAZNGoR}
\gll a-tɤ-ri kɤ-ʑɴɢoʁ \\
\textsc{1sg}.\textsc{poss}-\textsc{indef}.\textsc{poss}-thread \textsc{imp}-hang \\
\glt `Pull the threads for me.' (elicited)
\end{exe}

With an additional causative prefix (§\ref{sec:sig.caus.other.recursion}) in instrumental function (§\ref{sec:sig.caus.instrumental}), it specifically means `hang on a hook' as in (\ref{ex:kosWZNGoR}).

\begin{exe} 
\ex \label{ex:kosWZNGoR}
\gll tɤjŋoʁ kɯ tɯ-ŋga ko-sɯ-ʑɴɢoʁ \\
hook \textsc{erg} \textsc{indef}.\textsc{poss}-clothes \textsc{ifr}-\textsc{caus}-hang \\
\glt `He hung the clothes on the hook.' (elicited)
\end{exe}
%sɯɴqoʁ

\subsubsection{\forme{j-} allomorph} \label{sec:caus.j} 
In the Kamnyu dialect of Japhug, the verb \japhug{tsʰi}{drink} has the irregular causative form \japhug{jtsʰi}{give to drink} with the \forme{j-} allomorph. In dialects of Japhug which have not undergone the \forme{tʰi} \fl{} \forme{tsʰi} sound change, the base verb is \forme{tʰi} and its causative \forme{ɕtʰi} with the \forme{ɕ-} allomorph (§\ref{sec:caus.C}). This suggests that a sound change \forme{*ɕtsʰ-} \fl{} \forme{jtsʰ-} took place in this word by dissimilation of mode of articulation (§\ref{sec:shC.clusters}).

The causative \japhug{jtsʰi}{give to drink} is highly lexicalized, and can occur as input for other derivations such as antipassive (§\ref{sec:sig.caus.other.derivations}, §\ref{sec:antipassive.ditransitive}). The regular causative \forme{sɯ-tsʰi}  can be used with a neutral factitive meaning `make drink' (\ref{ex:qajW.smAnba.kAsWtshi}) or in the instrumental function `drink with/using' (§\ref{sec:sig.caus.instrumental}).

\begin{exe} 
\ex \label{ex:qajW.smAnba.kAsWtshi}
\gll qajɯsmɤnba [...] tɕe nɯnɯ tɤ-fka qʰe, tɕe pjɯ-ɤtɤr. [...] tɕe nɯnɯ ɯ-qʰu qʰe, nɯ kɤ-sɯ-tsʰi mɯ́j-kʰɯ. \\
leech { } \textsc{lnk} \textsc{dem} \textsc{aor}-be.full \textsc{lnk} \textsc{lnk} \textsc{ipfv}-fall {  }  \textsc{lnk} \textsc{dem} \textsc{3sg}.\textsc{poss}-after \textsc{lnk} \textsc{dem} \textsc{inf}-\textsc{caus}-drink \textsc{neg}:\textsc{sens}-be.possible \\
\glt `After it$_i$ has had its fill, the leech$_i$ [detaches and] falls down. After that, it is not possible to make it$_i$ drink [blood].' (28-kWpAz)
\japhdoi{0003714\#S133}
\end{exe}
 
 
\subsubsection{Irregular vowel fusion} \label{sec:caus.sAG}
Tthe causative verb \japhug{sɯxɕe}{send} (from \japhug{ɕe}{go}) has a stem II form \forme{sɤɣri} derived from the stem II \forme{ari} of its base verb (§\ref{sec:stem2}). This stem II appears to present the merger of the \forme{sɯɣ-} allomorph with the \forme{a-} prefixal element as \forme{sɤɣ\trt}, with preservation of the \forme{-ɣ-} element, instead of expected $\dagger$\forme{sɤri}.


\subsubsection{Other irregularities} \label{sec:sig.caus.irregular.other}
The denominal verb \japhug{sɤmbrɯ}{get angry} (§\ref{sec:denom.sA.proprietive}) has a causative form \japhug{sɤzmbrɯ}{anger} (\ref{ex:tAtasAzmbrW}) with the \forme{z-} allomorph (§\ref{sec:caus.z}) infixed rather than prefixed (occurring between the denominal \forme{sɤ-} prefix and the nominal root \forme{-mbrɯ}).

\begin{exe} 
\ex \label{ex:tAtasAzmbrW}
\gll nɯ maʁ kɯ tɤ-ta-sɤ<z>mbrɯ tu ɯ́-ŋu \\
 \textsc{dem} not.be:\textsc{fact} \textsc{erg} \textsc{aor}-1\fl{}2-<\textsc{caus}> exist:\textsc{fact} \textsc{qu}-be:\textsc{fact} \\
 \glt `Otherwise, is it the case that I/we have made you angry?' (2005 tAwakWcqraR)
 \end{exe}
 
The verb \japhug{zbraʁ}{attach together} (\ref{ex:kAzbraR} provides a definition of this verb) is possibly an instrumental causative form of \japhug{βraʁ}{attach to}, with irregular placement of the vowel-less allomorph \forme{z-} (§\ref{sec:caus.z}) on a monosyllabic stem and fortition of the \forme{β-} to \ipa{b} to avoid the impossible cluster $\dagger$\forme{zβr-}.

\begin{exe} 
\ex \label{ex:kAzbraR}
\gll ɯ-pʰoŋbu cʰonɤ si, tɤjtsi nɯnɯra tɯ-tɯ-xtɕɤr ku-kɤ-βzu nɯnɯ tɕe, kɤ-zbraʁ tu-kɯ-ti ŋu.  \\
\textsc{3sg}.\textsc{poss}-body \textsc{comit} tree pillar \textsc{dem}:\textsc{pl}  \textsc{simult}-\textsc{nmlz}:\textsc{action}-attach \textsc{ipfv}-\textsc{inf}-make \textsc{dem} \textsc{lnk} \textsc{inf}-attach \textsc{ipfv}-\textsc{genr}-say be:\textsc{fact} \\
\glt `(Whether a person or an animal), attaching his/its body together with a tree or a pillar is called \forme{zbraʁ}.' (150902 kAxtCAr)
\japhdoi{0006308\#S14}
\end{exe}
 
The \forme{sɤ-} allomorph (§\ref{sec:caus.sA}) is found on two bases with contracting \forme{a-}: \japhug{pe}{be good} and \japhug{rmi}{be called}, whose corresponding causative forms are  \japhug{sɤpe}{do well} and \japhug{sɤrmi}{give a name} (§\ref{sec:semi.transitive.causative}), respectively. In the case of \forme{sɤrmi}, this irregularity is a clue that it is in fact a denominal verb derived from the noun \japhug{tɤ-rmi}{name} by the causative denominal (§\ref{sec:sW.caus.history} ), rather than a causative directly derived from the intransitive verb \japhug{rmi}{be called}.

 The causative verb \japhug{sɯɣe}{invite} from \japhug{ɣi}{come} has the expected allomorph \forme{sɯ\trt}, but presents ablaut (\forme{ɣi} \fl{} \forme{ɣe}) with the same vowel alternation as that of stem II (§\ref{sec:stem2.form}). This is a trace of a vowel alternation system which used to be more widespread, and is better preserved in Zbu \citep{gong18these} and some varieties of Situ (\citealt[304, fn 10]{zhangsy18stem}).

  \subsubsection{\forme{a-}/\forme{z-} alternation} \label{sec:sigm.caus.a.z}
Verbs whose stem has three or more syllables, whose initial syllable is \forme{a-} followed by a prefixal syllable with a sonorant initial such as \forme{ɣɯ/ɣɤ-} or \forme{rɯ/rɤ\trt}, have sigmatic causative forms with deletion of the \forme{a-} element rather than regular vowel fusion (§\ref{sec:caus.sA}). For instance, the causative of \japhug{arɤtsʰi}{be cooked like rice gruel} is \japhug{zrɤtsʰi}{cook like rice gruel} instead of expected $\dagger$\forme{sɤrɤtsʰi}.

 Other pairs of the same type include  \japhug{aɣɯrɯru}{that can be done at the same time as} / \japhug{zɣɯrɯru}{do at the same time as},  \japhug{arɤtɕʰa}{be determined from} / \japhug{zrɤtɕʰa}{determine from} (see examples \ref{ex:pe.mApe.arAtCha} and \ref{ex:rtaR.mArtaR.zrAtCha}, §\ref{sec:multiclausal.complements}) and \japhug{aɣɯŋgɯŋgɯ}{have a lot of layers}	/ \japhug{zɣɯŋgɯŋgɯ}{put on a lot of layers}. 
 
 Some of the verbs with this \forme{a/z-} alternation are transparently denominal verbs with the compound prefixes \forme{arɤ-} (§\ref{sec:denom.arA}) and \forme{aɣɯ-} (§\ref{sec:denom.aGW.caus}).

 

\subsection{Lexicalized sigmatic causatives} \label{sec:sig.caus.lexicalized}
Some causative verbs are formally regular, but their semantic relationship with the base verb is not completely predictable.

The causative \forme{z-nɯna} of the intransitive verb \japhug{nɯna}{rest} has the predictable meaning `let rest' as in (\ref{ex:kAznWna.amAtAtWkhW}), but more commonly appears as a com\-ple\-ment-taking verb meaning `stop' as in (\ref{ex:kACar.toznWna}) (see also \ref{ex:Wqhu.CaNpCi}, §\ref{sec:temporal.postpositions}).

\begin{exe} 
\ex \label{ex:kAznWna.amAtAtWkhW}
\gll nɤ-pi ni kɯ ``nɯna-j nɤ nɯna-j" ʑo kɤtɯpa-ndʑi ri, maka ʑo kɤ-z-nɯna a-mɤ-tɤ-tɯ-kʰɯ ma,\\
\textsc{2sg}.\textsc{poss}-elder.sibling \textsc{du} \textsc{erg} rest:\textsc{fact}-\textsc{1pl} \textsc{add}  rest:\textsc{fact}-\textsc{1pl} \textsc{emph} say:\textsc{fact}-\textsc{du} \textsc{lnk} at.all \textsc{emph} \textsc{inf}-\textsc{caus}-\textsc{rest} \textsc{irr}-\textsc{neg}-\textsc{pfv}-2-agree \textsc{lnk} \\
\glt `Your two brothers will repeatedly say `let us rest', but you should never agree to let them rest.' (2003 qachGa)
\japhdoi{0003372\#S134}
\end{exe}

\begin{exe} 
\ex \label{ex:kACar.toznWna}
\gll ɯ-pi kɯ jo-ɣi tɕe, tɯ-ci kɤ-ɕar to-z-nɯna tɕe \\
\textsc{3sg}.\textsc{poss}-elder.sibling \textsc{erg} \textsc{ifr}-come \textsc{lnk} \textsc{indef}.\textsc{poss}-water inf-search \textsc{ifr}-stop \textsc{lnk} \\
\glt `His brother came and stopped looking for water.' (2002 nyimawodzer)
\end{exe}

The form \forme{sɯqaʁ}, which is analyzable as the causative of the transitive verb \japhug{qaʁ}{peel}, has the unexpected meaning `delimit the boundaries of (a place)' (\ref{ex:tAnWsWqaR}). The semantic change is unexplained.

\begin{exe} 
\ex \label{ex:tAnWsWqaR}
\gll  kʰa ɯ-sta tɤ-nɯ-sɯqaʁ-a \\
house \textsc{3sg}.\textsc{poss}-place \textsc{aor}-\textsc{auto}-delimit-\textsc{1sg} \\
\glt `I delimited the site [to build] the house.' (elicited)
\end{exe}

The secundative verb \japhug{nɯsɯkʰo}{rob, extort}, is the lexicalized autive (§\ref{sec:autoben.lexicalized}) of the causative \japhug{sɯkʰo}{cause to give} of the indirective verb \japhug{kʰo}{give} (§\ref{sec:ditransitive.indirective}).\footnote{This derivation dates back to the common ancestor of Northern Gyalrong languages, as shown by the Tshobdun cognate \forme{nsəkʰi} `snatch away' \citep[220]{jackson19tshobdun}. } The original meaning of \forme{nɯsɯkʰo} was presumably `$X$ causes $Y$ to give $Z$ to himself$_i$', its grammatical subject being thus originally both agent (causer) and recipient at the same time, reflecting the inversive function of the causative (§\ref{sec:ditransitive.causative}, §\ref{sec:sig.caus.inversive}).

The transitive subject, object and dative arguments of \forme{kʰo} correspond to the object, semi-object and subject of \forme{nɯsɯkʰo}, respectively. Example (\ref{ex:YWkWnWsWkhoa}) shows this verb with 2\fl{}\textsc{1sg} indexation, with a meaning that can still be interpreted as `you cause me to give it to you'.

\begin{exe}
\ex \label{ex:YWkWnWsWkhoa}
\gll  nɯ mɤkɯftsʰi ɲɯ-kɯ-nɯsɯkʰo-a pɯ\redp{}pɯ-ɕti qʰe, kɯki aʑo sɤlaŋpʰɤn ki pjɯ-qri-a ŋu \\
\textsc{dem} forcing \textsc{ipfv}-2\fl{}1-extort-\textsc{1sg} \textsc{cond}\redp{}\textsc{pst}.\textsc{ipfv}-be.\textsc{aff} \textsc{lnk} \textsc{dem}.\textsc{prox} \textsc{1sg} basin \textsc{dem}.\textsc{prox} \textsc{ipfv}-break[III]-\textsc{1sg} be:\textsc{fact} \\
\glt `If you [try] to take it from me forcibly, I will break this basin.' (150831 jubaopen-zh)
\japhdoi{0006294\#S92}
\end{exe}

The causative derivation from \forme{kʰo} to \forme{nɯsɯkʰo} differs from other inversive causatives in the obligatory presence of the autive \forme{nɯ\trt}, and the additional coercive meaning.

The verb \forme{nɯsɯkʰo} can serve as input for other derivations (§\ref{sec:antipassive.compatibility}). The autive prefix \forme{nɯ-} in this verb, although lexicalized, can be optionally reordered with regard to the antipassive prefix (§\ref{sec:antipassive.compatibility}, §\ref{sec:autoben.lexicalized}).

The verb \forme{sɯftsʰi}, formally the causative of the stative verb \japhug{ftsʰi}{feel better} (or `be good for nothing'), only occurs in negative form (§\ref{sec:obligatory.negative}) and means `force, coerce' as in (\ref{ex:mWtowGsWftshi}).

\begin{exe}
\ex \label{ex:mWtowGsWftshi}
 \gll   tɕeri tɤɕime nɯ kɯ, tɕe ɯ-wa nɯ kɯ mɯ-tó-wɣ-sɯftsʰi qʰe, nɯnɯ ɕɯŋgɯ ɯ-sŋi tʰɯtʰɤci pɯ-kɯ-fse nɯnɯra ɯ-wa ɯ-tɯ-fɕɤt pjɤ-βzu \\
 but girl \textsc{dem} \textsc{erg} \textsc{lnk} \textsc{3sg}.\textsc{poss}-father \textsc{dem} \textsc{erg} \textsc{neg}-\textsc{ifr}-\textsc{inv}-force \textsc{lnk} \textsc{dem} before \textsc{3sg}.\textsc{poss}-day something \textsc{pst}.\textsc{ipfv}-\textsc{sbj}:\textsc{pcp}-be.like \textsc{dem}:\textsc{pl} \textsc{3sg}.\textsc{poss}-father \textsc{3sg}.\textsc{poss}-\textsc{nmlz}:\textsc{action}-tell \textsc{ifr}-make \\
\glt `The girl, pressed by her father, told him everything that had happened in the day before that.' (140429 qingwa wangzi-zh)
\japhdoi{0003890\#S107}
\end{exe}

This meaning presumably derives from `cause to be unable to stand/bear', as the base verb \forme{ftsʰi} in negative form can have the meaning `cannot stand (the pain, discomfort caused by a disease' as in (\ref{ex:mWchAftshi}).

\begin{exe}
\ex \label{ex:mWchAftshi}
 \gll wuma ʑo ɲɯ-mŋɤm tɕendɤre mɯ-cʰɤ-ftsʰi tɕe pjɤ-ɣi tɕe \\
 really \textsc{emph} \textsc{sens}-hurt \textsc{lnk} \textsc{neg}-\textsc{ifr}-feel.better \textsc{lnk} \textsc{ifr}:\textsc{down}-come \textsc{lnk} \\
\glt `It hurt a lot, she could not stand it and came [down to Mbarkham for treatment].' (12-BzaNsa)
\japhdoi{0003484\#S91}
\end{exe}

The related adverb \japhug{mɤkɯftsʰi}{forcibly}, which originates from the negative stative infinitive of \forme{ftsʰi} (§\ref{sec:velar.inf.adverb}), is also used to express coercion (§\ref{sec:sig.caus.coercitive}).

 

\subsection{Morphosyntax} \label{sec:sig.caus.morphosyntax}

\subsubsection{Intransitive bases} \label{sec:sig.caus.intr}
\is{causative!sigmatic} \is{causative!intransitive verb}
The causative derivation increases the valency of the base verb by one argument. In the case of intransitive bases, the resulting verb is monotransitive. The intransitive subject of the base verb  corresponds to the object of the causative verb. For instance, in (\ref{ex:kutasAnbaR}) the causative verb \japhug{sɤnbaʁ}{hide} (vt) (from the intransitive \japhug{anbaʁ}{hide} (vi), \ref{ex:kanbaRa})  takes the portmanteau prefix 1\fl{}2 \forme{ta-} (§\ref{sec:indexation.local}), indexing the person hiding as object, and the causer (the person helping him/her, §\ref{sec:sig.caus.adjutative}) as transitive subject.

\begin{exe}
\ex \label{ex:kanbaRa}
\gll kɤ-anbaʁ-a \\
\textsc{aor}-hide-\textsc{1sg} \\
\glt `I hid.' (elicited)
\end{exe}

\begin{exe}
\ex \label{ex:kutasAnbaR}
\gll a-rɟit ra nɯ-ɣi-nɯ ɕti tɕetʰa, kɤ-ndza kowa tɯ́-wɣ-βzu ɕti tɕe ku-ta-sɯ-ɤnbaʁ ŋu \\
\textsc{1sg}.\textsc{poss}-child \textsc{pl} \textsc{vert}-come:\textsc{fact}-\textsc{pl} be.\textsc{aff}:\textsc{fact} later \textsc{inf}-eat manner 2-\textsc{inv}-make:\textsc{fact} be.\textsc{aff}:\textsc{fact} \textsc{lnk} \textsc{ipfv}-1\fl{}2-\textsc{caus}-hide be:\textsc{fact} \\
\glt `My children are about to come back and will try to eat you, let me hide you.' (2012 Norbzang)
\japhdoi{0003768\#S258}
\end{exe}
 
The causer of causative verbs derived from intransitive bases is morphosyntactically identical to the subject of a monotransitive verb both from the point of view of indexation and case marking, and takes ergative when overt as in (\ref{ex:kosAnbaR}).


\begin{exe}
\ex \label{ex:kosAnbaR}
\gll tɤ-mu nɯ kɯ  tɤ-pɤtso ra kʰri ɯ-pa zɯ ko-sɯ-ɤnbaʁ  \\
\textsc{indef}.\textsc{poss}-mother \textsc{dem} \textsc{erg} \textsc{indef}.\textsc{poss}-child \textsc{pl} bed \textsc{3sg}.\textsc{poss}-under \textsc{loc} \textsc{ifr}-\textsc{caus}-hide \\
\glt `The old woman hid the children under the bed.' (elicited)
\end{exe}

\subsubsection{Transitive bases} \label{sec:sig.caus.tr}
\is{causative!sigmatic} \is{causative!transitive verb}
The causativization of monotransitive verbs\footnote{The causativization of semi-transitive and ditransitive verbs is discussed in §\ref{sec:semi.transitive.causative} and §\ref{sec:ditransitive.causative}, and is not repeated here. } is more complicated. While the causer is always treated like a transitive subject in terms of indexation, ergative marking and relativization (§\ref{sec:ditransitive.causative}), the status of the causee (corresponding to the transitive subject of the base verb) and of the patientive argument (object of the base verbs) are less straightforward.

From the point of view of indexation (§\ref{sec:ditransitive.causative}), both causees (\ref{ex:pjWkWsWfCatandZi}) and patientive arguments (\ref{ex:tAkWsWRndWa}) can be indexed as direct objects (in these examples with 2\fl{}\textsc{1sg} configurations, §\ref{sec:indexation.local}).  
 
\begin{exe}
\ex \label{ex:pjWkWsWfCatandZi}
\gll  χpi pjɯ-fɕat-a, pjɯ-kɯ-sɯ-fɕat-a-ndʑi nɯ ʁo jɤɣ \\
story \textsc{ipfv}-tell-\textsc{1sg} \textsc{ipfv}-2\fl{}1-\textsc{caus}-tell-\textsc{1sg}-\textsc{du} \textsc{dem} \textsc{top}.\textsc{advers} be.acceptable:\textsc{fact} \\
\glt `I can tell a story$_i$, the two of you can have me tell it$_i$, but ...' (140511 1001 yinzi-zh)
\japhdoi{0003963\#S48}
\end{exe}

The form \forme{tɤ-kɯ-sɯ-ʁndɯ-a} is ambiguous, and can either be interpreted as `you caused me to be beaten (by him/someone)' as in (\ref{ex:tAkWsWRndWa}), but also as `you made me beat him' with the \textsc{2sg} as causee rather than patientive argument.
 
\begin{exe}
\ex \label{ex:tAkWsWRndWa}
\gll taʁndo mɯ́j-tɯ-tso tɕe, tɕe li tɤ-kɯ-sɯ-ʁndɯ-a \\
speech \textsc{neg}:\textsc{sens}-2-understand \textsc{lnk} \textsc{lnk} again \textsc{aor}-2\fl{}1-\textsc{caus}-hit-\textsc{1sg} \\
\glt `You are not listening [to what I say], you caused me to be beaten again.' (2003-kWBRa)
\end{exe}

The indexation of these verbs is determined by a person hierarchy: when the causee (or patientive) is first/second person, and the patientive (or causee)  is third person, the causative verb takes first/second object indexation (§\ref{sec:ditransitive.causative}).

From the point of view of case marking, patientive arguments never take the ergative, while causees can take it optionally (§\ref{sec:causee.kW}). In example  (\ref{ex:kW.pjAsWGWtnW}), the first causee \forme{ɕɤɣpɣa} `bird sp.' in left-dislocated position has no ergative marking, while the second one \japhug{qajdo}{crow} takes it (note that the plural indexation on \forme{pjɤ-sɯ-ɣɯt-nɯ} shows that \forme{qajdo} is causee and not causer).
 
\begin{exe}
\ex \label{ex:kW.pjAsWGWtnW}
\gll ɕɤɣpɣa nɯnɯ, smɤn saŋrɟɤz ra kɯ ci pjɤ-sɯ-ɣɯt-nɯ tɕe, nɯ ɕɤɣ ɯ-ku qʰe pjɤ-nɯ-ɕ-lɯɣ, tɕe qajdo kɯ ci pjɤ-sɯ-ɣɯt-nɯ ri, nɯnɯ ɕkrɤz ɯ-ku pjɤ-nɯ-ɕlɯɣ, \\
bird.sp. \textsc{dem} medicine buddha \textsc{pl} \textsc{erg} once \textsc{ifr}:\textsc{down}-\textsc{caus}-bring-\textsc{pl} \textsc{lnk} \textsc{dem} juniper \textsc{3sg}.\textsc{poss}-head \textsc{lnk} \textsc{ifr}:\textsc{down}-\textsc{auto}-\textsc{caus}-get.loose \textsc{lnk} crow \textsc{erg} once \textsc{ifr}:\textsc{down}-\textsc{caus}-bring-\textsc{pl} \textsc{lnk} \textsc{dem} oak \textsc{3sg}.\textsc{poss}-head  \textsc{ifr}:\textsc{down}-\textsc{auto}-\textsc{caus}-get.loose  \\
\glt `The buddhas$_i$ sent the juniper bird$_j$ to bring the medicine$_k$ [to Gesar], but it$_j$  dropped it$_k$ on a juniper tree, they$_i$ sent a crow$_l$, but it$_l$ dropped it$_k$ on an oak.' (2003 gesar)
\end{exe}

\subsubsection{Semi-reflexive} \label{sec:sig.caus.semi.reflexive}
\is{causative!semi-reflexive} \is{semi-reflexive!causative}
A special subtype of semi-reflexive indexation (§\ref{sec:incl.semi.reflexive}) is found in causative forms derived from transitive verbs, when a dual or plural number marker includes both the causer and the causee, as in (\ref{ex:toWGsWndzandZi.pjAra}) where in the 3$'$\fl{}\textsc{3du} form \forme{tó-wɣ-sɯ-ndza-ndʑi}) the dual suffix indexes the addition of the causer (\japhug{tɤ-rʑaβ}{wife}) and the causee (her husband).

\begin{exe}
\ex \label{ex:toWGsWndzandZi.pjAra}
\gll tɤ-rʑaβ nɯ kɯ iɕqʰa, tsʰaŋ ɯ-ŋgɯ la-nɯ-rku kɯ-mɯm nɯra cʰɤ-tɕɤt tɕe tó-wɣ-sɯ-ndza-ndʑi pjɤ-ra. \\
\textsc{indef}.\textsc{poss}-wife \textsc{dem} \textsc{erg} the.aforementioned cupboard \textsc{3sg}.\textsc{poss}-in \textsc{aor}:\textsc{upstream}:3\fl{}3-\textsc{auto}-put.in \textsc{sbj}:\textsc{pcp}-be.tasty \textsc{dem}:\textsc{pl} \textsc{ifr}:\textsc{downstream}-take.out \textsc{lnk} \textsc{ifr}-\textsc{inv}-\textsc{caus}-eat-\textsc{du} \textsc{ifr}.\textsc{ipfv}-be.needed \\
\glt `The wife had no choice but to take out the nice [food] that she had put in the cupboard and give it to [her husband and herself].' (150824 kelaosi-zh)
\japhdoi{0006276\#S73}
\end{exe}


\subsubsection{Negation} \label{sec:sig.caus.negation}
The relative position of the negative and causative prefixes in the prefixal chain is fixed (§\ref{sec:prefixal.chain}) and independent of the semantic scope.\footnote{The same is true of associated motion prefixes (§\ref{sec:sig.caus.AM}). } Thus, a negative causative verb can be either  interpreted as `not cause to $X$' or as `cause not to $X$' (prohibition).   For instance, \forme{mɤ-sɯx-ɕe-nɯ} (\textsc{neg}-\textsc{caus}-go:\textsc{fact}-\textsc{pl}) can either mean `they don't send him/them' or `they prevent/forbid/don't let him/them go' as in (\ref{ex:mAkWspa.mAsWXCenW}).
 
\begin{exe}
\ex \label{ex:mAkWspa.mAsWXCenW}
\gll nɯ ɕɤmɯɣdɯɣ kɤ-lɤt kɯ-mkʰɤz nɯra ra ma mɤ-kɯ-spa nɯra mɤ-sɯx-ɕe-nɯ. \\
\textsc{dem} gun \textsc{inf}-release \textsc{sbj}:\textsc{pcp}-be.expert \textsc{dem}:\textsc{pl} be.needed:\textsc{fact} \textsc{lnk} \textsc{neg}-\textsc{sbj}:\textsc{pcp}-be.able \textsc{dem}:\textsc{pl} \textsc{neg}-\textsc{caus}-go:\textsc{fact}-\textsc{pl} \\
\glt `They need people who are good at shooting with guns, they don't let those who are not able (to shoot) go.' (150829 KAGWcAno)
\japhdoi{0006420\#S13}
\end{exe}
 
%tɯrme ra tɕe nɯ ma a-mɤ-ɕ-tɤ-z-nɯsnɯɲaʁ ra
%140512_fushang_he_yaomo-zh,  171
The negative verb form \forme{mɯ-pa-sɯ-pʰɯt} (\textsc{neg}-pfv:3\flobv{}-\textsc{caus}-take.off) an also either mean `he did not make him/them destroy it' or `he prevented him/them from destroying it', as in (\ref{ex:mWpasWphWt}).

\begin{exe}
\ex \label{ex:mWpasWphWt}
\gll ɣʑo ɯ-kʰo ta-fsraŋ nɯnɯ, [ɯ-pi ni mɯ-pa-sɯ-pʰɯt] ra ɣɯ, ɣʑo ra ɣɯ nɯ-rɟɤlpu nɯ jo-ɣi. \\
bee \textsc{3sg}.\textsc{poss}-house \textsc{aor}:3\flobv{}-protect \textsc{dem} \textsc{3sg}.\textsc{poss}-elder.sibling \textsc{du} \textsc{neg}-pfv:3\flobv{}-\textsc{caus}-take.off \textsc{pl} \textsc{gen} bee \textsc{pl} \textsc{gen} \textsc{3pl}.\textsc{poss}-king \textsc{dem} \textsc{ifr}-come \\
\glt `The king of the bees whose hive he had protected and  prevented his brothers from destroying came.' (140510 fengwang-zh)
\japhdoi{0003939\#S129}
\end{exe}

% \begin{exe}
%\ex \label{ex:mAkWspa.mAsWXCenW}
%\gll mɯ́j-kɯ-sɯ-tsʰi-tɕi \\
%\glt `You are preventing us from drinking it.' (140514 guowang he lieying-zh)
% \end{exe}
% a-tɤ-tɯ-ndze jɤɣ ma ma-tɯ́-wɣ-sɯ-mtsɯr it will prevent you from being hungry

Another possible interpretation of the combination of causative and negative prefixes is `make sure $X$ does not need to $Y$', `take care of $X$ and ensure $Y$ does not need to happen' (where $X$ represents the causee and $Y$ the base verb), as in (\ref{amApWtWsqlWtnW}).

\begin{exe}
\ex \label{amApWtWsqlWtnW}
\gll   a-mu rcanɯ ndʑu tɯ-ldʑa cinɤ  a-mɤ-pɯ-tɯ-sɯ-qlɯt-nɯ ra \\
 mother \textsc{unexp}:\textsc{deg} chopsticks one-stick even  \textsc{irr}-\textsc{neg}-\textsc{pfv}-2-\textsc{caus}-break-\textsc{pl} be.needed:\textsc{fact} \\
 \glt   `Make sure that my mother does not even need to break chopsticks.' (an idiomatic expression meaning ``please take care of my mother's every need'', 2012 Norbzang)
\japhdoi{0003768\#S207}
\end{exe}

The same type of ambiguity is also found with the velar causative (§\ref{sec:velar.causative.negation}).

\subsubsection{Associated motion} \label{sec:sig.caus.AM} 
\is{causative!associated motion} \is{associated motion!causative}
As in the case of the negation (§\ref{sec:sig.caus.negation}), the relative ordering of the associated motion and causative prefixes in the template is fixed and independent of semantic scope, and both  `go/come and cause to $X$' and `cause to go/come and $X$' are possible interpretations.

Thus, either the causer or the causee can be the argument undergoing motion (§\ref{sec:AM.argument.motion}). For instance, the verb \forme{ɕ-ko-z-rɯru} in (\ref{ex:CkozrWru}) can either mean `he sent him to guard it' (causee motion, the correct interpretation in this context) or `he went and made him guard it' (causer motion).

\begin{exe}
\ex \label{ex:CkozrWru}
\gll rɟɤlpu kɯ ɯ-tɕɯ stu kɯ-xtɕi nɯ ɕ-ko-z-rɯru. \\
king \textsc{erg} \textsc{3sg}.\textsc{poss}-son most \textsc{sbj}:\textsc{pcp}-be.small \textsc{dem} \textsc{tral}-\textsc{ifr}-\textsc{caus}-guard \\
\glt `The king sent his youngest son to guard [the trees].' (140507 jinniao-zh)
\japhdoi{0003931\#S37}
\end{exe}

 
 
\subsubsection{Serial verb constructions} \label{sec:sig.caus.serial}
\is{causative!serial verb construction} \is{serial verb construction!causative}
In the serial verb construction with the similative verb \japhug{stu}{do like}, the instrumental causative can be repeated on both verbs, as in (\ref{ex:chWsWste.chWsWBde}).

\begin{exe}
\ex \label{ex:chWsWste.chWsWBde}
\gll  tɕe ɯ-jaʁ nɯ kɯ tʰɤlwa ki cʰɯ-sɯ-ste nɯ ɯ-tʰɤcu cʰɯ-sɯ-βde \\
\textsc{lnk} \textsc{3sg}.\textsc{poss}-hand \textsc{dem} \textsc{erg} earth \textsc{dem}.\textsc{prox} \textsc{ipfv}:\textsc{downstream}-\textsc{caus}-do.like[III] \textsc{dem} \textsc{3sg}.\textsc{poss}-downstream \textsc{ipfv}:\textsc{downstream}-\textsc{caus}-throw \\
\glt `[The mole] does this with its forepaws, and throws the earth below.' (28-qapar)
\japhdoi{0003720\#S148}
\end{exe}

In other serial verb constructions (§\ref{sec:svc.manner}), intransitive stative verbs generally have to be causativized to be used with transitive verbs to ensure transitivity harmony. These causative verbs expressing manner can appear before the lexical verbs as in (\ref{ex:pWzmAke}) (where the causative of \japhug{mɤku}{be first} expresses the meaning `do $X$ first', §\ref{sec:denom.mA}) or follow it (\ref{ex:koxtCAr.kosAsWG}, \ref{ex:toti.tosApe}).

\begin{exe}
\ex \label{ex:pWzmAke}
\gll a-tʂʰa ci pɯ-z-mɤke pɯ-rke \\
\textsc{1sg}.\textsc{poss}-tea once \textsc{imp}-\textsc{caus}-be.first[III] \textsc{imp}-put.in[III] \\
\glt `Serve me some tea first.' (elicited)
\end{exe} 

\begin{exe}
\ex \label{ex:koxtCAr.kosAsWG}
\gll  ɯ-mŋu nɯra koŋla ʑo ko-xtɕɤr ko-sɯ-ɤsɯɣ ʑo. \\
\textsc{3sg}.\textsc{poss}-opening \textsc{dem}:\textsc{pl} completely \textsc{emph} \textsc{ifr}-tie \textsc{ifr}-\textsc{caus}-be.tight \textsc{emph} \\
\glt `He tied the opening [of the bag] very tightly.' (150824 kelaosi-zh)
\japhdoi{0006276\#S279}
\end{exe} 

The irregular causative of \japhug{pe}{be good}, \forme{sɤ-pe} (§\ref{sec:sig.caus.irregular.other}), is common in the manner SVC (and in the corresponding complement clause construction, see \ref{ex:alAn.kAkho.WtAtWsApet} in §\ref{sec:sig.caus.complement}) to express the meaning `do $X$ well' (\ref{ex:toti.tosApe}).

\begin{exe}
\ex \label{ex:toti.tosApe}
\gll <luban> kɯ rcanɯ kɯ-pɯ-pe ʑo, maka mɤ-kɯ-ɤntɕʰoʁjɤr ʑo to-ti to-sɤ-pe.  \\
  \textsc{anthr} \textsc{erg} \textsc{unexp}:\textsc{deg} \textsc{inf}:\textsc{stat}-\textsc{emph}\redp{}be.good \textsc{emph} at.all \textsc{neg}-\textsc{inf}:\textsc{stat}-be.incomplete \textsc{emph} \textsc{ifr}-say \textsc{ifr}-\textsc{caus}-be.good \\
\glt `Luban said [the answers] very well, without missing anything.' (150902 luban-zh)
\japhdoi{0006268\#S56}
\end{exe} 

%kɯki tɯmbri ki tɤ-tɯ-ari qhe, koŋla ʑo kɤ-ndɤm kɤ-sɤsɯɣ qʰe tɕe, 

%maka ʑo kɤ-sɤfɕɤra, kɤ-sɤpe mɯ-pjɤ-cha-ndʑi,
%140514_beifeng, 7

\subsubsection{Complement clauses with sigmatic causative of manner} \label{sec:sig.caus.complement}
\is{causative!complement clause} \is{complement clause!causative}
Causative forms from adjectival stative verbs occur as com\-ple\-ment-taking verbs to express the manner in which the action takes place (§\ref{sec:causative.manner.complement}). This causative complement construction competes with the manner serial verb construction (§\ref{sec:sig.caus.serial}, §\ref{sec:svc.manner}) and the use of infinitive converbs (§\ref{sec:inf.converb}, §\ref{sec:manner.converbs}). The same construction is found with velar causatives (§\ref{sec:velar.caus.complement}).

For instance in (\ref{ex:alAn.kAkho.WtAtWsApet}) the meaning `give a good answer' is expressed with the causative \forme{sɤ-pe} of \japhug{pe}{be good} (§\ref{sec:sig.caus.irregular.other}) and an infinitival complement clause with the verb \forme{kɤ-kʰo} `to give'.

\begin{exe}
\ex \label{ex:alAn.kAkho.WtAtWsApet}
\gll   [a-lɤn kɤ-kʰo] ɯ-tɤ-tɯ-sɤ-pe-t nɤ tɕe ku-ta-wum ŋu \\
\textsc{1sg}.\textsc{poss}-answer \textsc{inf}-give \textsc{qu}-\textsc{aor}-2-\textsc{caus}-be.good-\textsc{pst}:\textsc{tr} \textsc{add} \textsc{lnk} \textsc{ipfv}-1\fl{}2-gather be:\textsc{fact} \\
\glt `If you give a good answer [to all my questions], I will take you [as my disciple].' (150902 luban-zh)
\japhdoi{0006268\#S44}
\end{exe}

With the exception of a few verbs like \japhug{sɤpe}{do well}, most com\-ple\-ment-taking causative verbs select the lexicalized orientation of the verb in the complement clause (§\ref{sec:orientation.raising}). For instance, in (\ref{ex:kAGndZWr.chAsAmYAm}), the causative verb \forme{cʰɤ-sɯ-ɤmɲɤm} has the \textsc{downstream} orientation preverb \forme{cʰɤ-} (§\ref{sec:kamnyu.preverbs}) selected by \japhug{ɣndʑɯr}{grind} in the infinitival clause.

\begin{exe}
\ex \label{ex:kAGndZWr.chAsAmYAm}
\gll kɤ-ɣndʑɯr cʰɤ-sɯ-ɤmɲɤm \\
\textsc{inf}-grind \textsc{ifr}:\textsc{downstream}-\textsc{caus}-be.homogeneous \\
\glt `He ground [the flour] very homogeneously.' (elicited)
\end{exe}

The causative form is also required when the infinitival complement clause contains a transitive verb with dummy subject (§\ref{sec:transitive.dummy}), as in (\ref{ex:kAlAt.YWzmAke}).

\begin{exe}
\ex \label{ex:kAlAt.YWzmAke}
\gll  paχɕi nɯ ɯ-jwaʁ kɤ-lɤt ɲɯ-z-mɤke, nɯ ɯ-qʰu tɕe ɲɯ-rɯmɯntoʁ, \\
apple \textsc{dem} \textsc{3sg}.\textsc{poss}-leaf \textsc{inf}-release \textsc{ipfv}-\textsc{caus}-be.first[III] \textsc{dem} \textsc{3sg}.\textsc{poss}-after \textsc{lnk} \textsc{ipfv}-bloom \\
\glt `The apple [tree] first grows leaves, and after that blooms.' (07-paXCi)
\japhdoi{0003430\#S11}
\end{exe}



\subsubsection{Auxiliary verbs} \label{sec:sig.caus.modal}
A few auxiliary verbs can take the sigmatic causative, and are used with complement clauses in a way that is different from those treated in §\ref{sec:sig.caus.complement}. 

The causative \forme{sɯxcʰa} from the modal verb \japhug{cʰa}{can} often occurs in inverse form with a dummy subject (§\ref{sec:transitive.dummy}) with the specific meaning of  `(cause to) be physically able to $X$', `(cause to) be strong enough to $X$', as in example (\ref{ex:mWjtWwGsWxcha}) (see also §\ref{sec:sWxcha}). The agent is not expressed, but implicitly refers to the heavy object in (\ref{ex:mWjtWwGsWxcha}).  

\begin{exe}
\ex \label{ex:mWjtWwGsWxcha}
\gll ɲɯ-rʑi tɕe [kɤ-fkur] mɯ́j-tɯ-wɣ-sɯx-cʰa. \\
\textsc{sens}-be.heavy \textsc{lnk}  \textsc{inf}-carry.on.the.back \textsc{neg}:\textsc{sens}-\textsc{2}-\textsc{inv}-\textsc{caus}-can \\
\glt `It is heavy, you won't be able to carry it on your back.' (elicitation)
\end{exe}	

 This com\-ple\-ment-taking verb is exceptional in being the only one in Japhug requiring coreference between object of the matrix clause and subject of the complement clause (§\ref{sec:sWxcha}).

This verb is attested in direct forms, as in (\ref{ex:mWYWsWxche}), but only in the meaning `cause to be able to bear' without infinitival complement.\footnote{The base verb \japhug{cʰa}{can}, among other functions, is attested with the meaning `be fine, be all right' (Chinese \ch{行}{xíng}{be all right}), in which case it does not take any complement clause or overt semi-object. }

\begin{exe}
\ex \label{ex:mWYWsWxche}
\gll kumpɣa pʰu nɯ ɲɯ-βʁa tɕe, mu nɯra mɯ-ɲɯ-sɯx-cʰe\\
fowl male \textsc{dem} \textsc{sens}-win \textsc{lnk} female \textsc{dem}:\textsc{pl} \textsc{neg}-\textsc{sens}-\textsc{caus}-can[III] \\
\glt `(Otherwise) the roosters are too strong, and the hens cannot bear it.' (150819 kumpGa)
\japhdoi{0006388\#S8}
\end{exe} 

The intransitive phasal \japhug{jɤɣ}{finish} (§\ref{sec:aspectual.complement}, unrelated to the modal verb \japhug{jɤɣ}{be allowed}, §\ref{sec:ra.khW.jAG.verb}) has the causative form \japhug{sɯɣjɤɣ}{finish}, used as a synonym of the transitive verb \japhug{stʰɯt}{finish} (§\ref{sec:aspectual.complement}).

Some modal verbs also appear in the same construction with a velar, instead of a sigmatic causative (§\ref{sec:velar.caus.modal}).

\subsubsection{Collocations} \label{sec:sig.caus.collocations}
\is{causative!collocation} \is{collocation!causative}
Verbs in noun-verb collocations can undergo sigmatic causative derivation. For instance, the combination of \japhug{tɯ-χpɯm}{knee} and \japhug{tsʰoʁ}{attach} (§\ref{sec:tshoR.lv}), which means `kneel' (\ref{ex:WXpWm.pjAtshoR}), can be causativized to \forme{tɯ-χpɯm+sɯ-tsʰoʁ} `cause/make/force to kneel' (\ref{ex:pjWsWtshoR.tusWndze}).

\begin{exe}
\ex \label{ex:WXpWm.pjAtshoR}
\gll ɯ-χpɯm pjɤ-tsʰoʁ \\
\textsc{3sg}.\textsc{poss}-knee \textsc{ifr}-attach \\
\glt `He knelt down.' \japhdoi{0003939\#S88}
\end{exe}

\begin{exe}
\ex \label{ex:pjWsWtshoR.tusWndze}
\gll ɯ-rkɯ nɯtɕu nɯ-χpɯm pjɯ-sɯ-tsʰoʁ, tɕe ɯ-ndzɤtsʰi ɯ-ro nɯ-kɯ-ri nɯnɯra, nɯ-ʁɤri pjɯ-χtɤr tɕe, tu-sɯ-ndze pjɤ-ŋu. \\
\textsc{3sg}.\textsc{poss}-side \textsc{dem}:\textsc{loc} \textsc{3pl}.\textsc{poss}-knee \textsc{ipfv}-\textsc{caus}-attach \textsc{lnk} \textsc{3sg}.\textsc{poss}-food \textsc{3sg}.\textsc{poss}-excess \textsc{aor}-\textsc{sbj}:\textsc{pcp}-remain \textsc{dem}:\textsc{pl} \textsc{3pl}.\textsc{poss}-front \textsc{ipfv}:\textsc{down}-scatter \textsc{lnk} \textsc{ipfv}-\textsc{caus}-eat[III] \textsc{ifr}.\textsc{ipfv}-be \\
\glt `[The evil prince] forced them to kneel at his side, and would throw them leftovers from his food and force them to eat them.' (150821 edu de wangzi-zh)
\japhdoi{0006402\#S58}
\end{exe}

The possessive prefix on the inalienable noun \japhug{tɯ-χpɯm}{knee}, which is obligatorily coreferential with the transitive subject in the base construction (\ref{ex:WXpWm.pjAtshoR}), must be coreferential with the causee in the causativized construction (\ref{ex:pjWsWtshoR.tusWndze}).

Causativization of complex predicates is also possible with the velar causative prefix (§\ref{sec:velar.causative.collocation}).

\subsection{Semantics of the causative} \label{sec:sig.caus.function}
The basic meaning of the sigmatic causative is a factitive `make $X$' or `have someone/something $X$'. 

Depending on the volition of the causer and causee and the nature of the action, several specific cases can be distinguished: missive `sent to' (§\ref{sec:sig.caus.missive}), coercive `force to' (§\ref{sec:sig.caus.coercitive}), adjutative `help $X$ing' (§\ref{sec:sig.caus.adjutative}), rogative `ask $X$ing' (§\ref{sec:sig.caus.rogative}), permissive `let, allow' (§\ref{sec:sig.caus.permissive}) and indirect causation (§\ref{sec:sig.caus.indirect}). This classification is intended as a convenient way of exploring the uses of the causative in Japhug, and makes no claim to reveal the instanciations of universal categories.

In addition, the sigmatic causative has extended and more grammaticalized functions, such as inversive (§\ref{sec:sig.caus.inversive}), instrumental (§\ref{sec:sig.caus.instrumental}) and also tropative (§\ref{sec:sig.caus.tropative}), as well as purely syntactic functions in serial verb constructions (§\ref{sec:sig.caus.serial} above) and complementation (§\ref{sec:sig.caus.complement}).

\subsubsection{Missive} \label{sec:sig.caus.missive}
\is{causative!missive} 
With motion (§\ref{sec:motion.verbs}) and manipulation verbs (§\ref{sec:manipulation.verbs}), the causative has a missive meaning. For instance, the causative \forme{sɯ-ɣɯt} from \japhug{ɣɯt}{bring} means `send' in the sense of `have someone bring $X$' (either by directly asking a person or by mail), as in (\ref{ex:jWfCWndzxi2}) (see also \ref{ex:kW.pjAsWGWtnW} above in §\ref{sec:sig.caus.tr}).

\begin{exe}
\ex \label{ex:jWfCWndzxi2}
\gll a-ŋga lɤ-tɯ-sɯ-ɣɯt-ndʑi nɯ, a-xtsa nɯ wuma ɲɯ-pe \\ 
\textsc{1sg}.\textsc{poss}-clothes \textsc{aor}-2-\textsc{caus}-bring-\textsc{du} \textsc{dem} \textsc{1sg}.\textsc{poss}-shoe \textsc{dem} really \textsc{sens}-be.good  \\
\glt `The clothes that you have sent me, the shoes are very nice.' (conversation, 15.04.18)
\end{exe}

The missive meaning of the causative is also found with non-motional bases taking an associated motion prefix, as in (\ref{ex:CkozrWru}) above (see also §\ref{sec:sig.caus.AM}).

\subsubsection{Coercive} \label{sec:sig.caus.coercitive}
\is{causative!coercive} 
The sigmatic causative also occurs to express coercive causation `force to' against the will of the causee, as in (\ref{ex:kWjNu.pasWtandZi}) (see also \ref{ex:pjWsWtshoR.tusWndze} in §\ref{sec:sig.caus.collocations}).

\begin{exe}
\ex \label{ex:kWjNu.pasWtandZi}
\gll tɕʰeme nɯ kɯjŋu kɯ-wxtɯ\redp{}wxti ʑo na-sɯ-ta-ndʑi ɲɯ-ŋu \\
girl \textsc{dem} oath \textsc{sbj}:\textsc{pcp}-\textsc{emph}\redp{}be.big \textsc{emph} \textsc{aor}:3\flobv{}-\textsc{caus}-put-\textsc{du} \textsc{sens}-be \\
\glt `They$_{DU}$ forced the girl to swear a big oath.' (2003 qachGa)
\japhdoi{0003372\#S142}
\end{exe}

This meaning occurs in particular with the adverb \japhug{mɤkɯftsʰi}{forcibly} (\ref{ex:mAkWftshi.tAsWndzata}) (see also \ref{ex:mAkWfsthi.chAsWrku}, §\ref{sec:velar.inf.adverb}), and the related lexicalized causative verb \japhug{sɯftsʰi}{force} (see \ref{ex:mWtowGsWftshi}, §\ref{sec:sig.caus.lexicalized}).

\begin{exe}
\ex \label{ex:mAkWftshi.tAsWndzata}
\gll  tɤ-pɤtso smɤn mɤkɯftsʰi tɤ-sɯ-ndza-t-a \\
\textsc{indef}.\textsc{poss}-child medicine forcibly \textsc{aor}-\textsc{caus}-eat-\textsc{pst}:\textsc{tr}-\textsc{1sg} \\
\glt `I forced the child to eat the medicine.' (elicited)
\end{exe}

In this function, the causee generally does not take the ergative, as shown by the examples above.
%βlama ci pjɤ-tu tɕe, 
%nɯ kɯ tɯdi to-lɤt, 
%tɯdi to-lɤt tɕe tɕe qapri nɯ nɯtɕu ko-z-rɤʑi.
%140522_Kamnyu_zgo


\subsubsection{Adjutative} \label{sec:sig.caus.adjutative}
\is{causative!adjutative} 
The adjutative meaning of the causative `help $X$ing' occurs with verbs whose action is in the interests of the causee, and can only be successfully performed by collaboration between causer and causee. A typical example is the causative \japhug{sɤnbaʁ}{hide}, `help to hide' from \japhug{anbaʁ}{hide} (vi) (see \ref{ex:kutasAnbaR} and \ref{ex:kosAnbaR} above, §\ref{sec:sig.caus.morphosyntax}).

Some verbs with irregular allomorphs have the adjutative function, in  particular \japhug{ɕpʰɣo}{flee with} (from the intransitive \japhug{pʰɣo}{flee}, §\ref{sec:caus.C}) and  \japhug{ʑŋga}{help wearing} (from \japhug{ŋga}{wear}, §\ref{sec:caus.Z}).

\subsubsection{Rogative} \label{sec:sig.caus.rogative}
\is{causative!rogative} 
Rogative causation occurs when the causer asks a causee for help in performing an action, for instance for medical treatment as in (\ref{ex:aZWG.tAznWsmana}). 

\begin{exe}
\ex \label{ex:aZWG.tAznWsmana}
\gll aʑɯɣ kɯnɤ tɤ-z-nɯsman-a qʰe tɤ-mna ɕti \\
\textsc{1sg}:\textsc{gen} also \textsc{aor}-\textsc{caus}-treat-\textsc{1sg} \textsc{lnk} \textsc{aor}-be.better be.\textsc{aff}:\textsc{fact} \\
\glt `I had my own [bellyache] treated and it got better.' (2011-13-qala)
\end{exe}

The rogative \forme{sɤ-} prefix derives from this use of the causative (§\ref{sec:rogative.derivation}). 

\subsubsection{Permissive} \label{sec:sig.caus.permissive} 
\is{causative!permissive} 
The sigmatic causative occurs with a permissive interpretation `let/allow $X$', when the causer does not prevent the causee from performing an action. This meaning of the causative is common for instance in 2\fl{}1 verb forms such as (\ref{ex:kukWzrAZianW.YWntshi}) and (\ref{ex:tukWsWrtoRi.ra}). 

\begin{exe}
\ex \label{ex:kukWzrAZianW.YWntshi}
\gll ku-kɯ-z-rɤʑi-a-nɯ ɲɯ-ntsʰi \\
\textsc{ipfv}-2\fl{}1-\textsc{caus}-stay-\textsc{1sg}-\textsc{pl} \textsc{sens}-be.better \\
\glt `Please let me stay [here].' (qajdoskAt 2002) \japhdoi{0003366\#S67}
\end{exe}
 
\begin{exe}
\ex \label{ex:tukWsWrtoRi.ra}
\gll a-mu nɤ-ndzɣi ɯ-rcʰɤβ tu-kɯ-sɯ-rtoʁ-i ra \\
\textsc{2sg}.\textsc{poss}-mother \textsc{2sg}.\textsc{poss}-fang \textsc{3sg}.\textsc{poss}-interstice \textsc{ipfv}-2\fl{}1-\textsc{caus}-look-\textsc{1sg} be.needed:\textsc{fact} \\
\glt `Mother, let us look at [the thing] that is in the interstice between your fangs?' (2012 Norbzang)
\japhdoi{0003768\#S265}
\end{exe}

Additional examples of permissive causatives include for instance  \forme{z-nɯna} `allow to rest' (§\ref{ex:kAznWna.amAtAtWkhW}, §\ref{sec:sig.caus.lexicalized}) and \forme{sɯ-ndza} `let eat, give to eat' (\ref{ex:toWGsWndzandZi.pjAra}, §\ref{sec:sig.caus.morphosyntax}).
 
\subsubsection{Indirect causation} \label{sec:sig.caus.indirect}
\is{causative!indirect} 
The sigmatic causative can be used to express the unintended result of an (erroneous) action on the part of the causer,  as in (\ref{ex:kWsWRndWa2}) (see also \ref{ex:tAkWsWRndWa}, §\ref{sec:sig.caus.morphosyntax}).

\begin{exe}
\ex \label{ex:kWsWRndWa2}
\gll  nɤʑo tɤ-ndze ma alo ma-lɤ-tɯ-tsɯm ma tʰa li kɯ-sɯ-ʁndɯ-a \\
\textsc{2sg} \textsc{imp}-eat[III] \textsc{lnk} upstream \textsc{neg}-\textsc{imp}:\textsc{upstream}-2-take.away \textsc{lnk} later again 2\fl{}1-\textsc{caus}-hit-\textsc{1sg} \\
\glt  `Eat it, don't take it up there, you would cause me to be beaten again.' (2003-kWBRa)
\end{exe}

It can even be used to express a detrimental result for the causer due to his negligence and failure to take proper preventive measures (\ref{ex:tuzmWrki}).
\largerpage
\begin{exe}
	\ex \label{ex:tuzmWrki}
	\gll kɯm mɯ-cʰɯ-pe qʰe ... kɯm kɯ-ɲɟɯ ɲɯ-βde qʰe laχtɕʰa ra tu-z-mɯrki \\
	door \textsc{neg}-\textsc{ipfv}-close \textsc{lnk} {  } door \textsc{sbj}:\textsc{pcp}-\textsc{acaus}:open \textsc{ipfv}-leave \textsc{lnk} thing \textsc{pl} \textsc{ipfv}-\textsc{caus}-steal[III] \\
	\glt  `(Due to being a heavy drinker), he would [forget] to close the door, leave it open, and [as a result] have his things stolen [by other people].' (17-lhazgron)
\end{exe}

It also occurs when the causer only suggests that the causee undertake an action of his own volition, with unexpected consequences for the causee. In (\ref{amphWz.nWtWsWphWt}), the causative on \forme{nɯ-tɯ-sɯ-pʰɯt} `you had it taken off' refers to the fact that the causer (a trickster rabbit from a traditional story) cheated the referent in \textsc{1sg} (a fierce bear) by suggesting that he treat his belly ache in such a way that he would wound his own bottom. Note that the absence of 2\fl{}1 indexation on the verb \forme{nɯ-tɯ-sɯ-pʰɯt} means that the bear does not regard himself as causee, and expresses it as if the wound resulted from an event independent of him.

\begin{exe}
\ex \label{amphWz.nWtWsWphWt} 
\gll pjɤ-kɯ-nɯβlu-a tɕe mɤ-jɤɣ ma tɕe a-xtu kɯ-mna sɤznɤ a-mpʰɯz ɯ-ntɕʰɯr nɯ-tɯ-sɯ-pʰɯt tɕe \\
\textsc{ifr}-2\fl{}1-cheat-\textsc{1sg} \textsc{lnk} \textsc{neg}-be.allowed:\textsc{fact} \textsc{lnk} \textsc{lnk} \textsc{1sg}.\textsc{poss}-belly \textsc{inf}:\textsc{stat}-be.better \textsc{comp} \textsc{1sg}.\textsc{poss}-buttock \textsc{3sg}.\textsc{poss}-piece \textsc{aor}-2-\textsc{caus}-take.off \textsc{lnk} \\
\glt `You cheated me, it is outrageous, not only did my belly not get better, but [by misleading me] you caused a piece of my buttocks to be ripped off.' (2011-13-qala)
\end{exe}

The causative can also express an event resulting not from an action, but from the absence of action on the causer's part, with unintended consequences for the patientive argument. For instance, in (\ref{ex:kWsWsata.YWNu}), the \textsc{2sg} causer (a bird) is about to (unwillingly) cause the \textsc{1sg} (patientive argument) to be killed by not appearing to the king. 

\begin{exe}
\ex \label{ex:kWsWsata.YWNu}
\gll  nɯ maʁ nɤ, mɯ́j-tɯ-ɣi tɕe kɯ-sɯ-sat-a ɲɯ-ŋu \\
\textsc{dem} not.be:\textsc{fact} \textsc{add} \textsc{neg}:\textsc{sens}-2-come \textsc{lnk} 2\fl{}1-\textsc{caus}-kill:\textsc{fact}-\textsc{1sg} \textsc{sens}-be \\
\glt `Otherwise, by not coming, you are about to get me killed.' (2005 Kunbzang)
\end{exe}

In (\ref{ex:YAzGAdita}), the causative expresses a non-expected result due to the negligence of the causer.

\begin{exe}
\ex \label{ex:YAzGAdita}
\gll tɤ-mtʰɯm ɲɤ-z-ɣɤdi-t-a   \\
\textsc{indef}.\textsc{poss}-meat \textsc{ifr}-\textsc{caus}-have.a.stench-\textsc{pst}:\textsc{tr}-\textsc{1sg} \\
 \glt  `I let the meat spoil.' (elicited)
\end{exe}

\subsubsection{Inversive} \label{sec:sig.caus.inversive}
\is{causative!inversive} 
The inversive function of the causative occurs with verbs expressing a transfer of property, reversing the recipient/source relation of the base verb. Two possible cases exist, depending on how source and recipient are encoded by the base verb's argument structure.

First, most inversive causatives are recipient-source inversive: the argument with oblique case of the base verb (the source of the transfer) corresponds to the causer of the causative verb, and the subject of the base verb (the recipient of the transfer) to the causee of the causative verb  (indexed as direct object) (see §\ref{sec:ditransitive.causative} for a more detailed discussion). Examples of goal inversives include the irregular \japhug{ɕɯrŋo}{lend} (§\ref{sec:caus.CW}) from \japhug{rŋo}{borrow} and \japhug{znɤŋgɯ}{lend} from \japhug{nɤŋgɯ}{borrow} (see §\ref{sec:ditransitive.indirective} on the semantic difference between \forme{nɤŋgɯ} and \japhug{rŋo}{borrow}), 

Second, the highly lexicalized \japhug{nɯsɯkʰo}{rob, extort} from \japhug{kʰo}{give} is a source-recipient inversive: the dative argument of the base verb (recipient) corresponds to the transitive subject (causer/recipient) of the causative verb, and the subject of the base verb to the direct object (source) of the causative verb (§\ref{sec:sig.caus.lexicalized}). For other secundative verbs, the rogative \forme{sɤ-} derivation is used instead to express source-recipient inversion (§\ref{sec:rogative.derivation}).

Although inversive causative mostly occurs with ditransitive bases, the causative of the monotransitive verb \japhug{χtɯ}{buy} can also be interpreted as a recipient-source inversive. Although \japhug{χtɯ}{buy} lacks a dative argument, the source can be optionally specified with the dative as in (\ref{ex:nWCki.kAXtW}) in the case of people, or as a locative adjunct, as in (\ref{ex:shangdian.kAXtW}).
 
 \begin{exe}
\ex \label{ex:nWCki.kAXtW}
\gll  <wugui> ɯ-kɯ-ntsɣe ɣɤʑu tɕe, nɯnɯra nɯ-ɕki li ɯ-ndza ra kɤ-χtɯ ɣɤʑu. \\
turtle \textsc{3sg}.\textsc{poss}-\textsc{sbj}:\textsc{pcp}-sell exist:\textsc{sens} \textsc{lnk} \textsc{dem}.\textsc{pl} \textsc{3pl}.\textsc{poss}-\textsc{dat} again \textsc{3sg}.\textsc{poss}-food \textsc{pl} \textsc{obj}:\textsc{pcp}-buy exist:\textsc{sens} \\
\glt `There are people$_i$ who sell turtles$_j$, and one [can] buy their$_j$ food from them$_i$.' (140510 wugui)
\japhdoi{0003951\#S33}
\end{exe}

\begin{exe}
\ex \label{ex:shangdian.kAXtW}
\gll <shangdian> ɯ-ŋgɯ kɤ-χtɯ ɣɤʑu \\
shop \textsc{3sg}.\textsc{poss}-inside \textsc{obj}:\textsc{pcp}-buy exist:\textsc{sens} \\
\glt `It [can] be bought in a shop.' (28-CAmWGdW)
\japhdoi{0003712\#S101}
\end{exe}

The causative form \forme{sɯ-χtɯ} has several meanings, including the factitive `cause to buy' and instrumental  `buy with' (see \ref{ex:WCki.CtosWXtW} below), but can also mean `sell'. This meaning is found in (\ref{ex:tAwGsWXtWa}) with 3\fl{}\textsc{1sg} indexation: the transitive subject of \forme{sɯχtɯ} in this construction is both causer and source, while its direct object is the recipient.

\begin{exe}
\ex \label{ex:tAwGsWXtWa}
\gll  pɤnmawombɤr kɯ [...] tɤ́-wɣ-sɯ-χtɯ-a ŋu \\
\textsc{anthr} \textsc{erg} { } \textsc{aor}-\textsc{inv}-\textsc{caus}-sell-\textsc{1sg} be:\textsc{fact} \\
\glt `It is Padma 'Od'bar who sold it to me.' (2012 Norbzang)
\japhdoi{0003768\#S133}
\end{exe} 

Case marking is completely different when \forme{sɯ-χtɯ} has the factitive function `cause to buy', as in (\ref{ex:WCki.CtosWXtW}): the source of the transaction (\forme{tɤ-mthɯm ɯ-kɯ-ntsɣe} `the meat seller') is marked with the dative as with the base verb above in (\ref{ex:nWCki.kAXtW}).\footnote{One should note however that the verb in (\ref{ex:WCki.CtosWXtW}) has the translocative prefix, and that the dative argument could also in principle be analyzed as a goal (§\ref{sec:AM.goal}), though in this case the locative \forme{ri} would rather be expected -- compare with examples \ref{ex:mbroXpa.ri.schWXtWnW}  and \ref{ex:kWre.ri.GWchWsWXtWnW} in §\ref{sec:round.trip.AM}.} The causative  \forme{sɯχtɯ} therefore has two different argument structures depending on whether it reflects the inversive (`sell', \ref{ex:tAwGsWXtWa}) or factitive/missive (with associated motion `send to buy' as in \ref{ex:WCki.CtosWXtW}, §\ref{sec:sig.caus.missive}) functions of this derivation.

\begin{exe}
\ex \label{ex:WCki.CtosWXtW}
\gll tɤ-mtʰɯm ɯ-kɯ-ntsɣe ɯ-ɕki tɕe tɤ-mtʰɯm ɕ-to-sɯ-χtɯ qʰe, \\
\textsc{indef}.\textsc{poss}-meat \textsc{3sg}.\textsc{poss}-\textsc{sbj}:\textsc{pcp}-sell \textsc{3sg}.\textsc{poss}-\textsc{dat} \textsc{loc} \textsc{indef}.\textsc{poss}-meat \textsc{tral}-\textsc{ifr}-\textsc{caus}-buy \textsc{lnk} \\
\glt `She sent him to buy meat at the butcher's.' (160701 poucet2)
\japhdoi{0006155\#S15}
\end{exe}

\subsubsection{Instrumental} \label{sec:sig.caus.instrumental}
\is{causative!instrumental} \is{instrumental!causative}
The causative prefix occurs on transitive verbs to mark an instrument, syntactically treated as a causee as in (\ref{ex:fenbi.kW.tosWrAtndZi}), where the Chinese loanword \ch{粉笔}{fěnbǐ}{chalk} occurs with ergative marking (§\ref{sec:instr.kW}). 

\begin{exe}
\ex \label{ex:fenbi.kW.tosWrAtndZi}
\gll tɕaχpa ʁnɯz nɯ kɯ, nɤki, <alibaba> ɣɯ ɯ-kɯm nɯtɕu <fenbi> kɯ, nɤkinɯ, kɯ-ɤrtɯm ci to-sɯ-rɤt-ndʑi \\
thief two \textsc{dem} \textsc{erg} \textsc{filler}  \textsc{anthr} \textsc{gen} \textsc{3sg}.\textsc{poss}-door \textsc{dem}:\textsc{loc} chalk \textsc{erg} \textsc{filler} \textsc{sbj}:\textsc{pcp}-be.round \textsc{indef} \textsc{ifr}-\textsc{caus}-write-\textsc{du} \\
\glt `The two thieves drew a circle on Alibaba's door.' (140512 alibaba-zh)
\japhdoi{0003965\#S121}
\end{exe}

However, the causative prefix is not required with overt instruments, as in (\ref{ex:WjaR.kW.jumWrRWz}).

\begin{exe}
\ex \label{ex:WjaR.kW.jumWrRWz}
\gll ɯ-jaʁ kɯ ju-mɯrʁɯz ɲɯ-cʰa, \\
\textsc{3sg}.\textsc{poss}-hand \textsc{erg} \textsc{ipfv}-scratch \textsc{sens}-can \\
\glt `[The mole] can scratch it with its claws.' (28-qapar)
\japhdoi{0003720\#S152}
\end{exe}

 \is{instrumental}
The co-existence of an overt instrument in the ergative with an overt transitive subject in the same clause (without any pause) is clumsy. Example (\ref{ex:sYWGjW.kW.pasWrAt}) for instance is considered infelicitous if borderline ungrammatical, and the alternative construction with two clauses (\ref{ex:sYWGjW.tando}) is highly preferred.

\begin{exe}
\ex 
\begin{xlist}
\ex \label{ex:sYWGjW.kW.pasWrAt}
\gll ??ɯʑo kɯ sɲɯɣjɯ kɯ tɤscoz pa-sɯ-rɤt \\
\textsc{3sg} \textsc{erg} pen \textsc{erg} letter \textsc{aor}:3\flobv{}-\textsc{caus}-write \\
\glt (intended meaning:) `S/he wrote the letter with the pen.' (elicited)
\ex \label{ex:sYWGjW.tando}
\gll ɯʑo kɯ sɲɯɣjɯ ta-ndo tɕe tɤscoz pa-sɯ-rɤt\\
\textsc{3sg} \textsc{erg} pen \textsc{aor}:3\flobv{}:\textsc{up}-take \textsc{lnk} letter \textsc{aor}:3\flobv{}-\textsc{caus}-write\\
\glt `S/he took the pen and wrote the letter with it.' (elicited)
\end{xlist}
\end{exe}

With intransitive bases (such as \japhug{βzi}{get drunk} in \ref{ex:lowGsWBzi}), the causative expresses a cause (§\ref{sec:manner.nominal.kW}) rather than an instrument.

\begin{exe}
\ex \label{ex:lowGsWBzi}
\gll cʰa kɯ ló-wɣ-sɯ-βzi \\ 
alcohol \textsc{erg} \textsc{ifr}-\textsc{inv}-\textsc{caus}-become.drunk \\
\glt `He became drunk from the alcohol.' (elicited)
\end{exe}

The irregular causative \japhug{ɕɯfkaβ}{cover with} (from the transitive \japhug{fkaβ}{cover}, §\ref{sec:caus.CW}) has an instrumental function, and the instrument is marked with the ergative as in (\ref{ex:pWCWfkaB}).

\begin{exe}
\ex \label{ex:pWCWfkaB}
\gll kʰɤlɤβ kɯ tɯtʰɯ pɯ-ɕɯ-fkaβ-a \\
cover \textsc{erg} pan \textsc{aor}-\textsc{caus}-cover-\textsc{1sg} \\
\glt `I covered the pan with a cover.' (elicited)
\end{exe}

Another irregular causative,  \japhug{ʑmbri}{play} (from \japhug{mbri}{cry, sing}), can also be interpreted as an instrumental causative, but this verb treats the instrument as the object (§\ref{sec:caus.Z}) without ergative.

%tɕe tɕe nɯnɯ stoʁ nɯ kɯ mbrɤz tu-sɤndu-j tɕe tɕe 
\subsubsection{Tropative} \label{sec:sig.caus.tropative}
\is{causative!tropative} \is{tropative!causative}
A handful of verbs with a sigmatic causative prefix have a tropative `find X, consider X' rather than a factitive meaning,  the same as what would have been expected from a  \forme{nɤ-} tropative derivation  (§\ref{sec:tropative}).

The most common tropative causative is  \japhug{znɤja}{find $X$ a shame}, `hate to part with' (\ref{ex:YWznAjandZi}) (in Chinese \ch{舍不得}{shěbùdé}{hate to part with}), which derives from the intransitive \japhug{nɤja}{be a shame} (\ref{ex:tWji.YWnAja}). 

\begin{exe}
\ex \label{ex:YWznAjandZi}
\gll ɯ-mu ɯ-wa ni kɯ ɲɯ-z-nɤja-ndʑi qʰe mɯ-ta-sɯ-ɣe-ndʑi, \\
\textsc{3sg}.\textsc{poss}-mother \textsc{3sg}.\textsc{poss}-father \textsc{du} \textsc{erg} \textsc{sens}-\textsc{caus}-be.a.shame-\textsc{du} \textsc{lnk} \textsc{neg}-\textsc{aor}:3\flobv{}:\textsc{up}-\textsc{caus}-come-\textsc{du} \\
\glt `Her parents could not stand to part with her, and did not let her come.' (14-siblings)
\japhdoi{0003508\#S275}
\end{exe}
 
\begin{exe}
\ex \label{ex:tWji.YWnAja}
\gll  tɕe tɯ-ji ɲɯ-nɤja.   \\
\textsc{lnk} \textsc{indef}.\textsc{poss}-field \textsc{sens}-be.a.shame \\
\glt `These fields, what a shame (nobody is taking care of them).' (150903-friche)
\japhdoi{0006400\#S7}
 \end{exe}
 

Other examples of tropative causative include \japhug{zɣɤtɕa}{consider to be wrong} from \japhug{ɣɤtɕa}{be wrong} (\ref{ex:pjWkWZGAGAtCa}, §\ref{sec:refl.intr}),  \japhug{zɣɤŋgi}{consider to be right} from \japhug{ɣɤŋgi}{be right} and  \japhug{znɤkɤro}{find okay} from \japhug{nɤkɤro}{be okay} (itself a denominal verb, §\ref{sec:denom.intr.nW}). The velar causative also has tropative uses (§\ref{sec:velar.causative.tropative}).

 The expected tropative forms $\dagger$\forme{nɤnɤja}, $\dagger$\forme{nɤɣɤtɕa}, $\dagger$\forme{nɤɣɤŋgi} and $\dagger$\forme{nɤnɤkɤro} are not attested.

In addition, the verbs \japhug{sɯpa}{consider} and \japhug{sɤrtsi}{count as}, causatives of \japhug{pa}{do} (§\ref{sec:pa.lv}, §\ref{sec:idph.pa}) and \japhug{artsi}{be counted as} (§\ref{sec:sig.caus.other.derivations}) are used in the periphrastic tropative construction (§\ref{sec:tropative.sWpa}).

The causative of the denominal verb \japhug{afsuja}{be of the same size} (§\ref{sec:denom.a}) has the near-tropative meaning `compare the length/size of $X$ and $Y$' (rather than `make $X$ and $Y$ have the same length' or `consider that $X$ and $Y$ have the same length') as in  (\ref{ex:tuWGsAfsuja}).

\begin{exe}
\ex \label{ex:tuWGsAfsuja}
\gll mbro ɯ-jme cʰo nɤ-kɤrme tú-wɣ-sɯ-ɤfsuja ra \\
horse \textsc{3sg}.\textsc{poss}-tail \textsc{comit} \textsc{2sg}.\textsc{poss}-hair \textsc{ipfv}-\textsc{inv}-\textsc{caus}-be.of.same.size be.needed:\textsc{fact} \\
\glt `Let us compare [the length of] the horse's tail and that of your hair.' (`Let us see which is longest.') (2014-kWlAG)
\end{exe}  

\subsubsection{Applicative} \label{sec:sig.caus.appl.comitative}
\is{causative!applicative} \is{applicative!causative}
The \forme{sɯ-} prefix has a value that can be described as applicative comitative when used with the motion verb \japhug{rɟɯɣ}{run}. 

Its causative form \forme{sɯrɟɯɣ}, in addition to the regular missive meaning `cause to gallop', can also serve as an allative transitive verb of manipulation `run away with' (§\ref{sec:manipulation.verbs}), as in (\ref{ex:WpCi.chAsWrJWG}).

\begin{exe}
	\ex \label{ex:WpCi.chAsWrJWG}
	\gll jɯlpa ɯ-pɕi tɕe cʰɤ-sɯ-rɟɯɣ.  \\
	village \textsc{3sg}.\textsc{poss}-outside \textsc{loc} \textsc{ifr}:\textsc{downstream}-\textsc{caus}-run \\
	\glt `He ran away from the village carrying her [on his back].' (150829 jidian-zh)
	\japhdoi{0006338\#S79}
\end{exe}
 

\subsection{Stative verbs} \label{sec:sig.caus.stative}
 Although the velar causative \forme{ɣɤ\trt}, rather than the sigmatic prefixes, occurs with most stative verbs, some stative verbs only appear with allomorphs of the sigmatic causative.\footnote{See §\ref{sec:sig.caus.serial} and §\ref{sec:sig.caus.complement} concerning some uses of the causativized stative verbs.  } \tabref{tab:causative.sW.stative} presents a list of representative examples.
 
 \begin{table} 
\caption{Examples of the sigmatic causative prefixes with stative verbs }\label{tab:causative.sW.stative} \centering
\begin{tabular}{lllllll} \lsptoprule
Base  verb&Causative  verb & \\
\midrule
\japhug{wɣrum}{be white} & \forme{sɯ-wɣrum} & §\ref{sec:caus.sWG} \\
\midrule
\japhug{ɲaʁ}{be black} & \forme{sɯɣ-ɲaʁ} & §\ref{sec:caus.sWG}\\
\midrule
\japhug{arŋi}{be green} & \forme{sɯ-ɤrŋi} & §\ref{sec:caus.sA} \\
\midrule
\japhug{ɣɯrni}{be red} & \forme{z-ɣɯrni} & §\ref{sec:caus.z} \\
\japhug{mɤrtsaβ}{be spicy} & \forme{z-mɤrtsaβ} \\
\midrule
\japhug{pe}{be good} & \forme{sɤ-pe} & §\ref{sec:sig.caus.irregular.other} \\
\midrule
\japhug{mŋɤm}{feel pain}  (of a body part) &  \forme{ɕɯ-mŋɤm} & §\ref{sec:caus.CW} \\ 
\lspbottomrule
\end{tabular}
\end{table}
   
Stative verbs with a prefixal syllable (\forme{mɤ\trt}, \forme{rɤ\trt}, \forme{ɣɯ-} etc), always appear with \forme{z\trt}, never with \forme{ɣɤ-} (except some examples with the prefixal element \forme{a-}). This constraint explains for instance why the causative of \japhug{mɤrtsaβ}{be spicy} is  in \forme{z-} rather than \forme{ɣɤ\trt}, while almost all other stative verbs denoting feelings or taste have a causative in \forme{ɣɤ\trt}, for instance \japhug{tɕur}{be sour} \fl{} \japhug{ɣɤtɕur}{make sour} and \japhug{tsri}{be salty} \fl{} \japhug{ɣɤtsri}{make salty'}.
  
Color stative verbs and stative verbs related to disease and pain (\japhug{ngo}{be ill}, \japhug{mŋɤm}{feel pain} etc) also form their causative with \forme{sɯ-} and its variants rather than with \forme{ɣɤ\trt}, as seen in the table above.
  
Only a handful of  stative verbs are compatible with both velar and sigmatic causative prefixes. The semantic contrast between them is treated in §\ref{sec:velar.causative.vs.sigmatic.causative}.
    
\subsection{Recursion} \label{sec:sig.caus.other.recursion}
\is{causative!recursion} \is{recursion!causative}
The causative is the only derivation in Japhug that can occur twice in the same form. A second causative prefix can appear on lexicalized causative verbs, in particular those with irregular allomorphs (§\ref{sec:sig.caus.irregular}), such as \japhug{ɕɯɴqoʁ}{hang} in (\ref{ex:tosWCWNqoR}), where the additional \forme{sɯ-} prefix has a factitive function. 

\begin{exe}
\ex \label{ex:tosWCWNqoR}
\gll mbro nɯnɯ ɯ-ku pjɤ-sɯ-pʰɯt tɕe, ɯ-ku nɯnɯ ʑara ɣɯ nɯ-kɤntɕʰaʁ ɣɯ ɯ-kɯm nɯtɕu to-sɯ-ɕɯ-ɴqoʁ. \\
horse \textsc{dem} \textsc{3sg}.\textsc{poss}-head \textsc{ifr}-\textsc{caus}-take.off \textsc{lnk} \textsc{3sg}.\textsc{poss}-head \textsc{dem} \textsc{3pl} \textsc{gen} \textsc{3pl}.\textsc{poss}-street \textsc{dem} \textsc{3sg}.\textsc{poss}-door \textsc{dem}:\textsc{loc} \textsc{ifr}:\textsc{up}-\textsc{caus}-\textsc{caus}-hang \\
\glt `She had [people] behead the horse, and hang its head on the city gate.' (140428 mu e guniang-zh)
\japhdoi{0003880\#S101}
  \end{exe} 
  
The rogative verb  \japhug{sɤjtsʰi}{ask for something to drink} (§\ref{sec:rogative.derivation}) also contains two instances of the sigmatic causative (at least historically), the irregular \forme{j-} (§\ref{sec:caus.j}), preceded by the passive \forme{a/ɤ-} whose vowel merges with another causative. The underlying form of this verb is thus \forme{sɯ-ɤ-j-tsʰi} (\textsc{caus}-\textsc{pass}-\textsc{caus}-drink).

However, double causativization also occurs with non-lexicalized causatives. For instance, in (\ref{ex:pjWGsWsWspoR}) and (\ref{ex:kuwGsWsWBGWt}), the \forme{sɯ-} prefix closest to the stem is factitive, and the one added to it has the instrumental function (§\ref{sec:sig.caus.instrumental}), meaning  `cause to $X$ with'. 

\begin{exe}
\ex \label{ex:pjWGsWsWspoR}
\gll nɯnɯ kɯ pjɯ́-wɣ-sɯ-sɯ-spoʁ ɲɯ-ŋu. \\
\textsc{dem} \textsc{erg} \textsc{ipfv}-\textsc{inv}-\textsc{caus}-\textsc{caus}-have.a.hole \textsc{sens}-be \\
\glt `One makes a hole [into it] with this.' (24-mbGo) \japhdoi{0003621}
 \end{exe} 
 
\begin{exe}
\ex \label{ex:kuwGsWsWBGWt}
\gll ci nɯ tɕe tɕe smi pjɯ́-wɣ-βlɯ tɕe nɯ smɯmba nɯ kɯ ɯ-rme nɯ kú-wɣ-sɯ-sɯ-βɣɯt ɲɯ-ra. \\
one \textsc{dem} \textsc{lnk} \textsc{lnk} fire \textsc{ipfv}-\textsc{inv}-burn \textsc{lnk} \textsc{dem} flame \textsc{dem} \textsc{erg} \textsc{3sg}.\textsc{poss}-hair \textsc{dem} \textsc{ipfv}-\textsc{inv}-\textsc{caus}-\textsc{caus}-burn.off \textsc{sens}-be.needed \\
\glt `The other [method to remove chicken feathers] is to make a fire and burn off the feathers with the flames.' (150907 kAnWtChWwWt)
\japhdoi{0006282\#S20}
 \end{exe} 
 
 In (\ref{ex:pjWGsWsWspoR}) and (\ref{ex:kuwGsWsWBGWt}), the base verbs \japhug{spoʁ}{have a hole} and \japhug{βɣɯt}{be burned off} (of hairs/feathers) are intransitive, but double causativization is also possible with transitive bases, as in (\ref{ex:pWtAsWsWrAt}).
 
 \begin{exe}
\ex \label{ex:pWtAsWsWrAt}
\gll  pɯ-ta-sɯ-sɯ-rɤt \\
\textsc{aor}-1\fl{}2-\textsc{caus}-\textsc{caus}-write \\
\glt `I made you write it$_i$ with it$_j$ .' 
  \end{exe} 

Another type of double causative occurs when sigmatic and velar causative prefixes are combined (§\ref{sec:velar.caus.other}).

\subsection{Compatibility with other derivations} \label{sec:sig.caus.other.derivations}
\is{causative!compatibility} 
As the most productive verbal derivation, the sigmatic is also the one that is compatible with the greatest number of other derivations.

\tabref{tab:sig.caus:after} lists the derivational prefixes that can follow the causative in the prefixal chain (following the regular allomorphy described in §\ref{sec:sig.caus.allomorphs}). Since all cases are exemplified and discussed in the relevant sections, the data is not reproduced here.  Despite the considerable number of prefixes that are compatible with the sigmatic causative, the reflexive \forme{ʑɣɤ-} (§\ref{sec:refl.caus}), \forme{sɤ-} antipassive (§\ref{sec:antipassive.sA}) and proprietive (§\ref{sec:proprietive}) derivations are never preceded by the sigmatic causative. The autive does not normally follow the causative, except in a handful of lexicalized examples (§\ref{sec:autoben.lexicalized}).

\begin{table}
\caption{Derivations following the sigmatic causative in the prefixal chain} \label{tab:sig.caus:after}
\begin{tabular}{llll}
\lsptoprule
Derivation & Form& Reference\\
\midrule
Applicative & \forme{z-nɯ/ɤ-} & §\ref{sec:appl.other.derivations} \\
Tropative & \forme{z-nɤ-} & §\ref{sec:tropative.other.derivations} \\
Velar causative & \forme{z-ɣɤ-} & §\ref{sec:velar.caus.other}\\
\midrule
Passive & \forme{sɯ-ɤ-} & §\ref{sec:passive.other.derivations}, §\ref{sec:rogative.derivation} \\
Reciprocal & \forme{sɯ-ɤ-} + reduplication &  §\ref{sec:reciprocal.other}  \\
Anticausative &\forme{sɯ(ɣ)}+prenasalization & §\ref{sec:anticausative.other.derivations}\\
Antipassive &\forme{z-rɤ-}  & §\ref{sec:antipassive.compatibility} \\
Antipassive &\forme{z-rɤ-}  & §\ref{sec:antipassive.compatibility} \\
Distributed property,  & \forme{sɯ-ɤmɯ-} & §\ref{sec:distributed.amW}, §\ref{sec:sAmW} \\
reciprocal \\
Facilitative & \forme{z-nɯɣɯ-} & §\ref{sec:facilitative.nWGW} \\
\midrule
Distributed action&\forme{z-nɤ-}  + reduplication& §\ref{sec:distributed.action.other} \\
\lspbottomrule
\end{tabular}
\end{table}

Fewer derivations can occur before the causative in the prefixal chain. The other valency-increasing derivations (applicative, tropative, velar causative) never take a sigmatic causative verb as input. Among valency-decreasing derivations, the anticausative prenasalization is never attested on causative verbs.

Three valency-decreasing derivations precede the causative only in a handful of lexicalized verbs. The antipassive and passive derivations are only found on \japhug{rɤjtsʰi}{give to someone to drink} (§\ref{sec:antipassive.ditransitive}) and \japhug{ajtsʰi}{be given to drink} (§\ref{passive.ditransitive}), from the irregular causative \japhug{jtsʰi}{give to drink} (§\ref{sec:caus.j}) of \japhug{tsʰi}{drink} and \japhug{sɤnɯsɯkʰo}{rob people} from the autive-causative \japhug{nɯsɯkʰo}{rob, extort} (§\ref{sec:antipassive.compatibility}). 

In addition, the progressive prefix \forme{asɯ-} probably derives from the combination of the passive with the sigmatic causative (§\ref{sec:progressive.history}), but cannot be analyzed this way synchronically.

The derivations that can productively take a sigmatic causative verb as input are presented in \tabref{tab:sig.caus:before}.

\begin{table}
\caption{Derivations preceding the sigmatic causative in the prefixal chain} \label{tab:sig.caus:before}
\begin{tabular}{llll}
\lsptoprule
Derivation & Form& Reference\\
\midrule
Reflexive& \forme{ʑɣɤ-sɯ(ɣ)-}   &  §\ref{sec:refl.caus}  \\
Reciprocal & \forme{a-sɯ(ɣ)-} + reduplication &  §\ref{sec:reciprocal.other}  \\
Facilitative & \forme{nɯɣɯ-sɯ(ɣ-)} & §\ref{sec:facilitative.nWGW} \\
\midrule
Autive & \forme{nɯ-sɯ(ɣ)-} & §\ref{sec:autoben.position} \\
\lspbottomrule
\end{tabular}
\end{table}

From Tables \ref{tab:sig.caus:after} and \ref{tab:sig.caus:before}, we see that only two derivations, the reduplicated reciprocal (§\ref{sec:reciprocal.other}) and the \forme{nɯɣɯ-} object-oriented facilitative (§\ref{sec:facilitative.nWGW}) can both freely precede and follow the sigmatic causative in the prefixal chain. Unlike negation (§\ref{sec:sig.caus.negation}) and associated motion (§\ref{sec:sig.caus.AM}) prefixes, whose semantic scope with the causative is independent of their relative position in the template, in the case of the reciprocal and facilitative, the relative position influences the semantic scope of the prefixes (§\ref{sec:inner.prefixal.chain}).
 
\section{Velar causative} \label{sec:velar.causative}
\is{causative!velar} 
The velar causative prefix \forme{ɣɤ-} (in the Kamnyu dialect, corresponding to \forme{wɤ-} in eastern dialects) is the second causative derivation found in Japhug. Unlike the sigmatic causative (§\ref{sec:sig.causative}), it is restricted to stative verb bases. Not all stative verbs, however, build their causative form with the velar causative. In particular, polysyllabic bases, including those whose non-final syllable is a derivation prefix or an unanalyzable element (such as \forme{a\trt}, \forme{nV\trt}, \forme{rV\trt}, \forme{sV-} etc) can only be causativized with sigmatic prefixes  (§\ref{sec:sig.caus.stative}). A handful of verbs are compatible with both velar and sigmatic causative prefixes (§\ref{sec:velar.causative.vs.sigmatic.causative}).
 
\tabref{tab:velar.caus} presents a representative sample of velar causative verbs, including adjectival stative verbs, existential verbs and modal verbs. The presence of Tibetan loanwords such as \japhug{dɤn}{be many}, \japhug{βdi}{be well} and \japhug{tsʰoz}{be complete}  (from \tibet{ལྡན་}{ldan}{possessing}, \tibet{བདེ་}{bde}{well} and \tibet{ཚངས་}{tsʰaŋs}{complete}) in the list shows that this prefix is productive.

\begin{table}
\caption{Examples of velar \forme{ɣɤ-} causative derivations} \label{tab:velar.caus}
\begin{tabular}{lll}
\lsptoprule
Base verb & Derived verb \\
\midrule
\japhug{dɤn}{be many} & \japhug{ɣɤdɤn}{increase} \\
\japhug{βdi}{be well} & \japhug{ɣɤβdi}{repair}, `make better' \\
%\japhug{χtso}{be clean} & \japhug{ɣɤχtso}{make clean} \\
\japhug{tsʰoz}{be complete} & \japhug{ɣɤtsʰoz}{make complete} \\
\japhug{wxti}{be big} & \japhug{ɣɤwxti}{make bigger} \\
\japhug{jom}{be broad} & \japhug{ɣɤjom}{broaden} \\
\japhug{mna}{be better}  & \japhug{ɣɤmna}{heal}, `make better' \\
\japhug{smi}{be cooked} & \japhug{ɣɤsmi}{cook} \\
\midrule
\japhug{me}{not exist} & \japhug{ɣɤme}{destroy} \\
\japhug{maʁ}{not be} & \japhug{ɣɤmaʁ}{cause not to be} \\
\japhug{ra}{be needed} & \japhug{ɣɤra}{cause to have to} \\
\japhug{kʰɯ}{be possible} & \japhug{ɣɤkʰɯ}{make it possible to} \\
\lspbottomrule
\end{tabular}
\end{table}
  
The velar causative \forme{ɣɤ-} prefix is homophonous with a few other derivational prefixes, including denominal (§\ref{sec:denom.GW}), deideophonic (§\ref{sec:GA.sA.deidph}) and subject-oriented facilitative (§\ref{sec:facilitative.GA}). The latter derives intransitive verbs whose stems are identical (with different conjugations however) to those of velar causatives, for instance \japhug{ɣɤwxti}{become big easily} vs. \japhug{ɣɤwxti}{make bigger}  from \japhug{wxti}{be big}.
  
%ɣɤjom
%ɣɤmbɤr 

The causative verbs from com\-ple\-ment-taking modal verbs such as \japhug{kʰɯ}{be possible} can occur with nominal objects, especially abstract nouns as in (\ref{ex:tWtsGe.mWYWkAGAkhW}), but are more commonly found with complement clauses as objects (§\ref{sec:velar.caus.modal}).
 \newpage
\begin{exe}
\ex \label{ex:tWtsGe.mWYWkAGAkhW}
 \gll ndʑi-tɯtsɣe ra mɯ-ɲɯ-kɤ-ɣɤ-kʰɯ ftɕaka ntsɯ tu-βze pjɤ-ŋu. \\
 \textsc{2du}.\textsc{poss}-commerce \textsc{pl} \textsc{neg}-\textsc{ipfv}-\textsc{inf}-\textsc{caus}-be.possible manner always \textsc{ipfv}-make[III] \textsc{ifr}.\textsc{ipfv}-be \\
 \glt `He was always trying [by all means] to make it impossible for them to manage their business.' (150825 baishe zhuan-zh)
 \japhdoi{0006342\#S99}
\end{exe} 

\subsection{Irregular allomorphs} \label{sec:causative.m}
\is{causative!allomorphy}\is{morphology!allomorphy}
Irregular allomorphs of the velar causative are very rare and highly lexicalized. The \forme{ɣɤ-} prefix clearly originates from earlier \forme{*wɐ\trt}, as shown by the form \forme{wɤ-} in eastern Japhug dialects and the cognate prefixes \forme{wɐ-} in Tshobdun and Zbu \citep{jackson14morpho}.  The vowel-less irregular allomorphs originate from earlier \forme{*w\trt}, nasalized to \ipa{m} before nasal and prenasalized stops. 

A \ipa{w-} allomorph, realized as \forme{β\trt}, is found in the verb \japhug{βri}{protect}, `save' (\ref{ex:tatWBrit}),\footnote{This verb can take as object the entity being protected as in (\ref{ex:tatWBrit}) (see also \ref{ex:iZora.tutafsraN}, §\ref{sec:indexation.local}), but also in some cases the entity one protects something from; for instance the participial clause \forme{qale ɯ-kɯ-βri} (wind \textsc{3sg}.\textsc{poss}-\textsc{sbj}:\textsc{pcp}-protect) means `(hedge) that protects from the wind'. } which derives from the intransitive \japhug{ri}{remain}, `be left' (see examples \ref{ex:zgo.tWrdoR}, §\ref{sec:multiple.CN} and \ref{ex:tWrdoR.YAri}, §\ref{sec:lability.pass}). The bare root \forme{ri} is also attested in the transitive verb \forme{ri} `save' (always in collocation with \japhug{tɯ-sroʁ}{life}, §\ref{sec:orphan.verb}), a zero derivation from \japhug{ri}{remain} (§\ref{sec:lability.pass}). 

\begin{exe}
\ex \label{ex:tatWBrit}
 \gll nɤ-pi ni tɤ-tɯ-βri-t ŋu \\
\textsc{2sg}.\textsc{poss}-elder.sibling \textsc{du} \textsc{aor}-2-save-\textsc{pst}:\textsc{tr} be:\textsc{fact} \\
\glt `You saved your two elder brothers.' (2003 qachGa)
\japhdoi{0003372\#S133}
\end{exe}

The lexicalized causative \forme{βri} can be subjected to reflexivization (§\ref{sec:refl.caus}) in \japhug{ʑɣɤβri}{protect oneself} (example \ref{ex:tWZo.kW.tWZo}, §\ref{sec:genr.pro}). 

The regular sigmatic causative \forme{sɯɣ-ri} from the base verb \japhug{ri}{remain} also exists, and  predictably means  `leave, not use up completely' as in (\ref{ex:tWndzrW.anWtWsWGri}). It is probable that \forme{βri} originally also had a meaning close to that of \forme{sɯɣ-ri}, and then changed to `save, protect'.   
\largerpage
\begin{exe}
\ex \label{ex:tWndzrW.anWtWsWGri}
 \gll ki a-tɤ-tɯ-ndze qʰe qʰe ɯ-qa nɯtɕu tɯ-ndzrɯ jamar ci ʑo a-nɯ-tɯ-sɯɣ-ri ma \\
 \textsc{dem}.\textsc{prox} \textsc{irr}-\textsc{pfv}-2-eat[III] \textsc{lnk} \textsc{lnk} \textsc{3sg}.\textsc{poss}-bottom \textsc{dem}:\textsc{loc} \textsc{indef}.\textsc{poss}-nail about \textsc{indef} \textsc{emph}  \textsc{irr}-\textsc{pfv}-2-\textsc{caus}-remain \textsc{lnk} \\
\glt `When you eat this, leave a [quantity of] about a nail [from it in the bowl].' (2003kandZislama)
\end{exe}
 
The \forme{m-} allomorph of the velar causative occurs in the verb \japhug{mɲo}{prepare} (see \ref{ex:aZo.tAmYota} in §\ref{sec:A.kW}, and the discussion in §\ref{sec:z.nmlz}), a causative of the intransitive stative verb \japhug{ɲo}{be prepared} (with nasalization \forme{*w-ɲo} \fl{} \forme{mɲo}). This  intransitive verb is particularly common in the participle form \forme{kɯ-ɲo} `already prepared, ready to (eat)' (corresponding to Chinese \ch{现成}{xiànchéng}{ready-made}), as in (\ref{ex:kWYo.tundzaj}).  

\begin{exe}
\ex \label{ex:kWYo.tundzaj}
 \gll ma jinde tɕe ku-χsu-j me kɯ-ɲo ntsɯ tu-ndza-j ɕti ma \\
 \textsc{lnk} nowadays \textsc{lnk} \textsc{ipfv}-raise-\textsc{1pl} not.exist:\textsc{fact} \textsc{sbj}:\textsc{pcp}-be.ready always \textsc{ipfv}-eat-\textsc{1pl} be.\textsc{aff}:\textsc{fact} \textsc{lnk} \\
 \glt `Nowadays we do not raise [chicken] anymore, we eat `ready-made' [eggs].'  (22-kumpGa)
\japhdoi{0003588\#S79}
 \end{exe}
 
Like \forme{βri} above, the causative \forme{mɲo} can be reflexivized to \japhug{ʑɣɤmɲo}{prepare oneself} (§\ref{sec:refl.caus}).  A regular causative form \forme{sɯɣ-ɲo} `prepare' is also attested, but it is much rarer than \forme{mɲo}, and the semantic difference between these two causative forms has not been elucidated.

Another example of the nasalized \forme{m-} allomorph is the rare transitive verb \japhug{mdzar}{drip dry}  (\ref{ex:pWmdzara})\footnote{This verb also appears in participial form in the compound \japhug{kɯndzarmɯ}{type of rain} (see \ref{ex:kWndzarmW}, §\ref{sec:lexicalized.subject.participle}). } which derives from the intransitive verb \japhug{ndzar}{drip dry}  (\ref{ex:apWndzar}).\footnote{The intransitive \japhug{ndzar}{drip dry} itself is the anticausative form of \japhug{sar}{filter out}, `drip dry' (§\ref{sec:anticausative.pairs}). } The onset \forme{mdz-} is phonologically \ipa{mndz\trt}, with the causative \forme{*w-} prefix nasalized by the prenasalized voiced affricate \ipa{ndz}, a unitary phoneme (§\ref{sec:NC.clusters}).

\begin{exe}
\ex \label{ex:apWndzar}
\gll  tɯ-ŋga nɯ-χtɕi-t-a tɕe a-pɯ-ndzar tɕe cʰɯ́-wɣ-ɕkʰo jɤɣ \\
\textsc{indef}.\textsc{poss}-clothes \textsc{aor}-wash-\textsc{pst}:\textsc{tr}-\textsc{1sg} \textsc{lnk} \textsc{irr}-\textsc{pfv}-drip.dry \textsc{lnk} \textsc{ipfv}:\textsc{downstream}-\textsc{inv}-spread be.allowed:\textsc{fact} \\
\glt `I have washed the clothes, let them first drip dry before putting them in the sun to dry.' (elicited)
\end{exe}
 
\begin{exe}
\ex \label{ex:pWmdzara}
\gll mbrɤz kɯ-fse lo, <cai> kɯ-fse nɯnɯra tɕe, ɯ-ŋgɯ tɯ-ci kɯ-tu nɯra pjɯ́-wɣ-sɯ-ɤrɕo tɕe, nɯnɯ ``pɯ-mdzar-a" tu-kɯ-ti ŋu. \\
rice \textsc{sbj}:\textsc{pcp}-be.like \textsc{sfp} dish  \textsc{sbj}:\textsc{pcp}-be.like  \textsc{dem}:\textsc{pl} \textsc{lnk}  \textsc{3sg}.\textsc{poss}-inside \textsc{indef}.\textsc{poss}-water \textsc{sbj}:\textsc{pcp}-exist \textsc{dem}:\textsc{pl} \textsc{ipfv}-\textsc{inv}-\textsc{caus}-be.finished \textsc{lnk} \textsc{dem} \textsc{aor}-drip.dry-\textsc{1sg} \textsc{ipfv}-\textsc{genr}-say be:\textsc{fact} \\
\glt `When one has completely removed the water in the rice or in a dish, one says \forme{pɯ-mdzar-a}.' (definition)
\end{exe} 
 
 
\subsection{Morphosyntax} \label{sec:velar.causative.morphosyntax}

\subsubsection{Negation} \label{sec:velar.causative.negation}
The combination of the velar causative with negation prefixes is ambiguous. As in the case of the sigmatic causative (§\ref{sec:sig.caus.negation}), the scope of the negation can either include the causation (`not cause to $X$') or not (`cause not to $X$'). For instance, the causative  \japhug{ɣɤwxti}{make bigger} in negative form can either be `not make bigger', or  `make smaller' as in (\ref{ex:matAtWGAwxti}).

\begin{exe}
\ex \label{ex:matAtWGAwxti}
\gll  ɯ-pʰɯ ɲɯ-wxti tɕe, nɯra tʰamtɕɤt ma-tɤ-tɯ-ɣɤ-wxti \\
\textsc{1sg}.\textsc{poss}-price \textsc{sens}-be.big \textsc{lnk} \textsc{dem}:\textsc{pl} all \textsc{neg}-\textsc{imp}-2-\textsc{caus}-be.big \\
\glt  `It is expensive, make it less expensive.' (2010-12)
\end{exe} 
 
With the velar causative from modal verbs, the broad scope interpretation of the negation is very common. For instance,  \japhug{ɣɤra}{cause to have to} and  \japhug{ɣɤkʰɯ}{make it possible to} in the negative form generally mean `cause not to have to' and `make it impossible to' (§\ref{sec:velar.caus.modal}).

\largerpage
\subsubsection{Complement clauses with causative of manner} \label{sec:velar.caus.complement}
As with the sigmatic causative (§\ref{sec:sig.caus.complement}), the velar causative can derive com\-ple\-ment-taking verbs expressing manner (§\ref{sec:causative.manner.complement}), selecting complement clauses with velar infinitive (§\ref{sec:inf.complementation}) or bare infinitive (§\ref{sec:bare.inf.complement}).

The examples in (\ref{ex:GAtChom}) illustrate the possible constructions with the causative \japhug{ɣɤtɕʰom}{cause to be too much}, `do $X$ too much' from \japhug{tɕʰom}{be too much} and the verb \japhug{tsʰi}{drink} expressing the main action in the complement clauses. The velar infinitive (\ref{ex:kAtshi.koGAtChom}) and bare infinitive (\ref{ex:Wtshi.loGAtChom}) constructions have the same meaning `drink too much', and the causative verb takes the orientation prefix normally selected by the verb in the complement clause, in this case either \textsc{eastwards} (\ref{ex:kAtshi.koGAtChom}) or \textsc{upstream} (\ref{ex:Wtshi.loGAtChom}). 

\begin{exe}
\ex \label{ex:GAtChom}
\begin{xlist}
\ex \label{ex:kAtshi.koGAtChom}
\gll [cʰa kɤ-tsʰi] ko-ɣɤ-tɕʰom \\
alcohol \textsc{inf}-drink \textsc{ifr}-\textsc{caus}-be.too.much \\
\ex \label{ex:Wtshi.loGAtChom}
\gll [cʰa ɯ-tsʰi] lo-ɣɤ-tɕʰom \\
alcohol \textsc{3sg}.\textsc{poss}-\textsc{bare}.\textsc{inf}:drink \textsc{ifr}-\textsc{caus}-be.too.much \\
\glt `He drank too much alcohol.' (elicited)
\ex \label{ex:chAtshi.koGAtChom2}
\gll cʰɤtsʰi ko-ɣɤ-tɕʰom \\
alcohol.drinking \textsc{ifr}-\textsc{caus}-be.too.much \\
\glt `He had drunk too much alcohol.' (150829 jidian-zh)
\japhdoi{0006338\#S16}
\end{xlist}
\end{exe}

In addition, example (\ref{ex:chAtshi.koGAtChom2}) shows that object-verb action nominal compounds (§\ref{sec:object.verb.compounds}, §\ref{sec:action.nominal.compounds}) can occur with causative verbs instead of complement clauses.

While action nominal compounds are rare, the causative complement construction with velar and bare infinitive is very common and compatible with all verb categories. Even noun-verb collocations such as \forme{tɯ-xɕɤt+lɤt} `exert strength, do $X$ forcibly' can occur in these complement clauses as in (\ref{ex:WxCAt.WlAt.toGAtChom}) (with an  \textsc{upwards} orientation preverb).

\begin{exe}
\ex \label{ex:WxCAt.WlAt.toGAtChom}
\gll  rgɤtpu nɯ kɯ ɯ-tɕɯ ja-stʰoʁ tɕe, ɯ-xɕɤt ɯ-lɤt to-ɣɤ-tɕʰom tɕe \\
old.man \textsc{dem} \textsc{erg} \textsc{3sg}.\textsc{poss}-son \textsc{aor}:3\flobv{}-push \textsc{lnk} \textsc{3sg}.\textsc{poss}-strength \textsc{3sg}.\textsc{poss}-release \textsc{ifr}-\textsc{caus}-be.too.much \textsc{lnk} \\
\glt `The old man pushed his son, but exerted too much strength and...'   (150831 jubaopen-zh)
\japhdoi{0006294\#S143}
\end{exe}

All velar causative verbs, even those that are generally used with a concrete factitive meaning, can be used in these com\-ple\-ment-taking constructions. For instance, \forme{ɣɤ-βdi} (from \japhug{βdi}{be well}), which can mean `repair' with a nominal object (see \ref{ex:WZAG.nWra}, §\ref{sec:pronouns.gen}), occurs in the sense of `do $X$ well, do $X$ nicely' with a complement clause as in (\ref{ex:akAtWGABdi}).

\begin{exe}
\ex \label{ex:akAtWGABdi}
\gll  kʰa ɯ-ʁɤri nɯtɕu ɯ-fkrɤm a-kɤ-tɯ-ɣɤ-βdi \\
house \textsc{3sg}.\textsc{poss}-front \textsc{dem}:\textsc{loc} \textsc{3sg}.\textsc{poss}-\textsc{bare}.\textsc{inf}:place \textsc{irr}-\textsc{pfv}-2-\textsc{caus}-be.well \\
\glt `Place these nicely [in order] in front of the house.' (smanmi 2003.1)
\end{exe}

\subsubsection{Modal and existential verbs} \label{sec:velar.caus.modal}
Velar causative forms of modal verbs, in particular \japhug{ɣɤra}{cause to have to} and  \japhug{ɣɤkʰɯ}{make it possible to}, are found with velar infinitive or finite complement clauses (§\ref{sec:sWpa.sABzu}) like the corresponding base verbs \japhug{ra}{be needed} and \japhug{kʰɯ}{be possible} (§\ref{sec:ra.khW.jAG.verb}) as in (\ref{ex:mWtAGAkhWta}), but not with bare or dental infinitives.

\begin{exe}
\ex \label{ex:mWtAGAkhWta}
\gll  [kɤ-nɯ-ɬoʁ] mɯ-tɤ-ɣɤ-kʰɯ-t-a. \\
  \textsc{inf}-\textsc{auto}-come.out \textsc{neg}-\textsc{aor}-\textsc{caus}-be.possible-\textsc{pst}:\textsc{tr}-\textsc{1sg} \\
\glt `I prevented it/him from coming out.' (elicited)
\end{exe}

The object of the causative verb is most often the (transitive or intransitive) subject of the complement clause as in (\ref{ex:mWtAGAkhWta}), but not necessarily; in (\ref{ex:kAtu.tAGAkhWta}), the object of \forme{tɤ-ɣɤ-kʰɯ-t-a} is the possessor of the subject (in the possessive construction, §\ref{sec:possessive.mihi.est}).

\begin{exe}
\ex \label{ex:kAtu.tAGAkhWta}
\gll [ɯ-rŋɯl kɤ-tu] tɤ-ɣɤ-kʰɯ-t-a. \\
\textsc{3sg}.\textsc{poss}-money \textsc{inf}-exist \textsc{aor}-\textsc{caus}-be.possible-\textsc{pst}:\textsc{tr}-\textsc{1sg} \\
\glt `I made it possible for him/her to have money.'  (elicited)
\end{exe}

In (\ref{ex:mWYAGAra}), the causative \forme{ɣɤra} occurs with a clause containing the form \forme{kɯ-ra} that can either be interpreted as a relative (with a subject participle) or as a complement clause (in which case the \forme{kɯ-} prefix is preferably analyzed as a stative infinitive).

\begin{exe}
\ex \label{ex:mWYAGAra}
\gll <xianling> nɯ nɯ-rga qʰendɤre, [...] [nɯ-ʁjoʁ kɯ-ra] nɯ mɯ-ɲɤ-ɣɤ-ra qʰe, \\
magistrate \textsc{dem} \textsc{aor}-be.happy \textsc{lnk} { } \textsc{3pl}.\textsc{poss}-servant \textsc{sbj}:\textsc{pcp}-be.needed \textsc{dem} \textsc{neg}-\textsc{ifr}-\textsc{caus}-be.needed \textsc{lnk} \\
\glt `The magistrate was happy, and cancelled the duties that they had to do.' (150904 cuzhi-zh)
\japhdoi{0006322\#S173}
\end{exe}
 
 The existential verb \forme{tu}  (§\ref{sec:existential.basic}) can also be subjected to the \forme{ɣɤ-} derivation, which yields the verb \japhug{ɣɤtu}{cause to have}. This form is used to causativize possessive constructions (§\ref {sec:possessive.constructions}) or noun-verb collocations (§\ref{sec:intr.light.verbs}) as in (\ref{ex:axCAt.Zo.tuGAte}).
 
\begin{exe}
\ex \label{ex:axCAt.Zo.tuGAte}
\gll  a-xɕɤt ʑo tu-ɣɤte tɕe \\
\textsc{1sg}.\textsc{poss}-strength \textsc{emph} \textsc{ipfv}-\textsc{caus}-exist[III] \textsc{lnk} \\
\glt `[The rain] gives me (`makes me have') strength.' (150819 woniu-zh)
\japhdoi{0006254\#S49}
\end{exe}

\subsubsection{Collocations} \label{sec:velar.causative.collocation}
\is{causative!collocation} \is{collocation!causative}
Complex predicates comprising a noun-verb collocation can undergo causativization with the \forme{ɣɤ-} prefix.\footnote{The causativization of complex predicates with the sigmatic prefixes is treated in §\ref{sec:sig.caus.collocations}. }

For instance, the combination of the lexicalized object participle \forme{kɤ-ti} (\textsc{obj}:\textsc{pcp}-say) with the existential verb \japhug{me}{not exist}, which means  `have nothing to say' or `be unable to say for sure' (see example \ref{ex:akAti.maNe}, §\ref{sec:lexicalized.object.participle}), can be causativized to \forme{kɤ-ti+ɣɤ-me} `cause $X$ to have nothing to say' as (\ref{ex:WkAti.naGame}).

\begin{exe}
\ex \label{ex:WkAti.naGame}
\gll rɟɤlpu nɯ ɯ-kɤ-ti na-ɣɤ-me \\
king \textsc{dem} \textsc{3sg}.\textsc{poss}-\textsc{obj}:\textsc{pcp}-say \textsc{aor}:3\flobv{}-\textsc{caus}-not.exist \\
\glt `He made the king unable to say anything.' (2005 tAwakWcqraR)
\end{exe}
 

\subsection{Semantics} \label{sec:velar.causative.semantics}
 
\subsubsection{Tropative} \label{sec:velar.causative.tropative}
\is{causative!tropative} \is{tropative!causative}
 The velar causative does have a tropative interpretation in specific contexts, like the sigmatic causative (§\ref{sec:sig.caus.tropative}). The verb \japhug{ɣɤwxti}{make bigger} from \japhug{wxti}{be big}  for instance can mean `find/consider to be bigger' as in (\ref{ex:mWpjAnWGAwxti}).

 \begin{exe}
\ex \label{ex:mWpjAnWGAwxti}
\gll kɯmaʁ ɯ-ɣi nɯra nɯ-rtsawa mɯ-pjɤ-nɯ-ɣɤ-wxti \\
other \textsc{3sg}.\textsc{poss}-relative \textsc{dem}:\textsc{pl} \textsc{3pl}.\textsc{poss}-importance \textsc{neg}-\textsc{ifr}.\textsc{ipfv}-\textsc{auto}-\textsc{caus}-be.big \\
\glt `He did not consider his other relatives to be as important [as his wife].' (kWjujmAlu 2003)
\end{exe} 

A tropative velar causative with a slightly lexicalized meaning is \japhug{ɣɤkʰe}{depreciate, demean}  (\ref{ex:WkWGAkhe.YWxcat}) (see also \ref{ex:ndZitAresAtCWtCAt}, §\ref{sec:lexicalized.oblique.participle}), from \japhug{kʰe}{be stupid}.
 
 \begin{exe}
\ex \label{ex:WkWGAkhe.YWxcat}
\gll  ɯ-kɯ-n-nɤmqe, ɯ-kɯ-nɤre, ɯ-kɯ-ɣɤ-kʰe nɯra ɲɯ-xcat ʑo. \\
\textsc{3sg}.\textsc{poss}-\textsc{auto}-scold \textsc{3sg}.\textsc{poss}-laugh.at \textsc{3sg}.\textsc{poss}-\textsc{caus}-be.stupid \textsc{dem}:\textsc{pl} \textsc{sens}-be.many \textsc{emph} \\
\glt `There were many people scolding him, making fun of him, calling him stupid.' (150829 phaRrgot)
\japhdoi{0006414\#S11}
 \end{exe} 
 
\subsubsection{Velar vs. sigmatic causatives} \label{sec:velar.causative.vs.sigmatic.causative}
\is{causative!velar}  \is{causative!sigmatic} 
Some stative verbs are compatible with both sigmatic (§\ref{sec:sig.caus.stative}) and velar causative prefixes, an observation which raises the question of the semantic distinction between these two derivations when contrastive.\footnote{The contrast between the two causative derivations in the case of the irregular allomorphs \forme{β-} and \forme{m-} is discussed in §\ref{sec:causative.m}. }

\citet{jackson06paisheng, jackson14morpho}, with regard to the causative prefixes \forme{sə} and \forme{wɐ-} in Tshobdun, proposes that in the case of some stative verbs, the former indicates an increase of degree (\ref{ex:nekAsEGchi}), while the latter expresses a change of state (\ref{ex:nekAwAchi}). 

\begin{exe}
\ex 
\begin{xlist}
\ex \label{ex:nekAsEGchi}
\gll cʰɐ́ɟi ne-kɐ-səɣ-cʰiʔ=nəʔ mimʔ=cə \\
beer \textsc{ipfv}-\textsc{genr}-\textsc{caus}-be.sweet=\textsc{dem} be.tasty=\textsc{med} \\
\glt ‘Beer is tasty when one allows it to sweeten (naturally and gradually).’
\ex \label{ex:nekAwAchi}
\gll cʰɐ́ɟi ne-kɐ-wɐ-cʰiʔ=nəʔ mimʔ=cə \\
beer \textsc{ipfv}-genr-\textsc{caus}-be.sweet=\textsc{dem} be.tasty=\textsc{med} \\
\glt ‘Beer is tasty when one sweetens it (e.g. by adding sugar).’
\end{xlist}
\end{exe}

In Japhug, it is not completely clear whether a semantic contrast of the same type is attested. Minimal pairs such as \forme{sɯx-cʰi} and \forme{ɣɤ-cʰi} from \japhug{cʰi}{be sweet} or \forme{sɯx-tɕur} and \forme{ɣɤ-tɕur} from \japhug{tɕur}{be sour} do exist, but no consistent semantic difference appears to exist between them.

Examples (\ref{ex:pjWsWxtCur}) and (\ref{ex:YWsWxtCur}) suggest that the causative  \forme{sɯx-tɕur} means `make sour' rather than `make more sour' as would be expected following Sun's analysis of Tshobdun.\footnote{In the case of (\ref{ex:pjWsWxtCur}), it is possible that \forme{tɤjko mɯ-tɤ-tɕur} can be contextually translated as `when the pickle is not sour \textit{enough}', and that therefore the verb \forme{pjɯ-sɯx-tɕur} does indeed mean `make more sour'. \iai{Tshendzin} proposed conflicting interpretations of this example. The occurrence of \forme{sɯx-tɕur} in (\ref{ex:YWsWxtCur}) however is incompatible with an analysis in terms of heightened degree. } 

 \begin{exe}
\ex \label{ex:pjWsWxtCur}
\gll tɕe tɤjko mɯ-tɤ-tɕur tɕe, ɴɢolo ɯ-mat nɯ ɲɯ́-wɣ-pʰɯt tɕe, tɕe tɤrca pjɯ́-wɣ-ɣɤ-la tɕe, tɕe tɤjko pjɯ-sɯx-tɕur cʰa. \\
\textsc{lnk} pickle \textsc{neg}-\textsc{aor}-be.sour \textsc{lnk}  Ribes.stenocarpum \textsc{3sg}.\textsc{poss}-fruit \textsc{dem} \textsc{ipfv}-\textsc{inv}-take.off \textsc{lnk} \textsc{lnk}  together \textsc{ipfv}-\textsc{inv}-\textsc{caus}-soak \textsc{lnk} \textsc{lnk} pickle \textsc{ipfv}-\textsc{caus}-be.sour can:\textsc{fact} \\
\glt `When the pickle$_i$ is not sour, one picks fruits from the \textit{Ribes stenocarpum}$_j$, soaks them together [with it$_i$], and it$_j$ can make the pickle sour.' (18-NGolo)
\japhdoi{0003530\#S25}
\end{exe}

 \begin{exe}
\ex \label{ex:YWsWxtCur}
\gll ɯ-tɯ-tɕur kɯ tɯ-kɯr ɯ-ŋgɯ lú-wɣ-rku qʰe maka ɲɯ-sɯ-ɤmɯ-zɣɯt qʰe, tɯ-pʰoŋbu ra kɯnɤ ɲɯ-sɯx-tɕur kɯ-fse ɕti\\
\textsc{3sg}.\textsc{poss}-\textsc{nmlz}:\textsc{deg}-be.sour \textsc{erg} \textsc{genr}.\textsc{poss}-mouth \textsc{3sg}.\textsc{poss}-inside \textsc{ipfv}:\textsc{upstream}-\textsc{inv}-put.in \textsc{lnk} at.all  \textsc{ipfv}-\textsc{caus}-\textsc{distr}-reach \textsc{lnk} \textsc{genr}.\textsc{poss}-body \textsc{pl} also \textsc{sens}-\textsc{caus}-be.sour \textsc{sbj}:\textsc{pcp}-be.like be.\textsc{aff}:\textsc{fact}\\
\glt `It is so sour that when one puts it into one's mouth, it makes everything [sour in an even way], as if one's whole body becomes sour.' (09-mi)
\japhdoi{0003466\#S60}
\end{exe}

In addition, example (\ref{ex:pWGAtCura}) shows that the velar causative  \forme{ɣɤ-tɕur} is not incompatible with the meaning `make more sour'. It is possible that other semantic parameters (such as direct/indirect causation and volitionality) are at play in the choice of the two prefixes.

 \begin{exe}
\ex \label{ex:pWGAtCura}
\gll  <cai> ɯ-tɯ-tɕur mɯ́j-rtaʁ tɕe, <cu> pɯ-lat-a tɕe pɯ-ɣɤ-tɕur-a \\
dish \textsc{3sg}.\textsc{poss}-\textsc{nmlz}:\textsc{deg}-be.sour \textsc{neg}:\textsc{sens}-be.enough \textsc{lnk}, vinegar \textsc{aor}-release-\textsc{1sg} \textsc{lnk} \textsc{aor}-\textsc{caus}-be.sour-\textsc{1sg} \\
\glt `As the dish was not sour enough, I added vinegar to make it sourer.' (elicited)
\end{exe}

The only verb root that has sigmatic and velar causative forms with clearly different meanings is \forme{mto}, a labile verb meaning `see' when conjugated transitively, and `have sharp eyesight' when intransitive (§\ref{sec:lability.categories}). The sigmatic causative \forme{sɯ-mto}, based on the transitive use of \forme{mto}, means `let see, show' as in (\ref{ex:pjWtWwGsWmto.cha}).

%\japhug{mto}{see}

\begin{exe}
\ex \label{ex:pjWtWwGsWmto.cha}
\gll  kɯki tɤ-ndɤm tɕe, [...] ʑimkʰɤm ɯ-ku, tɕʰi kɯ-tu ʑo nɯ pjɯ-tɯ́-wɣ-sɯ-mto cʰa \\
\textsc{dem}.\textsc{prox} \textsc{imp}-take[III] \textsc{lnk} { } world \textsc{3sg}.\textsc{poss}-head what \textsc{sbj}:\textsc{pcp}-exist \textsc{emph} \textsc{dem} \textsc{ipfv}-2-\textsc{inv}-\textsc{caus}-see can:\textsc{fact} \\
\glt `Take this [spyglass], it will make you able to see everything that exists in the world.' (140508 benling gaoqiang de si xiongdi-zh)
\japhdoi{0003935\#S74}
\end{exe} 

The velar causative  \japhug{ɣɤmto}{cause to recover eyesight} is based on the stative use `have sharp eyesight'. Just as the base verb requires the noun \japhug{tɯ-mɲaʁ}{eye} as intransitive subject (example \ref{ex:WmYaR.YWmto}, §\ref{sec:lability.categories}), \forme{ɣɤmto} requires it as object as in (\ref{ex:tuGAmtAm.cha}).

\begin{exe}
\ex \label{ex:tuGAmtAm.cha}
\gll  nɯnɯ kɯ maka nɯ-mɲaʁ tu-ɣɤ-mtɤm cʰa ri, \\
\textsc{dem} \textsc{erg} completely \textsc{3pl}.\textsc{poss}-eye \textsc{ipfv}-\textsc{caus}-have.sharp.eyesight[III] can:\textsc{fact} \textsc{lnk} \\
\glt `That [plant] can cure their blindness (cause their eyes to recover eyesight).' (140517 mogui de jing-zh)
\japhdoi{0004022\#S81}
\end{exe} 

\subsection{Compatibilities with other derivations} \label{sec:velar.caus.other}
\is{causative!compatibility} 
Due to the constraint on monosyllabic bases (§\ref{sec:sig.caus.stative}), the velar causative can only take bare verb roots (or at least verb roots containing frozen prefixes) as input, unlike the sigmatic causative which can precede a dozen derivational prefixes (§\ref{sec:sig.caus.other.derivations}).

The velar causative prefix can be preceded by the reflexive \forme{ʑɣɤ-} as in (\ref{ex:YWZGAGAxtCindZi}) (see also \ref{ex:atAtWGAZo.AtAZGAGAmbjom}, §\ref{sec:refl.caus.volitional} and §\ref{sec:refl.trop.caus}).

\begin{exe}
\ex \label{ex:YWZGAGAxtCindZi}
\gll  iɕqʰa tɯ-xtsa nɯni tɤ-kɤ-sŋaʁ pjɤ-ɕti tɕe tɕendɤre, ɲɤ-ʑɣɤ-ɣɤ-xtɕi-ndʑi \\
the.aforementioned \textsc{indef}.\textsc{poss}-shoe \textsc{dem}:\textsc{du} \textsc{aor}-\textsc{obj}:\textsc{pcp}-enchant \textsc{ifr}.\textsc{ipfv}-be.\textsc{aff} \textsc{lnk} \textsc{lnk} \textsc{ifr}-\textsc{refl}-\textsc{caus}-be.small-\textsc{du} \\
\glt `The shoes had been enchanted, and became small by themselves.' (160706 poucet6)
\japhdoi{0006109\#S92}
\end{exe} 

It can also undergo reduplicated reciprocal derivation (\ref{ex:chAGArlWrlaRnW}, §\ref{sec:reciprocal.other}),  and occurs with the autive \forme{nɯ-} (\ref{ex:mWpjAnWGAwxti} in §\ref{sec:velar.causative.tropative} above) and the \forme{z-} allomorph of the sigmatic causative (§\ref{sec:caus.z}) as in (\ref{ex:YAtWzGABdit}).
 
\begin{exe}
\ex \label{ex:YAtWzGABdit}
\gll  tɕe tʂu ri ɲɤ-tɯ-z-ɣɤ-βdi-t ɕti tɕe, tɕe nɯtɕu a-kɤ-tɯ-ɕe \\
\textsc{lnk} road also \textsc{ifr}-2-\textsc{caus}-\textsc{caus}-be.well-\textsc{pst}:\textsc{tr} be.\textsc{aff}:\textsc{fact} \textsc{lnk} \textsc{lnk} \textsc{dem}:\textsc{loc} \textsc{irr}-\textsc{pfv}:\textsc{east}-2-go \\
\glt `[Since] you have already had the road repaired (by someone else), [take that road] to go [east].' (2011-04-smanmi)
 \end{exe} 
 
 The tropative causative \japhug{ɣɤkʰe}{depreciate, demean} (derived from \japhug{kʰe}{be stupid}, §\ref{sec:velar.causative.tropative}) also has the \forme{sɤ-} antipassive form (§\ref{sec:antipassive.sA}) \forme{sɤz-ɣɤ-kʰe} (\textsc{apass}-\textsc{caus}-be.stu\-pid) `demean people'.


\section{Applicative} \label{sec:applicative}
  \is{applicative}
The applicative \forme{nɯ-} is a valency-increasing derivation by means of which an oblique argument, an adjunct or even a non-participant (including comitative or dative adjuncts and semi-objects, §\ref{sec:applicative.promoted}) is promoted to object function. In Japhug, the base verb is always morphologically intransitive, and the applicative verb transitive (§\ref{sec:transitivity.morphology}). The subject of the applicative verb corresponds to the same referent as that of the base verb, though it receives ergative case marking instead of absolutive (see examples \ref{ex:YWkWnWGmu} and \ref{ex:YWmunW.chWphGonW} in §\ref{sec:applicative.promoted} below).

The \forme{nɯ-} applicative is only attested by a limited number of examples (exhaustively listed in \tabref{tab:applicative}; the allomorphy is discussed in §\ref{sec:allomorphy.applicative}), but the fact that it includes the Tibetan loanword \japhug{rga}{like}, `be happy' (from \tibet{དགའ་}{dga}{be happy}) shows that it has some degree of productivity. 

Cognates of the applicative prefix are found in other Gyalrong languages (in Tshobdun, see \citealt{jackson06paisheng}), and one potential example is found in Khroskyabs (\citealt[361]{lai17khroskyabs}). This prefix probably originates from the denominal \forme{nɯ-} (§\ref{sec:denom.tr.nW}), and replaced the older suffixal applicative, which only remains in a handful of examples (§\ref
{sec:applicative})

\begin{table}
\caption{Examples of the \forme{nɯ-} applicative prefix}\label{tab:applicative} 
\begin{tabular}{lllllllll} 
\lsptoprule
Base verb  & Derived  verb &\\
\midrule
\japhug{aʑɯʑu}{wrestle}	& \japhug{nɤʑɯʑu}{wrestle with} \\
\japhug{akʰu}{call} &\japhug{nɤkʰu}{invite}  \\
\japhug{akʰɤzŋga}{call}, `shout' & \japhug{nɤkʰɤzŋga}{shout at}  \\
\japhug{andzɯt}{bark} & \japhug{nɤndzɯt}{bark at}  \\
\japhug{amdzɯ}{sit} & \japhug{nɤmdzɯ}{look after}  \\
\japhug{aɣro}{play} & \japhug{nɤɣro}{play with}  \\
\japhug{stu}{believe} (vi) &\japhug{nɤstu}{believe} (vt)  \\
\midrule
\japhug{mbɣom}{be in a hurry} & \japhug{nɯmbɣom}{look forward to}  \\
\japhug{ŋke}{walk} & \japhug{nɯŋke}{look for}  \\
\japhug{rga}{like} (vi) & \japhug{nɯrga}{like} (vt)  \\
\japhug{sŋom}{envy} (vi) & \japhug{nɯsŋom}{envy} (vt)  \\
 \japhug{zdɯɣ}{suffer} & \japhug{nɯzdɯɣ}{worry about} \\
\midrule
\japhug{bɯɣ}{miss home} (vi) & \japhug{nɯɣbɯɣ}{miss} (vt)  \\
\japhug{mu}{be afraid} & \japhug{nɯɣmu}{be afraid of}  \\
\lspbottomrule
\end{tabular}
\end{table}

In addition to these examples, the \forme{nɤ-} deideophonic verbs can be analyzed as applicative derivations from their \forme{a-} deideophonic counterpart (§\ref{sec:a.nA.deidph}).

\subsection{The syntactic and semantic functions of the promoted argument} \label{sec:applicative.promoted}

Despite the limited number of applicative verbs, there is a considerable diversity in the syntactic functions of the non-core arguments (of the base verbs) that are promoted to object status by the applicative derivation.

\subsubsection{Promotion of comitative argument} \label{sec:applicative.comitative}
 \is{promotion to object function}
The verb \japhug{aʑɯʑu}{wrestle}, historically a reciprocal verb (§\ref{sec:redp.lexicalized}), requires a non-singular subject and can select a comitative argument in \forme{cʰo} (§\ref{sec:comitative}). The object of the applicative form \japhug{nɤʑɯʑu}{wrestle with} corresponds to this comitative argument, as shown by the minimal pair (\ref{ex:YAZWZundZi}) vs. (\ref{ex:YAznAZWZu}).

\begin{exe}
\ex \label{ex:YAZWZundZi}
\gll ɯ-zda cʰo ɲɯ-ɤʑɯʑu-ndʑi \\
\textsc{3sg}.\textsc{poss}-companion \textsc{comit} \textsc{sens}-wrestle-\textsc{du}  \\
\glt `He is wrestling with his friend.' (elicited)
\end{exe}

\begin{exe}
\ex \label{ex:YAznAZWZu}
\gll ɯ-zda ɲɯ-ɤz-nɯ-ɤʑɯʑu \\
\textsc{3sg}.\textsc{poss}-companion \textsc{sens}-\textsc{prog}-\textsc{appl}-wrestle \\
\glt `He is wrestling his friend.' (elicited)
\end{exe}

\subsubsection{Promotion of semi-object (stimulus)} \label{sec:applicative.semi.object}
\is{promotion to object function} 
In the case of the semi-transitive \japhug{rga}{like}, the argument added by the applicative is the semi-object (§\ref{sec:semi.transitive}). The base verb \forme{rga} and its applicative form \japhug{nɯrga}{like} are in some contexts semantically identical, for instance in the pseudo-clefts  (§\ref{sec:pseudo.cleft}) \forme{stu ji-kɤ-rga} and \forme{stu ji-kɤ-nɯ-rga} in (\ref{ex:stu.jikArga.Cti}) and (\ref{ex:stu.jikAnWrga.Cti}) which both mean `the one that we like most' (both objects and semi-objects can be relativized with the object participle §\ref{sec:object.participle.relatives}, and the possessive prefix in both cases refers to the subject §\ref{sec:object.participle.possessive}).


\begin{exe}
\ex \label{ex:stu.jikArga.Cti}
\gll tɕe iʑo kɯrɯ ra tɕe tʂʰɤlu nɯ stu ji-kɤ-rga ɕti \\
\textsc{lnk} \textsc{1pl} Tibetan \textsc{pl} \textsc{lnk} milk.tea \textsc{dem} \textsc{most} \textsc{1sg}.\textsc{poss}-\textsc{obj}:\textsc{pcp}-like be.\textsc{aff}:\textsc{fact} \\
\glt `Milk tea is [the type of tea] that we Tibetans like best.' (05-qaZo)
\japhdoi{0003404\#S152}
\end{exe}

\begin{exe}
\ex \label{ex:stu.jikAnWrga.Cti}
\gll tʰaχtsa nɯ iʑo kɯrɯ tɕʰeme ra ɣɯ, nɯnɯ mɤlɤn ʑo pjɯ-tu kɯ-ra tɕe, stu ji-kɤ-nɯ-rga ɕti, \\
coloured.belt \textsc{dem} \textsc{1pl} Tibetan woman \textsc{pl} \textsc{gen} \textsc{dem} absolutely \textsc{emph} \textsc{ipfv}-exist \textsc{sbj}:\textsc{pcp}-be.needed \textsc{lnk} most \textsc{1pl}.\textsc{poss}-\textsc{obj}:\textsc{pcp}-\textsc{appl}-like be.\textsc{aff}:\textsc{fact} \\
\glt `Coloured belts are something that we Tibetan woman must absolutely have, and it is what we like best.' (thaXtsa 2002)
\end{exe}

In addition to morphological (§\ref{sec:transitivity.morphology}) differences, as well as absolutive vs. ergative marking of the subject (§\ref{sec:A.kW}), the base verb \forme{rga} and its applicative \forme{nɯrga} differ from each other in three regards.
 

First of all, \forme{rga} displays lability between a semi-transitive use `like' and a stative intransitive use meaning `be happy' (compare \ref{ex:cha.pjArga} and \ref{ex:wuma.Zo.pjArgandZi} in §\ref{sec:semi.tr.labile}), while the applicative \forme{nɯrga} does not mean `be happy because/for'.

Second, whereas the base verb \forme{rga} can take infinitival complement clauses (§\ref{sec:inf.complementation}, §\ref{sec:velar.inf.coreference}), as in (\ref{ex:kAnArtoXpjAt.pWrgaa2}) (see also \ref{ex:tApAtso.pWkWNu}, §\ref{sec:genr.3pl}), its applicative form \forme{nɯ-rga} cannot: the applicative derivation thus \textit{removes} com\-ple\-ment-taking ability (at least in this case).

\begin{exe}
\ex \label{ex:kAnArtoXpjAt.pWrgaa2}
\gll ma aʑo [qajɯ nɯra kɤ-nɤrtoχpjɤt] pɯ-rga-a tɕe  \\
\textsc{lnk} \textsc{1sg} bug \textsc{dem}:\textsc{pl} \textsc{inf}-observe \textsc{pst}.\textsc{ipfv}-like-\textsc{1sg} \textsc{lnk} \\
\glt `I liked to observe bugs.' (26-quspunmbro)
\japhdoi{0003684\#S15}
\end{exe}

Third, with first or second person objects, only the applicative \forme{nɯrga} is possible. The local scenario 1\fl{}2 (\ref{ex:YWtanWrga}) and 2\fl{}1 configurations (example \ref{ex:nAstu.Zo.WYWkWnWrgaa}, §\ref{sec.IPN.adverbs}), as well as the mixed scenario inverse configurations (\ref{ex:apWwGnWrgaa}) cannot be expressed with the base verb \forme{rga}. 

\begin{exe}
\ex \label{ex:YWtanWrga}
\gll ɲɯ-ta-nɤ-pe ɕti qʰe, aʑo ɲɯ-ta-nɯ-rga \\
\textsc{sens}-1\fl{}2-\textsc{trop}-be.good be.\textsc{aff}:\textsc{fact} \textsc{lnk} \textsc{1sg} \textsc{sens}-1\fl{}2-\textsc{appl}-like \\
\glt `I like you, I love you.' (160630 abao-zh)
\japhdoi{0006197\#S108}
\end{exe}

\begin{exe}
\ex \label{ex:apWwGnWrgaa}
\gll nɯnɯ rɟɤlpu ɯ-tɕɯ nɯ kɯ a-pɯ́-wɣ-nɯ-rga-a ra \\
\textsc{dem} king \textsc{3sg}.\textsc{poss}-son \textsc{dem} \textsc{erg} \textsc{irr}-\textsc{ipfv}-\textsc{inv}-\textsc{appl}-like-\textsc{1sg} be.needed:\textsc{fact} \\
\glt `May the prince love me!' (150819 haidenver-zh)
\japhdoi{0006314\#S497}
\end{exe}

\subsubsection{Promotion of dative argument} \label{sec:applicative.dative}
 \is{promotion to object function}
The verbs \japhug{andzɯt}{bark} and \japhug{akʰɤzŋga}{call} optionally select a dative argument (\forme{tɯrme mɤ-kɤ-nɯfse nɯ} `the person that it does not know' in \ref{ex:WCki.YAndzWt}), which is promoted to object status by the applicative (\ref{ex:YAznAndzWt}).

\begin{exe}
\ex \label{ex:WCki.YAndzWt}
\gll tɯrme mɤ-kɤ-nɯfse nɯ ɯ-ɕki ɲɯ-ɤndzɯt \\
person \textsc{neg}-\textsc{obj}:\textsc{pcp}-know \textsc{dem} \textsc{3sg}.\textsc{poss}-\textsc{dat} \textsc{sens}-bark \\
\glt `[The dog] is barking at the unknown person.' (elicited)
\end{exe}

\begin{exe}
\ex \label{ex:YAznAndzWt}
\gll tɯrme mɤ-kɤ-nɯfse nɯ ɲɯ-ɤz-nɯ-ɤndzɯt \\
person \textsc{neg}-\textsc{obj}:\textsc{pcp}-know \textsc{dem} \textsc{sens}-\textsc{prog}-\textsc{appl}-bark \\
\glt `[The dog] is barking at the unknown person.' (elicited)
\end{exe}

\subsubsection{Introduction of new referent}
For most applicative verbs, the object corresponds to an entirely new referent, without equivalent in the argument structure of the base verb. Based on the semantic role of the added argument, four sub-cases can be distinguished.

The semi-transitive verb \japhug{stu}{believe}, like \japhug{rga}{like}, is semi-transitive (§\ref{sec:semi.transitive}). Its semi-object is either a complement clause or a noun such as \forme{ɯ-rju} `his words', expressing the content of the utterance that is believed by the subject. Its (morphologically irregular, §\ref{sec:allomorphy.applicative}) applicative \forme{nɤstu} takes as object the person uttering the words that are believed by the subject, as in (\ref{ex:WYWkWnAstua}), which is not expressed as an argument of the base verb (with \japhug{stu}{believe} the only way to express the person saying the words that are believed is as possessor of the semi-object).

\begin{exe}
\ex \label{ex:WYWkWnAstua}
\gll nɯnɯ aʑo ɯ-ɲɯ-kɯ-nɤ-stu-a nɤ, tɕendɤre, nɯɕimɯma ʑo ɕe-tɕi tɕe, \\
\textsc{dem} \textsc{1sg} \textsc{qu}-\textsc{sens}-2\fl{}1-\textsc{appl}-believe-\textsc{1sg} \textsc{add} \textsc{lnk} immediately \textsc{emph} go:\textsc{fact}-\textsc{1du} \textsc{lnk} \\
\glt `If you believe me, let us go [there] immediately.' (140425 shizi huli he lu-zh)
\japhdoi{0003794\#S41}
\end{exe}

Examples like (\ref{ex:manWtWnAste}) with the relative clause \forme{ɯʑo kɯ ta-tɯt} `(the words) that he said' appearing before the verb could seem to imply that \forme{nɤstu} can also select as object the words uttered. If this analysis were correct, the applicative \forme{nɤstu} would be similar to \japhug{nɯrga}{like} above in promoting the semi-object of its base verb as object. However, if the relative clause takes a second person subject as in (\ref{ex:tAtWtWt.nW.mAstua}), it is possible either to use the base verb \japhug{stu}{believe} or the applicative \forme{nɤstu} with a second person object (\forme{mɤ-ta-nɤstu} `I don't believe you'), but not with a third person object, showing that in (\ref{ex:manWtWnAste}) the relative clause is not the object, but an adjunct (the object of \forme{nɤstu} in this example is the referent corresponding to the subject of the relative clause).

\begin{exe}
\ex \label{ex:manWtWnAste}
\gll [ɯʑo kɯ ta-tɯt] nɯ ma-nɯ-tɯ-nɤ-ste \\
\textsc{1sg} \textsc{erg} \textsc{aor}:3\flobv{}-say[II] \textsc{dem} \textsc{neg}-\textsc{imp}-2-\textsc{appl}-believe[III] \\
\glt `Don't believe what he said.' (elicited)
\end{exe}


\begin{exe}
\ex \label{ex:tAtWtWt.nW.mAstua}
\gll [nɤʑo tɤ-tɯ-tɯt] nɯ mɤ-stu-a \\
\textsc{2sg} \textsc{aor}-2-say[II] \textsc{dem} \textsc{neg}-believe:\textsc{fact}-\textsc{1sg} \\
\glt `I don't believe what you said.' (elicited)
\end{exe}

Other intransitive verbs with experiencer subject, such as \japhug{bɯɣ}{miss home}, \japhug{mu}{be afraid} and \japhug{sŋom}{envy}, have applicative forms that promote the stimulus as object. The base verb forms cannot select a noun or a complement clause to specify this stimulus, even as adjunct; the stimulus can only be indirectly expressed in a separate clause. In (\ref{ex:YWmunW.chWphGonW}) for instance, the clause \forme{tɤ-tɕɯ ... tu-ɣɤɕqali-nɯ} `the men ... shout' describes the reason for the fear of the animals; though it could be analyzed as the stimulus of the verb \forme{ɲɯ-mu-nɯ}, the relationship between the two clauses is simply one of temporality/causation, and the verb \japhug{mu}{be afraid} is unable to take any overt stimulus.

\begin{exe}
\ex \label{ex:YWmunW.chWphGonW}
\gll  tɤ-tɕɯ tʰamtɕɤt kɯ-dɯ\redp{}dɤn nɯ ɣɯrnɤɣɯr ʑo tu-ɣɤɕqali-nɯ tɕe, tɕendɤre rɯdaʁ nɯra ɲɯ-mu-nɯ tɕe cʰɯ-pʰɣo-nɯ ɲɯ-ŋu. \\
\textsc{indef}.\textsc{poss}-boy all \textsc{sbj}:\textsc{pcp}-be.many \textsc{dem} \textsc{idph}(III):noisy.and.crowded \textsc{emph} \textsc{ipfv}-shout-\textsc{pl} \textsc{lnk} \textsc{lnk} wild.animal \textsc{dem}:\textsc{pl} \textsc{ipfv}-be.afraid-\textsc{pl} \textsc{lnk} \textsc{ipfv}:\textsc{downstream}-flee-\textsc{pl} \textsc{sens}-be \\
\glt `The men (i.e. the hunters) all shout together in great numbers, and the wild animals being afraid flee downstream.' (150829 KAGWcAno)
\japhdoi{0006420\#S9}
\end{exe}


The corresponding applicative verb \japhug{nɯɣmu}{be afraid of} marks the experiencer in the ergative, and takes the stimulus (in \ref{ex:YWkWnWGmu}, \japhug{tɯrme}{person}) as object (note the generic object indexation, §\ref{sec:indexation.generic.tr}). The referent promoted as object by the applicative derivation in this verb is the same as that promoted to subject status by the proprietive \forme{sɤ(ɣ)-} (see \ref{ex:pjAsAGmu.pjAnWGmu}, §\ref{sec:proprietive}).

\begin{exe}
\ex \label{ex:YWkWnWGmu}
\gll nɯnɯ kɯ tɯrme wuma ʑo ɲɯ-kɯ-nɯɣ-mu. \\
\textsc{dem} \textsc{erg} people really \textsc{emph} \textsc{ipfv}-\textsc{genr}:S/O-\textsc{appl}-be.afraid \\
\glt `It is very afraid of people.' (24-ZmbrWpGa)
\japhdoi{0003628\#S26}
\end{exe}

The verb \japhug{aɣro}{play} (almost always attested as \forme{a<nɯ>ɣro} with infixation of the autive, §\ref{sec:autoben.position}), rarely occurs in the singular, and can select comitative arguments (§\ref{sec:comitative}), as in (\ref{ex:cho.YAnWGrondZi}).

\begin{exe}
\ex \label{ex:cho.YAnWGrondZi}
\gll <xiaocui> nɯ tɤ-rɯstɯnmɯ ɯ-qʰu qʰe tɕe iɕqʰa [<yuanfeng> nɯ, taʁndo mɤ-kɯ-tso nɯ cʰo] sɲikuku ʑo tɯtɯrca ɲɯ-ɤ<nɯ>ɣro-ndʑi qʰe \\
\textsc{anthr} \textsc{dem} \textsc{aor}-marry \textsc{3sg}.\textsc{poss}-after \textsc{lnk} \textsc{lnk} the.aforementioned  \textsc{anthr} \textsc{dem} speech \textsc{neg}-\textsc{sbj}:\textsc{pcp}-understand \textsc{dem} \textsc{comit} every.day \textsc{emph} together \textsc{ipfv}-<\textsc{auto}>play-\textsc{du} \textsc{lnk} \\
\glt `After she married, Xiaocui played every day with Yuanfeng, who did not understand speech.' (150909 xiaocui-zh)
\japhdoi{0006386\#S60}
\end{exe}

 However, unlike \japhug{aʑɯʑu}{wrestle} above, its applicative \japhug{nɤɣro}{play with} does not promote the comitative argument (the person one plays with) to object status. Rather, it selects the instrument (the toy one plays with) as object. In (\ref{ex:YWnAGrondZi}), the object of \forme{nɤɣro} is not overt in the same clause, but anaphorically refers to \japhug{kɯmtɕʰɯ}{toy} in the previous clause.

\begin{exe}
\ex \label{ex:YWnAGrondZi}
\gll  mɤʑɯ kɯmtɕʰɯ kɯ-fse tɤ-tu nɤ, tɯɣrɤz ɲɯ-nɯ-ɤɣro-ndʑi nɯra pjɤ-ŋgrɤl,\\
again toy \textsc{sbj}:\textsc{pcp}-be.like \textsc{aor}-exist \textsc{add} together \textsc{ipfv}-\textsc{appl}-play-\textsc{du} \textsc{dem}:\textsc{pl} \textsc{ipfv}.\textsc{ifr}-be.usually.the.case \\
\glt `Whenever there was a toy$_i$, they(\textsc{du}) played with it$_i$ together.' (lWlu)
\end{exe}

\subsubsection{A problematic case} \label{sec:nWzdWG}
The relationship between the transitive verb  \japhug{nɯzdɯɣ}{worry about} and the base verb \japhug{zdɯɣ}{suffer} (from \tibet{སྡུག་}{sdug}{suffering}) is slightly different from the preceding cases, and it is disputable whether this verb is to be classified as applicative. The intransitive verb \forme{zdɯɣ} has two different meanings: `be sad' when used with the nouns \japhug{tɯ-sɯm}{mind} or \japhug{tɯ-sni}{heart} as subjects, and the experiencer as possessor as in (\ref{ex:Wsni.YAzdWG}), and `endure hardship' when taking a human (or non-human animal) subject (\ref{ex:pjAzdWGndZi}). The meaning `worry' of the verb \forme{nɯzdɯɣ} is not directly derivable from either. It is close to `to be sad about', but the transitive subject of \forme{nɯzdɯɣ} corresponds to the possessor of the subject of \forme{zdɯɣ}, making it a unique type of derivation.

\begin{exe}
\ex \label{ex:Wsni.YAzdWG}
\gll  tɕʰemɤpɯ nɯ ɯ-sni ɲɤ-zdɯɣ \\
little.girl \textsc{dem} \textsc{3sg}.\textsc{poss}-heart \textsc{ifr}-suffer \\
\glt `The little girl was very sad.' (140504 huiguniang-zh)
\japhdoi{0003909\#S96}
\end{exe}

\begin{exe}
\ex \label{ex:pjAzdWGndZi}
\gll  laʁnɯ-sŋi pjɤ-zdɯɣ-ndʑi tɕe, \\
one.or.two-day \textsc{ifr}.\textsc{ipfv}-suffer \textsc{lnk} \\
\glt `They had [worked] hard for several days.' (qajdoskAt 2002) \japhdoi{0003366\#S18}
\end{exe}


\subsection{Allomorphy} \label{sec:allomorphy.applicative}
\is{applicative!allomorphy}\is{morphology!allomorphy} \is{allomorphy!applicative}
The applicative prefix has three regular allomorphs: \forme{nɯ\trt}, \forme{nɯɣ-} and \forme{nɤ-}.\footnote{A similar allomorphy is found in other Gyalrong languages, see \citet{jackson06paisheng} on Tshobdun.} The allomorph \forme{nɯɣ-} has the same distribution as the \forme{sɯɣ-} allomorph of the causative (§\ref{sec:caus.sWG}), occurring with monosyllabic intransitive verb roots whose onset does not contain a cluster and/or a velar consonant. The allomorph \forme{nɤ-} (homophonous with the tropative, §\ref{sec:tropative.allomorphy}) is due to vowel fusion with the contracting \forme{a-} in some verb stems (§\ref{sec:contraction}; the de-contracted form \forme{nɯ-ɤ-} is used in some of the discussion in this section), with the exception of the verb \japhug{stu}{believe} whose applicative is irregular. 

The allomorph \forme{nɯ-} occurs in all other contexts. It is homophonous with the autive (§\ref{sec:autobenefactive}), the vertitive (§\ref{sec:vertitive}) and various other prefixes, including the \textsc{westwards} orientation preverbs (§\ref{sec:kamnyu.preverbs}) and the denominal \forme{nɯ-} (§\ref{sec:denom.nW}), but ambiguity is rare as these prefixes belong to different slots (§\ref{sec:prefixal.chain}).\footnote{For an example of partial ambiguity between autive and applicative, see examples (\ref{ex:CtAnWNke}) and (\ref{ex:tukWnWNke}) in §\ref{sec:applicative.lexicalized}. } 


\subsection{Lexicalized applicatives} \label{sec:applicative.lexicalized} 
Three of the applicative verbs in \tabref{tab:applicative} are highly lexicalized and cannot be considered to be synchronically analyzable as related to their base verbs.
 
First, the transitive verb \japhug{nɤkʰu}{invite} (to one's home as a guest, see examples in §\ref{sec:bare.inf.coreference} and §\ref{sec:object.participles.complement}) originates from the applicative \forme{nɯ-ɤkʰu} of \japhug{akʰu}{call}, meaning `call (someone), shout at'. The verb \forme{akʰu} can select a locative goal (referring to the direction where one calls), as in (\ref{ex:CtokAkhuci}), and its applicative \forme{nɯ-ɤkʰu} presumably originally promoted this oblique argument to object status.

\begin{exe}
\ex \label{ex:CtokAkhuci}
\gll kʰa nɯtɕu ɕ-to-k-ɤkʰu-ci \\
house \textsc{dem}:\textsc{loc} \textsc{tral}-\textsc{ifr}-\textsc{peg}-call-\textsc{peg} \\
\glt `He went and called in the direction of the house.' (2011-05-nyima)
\end{exe}

At that earlier stage, \forme{nɯ-ɤkʰu} probably used to have a meaning similar to that of \forme{nɯ-ɤkʰɤzŋga} `shout at', the applicative of \japhug{akʰɤzŋga}{call}, `shout' (a verb related to and synonymous with \japhug{akʰu}{call}). As shown by (\ref{ex:tuwGnAkhAzNganW}), the applicative \japhug{nɤkʰɤzŋga}{shout at} selects the goal/addressee as object (the plural marking on the verb form which here indexes the object being due to the presence of the inverse prefix, see 3$'$\fl{}\textsc{3pl} in §\ref{sec:indexation.non.local}). 

\begin{exe}
\ex \label{ex:tuwGnAkhAzNganW}
\gll  tɕʰeme ɯ-skɤt kɯ-snɯ\redp{}sna ci kɯ ʑo tú-wɣ-nɯ-ɤkʰɤzŋga-nɯ ntsɯ. \\
girl \textsc{3sg}.\textsc{poss}-voice \textsc{sbj}:\textsc{pcp}-\textsc{emph}\redp{}nice \textsc{indef} \textsc{erg} \textsc{emph} \textsc{ipfv}-\textsc{inv}-\textsc{appl}-call-\textsc{pl} always \\
\glt `A girl who had a beautiful voice was calling them.' (2003kandZislama)
\end{exe}

The applicative \forme{nɯ-ɤkʰu} however underwent the semantic change `call $X$' $\Rightarrow$ `call $X_i$ to invite him/her$_i$ to come as guest' $\Rightarrow$ `invite $X$ to come as guest', so that its etymological relationship with its base verb \japhug{akʰu}{call} is not synchronically obvious anymore.


Second,  \japhug{nɯŋke}{look for} is formally an applicative of the atelic motion verb \japhug{ŋke}{walk} (§\ref{sec:motion.verbs}). The promoted object corresponds to the aim of the motion (the entity that the subject is searching), as in (\ref{ex:CtAnWNke}).

\begin{exe}
\ex \label{ex:CtAnWNke}
\gll   ji-<baogao> tɤ-lɤt, <piaozi> ɕ-tɤ-nɯŋke, <dianzhan> βzu-j ŋu \\
\textsc{1pl}.\textsc{poss}-report \textsc{imp}-release money \textsc{tral}-\textsc{imp}-look.for electric.station make:\textsc{fact}-\textsc{1pl} be:\textsc{fact} \\
\glt `Make a report for us, go and look for money, and we will build an electrical station.' (2010-09)
\end{exe}

This verb should not be confused with the regular autive \forme{nɯ-ŋke} of the verb \japhug{ŋke}{walk}, which is intransitive, as shown by the generic \forme{kɯ-} in (\ref{ex:tukWnWNke}) (§\ref{sec:intr.23}).

\begin{exe}
\ex \label{ex:tukWnWNke}
\gll tɯ-ji ɯ-ŋgɯ aʁɤndɯndɤt tu-kɯ-nɯ-ŋke kʰɯ tɕe \\
\textsc{indef}.\textsc{poss}-field \textsc{3sg}.\textsc{poss}-in everywhere \textsc{ipfv}-\textsc{genr}:S/O-\textsc{auto}-walk be.possible:\textsc{fact} \textsc{lnk} \\
\glt `One can walk [on this path] everywhere in the fields [as one wishes].' (15-06-05)
\end{exe}

Third, the transitive verb \japhug{nɤmdzɯ}{look after} historically derives from the intransitive \japhug{amdzɯ}{sit} by the applicative derivation. Its original meaning probably was `sit by', with a locative adjunct promoted to object status, but in Japhug it rather means `stay near $X$ and look after $X$' without necessary implication of remaining seated, as in (\ref{ex:kuWGnAmdzW.YWra}).

\begin{exe}
\ex \label{ex:kuWGnAmdzW.YWra}
\gll tɤ-lu koŋla ʑo kú-wɣ-nɤmdzɯ ɲɯ-ra ma ʁlɤwur ʑo tu-mbuz ɲɯ-ŋu \\
\textsc{indef}.\textsc{poss}-milk completely \textsc{emph} \textsc{ipfv}-\textsc{inv}-look.after \textsc{sens}-be.needed \textsc{lnk} immediately \textsc{emph} \textsc{ipfv}-spill.out \textsc{sens}-be \\
\glt `One has to look after the milk [in the pan], otherwise it will overflow as soon as [it boils].' (elicited)
\end{exe}
 
 
\subsection{Applicatives and other derivations} \label{sec:appl.other.derivations}
\is{applicative!compatibility}
The applicative derivation cannot take any other derived verb form as input. The only partial exception is the applicative \japhug{nɤʑɯʑu}{wrestle with}, whose base form \japhug{aʑɯʑu}{wrestle} historically was a reciprocal verb (§\ref{sec:redp.lexicalized}), but since it is an already fossilized reciprocal, it does not demonstrative the ability of the applicative to apply to reciprocal forms in general.

On the other hand, applicative verb forms can be subjected to further derivations, including the reflexive \forme{ʑɣɤ-} (\forme{ʑɣɤ-nɤ-stu} \textsc{refl}-\textsc{appl}-believe `believe in oneself', see example \ref{ex:mWYAZGAnAstu}, §\ref{sec:refl.tropative}) and the sigmatic causative (\ref{ex:YWznWsNomanW}).\footnote{In this example, the \textsc{1sg} is both causer (indexed as transitive subject) and patient, while the causee (`the people') is indexed as object. }. 

\begin{exe}
\ex \label{ex:YWznWsNomanW}
\gll tɯrme ra, ʑimkʰɤm ʑo, ɲɯ-z-nɯ-sŋom-a-nɯ cʰa-a \\
people \textsc{pl} many \textsc{emph} \textsc{ipfv}-\textsc{caus}-\textsc{appl}-envy-\textsc{1sg}-\textsc{pl} can:\textsc{fact}-\textsc{1sg} \\
\glt `I will be able to make a lot of people envy [me for my dress].' (jinai de guniang-zh)
\end{exe}

The problematic verb  \japhug{nɯzdɯɣ}{worry about}  (§\ref{sec:nWzdWG}) can in addition serve as input to the proprietive derivation (§\ref{sec:proprietive.compatibility}).

\section{Tropative} \label{sec:tropative}
  \is{tropative}
The tropative\footnote{This term is taken from Arabic linguistics, see for instance \citet{larcher96}. Another possible term for this derivation would be `estimative'. } \forme{nɤ-} prefix is a valency-increasing derivation creating a transitive verb meaning `find/consider to be $X$' out of an intransitive stative verb.\footnote{Cognate prefixes are found in other Gyalrong languages, (\citealt[5--6]{jackson06paisheng}, \citealt{jacques13tropative}).} 
 
The tropative resembles the causative derivations in that the subject of the base verb becomes the object of the derived verb, and the added argument is the transitive subject (unlike the applicative). However, the semantic role of the transitive subject is not an agent (causer), but an experiencer.

For instance, the tropative of the stative verb \japhug{mpɕɤr}{be beautiful} is the transitive \japhug{nɤmpɕɤr}{find beautiful}, whose object is the entity considered to be beautiful (\forme{iɕqʰa tɕʰeme} `the girl' in \ref{ex:YAnAmpCAr.YAnApe}), and whose transitive subject in the ergative is the person feeling the beauty of the object (\forme{tɤru ɯ-tɕɯ} `the prince' in \ref{ex:YAnAmpCAr.YAnApe}).


\begin{exe}
\ex \label{ex:YAnAmpCAr.YAnApe}
\gll tɕe nɯ tɤru ɯ-tɕɯ nɯ kɯ nɯɕimɯma ʑo iɕqʰa tɕʰeme nɯ ɲɤ-nɤ-mpɕɤr,
tɕe ɲɤ-nɤ-pe. \\
\textsc{lnk} \textsc{dem} chieftain \textsc{3sg}.\textsc{poss}-son \textsc{dem} \textsc{erg} immediately \textsc{emph} the.aforementioned girl \textsc{dem} \textsc{ifr}-\textsc{trop}-be.beautiful \textsc{lnk} \textsc{ifr}-\textsc{trop}-be.good \\
\glt `The prince immediately found the girl very beautiful and fell in love with her.' (140518 huifei de muma-zh)
\japhdoi{0004026\#S51}
\end{exe}

The tropative derivation is extremely productive, and \tabref{tab:tropative} illustrates a few representative examples.

\begin{table}
\caption{Examples of the \forme{nɤ-} tropative derivation}\label{tab:tropative}
\begin{tabular}{lllllllll}
\lsptoprule 
basic verb  & derived  verb &\\
\midrule
\japhug{rtaʁ}{be enough} & \japhug{nɤrtaʁ}{find sufficient} \\
\japhug{wxti}{be big} & \japhug{nɤwxti}{find big} \\
\japhug{zri}{be long} & \japhug{nɤzri}{find long} \\
\japhug{pe}{be good} & \japhug{nɤpe}{consider to be good, like} \\
\japhug{mnɤm}{smell} (vi) & \japhug{nɤmnɤm}{smell} (vt) \\
       \midrule
\japhug{cʰi}{be sweet} & \japhug{nɤxcʰi}{find sweet} \\
\japhug{maʁ}{not be} & \japhug{nɤɣmaʁ}{consider wrong} \\
\japhug{mbat}{be easy} & \japhug{nɤɣmbat}{finish easily} \\
\lspbottomrule
\end{tabular}
\end{table}

The semantics of the derived verb is not always simply `consider/find $X$'. In the case of stative verbs whose meaning is neutral (not explicitly positive like `beautiful'), the tropative often has the additional meaning `find too $X$', as in (\ref{ex:WYWtWnAxtCi}) for instance.
\largerpage
\begin{exe}
\ex \label{ex:WYWtWnAxtCi}
\gll nɤ-sɤtɕʰa ɯ-ɲɯ-tɯ-nɤ-xtɕi nɤ, aʑo a-βlu ci tu tɕe, \\
\textsc{2sg}.\textsc{poss}-place \textsc{qu}-\textsc{sens}-2-\textsc{trop}-be.small \textsc{add} \textsc{1sg} \textsc{1sg}.\textsc{poss}-trick \textsc{indef} exist:\textsc{fact} \textsc{lnk} \\
\glt `If you find this place too small for you, I have an idea.' (150829 taishan zhi zhu-zh,203)
\japhdoi{0006350\#S196}
\end{exe}

The tropative \forme{nɤ-} cannot be prefixed to non-adjectival stative verbs like copulas (\japhug{ŋu}{be}, \japhug{ɕti}{be}) or existential verbs (\japhug{tu}{exist}, \japhug{me}{not exist}). A meaning such as `consider $X$ to be $Y$', where both $X$ and $Y$ are nouns, cannot be expressed in Japhug using a tropative derivation (forms such as  $\dagger$\forme{nɤ-ŋu} or $\dagger$\forme{nɤ-tu} are utterly incorrect), and a synthetic construction must be used instead (§\ref{sec:tropative.sWpa}).

An apparent exception could seem to be \japhug{nɤɣmaʁ}{consider wrong} from the negative copula \japhug{maʁ}{not be}. However, this verb never means `consider $X$ not to be $Y$', and it seems that one of the original meanings of \ipa{maʁ}  was `not to be right', as shown by a lexicalized form such as the fossilized participle \japhug{kɯmaʁ}{bad thing} (from `(something) which is not right', §\ref{sec:lexicalized.subject.participle}). In this example the tropative preserved the original meaning of the verb, while the base verb underwent an independent semantic change (already at the common Gyalrong stage).

The ability to undergo tropativization is thus a criterion for identifying a sub-class of adjectives among stative verbs (with the exception of some derived adjectives, which are not compatible with the tropative possibly for morphological reasons, §\ref{sec:tropative.other.derivations}).

\subsection{Allomorphy} \label{sec:tropative.allomorphy}
  \is{tropative!allomorphy}\is{morphology!allomorphy}  \is{allomorphy!tropative}
As shown by \tabref{tab:tropative}, aside from the regular \forme{nɤ-} allomorph, a few verbs select a \forme{nɤɣ-} / \forme{nɤx-} allomorph. A similar allomorphy is observed on the sigmatic causative (§\ref{sec:caus.sWG}) and the applicative (§\ref{sec:allomorphy.applicative}) prefixes. The \forme{sɯ-} / \forme{sɯɣ-} alternation of the sigmatic causative is still productive: the latter allomorph occurs when the original verb is intransitive, without an initial consonant cluster and without initial velar or uvular. It is possible that a similar distribution used to exist at a former stage for the \forme{nɤ-} / \forme{nɤɣ-} allomorphs, but the data at hand do not permit a firm conclusion.

The \forme{nɤ-} allomorph of the tropative is homophonous with that of the applicative before \forme{a-} contracting verbs (§\ref{sec:contraction}, §\ref{sec:allomorphy.applicative}), but there are no cases of forms that are ambiguous between these two derivations.

When two base verbs have stems that only differ in the presence vs. absence of an \forme{a-} prefixal element, their tropative form is identical. For instance, the stative verbs \japhug{amtɕoʁ}{be pointy} and \japhug{mtɕoʁ}{be sharp} (§\ref{sec:a.non.passive.denominal}) have the same tropative \forme{nɤmtɕoʁ}, which can be interpreted as either `find pointy' or `find sharp'.

\subsection{Past imperfective} \label{sec:tropative.pst.ipfv}
\is{Past Imperfective!tropative} \is{tropative! Past Imperfective}
Tropative verbs stand out among transitive verbs in that they are compatible with Past Imperfective \forme{pɯ-} and Inferential imperfective \forme{pjɤ-} forms as in (§\ref{sec:pWwGnAmWm}) (see also \ref{ex:mWpWnAsAscita} in §\ref{sec:tropative.other.derivations} below), unlike most transitive verbs which require the progressive \forme{asɯ-} to occur with these TAME prefixes (§\ref{sec:pst.ifr.ipfv}).


 \begin{exe}
\ex \label{sec:pWwGnAmWm}
\gll ɯ-tɯ-tɕur mɤ-tɕʰom tɕe, pɯ́-wɣ-nɤ-mɯm ɕti.  \\
\textsc{3sg}.\textsc{poss}-\textsc{nmlz}:\textsc{deg}-sour \textsc{neg}-be.exceedingly:\textsc{fact} \textsc{lnk} \textsc{pst}.\textsc{ipfv}-\textsc{inv}-\textsc{trop}-be.tasty be.\textsc{aff}:\textsc{fact} \\
 \glt `It is not too sour, and we used to find it tasty.' (17-ndZWnW)
 \japhdoi{0003524\#S54}
\end{exe}  

The use of the Inferential or the Perfective with tropative verbs indicates a change of state. For instance, \forme{to-nɤ-mɯm} \textsc{ifr}-\textsc{trop}-be.tasty) means `(he used not to find it tasty, but now) he finds it tasty'.
 
\subsection{Lexicalized tropatives} \label{sec:tropative.lexicalized}
Some tropative verbs have specialized meanings that are not completely predictable from the base verb. Thus, \forme{nɤ-pe} (tropative of \japhug{pe}{be good}), in addition to its regular meaning `consider to be good', can also used in the sense of `love', as in (\ref{ex:YAnAmpCAr.YAnApe}) above. The tropative of the modal auxiliary \japhug{ntsʰi}{have better} (§\ref{sec:ra.khW.jAG.verb}), \forme{nɤntsʰi}, also has this meaning (\ref{ex:anWwGnAntshia}).\footnote{Since the etymological relationship between the base verb \japhug{ntsʰi}{have better}  and \japhug{nɤntsʰi}{love} is not synchronically transparent, the \forme{nɤ-} is not analyzed as a prefix in the glosses. }

\begin{exe}
\ex \label{ex:anWwGnAntshia}
\gll aʑo a-nɯ́-wɣ-nɤntsʰi-a ra \\
\textsc{1sg} \textsc{irr}-\textsc{pfv}-\textsc{inv}-love-\textsc{1sg} be.needed:\textsc{fact} \\
\glt `[Aladin thought] `May the princess love me.'' (140511 alading-zh)
\japhdoi{0003953\#S187}
\end{exe}

 The transitive verb \japhug{nɤkʰe}{bully} from \japhug{kʰe}{be stupid}, probably also used to be a tropative verb `consider to be stupid', with a quite unpredictable semantic evolution.

The perception verb \japhug{nɤmnɤm}{smell} (vt) is formally a tropative derived from the intransitive \japhug{mnɤm}{smell} (vi), a verb that can only take nouns meaning `smell' (such as the inalienably possessed noun \japhug{tɤ-di}{smell}) as subject, as shown by (\ref{ex:Wdi.mnAm}).

\begin{exe}
\ex \label{ex:Wdi.mnAm}
\gll nɯnɯ ɕɤɣ nɯ wuma ʑo pe, tɕe pjɯ́-wɣ-βlɯ tɕe ɯ-di wuma mnɤm \\
\textsc{dem} juniper \textsc{dem} really \textsc{emph} be.good:\textsc{fact} \textsc{lnk} \textsc{ipfv}-\textsc{inv}-burn \textsc{lnk} \textsc{3sg}.\textsc{poss}-smell really have.a.smell:\textsc{fact} \\
\glt `The juniper is very nice, when one burns it, it has a strong smell.' (08-CAG)
\japhdoi{0003442\#S13}
\end{exe}

The tropative \forme{nɤmnɤm} can take as object a noun meaning `smell', with the referent whose smell is perceived encoded as a possessive prefix on the object (for instance the \textsc{3pl} prefix on the noun \forme{nɯ-di} `their smell' in \ref{ex:nWdi.tunAmnAm}). However, unlike its base verb, \forme{nɤmnɤm} can also directly select as object the referent whose smell is perceived, including even a first or second person as in (\ref{ex:tutanAmnAm}).

\begin{exe}
\ex \label{ex:nWdi.tunAmnAm}
\gll srɯtpʰu nɯ kɯ, ŋotɕu rɤʑi-nɯ pjɤ-sɯχsɤl matɕi, nɯ-di tu-nɤ-mnɤm pjɤ-ɕti tɕe pjɤ-mtsʰɤm. \\
râkshasa \textsc{dem} \textsc{erg} where stay:\textsc{fact}-\textsc{pl} \textsc{ifr}-realize \textsc{lnk} \textsc{3pl}.\textsc{poss}-smell \textsc{ipfv}-\textsc{trop}-have.a.smell \textsc{ifr}.\textsc{ipfv}-be.\textsc{aff} \textsc{lnk} \textsc{ifr}-feel \\
\glt `The ogre realized where they where, as it was sniffing them out and perceived [their smell].' (160706 poucet6)
\japhdoi{0006109\#S49}
\end{exe}


\begin{exe}
\ex \label{ex:tutanAmnAm}
\gll  jɤ-ɣi tɕe, pɤjkʰu tu-ta-nɤ-mnɤm \\
\textsc{imp}-come \textsc{lnk} still \textsc{ipfv}:\textsc{up}-1\fl{}2-\textsc{trop}-have.a.smell \\
\glt `Come (here), I will smell you (to see if you have had alcohol).' (140506 loBzi)
\japhdoi{0003923\#S11}
\end{exe}

Example (\ref{ex:nWdi.tunAmnAm}) also shows that  \forme{nɤmnɤm} expresses volitional olfactory perception (looking for something by paying attention to smell). Its non-volitional counterpart is \forme{mtsʰɤm} in the sense of `perceive (a smell) inadvertently; find (by smell)'.\footnote{The verb \forme{mtsʰɤm} expresses non-visual non-volitional perception (§\ref{sec:preverb.perception}), not only auditory (in the meaning `hear'), but also olfactory as in (\ref{ex:nWdi.tunAmnAm}). }


\subsection{Compatibility with other derivations} \label{sec:tropative.other.derivations}
  \is{tropative!compatibility}
In addition to underived stative verbs as in \tabref{tab:tropative} above, the tropative derivation can also take verbs with the proprietive \forme{sɤ-} derivation (§\ref{sec:proprietive})


One of the most common of such verbs is \japhug{nɤsɤscit}{find pleasant} (of a place, a situation, an event) from the proprietive \japhug{sɤscit}{be pleasant} (`be such that people feel happy with/in it') of the stative verb \japhug{scit}{be happy}. In the excerpt from a conversation in (\ref{ex:sAscit.nAsAscit}), we see that the tropative \forme{mɯ-pɯ-nɤ-sɤ-scit-a} (\ref{ex:mWpWnAsAscita}) occurs in answer to a question with the proprietive verb \forme{ɯ-pɯ́-sɤ-scit} (\ref{ex:WpWsAscit}). 


\begin{exe}
\ex \label{ex:sAscit.nAsAscit}
\begin{xlist}
\ex \label{ex:WpWsAscit}
\gll (T) atu tɤʁaʁ ɯ-pɯ́-sɤ-scit? [...] \\
{ } up.there party \textsc{qu}-\textsc{pst}.\textsc{ipfv}-\textsc{prop}-be.happy \\
\glt `Was the party up there nice?'
\ex \label{ex:pWsAscit.alhaR}
\gll (L)  pɯ-sɤ-scit a-ɬaʁ, mɤ-kɯ-sɤ-scit pɯ-me \\
{ } \textsc{pst}.\textsc{ipfv}-\textsc{prop}-be.happy \textsc{1sg}.\textsc{poss}-MZ \textsc{neg}-\textsc{sbj}:\textsc{pcp}-\textsc{prop}-be.happy \textsc{pst}.\textsc{ipfv}-not.exist \\
\glt `It was nice, mother-in-law, there was nothing that was not nice.'
\ex \label{ex:mWpWnAsAscita}
\gll (A) aʑo ndɤre mɯ-pɯ-nɤ-sɤ-scit-a, lo-nɯtɕʰomba-a \\
{ } \textsc{1sg} \textsc{lnk} \textsc{neg}-\textsc{pst}.\textsc{ipfv}-\textsc{trop}-\textsc{prop}-be.happy-\textsc{1sg} \textsc{ifr}-have.a.cold-\textsc{1sg} \\
\glt `As far as I am concerned, I did not find [that party] nice, I caught a cold.' (TaRrdo2003)
\end{xlist}
\end{exe}

The antonym of  \forme{nɤsɤscit}, \japhug{nɤsɤɣdɯɣ}{find unpleasant}, is also the tropative of a proprietive verb \japhug{sɤɣdɯɣ}{be unpleasant} from the base verb \japhug{dɯɣ}{have enough of}, `be upset'. Both \forme{scit} and \forme{dɯɣ} are Tibetan loanwords (from \tibet{སྐྱིད་}{skʲid}{be happy} and \tibet{སྡུག་}{sdug}{suffer}, respectively), showing the productivity of this double derivation. While no other examples of this double derivation are found in the corpus, it is possible to elicit additional examples (§\ref{sec:proprietive.compatibility}).

\begin{exe}
\ex \label{ex:fsapaR.ra.kW.nAsAGdWGnW}
\gll nɯnɯ ɯ-rme ɯ-ŋgɯ ri ku-ɕe tɕe tɕendɤre ku-ɕe nɤ ku-ɕe tɕe tɕe ɯ-ndʐi ɯ-ŋgɯ ri ku-otsa ɲɯ-ŋu. tɕe fsapaʁ ra kɯ nɯ nɤ-sɤɣ-dɯɣ-nɯ ŋgrɤl ma \\
\textsc{dem} \textsc{3sg}.\textsc{poss}-hair \textsc{2sg}.\textsc{poss}-in \textsc{loc} \textsc{ipfv}:\textsc{east}-go \textsc{lnk} \textsc{lnk} \textsc{ipfv}:\textsc{east}-go \textsc{add} \textsc{ipfv}:\textsc{east}-go \textsc{lnk} \textsc{lnk} \textsc{3sg}.\textsc{poss}-skin \textsc{3sg}.\textsc{poss}-in \textsc{loc} \textsc{ipfv}-prick \textsc{sens}-be \textsc{lnk} animal \textsc{pl} \textsc{erg} \textsc{dem} \textsc{trop}-\textsc{prop}-be.upset:\textsc{fact}-\textsc{pl} be.usually.the.case:\textsc{fact} \textsc{lnk} \\
\glt `[Its seeds] go into the hair [of the sheep], deeper and deeper, and stick into their skin, and the animals find this unpleasant.' (19-khWlu)
\japhdoi{0003540\#S96}
\end{exe}

The tropative-proprietive verbs \forme{nɤsɤscit} and \forme{nɤsɤɣdɯɣ} take as transitive subject (with the ergative in \ref{ex:fsapaR.ra.kW.nAsAGdWGnW}) the experiencer (like the base verbs \japhug{scit}{be happy} and \japhug{dɯɣ}{have enough of}) and as object the stimulus (the situation or place causing the feeling, like the proprietive verbs \japhug{sɤscit}{be pleasant} and \japhug{sɤɣdɯɣ}{be unpleasant}).

The tropative is also attested with the subject-oriented facilitative \forme{ɣɤ-} (§\ref{sec:facilitative.GA}). For instance, from \japhug{ɣɤβzi}{become drunk easily} (facilitative of \japhug{βzi}{get drunk}) one can derive the tropative \forme{nɤɣɤβzi} `consider that $X$ becomes drunk easily' (\ref{ex:YWtAnAGABzi}).

\begin{exe}
\ex \label{ex:YWtAnAGABzi}
\gll nɤʑo ndɤre ɲɯ-ta-nɤ-ɣɤ-βzi  \\
\textsc{2sg} \textsc{lnk} \textsc{sens}-1\fl{}2-\textsc{trop}-\textsc{facil}-be.drunk \\
\glt `You, I think that you get drunk easily.' (elicited)
\end{exe}

Distributed property verbs such \japhug{amɯzɣɯt}{be evenly distributed} (§\ref{sec:distributed.amW}) can undergo the tropative derivation (\forme{nɤmɯzɣɯt} `consider that $X$ is evenly distributed', for instance about colours), though such uses are uncommon.

The tropative cannot take as input any other type of derived stative verbs, such as the passive (§\ref{sec:passive}), and more surprisingly the \forme{nɯɣɯ-} facilitative (§\ref{sec:facilitative.nWGW}).

Tropative verbs can in their turn serve as input for other derivations, including causative (§\ref{sec:sig.caus.other.derivations}), reflexive (§\ref{sec:refl.tropative}) and reciprocal (§\ref{sec:reciprocal.other}). A related derivation whose meaning is close to that of a reflexivized tropative is the auto-eva\-lua\-tive \forme{znɤ-} (§\ref{sec:autoevaluative}).

\subsection{Other tropative constructions}  \label{sec:tropative.other.construction}
\is{causative!tropative} \is{tropative!causative}
Apart from the tropative \forme{nɤ\trt}, tropative meaning can be expressed in Japhug by the sigmatic causative in a few cases (§\ref{sec:sig.caus.tropative}), the velar causative (§\ref{sec:velar.causative.tropative}) and by periphrastic constructions (§\ref{sec:tropative.sWpa}). 

Other derivations with a tropative meaning include the \forme{rɤ(ɣ)-} prefix in \japhug{rɤɣrɯɣ}{cherish} (from \japhug{rɯɣ}{be precious}, §\ref{sec:rA.non.apass}), and the reflexive  \forme{ʑɣɤ-} of some intransitive verbs (§\ref{sec:refl.intr}).
