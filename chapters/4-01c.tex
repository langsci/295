\chapter{Negation} \label{chap:negation}
Negation in Japhug is mainly expressed by negative prefixes (§\ref{sec:negation}). It is symmetrical \citep{miestamo05negation}: the presence of these prefixes is not systematically correlated with finiteness or TAME alternations.

In addition to negative prefixes, a periphrastic negative construction with sentence-final negative auxiliary is also attested (§\ref{sec:periphrastic.negation}), and it is required in particular to mark double negation (§\ref{sec:double.negation}).




\section{Negative prefixes} \label{sec:negation}

\subsection{Allomorphy} \label{sec:neg.allomorphs}
Four negative prefixes are found in Japhug: \forme{mɤ\trt}, \forme{mɯ\trt}, \forme{ma-} and \forme{mɯ́j-}. Their distribution is determined by TAME and finiteness.

The \forme{ma-} prefix is restricted to prohibitive verb forms (§\ref{sec:prohibitive.morpho}), and always combined with type A orientation preverbs (§\ref{sec:kamnyu.preverbs}). 

The stress-bearing (§\ref{sec:stress.prefixal.chain}) \forme{mɯ́j-} prefix (see \ref{ex:WmAtaZa} below) is a portmanteau of negation and Sensory evidental (§\ref{sec:sensory}). The Sensory prefix \forme{ɲɯ-} has the expected negative form \forme{mɯ-ɲɯ-} only in the case of contracting verbs (§\ref{sec:sensory.morphology}). Otherwise, when the  prefixal sequence \forme{mɯ-ɲɯ-} occurs, the \forme{ɲɯ-} preverb marks the Imperfective (§\ref{sec:ipfv.morphology}).

The form \forme{mɤ-} is found on non-finite verbal forms without any orientation preverb (§\ref{sec:subject.participle.other.prefixes}, §\ref{sec:object.participle.other.prefixes}, §\ref{sec:infinitives.other.prefixes}, §\ref{sec:degree.nominal.prefixes}), in Factual Non-Past and Irrealis form (see \ref{ex:amAGWtAtWwGndza} above in §\ref{sec:outer.prefixal.chain}), and also when preceded by the interrogative \forme{ɯ-} (§\ref{sec:interrogative.W.morpho}) as in (\ref{ex:WmAtaZa}) and the Proximative \forme{jɯ-} (§\ref{sec:proximative}) as in  (\ref{ex:jWmACtAthuta}).


\begin{exe}
	\ex \label{ex:WmAtaZa}
	\gll tɕe tɯ-ŋke ɯ-mɤ-ta-ʑa, ɯ-mɤ-nɯ-mɯnmu qʰe mɯ́j-sɤ-mto \\
	\textsc{lnk} \textsc{inf}:II-walk \textsc{qu}-\textsc{neg}-\textsc{aor}:3\flobv{}-start \textsc{qu}-\textsc{neg}-\textsc{aor}-move \textsc{lnk} \textsc{neg}:\textsc{sens}-\textsc{prop}-see \\
	\glt `If it has not started walking, if it has not moved, it is not visible.' (26-NalitCaRmbWm)
\end{exe}

\begin{exe}
	\ex \label{ex:jWmACtAthuta}
	\gll  jɯfɕɯr jɯ-mɤ-ɕ-tɤ-tʰu-t-a ʑo \\
	yesterday \textsc{proxm}-\textsc{neg}-\textsc{tral}-\textsc{aor}-ask-\textsc{pst}:\textsc{tr}-\textsc{1sg} \textsc{emph} \\
	\glt `Yesterday I almost did not go and ask about it.' (elicited)
\end{exe}

The allomorph \forme{mɯ-} is found elsewhere, including non-finite verb forms with orientation preverbs (§\ref{sec:subject.participle.other.prefixes}, §\ref{sec:object.participle.other.prefixes}, §\ref{sec:inf.exist}) and all finite verb forms with orientation preverbs other than the Prohibitive and the verbal forms where slot -6 is filled (Irrealis, Interrogative, Proximative etc).

The partial reduplication (§\ref{sec:verb.initial.redp}) of \forme{mɯ-} (which occurs in particular in the protasis of conditionals, §\ref{sec:redp.protasis}) does not yield expected $\dagger$\forme{mɯ\redp{}mɯ\trt}, but rather \forme{mɯ\redp{}mɤ-} with vowel alternation: compare for instance \forme{mɯ-nɯ-si} `she did not die' with the \forme{mɯ-} prefix (as expected in the Aorist) with \forme{mɯ\redp{}mɤ-nɯ-si-a} `if I do not die' in (\ref{ex:mWmAnWsia}). 


\begin{exe}
	\ex \label{ex:mWnWsi}
	\gll  tɤ-mu nɯ ɣɯjpa kɯre mɤɕtʂa mɯ-nɯ-si \\
	\textsc{indef}.\textsc{poss}-mother \textsc{dem} this.year \textsc{dem}.\textsc{loc} until \textsc{neg}-\textsc{aor}-die \\
	\glt `The old woman only died this year.' (`she did not die until this year') (14-siblings, 353)
\end{exe}

\begin{exe}
	\ex \label{ex:mWmAnWsia}
	\gll mɯ\redp{}mɤ-nɯ-si-a nɤ, a-tɤ-kɯ-nɯlaʁrdaβ-a ra \\
	\textsc{cond}\redp{}\textsc{neg}-\textsc{aor}-die-\textsc{1sg} \textsc{add} \textsc{irr}-\textsc{pfv}-2\fl{}1-hit.with.forelegs-\textsc{1sg} be.needed:\textsc{fact} \\
	\glt `If [after that] I have not died [yet], hit me with your forelegs.' (2003kAndzwsqhaj2, 83)
\end{exe}

The Interrogative \forme{ɯ-} prefix, like conditional reduplication, requires the \forme{mɤ-} negative prefix (see \ref{ex:WmAtaZa} above), and this commonality in morphophonology is correlated with a similarity in function, since both the prefix \forme{ɯ-} and reduplication are used to mark the verb of the protasis of conditional clauses (§\ref{sec:real.conditional}).


The -6 slot \forme{ɯmɤ-} prefix of possible modality (§\ref{sec:WmA}) has a surface form identical to the combination of the interrogative \forme{ɯ-} with the negative prefix \forme{mɤ\trt}, as in (\ref{ex:tundze.WmANu}). The \forme{ɯmɤ-} synchronically differs from \forme{ɯ-mɤ-} (from which it historically derives) in that it can occur with the peg circumfix (§\ref{sec:peg.circumfix}, §\ref{sec:WmA.kW.ci}). It is not compatible with a negative prefix.

\begin{exe}
	\ex \label{ex:tundze.WmANu}
	\gll  qajɯ kɯ-fse ra tu-ndze ɯmɤ-ŋu ma \\
	bug \textsc{sbj}:\textsc{pcp}-be.like \textsc{pl} \textsc{ipfv}-eat[III] \textsc{prob}-be:\textsc{fact} \textsc{lnk} \\
	\glt `It presumably/maybe eats bugs.' (23-pGAYaR, 35)
\end{exe}

Negative prefixes are restricted to verb forms, and cannot be prefixed on nouns (§\ref{sec:negation.noun}).  They occur however on non-finite verb forms including participles (§\ref{sec:subject.participle.other.prefixes}, §\ref{sec:object.participle.other.prefixes}, §\ref{sec:oblique.participle.orientation}), infinitives (§\ref{sec:infinitives.other.prefixes}, §\ref{sec:inf.exist}, §\ref{sec:dental.inf.polarity}), degree nominals (§\ref{sec:degree.nominal.prefixes}), converbs (§\ref{sec:gerund.neg}, §\ref{sec:purposive.converb}), but not action nominals (§\ref{sec:action.nominals}) and  fossilized deverbal nouns (§\ref{sec:fossil.nmlz}).

\subsection{Suppletive negative verbs} \label{sec:suppletive.negative}
Negative prefixes can occur on most verbs, with the exception of copulas and existential verbs, which have suppletive negative forms (§\ref{sec:intr.person.irregular}), as illustrated in \tabref{tab:neg.suppletion}.\footnote{Another verb lacking negative forms is  \japhug{kɤtɯpa}{tell}, though for a different reason (§\ref{sec:irregular.transitive}). } The contrast between the neutral copula \japhug{ŋu}{be} and the Emphatic Affirmative \japhug{ɕti}{be} is neutralized in the negative, where only one negative copula \japhug{maʁ}{not be} is present (§\ref{sec:copula.basic}).

\begin{table}
	\caption{Suppletive negative verbs} \label{tab:neg.suppletion}
	\begin{tabular}{lllll}
		\lsptoprule
		& Affirmative & Negative \\
		\midrule
		Copula & \japhug{ŋu}{be},   &\japhug{maʁ}{not be} \\
		&\japhug{ɕti}{be} (emphatic affirmative) & \\
		\tablevspace
		Existential & \japhug{tu}{exist} &\japhug{me}{not exist} \\
		\tablevspace
		Sensory existential & \japhug{ɣɤʑu}{exist} &\japhug{maŋe}{not exist} \\
		\lspbottomrule
	\end{tabular}
\end{table}

For instance, the negation of \japhug{ŋu}{be} and \japhug{tu}{exist} can only be \japhug{maʁ}{not be} (\ref{ex:aZo.maRa}) and \japhug{me}{not exist} (\ref{ex:tWtshot.pWme}).

\begin{exe}
	\ex \label{ex:aZo.maRa}
	\gll aʑo maʁ-a \\
	\textsc{1sg} not.be:\textsc{fact}-\textsc{1sg} \\
	\glt `I am not [that girl].' (2003sras, 66)
\end{exe}

\begin{exe}
	\ex \label{ex:tWtshot.pWme}
	\gll kɯɕɯŋgɯ tɕe tɯtsʰot pɯ-me tɕe, \\
	former.times \textsc{loc} clocks \textsc{pst}.\textsc{ipfv}-not.exist \textsc{lnk} \\
	\glt `In former times, there were no clocks.' (29-LAntshAm, 65)
\end{exe}

Combining \forme{ŋu} and \forme{tu} with negative prefixes (for instance $\dagger$\forme{mɤ-ŋu-a} instead of \forme{maʁ-a} `I am not/It is not me' and $\dagger$\forme{mɯ-pɯ-tu} instead of \forme{pɯ-me}) is utterly incorrect and categorically rejected by all speakers. Likewise, negative copulas cannot take negative prefixes ($\dagger$\forme{mɤ-maʁ-a} is agrammatical). 

It is however possible to combine affirmative copulas or existential verbs with their negative counterparts as postverbal periphrastic negations (§\ref{sec:periphrastic.negation}). The copulas \forme{ŋu} and \forme{maʁ} occur together in the phrase \forme{ŋu cinɤ maʁ kɯ} `in any case it is not true' (example \ref{ex:Nu.cinA.maR.kW}, §\ref{sec:cinA}). The negative existential \japhug{me}{not.exist} is very commonly found with the participle of its antonym \forme{kɯ-tu} to express emphasis on the non-existence, as in (\ref{ex:pWkWtu.me}) in comparison with (\ref{ex:tWtshot.pWme}) (§\ref{sec:existential.basic}).

\begin{exe}
	\ex \label{ex:pWkWtu.me}
	\gll  kɯɕɯŋgɯ mkʰɯrlu kɯ-fse ra pɯ-kɯ-tu me. \\
	former.times machine \textsc{sbj}:\textsc{pcp}-be.like \textsc{pl} \textsc{pst}.\textsc{ipfv}-\textsc{sbj}:\textsc{pcp}-exist not.exist:\textsc{fact} \\
	\glt `In former times, there were no machines or things like that at all.' (140430 tWfkur, 4)
\end{exe}

\subsection{Verbs requiring the negative prefixes} \label{sec:obligatory.negative}
Some verbs are defective and lack affirmative forms: they only appear in conjunction with negative prefixes. Two categories can be distinguished.

First, a handful of defective verb roots are only attested with negative prefixes, listed in \tabref{tab:defect.verbs.neg}. The verb roots in this table are not found in any derived form without negation, except for \japhug{mɤ-xsi}{it is not known}, a highly defective verb (§\ref{sec:irregular.transitive}) historically related to the transitive verb \japhug{sɯz}{know} (whose paradigm is not defective). 

The complex collocation \japhug{\textsc{neg}+spa=\textsc{neg}+rka=tu/me}{be guilty/innocent} contains a bipartite verb (§\ref{sec:bipartite}) whose first component \forme{-spa} may be related to the modal auxiliary \japhug{spa}{be able to} (§\ref{sec:abilitative.lexicalized}, §\ref{sec:spa.verb}) and whose second component \forme{-rka} is an orphan verb.

\begin{table}
	\caption{Verb roots requiring a negative prefix} \label{tab:defect.verbs.neg}
	\begin{tabular}{lllll}
		\lsptoprule
		Root & Verb & Factual Non-Past \textsc{3sg} \\
		\midrule
		\forme{-ʑɯ} &\japhug{\textsc{neg}+ʑɯ}{not just be} & \forme{mɤ-ʑɯ}  \\
		\forme{-tɕʰɤz} &\japhug{\textsc{neg}+tɕʰɤz}{be contrary to religion} & \forme{mɤ-tɕʰɤz} \\
		\forme{-rka} &\japhug{\textsc{neg}+spa=\textsc{neg}+rka=tu/me}{be guilty/innocent} & \forme{mɤ-spe mɤ-rke me} \\
		\forme{-(x)si} & \japhug{\textsc{neg}-xsi}{it is not known} & \forme{mɤ-xsi}\\
		\lspbottomrule
	\end{tabular}
\end{table}

The verb \japhug{\textsc{neg}+ʑɯ}{not just be}, `not be/have only' (\ref{ex:mAZW.ma}) is most commonly used in one of the comparative constructions  (§\ref{sec:mAZW}) and has even been further grammaticalized as an adverb \japhug{mɤʑɯ}{even more} (\ref{ex:mAZW.Zo.chWtshu}).

\begin{exe}
	\ex \label{ex:mAZW.ma}
	\gll kʰɤjmu nɯtɕu mɤ-ʑɯ ma, tɤrɲɟo tu tɕe, \\
	kitchen \textsc{dem}:\textsc{loc} \textsc{neg}-be.just \textsc{lnk} shelf exist:\textsc{fact} \textsc{lnk} \\
	\glt `In the kitchen, there are not just [the aforementioned objects], 
	there is also a shelf (to store the cooking implements).' (2011-11-kha2, 22)
\end{exe}

\begin{exe}
	\ex \label{ex:mAZW.Zo.chWtshu}
	\gll nɯnɯ paʁ kɯ tu-ndze tɕe, mɤʑɯ ʑo cʰɯ-tsʰu cʰa \\
	\textsc{dem} pig \textsc{erg} \textsc{ipfv}-eat[III] \textsc{lnk} even.more \textsc{emph} \textsc{ipfv}-be.fat can:\textsc{fact} \\
	\glt `When the pig eats it [acorns], it can grow even fatter.' (08-CkrAz, 49)
\end{exe}

Second, most \forme{sɯ-/z-} abilitative verbs (§\ref{sec:abilitative}) are only found in negative forms. For instance the abilitative \forme{z-nɤjo} from \japhug{nɤjo}{wait} only occurs with a negation in the meaning `cannot wait to' (due to hurry/impatience) as in (\ref{ex:juzGWt.mWpjAznAjo}).

\begin{exe}
	\ex \label{ex:juzGWt.mWpjAznAjo}
	\gll  spjaŋkɯ nɯ kɯ  ju-zɣɯt mɯ-pjɤ-z-nɤjo ʑo tɕe, \\
	wolf \textsc{dem} \textsc{erg} \textsc{ipfv}-arrive \textsc{neg}-\textsc{ifr}.\textsc{ipfv}-\textsc{abil}-wait \textsc{emph} \textsc{lnk} \\
	\glt `The wolf could not wait [for the fox] to arrive.' (140516 huli de baofu-zh, 43)
\end{exe}

This constraint is also observed with the lexicalized abilitative \japhug{spʰɯt}{can cut} (from \japhug{pʰɯt}{cut, pluck}, §\ref {sec:abilitative.lexicalized}) as in (\ref{ex:WndzrW.mWjsphWt}).

\begin{exe}
	\ex \label{ex:WndzrW.mWjsphWt}
	\gll tsɯntu kɯ ɯ-ndzrɯ mɯ́j-spʰɯt ma ɯ-tɯ-rko ɯ-tɯ-jaʁ ɲɯ-sɤre ʑo \\
	scissors \textsc{erg} \textsc{3sg}.\textsc{poss}-nail \textsc{neg}:\textsc{sens}-can.cut \textsc{lnk} \textsc{3sg}.\textsc{poss}-\textsc{nmlz}:\textsc{deg}-be.hard \textsc{3sg}.\textsc{poss}-\textsc{nmlz}:\textsc{deg}-be.thick \textsc{sens}-be.ridiculous \textsc{emph} \\
	\glt `The scissors cannot cut through her nails, as they are extremely hard and thick.' (2012, heard in context) 
\end{exe}


\subsection{Lexicalized negation} \label{sec:lexicalized.negative}
Some verb roots have negative forms with unpredictable lexicalized meanings. Three subtypes can be distinguished.

First, the negative form of the underived form of the root can have an extended meaning. For instance, the verb \japhug{rkaŋ}{be strong}, `be in good physical condition' can mean `be pregnant' in the negative as in (\ref{ex:mWYArkaN}).

\begin{exe}
	\ex \label{ex:mWYArkaN}
	\gll ki tɕʰeme ki mɯ-ɲɤ-rkaŋ \\
	\textsc{dem}.\textsc{prox} girl \textsc{dem}.\textsc{prox} \textsc{neg}-\textsc{ifr}-be.in.good.shape \\
	\glt `This woman became pregnant.' (elicited)
\end{exe}

Second, the special meaning of the negation appears in derivations: for instance, the stative verb \japhug{ftsʰi}{feel better} has a negative causative \japhug{\textsc{neg}+sɯftsʰi}{force}, `coerce'  (§\ref{sec:sig.caus.lexicalized}) and a lexicalized negative participle \japhug{mɤkɯftsʰi}{forcibly} used as an adverb (§\ref{sec:velar.inf.adverb}). 


Third, some lexicalized noun-verb collocations require a negative form (§\ref{sec:other.collocation.intr}), for instance \forme{tɯ-skʰrɯ=\textsc{neg}-βdi} `be pregnant' from \japhug{tɯ-skʰrɯ}{body} and \japhug{βdi}{be well} as in (\ref{ex:ndZiskhrW.mWYABdi}).

\begin{exe}
	\ex \label{ex:ndZiskhrW.mWYABdi}
	\gll ndʑi-rʑaβ ʁnaʁna ʑo ndʑi-skʰrɯ mɯ-ɲɤ-βdi  \\
	\textsc{3du}.\textsc{poss}-wife both \textsc{emph} \textsc{3du}.\textsc{poss}-body \textsc{neg}-\textsc{ifr}-be.well \\
	\glt `Both of their wives got pregnant.' (2005 Lobzang, 2) 
\end{exe}  

\section{Periphrastic negation} \label{sec:periphrastic.negation}
Aside from negative prefixes (\ref{ex:mWpWmtota}), negation can be expressed by the negative copula \japhug{maʁ}{not be} (\ref{ex:pWmtota.maR}) and the negative existential verb \japhug{me}{not exist} (\ref{ex:pWmtota.me}) in postverbal position. 

\begin{exe}
	\ex  \label{ex:negation.3}
	\begin{xlist}
		\ex \label{ex:mWpWmtota}
		\gll mɯ-pɯ-mto-t-a \\
		\textsc{neg}-\textsc{aor}-see-\textsc{pst}:\textsc{tr}-\textsc{1sg} \\
		\glt `I have not seen it.' (several examples)
		\ex \label{ex:pWmtota.maR}
		\gll pɯ-mto-t-a maʁ\\
		\textsc{aor}-see-\textsc{pst}:\textsc{tr}-\textsc{1sg} not.be:\textsc{fact} \\
		\glt `I have not seen it, (but rather...).' 
		\ex \label{ex:pWmtota.me}
		\gll pɯ-mto-t-a me \\
		\textsc{aor}-see-\textsc{pst}:\textsc{tr}-\textsc{1sg} not.exist:\textsc{fact} \\
		\glt `I have seen none/I haven't seen anything.' (several examples)
	\end{xlist}
\end{exe}

The negative copula \japhug{maʁ}{not be} occurs postverbally in Periphrastic TAME categories requiring a copula (§\ref{sec:ipfv.periphrastic.TAME}), or to express emphatic negation, contrasting with an assertative postverbal copula (§\ref{sec:affirmative.copula.function}) as in (\ref{ex:YWGAwua.maR}).

\begin{exe}
	\ex 
	\begin{xlist}
		\ex \label{ex:tChindzxa.YWtWGAwu}
		\gll tɕʰindʐa ɲɯ-tɯ-ɣɤwu ŋu? \\
		why \textsc{sens}-2-cry be:\textsc{fact} \\
		\glt `Why are you crying?' 
		\ex \label{ex:YWGAwua.maR}
		\gll ɲɯ-ɣɤwu-a maʁ nɤ, tɯ-mɯ pjɯ-lɤt ɲɯ-ɕti ma \\
		\textsc{sens}-cry-\textsc{1sg} not.be:\textsc{fact} \textsc{sfp} \textsc{indef}.\textsc{poss}-sky \textsc{ipfv}-release \textsc{sens}-be.\textsc{aff} \textsc{lnk} \\
		\glt `It is not that I am crying, [I look like I am crying because] it has been raining.' (2005-stod-kunbzang, 391)
	\end{xlist}
\end{exe}

The postverbal negative existential \forme{me}  (§\ref{sec:existential.basic}) has a universal negative meaning `nothing' or negative indefinite `none, not any' as in (\ref{ex:pWmtota.me}) (§\ref{sec:negation.existential}). 

\begin{exe}
	\ex \label{ex:pWmtota.me2}
	\gll  tɕe ɯ-mdoʁ tɕʰi ʑo fse mɤ-xsi, a-kɤ-ti me ma mɯ-pɯ-mto-t-a. ɯ-ndʐi kɯnɤ pɯ-mto-t-a me \\
	\textsc{lnk} \textsc{3sg}.\textsc{poss}-colour what \textsc{emph} be.like:\textsc{fact} \textsc{neg}-\textsc{genr}:know \textsc{1sg}.\textsc{poss}-\textsc{obj}:\textsc{pcp}-say not.exist:\textsc{fact} \textsc{lnk} \textsc{neg}-\textsc{aor}-see-\textsc{pst}:\textsc{tr}-\textsc{1sg} \textsc{3sg}.\textsc{poss}-skin also \textsc{aor}-see-\textsc{pst}:\textsc{tr}-\textsc{1sg} not.exist:\textsc{fact} \\
	\glt `I don't know which colour it$_i$ has, I can't say because I have not seen it$_i$, I have not even seen any of its$_i$ hides.' (27-kikakCi, 22)
\end{exe}

It is also found in one of the superlative constructions, illustrated by example (\ref{ex:tuqhea.me}) (§\ref{sec:negative.existential.superlative}).

\begin{exe}
	\ex \label{ex:tuqhea.me}
	\gll βʑɯ kɯ-fse tu-qʰe-a me \\
	mouse \textsc{sbj}:\textsc{pcp}-be.like \textsc{ipfv}-hate[III]-\textsc{1sg} not.exist:\textsc{fact} \\
	\glt `Mice is what I hate most.' (`there is nothing that I hate like a mouse') (140427 bianfu yu huangshulang-zh, 13)
\end{exe}

%\begin{exe}
%\ex \label{ex:pWwGsWz.me}
%\gll nɯ ma kɯ-pe tɕi pɯ́-wɣ-sɯz me, mɤ-kɯ-pe tɕi pɯ́-wɣ-sɯz me. \\
%\textsc{dem} apart.from \textsc{sbj}:\textsc{pcp}-be.good also \textsc{aor}-\textsc{inv}-know not.exist:\textsc{fact} \textsc{neg}-\textsc{sbj}:\textsc{pcp}-be.good also \textsc{aor}-\textsc{inv}-know not.exist:\textsc{fact} \\
%\glt (26-qambalWla, 55)
%\end{exe} 
\section{Double negation} \label{sec:double.negation}
Since negative prefixes are not recursive, there are only two ways to express double negation in Japhug: either is by combining a negative verb form with a negative auxiliary (\japhug{me}{not exist} or \japhug{maŋe}{not exist}), or a negative verb form in a complement clause with a negative complement-taking verb. 

Double negation with existential verbs can indicate universal quantification, in particular when the verb taking the negative prefix is transitive (or semi-transitive) and no overt object (or semi-object) is present as in (\ref{ex:mAndze.Zo.me}) (see also §\ref{sec:negation.existential}).

\begin{exe}
	\ex \label{ex:mAndze.Zo.me}
	\gll paʁ aɣɯɣli ma rcanɯ mɤ-ndze ʑo me  \\
	pig produce.a.lot.of.manure:\textsc{fact} \textsc{lnk} \textsc{unexp}:\textsc{foc} \textsc{neg}-eat[III] \textsc{emph} not.exist:\textsc{fact} \\
	\glt `Pigs produce a lot of manure, as they eat everything (there is nothing they don't eat).' (05-paR, 28)
\end{exe}
%nɤʑo mɤ-tɯ-nɯskɤt me
Negative participial forms with a negative auxiliary express mild assertion, as in (\ref{ex:mAkWpe.me}) and (\ref{ex:mAkWkhW.me}).

\begin{exe}
	\ex \label{ex:mAkWpe.me}
	\gll aʑɯɣ mɤ-kɯ-pe me \\
	\textsc{1sg}:\textsc{gen} \textsc{neg}-\textsc{sbj}:\textsc{pcp}-be.good not.exist:\textsc{fact} \\
	\glt `I am fine (I don't have any particular problem).' (140506 shizi he huichang de bailingniao-zh, 73)
\end{exe}

\begin{exe}
	\ex \label{ex:mAkWkhW.me}
	\gll mɤ-kɯ-kʰɯ me \\
	\textsc{neg}-\textsc{sbj}:\textsc{pcp}-be.possible not.exist:\textsc{fact} \\
	\glt `There is nothing wrong with it.' (common in metalinguistic judgments about the grammaticality of sentences)
\end{exe}

Another type of double negation construction is observed when both the com\-plement-taking verb and the verb in the complement clause take a negative prefix. The most common verb in this type of configuration is the modal verb \japhug{kʰɯ}{be possible} (§\ref{sec:ra.khW.jAG.verb}), either with finite complements (§\ref{sec:finite.complement}) as in (\ref{ex:mAGia.mAkhW}), or with negative infinitival complements (example \ref{ex:mAkACe.mAkhW}, §\ref{sec:infinitives.other.prefixes}), expressing the meaning `have no choice but to $X$'.


\begin{exe}
	\ex \label{ex:mAGia.mAkhW}
	\gll  nɤ-rca mɤ-ɣi-a mɤ-kʰɯ qʰe \\
	\textsc{2sg}.\textsc{poss}-together.with \textsc{neg}-come:\textsc{fact}-\textsc{1sg} \textsc{neg}-be.possible:\textsc{fact} \textsc{lnk} \\
	\glt `It have no other choice but to go with you/follow you.' (Nyima Wodzer 2003.2, 94-95)
\end{exe}

Other verbs attested in this type of construction include \japhug{nɤz}{dare} (§\ref{sec:nAz.verb}) as in (\ref{ex:mAkAGi.mAnAzi}) and \japhug{cʰa}{can} (with the meaning `cannot help but' as in \ref{ex:mACWkAtshi}, §\ref{sec:am.complement}).

\begin{exe}
	\ex \label{ex:mAkAGi.mAnAzi}
	\gll rɟɤlpu fka ɕti tɕe, mɤ-kɤ-ɣi mɤ-nɤz-i ri \\
	king order be.\textsc{aff}:\textsc{fact} \textsc{lnk} \textsc{neg}-\textsc{inf}-come \textsc{neg}-dare:\textsc{fact}-\textsc{1pl} \textsc{lnk} \\
	\glt `It is the king's order, we do not dare not to come.' (Norbzang 2005, 400)
\end{exe}


\section{Negation and parts of speech other than verbs} 
Verbs are the only part of speech than can take negative prefixes. This section describes the constructions used in Japhug to express meaning corresponding to that of negative nouns or pronouns, and also discusses negative intensifiers adverbs.

\subsection{Nouns} \label{sec:negation.noun}
Although there is a privative nominal derivation in Japhug (§\ref{sec:privative}), there is no way of building a negative noun like English `non-$X$' or  `un-$X$' meaning `which is not $X$' (rather than $X$-less). The only way to express such a meaning is build using a participial relative (§\ref{sec:subject.participles}) with the negative copula \forme{$X$ kɯ-maʁ} `something/someone who/that is not $X$' (§\ref{sec:copula.basic}), for instance \forme{tɤ-rɟit kɯ-maʁ} `someone who is not a child' in (\ref{ex:tArJit.kWmaR}).

\begin{exe}
	\ex \label{ex:tArJit.kWmaR}
	\gll ɯʑora nɯ-rɟit ʑo cʰɯ-βri-nɯ qʰe kɯmaʁ ra nɯ-rɟit nɯ pɯpɯŋunɤ [tɤ-rɟit kɯ-maʁ] tú-wɣ-sɯpa qʰe \\
	\textsc{3pl} \textsc{3pl}.\textsc{poss}-offspring \textsc{emph} \textsc{ipfv}-protect-\textsc{pl} \textsc{lnk} other \textsc{pl} \textsc{3pl}.\textsc{poss}-offspring \textsc{dem} \textsc{top} \textsc{indef}.\textsc{poss}-child \textsc{sbj}:\textsc{pcp}-not.be \textsc{impf}-\textsc{inv}-consider \textsc{lnk} \\
	\glt `(At school), [some parents] protect their own children, and as for the children of other people, they consider them as if they were not children (`as non-children').' (140501 01, 59)
\end{exe}

%{sec:affirmative.copula.function}


\subsection{Pronouns} \label{sec:negative.pronoun}
Japhug lacks negative pronouns (§\ref{sec:indef.pro}), and headless relative clauses (§\ref{sec:headless.relatives.quantification}) combined with a negative existential copula (§\ref{sec:suppletive.negative})  are used to express the meanings `nothing', `nobody' (\ref{ex:nAkWCaB.me}, \ref{ex:WkWntChoz.maNe}) or `nowhere' (\ref{ex:WsArNgW.maNe}).

\begin{exe}
	\ex \label{ex:nAkWCaB.me}
	\gll nɤ-kɯ-ɕaβ me \\
	\textsc{2sg}.\textsc{poss}-\textsc{sbj}:\textsc{pcp}-catch.up not.exist:\textsc{fact} \\
	\glt `Nobody will [be able] to catchup with you.' (2003 qachGa, 109)
\end{exe}

\begin{exe}
	\ex \label{ex:WsArNgW.maNe}
	\gll tɯmɯrkʰa tɕe ɯ-sɤ-rŋgɯ maŋe \\
	evening \textsc{loc} \textsc{3sg}.\textsc{poss}-\textsc{obl}:\textsc{pcp}-lie.down not.exist:\textsc{sens} \\
	\glt `In the evening, it has no place to stay.' (26-NalitCaRmbWm, 38)
\end{exe}

In negative existential constructions, the headless relative can also mean `not any of' as \forme{kɤ-mto} `any (amadou) to be seen' in (\ref{ex:WkWntChoz.maNe}).

\begin{exe}
	\ex \label{ex:WkWntChoz.maNe}
	\gll jinde tɕe nɯ [ɯ-kɯ-ntɕʰoz] maŋe qʰe, [kɤ-mto] kɯnɤ maŋe \\
	nowadays \textsc{lnk} \textsc{dem} \textsc{3sg}.\textsc{poss}-\textsc{sbj}:\textsc{pcp}-use not.exist:\textsc{sens} \textsc{lnk} \textsc{obj}:\textsc{pcp}-see also not.exist:\textsc{sens} \\
	\glt `Nowadays \textbf{nobody} uses [amadou], there isn't even \textbf{any} to be seen.' (15-babW, 232)
\end{exe}

\subsection{Adverbs} \label{sec:negative.adverbs}
Japhug lacks an all-purpose negative adverb, but there a few negative intensifiers.

The adverb \forme{maka} is generally used with negative verb forms to express emphatic negation `not ... at all' as in (\ref{ex:maka.mWpWjAG}), in particular with negative existential verbs (§\ref{sec:negation.existential}).

\begin{exe}
	\ex \label{ex:maka.mWpWjAG}
	\gll   tɕʰeme kʰɤɕkʰɤr ku-kɯ-ɕe maka mɯ-pɯ-jɤɣ \\
	girl man.seating.place \textsc{ipfv}:\textsc{east}-\textsc{genr}:S/O-go at.all \textsc{neg}-\textsc{pst}.\textsc{ipfv}-be.allowed \\
	\glt `Ladies were not allowed to go to the men's seating place.' (31-khAjmu, 39)
\end{exe} 

It is however attested in non-negative sentences as in (\ref{ex:maka.nWnACqe}), in adversative contexts.

\begin{exe}
	\ex \label{ex:maka.nWnACqe}
	\gll  jɤxtsʰi ndɤre maka nɯ-nɤɕqe ra \\
	this.time \textsc{advers} at.all \textsc{imp}-endure[III] be.needed:\textsc{fact} \\
	\glt `This time (unlike the previous times), you have absolutely to bear [the cold of the moon and the heat of the sun without making a word, otherwise we will not succeed].' (tWxtsa, 211)
\end{exe} 

This adverb is built from the root \forme{-ka} found in the distributive determiner  \japhug{tɯka}{each} (§\ref{sec:raNri}) and the distributive pronoun  \japhug{ʑaka}{each his own}  (§\ref{sec:distributive.pronouns}). The first syllable \forme{ma-} resembles the negative prefix \forme{mɤ-} (§\ref{sec:negation}), but given the fact that negative prefixes (§\ref{sec:negation}) are strictly restricted to verb forms, it is not likely that the first syllable of \forme{maka} is from a negative prefix.

Rather, it is from the intensifier \forme{mu},\footnote{This form is related to the denominal verb \japhug{mɤmu}{be the most important}.  } also attested as intensifier `not... at all' in negative contexts (\ref{ex:muZo.me}), in \textit{status constructus} form \forme{mɤ-} (§\ref{sec:vowel.alternations.compounds}), followed by vowel assimilation with the following syllable \forme{-ka}.

\begin{exe}
	\ex \label{ex:muZo.me}
	\gll  nɯ ma kɯ-ra mu ʑo me \\
	\textsc{dem} apart.from \textsc{sbj}:\textsc{pcp}-be.needed at.all \textsc{emph} not.exist:\textsc{fact} \\
	\glt `I don't need anything else at all.' (2003 tWxtsa, 121)
\end{exe} 

We find in addition the compound form \forme{mucin} `not even one at all' (\ref{ex:mucin.Zo.marCAttCi}) from \forme{mu} and  \japhug{cinɤ}{(not) even one} (§\ref{sec:cinA}).

\begin{exe}
	\ex \label{ex:mucin.Zo.marCAttCi}
	\gll  kɯmdza ra mucin ʑo mɤ-arɕɤt-tɕi  \\
	relative \textsc{pl} at.all \textsc{emph} \textsc{neg}-have.a.kinship.relationship:\textsc{fact}-\textsc{1du} \\
	\glt `We don't have any kinship relationship at all.' (12-BzaNsa, 57)
\end{exe} 
