\chapter{Other types of multiclausal constructions} \label{chap:temporal.conditional}


\section{Classification of multiclausal constructions }
This chapter, based on \citet{jacques14linking}, discusses multiclausal constructions other than (headless or noun modifying) relative clauses, complement clauses in core argument function and clauses involving the expression of degree (discussed in chapters \ref{chap:relatives}, \ref{chap:complement.clauses} and \ref{chap:degree}, respectively).\footnote{Some relative and complement clauses in essive function, or with oblique cases, are however discussed in this chapter (§\ref{sec:marked.subordinate}). } 

The constructions treated in this chapter involve different degrees of subordination, from highly dependent (§\ref{sec:marked.subordinate}) to loose parataxis (§\ref{sec:coordination}). This section provides an overview of the criteria that can be used to classify subordination subtypes in Japhug.

\subsection{Marked subordinate clauses}  \label{sec:marked.subordinate}
Some subordinate clauses can be distinguished from main clauses by overt morphological marking. Four types of overt marks of subordination can be distinguished.

First, converbs (§\ref{sec:inf.converb}), found in particular in temporal (§\ref{sec:immediate.subsequence}, §\ref{sec:simultaneity}) and manner (§\ref{sec:manner.converbs}) clauses), being non-finite forms, cannot serve as the predicate of main clauses. 

Second, verb-initial reduplication (§\ref{sec:redp.protasis}) and the prefix \forme{ɯ-} in non-Interrogative clauses (§\ref{sec:interrogative.W}) are non-ambiguous markers of the protasis of a conditional construction (§\ref{sec:real.conditional}).

Third, relator nouns with prenominal complements (§\ref{sec:complement.taking.nouns}) or prenominal relatives (§\ref{sec:Wspa.relative}) in absolutive, locative or ergative form are used to build temporal (§\ref{sec:subsequence.neutral}, §\ref{sec:temporal.reference}) and causality (§\ref{sec:causal.clauses}, §\ref{sec:purposive.clauses}) clauses.

Fourth, postpositions can either contribute to subordination marking in combination with relator nouns (for instance, the ergative in causal clauses, §\ref{sec:causal.clauses}), or serve on their own as subordinating markers, as in the case of the locative position in temporal clauses (\ref{ex:jari.nWtCu}, see also §\ref{sec:temporal.clauses}).

 \begin{exe}
\ex \label{ex:jari.nWtCu}
\gll a-ʁi kɯ [jɤ-ari] nɯtɕu tɕe, nɤkinɯ, qapi jo-rɤmbɯmbri qʰe, tɕendɤre nɯ jo-nɯɴqʰu-j qhe, tɕe kʰa jɤ-azɣɯt-i \\
\textsc{1sg}.\textsc{poss}-younger.sibling \textsc{erg} \textsc{aor}-go[II] \textsc{dem}:\textsc{loc} \textsc{loc} \textsc{filler} white.stone \textsc{ifr}-drop.while.going \textsc{lnk} \textsc{lnk} \textsc{dem} \textsc{ifr}-follow-\textsc{1pl} \textsc{lnk} \textsc{lnk} house \textsc{aor}-reach-\textsc{1pl} \\
\glt `\textbf{When he$_i$ went (to the forest)}, our younger brother$_i$ dropped white stones$_j$ one by one on the way as he was going, and we followed them$_j$ and found our way home.' (160701 poucet2, 42)
\end{exe}

Example (\ref{ex:jari.nWtCu}) also illustrates that in addition to the markers listed above, word order and case marking are important additional clues of subordinating status. The subject \forme{a-ʁi} `my younger brother' takes ergative case following the transitive verb \forme{jo-rɤmbɯmbri} `he dropped them one by one on the way' (§\ref{sec:long.distance.kW}), instead of absolutive case as would be expected if it belonged to the subordinate clause \forme{jɤ-ari nɯtɕu} `when he went there' (since \forme{jɤ-ari} is intransitive), showing that this clause is embedded within the main clause. In some cases the case marking mismatch is the only evidence that a clause is subordinate (§\ref{sec:embedded.clause}).

\subsection{Correlative clauses}  \label{sec:correlative.clauses}
Correlative constructions comprise two or more clauses with a parallel syntactic structure, each obligatorily marked by overt coordinating markers. With the exception of correlative relatives (§\ref{sec:interrogative.relative}), correlative constructions are not very widespread in Japhug: usually only one of the two clauses has obligatory marking.

The correlative additive focus markers \forme{ri} and \forme{tɕi} (§\ref{sec:ri.additive}, §\ref{sec:addition.clauses}) are one of the clearest cases of correlative constructions. As shown by (\ref{ex:pjWnWCWrNYJo.pjWNgra}), they follow the noun phrase on which they have scope, and the rest of the clause, including the verb, can be repeated as in (\ref{ex:WmdoR.tCi.mAkWnaXtCWG}).


\begin{exe}
\ex \label{ex:WmdoR.tCi.mAkWnaXtCWG}
\gll ʑakastaka ɯ-mdoʁ \textbf{tɕi} mɤ-kɯ-naχtɕɯɣ ɣɤʑu, ɯ-tsʰɯɣa \textbf{tɕi} mɤ-kɯ-naχtɕɯɣ ɣɤʑu. \\
each.his.own \textsc{3sg}.\textsc{poss}-colour also \textsc{neg}-\textsc{sbj}:\textsc{pcp}-be.the.same exist:\textsc{sens} \textsc{3sg}.\textsc{poss}-shape also \textsc{neg}-\textsc{sbj}:\textsc{pcp}-be.the.same exist:\textsc{sens} \\
\glt `Each of them (species of starfishes) has its own different colour and different shape.' (180421 haixing, 45)
\end{exe}

Other examples of correlative construction include the relator noun \japhug{ɯ-jɯja}{along with} with incremental initial reduplication (§\ref{sec:redp.gradual.increase}) in the second clause (§\ref{sec:simultaneity}).

\subsection{Periphrastic tenses and subordination}   \label{sec:periphrastic.subordination}
Japhug has quite a few periphrastic TAME categories, combining a verb in the Imperfective or the Factual with a copula (§\ref{sec:ipfv.periphrastic.TAME}, §\ref{sec:irrealis.periphrastic}, §\ref{sec:pst.ifr.ipfv.periphrastic}, §\ref{sec:proximative.periphrastic}). It is common to observe chains of verbs in the Imperfective sharing a single copula. For instance, in (\ref{ex:pjWnWCWrNYJo.pjWNgra}), the copula \forme{ɲɯ-ɕti} in the Sensory has scope over two verbs, \forme{pjɯ-nɯɕɯrɲɟo} and \forme{pjɯ-ŋgra}. Such chains can be long, and comprise more than ten verbs in the Imperfective (example \ref{ex:chain.pjANu}, §\ref{sec:ipfv.periphrastic.TAME}).


\begin{exe}
\ex \label{ex:pjWnWCWrNYJo.pjWNgra}
\gll tɕeri qartsɯ tɕe tɕe ɯ-jwaʁ nɯ pjɯ-nɯɕɯrɲɟo tɕe pjɯ-ŋgra \textbf{ɲɯ-ɕti}. \\
\textsc{lnk} winter \textsc{loc} \textsc{lnk} \textsc{3sg}.\textsc{poss}-leaf \textsc{dem} \textsc{ipfv}-redden \textsc{lnk} \textsc{ipfv}-\textsc{acaus}:cause.to.fall \textsc{sens}-be.\textsc{aff} \\
\glt `In winter, its leaves redden and fall.' (14-sWNgWJu, 12)
\end{exe}   

In (\ref{ex:CtunWrdoR.tundze}), the copula \forme{ɲɯ-ŋu} occurs two times: the first occurrence has scope over one verb (\forme{tu-nɯ-ɬoʁ}), and the second one over two verbs (\forme{ɕ-tu-nɯrdoʁ} and \forme{tu-ndze}).

\begin{exe}
\ex \label{ex:CtunWrdoR.tundze}
\gll  nɯnɯ la-ɕlu-nɯ cʰo la-lɣa-nɯ tɕe tɕe qandʐe tu-nɯ-ɬoʁ \textbf{ɲɯ-ŋu}. tɕe nɯ ɕ-tu-nɯrdoʁ tɕe tu-ndze \textbf{ɲɯ-ŋu}. \\
\textsc{dem} \textsc{aor}:3\flobv{}-plough \textsc{comit}  \textsc{aor}:3\flobv{}-dig \textsc{lnk} \textsc{lnk} earthworm \textsc{ipfv}-\textsc{auto}-come.out \textsc{sens}-be \textsc{lnk} \textsc{dem} \textsc{tral}-\textsc{ipfv}-collect \textsc{lnk} \textsc{ipfv}-eat[III] \textsc{sens}-be \\
\glt `When [people] plough and dig the earth, earthworms come out, and crows go (into the fields) and pick [the earthworms] and eat them (one by one).' (140511 qajdo kW qandzxe tundze, 21-22)
\end{exe}   

Although all verbs in examples (\ref{ex:pjWnWCWrNYJo.pjWNgra}) and (\ref{ex:CtunWrdoR.tundze}) are finite, only the last verb of the chain can be considered to be fully conjugated, while the non-final verbs (\forme{pjɯ-nɯɕɯrɲɟo} in \ref{ex:pjWnWCWrNYJo.pjWNgra} and \forme{ɕ-tu-nɯrdoʁ} in (\ref{ex:CtunWrdoR.tundze}) lack an element of their conjugation. For this reason, it is possible to consider that the last member of a periphrastic TAME chain is the main verb, and that all preceding clauses are subordinate.

\subsection{Unmarked embedded clauses} \label{sec:embedded.clause}
Embedded clauses such as  \forme{tɕʰi a-tɤ-fse-a} in  (\ref{ex:atAfsea.mAwGmtoa}) lack any formal subordinating morphology. However, their subordinating status is shown by the fact that they occur inside the main clause, located between two constituents: in (\ref{ex:atAfsea.mAwGmtoa}), the embedded clause is preceded by the transitive subject \forme{tɤɕime kɯ} (which is not an argument of that clause) and followed by the main verb.


\begin{exe}
\ex \label{ex:atAfsea.mAwGmtoa}
\gll tɤɕime kɯ [tɕʰi a-tɤ-fse-a] tɕe mɤ́-wɣ-mto-a kɯ \\
princess \textsc{erg} what \textsc{irr}-\textsc{pfv}-be.like-\textsc{1sg} \textsc{lnk} \textsc{neg}-\textsc{inv}-see:\textsc{fact}-\textsc{1sg} \textsc{sfp} \\
\glt `What should I do [in order] not to be seen by the princess?' (140505 xiaohaitu-zh, 78)
\end{exe}   

In (\ref{ex:tCe.WmYaR.YAznAmbju}), the clause \forme{ɯ-mɲaʁ ɲɤ-z-nɤmbju} is inserted between the main verb \forme{pjɤ-mto} and its transitive subject \forme{rɟɤlpu nɯ kɯ}. This transitive subject is coreferent with the \textsc{3sg} possessor of \forme{ɯ-mɲaʁ}, object of the verb \forme{ɲɤ-z-nɤmbju} `it dazzled it' in the embedded clause, and the presence of the ergative \forme{kɯ} shows that it receives its case marking from the main verb rather than from the embedded subordinate clause.  
 
\begin{exe}
\ex \label{ex:tCe.WmYaR.YAznAmbju}
\gll tɤ-wi nɯ kɯ tɤ-ri nɯ rɟɤlpu ɯ-tɕʰaʁla scʰiz pjɤ-k-ɤz-nɤndɯndo-ci tɕe, rɟɤlpu nɯ kɯ [ɯ-mɲaʁ ɲɤ-z-nɤmbju] ʑo pjɤ-mto tɕe,  \\
\textsc{indef}.\textsc{poss}-grandmother \textsc{dem} \textsc{erg} \textsc{indef}.\textsc{poss}-thread \textsc{dem} king \textsc{3sg}.\textsc{poss}-yard \textsc{approx}.\textsc{loc} \textsc{ifr}.\textsc{ipfv}-\textsc{peg}-\textsc{prog}-take.here.and.there-\textsc{peg} \textsc{lnk} king  \textsc{dem} \textsc{erg} \textsc{3sg}.\textsc{poss}-eye \textsc{ifr}-\textsc{caus}-be.bright \textsc{emph} \textsc{ifr}-see \textsc{lnk} \\
\glt `The old woman was taking the (well-spun) thread$_i$ here and there in the yard of the king, and the king$_j$ saw it$_i$ as it$_i$ dazzled his$_j$ eyes (i.e. it$_i$ was so well spun that it was dazzling bright).'  (Norbzang 2012, 151-152)
\end{exe} 


Unmarked embedding is also found in serial verb constructions. In (\ref{ex:ki.tuste.luznaRje}), the manner clause with \japhug{stu}{do like} (§\ref{sec:svc.similative.verb}) occurs between the instrument \forme{ɯ-jaʁ kɯ} `with its paw' and \forme{lu-z-naʁje} `it reaches into it'. The presence of the causative prefix \forme{z-} indicates that the instrumental phrase is selected by \forme{lu-z-naʁje} (§\ref{sec:sig.caus.instrumental}), not by \forme{tu-ste}.

\begin{exe}
\ex \label{ex:ki.tuste.luznaRje}
 \gll tɕe ɯ-jaʁ kɯ [ki tu-ste] lu-z-naʁje ɲɯ-ŋu  \\
 \textsc{lnk} \textsc{3sg}.\textsc{poss}-hand \textsc{erg} \textsc{dem}:\textsc{prox} \textsc{ipfv}-do.like[III]  \textsc{ipfv}-reach.into[III] \textsc{sens}-be   \\
\glt `[The cat] reaches with its paw [into the whole] like this.' (27-spjaNkW, 48)
\end{exe}

The examples (\ref{ex:atAfsea.mAwGmtoa}), (\ref{ex:tCe.WmYaR.YAznAmbju}) and (\ref{ex:ki.tuste.luznaRje}) above illustrate that unmarked embedded finite clauses have a considerable variety of semantic functions, including purposive clauses (§\ref{sec:purposive.clauses}), temporal clauses of simultaneity (§\ref{sec:simultaneity}) and manner clauses (§\ref{sec:svc.manner}), respectively.


\subsection{Serial verb constructions} \label{sec:svc}
In Japhug, as in Tshobdun (\citealt[490--491]{sun12complementation}), we find serial verb constructions comprising two verbs sharing TAME category and subject (and often, but not in all cases, objects). 
 
One of the verbs expresses the main action, and the other describes the manner in which the action is performed (§\ref{sec:svc.manner}). Unlike Tshobdun, there is no constraint in Japhug against inserting a linker such as \forme{tɕe} between the two verbs in the serial construction, as shown by example (\ref{ex:ki.tuste.tuZmbri}).\footnote{This construction cannot be considered monoclausal, and thus differs from what is usually understood as `serial verb constructions' in many languages (\citealt[6]{aikhenvald06svc}), but I keep Sun's terminology for want of a better term. }

\begin{exe}
\ex \label{ex:ki.tuste.tuZmbri}
\gll  [ɯ-ʁar nɯ ki tu-ste] tɕe [tu-ʑ-mbri] ɲɯ-ŋu \\
\textsc{3sg}.\textsc{poss}-wing \textsc{dem} \textsc{dem}.\textsc{prox} \textsc{ipfv}-do.like[III] \textsc{lnk} \textsc{ipfv}-\textsc{caus}-make.noise \textsc{sens}-be \\
\glt `[The grasshopper] makes noise by [moving] its wings like this.' (26-kWrNukWGndZWr, 78)
\end{exe}

In some cases, embedding of the first clause into the second one can occur (§\ref{sec:embedded.clause}), suggesting that a syntactic hierarchy exists between them.

There are three main types of serial verb constructions in Japhug, involving deideophonic verbs (§\ref{sec:svc.deideophonic}), similative verbs (§\ref{sec:svc.deideophonic}) and bipartite verbs (treated in §\ref{sec:bipartite} in another chapter). Constructions that are superficially similar to serial verb constructions but better analyzed as finite complement clauses are discussed in §\ref{sec:svc.finite.agreement}.


\subsection{Coordination and parataxis} \label{sec:coordination}
When none of the four set of criteria described above (§\ref{sec:correlative.clauses}, §\ref{sec:periphrastic.subordination}, §\ref{sec:embedded.clause}, §\ref{sec:svc}) are applicable, there remain a residue of clauses in parataxis or linked by \forme{tɕe} (§\ref{sec:tCe.postposition}) or \forme{qʰe}, without a clear subordinating hierarchy. 

In (\ref{ex:pjWsat.juGWt}), all verbs are in the Imperfective, without correlative element (§\ref{sec:correlative.clauses}), final copula (§\ref{sec:periphrastic.subordination}) or embedding (§\ref{sec:embedded.clause}). The verbs \forme{pjɯ-sat} `it kills it' and \forme{ju-ɣɯt} `it brings it' share the same subject and object, but unlike in serial constructions (§\ref{sec:svc}) where both verbs express two aspects of the same action, \forme{pjɯ-sat} and \forme{ju-ɣɯt} refer to two actions occurring one after the other.


\begin{exe}
\ex \label{ex:pjWsat.juGWt}
\gll tɕe ɯ-pɕi ju-ɕe tɕe, ci ci pɣɤtɕɯ pjɯ-sat ju-ɣɯt, ci ci βʑɯ pjɯ-sat ju-ɣɯt, ci ci qaɲi ra pjɯ-sat tɕe ju-ɣɯt. \\
\textsc{lnk} \textsc{3sg}.\textsc{poss}-outside \textsc{ipfv}-go \textsc{lnk} one one bird \textsc{ipfv}-kill \textsc{ipfv}-bring one one mouse \textsc{ipfv}-kill \textsc{ipfv}-bring one one mole \textsc{pl} \textsc{ipfv}-kill \textsc{lnk} \textsc{ipfv}-bring \\
\glt `(When her kittens are hungry), [the cat mother] goes out and sometimes kills a bird and brings it, sometimes kills a mouse and brings it, sometimes kills a mole or something and brings it [to them].' (21-lWLU, 37)
\end{exe}

Simple coordination and parataxis either express temporal subsequence (§\ref{sec:subsequence.neutral}) as in (\ref{ex:pjWsat.juGWt}) between the first clause and all the following ones, and between \forme{pjɯ-sat} and \forme{ju-ɣɯt} in each of the three clauses, or disjunction (§\ref{sec:disjunction.clauses}), as that between the three pairs of clauses in (\ref{ex:pjWsat.juGWt}). 

Coordination with \forme{tɕe} and \forme{qʰe} can in addition be used to indicate logical consequence (§\ref{sec:consequence}) and neutral addition (§\ref{sec:neutral.addition}).

\subsection{Tail-head linkeage} \label{sec:tail.head.linkeage}
Tail-head linkage is a type of linking strategy whereby an element (generally the verb) of one clause  is repeated in the following clause (see \citealt{vries05tailhead} for a typological overview). Such constructions are well-attested in languages of Western Sichuan (see for instance \citealt[688--693]{zhangsh13ersu}). In Japhug, they occur predominantly with parataxis and loose temporal succession linking with finite clauses coordinated by linkers such as \forme{tɕe} or \forme{qʰe}. It is a very common strategy both for ensuring narrative coherence, and providing time for the storyteller to prepare the narration of the following events without hesitating and using speech fillers (§\ref{sec:fillers}). 

Tail-head linkage can involve an entire sentence, as  in (\ref{ex:YAmbinW.x2}), but often leaves out a constituent: for instance in (\ref{ex:sWNgW.zW.joCe2}) the intransitive subject \forme{tɤ-pɤtso nɯnɯ} is not repeated.

\begin{exe}
\ex \label{ex:YAmbinW.x2}
\gll [nɯ-me stu kɯ-xtɕi nɯ ɲɤ-mbi-nɯ], tɕe [nɯ-me stu kɯ-xtɕi nɯ ɲɤ-mbi-nɯ tɕe], tɕe tɕʰeme nɯ to-nɯmbrɤpɯ, \\
\textsc{3pl}.\textsc{poss}-daughter most \textsc{sbj}:\textsc{pcp}-be.small \textsc{dem} \textsc{ifr}-give-\textsc{pl}  \textsc{lnk} \textsc{3pl}.\textsc{poss}-daughter most \textsc{sbj}:\textsc{pcp}-be.small \textsc{dem} \textsc{ifr}-give-\textsc{pl}  \textsc{lnk}  \textsc{lnk} girl \textsc{dem} \textsc{ifr}-ride \\
\glt `They gave [him] their daughter (in marriage), and as they gave [him] their daughter (in marriage),  the girl mounted [a horse].' (2002 qaCpa, 62)
\end{exe}


\begin{exe}
\ex \label{ex:sWNgW.zW.joCe2}
\gll tɕendɤre [tɤ-pɤtso nɯnɯ li sɯŋgɯ zɯ jo-ɕe]. [sɯŋgɯ zɯ jo-ɕe] tɕe tɕendɤre, pʰaʁrgot nɯ kɯ tɤ-pɤtso nɯ pa-mto tɕe \\
  \textsc{dem} boy \textsc{dem} again forest \textsc{loc} \textsc{ifr}-go  forest \textsc{loc} \textsc{ifr}-go \textsc{lnk} \textsc{lnk} boar \textsc{dem} \textsc{erg} boy \textsc{dem} \textsc{ifr}:3\flobv{}-see \textsc{lnk}   \\
\glt `The boy went again into the forest, and as he went into the forest, the boar saw the boy.' (140428 yonggan de xiaocaifeng-zh, 232-233)
\end{exe}

In some cases, the verb form is slightly different in the repeated clause. For instance in (\ref{ex:kowGsAmdzWnW.kasAmdzW}), the verb form in the first clause is in inverse configuration 3$'$\fl{}\textsc{3pl} (the TAME is probably Inferential, though in this context the contrast between Inferential and Aorist is neutralized, §\ref{sec:allomorphy.inv}) while that in the second clause is in direct configuration \textsc{3sg}\flobv{}.\footnote{Aorist is used in the second clause of (\ref{ex:kowGsAmdzWnW.kasAmdzW}) to mark a point of temporal reference, §\ref{sec:aor.temporal}, §\ref{sec:temporal.reference}.}

\begin{exe}
\ex \label{ex:kowGsAmdzWnW.kasAmdzW}
\gll sqʰi ɯ-rkɯ nɯtɕu kó-wɣ-sɯ-ɤmdzɯ-nɯ. sqʰi ɯ-rkɯ nɯtɕu ka-sɯ-ɤmdzɯ tɕe, (...) ra to-ti \\
tripod \textsc{3sg}.\textsc{poss}-side \textsc{dem}:\textsc{loc} \textsc{ifr}-\textsc{inv}-\textsc{caus}-sit-\textsc{pl} tripod \textsc{3sg}.\textsc{poss}-side \textsc{dem}:\textsc{loc} \textsc{aor}:3\flobv{}-\textsc{caus}-sit \textsc{lnk} { } \textsc{pl} \textsc{ifr}-say \\
\glt `[The woman] had them sit by the hearth. She had them sit by the hearth, and said (...).' (160703 poucet3, 48-49)
\end{exe}

%{sec:cataph.pron} \forme{nɤki}
\section{Conditional constructions} \label{sec:conditionals}
Conditional constructions are biclausal, comprising a subordinate clause (the protasis) and a main clause (the apodosis). The apodosis describes a result which takes place if the condition in the protasis is fulfilled. Depending on whether the protasis is a fact or a hypothetical situation, several types of conditionals can be distinguished: real, concessive, counterfactual and hypothetical. Some temporal clauses, such as iterative coincidence (§\ref{sec:iterative.coincidence.clause}), can also be considered to be a subtype of conditionals, but are treated in §\ref{sec:temporal.clauses}.

\subsection{Real conditionals} \label{sec:real.conditional}
In real conditional constructions, the verb in the protasis has either initial reduplication (§\ref{sec:redp.protasis}) or the Interrogative \forme{ɯ-} prefix (§\ref{sec:interrogative.W.function}) and is followed the additive postposition \forme{nɤ} (§\ref{sec:additive.nA}),  as shown by (\ref{ex:padma.tWtWNu.nA}) and (\ref{ex:mWmAjAtWGWt.nA}) on the one hand, and (\ref{ex:Wtope.nA}) on the other without semantic difference.

Non-past TAME categories, in particular the Factual Non-Past (\ref{ex:padma.tWtWNu.nA}) or the Sensory (\ref{ex:YWYWmaR.nA}) are used when the protasis concern a present state or an ongoing action whose truth value is known to the speaker, as in (\ref{ex:padma.tWtWNu.nA}) and (\ref{ex:YWYWmaR.nA}).

\begin{exe}
\ex \label{ex:padma.tWtWNu.nA}
 \gll [pɤnmawombɤr tɯ\redp{}tɯ-ŋu] nɤ, pɯ-ta-sɯxɕɤt nɯ nɯ-ndɯn ra \\
  \textsc{anthr} \textsc{cond}\redp{}2-be:\textsc{fact} \textsc{add} \textsc{aor}-1\fl{}2-teach \textsc{dem} \textsc{imp}-read be.needed:\textsc{fact} \\
 \glt `If you (really) are Padma 'Od'bar, recite [the mantra] that I have taught you.' (Norbzang 2005, 249)
\end{exe}

\begin{exe}
\ex \label{ex:YWYWmaR.nA}
 \gll [tɯrme ɲɯ\redp{}ɲɯ-maʁ] nɤ, a-ku ɲɯ-sɤpʰar-a ŋu tɕe tɕe, jɤ-pʰɣo \\
 person \textsc{cond}\redp{}\textsc{sens}-not.be \textsc{add} \textsc{1sg}.\textsc{poss}-head \textsc{ipfv}-shake-\textsc{1sg} be:\textsc{fact} \textsc{lnk} \textsc{lnk} \textsc{imp}-flee \\
\glt `If (the thing in the tree) is not a man, I will shake my head, and you (had better) flee.' (khu 2012, 50)
\end{exe}

The Aorist in the protasis (§\ref{sec:aor.cond}) can express potential future events  (\ref{ex:mWmAjAtWGWt.nA}), but also generic conditions (\ref{ex:mWmAnWwGsWqhrWt.tCe}).

\begin{exe}
\ex \label{ex:mWmAjAtWGWt.nA}
 \gll [tɯrɟɯ laχtɕʰa cʰo rŋɯl ra mɯ\redp{}mɤ-jɤ-tɯ-ɣɯt] nɤ, nɤkinɯ, pjɯ-ta-sat ŋu \\
 riches thing \textsc{comit} silver \textsc{pl} \textsc{cond}\redp{}\textsc{neg}-\textsc{aor}-2-bring \textsc{add} \textsc{filler} \textsc{ipfv}-1\fl{}2-kill be:\textsc{fact} \\
 \glt `If you do not bring goods and money, we will kill you.' (160706 poucet6, 108)
 \end{exe}

The Inferential occurs in the two protases in (\ref{ex:Wtope.nA}) in a generic context to specifically express a condition which only becomes testable after a special procedure (removing bandages) has been applied, when the action or change of state in the protasis cannot be directly observed.

\begin{exe}
\ex \label{ex:Wtope.nA}
 \gll   kɯɕnɯ-rʑaʁ jamar tɤ-tsu tɕe tɕe pjɯ-rle tɕe tu-rtoʁ tɕe, tɕe ɯ-tó-pe nɤ tɕe nɯ ma mɤ-ra,
ɯ-mɯ́-to-pe nɤ tɕe li tu-xtɕɤr ŋu. \\
seven-day about \textsc{aor}-pass \textsc{lnk} \textsc{lnk} \textsc{ipfv}-untie[III] \textsc{lnk} \textsc{ipfv}-look \textsc{lnk} \textsc{lnk} \textsc{qu}-\textsc{ifr}-be.good \textsc{add} \textsc{lnk} \textsc{dem} apart.from \textsc{neg}-be.needed:\textsc{fact} \textsc{qu}-\textsc{neg}-\textsc{ifr}-be.good \textsc{add} \textsc{lnk} again \textsc{ipfv}-attach be:\textsc{fact} \\
\glt `After seven days, [the doctor] unties [the splint] and looks, if [the fracture] has healed, [the splint] is not needed anymore, if it has not improved, [the doctor] attaches [the splint] again.' (140426 laxthab, 11-13)
\end{exe}
 
Although the postposition coming after the verb in the protasis is almost always \forme{nɤ} as in the examples above, this is not a requirement; for instance in (\ref{ex:mWmAnWwGsWqhrWt.tCe}) the form \forme{mɯ\redp{}mɤ-nɯ́-wɣ-sɯ-qʰrɯt} `if one does not scrape it with it' in the protasis is rather followed by \forme{qʰe}.

 \begin{exe}
\ex \label{ex:mWmAnWwGsWqhrWt.tCe}
 \gll tɕe [tɯ-ndzrɯ kɯ mɯ\redp{}mɤ-nɯ́-wɣ-sɯ-qʰrɯt] qʰe mɯ́j-ŋgra. \\
 \textsc{lnk} \textsc{genr}.\textsc{poss}-nail \textsc{erg} \textsc{cond}\redp{}\textsc{neg}-\textsc{aor}-\textsc{inv}-\textsc{caus}-scrape \textsc{lnk} \textsc{neg}:\textsc{sens}-\textsc{acaus}:cause.to.fall \\
 \glt `[Nits] (are firmly attached and) will not detach unless one scrapes them with one's nail.' (21-mdzadi, 117)
 \end{exe}

In reduplicated conditional forms, the negative prefix \forme{mɯ-} has the special form \forme{mɯ\redp{}mɤ-} (§\ref{sec:neg.allomorphs}), while no such vowel alternation occurs with negative prefixes in Interrogative forms (\forme{ɯ-mɯ́-to-pe} in \ref{ex:Wtope.nA}) or suppletive negative verbs (\ref{ex:YWYWmaR.nA}).

Apart from reduplicated conditional and Interrogative, a third way of marking the protasis in a real conditional construction is the Irrealis (§\ref{sec:irrealis.conditional}). While the Irrealis usually marks counterfactuals (§\ref{sec:counterfactual}) conditionals, it also occurs in real conditionals as in (\ref{ex:anWYatnW.tCe}) and (\ref{ex:atAngo.tCe}), in particular in the case of possible but not highly frequent events. The Irrealis only very rarely followed by \forme{nɤ}, and most commonly occurs with the linker \forme{tɕe}.

\begin{exe}
\ex \label{ex:anWYatnW.tCe}
 \gll a-nɯ-ɲat-nɯ tɕe tɯ-tɕʰa nɤ tɯ-tɕʰa nɯ, nɤki <dianxian> ɯ-taʁ,  qʰe sɯku ɯ-taʁ nɯtɕu tu-nɯna-nɯ tɕe \\
 \textsc{irr}-\textsc{pfv}-be.tired-\textsc{pl} \textsc{lnk} one-pair \textsc{lnk} one-pair \textsc{dem} electric.wire \textsc{3sg}.\textsc{poss}-on \textsc{lnk} treetop \textsc{3sg}.\textsc{poss}-on \textsc{dem} \textsc{loc} \textsc{ipfv}-rest-\textsc{pl} \textsc{lnk} \\
\glt `If/Whenever  [the swallows] are tired (from flying), they rest in pairs on electric wires or on trees.' (03-mWrmWmbjW-zh, 54-55)
\end{exe}


\begin{exe}
\ex \label{ex:atAngo.tCe}
 \gll kʰɯna nɯ a-tɤ-ngo tɕe tɕendɤre ɲɯ-sɲu kɯ-fse ɲɯ-ŋu \\
 dog \textsc{dem} \textsc{irr}-\textsc{pfv}-be.ill \textsc{lnk} \textsc{lnk} \textsc{ipfv}-be.mad \textsc{sbj}:\textsc{pcp}-be \textsc{sens}-be \\
\glt `If a dog gets ill (from rabbies), it will become like mad.' (29-chWsYu, 3-4)
\end{exe}

 % ɯ-ɲɯ́-wɣ-sat-a nɤ a-pɯ́-wɣ-nɯ-sat-a 
 
 
\subsection{Necessary condition} \label{sec:only.if}
Necessary condition (`only if') can be expressed using the postposition \japhug{kóʁmɯz}{only after} (§\ref{sec:temporal.postpositions}, §\ref{sec:immediate.subsequence}) as connecting element between the two clauses. The protasis selects the Irrealis, the apodosis is in the Imperfective and the whole construction is under the scope of the modal verb \japhug{ra}{be needed} as in  (\ref{ex:nWkoRmWznA.only.if}) and (\ref{ex:nWkoRmWznA.only.if2}).

\begin{exe}
	\ex \label{ex:nWkoRmWznA.only.if}
	\gll nɯ kɤsɯfse kɯ a-pɯ-tsʰi-nɯ tɕe nɯ 	kóʁmɯz nɤ kɤsɯfse cʰɯ-tsʰu-nɯ ra tɕe,  \\
	\textsc{dem} \textsc{all} \textsc{erg} \textsc{irr}-\textsc{pfv}:\textsc{down}-drink-\textsc{pl} \textsc{lnk} \textsc{dem}  only.then \textsc{add} all \textsc{ipfv}-be.fat-\textsc{pl} be.needed:\textsc{fact} \textsc{lnk} \\
	\glt `It is only if all [of the pigs] get to eat (hogwash) that they will all grow fat.' 	(i.e. a person has to be appointed to prevent the pigs from fighting, otherwise the weaker pigs would get nothing to eat).' (160708 paRtshi WkWrWru, 13)
\end{exe}

\begin{exe}
	\ex \label{ex:nWkoRmWznA.only.if2}
	\gll zraβ a-pɯ-tu kóʁmɯz nɤ tsʰɤnmu cʰɯ-rɤpɯ tɕe tɕe nɯ ɲɯ-mpʰɯl ra.\\
	male.goat \textsc{irr}-\textsc{ipfv}-exist only.then \textsc{add} female.goat \textsc{ipfv}-bear.young \textsc{lnk} \textsc{lnk} \textsc{dem} \textsc{ipfv}-reproduce be.needed:\textsc{fact} \\
	\glt `Nanny goats can bear young and reproduce only if a male goat is present.' (05-qaZo, 6)
\end{exe}
 
 \subsection{Concessive conditional} \label{sec:concessive.conditional}
There are three types of concessive conditional constructions: scalar (`even if'), alternative (`whether ... or') and universal (`whatever, whenever, wherever etc') in Japhug. These conditional present two morphological commonalities concerning the verb in the protasis. 

First, in the majority of cases, this verb  takes the Autive prefix \forme{-nɯ-}  §\ref{sec:autoben.spontaneous}, \citealt[298--300]{jacques14linking}), sometimes with emphatic gemination to \forme{-nnɯ-}. Second, it never takes  conditional initial reduplication (§\ref{sec:redp.protasis}) or the Interrogative prefix (§\ref{sec:interrogative.W.function}) unlike the verb in the protasis of real conditionals (§\ref{sec:real.conditional}).  

 
\subsubsection{Scalar concessive conditional} \label{sec:scalar.concessive.conditional}
In scalar concessive conditionals, the protasis is followed by the scalar focus marker \japhug{kɯnɤ}{also, even} (§\ref{sec:kWnA}), and contains a verb  in the Past Imperfective (\ref{ex:nWkWGAdi.kWnA}) or the Imperfective (\ref{ex:chWwGnWBlW.kWnA}) with the autive prefix (§\ref{sec:autoben.spontaneous}).
 
\begin{exe}
\ex \label{ex:nWkWGAdi.kWnA}
 \gll tɤ-mtʰɯm ndɤre, [nɯ-kɯ-ɣɤdi pɯ-nɯ-ŋu] kɯnɤ tu-ndze ɕti. \\
 \textsc{indef}.\textsc{poss}-meat \textsc{lnk} \textsc{aor}-\textsc{sbj}:\textsc{pcp}-be.smelly \textsc{pst}.\textsc{ipfv}-\textsc{auto}-be also \textsc{ipfv}-eat[III] be.\textsc{aff}:\textsc{fact} \\
\glt `[Crows] eat meat even if it has become smelly.' (22-qajdo, 21)
 \end{exe}
 
\begin{exe}
\ex \label{ex:chWwGnWBlW.kWnA}
 \gll   [cʰɯ́-wɣ-nɯ-βlɯ] kɯnɤ, tu-nɯt ʁo ŋu ri, ɯ-ʁrɤt nɯ ɲaʁ ʑo qʰe, maka ɲɯ-ɣɤ-mpje mɤ-cʰa. \\
  \textsc{ipfv}-\textsc{inv}-burn also \textsc{ipfv}-be.ignited \textsc{advers} be:\textsc{fact} \textsc{lnk}
\textsc{3sg}.\textsc{poss}-charcoal \textsc{dem}  be.black:\textsc{fact} \textsc{emph} \textsc{lnk} at.all \textsc{ipfv}-\textsc{caus}-be.warm[III] \textsc{neg}-can:\textsc{fact} \\
 \glt Even when one burns it, although it does ignite, its charcoal is black and it does not warm anything. (17-thowum, 8-10)
\end{exe}
 
 \subsubsection{Alternative concessive conditional} \label{sec:alt.concessive.conditional}
Alternative concessive conditional constructions express that the outcome in apodosis will occur irrespective of a list of alternative possibilities, indicated by several protases. 


 In (\ref{ex:pWnnWNu4}), each of the protases contains copula \forme{pɯ-nɯ-ŋu} in Past Imperfective (§\ref{sec:pst.ifr.ipfv.morphology}) Autive (§\ref{sec:autoben.spontaneous}) form, and share a single apodosis. In the clause \forme{qajɯ kɯ tu-ndze pɯ-nnɯ-ŋu}, this copula is combined with the Imperfective verb \forme{tu-ndze} `it eats it' to build a Periphrastic Past Imperfective construction (§\ref{sec:pst.ifr.ipfv.periphrastic}).

\begin{exe}
\ex  \label{ex:pWnnWNu4}
\gll [tɯ-ɕɣa pɯ-kɯ-ɴɢrɯ pɯ-nnɯ-ŋu], [pɯ-kɯ-ɣɤtsɯr pɯ-nnɯ-ŋu] qʰe, [qajɯ kɯ tu-ndze pɯ-nnɯ-ŋu], nɯfse tu-kɯ-mŋɤm pɯ-nnɯ-ŋu], nɯnɯ kɯ wuma ʑo nɯsmɤn. \\
\textsc{indef}.\textsc{poss}-tooth \textsc{ipfv}-\textsc{sbj}:\textsc{pcp}-\textsc{acaus}:break \textsc{pst}.\textsc{ipfv}-\textsc{auto}-be
\textsc{aor}-\textsc{sbj}:\textsc{pcp}-crack \textsc{pst}.\textsc{ipfv}-\textsc{auto}-be \textsc{lnk}
bug \textsc{erg} \textsc{ipfv}-eat[III] \textsc{pst}.\textsc{ipfv}-\textsc{auto}-be 
like.that \textsc{ipfv}-\textsc{sbj}:\textsc{pcp}-hurt \textsc{pst}.\textsc{ipfv}-\textsc{auto}-be
\textsc{dem} \textsc{erg} very \textsc{emph} heal:\textsc{fact} \\
\glt `Whether one's tooth is broken, cracked, decayed or whether it simply hurts, he (a particular dentist) treats it very well.' (27-tApGi, 142-145)
\end{exe}

The presence of a periphrastic TAME category with a copula is not required. In (\ref{ex:kAnWrNgWnW.3}), the protases contain an Aorist Autive verb form \forme{kɤ-nɯ-rŋgɯ-nɯ}.

\begin{exe}
\ex \label{ex:kAnWrNgWnW.3}
\gll [stɤmku kɯ-fse kɤ-nɯ-rŋgɯ-nɯ], [sɯku ɯ-pa kɤ-nɯ-rŋgɯ-nɯ], [praʁ-pa kɤ-nɯ-rŋgɯ-nɯ], ɯnɯnɯ nɯ-ŋga ɯ-taʁ pjɯ-ta-nɯ. \\
pasture \textsc{sbj}:\textsc{pcp}-be.like \textsc{aor}-\textsc{auto}-lie.down-\textsc{pl} tree.top \textsc{3sg}.\textsc{poss}-under  \textsc{aor}-\textsc{auto}-lie.down-\textsc{pl} cliff-under \textsc{sbj}:\textsc{pcp}-be.like \textsc{aor}-\textsc{auto}-lie.down-\textsc{pl} \textsc{dem} \textsc{3pl}.\textsc{poss}-clothes \textsc{3sg}.\textsc{poss}-on \textsc{ipfv}-put-\textsc{pl} \\
\glt `Whether [travelers] sleep in pastures, under trees or in caves, they put it on their blankets.' (30-mboR, 38-39)
\end{exe} 

The alternative is often between the affirmative and negative versions of the same event (polar alternative concession), a meaning close to that of the scalar concessive conditional. In such cases, the conditional constructions has two protases, the first in affirmative form, and the second with the corresponding negative form as in (\ref{ex:panWri.nA}). The same apodosis can be repeated after each protasis.
 

\begin{exe}
\ex \label{ex:panWri.nA}
\gll [tɯ-sɯm pɯ-a<nɯ>ri] nɤ ju-kɯ-ɕe, [mɯ-pɯ-a<nɯ>ri] nɤ ju-kɯ-ɕe pɯ-ra \\
\textsc{genr}.\textsc{poss}-mind \textsc{pst}.\textsc{ipfv}-<\textsc{auto}>go[II] \textsc{lnk} \textsc{ipfv}-\textsc{genr}:S/O-go, \textsc{neg}-\textsc{pst}.\textsc{ipfv}-<\textsc{auto}>go[II] \textsc{lnk} \textsc{ipfv}-\textsc{genr}:S/O-go \textsc{pst}.\textsc{ipfv}-be.needed \\
\glt `One had to go whether one liked it or not.' (14-siblings, 215)
\end{exe} 
 
The alternative concessive conditional is one of the few constructions where even transitive verbs such as \japhug{nɤla}{agree} occur in the Past Imperfective (§\ref{sec:pst.ifr.ipfv.morphology}).\footnote{The verb \japhug{nɤla}{agree} selects the \textsc{upwards} orientation, as shown by the form \forme{to-nɤla} `he agreed' in example (\ref{ex:mAkArWsWso.kW.Zo}), §\ref{sec:inf.converb}. } 

\begin{exe}
\ex  \label{ex:pannAla.nA}
\gll  [pa-n-nɤla] ɕe-a, [mɯ-pa-n-nɤla] nɤ ɕe-a ra    \\
 \textsc{pst}.\textsc{ipfv}:3\fl{}3-\textsc{auto}-agree \textsc{lnk} \textsc{ipfv}:go-\textsc{1sg}
  \textsc{neg}-\textsc{pst}.\textsc{ipfv}:3\fl{}-\textsc{auto}-agree \textsc{lnk} \textsc{ipfv}:go-\textsc{1sg} be.needed:\textsc{fact} \\
\glt `I will go whether he agrees or not.' (elicited)
\end{exe}

A more concise type of alternative concessive construction shown in (\ref{ex:tanAla.pWnWNu}) avoids repeating the content of the protasis and of the apodosis, and marks the alternative by combining the affirmative (\forme{pɯ-nɯ-ŋu}) and negative (\forme{pɯ-nɯ-maʁ}) copulas after the verb in the protasis.

\begin{exe}
\ex  \label{ex:tanAla.pWnWNu}
\gll  [ta-nɤla pɯ-nɯ-ŋu pɯ-nɯ-maʁ] ɕe-a ra \\
\textsc{aor}:3\fl{}3-agree \textsc{pst}.\textsc{ipfv}-\textsc{auto}-be \textsc{pst}.\textsc{ipfv}-\textsc{auto}-not.be \textsc{ipfv}:go-\textsc{1sg} be.needed:\textsc{fact} \\
\glt `I will go whether he agrees or not.' (elicited)
\end{exe}
 
Another way to indicate alternative concession is the interrogative particle \forme{ɕi} (§\ref{sec:fsp.interrog}) as in (\ref{ex:pjWnAndAG}).

\begin{exe}
\ex  \label{ex:pjWnAndAG}
\gll [nɯŋa ŋu] ɕi, [mbro ŋu] ma, pjɯ-nɤndɤɣ ɲɯ-ŋgrɤl  \\
cow be:\textsc{fact} \textsc{sfp} horse  be:\textsc{fact} \textsc{lnk} \textsc{ipfv}-be.poisoned \textsc{sens}-be.usually.the.case \\
\glt `Whether it is a cow or a horse, they end up poisoned.' (25-qarmWrwa, 21)
\end{exe}

In addition to these constructions, polar alternative concession `whether or not' can be expressed by a construction combining a negative existential verb with biclausal complements, comprising the same verb in bare root form followed by the bare root prefixed with the negative prefix \forme{mɤ\trt}, as in (\ref{ex:Ce.mACe.tWme}). This unusual construction is discussed in more detail in §\ref{sec:bare.inf.negative}.

\begin{exe}
\ex  \label{ex:Ce.mACe.tWme}
\gll [[ɕe] [mɤ-ɕe] tɯ-me] ma kɤ-nɤma tɯ-cʰa me qʰe naχtɕɯɣ ɕti\\
\textsc{bare}.\textsc{inf}:go \textsc{neg}-\textsc{bare}.\textsc{inf}:go 2-not.exist:\textsc{fact} \textsc{lnk} \textsc{inf}-work 2-can:\textsc{fact} not.exist:\textsc{fact} \textsc{lnk} be.the.same:\textsc{fact} be:\textsc{fact} \\
\glt `[It does not matter] whether you go or not, you are not able to do anything, it amounts to the same.' (elicited)
\end{exe}

\subsubsection{Universal concessive conditional} \label{sec:universal.concessive.conditional}
Universal concessive conditional constructions comprise a protasis with an interrogative pronoun (in free-choice indefinite function, §\ref{sec:interrogative.indef}, §\ref{sec:headless.relatives.quantification}) and a main verb in the Aorist or Past Imperfective. In addition, the autive (§\ref{sec:autoben.spontaneous}) occurs either in the protasis, with emphatic reduplication as (\ref{ex:cai.tAwGnWBzWBzu}) or (\ref{ex:tChi.pWnWfsAfse.Zo}) or in the main clause as in (\ref{ex:thAjtCu.WRjiz.tAGe}). The emphatic marker \forme{ʑo} (§\ref{sec:emphatic.Zo}) often follows the subordinate clause.


\begin{exe}
\ex \label{ex:cai.tAwGnWBzWBzu}
\gll kɯmɕku nɯ tɕe, ɯ-dɯχɯn wuma ʑo mɯm tɕe, [...]  [tɕʰi <cai> tɤ́-wɣ-nɯ-βzɯ\redp{}βzu] ʑo pjɯ-tu ra. \\
garlic \textsc{dem} \textsc{lnk} \textsc{3sg}.\textsc{poss}-fragrance really \textsc{emph} be.tasty:\textsc{fact} \textsc{lnk} { } what dish \textsc{aor}-\textsc{inv}-\textsc{auto}-\textsc{emph}\redp{}make \textsc{emph} \textsc{ipfv}-exist  be.needed:\textsc{fact} \\
\glt `Garlic has a nice smell, (...), whatever dish one prepares, there has to be [garlic] (in it).' (07-kWmCku, 26)
\end{exe}

 \begin{exe}
\ex \label{ex:thAjtCu.WRjiz.tAGe}
\gll  tɕʰomba tɕe [tʰɤjtɕu ɯ-ʁjiz tɤ-ɣe] qʰe ju-nnɯ-ɣi ɕti ma \\
cold \textsc{lnk} when \textsc{3sg}.\textsc{poss}-wish \textsc{aor}-come[II] \textsc{lnk} \textsc{ipfv}-\textsc{auto}-come be.\textsc{aff}:\textsc{fact} \textsc{lnk} \\
\glt `A cold (the disease) comes whenever it wants (i.e. it occurs spontaneously).' (22-tAmbrWm, 7)
\end{exe}

 \begin{exe}
\ex \label{ex:tChi.pWnWfsAfse.Zo}
\gll [tɕʰi pɯ-nɯ-fsɤ\redp{}fse] ʑo nɤ-rca tu-kɯ-tsɯm-a ra \\
what \textsc{pst}.\textsc{ipfv}-\textsc{auto}-\textsc{emph}\redp{}be.like \textsc{emph} \textsc{2sg}.\textsc{poss}-together \textsc{ipfv}-2\fl{}1-take.away-\textsc{1sg} be.needed:\textsc{fact} \\
\glt `In any case (whatever the circumstances are like), take me with you.' (07-deluge, 59)
\end{exe}

Examples of universal concessive conditionals without autive prefix are however also attested, as in (\ref{ex:NotCu.nWnWlhWlhoR}).

\begin{exe}
\ex \label{ex:NotCu.nWnWlhWlhoR}
\gll [ŋotɕu nɯ-ɬɯ\redp{}ɬoʁ] ʑo wuma ʑo sɤɣdɯɣ \\
where \textsc{aor}-\textsc{emph}\redp{}come.out \textsc{emph} really \textsc{emph} be.annoying:\textsc{fact} \\
\glt `No matter where it grows, it is very annoying.' (5-khArWm, 19)
\end{exe}

Universal concessive conditionals are semantically close to correlative relative clauses with free-choice interrogative pronouns (§\ref{sec:interrogative.relative}) such as (\ref{ex:tChi.tWtAstuta}). 

\begin{exe}
\ex \label{ex:tChi.tWtAstuta}
\gll kɯz tɕe [aʑo tɕʰi tɯ\redp{}tɤ-stu-t-a] nɯ tɤ-ste je \\
\textsc{interj} \textsc{lnk} \textsc{1sg} what \textsc{total}\redp{}\textsc{aor}-do.like-\textsc{pst}:\textsc{tr}-\textsc{1sg} \textsc{dem} \textsc{imp}-do.like[III] \textsc{sfp} \\
\glt `Come on, do everything in the same way as me.' (i.e. in whatever way I act, act in this way) (140511 xinbada-zh, 252)
\end{exe}

First, the verb \forme{tɯ\redp{}tɤ-stu-t-a} totalitative initial reduplication (§\ref{sec:totalitative.relatives}), a specificity of relative clauses. 

Second, the interrogative pronoun \forme{tɕʰi} has a different syntactic function in these constructions. In the correlative relative (\ref{ex:tChi.tWtAstuta}), \forme{tɕʰi} is the relativized element, and serves as direct object of both the verb in the subordinate clause and that in the main clause.\footnote{See §\ref{sec:ditransitive.secundative} for an account of the argument structure of the similative verb \japhug{stu}{do like}. } In the universal concessive conditionals (\ref{ex:cai.tAwGnWBzWBzu}) and (\ref{ex:tChi.pWnWfsAfse.Zo}), \forme{tɕʰi} in the subordinate clauses has no syntactic role in the main clause.
  
 \subsection{Counterfactuals} \label{sec:counterfactual}
In counterfactual constructions, the protasis describes a condition that is known to be false, and the apodosis indicates an outcome that would have occurred if the condition had been true. In Japhug, counterfactual conditionals have strict requirements on TAME marking in both the protasis and the apodosis. The verb in the protasis is in the  Irrealis (§\ref{sec:irrealis.conditional}), and the Past Imperfective is required in the apodosis even for transitive dynamic verbs (§\ref{sec:pst.ifr.ipfv.apodosis}), as illustrated by (\ref{ex:GWtukWqura.apWNu.pWnApeta}) and (\ref{ex:apWtu.apWNua.pWra}).\footnote{In a previous publication (\citealt[301]{jacques14linking}), I claimed that there were counterfactual constructions with a verb in the Factual Non-Past in the apodosis (example 99), but this was an erroneous interpretation.  }
 
 \begin{exe}
\ex \label{ex:GWtukWqura.apWNu.pWnApeta}
\gll [aʑo nɤʑo kɯ iɕqʰa nɯtɕu ɣɯ-tu-kɯ-qur-a a-pɯ-ŋu] tɕe, nɯnɯ wuma ʑo \textbf{pɯ-nɤ-pe-t-a} ma \\
\textsc{1sg} \textsc{2sg} \textsc{erg} just.before \textsc{dem}:\textsc{loc} \textsc{cisl}-\textsc{ipfv}-2\fl{}1-help-\textsc{1sg} \textsc{irr}-\textsc{pst}.\textsc{ipfv}-be \textsc{lnk} \textsc{dem} really \textsc{emph} \textsc{pst}.\textsc{ipfv}-\textsc{trop}-be.good-\textsc{pst}:\textsc{tr}-\textsc{1sg} \textsc{lnk} \\
\glt `If you had come and helped me right before, I would have appreciated it.' (140427 liangge shibiang he qiangdao-zh, 24-25)
\end{exe}
 
\begin{exe}
\ex \label{ex:apWtu.apWNua.pWra}
\gll tʰam tɕe nɤki χawo [aʑo a-mi a-pɯ-tu] tɕe, nɯ rɟɤlpu ɯ-tɕɯ nɯ ɯ-rkɯ nɯtɕu kɯ-rɤʑi nɯ aʑo \textbf{a-pɯ-ŋu-a} \textbf{pɯ-ra} \\
now \textsc{lnk} \textsc{filler} if.only \textsc{1sg} \textsc{1sg}.\textsc{poss}-leg \textsc{irr}-\textsc{ipfv}-exist \textsc{lnk} \textsc{dem} king \textsc{3sg}.\textsc{poss}-son \textsc{dem} \textsc{3sg}.\textsc{poss}-side \textsc{dem}:\textsc{loc} \textsc{sbj}:\textsc{pcp}-stay \textsc{dem} \textsc{1sg} \textsc{irr}-\textsc{ipfv}-be-\textsc{1sg} \textsc{pst}.\textsc{ipfv}-be.needed \\
\glt `If only I had legs now, the one next to the prince would have had to be me.' (said by the little mermaid, who has no legs) (150819 haidenver-zh, 149)
\end{exe}

  
 
 \section{Temporal clauses} \label{sec:temporal.clauses}

 

 \subsection{Iterative coincidence} \label{sec:iterative.coincidence.clause}
Iterative coincidence is a biclausal construction comprising a temporal subordinate clause ($A$) and a main clause ($B$), expressing that each time the event in $A$ takes place, that in $B$ necessarily follows, and that this has taken place several times in the past. It can be generally translated as `each time $A$ then $B$'. As shown by (\ref{ex:jWjAGenW.Zo}) and (\ref{ex:tWmW.kWkAlAt}), the temporal clause generally takes a verb in the Aorist with initial reduplication (§\ref{sec:iterative.coincidence}) like the protasis of a conditional construction (§\ref{sec:redp.protasis}, §\ref{sec:real.conditional}), followed by the emphatic marker \forme{ʑo} (§\ref{sec:emphatic.Zo}).

 
\begin{exe}
\ex \label{ex:jWjAGenW.Zo}
\gll [tɯrme ra jɯ\redp{}jɤ-ɣe-nɯ] ʑo tu-nɯrmɤʑu pjɤ-ŋu tɕe \\
people \textsc{pl}  \textsc{iter}\redp{}\textsc{aor}-come[II]-\textsc{pl} \textsc{emph} \textsc{ipfv}-show.off \textsc{ifr}.\textsc{ipfv}-be \textsc{lnk} \\
\glt `Each time people came, he would show off.' (2011-10-qajdo, 2)
\end{exe}

\begin{exe}
   \ex \label{ex:tWmW.kWkAlAt}
   \gll  tɯ-mɯ  kɯ\redp{}ka-lɤt ʑo zdɯmlaʁrɯʁrɯ ju-nɯ-ɬoʁ ŋu   \\
sky \textsc{iter}\redp{}\textsc{aor}-release \textsc{emph} snail \textsc{ipfv}-\textsc{auto}-come.out be:\textsc{fact} \\
\glt `Each time it rains,  snails come out.' (elicited)
\end{exe}

Alternatively, unreduplicated Aorist in its function to mark a temporal reference point (§\ref{sec:aor.temporal}, §\ref{sec:temporal.reference}) can be interpretable as  iterative coincidence when the verb in the main clause is in the Imperfective, as in (\ref{ex:tWtWrca.luzonW2}). 

\begin{exe}
\ex \label{ex:tWtWrca.luzonW2}
\gll tɕe [lɤ-zo-nɯ] qʰe tɯtɯrca lu-zo-nɯ, [tʰɯ-nɯqambɯmbjom-nɯ] qʰe tɯtɯrca cʰɯ-nɯqambɯmbjom-nɯ, \\
\textsc{lnk} \textsc{aor}:\textsc{upstream}-land-\textsc{pl}  \textsc{lnk} together  \textsc{ipfv}:\textsc{upstream}-land-\textsc{pl} \textsc{aor}:\textsc{downstream}-fly-\textsc{pl}  \textsc{lnk} together  \textsc{ipfv}:\textsc{downstream}-fly-\textsc{pl} \\
\glt `When they land they all land together, when they fly they all fly together.' (23-scuz, 79-80)
\end{exe}
 
 
\subsection{Precedence}
Three types of constructions are used to express temporal precedence between the event described in the main clause precedes that of the temporal subordinate clause: neutral precedence, immediate precedence and terminative. 

These temporal clauses normally occur before the main clause, hence resulting in a non-iconic temporal relationship between the two clauses, since the first (subordinate) clause describes an event occurring \textit{after} that in the second one (the main clause).

 \subsubsection{Neutral precedence} \label{sec:precedence.CWNgW}
There is only one available construction in Japhug to express that the action of the main occurs before that of the temporal clause without further aspectual specifications: clauses with the postposition \japhug{ɕɯŋgɯ}{before} (§\ref{sec:temporal.postpositions}). The verb in \forme{ɕɯŋgɯ} clauses is required to be in the Imperfective (§\ref{sec:ipfv.temporal}). In (\ref{ex:YWsi.CWNgW}) for instance, replacing the Imperfective form \forme{ɲɯ-si} by the corresponding Aorist (\forme{nɯ-si}), an infinitive (\forme{kɤ-si} or any other finite or non-finite form would result in a utterly agrammatical sentence.

\begin{exe}
\ex \label{ex:YWsi.CWNgW}
\gll [ɲɯ-si] ɕɯŋgɯ pɯ-nɯ-ɴɢɤt-ndʑi  \\
\textsc{ipfv}-die before \textsc{aor}-\textsc{auto}-\textsc{acaus}:separate-\textsc{du} \\
\glt `They had divorced before she died.' (14-siblings, 331)
\end{exe}

A syntactic error involving a verb in a temporal clause in \japhug{ɕɯŋgɯ}{before}  is presented in §\ref{sec:syntactic.errors}.

 \subsubsection{Immediate precedence}
Immediate precedence can be expressed by three constructions.

First, a clause in the Factual Non-Past (§\ref{sec:factual}) with linker \forme{tɤkʰa} `about to' can describe a action that took place just before that of the main clause, as in (§\ref{ex:GindZi.tAkha}) and (§\ref{ex:amboR.tAkha}).
 
\begin{exe}
\ex \label{ex:GindZi.tAkha}
\gll ɬamu kɯ [ɣi-ndʑi] tɤkʰa tɕe pɯwɯ ɯ-ɕki ɯʑo kɯ ta-tɯt nɯra ci to-sɯʁjit \\
\textsc{anthr} \textsc{erg} come:\textsc{fact}-\textsc{du} about.to \textsc{lnk} donkey \textsc{3sg}-\textsc{dat} \textsc{3sg} \textsc{erg} \textsc{aor}:3\fl{}3-say[II] \textsc{dem} \textsc{ifr}-remember \textsc{lnk} \\
\glt  `Lhamo remembered [what] she had said to her donkey as they were about to depart (to come here).' (2002 qajdoskAt, 64-5)
\end{exe}

\begin{exe}
\ex \label{ex:amboR.tAkha}
\gll [amboʁ] tɤkʰa tɕe tɕe ɲɯ-mu-a tɕe, tɕe a-jaʁ ɲɯ-mɯnmu ɲɯ-ɕti qʰe  \\
burst:\textsc{fact}  about.to \textsc{lnk}  \textsc{lnk} \textsc{sens}-be.afraid-\textsc{1sg}  \textsc{lnk}  \textsc{lnk} \textsc{1sg}.\textsc{poss}-hand \textsc{ipfv}-move \textsc{sens}-be:\textsc{aff} \textsc{lnk} \\
\glt  `(When I was aiming), as [the gun] was about to burst (i.e. to go off), I was afraid and my hand moved.' (28-CAmWGdW, 134-135)
\end{exe}


Alternatively, the \forme{jɯ-} Proximative prefix (§\ref{sec:proximative}) and the Periphrastic Proximative (§\ref{sec:proximative.periphrastic}) can be used to express this meaning, as illustrated by (\ref{ex:CAr.jWjazGWt}) and (\ref{ex:qanW.pjANu}).


\begin{exe}
\ex \label{ex:CAr.jWjazGWt}
\gll   [ɕɤr jɯ-jɤ-azɣɯt] tɕe (...) ɯ-kɤrme nɯ pjɤ-ɕɯ-ɴqoʁ \\
night \textsc{prox}-\textsc{aor}-arrive \textsc{lnk} { } \textsc{3sg}.\textsc{poss}-hair \textsc{dem} \textsc{ifr}-\textsc{caus}-hang \\
\glt `Just before the night fell, [the witch] hung her hair [on the window of the tower].' (140506 woju guniang-zh, 145)
\end{exe}


\begin{exe}
\ex \label{ex:qanW.pjANu}
\gll  tɯrmɯ ko-ɣi tɕe ʑatsa qanɯ pjɤ-ŋu \\
evening \textsc{ifr}-come \textsc{lnk} soon be.dark:\textsc{fact} \textsc{ipfv}.\textsc{ifr}-be \\
\glt `The evening came and it was about to be dark (i.e., it was getting dark).' (140510 fengwang-zh, 55)
\end{exe}

 \subsubsection{Terminative} \label{sec:terminative.clause}
The terminative postposition  \japhug{mɤɕtʂa}{until}  (§\ref{sec:terminative}) occurs with finite clauses to express the end point of the event described in the main clause. In terminative clauses, verbs nearly always take a negative prefix as in (\ref{ex:mWthaCskWt.mACtsxa}) and (\ref{ex:mWtAtWta.mACtsxa}), and the polarity contrast is neutralized.

\begin{exe}
\ex \label{ex:mWthaCskWt.mACtsxa}
\gll nɯ-kɤ-kʰo nɯ [mɯ-tʰa-ɕkɯt] mɤɕtʂa tu-ndze ɲɯ-ɕti. \\
\textsc{aor}-\textsc{obj}:\textsc{pcp}-give \textsc{dem} \textsc{neg}-\textsc{aor}:3\flobv{}-eat.completely until \textsc{ipfv}-eat[III] \textsc{sens}-be.\textsc{aff} \\
\glt `[The monkey] (cannot control its urge to eat, and) eats [the things that people] 
have given him it none are left (until it has completely finished eating them).' (19-GzW, 
\end{exe}

\begin{exe}
\ex \label{ex:mWtAtWta.mACtsxa}
\gll [``jɤ-ɕe jɤɣ'' mɯ-tɤ-tɯt-a] mɤɕtʂa ma-nɯ-tɯ-mɯnmu ra \\
\textsc{imp}-go be.allowed:\textsc{fact} \textsc{neg}-\textsc{aor}-say[II]-\textsc{1sg} until \textsc{neg}-\textsc{imp}-2-move be.needed:\textsc{fact} \\
\glt `Don't move until I tell [you] that you can go.' (qala 2002, 58)
\end{exe}

Non-negative terminative clauses are rare but attested, as in (\ref{ex:tundzanW.mACtsxa}), especially when the main clause is in negative form itself.

\begin{exe}
\ex \label{ex:tundzanW.mACtsxa}
\gll [ʑimkʰɤm ʑo tu-ndza-nɯ] mɤɕtʂa kɤ-mqlaʁ mɯ́j-βze. \\
long.time \textsc{emph} \textsc{ipfv}-chew-\textsc{pl} until \textsc{inf}-swallow \textsc{neg}:\textsc{sens}-make[III] \\
\glt `[The animals] do not swallow until they have chewed it for along time.' (19-qachGa mWntoR, 204)
\end{exe}

 \subsection{Subsequence} 
Temporal succession can be expressed by simple coordination, the order of the clauses mirroring the temporal sequence of the events they describe (§\ref{sec:coordination}). The present section however focuses on subordinating construction devoted to encoding temporal subsequence between the subordinate clause and the main clause. 

 \subsubsection{Neutral subsequence} \label{sec:subsequence.neutral}
 The most common way to specify temporal succession between two clauses is to use the locative/temporal relator noun  \japhug{ɯ-qʰu}{after, behind}  (§\ref{sec:other.locative.relator}). The temporal clauses are in finite form, in particular in the Aorist as in (\ref{ex:pWsthWta.Wqhu}) and (\ref{ex:smWntsxWG}).
 
 
 \begin{exe}
\ex \label{ex:pWsthWta.Wqhu}
\gll [<weixiao> kɯβde-xpa pɯ-stʰɯt-a] ɯ-qʰu tɕe tɕendɤre, <fenpeigongzuo> tɕe ʁdɯrɟɤt lɤ́-wɣ-lat-a-nɯ \\
 nursing.school four-year \textsc{aor}-finish-\textsc{1sg} \textsc{3sg}.\textsc{poss}-after \textsc{lnk} \textsc{lnk} assign.work \textsc{lnk}  \textsc{topo} \textsc{aor}:\textsc{upstream}-\textsc{inv}-release-\textsc{1sg}-\textsc{pl} \\
 \glt `After I finished the four years of nursing school, they assigned me a work position and sent me to Gdongbrgyad.' (140501 tshering skyid, 118)
 \end{exe}

The relator noun \forme{ɯ-qʰu} can be optionally followed by the linker \forme{tɕe}, by locative postpositions such as \forme{ri}  (§\ref{sec:core.locative}) and can even be modified by the adverb \japhug{tsa}{a little} as in (\ref{ex:smWntsxWG}).

 \begin{exe}
\ex \label{ex:smWntsxWG}
\gll [smɯntʂɯɣ nɯnɯ tɤ-ɬoʁ] ɯ-qʰu tsa ri tɕe tɕe, qandʐe tu-ɬoʁ ŋu. \\
Pleiades \textsc{dem} \textsc{aor}-come.out \textsc{3sg}.\textsc{poss}-after a.little \textsc{loc} \textsc{lnk} \textsc{lnk} earthworm \textsc{ipfv}-come.out be:\textsc{fact} \\
 \glt  `The [constellation of the] earthworm appears a little after the Pleiades have come out.'  (29-mWBZi, 26)
\end{exe}
 
Alternatively, and more rarely, the relator \japhug{ɯ-mpʰru}{after, following} (§\ref{sec:relator.temporal}) can be used to indicate temporal subsequence, as in (\ref{ex:kalAt.Wmphru}). 

 \begin{exe}
\ex \label{ex:kalAt.Wmphru}
\gll [tɯ-mɯ ka-lɤt] ɯ-mpʰru nɯ tu.    \\
 \textsc{indef}.\textsc{poss}-sky \textsc{aor}:3\flobv{}-release 3\textsc{sg}.\textsc{poss}-after \textsc{dem} exist:\textsc{fact} \\
\glt  `It is found after it has rained.' (23-mbrAZim, 57)
\end{exe}

  
 \subsubsection{Immediate subsequence} \label{sec:immediate.subsequence}
Temporal clauses with the immediate converb (§\ref{sec:immediate.converb}) describe an event immediately followed by the action referred to in the main clause (\ref{ex:tutWlhoR.chWphWtnW}).
 
\begin{exe}
\ex \label{ex:tutWlhoR.chWphWtnW} 
\gll   tɕe [nɯ tu-tɯ-ɬoʁ] ʑo qʰe cʰɯ-pʰɯt-nɯ cʰɯ-βde-nɯ ɕti. \\
\textsc{lnk} \textsc{dem} \textsc{ipfv}-\textsc{imm}:\textsc{conv}-come.out \textsc{emph} \textsc{lnk} \textsc{ipfv}-take.out-\textsc{pl} \textsc{ipfv}-throw-\textsc{pl} be.\textsc{aff}:\textsc{fact} \\
\glt `As soon as it comes out (in fields), [the farmers] pluck it off and throw it away.' (16-CWrNgo, 35
 \end{exe}  
  
The main clause can also contain a stative verb, depicting a temporary state beginning just after the event of the converbial clause (\ref{ex:tutWlhoR.rqhArqhAt}).
 
\begin{exe}
\ex \label{ex:tutWlhoR.rqhArqhAt} 
\gll tɕe [tu-tɯ-ɬoʁ] tɕe rqʰɤrqʰɤt ʑo ɲɯ-pa  \\
\textsc{lnk} \textsc{ipfv}-\textsc{imm}:\textsc{conv}-come.out \textsc{lnk} \textsc{idph}(II):fresh.and.firm \textsc{emph} \textsc{sens}-\textsc{aux} \\
\glt `When [this mushroom] has just come out, it is nice and firm.' (24-zwArqhAjmAG, 15)
  \end{exe}  
  
Another way to express immediate subsequence is a temporal clause with the temporal postposition \japhug{ɕimɯma}{immediately after} (§\ref{sec:temporal.postpositions}) taking a clause with a verb in the Aorist. This construction is semantically very close to the immediate converb, but more commonly found when the main verb is a stative verb like \japhug{arʁɯrʁu}{be wrinkled} in (\ref{ex:tAlhoR.CimWma}).
 
\begin{exe}
\ex \label{ex:tAlhoR.CimWma} 
\gll  nɯ tɤ-ɬoʁ ɕimɯma tɕe ɲɯ-ɤrʁɯrʁu \\
\textsc{dem} \textsc{aor}-come.out immediately.after \textsc{lnk} \textsc{sens}-be.wrinkled \\
\glt  `When [the stalk of the fern] has just come out, it is all wrinkled.' (13-NanWkWmtsWG, 13)
 \end{exe}  
 
 The postposition \japhug{kóʁmɯz}{only after}  (§\ref{sec:temporal.postpositions}), which imposes no requirement on the TAME category of the verb in the subordinate clause, indicates that the event in the subordinate clause is a prerequisite for that in the main clause to occur, as shown by (\ref{ex:kuxtCArnW.koRmWz}).
 
 \begin{exe}
\ex \label{ex:kuxtCArnW.koRmWz} 
\gll   [ɯ-mi ra ku-xtɕɤr-nɯ] kóʁmɯz tɤ-lu pjɯ-tɕɤt-nɯ ɲɯ-ra \\
 \textsc{lnk} \textsc{ipfv}-tie.up-\textsc{pl} \textsc{3sg}.\textsc{poss}-foot \textsc{pl} \textsc{ipfv}-attach-\textsc{pl} only.after \textsc{indef}.\textsc{poss}-milk \textsc{ipfv}-take.out-\textsc{pl}  \textsc{sens}-be.needed \\
\glt  `People need to tie the legs [of the female hybrid yak] before milking.' (i.e., they milk it only after they have tied its legs).   (05-qambrW, 19)
  \end{exe}  
 
 
\subsubsection{Ingressive} \label{sec:since.clause} 
 The ingressive postpositions  \japhug{ɕaŋpɕi}{since}, `from ... on' and \japhug{pɕintɕɤt}{since} (§\ref{sec:temporal.postpositions}) can take clauses with a verb in the Aorist, indicating the beginning of a period lasting up until to the current point of temporal reference (utterance time, or a reference point in the past in the case of narratives). The main clause is generally in negative form, as in (\ref{ex:jari.CaNpCi}) and (\ref{ex:lazGWtnW.CaNpCi}).

\begin{exe}
\ex \label{ex:jari.CaNpCi} 
\gll  [nɤ-wa nɯ kɯ-fse jɤ-ari] ɕaŋpɕi nɯ tɕe mɯ-jɤ-a<nɯ>zɣɯt ɕti tɕe \\
\textsc{2sg}.\textsc{poss}-father \textsc{dem} \textsc{sbj}:\textsc{pcp}-be.like \textsc{aor}-go[II] since \textsc{dem} \textsc{lnk} \textsc{neg}-\textsc{aor}-<\textsc{vert}>arrive be.\textsc{aff}:\textsc{fact} \textsc{lnk} \\
\glt `Ever since your father went (to the mission) like that, he has not come back.' (Norbzang 2005, 206)
\end{exe}  

\begin{exe}
\ex \label{ex:lazGWtnW.CaNpCi} 
\gll   [ci nɯra lɤ-azɣɯt-nɯ] ɕaŋpɕi nɯ tɕe, tɕʰeme nɯ tɯ-ɣjɤn ci kɯnɤ nɯ-nɤre kɤ-mtsʰɤm pɯ-me \\
\textsc{indef} \textsc{dem}:\textsc{pl} \textsc{aor}:\textsc{upstream}-arrive-\textsc{pl} since \textsc{dem} \textsc{lnk} girl \textsc{dem} one-time \textsc{indef} also \textsc{aor}-laugh \textsc{obj}:\textsc{pcp}-hear \textsc{pst}.\textsc{ipfv}-not.exist \\
\glt `Ever since the others had come back (to the king's palace), the girl (that they had brought with them) had not been heard laughing even once.' (qachGa 2003, 321)
\end{exe}  

 
\subsection{Concurrence}

\subsubsection{Temporal reference point} \label{sec:temporal.reference}
Aside from indicating the temporal relation and ordering between the events in the subordinate and the main clauses, temporal clauses can also serve to specify a temporal reference point for the main clause. There are four constructions of this type.

First, one of the functions of the Aorist is precisely to fix a temporal reference, for both past and future events (§\ref{sec:aor.temporal}). 

Second, prenominal clauses with temporal relator nouns in \textsc{3sg} possessive forms such as  \forme{ɯ-sŋi} `the day when...', \forme{ɯ-xpa} `the year when...' , \forme{ɯ-raŋ} `the time when...' or other ones as in (\ref{ex:pWnWmdarndZi.WGmWr}) can mark the time period when the action in the main clause takes place. These  clauses can be formally as a subtype of finite prenominal relative clauses (§\ref{sec:time.relativization}).
 
 \begin{exe}
\ex \label{ex:pWnWmdarndZi.WGmWr}
\gll [pɯ-nɯmdar-ndʑi] ɯ-ɣmɯr nɯtɕu tɕe  mtsʰu nɯ cʰɯmcʰɯm ʑo, tɕe, tɯ-skɤm pjɤ-sɤʑa \\
\textsc{aor}-jump-\textsc{du} \textsc{3sg}.\textsc{poss}-evening \textsc{dem}:\textsc{loc} \textsc{loc} lake \textsc{dem} \textsc{idph}(II):slowly \textsc{emph} \textsc{lnk} \textsc{inf}:II-dry.up \textsc{ifr}:\textsc{down}-start \\
\glt `In the evening when they jumped [into the lake], the level of the lake started to drop slowly and the lake disappeared' (2003, Nyimawozer 2, 105)
\end{exe}

Third, the locative postpositions \forme{ri}, \forme{tɕu} and \forme{zɯ} (§\ref{sec:core.locative}) can also occur with finite clauses, with a temporal meaning even without temporal relator noun as in (\ref{ex:pWkWxtCi.ri}).

 \begin{exe}
\ex \label{ex:pWkWxtCi.ri}
\gll  ɯnɯnɯtɕu ʁo [tɯʑo pɯ-kɯ-xtɕi] ri rcanɯ, mtɕʰi kɯ-wxtɯ\redp{}wxti ʑo pɯ-tu. \\
\textsc{dem}:\textsc{loc} \textsc{advers}:\textsc{top} \textsc{genr} \textsc{pst}.\textsc{ipfv}-\textsc{genr}:S/O-be.small \textsc{loc} \textsc{unexp}:\textsc{foc} sea.buckthorn \textsc{sbj}:\textsc{pcp}-\textsc{emph}\redp{}be.big \textsc{emph} \textsc{pst}.\textsc{ipfv}-exist \\
\glt `There, when we were young, there was a huge sea buckthorn.' (140522 Kamnyu zgo, 344)
 \end{exe}
 
Fourth, the temporal postposition   \japhug{jɤz}{when} and its variant \forme{jɤznɤ} (§\ref{sec:temporal.postpositions}) can take finite clauses as in (\ref{ex:pWxtCij.jAz}).
    
 \begin{exe}
\ex \label{ex:pWxtCij.jAz}
\gll ma [iʑora pɯ-xtɕi-j] jɤz  tu-ndza-j mɯ-pɯ-ŋgrɤl ma \\
\textsc{lnk} \textsc{1pl} \textsc{pst}.\textsc{ipfv}-be.small-\textsc{1pl} when \textsc{ipfv}-eat-\textsc{1pl} \textsc{neg}-\textsc{pst}.\textsc{ipfv}-be.usually.the.case \textsc{lnk} \\
\glt `When we were little, we did not eat it.' (13-NanWkWmtsWG, 177)
 \end{exe}

 
\subsubsection{Simultaneity} \label{sec:simultaneity}
 There are two types of subordinate clauses expressing an ongoing action or event taking place at the same time as that of the main verb without serving as a point of temporal reference, unlike the constructions discussed in §\ref{sec:temporal.reference} above.

 First, finite clauses with the relator nouns \japhug{ɯ-kʰɯkʰa}{while}  and \japhug{ɯ-jɯja}{along with}, `while' describe events that occur together with, and serve as background to the action of the main clause, as in (\ref{ex:tunWsmAn.WkhWkha}) and (\ref{ex:jurJWG.WkhWkha}).  There are no coreference restrictions on the arguments of the subordinate and the main clauses.

\begin{exe}
\ex \label{ex:tunWsmAn.WkhWkha}
\gll tɕendɤre [tu-nɯsmɤn] ɯ-kʰɯkʰa tu-rɤma-nɯ  \\
\textsc{lnk} \textsc{ipfv}-treat \textsc{3sg}-the.same.time \textsc{ipfv}-work-\textsc{pl} \\
\glt [The lepers]$_i$ worked (there) while [the doctor] was treating them$_i$.' (25-khArWm, 68)
\end{exe}

\begin{exe}
\ex \label{ex:jurJWG.WkhWkha}
\gll   [nɯnɯ ju-rɟɯɣ] ɯ-kʰɯkʰa ɯ-se ku-tsʰi ɲɯ-ɕti.  \\
\textsc{dem} \textsc{ipfv}-run \textsc{3sg}-the.same.time \textsc{3sg}.\textsc{poss}-blood \textsc{ipfv}-drink  \textsc{sens}-be:\textsc{aff} \\
\glt `[The lion] drinks [its prey's]$_i$ blood while it$_i$ is (still) running.' (20-sWNgi, 53)
\end{exe}

The relator \japhug{ɯ-jɯja}{along with} differs from \forme{ɯ-kʰɯkʰa} in that it implies a simultaneous gradual change of degree in both the event or state of the subordinate clause and that of the main clause. The verb of the subordinate clause is generally in the Perfective (though a few Imperfective forms are also attested), while that of the main clause can be in any TAM form, in particular with incremental initial reduplication (§\ref{sec:redp.gradual.increase}) indicating gradual increase as in (\ref{ex:tAwxti.WjWja}).

\begin{exe}
\ex \label{ex:tAwxti.WjWja}
\gll [ɯʑo tɤ-wxti] ɯ-jɯja tɕe ɯ-jwaʁ nɯnɯ ɲɯ\redp{}ɲɯ-ndɯβ ʑo ɲɯ-ŋu.  	\\
\textsc{3sg} \textsc{aor}-be.big  \textsc{3sg}.\textsc{poss}-along \textsc{lnk} \textsc{3sg}.\textsc{poss}-leaf \textsc{dem} \textsc{incr\redp{}ipfv}-be.tiny \textsc{emph} \textsc{sens}-be \\
\glt `As it grows big, its leaves become more and more tiny.' (09-mi, 18)
\end{exe}

The clause in \forme{ɯ-jɯja}, rather than a gradual change of state across time, can indicate change across space, as in (\ref{ex:tAmbro.WjWja}), where \forme{tɤ-mbro} does not mean that the mountain becomes higher, but rather that the person observing the plants moves higher in the mountain.

\begin{exe}
\ex \label{ex:tAmbro.WjWja}
\gll  zgoku tɤ-mbro ɯ-jɯja nɯ ʑmbri tu-tɯ-ldʑɯz ʑo ŋu \\
mountain \textsc{aor}-be.high \textsc{3sg}.\textsc{poss}-along \textsc{dem} willow \textsc{incr}\redp{}-\textsc{ipfv}-be.flexible \textsc{emph} be:\textsc{fact} \\
\glt `The higher [ones goes] in the mountain, the more the [wood] of the willows is flexible.' (07-Zmbri, 43)
\end{exe}

Second, gerundive clauses (§\ref{sec:gerund.clauses}) are an alternative possibility to indicate a background event or state occurring concurrently with the event of the main clause, as in (\ref{ex:sAznWGmWGmu.kuXse.YWra}), though in some cases gerunds express manner rather than temporal overlap (§\ref{sec:manner.clauses}).
 
\begin{exe}
\ex \label{ex:sAznWGmWGmu.kuXse.YWra}
\gll nɯnɯ ɲɯ-nɯɣ-me rɯtɕi, nɯ kɯnɤ ku-χse ɲɯ-ra.  tɕe [sɤz-nɯɣ-mɯ\redp{}ɣmu] ʑo ku-χse ɲɯ-ra. \\
\textsc{dem} \textsc{ipfv}-\textsc{appl}-fear[III] \textsc{lnk} \textsc{dem} also \textsc{ipfv}-feed[III] \textsc{sens}-be.needed \textsc{lnk} \textsc{ger}-\textsc{appl}-fear \textsc{emph} \textsc{ipfv}-feed[III]  \textsc{sens}-be.needed \\
\glt `Even though it$_i$ is afraid of it$_j$, it$_i$ has to feed it$_i$ , and it$_i$ has to feed it$_i$  while being afraid of it.$_i$ '  (24-ZmbrWpGa, 110)
\end{exe}

 

\subsubsection{Opportunity} \label{sec:RaznA}
The postposition \japhug{ʁaz}{while ... still} (and its variant \forme{ʁaznɤ}; its etymology is discussed in §\ref{sec:denominal.postposition.s}) has a meaning close to that of Chinese \ch{趁着}{chènzhe}{while ... still}, `taking the opportunity of...'. It requires a finite clause as in (\ref{ex:GWrNi.RaznA}) and (\ref{ex:YWnWZWB.RaznA}).

\begin{exe}
\ex \label{ex:GWrNi.RaznA}
\gll iɕqʰa tɯ-ndʐi nɯ nɤki, [ɣɯrŋi] ʁaznɤ nɯnɯ nɯnɯtɕu pjɯ-tʂɯβ-nɯ tɕe tɕe \\
the.aforementioned \textsc{indef}.\textsc{poss}-skin \textsc{dem} \textsc{filler} be.wet:\textsc{fact} while \textsc{dem} \textsc{dem}:\textsc{loc} \textsc{ipfv}-sew-\textsc{pl} \textsc{lnk} \textsc{lnk} \\
\glt `They would sew the skin while it was still wet.' (06-BGa, 53)
\end{exe}

The transitive verb \japhug{naχtʰɤβ}{take the opportunity of} can select as object a finite clause or a participial clause with a similar meaning: in  (\ref{ex:YWnWZWB.RaznA}), both constructions redundantly occur. 

\begin{exe}
\ex \label{ex:YWnWZWB.RaznA}
\gll  nɯʑora tɕe ki ɲɯ-nɯʑɯβ ʁaznɤ, [kɯ-nɯʑɯβ] nɯ kɤ-naχtʰɤβ-nɯ tɕe, jɤ-pʰɣo-nɯ! \\
\textsc{2pl} \textsc{lnk} \textsc{dem}.\textsc{prox} \textsc{sens}-sleep while \textsc{sbj}:\textsc{pcp}-sleep \textsc{dem} \textsc{imp}-take.the.opportunity-\textsc{pl} \textsc{lnk} \textsc{imp}-flee-\textsc{pl} \\
\glt `Flee while this one [the ogre] is sleeping.' (160706 poucet6, 83-84)
\end{exe}
 


\section{Manner clauses} \label{sec:manner.clauses}
% alternative to causative complement-taking verbs (§\ref{sec:causative.manner.complement}).

\subsection{Serial verb construction} \label{sec:svc.manner}
The main function of serial verb constructions (§\ref{sec:svc}) in Japhug is to express manner. The most grammaticalized serial constructions involve deideophonic and similative verbs.


Serial verbs constructions can themselves occur as complements of a single complement-taking verb, resulting in a multiclausal complement (§\ref{sec:multiclausal.complements}).

\subsubsection{Deideophonic verbs} \label{sec:svc.deideophonic}
Deideophonic verbs (§\ref{sec:voice.deideophonic}) commonly occur in serial verb constructions. The transitive deideophonic verb \japhug{nɯdrɯβ}{gore again and again} for instance, is only attested in serial construction with \japhug{tɕʰɯ}{gore} as in (\ref{ex:totChW}). The ideophonic verb can either follow (\ref{ex:totChW}) or precede the main verb (\ref{ex:totChW2}), the latter construction being by far more common. 

\begin{exe}
\ex \label{ex:totChW}
\gll iɕqʰa srɯnmɯ nɯ to-tɕʰɯ to-nɯdrɯβ tɕe pjɤ-sat \\
the.aforementioned râkshasî \textsc{dem} \textsc{ifr}-gore \textsc{ifr}-repeatedly.gore  \textsc{lnk} \textsc{ifr}-kill \\
\glt `[The rhinoceros] gored the râkshasî repeatedly and killed her.' (28-smAnmi, 403)
\end{exe}

\begin{exe}
\ex \label{ex:totChW2}
\gll srɯnmɯ nɯ to-nɯdrɯβ ʑo to-tɕʰɯ \\
 râkshasî \textsc{dem}  \textsc{ifr}-repeatedly.gore  \textsc{emph}  \textsc{ifr}-gore \\
 \glt `[The rhinoceros] gored the râkshasî repeatedly and killed her.' (elicited on the basis of \ref{ex:totChW})
\end{exe}	

\subsubsection{Similative verbs} \label{sec:svc.similative.verb}
The similative verbs \japhug{stu}{do like} and \japhug{fse}{be like}\footnote{I adopt the term `similative verb' \citep{creissels17similative} rather than `manner deixis verbs' that I used in previous publications, since these verbs do not express deixis on their own, and require a demonstrative. } commonly occur in a serial verb construction to indicate the way in which the action takes place. They are always the first verb of the series. 

The ditransitive verb \forme{stu} (§\ref{sec:ditransitive.secundative}) is found with transitive verbs, and shares its subject and object with them, as shown by (\ref{ex:ki.tuWGstu.pjWGqlWt}) (generic transitive subject) and (\ref{ex:kuWGstuanW}) (\textsc{3pl}\fl{}\textsc{1sg}).

\begin{exe}
\ex \label{ex:ki.tuWGstu.pjWGqlWt}
\gll [ɯ-ru nɯ ki tú-wɣ-stu] [pjɯ́-wɣ-qlɯt] \\
\textsc{3sg}.\textsc{poss}-stalk \textsc{dem} \textsc{dem}:\textsc{prox} \textsc{ipfv}-\textsc{inv}-do.like \textsc{ipfv}-\textsc{inv}-break \\
\glt `One breaks its stalk like this.' (14-tasa, 81)
\end{exe}	

In addition, in some cases it takes the orientation of the other verb (\textsc{eastwards} in \ref{ex:kuWGstuanW}), rather than its intrinsic orientation (\textsc{upwards} as in \ref{ex:ki.tuWGstu.pjWGqlWt}).

\begin{exe}
\ex \label{ex:kuWGstuanW}
\gll [aʑo kɯki ntsɯ kú-wɣ-stu-a-nɯ] tɕe, [kú-wɣ-znɯkʰrɯm-a-nɯ] \\
 \textsc{1sg} \textsc{dem}:\textsc{prox} always \textsc{ipfv}-\textsc{inv}-do.like-\textsc{1sg}-\textsc{pl} \textsc{lnk} \textsc{ipfv}-\textsc{inv}-punish-\textsc{1sg}-\textsc{pl} \\
 \glt `They tortured me like this.' (Gesar, 278)
\end{exe}	

The demonstratives \forme{ki} and \forme{kɯki} in (\ref{ex:ki.tuWGstu.pjWGqlWt}) and (\ref{ex:kuWGstuanW}) are semi-objects that are not shared with the other verb.

With intransitive verbs, \japhug{fse}{be like} is used instead of \forme{stu}. It shares its intransitive subject with the other verb, as shown by (\ref{ex:ki.fsea}) where both \forme{fse-a} and \forme{ndzur-a} have \textsc{1sg} indexation.

\begin{exe}
\ex \label{ex:ki.fsea}
\gll aʑo nɯ sŋiɕɤr ʑo kutɕu [ki fse-a] [ndzur-a] ntsɯ ɲɯ-ra tɕe \\
\textsc{1sg} \textsc{dem} night.and.day \textsc{emph} here \textsc{dem}:\textsc{prox} be.like:\textsc{fact}-\textsc{1sg} stand:\textsc{fact}-\textsc{1sg} always \textsc{sens}-be.needed like \\
\glt `I have to stand like this night and day.' (The divination, 2002, 44)
\end{exe}

There are cases when \forme{fse} rather than \forme{stu} occurs with transitive verbs, and which could superficially appear to be cases of serial verb constructions with transitivity mismatch.  In (\ref{ex:chWwGnWchAmdaj}) the transitive verb \japhug{nɯcʰɤmda}{drink with a straw} is preceded by \forme{fse}, with \textsc{1pl} coreference between the object of the former and the intransitive subject of the latter. 

\begin{exe}
\ex \label{ex:chWwGnWchAmdaj}
\gll tʰɯ-rgɤz-i, sna nɯ-me-j tɕe [ki tɤ-fse-j] tɕe [cʰɯ́-wɣ-nɯcʰɤmda-j] ɕti  \\
\textsc{aor}-be.old-\textsc{1pl} be.good \textsc{aor}-not.exist-\textsc{1pl} \textsc{lnk} \textsc{dem}.\textsc{prox} \textsc{aor}-be.like-\textsc{1sg} \textsc{lnk} \textsc{ipfv}:\textsc{downstream}-\textsc{inv}-drink.with.a.straw-\textsc{1sg} be.\textsc{aff}:\textsc{fact} \\
\glt `Now that we have become old and useless, we became like this, they drink our [blood] with straws (planted in our back).' (Norbzang 2005, 78)
\end{exe}

However, note that although \forme{tɤ-fse-j} and \forme{cʰɯ́-wɣ-nɯcʰɤmda-j} share one argument, their TAME category is different, and the two clauses constitute a simple case of coordination rather than a serial verb construction: \forme{ki tɤ-fse-j} means `we became like this' rather than `(they treated) us like this'. The corresponding genuine serial verb construction is shown by (\ref{ex:ki.tWwGstu.tWGnWchAmda}), where \forme{nɯcʰɤmda} occurs with \forme{stu} as expected, and the two verbs have the same person-number configuration and TAME category (like \ref{ex:kuWGstuanW} and \ref{ex:ki.tuWGstu.pjWGqlWt} above).

\begin{exe}
\ex   \label{ex:ki.tWwGstu.tWGnWchAmda}
\gll tʰɯ-tɯ-rgɤz tɕe [ki tɯ́-wɣ-stu] [tɯ́-wɣ-nɯcʰɤmda] ɕti tɕe, \\
\textsc{aor}-2-be.old \textsc{lnk} \textsc{dem}.\textsc{prox} 2-\textsc{inv}-do.like:\textsc{fact} 2-\textsc{inv}-drink.with.a.straw:\textsc{fact} be.\textsc{aff}:\textsc{fact} \textsc{lnk} \\
\glt `When you have become old, they will drink you [your blood] like this with a straw (planted on your back)' (Norbzang 2012, 67)
\end{exe}

Some non-deideophonic verbs expressing manner such as \japhug{nɤxɕɤt}{do with force} (like its Tshobdun cognate \japhug{nɐʃeʃet}{exert oneself}, \citealt[490--491]{sun12complementation}) can also be used in a serial verb construction.

\subsubsection{Other verbs of manner} \label{sec:svc.manner.other}
Other verbs of manner can also occur in a serial construction, for instance the distributed action verb \japhug{amɯzɣɯt}{be evenly distributed} (examples \ref{ex:YAmWzGWt.YWlhoR} and\ref{ex:YWsAmWzGWt.YWmar}, §\ref{sec:distributed.amW}) or atelic motion verbs like \japhug{ŋke}{walk} (\ref{ex:joNke.joCe}, §\ref{sec:motion.verbs}) or \japhug{nɯqambɯmbjom}{fly}  (example \ref{ex:junWqambWmbjomndZi.junACWCendZi}, §\ref{sec:distributed.action}).

\subsubsection{Simultaneous action} \label{sec:svc.simultaneous}
Serial verb constructions can be used to describe a secondary action simultaneous with the main action, optionally marked with the emphatic \forme{ʑo} (§\ref{sec:emphatic.Zo}) as \forme{pjɤ-sŋur} `it was snoring' in (\ref{ex:pjAsNur.Zo.pjAnWZWB}).

\begin{exe}
\ex \label{ex:pjAsNur.Zo.pjAnWZWB}
\gll pjɤ-sŋur ʑo pjɤ-nɯʑɯβ ɕti ma, maka ʑo mɯ-pjɤ-sɯχsɤl. \\
\textsc{ifr}.\textsc{ipfv}-snore \textsc{emph} \textsc{ipfv}.\textsc{ifr}-sleep be.\textsc{aff}:\textsc{fact} \textsc{lnk} at.all \textsc{emph}  \textsc{neg}-\textsc{ifr}-realize \\
\glt `[The aquatic monster] was snoring in its sleep, and did not notice anything.' (140508 benling gaoqiang de si xiongdi-zh, 190)
\end{exe}



\subsubsection{Degree} \label{sec:svc.degree}
Serial constructions can also describe the degree, intensity or extent of an action (§\ref{sec:degree.svc}), even for non-gradable predicates. In these constructions, the verb describing the main action or state is in the first position, and a stative verb of degree (such as \japhug{arɕo}{be finished}, \japhug{tɕʰom}{be too much} or \japhug{rtaʁ}{be enough}) occurs in second position, as in (\ref{ex:nWsinW.narConW}).

\begin{exe}
\ex \label{ex:nWsinW.narConW}
\gll   wuma ʑo kɯ-tso nɯra nɯ-si-nɯ nɯ-arɕo-nɯ ɕti \\
really \textsc{emph} \textsc{sbj}:\textsc{pcp}-understand \textsc{dem}:\textsc{pl} \textsc{aor}-die-\textsc{pl} \textsc{aor}-be.finished-\textsc{pl} be.\textsc{aff}:\textsc{fact} \\
\glt `The [elders] who knew [traditional stories] really well have all died.' (conversation, 2016-03-20)
\end{exe}


%%\begin{exe}
%%\ex \label{ex:ZYWlata}
%% \gll nɯnɯ ʑakastaka kɯ-tu nɯnɯ ɣɯ nɯ-<gongfen> ra ʑ-ɲɯ-lat-a ɕ-tu-kʰat-a pɯ-ra. \\
%%\textsc{dem} each \textsc{sbj}:\textsc{pcp}-exist \textsc{dem} \textsc{gen} \textsc{3sg}.\textsc{poss}-work.point \textsc{pl} \textsc{tral}-\textsc{ipfv}-throw-\textsc{1sg} \textsc{tral}-\textsc{ipfv}-do.everywhere-\textsc{1sg} \textsc{pst}.\textsc{ipfv}-be.needed  \\
%%\glt `I had to go everywhere to count work points for each one who was in the commune.' (2010-09, 79)
%%\end{exe}
%
%%\begin{exe}
%% \ex  \label{sec:chWCe.chWsAtCitsxi}
%% \gll  tɕe nɯtɕu tɕe tɕe ko-rɤʑi ma cʰɯ-ɕe cʰɯ-sɤtɕitʂi mɤ-kɯ-khɯ ɲɤ-k-ɤβzu-ci. \\
%% \textsc{lnk} \textsc{dem}:\textsc{loc} \textsc{loc} \textsc{lnk} \textsc{ifr}-stay \textsc{lnk} \textsc{ipfv}:\textsc{downstream}-go \textsc{ipfv}:\textsc{downstream}-continue \textsc{neg}-\textsc{sbj}:\textsc{pcp}-be.possible \textsc{ifr}-\textsc{peg}-become-\textsc{peg} \\
%%\end{exe}
%
%%\begin{exe}
%%\ex 
%%\gll  tɯrme kɯnɤ nɯ kɯ-fse, tsuku kɯ tu-nɯɕɤmɯɣdɯ-nɯ,  tsuku kɯ laʁjɯɣ tu-lɤt-nɯ, tsuku kɯ rdɤstaʁ tu-lɤt-nɯ tɕe pjɯ-sat-nɯ ŋgrɤl. \\
%%  people also \textsc{dem} \textsc{sbj}:\textsc{pcp}-be.like some \textsc{erg} \textsc{ipfv}-shoot-\textsc{pl} some \textsc{erg} cane \textsc{ipfv}-release-\textsc{pl} some \textsc{erg} stone \textsc{ipfv}-release-\textsc{pl} \textsc{lnk} \textsc{ipfv}-kill-\textsc{pl} be.usually.the.case:\textsc{fact} \\
%%\glt `People kill also it either by shooting it, hitting it with a cane or throwing stones at it.' (27-spjaNkW, 127-129)
%%\end{exe}
%
%
%%\begin{exe}
%%\ex \label{ex:YAGAme.Zo.chAtsxWB}
%% \gll nɯŋa ɯ-ndʐi nɯnɯ \rouge{cʰɤ-rɤɣdɯt}, tɕendɤre iɕqʰa nɯ \rouge{cʰɤ-tʂɯβ}. tɕe kɯ-spoʁ \rouge{ɲɤ-ɣɤ-me} ʑo \rouge{cʰɤ-tʂɯβ} qʰe,  ɯʑo ɯ-ŋgɯ nɯtɕu \rouge{ko-ɕe}.  \\
%% cow \textsc{3sg}.\textsc{poss}-skin \textsc{dem} \textsc{ifr}-peel.skin \textsc{lnk} \textsc{filler} \textsc{dem} \textsc{ifr}-sew \textsc{lnk} \textsc{sbj}:\textsc{pcp}-have.a.hole \textsc{ifr}-\textsc{caus}-not.exist \textsc{emph} \textsc{ifr}-sew \textsc{lnk}  \\
%% \glt `(The youngest brother) skinned the cow (that they had killed), and sewed (its hide). He sewed it in such a way as not to leave any holes, and went into the cowhide. ' (31-deluge, 21-25)
%%\end{exe}

\subsection{Manner converbs} \label{sec:manner.converbs}
Infinitive converbial clauses (§\ref{sec:inf.converb}) can indicate the manner of the action of the main clause, or a background event. The converb can be optionally followed by the ergative \forme{kɯ} and/or the emphatic \forme{ʑo} (§\ref{sec:emphatic.Zo}), as in (\ref{ex:kArJWG.kW.Zo2}) and (\ref{ex:kAfkur.RJa.kW}).

\begin{exe}
\ex \label{ex:kArJWG.kW.Zo2}
\gll [kɤ-rɟɯɣ] (kɯ) (ʑo) jo-ɕe \\
\textsc{inf}-run \textsc{erg} \textsc{emph} \textsc{ifr}-go \\
\glt `He went running.' (elicited)
\end{exe}

\begin{exe}
\ex \label{ex:kAfkur.RJa.kW}
\gll tɕe kɯɕɯŋgɯ ji-si nɯra [kɤ-fkur] ʁɟa kɯ jú-wɣ-sɯ-ɤzɣɯt-nɯ pɯ-ra. \\
\textsc{lnk} in.former.times \textsc{1pl}.\textsc{poss}-wood \textsc{dem}:\textsc{pl} \textsc{inf}-carry.on.the.back completely \textsc{erg} \textsc{ipfv}-\textsc{inv}-\textsc{caus}-reach-\textsc{pl} \textsc{pst}.\textsc{ipfv}-be.needed \\
\glt `In former times, one used to transport firewood exclusively by carrying it on one's back.' (140430 tWfkur, 21)
\end{exe}


The converb is often in negative form, meaning `without ...ing' as in  (\ref{ex:WGi.ra.nWmAkAsWz}) and (\ref{ex:mAkArWsWso.kW}).

\begin{exe}
\ex \label{ex:WGi.ra.nWmAkAsWz}
\gll  [ɯ-ɣi ra nɯ-mɤ-kɤ-sɯz] nɯ rŋɯl nɯ ɲɤ-mbi. \\
\textsc{3sg}.\textsc{poss}-relative \textsc{pl} \textsc{3pl}.\textsc{poss}-\textsc{neg}-\textsc{inf}-know \textsc{dem} silver \textsc{dem} \textsc{ifr}-give \\
\glt `She$_i$ gave him money without her$_i$ relatives knowing (about it).' (28-qAjdoskAt, 170)
\end{exe}


\begin{exe}
\ex \label{ex:mAkArWsWso.kW}
\gll  tɤ-mu nɯ kɯ rcanɯ, maka ʑo mɤ-kɤ-rɯsɯso kɯ to-nɤla \\
\textsc{indef}.\textsc{poss}-mother \textsc{dem} \textsc{erg} \textsc{unexp}:\textsc{foc} at.all \textsc{emph} \textsc{neg}-\textsc{inf}-think \textsc{erg} \textsc{ifr}-agree \\
\glt `The woman accepted without thinking at all.' (150907 yingning-zh, 122)
\end{exe}

The manner infinitive clauses can express the degree of the state or action in the main clause, as in (\ref{ex:kAcW.mAsAcha}) and (\ref{ex:tChi.kAcha.Zo2}). 

\begin{exe}
\ex \label{ex:kAcW.mAsAcha}
\gll [tɯ-mɲaʁ kɤ-cɯ mɤ-kɤ-sɤ-cʰa] ʑo mŋɤm  \\
\textsc{genr}.\textsc{poss}-eye \textsc{inf}-open \textsc{neg}-\textsc{inf}-\textsc{prop}-can \textsc{emph} hurt:\textsc{fact} \\
\glt `[This disease] hurts so much that one cannot open one's eyes.' (25-kACAl, 34)
\end{exe}

\begin{exe}
\ex \label{ex:tChi.kAcha.Zo2}
\gll βʑɯ nɯ kɯ [tɕʰi kɤ-cʰa] ʑo to-nɯrdoʁ nɤ to-nɯrdoʁ \\
mouse \textsc{dem} \textsc{erg} what \textsc{inf}-can \textsc{emph} \textsc{ifr}-collect \textsc{add} \textsc{ifr}-collect \\
\glt `The mouse collected as much [fruits] as it could.' (lWlu 2002, 32)
\end{exe}


The stative/impersonal infinitives in \forme{kɯ-} are also used as converbs for some anticausative verbs, modal auxiliary and existential verbs (§\ref{sec:inf.converb}) as in (\ref{ex:kAlAt.mAkWra}), and some have become lexicalized as adverbs (§\ref{sec:velar.inf.adverb}).

\begin{exe}
\ex \label{ex:kAlAt.mAkWra}
\gll nɯ [tɯ-jaʁ kɤ-lɤt mɤ-kɯ-ra] tɯ-ci kɯ cʰɯ-sɯ-mtɕɯr pɯ-ŋgrɤl \\
\textsc{dem} \textsc{genr}.\textsc{poss}-had \textsc{inf}-release \textsc{neg}-\textsc{inf}:\textsc{stat}-be.needed \textsc{indef}.\textsc{poss}-water \textsc{erg} \textsc{ipfv}:\textsc{downstream}-\textsc{caus}-turn \textsc{pst}.\textsc{ipfv}-be.usually.the.case \\
\glt `The water would make the [toy waterwheel] turn without needing to use one's hand.' (08-kWmtChW, 23)
\end{exe}

In addition, gerundive clauses (§\ref{sec:gerund.clauses}), which are used to describe actions that are simultaneous with that of the main clause (§\ref{sec:simultaneity}), occur in clauses that express manner more than temporal overlap, as in (\ref{ex:sAmtsMmtsWr.kurAZita}).

\begin{exe}
\ex \label{ex:sAmtsMmtsWr.kurAZita}
\gll  [kutɕu sɤ-mtsɯ\redp{}mtsɯr] ku-rɤʑit-a tɕe, jisŋi ndɤ tɯmɯkɯmpɕi kɯ pɯ́-wɣ-nɯ-mbi-a ɕti  \\
\textsc{dem}.\textsc{prox}:\textsc{loc} \textsc{ger}-be.hungry \textsc{ipfv}-stay-\textsc{1sg} \textsc{lnk} today \textsc{advers} heaven \textsc{erg} \textsc{aor}:\textsc{down}-\textsc{inv}-\textsc{auto}-give-\textsc{1sg} be.\textsc{aff}:\textsc{fact} \\
\glt `I am hungry while staying here, but today the heavens have sent me down [these humans to eat].' (Norbzang 2012, 291)
\end{exe}

%mɯntoʁ kɯ-dɯ-dɤn ʑo, nɤkinɯ, sɤ-rɟɯ-rɟaʁ kɯ ɯ-ŋgɯ jo-ɣi-nɯ tɕe, 
 

 \section{Causality} \label{sec:causality}

 \subsection{Consequence} \label{sec:consequence}
 The only specific marker of consequence is the linker \japhug{núndʐa}{for this reason} (\ref{ex:qapGAmtWmtW}).

\begin{exe}
\ex \label{ex:qapGAmtWmtW}
\gll ɯ-mtɯ ɣɤʑu tɕe, tɕe núndʐa qapɣɤmtɯmtɯ tu-ti-nɯ ɲɯ-ŋu \\
\textsc{3sg}.\textsc{poss}-crest exist:\textsc{sens} \textsc{lnk} \textsc{lnk} for.this.reason hoopoe \textsc{ipfv}-say-\textsc{pl} \textsc{sens}-be \\
\glt `It has a crest, and for this reason it is called `hoopoe'.' (23-qapGAmtWmtW, 20)
\end{exe}

This linker results from the fusion (§\ref{sec:3sg.possessive.form}) of the noun \japhug{ɯ-ndʐa}{reason} used in causal clauses (§\ref{sec:causal.clauses}) with the distal demonstrative \forme{nɯ} in anaphoric function `the reason of (the preceding sentence)' (§\ref{sec:anaphoric.demonstrative.pro}).\footnote{The phrase \forme{nɯnɯ ɯ-ndʐa} without vowel fusion is also found in exactly the same context in the same text. }

However, consequence is more generally simply indicated by the linkers \forme{tɕe} and/or \forme{qʰe} (§\ref{sec:coordination}), as illustrated by (\ref{ex:pWCtia.qhe.qhe}).

\begin{exe}
\ex \label{ex:pWCtia.qhe.qhe}
\gll tɕʰemɤ-pɯ pɯ-ɕti-a \textbf{qʰe} tɤ-tɕɯ ra nɯ-ɕki ku-rɤʑi-a mɯ́j-naz-a \textbf{qʰe} kɤ-ʑɣɤ-ɕɯ-fka mɯ-pɯ-naz-a \\
girl-\textsc{dim} \textsc{pst}.\textsc{ipfv}-be-\textsc{1sg} \textsc{lnk} \textsc{indef}.\textsc{poss}-son \textsc{pl} \textsc{3pl}.\textsc{poss}-\textsc{dat} \textsc{ipfv}-stay-\textsc{1sg} \textsc{neg}:\textsc{sens}-dare-\textsc{1sg} \textsc{lnk} \textsc{inf}-\textsc{refl}-\textsc{caus}-be.full \textsc{neg}-\textsc{pst}.\textsc{ipfv}-dare-\textsc{1sg} \\
\glt `As I was a little girl, I did not dare to stay at a boy's place, and thus did not dare to eat my fill.' (17-lhazgron, 45-46)
\end{exe}
 
 \subsection{Cause} \label{sec:causal.clauses}
The causal linker \japhug{matɕi}{because} is the main way to indicate cause in Japhug. In this construction, it is not clear which clause is the main clause, since the linker can be prosodically linked to both the preceding, or the following one, even with a pause after \forme{matɕi} as in (\ref{ex:matCi.YWxtCi}). The clause preceding \forme{matɕi} expresses the result/consequence (consequence clause), and the one following it the cause (causal clause).


\begin{exe}
\ex \label{ex:matCi.YWxtCi}
\gll ma nɯnɯ kʰro mɤ-sɤ-mto. matɕi ɲɯ-xtɕi. \\
\textsc{lnk} \textsc{dem} much \textsc{neg}-\textsc{prop}-see:\textsc{fact} because \textsc{sens}-be.small \\
\glt `That [species of ant] is barely visible, because it is [so] small.'
\end{exe} 

Example (\ref{ex:matCi.mAndze.Zo.me}) cannot be analyzed as an attestation of a preposed \forme{matɕi} causal clause, since the clause that follows \forme{qʰe ɯ-ɣli dɤn} is a redundant consequence clause (§\ref{sec:consequence}), repeating the real consequence clause \forme{paʁ ɣɯ ɯ-ɣli dɤn}, which is located in the expected place before the linker \forme{matɕi}.


\begin{exe}
\ex \label{ex:matCi.mAndze.Zo.me}
\gll paʁ ɣɯ ɯ-ɣli dɤn [matɕi mɤ-ndze ʑo me] qʰe ɯ-ɣli dɤn  \\
pig \textsc{gen} \textsc{3sg}.\textsc{poss}-manure be.many:\textsc{fact} because \textsc{neg}-eat[III]:\textsc{fact} \textsc{emph} not.exist:\textsc{fact} \textsc{lnk} \textsc{3sg}.\textsc{poss}-manure  be.many:\textsc{fact} \\
\glt `Pigs create a lot of manure because they eat anything, and so they create a lot of manure.' 05-paR, 106-107)
\end{exe}

The shorter form \forme{ma}\footnote{The two forms are historically related: \forme{matɕi} is probably a combination of \forme{ma} with the additive topic marker \forme{tɕi} (§\ref{sec:ri.additive}). } can also mark cause as in (\ref{ex:mAcha.ma.mpW}). However, \forme{ma} has many additional functions, including marking precautioning (§\ref{sec:precautioning.clauses}), adversative (§\ref{sec:adversative.clauses}) and exceptive (§\ref{sec:exceptive.clauses}) clauses.

\begin{exe}
\ex \label{ex:mAcha.ma.mpW}
\gll ɯ-jwaʁ nɯ (...) tu-ostɤko ʑo mɤ-cʰa ma mpɯ. \\
\textsc{3sg}.\textsc{poss}-leaf \textsc{dem} { } \textsc{ipfv}-be.straight \textsc{emph} \textsc{neg}-can \textsc{lnk} be.soft:\textsc{fact} \\
\glt `Its leaves do not grow straight (i.e. they hang down), as they are soft.' (07-kWmCku, 12)
\end{exe}

Alternatively, the relator noun \japhug{ɯ-ndʐa}{reason} (§\ref{sec:IPN.cause}, §\ref{sec:nouns.cause.complement}) with ergative (and/or locative) postpositions can serve to build causal clauses, as in (\ref{ex:tAtWta.Wndzxa.zW.kW}). In some rare cases, \forme{ɯ-ndʐa}-clauses can have a purposive meaning (example \ref{ex:mWYWkABzu.Wndzxa}, §\ref{sec:nouns.cause.complement}).

\begin{exe}
\ex \label{ex:tAtWta.Wndzxa.zW.kW}
\gll [``nɤʑo ɯ-rɯz ɣɤʑu" tɤ-tɯt-a ɯ-ndʐa] zɯ kɯ a-pa a-ma ni kɯ ``mtsʰukʰa ɯ-ŋgɯ tɕe (...) tɤrca pɯ-ɕe ma mɤ-jɤɣ'' \\
\textsc{2sg} \textsc{3sg}.\textsc{poss}-supernatural.power exist:\textsc{sens} \textsc{aor}-say[II]-\textsc{1sg} \textsc{3sg}.\textsc{poss}-reason \textsc{loc} \textsc{erg} \textsc{1sg}.\textsc{poss}-father.\textsc{hon} \textsc{1sg}.\textsc{poss}-mother.\textsc{hon} \textsc{du} \textsc{erg} lake \textsc{3sg}.\textsc{poss}-in \textsc{loc} {  } together \textsc{imp}:\textsc{down}-go apart.from \textsc{neg}-be.allowed:\textsc{fact} \\
\glt `Because I said that you had supernatural powers (§\ref{sec:hybrid indirect}), my parents [said] `go together with him into the lake.'(Nyima wodzer 2003.2, 70-73)
\end{exe}

Instead of being marked with the ergative, the causal clauses can also serve as predicate of a copular construction (example \ref{ex:tCetha.nWmbrAt}, §\ref{sec:aor.temporal}). Predicative causal clauses occur often standalone as in (\ref{ex:tutWtinW.ndzxa.Nu}), without any overt clause expressing the result.


\begin{exe}
\ex \label{ex:tutWtinW.ndzxa.Nu}
\gll [``nɯ ma-tɤ-tɯ-ste, nɯ ma-tɤ-tɯ-fse" tu-tɯ-ti-nɯ] ndʐa ŋu wo \\
\textsc{dem} \textsc{neg}-\textsc{imp}-2-do.like[III] \textsc{dem}  \textsc{neg}-\textsc{imp}-2-be.like \textsc{ipfv}-2-say-\textsc{pl} reason be:\textsc{fact} \textsc{sfp} \\
\glt `(If your son wants so much to see his grandparents), this is because you (his parents always) tell him `Don't act like that, don't be like that.' (unlike grandparents, who are more lenient).' (conversation 16-08-11)
\end{exe}

Apart from \forme{ndʐa}, the relator nouns \japhug{ɯ-tʰɯrʑi}{mercy} and \japhug{ɯ-xɕɤt}{strength}
can also be used to indicate cause on both noun phrases (§\ref{sec:IPN.cause}) and subordinate clauses.

The noun \japhug{ɯ-tʰɯrʑi}{mercy} is specifically used to indicate beneficial actions thanks to which a desirable result is obtained (\ref{ex:tanWsmAn.WthWrZi}).

\begin{exe}
\ex \label{ex:tanWsmAn.WthWrZi}
\gll [<guojia> kɯ ta-nɯsmɤn] ɯ-tʰɯrʑi tɕe tɕe  (...) tham tɕe wuma ʑo tɤ-pe-nɯ. \\
country \textsc{erg} \textsc{aor}:3\flobv{}-heal \textsc{3sg}.\textsc{poss}-mercy \textsc{lnk} \textsc{lnk} { } now \textsc{lnk} really \textsc{emph} \textsc{aor}-be.good-\textsc{pl} \\
\glt `Thanks to the fact that [our] country['s governement] has healed them, (...) now they are much better.' (140522 RdWrJAt, 139)
\end{exe}

The locution \forme{$X$ ɯ-xɕɤt kɯ $Y$} with the ergative specifically means `do $X$ so much/to the extent that $Y$', as in (\ref{ex:khu.kW.tAwGndzanW.WxCAt}), specifying not only that $X$ is the cause of $Y$, but in addition that $X$ was performed to/with a sufficiently high degree/frequency/ to make the action/situation $Y$ possible.\footnote{The locution \forme{$X$ ɯ-xɕɤt kɯ} is reminds of French \forme{à force de $X$}. } More data on degree causal constructions is provided in §\ref{sec:degree.consequence}.

\begin{exe}
\ex \label{ex:khu.kW.tAwGndzanW.WxCAt}
\gll tsʰupa ci pjɤ-tu tɕe tɕendɤre, [kʰu kɯ tɤ́-wɣ-ndza-nɯ] ɯ-xɕɤt kɯ tɤ-mu kɤtsa ci pjɤ-ri-ndʑi tɕe \\
village \textsc{indef} \textsc{ifr}.\textsc{ipfv}-exist \textsc{lnk} \textsc{lnk} tiger \textsc{erg} \textsc{aor}-\textsc{inv}-eat-\textsc{pl} \textsc{3sg}.\textsc{poss}-strength \textsc{erg} \textsc{indef}.\textsc{poss}-mother \textsc{coll}:family \textsc{indef} \textsc{ifr}.\textsc{ipfv}-remain-\textsc{du} \textsc{lnk} \\
\glt `There was a village, but a tiger had eaten them (i.e. the villagers) so that only a mother and her daughter remained.' (khu 2012, 2)
\end{exe}
 
\subsection{Prerequisite} \label{sec:prerequisite.clause}
Prerequisite constructions do not express the direct cause of the action/situation in the main clause, but simply the basic conditions for that action or situation to be possible, like \ch{既然}{jìrán}{since} in Chinese. This meaning is expressed by a clause in the Aorist as in (\ref{ex:tAcha.tANu}), formally similar to a temporal clause (§\ref{sec:aor.temporal}, §\ref{sec:temporal.reference}).

\begin{exe}
\ex \label{ex:tAcha.tANu}
\gll <huangdi> (...) tú-wɣ-sɯ-ndo-a tɤ-cʰa tɤ-ŋu tɕe (...) <jiaohuang> nɯ ʑgrɯɣ tú-wɣ-sɯ-ndo-a cʰa tɕe, ɕ-tɤ-ti ra \\
emperor {  } \textsc{ipfv}-\textsc{inv}-\textsc{caus}-take-\textsc{1sg} \textsc{aor}-can \textsc{aor}-be \textsc{lnk} {  } pope \textsc{dem} certainly \textsc{ipfv}-\textsc{inv}-\textsc{caus}-take-\textsc{1sg} can:\textsc{fact} \textsc{lnk} \textsc{tral}-\textsc{imp}-go be.needed:\textsc{fact} \\
\glt `Since [the golden fish] has succeeded in making me emperor, he will certainly succeed in making me pope, go and tell him that.' (140430 yufu he tade qizi-zh, 204)
\end{exe}

The minimal clause \forme{nɯ tɤ-ŋu tɕe} can have a prerequisite interpretation `given these circumstances, in this case' as in (\ref{ex:nW.tANu.tCe}).

\begin{exe}
\ex \label{ex:nW.tANu.tCe}
\gll nɯ tɤ-ŋu tɕe tɕe, si lú-wɣ-ɣɤjɯ, smi a-tɤ-wxti tɕe ɲɯ-pʰɤn \\
\textsc{dem} \textsc{aor}-be \textsc{lnk} \textsc{lnk} wood \textsc{ipfv}-\textsc{inv}-add fire \textsc{irr}-\textsc{pfv}-be.big \textsc{lnk} \textsc{sens}-be.efficient \\
\glt `In this case, if one adds more wood, and makes the fire bigger, it will work.' (150827 taisui-zh, 63)
\end{exe}


 \subsection{Purposive clauses} \label{sec:purposive.clauses}
 Five main constructions are available in Japhug to express purpose.\footnote{This section does not include the  participial clauses (§\ref{sec:subject.participle.complementation}) used as supine purposive clauses of motion verbs  (§\ref{sec:purposive.clause.motion.verbs}).
 }

First, the purposive converb (§\ref{sec:purposive.converb}) is a non-finite verb form dedicated to expressing the purpose of the action in the main clause. It is built by combining the prefix \forme{sɤ(z)-} with a B-type preverb (§\ref{sec:kamnyu.preverbs}), the reduplicated verb stem and a possessive prefix coreferent with one of the core arguments of the purposive clause (and almost always in negative form). Purposive converbs are productive: even denominal verbs of Tibetan origin such as  \japhug{nɯtɕʰomba}{catch a cold}  (\ref{ex:WmAtusAznWtChombWmba}) can derive this type of converbs. However, they are barely attested in the Japhug corpus, and simpler constructions are preferred to express purpose.


\begin{exe}
\ex \label{ex:WmAtusAznWtChombWmba}
\gll  a-tɕɯ ɯ-mɤ-tu-sɤz-nɯtɕʰombɯ\redp{}mba tɯ-ŋga kɯ-jaʁ tɤ-ʑ-ŋga-t-a \\
\textsc{1sg}.\textsc{poss}-son \textsc{3sg}.\textsc{poss}-\textsc{neg}-\textsc{ipfv}-\textsc{purp}:\textsc{conv} \textsc{indef}.\textsc{poss}-clothes \textsc{sbj}:\textsc{pcp}-be.thick \textsc{aor}-\textsc{caus}-wear-\textsc{pst}:\textsc{tr}-\textsc{1sg} \\
\glt `In order to prevent my son from catching a cold, I made him wear heavy clothes.' (elicited)
\end{exe}

Second, the  the verb \japhug{nɯmga}{want from}  can be combined with an infinitival clause to express purpose, as in (\ref{ex:mWYWkABzu.kANWmga}) to express the aim of the action in the main clause. 


\begin{exe}
\ex \label{ex:mWYWkABzu.kANWmga}
\gll tɕe [kɯpɤz nɯ mɯ-ɲɯ-kɤ-βzu kɤ-nɯmga], iʑɤra, ji-mthɯm nɯra <binggui> tu-χtɯ-j tɕe nɯ ɯ-ŋgɯ ri pjɯ-nɯ-rku-j ɕti ma \\
\textsc{lnk} type.of.bug \textsc{dem} \textsc{neg}-\textsc{ipfv}-\textsc{inf}-grow \textsc{inf}-want.from \textsc{1pl} \textsc{1pl}.\textsc{poss}-meat \textsc{dem}:\textsc{pl} refrigerator  \textsc{ipfv}-buy-\textsc{1pl} \textsc{lnk} \textsc{dem} \textsc{3sg}.\textsc{poss}-in \textsc{loc} \textsc{ipfv}-auto-put.in-\textsc{1pl} be.\textsc{aff}:\textsc{fact} \textsc{lnk} \\
\glt `In order to prevent \forme{kɯpɤz} bugs from growing (in the meat), we buy refrigerators and put our meat in it.' (28-kWpAz, 46)
\end{exe}

The verb \forme{nɯmga} in this function is generally a velar infinitive \forme{kɤ-nɯmga}  in converbial function (§\ref{sec:inf.converb}) as in (\ref{ex:mWYWkABzu.kANWmga}) and (\ref{ex:kABzjoz.kAnWmga}), but a finite verb is also possible (\ref{ex:kABzjoz.kAnWmgata}). The infinitival clause can be focalized, serving as the predicate of a copular construction in \forme{ŋu} (\ref{ex:kABzjoz.kAnWmga.Nu}).

\begin{exe}
\ex 
\begin{xlist}
\ex \label{ex:kABzjoz.kAnWmga}
\gll kɯrɯ-skɤt kɤ-βzjoz kɤ-nɯmga kɯ, mbarkʰom mɤɕtʂa jɤ-ɣe-a ŋu. \\
Gyalrong-language \textsc{inf}-learn \textsc{in}-want.from  \textsc{erg}  \textsc{topo} until \textsc{aor}-come[II]-\textsc{1sg} be:\textsc{fact} \\
\ex \label{ex:kABzjoz.kAnWmgata}
\gll  kɯrɯ-skɤt kɤ-βzjoz kɤ-nɯmga-t-a tɕe, mbarkʰom mɤɕtʂa jɤ-ɣe-a ŋu. \\
Gyalrong-language \textsc{inf}-learn \textsc{aor}-want.from-\textsc{pst}:\textsc{tr}-\textsc{1sg} \textsc{lnk}  \textsc{topo} until \textsc{aor}-come[II]-\textsc{1sg} be:\textsc{fact} \\
\glt `I came all the way up to Mbarkham to learn the Gyalrong language.' (elicited)
\ex \label{ex:kABzjoz.kAnWmga.Nu}
\gll mbarkʰom ju-ɣi-a nɯ, kɯrɯ-skɤt kɤ-βzjoz kɤ-nɯmga ŋu \\
\textsc{topo} \textsc{ipfv}-come-\textsc{1sg} \textsc{dem} Gyalrong-language \textsc{inf}-learn \textsc{inf}-want.from be:\textsc{fact} \\
\glt `(The reason why) I come to Mbarkham is to learn the Gyalrong language.' (elicited)
\end{xlist}
\end{exe}

Third, the relator noun \japhug{ɯ-spa}{its material} (§\ref{sec:lexicalized.oblique.participle}, §\ref{sec:Wspa.relative}) can take object (\ref{ex:WkAnAmYo.Wspa}) and subject (\ref{ex:WkWBzjoz.Wspa}) participial clauses with a purposive meaning. These constructions are to be analyzed as essive participial clauses (§\ref{sec:participial.clause.essive}).

\begin{exe}
\ex \label{ex:WkAnAmYo.Wspa}
\gll tɯrme ra kɯ ɯ-<shipin> jo-lɤt-nɯ tɕe [nɯnɯ ɯ-kɤ-nɤmɲo] ɯ-spa jo-lɤt-nɯ ɲɯ-ŋu. \\
people \textsc{pl} \textsc{erg} \textsc{3sg}.\textsc{poss}-video \textsc{ifr}-release-\textsc{pl} \textsc{lnk} \textsc{dem} \textsc{3sg}.\textsc{poss}-\textsc{obj}:\textsc{pcp}-watch \textsc{3sg}.\textsc{poss}-material \textsc{ifr}-release-\textsc{pl} \textsc{sens}-be \\
\glt `People send him this video for him to watch it.' (conversation 2019-02-26)
\end{exe}

\begin{exe}
\ex \label{ex:WkWBzjoz.Wspa}
\gll [kɯrɯ-skɤt ɯ-kɯ-βzjoz] ɯ-spa, mbarkʰom mɤɕtʂa jɤ-ɣe-a ŋu. \\
Gyalrong-language \textsc{3sg}.\textsc{poss}-\textsc{sbj}:\textsc{pcp}-learn \textsc{3sg}.\textsc{poss}-material  \textsc{topo} until \textsc{aor}-come[II]-\textsc{1sg} be:\textsc{fact} \\
\glt `I came all the way up to Mbarkham to learn the Gyalrong language.' (elicited)
\end{exe}

Essive participial clauses without \forme{ɯ-spa} can also have a purposive meaning, as in (\ref{ex:nWkWqur.tundonW}).

\begin{exe}
\ex \label{ex:nWkWqur.tundonW}
\gll  kɯ-ɣɤrʁaʁ ra kɯ tu-ndo-nɯ. [nɯ-kɯ-qur] tu-ndo-nɯ pjɤ-ŋgrɤl. \\
\textsc{sbj}:\textsc{pcp}-hunt \textsc{pl} \textsc{erg} \textsc{ipfv}-take-\textsc{pl} 3pl.\textsc{poss}-\textsc{sbj}:\textsc{pcp}-help \textsc{ipfv}-take-\textsc{pl} \textsc{ipfv}.\textsc{ifr}-be.usually.the.case \\
\glt `(In former times), hunters would take [dogs] (on a hunt). They would take dogs to help them (i.e., as their helpers).' (05-khWna, 39-40)
\end{exe}

Fourth, a biclausal construction comprising a clause containing a similative verb (\japhug{fse}{be like} or \japhug{stu}{do like}) in finite form with the interrogative pronoun (§\ref{sec:tChi}), and another finite clause $Y$ linked by \forme{tɕe}, specifically means `what should $X$ do in order to $Y$' as in (\ref{ex:tufsej.jitWci.GAZu}) and (\ref{ex:atAstunW.nWtWci.GAZu}).

 \begin{exe}
 \ex  
 \begin{xlist}
\ex \label{ex:tufsej.jitWci.GAZu}
\gll  iʑora ɣɯ tɕʰi tu-fse-j tɕe ji-tɯ-ci ɣɤʑu (...) tu-tɯ-tʰe ɯ-tɯ́-cʰa \\
\textsc{1pl} \textsc{gen} what \textsc{ipfv}-be.like-\textsc{1pl} \textsc{lnk} \textsc{1pl}.\textsc{poss}-\textsc{indef}.\textsc{poss}-water exist:\textsc{sens} { } \textsc{ipfv}-2-ask[III] \textsc{qu}-2-can:\textsc{fact} \\
\glt `Can you ask for us what we [need to] do in order to have water?' (said by villagers living in a desert where there is no water, divination 2005, 14)
\ex \label{ex:atAstunW.nWtWci.GAZu}
\gll nɯnɯra kɯ tɕhi a-tɤ-stu-nɯ tɕe nɯ-tɯ-ci ɣɤʑu? \\
\textsc{dem} \textsc{erg} what \textsc{irr}-\textsc{pfv}-do.like-\textsc{pl} \textsc{lnk} \textsc{3pl}.\textsc{poss}-\textsc{indef}.\textsc{poss}-water exist:\textsc{sens} \\
\glt `What should they do in order to have water?' (divination 2005, 38)
 \end{xlist}
\end{exe} 

 The clause in \forme{fse} can be embedded within the clause $Y$, as shown by example (\ref{ex:atAfsea.mAwGmtoa}) (§\ref{sec:embedded.clause}).


 Fifth, purposive meaning can be expressed by a clause in the Irrealis (§\ref{sec:irrealis} with the adverb \japhug{tɕetʰa}{later} as in (\ref{ex:tCetha.amAlAZGWt}). This type of construction resembles precautioning clauses (§\ref{sec:precautioning.clauses}), but with opposite polarity.

 \begin{exe}
\ex \label{ex:tCetha.amAlAZGWt}
\gll lu-zɣɯt ɕɯŋgɯ tɕe nɯtɕu ku-ɣɤrat-a tɕetʰa a-mɤ-lɤ-zɣɯt \\
\textsc{ipfv}:\textsc{upstream}-arrive before \textsc{lnk} \textsc{dem}:\textsc{loc} \textsc{ipfv}-throw-\textsc{1sg} later \textsc{irr}-\textsc{neg}-\textsc{pfv}:\textsc{upstream}-arrive \\
\glt `Before it arrives, I will throw [the enchanted white stone], so that it does not arrive.' (25-kAmYW-XpAltCin, 62)
\end{exe}

Finally, clauses with the relator \japhug{ɯ-ndʐa}{reason} can also have a purposive interpretation (example \ref{ex:mWYWkABzu.Wndzxa}, §\ref{sec:nouns.cause.complement}, which resembles causal clauses §\ref{sec:causal.clauses}).

\subsection{Justification clauses} \label{sec:justification.clauses} 
Justification clauses \citep{lopes09justification} differ from strictly clausal clauses in that there is no direct causal  relationship between the main clause and the justification clause. Rather, the truth value of the latter allows to make an inference concerning the truth value of the latter. 

In Japhug, there is no dedicated construction to express such meaning, and as in many languages, the clausal linkers \forme{ma} and \forme{matɕi} (§\ref{sec:causal.clauses}) can be used (\ref{ex:matCi.justification}). 

\begin{exe}
\ex \label{ex:matCi.justification}
\gll cʰa tɕi ko-tsʰi matɕi lo-βzi ɕti ri \\
alcohol also \textsc{ifr}-drink \textsc{lnk} \textsc{ifr}-be.drunk be.\textsc{aff}:\textsc{fact} \textsc{lnk} \\
\glt `He had also drunk alcohol, since he was drunk.' (140506 loBzi, 13)
 \end{exe}

The justification meaning is clearest when the verb of the main clause takes the Probabilitative \forme{ɯmɤ-} (§\ref{sec:WmA.kW.ci}) or the Rhetorical Interrogative \forme{ɯβrɤ-} (§\ref{sec:WBrA.kW.ci}) prefixes with peg circumfix, and that in the clause following \forme{ma} is in the Sensory, as in (\ref{ex:WmAkWNuci.justification}) and (\ref{ex:WBrApWtWnWZWBci.justification}).
 
\begin{exe}
\ex \label{ex:WmAkWNuci.justification}
 \gll ɕɤxɕo pɤnmawombɤr jo-nɯ-ɣi ɯmɤ-kɯ-ŋu-ci ma, kʰa ɲɯ-ɣɤkʰɯ-nɯ \\
these.days  \textsc{anthr} \textsc{ifr}-\textsc{vert}-come \textsc{prob}-\textsc{peg}-be-\textsc{peg} \textsc{lnk} house \textsc{sens}-have.smoke-\textsc{pl} \\
\glt `Maybe Padma 'Od'bar came back in the last few days, since there is smoke coming out from their house.' (Norbzang 2012, 226-227)
\end{exe} 

 \begin{exe}
\ex \label{ex:WBrApWtWnWZWBci.justification}
 \gll  jɯfɕɯɕɤr ɯβrɤ-pɯ-tɯ-nɯʑɯβ-ci ma nɤ-mɲaʁ ɲɯ-ɣɯrni \\
 last.night \textsc{rh}.\textsc{q}-\textsc{pst}.\textsc{ipfv}-2-sleep-\textsc{peg} \textsc{lnk} \textsc{2sg}.\textsc{poss}-eye \textsc{sens}-be.red \\
 \glt `It seems that you did not sleep (well) last night, since your eyes are red.' (elicited)
\end{exe} 

 The use of \forme{ma} as a sentence final particle with the modal tenses (§\ref{sec:fsp.epistemic}) possibly originate from constructions as in (\ref{ex:WmAkWNuci.justification}) and (\ref{ex:WBrApWtWnWZWBci.justification}) with elided justification clauses.
 
\subsection{Precautioning clauses} \label{sec:precautioning.clauses} 
Precautioning constructions comprise a clause describing an undesirable result, and another clause referring to a measure that can be taken to prevent this result. The precautioning meaning can be expressed with a negative purposive clause (§\ref{sec:purposive.clauses}); in particular, purposive converbs, which are almost always in negative forms (§\ref{sec:purposive.converb}), could be described as a type of precautioning clauses.

However, the most common way of expressing precautioning meaning in Japhug is by coordinating a clause in the Irrealis (§\ref{sec:irrealis}), the Imperative (§\ref{sec:imperative}) or the Prohibitive (§\ref{sec:prohibitive}) indicating the preventive measure, with a clause in the Factual Non-Past (§\ref{sec:factual}) referring to the undesired event. The two clauses are coordinated by the linker \forme{ma} (\ref{ex:ma.tWndzxaB}) and/or the adverb \japhug{tʰa}{later} or \forme{tɕetʰa}, as illustrated by (\ref{ex:matABzea}) and (\ref{ex:ma.tCetha.junWCe}), a crosslinguistically common strategy to express precautioning meaning \citep{angelo16beware}.

\begin{exe}
\ex \label{ex:ma.tWndzxaB}
\gll  ma-jɤ-tɯ-ɕe ma tɯ-ndʐaβ \\
\textsc{neg}-\textsc{imp}-2-go \textsc{lnk} 2-\textsc{acaus}:cause.to.roll.down:\textsc{fact} \\
\glt `Don't go (to the temple on the mountain to do circumambulations), you might fall down.' (conversation, conversation 15-12-05)
\end{exe}

\begin{exe}
\ex \label{ex:matABzea}
\gll nɯ kʰramba ma-tɤ-βze-a ra ma tɕe <lishijizai> pjɯ-tɯ-βze ɕti tɕetʰa, nɯnɯ rcanɯ <zuzubeibei> kɯ ɣɯ-nɤmqe-a-nɯ.  \\
\textsc{dem}  lie \textsc{neg}-\textsc{imp}-make[III]-\textsc{1sg} be.needed:\textsc{fact} \textsc{lnk} \textsc{lnk}   history record \textsc{ipfv}-2-make[III] be:\textsc{aff}:\textsc{fact} \textsc{lnk} later generations \textsc{erg}  \textsc{inv}-scold:\textsc{fact}-\textsc{1sg}-\textsc{pl} \\
\glt `I cannot tell lies, as   you are making a historical record, and previous and future generations would scold me.'  (27-kikakCi, 224)
 \end{exe}

\begin{exe}
\ex \label{ex:ma.tCetha.junWCe}
\gll a-mɤ-tʰɯ-sta ma tɕetʰa ju-nɯ-ɕe ɕti \\
\textsc{irr}-\textsc{neg}-\textsc{pfv}-wake \textsc{lnk} later \textsc{ipfv}-\textsc{vert}-go be.\textsc{aff}:\textsc{fact} \\
\glt `May she not wake up (Don't wake her up), otherwise she will go away.' (150818 muzhi guniang-zh, 106)
\end{exe}  

There is no categorical requirement for the first clause to contain a modal TAME category. In (\ref{ex:kWxtshWm.mApe.ma}), we find instead the negative form \forme{mɤ-pe} `it is not good' with a non-finite clause that can either be interpreted as a stative infinitive (`it is not good (if) they are thin') or a subject participle (`the thin ones are not good').

\begin{exe}
\ex \label{ex:kWxtshWm.mApe.ma}
\gll roʁre nɯ kɯnɤ [wuma ʑo kɯ-xtsʰɯm] mɤ-pe ma tɕe pjɯ-qlɯt-nɯ. \\
fence.rail \textsc{dem} also really \textsc{emph} \textsc{inf}:\textsc{stat}-be.thin \textsc{neg}-be.good \textsc{lnk} \textsc{lnk} \textsc{ipfv}-break-\textsc{pl} \\
\glt `As for fence rails also, it is not good [where they are] too thin, otherwise [the hybrid yaks in the cowshed] will break them.' (150902 mkhoN, 42)
\end{exe}  

\section{Other constructions}
 
\subsection{Adversative} \label{sec:adversative.clauses}

\subsubsection{Concession} \label{sec:concessive.clauses}
The most common way of expressing concession is the linker \forme{tɕeri} `but, however', which can occur in sentence-initial position, after a pause as in (\ref{ex:tCeri.WmAlAjaR.GAZu}).

\begin{exe}
\ex \label{ex:tCeri.WmAlAjaR.GAZu}
\gll ɯ-βri nɯra qapri ɯ-βri wuma ʑo ɲɯ-fse, ɯ-rme ri kɯ-tu maŋe. tɕeri ɯ-mɤlɤjaʁ ɣɤʑu. \\
\textsc{3sg}.\textsc{poss}-body \textsc{dem}:\textsc{pl} snake \textsc{3sg}.\textsc{poss}-body really \textsc{emph} \textsc{sens}-be.like \textsc{3sg}.\textsc{poss}-hair also \textsc{sbj}:\textsc{pcp}-exist not.exist:\textsc{sens} \textsc{lnk} \textsc{3sg}.\textsc{poss}-limb exist:\textsc{sens} \\
\glt `[The gecko's] body looks a little bit like the body of a snake, and it too has no hair. However, it has limbs.' (28-tshAwAre, 10)
\end{exe}  

The shorter variant \forme{ri}\footnote{The form \forme{tɕeri} is a combination of the linker \forme{tɕe} with \forme{ri}; the concessive meaning of \forme{ri} itself possibly derives from its function as a locative postposition, used with a finite clause as a temporal subordinator as in (\ref{ex:pWkWxtCi.ri}) (§\ref{sec:temporal.reference}). } is also used to indicate concession (\ref{ex:tu.ri.mAdAn}) when used between coordinated clauses. When \forme{ri} follows noun phrases or complement clauses, it is a different marker, either that of additive focus `also' as in (\ref{ex:tCeri.WmAlAjaR.GAZu}) above (§\ref{sec:ri.additive}, §\ref{sec:addition.clauses}) or a locative postposition (§\ref{sec:core.locative}).

\begin{exe}
\ex \label{ex:tu.ri.mAdAn}
\gll kuwu nɯ iʑo kutɕu ji-ʑimkʰɤm tɕe tu ri mɤ-dɤn \\
bearded.vulture \textsc{dem} \textsc{1sg} \textsc{dem}.\textsc{prox}:\textsc{loc} \textsc{1pl}.\textsc{poss}-region \textsc{loc} exist:\textsc{fact} \textsc{lnk} \textsc{neg}-be.many:\textsc{fact} \\
\glt `Bearded vultures are found here in our region, but not many.' (2011-08-kuwu, 28)
\end{exe}  
 
 The linker \forme{ma} also has a concessive meaning in some contexts, as in (\ref{ex:toznWGmaz.ma.kAsat.mWpjAcha}). This function possibly derives from its use as an exceptive postposition meaning `apart from' (§\ref{sec:exceptive}, §\ref{sec:exceptive.clauses}).
 
\begin{exe}
\ex \label{ex:toznWGmaz.ma.kAsat.mWpjAcha}
\gll qaliaʁ nɯ to-z-nɯɣmaz ma kɤ-sat nɯ mɯ-pjɤ-cʰa\\
eagle \textsc{dem} \textsc{ifr}-\textsc{caus}-have.a.wound \textsc{lnk} \textsc{inf}-kill \textsc{dem} \textsc{neg}-\textsc{ifr}-can\\
\glt `He wounded the eagle, but could not kill it.' (150902 hailibu-zh, 21)
\end{exe}  
 
The topic marker \forme{ndɤre} (§\ref{sec:adversative.topic}) can also indicate adversative meaning, with scope overt one constituent of the sentence, in (\ref{ex:ndAre.mAsna}) over an infinitival clause.
 
\begin{exe}
\ex \label{ex:ndAre.mAsna}
\gll paʁ kɯ tɕi ndze, nɯŋa kɯ tɕi ndze. tɕe [tɯrme kɤ-ndza] ndɤre mɤ-sna.   \\
pig \textsc{erg} too eat[III]:\textsc{fact} cow  \textsc{erg} too eat[III]:\textsc{fact} \textsc{lnk} people \textsc{inf}-eat  \textsc{contrast}:\textsc{foc}  \textsc{neg}-be.good:\textsc{fact} \\
\glt `Pigs eat it, cows eat it, but it is not good for people to eat.'  (17-ndZWnW, 125-126)
\end{exe} 

In the register of traditional stories, the locution \forme{jinbala zɯ} is also employed to mark adversative clauses as in (\ref{ex:jinbala.zW}). The linker \forme{jinbala} is borrowed from Tibetan \tibt{ཡིན་པ་ལ་}{jin.pa.la}, a non-finite form of the copula \tibet{ཡིན་}{jin}{be} which also has adversative meaning in the classical language.

\begin{exe}
\ex \label{ex:jinbala.zW}
\gll tɕe rɟɤlpu nɯ nɯ-rga jinbala zɯ ``e, a-tɕɯ ki stɤβtsʰɤt ci mɯ-ɕɯ-cʰa kɯ" ɲɤ-sɯso tɕe \\
\textsc{lnk} king \textsc{dem} \textsc{aor}-be.happy although \textsc{loc} \textsc{interj} \textsc{1sg}.\textsc{poss}-son \textsc{dem}.\textsc{prox} contest \textsc{indef} \textsc{neg}-\textsc{apprehensive}-can \textsc{sfp} \textsc{ifr}-think \textsc{lnk} \\
\glt `Although the king was happy (about that), he thought `I am worried that my son will not win the contest.' (2003 sras, 84)
\end{exe} 

Finally, velar infinitive converbs (§\ref{sec:inf.converb}) with the ergative can have an adversative interpretation,  as in (\ref{ex:kWnA.mAkAti.kW}), where the converbial clause is better translated by `although ...' than `without...'.\footnote{In the Chinese original, \ch{…便主动要求嫁给国王}{biàn zhǔdòng yāoqiú jiàgěi guówáng}{... and she asked to marry the king of her own initiative}, what corresponds to the converbial clause is the adverb \ch{主动}{zhǔdòng}{of ...'s own initiative}. This construction cannot be a calque.}

\begin{exe}
\ex \label{ex:kWnA.mAkAti.kW}
\gll  ``jɤ-ɣi" kɯnɤ mɤ-kɤ-ti kɯ, ``nɤ-rʑaβ ɣɯ-ku-βze-a" to-ti, rɟɤlpu nɯ ɯ-ɕki. \\
\textsc{imp}-come also \textsc{neg}-\textsc{inf}-say \textsc{erg} \textsc{2sg}.\textsc{poss}-wife \textsc{cisl}-\textsc{ipfv}-make[III]-\textsc{1sg} \textsc{ifr}-say king \textsc{dem} \textsc{3sg}.\textsc{poss}-\textsc{dat} \\
\glt `Although [the king had not] said `come' (i.e.  `without the king saying `come''), she told the king `I am coming to become your wife.'  (140511 yinzi-zh, 21)
\end{exe}

\subsubsection{Rectification} \label{sec:rectification.clauses}
The negative copula \japhug{maʁ}{not be} followed by the ergative \forme{kɯ} negates the first clause ($A$), and indicates that the events described by following clause(s) $B$ instead are true. The meaning of this construction can be glossed `not $A$, but rather $B$' as in (\ref{ex:CkurAloR.YWmaR.kW}). 

\begin{exe}
\ex \label{ex:CkurAloR.YWmaR.kW}
\gll  kɯjka nɯnɯ sɯŋgɯ cʰiz ɕ-ku-rɤloʁ ɲɯ-maʁ kɯ, (...) kuxtɕo ɯ-mbe ɯ-ŋgɯ ri ku-rɤloʁ \\
chough \textsc{dem} forest \textsc{approx}:\textsc{loc} \textsc{tral}-\textsc{ipfv}-make.nest \textsc{sens}-not.be \textsc{erg} { } basket \textsc{3sg}.\textsc{poss}-old.one \textsc{3sg}.\textsc{poss}-in \textsc{loc} \textsc{ipfv}-make.nest \\
\glt `The chough does not go in the forest and makes a nest there, but rather makes a nest (...) in old baskets, (...)' (22-CAGpGa, 48-53)
\end{exe} 

The adversative additive adverb \japhug{mɤ́ɣrɤz}{instead}, `on the contrary' (and its variant \forme{mɤ́ɣrɤz nɤ}), semantically similar to Chinese \ch{反而}{fǎn'ér}{instead}, specifically indicates an undesirable result that occurs contrary to expectation instead of the intended result.

The clause expressing the intended result can take the comparative postposition (standard marker) \japhug{sɤz}{compared with} (§\ref{sec:comparative}), as in (\ref{ex:ndAre.mAsna}).

\begin{exe}
\ex \label{ex:sAznA.mAGrAZ}
\gll pjɤ-kɯ-nɯβlu-a tɕe, nɯnɯ kɯ-pʰɤn sɤznɤ, mɤ́ɣrɤz a-mpʰɯz ɯ-ntɕʰɯr pa-nɯ-pʰɯt \\
\textsc{ifr}-2\fl{}1-cheat-\textsc{1sg} \textsc{lnk} \textsc{dem} \textsc{sbj}:\textsc{pcp}-be.efficient \textsc{comp} instead \textsc{1sg}.\textsc{poss}-bottom \textsc{3sg}.\textsc{poss}-piece \textsc{aor}:3\flobv{}-\textsc{auto}-take.off \\
\glt `You cheated me, not only was [your healing method] not efficient [to treat my illness], but it took off a chunk of my bottom. (140427 qala cho kWrtsAG, 44)
\end{exe} 

In most cases however, \forme{mɤ́ɣrɤz} occurs without any additional marker, as in (\ref{ex:mAGrAz.YWmNAm}).

\begin{exe}
\ex \label{ex:mAGrAz.YWmNAm}
\gll  ɕɤr tɕe ʑŋgri nɯ nɯ-mɤrʑaβ ʑo tɕe tɯ-kɤrme cʰɯ́-wɣ-rɤɕi ri, cʰɯ-kɯ-zri maŋe, mɤ́ɣrɤz ɲɯ-mŋɤm ma. \\
night \textsc{loc} star \textsc{dem} \textsc{aor}-marry \textsc{emph} \textsc{lnk} \textsc{genr}.\textsc{poss}-hair \textsc{ipfv}-\textsc{inv}-pull \textsc{lnk} \textsc{ipfv}-\textsc{pcp}:\textsc{sbj}-be.long not.exist:\textsc{sens} instead \textsc{sens}-hurt \textsc{sfp} \\
\glt `In the night when there was a shooting star, we pulled our hair (thinking that our hair would grow longer), but not only did [our hair] not grow longer, [but the only thing that it did was] it hurt.' (29-mWBZi, 108)
\end{exe} 

 \subsubsection{Evidential rectification} \label{sec:evd.rectification.clauses}
 The Aorist \forme{tɤ-mda} (§\ref{sec:aor}) of the verb \japhug{mda}{arrive} (of time),  without complement clause, and following a clause with the adversative linker \forme{ri}, has the special meaning `actually', `it turns out that...', `but in fact ...'. It is specifically used when a referent realizes that that the reality (the rectification clause) was different from his/her/its original belief or expectation. 

This construction is common in stories translated from Chinese, as in (\ref{ex:tAmda.pjACti}) and (\ref{ex:tAmda.pjAtWNu}), where \forme{tɤ-mda} corresponds to the Chinese linkers \ch{结果}{jiéguǒ}{in the end} or  \ch{原来}{yuánlái}{it turns out that}.\footnote{For instance, the original of (\ref{ex:tAmda.pjAtWNu}) is \ch{原来是我的外甥来了}{yuánlái shì wǒ de wàishēng láile}{it turns out that (the one who came, you) is my nephew}. }  

\begin{exe}
\ex \label{ex:tAmda.pjACti}
\gll <huangshang> nɯ kɯ ko-mɟa tɕe to-rtoʁ ri, nɤkinɯ, tɤ-mda tɕe,  nɯnɯ nɤki, ɕoʁɕoʁ kɯ-qarŋe kɯ tɤ-kɤ-βzu pjɤ-ɕti. \\
emperor \textsc{dem} \textsc{erg} \textsc{ifr}-catch \textsc{lnk} \textsc{ifr}-look \textsc{lnk} \textsc{filler} \textsc{aor}-be.the.time \textsc{lnk} \textsc{dem} \textsc{filler} paper \textsc{sbj}:\textsc{pcp}-be.yellow \textsc{erg} \textsc{aor}-\textsc{obj}:\textsc{pcp}-make \textsc{ifr}.\textsc{ipfv}-be.\textsc{aff} \\
\\
\glt `[Wang Taichang$_i$ was accused of holding an imperial robe, a crime punishable by death] The emperor$_j$ [had] him$_i$ arrested, but when he$_j$ examined [the so-called imperial robe]$_k$ it turned out that it$_k$ was [just] made of yellow paper.' (150909 xiaocui-zh, 93-95)
\end{exe} 
 
 The rectification clause is always in the Inferential (§\ref{sec:ifr}), not only in narration as in (\ref{ex:tAmda.pjACti}), but also when reflecting the point of view of the speaker as in (\ref{ex:tAmda.pjAtWNu}). It can be used with the mistaken expectation construction (§\ref{sec:cimame}).
 
\begin{exe}
\ex \label{ex:tAmda.pjAtWNu}
\gll  aʑo ``nɯ ɕɯ tɯ-ŋu kɯ" nɯ-sɯso-t-a ri,  tɤ-mda tɕe nɤkinɯ, a-ftsa pjɤ-tɯ-ŋu  \\
\textsc{1sg} \textsc{dem} who 2-be:\textsc{fact} \textsc{sfp} \textsc{aor}-think-\textsc{pst}:\textsc{tr}-\textsc{1sg} \textsc{lnk} \textsc{aor}-be.the.time \textsc{lnk} \textsc{filler} \textsc{1sg}.\textsc{poss}-ZCh \textsc{ifr}.\textsc{ipfv}-2-be \\
\glt `I was wondering who you were, but it turns out that you are my nephew!' (150907 yingning-zh, 73-74)
\end{exe} 

The adverb \forme{təmdánə} `actually' in Tshobdun \citep[44;802]{jackson19tshobdun} is probably grammaticalized from a cognate construction.

 \subsection{Addition} \label{sec:addition.clauses}
 
\subsubsection{Neutral addition} \label{sec:neutral.addition}
Neutral additive constructions describe events that are related but for which neither a temporal sequence, a causal nor a hierarchical relationship can be assumed. This type of meaning can be expressed by coordinated clauses (§\ref{sec:coordination}) with the linkers \forme{tɕe} or \forme{qʰe} as in (\ref{ex:pjAmenW.tCe}) or parataxis (\ref{ex:YWwxti.YWtshu}).


 \begin{exe}
\ex \label{ex:pjAmenW.tCe}
\gll ʑara χsɯm ma pjɤ-me-nɯ tɕe tɕendɤre nɯ-nɯŋa ci pjɤ-tu. \\
they three apart.from \textsc{ifr}.\textsc{ipfv}-not.exist-\textsc{pl} \textsc{lnk} \textsc{lnk} \textsc{3pl}.\textsc{poss}-cow \textsc{indef} \textsc{ifr}.\textsc{ipfv}-exist \\
\glt `There were only the three of them, and they had a cow.' (07-deluge, 3)
\end{exe}
 

\begin{exe}
\ex \label{ex:YWwxti.YWtshu}
\gll ɯ-pʰoŋbu ra ɲɯ-wxti, ɲɯ-tsʰu ʑo.  \\
\textsc{3sg}.\textsc{poss}-body \textsc{pl} \textsc{sens}-big \textsc{sens}-fat \textsc{emph} \\
\glt `Its body is big and fat.' (26-GZo, 12)
\end{exe}
 
 A more specific way to indicate neutral addition is the comitative \japhug{cʰo}{and, with} and its variants \forme{cʰondɤre} and \forme{cʰonɤ} (§\ref{sec:comitative}), which are used as a clause linkers in addition to its function as noun phrase coordinator.
 
\begin{exe}
\ex \label{ex:YWwxti.cho.YWwxti}
\gll  mbro sɤznɤ ɲɯ-wxti. ɲɯ-mbro cʰo ɲɯ-wxti. \\
horse comp \textsc{sens}-be.big  \textsc{sens}-be.high  \textsc{comit} \textsc{sens}-be.big \\
\glt `It is larger than a horse. Higher and larger.'  (19-rNamoN, 15)
\end{exe}
 
 The coordinated clauses do not need to have a completely parallel syntactic structure: in (\ref{ex:YWmpCAr.cho.khatoR.GAZu}), the clause preceding \forme{cʰo} has an adjectival stative predicate, while the second one contains an existential verb with the nominal \japhug{kʰatoʁ}{variegated} (§\ref{sec:tibetan.colours}).
 
\begin{exe}
\ex \label{ex:YWmpCAr.cho.khatoR.GAZu}
\gll   ɯ-ku nɯra rcanɯ, wuma ʑo ɲɯ-mpɕɤr cʰo kʰatoʁ ʑo ɣɤʑu \\
\textsc{3sg}.\textsc{poss}-head \textsc{dem}:\textsc{pl} \textsc{unexp}:\textsc{foc} really \textsc{emph} \textsc{sens}-be.beautiful \textsc{comit} variegated \textsc{emph} exist:\textsc{sens} \\
\glt `Its head, it is very beautiful, and variegated.' (24-qro, 84)
\end{exe}

The comitative \forme{cʰo} can follow the linkers \forme{qʰe} and \forme{tɕe} when used to link clauses as in (\ref{ex:Zo.qhe.cho}), but notice that the emphatic marker (§\ref{sec:emphatic.Zo}) and the linker \forme{ʑo qʰe} are repeated in both the clause preceding \forme{cʰo} and the one following it.

\begin{exe}
\ex \label{ex:Zo.qhe.cho}
\gll tɕe nɯ ɯ-rɣi a-mɤ-pɯ-ɕe ra ma pjɯ-tsɣi mɤ-cʰa tɕe tɕendɤre a-nɯ-ɤci \textbf{ʑo} \textbf{qʰe} cʰo ftɕar a-kɤ-ndzoʁ \textbf{ʑo} \textbf{qʰe} li tu-ɬoʁ ɕti \\
\textsc{lnk} \textsc{dem} \textsc{3sg}.\textsc{poss}-grain \textsc{irr}-\textsc{neg}-\textsc{pfv}:\textsc{down}-go be.needed:\textsc{fact} \textsc{lnk} \textsc{ipfv}-be.rotten \textsc{neg}-can:\textsc{fact} \textsc{lnk} \textsc{lnk} \textsc{irr}-\textsc{pfv}-get.wet \textsc{emph} \textsc{lnk} \textsc{comit} summer \textsc{irr}-\textsc{pfv}-\textsc{acaus}:attach \textsc{emph} \textsc{lnk} again \textsc{ipfv}-come.out be.\textsc{aff}:\textsc{fact} \\
\glt `One should not let its grains go into [the ground], because they cannot rot, and when they get wet and the spring comes, they grow again.' (-08-qaJAGi, 48)
\end{exe}

 \subsubsection{Correlative addition} \label{sec:correlative.addition}
The correlative (§\ref{sec:correlative.clauses}) additive focus markers \forme{tɕi} and \forme{ri} (§\ref{sec:ri.additive}) follow noun phrases in a series of two or more clauses in parataxis. The main verbs of these clauses are often identical and redundant, but are not always necessarily so, as in (\ref{ex:tCi.YWtshi.tCi.YWndze}), where a clause with \japhug{tsʰi}{drink} is conjoined with clauses with \japhug{ndza}{eat} as main verb.

 \begin{exe}
\ex \label{ex:tCi.YWtshi.tCi.YWndze}
\gll [ɕa] tɕi ɲɯ-ndze, [ɕɤci] tɕi ɲɯ-tsʰi, [tɤ-lu ta-mar] tɕi ɲɯ-ndze  \\
meat also \textsc{sens}-eat[III] meat.stew also \textsc{sens}-drink  \textsc{indef}.\textsc{poss}-milk \textsc{indef}.\textsc{poss}-butter also \textsc{sens}-eat[III] \\
\glt  `[Pigs] eat meat, slurp meat stew, and also eat butter and have milk.' (05-paR, 30)
\end{exe}

These markers can also have scope over verbs, in the existential construction as in (\ref{ex:ri.kWmWm.ri}) or with a modal auxiliary verb as main predicate as (\ref{ex:turWCmi.ri}). The meaning of this construction is `both $X$ and $Y$' when used with positive copulas or modal verbs, and `neither $X$ nor $Y$' with negative ones.  

 \begin{exe}
\ex \label{ex:ri.kWmWm.ri}
 \gll   kɯroz kɯ-mɯm ri maŋe, kɯroz mɤ-kɯ-ɣɤ-mɲɤt ri maŋe qʰe, \\
specially \textsc{sbj}:\textsc{pcp}-be.tasty also not.exist:\textsc{sens} specially \textsc{neg}-\textsc{sbj}:\textsc{pcp}-\textsc{facil}-be.spoiled also not.exist:\textsc{sens} \textsc{lnk} \\
 \glt ` `[Scoring bread] neither [makes it] specially tasty nor really [prevents it from] spoiling.'   (160706 thotsi, 27)
  \end{exe}

 \begin{exe}
\ex \label{ex:turWCmi.ri}
 \gll   tɕendɤre tu-rɯɕmi ri mɤ-kɯ-kʰɯ, cʰɯ-nɯrɤɣo ri mɤ-kɯ-kʰɯ ci ɲɤ-k-ɤβzu-ci. \\
 \textsc{lnk} \textsc{ipfv}-speak also \textsc{neg}-\textsc{sbj}:\textsc{pcp}-be.possible  \textsc{ipfv}-sing also \textsc{neg}-\textsc{sbj}:\textsc{pcp}-be.possible  \textsc{indef} \textsc{ifr}-\textsc{peg}-become-\textsc{peg} \\
 \glt `She became unable to speak and to sing.' (150819 haidenver-zh, 301)
 \end{exe}
 
\subsubsection{Incremental addition} \label{sec:incremental.addition}
Four constructions meaning `not only ... but also ...' (called in Chinese \ch{递进复句}{dìjìn fùjù}{incremental complex clause}) are found in Japhug.

First, the locutions \forme{ʁo alala ri} or \forme{ʁo alala ma}, comprising the adversative topic marker \forme{ʁo} (§\ref{sec:adversative.topic}) and the adversative \forme{ri} (§\ref{sec:concessive.clauses}), can express incremental addition as in (\ref{ex:pjAmpCAr.Roalalari}).

\begin{exe}
\ex \label{ex:pjAmpCAr.Roalalari}
\gll ɯʑo [pjɤ-mpɕɤr] ʁo alala ri, rɤɣo ri pjɤ-mkʰɤz, tɯ-rɟaʁ ri pjɤ-mkʰɤz. \\
\textsc{3sg} \textsc{ipfv}.\textsc{ifr}-be.beautiful \textsc{advers} not.only \textsc{lnk} song also \textsc{ifr}.\textsc{ipfv}-be.expert \textsc{nmlz}:\textsc{action}-dance also \textsc{ifr}.\textsc{ipfv}-be.expert  \\
\glt `Not only was he good-looking, he was also very talented at singing and dancing.' (160702 luocha-zh, 4)
\end{exe}

Second, the relator noun \japhug{ɯ-tɤjɯ}{addition}, an alienabilized abstract noun (§\ref{sec:tA.abstract.nouns}) from which the denominal verb \japhug{ɣɤjɯ}{add} was originally derived, has the grammaticalized meaning `in addition to $X$', `not only $X$, but also $Y$' when used as clausal linker as in (\ref{ex:WtAjW.tCe.YWsAzoNzoN}).

\begin{exe}
\ex \label{ex:WtAjW.tCe.YWsAzoNzoN}
\gll  [ɕɯ-mŋɤm] ɯ-tɤjɯ tɕe ɲɯ-sɤzoŋzoŋ ʑo ŋu \\
\textsc{caus}-hurt:\textsc{fact} \textsc{3sg}.\textsc{poss}-addition \textsc{lnk} \textsc{sens}-cause.numbing.sensation \textsc{emph} be:\textsc{fact} \\
\glt `Not only does [nettle] hurt, it also causes a numbing sensation.' (140428 mtshalu, 6)
\end{exe}

The noun \forme{ɯ-tɤjɯ} can also follow the demonstrative \forme{nɯ} (§\ref{sec:anaphoric.demonstrative.pro}), anaphorically referring to the preceding clause(s) (\ref{ex:nW.WtAjW.tCe}).

\begin{exe}
\ex \label{ex:nW.WtAjW.tCe}
\gll  rɟɤlpu nɯ kɯ wuma ʑo (...) tó-wɣ-raχtɕɤz, ɲɤ́-wɣ-mgrɯn. tɕe nɯ ɯ-tɤjɯ tɕe tɕendɤre li iɕqʰa nɯ, ɯ-rʑaβ ra ɲɤ-ɕar. \\
king \textsc{dem} \textsc{erg} really \textsc{emph} {  } \textsc{ifr}-\textsc{inv}-cherish \textsc{ifr}-\textsc{inv}-receive.as.guest \textsc{lnk} \textsc{dem} \textsc{3sg}.\textsc{poss}-addition \textsc{lnk} \textsc{lnk} again \textsc{filler} \textsc{dem} \textsc{3sg}.\textsc{poss}-wife \textsc{pl} \textsc{ifr}-search \\
\glt `The king (...) treated him well as a guest, and in addition found a wife for him.' (140511 xinbada-zh, 33-35)
\end{exe}

Third, the locution \forme{mɤra ma}, which is possibly grammaticalized form the negative form of the modal verb \japhug{ra}{be needed}, `be necessary' (§\ref{sec:ra.khW.jAG.verb}) with the adversative linker \forme{ma} (§\ref{sec:concessive.clauses}), also has the same meaning as the constructions described above.

\begin{exe}
\ex \label{ex:mAra.ma.nW.sAznA}
\gll tɯtsɣe ɯ-kɯ-βzu nɯ kɯ paχɕi mɯ-ɲɤ-mbi mɤra ma nɯ sɤznɤ to-nɤmqe tɕe jo-sɯx-ɕe. \\
commerce \textsc{3sg}.\textsc{poss}-make \textsc{dem} \textsc{erg} apple \textsc{neg}-\textsc{ifr}-give not.only \textsc{lnk} \textsc{dem} \textsc{comp} \textsc{ifr}-scold \textsc{lnk} \textsc{ifr}-\textsc{caus}-go \\
\glt `The merchant not only did not give him an apple, but scolded him and sent him away.' (150904 zhongli-zh, 15)
\end{exe}

Fourth, the form \forme{mɤkɯjɤɣ kɯ}, from the negative participle or infinitive of either the modal auxiliary \japhug{jɤɣ}{be allowed} (§\ref{sec:ra.khW.jAG.verb}) or the phasal verb \japhug{jɤɣ}{finish} (§\ref{sec:phasal.complements}), also means `not only $X$, but' as in (\ref{ex:mAkWjAG.kW}).

 \begin{exe}
\ex \label{ex:mAkWjAG.kW}
\gll tɤ-mtʰɯm nɯra tu-ndze mɤkɯjɤɣ kɯ, ɯ-di ɲɯ-ɕɯ-mnɤm \\
\textsc{indef}.\textsc{poss}-meat \textsc{dem}:\textsc{pl} \textsc{ipfv}-eat[III] not.only \textsc{erg} \textsc{3sg}.\textsc{poss}-smell \textsc{sens}-\textsc{caus}-smell \\
\glt  `Not only does [the mouse] eat meat, it also stinks it up.' (27-spjaNkW, 198)
\end{exe}


When the first clause is in negative form as in (\ref{ex:mAra.ma.nW.sAznA}), the meaning of incremental additive construction is similar to the adversative additive construction with \japhug{mɤ́ɣrɤz}{instead} `not only $\neg X$, but on the contrary $Y$' (§\ref{sec:rectification.clauses}). The locution \forme{nɯ sɤz(nɤ)} `rather than that, instead' can be added in the second clause.

These first three additive markers also found on noun phrases, instead of subordinate clauses (§\ref{sec:incremental.add.np}).


\subsection{Exceptive} \label{sec:exceptive.clauses}

\subsubsection{\japhug{laʁma}{apart from the fact that}}
The postposition \japhug{laʁma}{apart from the fact that}, which derives from the exceptive postposition \japhug{ma}{apart from} (§\ref{sec:exceptive}), cannot take a noun phrase, and requires a finite clause instead, as in (\ref{ex:GWrni.laRma})

\begin{exe}
\ex \label{ex:GWrni.laRma}
\gll tɕe [ɯ-rqʰu nɯ ɣɯrni] laʁma ɯ-ŋgɯ nɯ sɤjku cʰo ɲɯ-naχtɕɯɣ-ndʑi ri \\
\textsc{lnk} \textsc{3sg}.\textsc{poss}-skin \textsc{dem} be.red:\textsc{fact} apart.from \textsc{3sg}.\textsc{poss}-inside \textsc{dem} white.birch \textsc{comit} \textsc{sens}-be.the.same-\textsc{du} \textsc{lnk} \\
\glt `Apart from the fact that its bark is red, its inside is the same as that of the white birch.' (06-mbrAj, 13)
\end{exe} 

The semantic scope of \forme{laʁma} can however be a noun phrase, if the verb in the exceptive clause is repeated in the main clause, as in (\ref{ex:mAndze.laRma}).

\begin{exe}
\ex \label{ex:mAndze.laRma}
\gll  [kɤ-kɤ-pu kɤ-kɤ-sqa kɯ-fse nɯra mɤ-ndze] laʁma, nɯ ɯ-ro nɯ lonba tu-ndze ɕti. \\
\textsc{aor}-\textsc{obj}:\textsc{pcp}-bake \textsc{aor}-\textsc{obj}:\textsc{pcp}-cook  \textsc{sbj}:\textsc{pcp}-be.like \textsc{dem}:\textsc{pl} \textsc{neg}-eat:\textsc{fact} apart.from \textsc{dem} \textsc{3sg}.\textsc{poss}-rest \textsc{dem} all \textsc{ipfv}-eat be.\textsc{aff}:\textsc{fact} \\
\glt `Apart from the fact that it does not eat food that has been baked or cooked, it eats everything else (=apart from cooked food, it eats everything else).' (19-GzW, 11)
\end{exe} 

\subsubsection{\japhug{tɕʰimaʁnɤ}{at least}}
The linker \forme{tɕʰimaʁnɤ} or \forme{tɕʰimaʁ}, optionally combined with \japhug{tsaʁ}{just, only}, occurs in imperative or hortative sentences with the meaning `at least'. As illustrated by (\ref{ex:tChimaRnA.CWrAfCAttCi}) (where \forme{tsaʁ} occurs in one version of the story, and does not occur in the other) and (\ref{ex:tChimaR.tsaR}), \forme{tɕʰimaʁnɤ} is placed at the beginning of the clause, while \forme{tsaʁ} either follows the verb or a constituent overt which it has scope.

\begin{exe}
\ex \label{ex:tChimaRnA.CWrAfCAttCi}
\gll wortɕʰi wojɤr ʑo tɕʰimaʁnɤ a-ɣi ra nɯ-pʰe ɕɯ-rɤfɕɤt-tɕi (tsaʁ) ma tɕetʰa ɣɯ-nɯzdɯɣ-a-nɯ \\
please please \textsc{emph} at.least \textsc{1sg}.\textsc{poss}-relative \textsc{pl} \textsc{3pl}.\textsc{poss}-\textsc{dat} \textsc{tral}-tell:\textsc{fact}-\textsc{1du} just \textsc{lnk} later \textsc{inv}-worry.about:\textsc{fact}-\textsc{1sg}-\textsc{pl} \\
\glt `Please, at least let the two of us go and inform my relatives, otherwise they will be worried about me.' (qachGa 2012, 76)
\end{exe} 

\begin{exe}
\ex \label{ex:tChimaR.tsaR}
\gll  tɕʰimaʁ ``pɯ-tɯ-χɕu" tsaʁ tu-ti-a ma \\
at.least \textsc{pst}.\textsc{ipfv}-2-be.strong just \textsc{ipfv}-say-\textsc{1sg} \textsc{lnk} \\
\glt `I [should] at least say `thank you'.' (2014-kWlAG, 624)
\end{exe} 

The linker \forme{tɕʰimaʁnɤ} is built from the interrogative pronoun \japhug{tɕʰi}{what} (§\ref{sec:tChi}), the negative copula \japhug{maʁ}{not be} (§\ref{sec:suppletive.negative}) and the postposition \forme{nɤ} (§\ref{sec:additive.nA}). The original meaning of this locution was probably `whatever it is not', with the free-choice indefinite function of \forme{tɕʰi} (§\ref{sec:interrogative.indef}).

\subsection{Disjunction} \label{sec:disjunction.clauses}
Exclusive disjunction is expressed by the linker \forme{nɯmaʁnɤ}, either between the two alternative clauses as in (\ref{ex:chWBdenW.YWmbinW}) or in a correlative construction (§\ref{sec:correlative.clauses}), repeated before each clause (\ref{ex:nWmaRnA.tuwGnWxsWr}).

\begin{exe}
\ex \label{ex:chWBdenW.YWmbinW}
\gll tɕe cʰɯ-βde-nɯ nɯmaʁnɤ fsapaʁ ɲɯ-mbi-nɯ ŋgrɤl ma \\
\textsc{lnk} \textsc{ipfv}-throw-\textsc{pl} otherwise animal \textsc{ipfv}-give-\textsc{pl} be.usually.the.case:\textsc{fact} \textsc{lnk} \\
\glt `(People uproot it and) either they throw it away, or give it to the animals (to eat).' (12-Zmbroko, 120-122)
\end{exe} 

\begin{exe}
\ex \label{ex:nWmaRnA.tuwGnWxsWr}
\gll nɯmaʁnɤ tú-wɣ-nɯ-xsɯr, nɯmaʁnɤ <ban> tú-wɣ-βzu tɕe. \\
otherwise \textsc{ipfv}-\textsc{inv}-\textsc{auto}-fry otherwise mix \textsc{ipfv}-\textsc{inv}-make \textsc{lnk} \\
\glt `People either fry it, or mix it in salad.' (conversation 14-05-10)
\end{exe} 

The linker \japhug{nɯmaʁnɤ}{otherwise} has stress on the second syllable (\forme{nɯmáʁnɤ}), and transparently comes from the demonstrative \forme{nɯ} (§\ref{sec:anaphoric.demonstrative.pro}), the copula \japhug{maʁ}{not be} (§\ref{sec:suppletive.negative}) and the adposition \forme{nɤ}, probably from what originally was the protasis of a conditional construction `if it is not', with initial reduplication $\dagger$\forme{nɯ mɯ\redp{}maʁ nɤ} or Interrogative $\dagger$\forme{nɯ ɯ́-maʁ nɤ} (§\ref{sec:real.conditional}, §\ref{sec:redp.protasis}). 

In questions, disjunction between several clauses can be express by adding the interrogative particle \forme{ɕi} after each clause except the last, as in (\ref{ex:tWnWCe.Ci}).

\begin{exe}
\ex \label{ex:tWnWCe.Ci}
\gll χsɤr rɟɤskɤt ɯ-taʁ tɯ-nɯ-ɕe ɕi, rŋɯl rɟɤskɤt ɯ-taʁ tɯ-nɯ-ɕe ɕi, ɕom rɟɤskɤt ɯ-taʁ tɯ-nɯ-ɕe ɕi, 
si rɟɤskɤt ɯ-taʁ tɯ-nɯ-ɕe? \\
gold stairs \textsc{3sg}.\textsc{poss}-on 2-\textsc{auto}-go:\textsc{fact} \textsc{sfp} silver stairs \textsc{3sg}.\textsc{poss}-on 2-\textsc{auto}-go:\textsc{fact} \textsc{sfp} iron stairs \textsc{3sg}.\textsc{poss}-on 2-\textsc{auto}-go:\textsc{fact} \textsc{sfp} wood stairs \textsc{3sg}.\textsc{poss}-on 2-\textsc{auto}-go:\textsc{fact} \\
\glt `Will you go on the golden stairs, the silver stairs, the iron stairs, or the wooden stairs? (2005 Kunbzang, 214)
\end{exe} 

Both \japhug{nɯmaʁnɤ}{otherwise} and \forme{ɕi} are only very rarely used with noun phrases instead of clauses (§\ref{sec:disjunction.nouns}).

