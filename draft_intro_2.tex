\chapter*{A grammar of Japhug\\
Guillaume Jacques\\
13-04-2019}  
\sloppy
This draft contains 13 chapters of a grammar of Japhug (Sino-Tibetan/Trans-Himalayan, China, Sichuan, Mbarkham county). It focuses on the dialect of Kamnyu village (\forme{kɤmɲɯ}, in Chinese \zh{干木鸟村}, coordinates: 32.2068034,101.9481512).

The examples are mainly based on a text corpus partially available on the Pangloss Archive (\url{http://lacito.vjf.cnrs.fr/pangloss/corpus/list\_rsc.php?lg=Japhug}, though additional examples from casual conversations and elicitations are also included.

The glosses follow the Leipzig glossing rules. Other abbreviations in this work include: \textsc{auto}  autobenefactive-spontaneous, \textsc{anticaus} anticausative, \textsc{antipass} antipassive, \textsc{appl} applicative, \textsc{emph} emphatic, \textsc{fact} factual, \textsc{genr} generic, \textsc{ifr} inferential, \textsc{indef} indefinite, \textsc{inv} inverse,  \textsc{lnk} linker, p.n. personal name, pl.n. place name, \textsc{poss} possessor, \textsc{ego.prs} egophoric present, \textsc{prog} progressive, \textsc{sens} sensory. 

This document comprises 13 chapters, including the noun phrase, and verbal morphology (everything excluding TAME). In total, the following chapters are planned:

\begin{itemize}
\item 1-01 The Japhug language and its speakers
\item 1-02 A sketch of Japhug grammar
\item 2-01 Phonological inventory 
\item 2-02 Consonant clusters and partial reduplication
\item 2-03 Syllable structure and sandhi
\item 3-01 \textbf{Nominal morphology}
\item 3-02 \textbf{Pronouns, Demonstratives and Indefinites}
\item 3-03 \textbf{Numerals and counted nouns}
\item 3-04 \textbf{Postpositions and relator nouns}
\item 3-05 \textbf{The noun phrase}
\item 3-06 Ideophones
\item 3-07 Adverbs and sentence final particles
\item 4-01 \textbf{The verbal template}
\item 4-02 \textbf{Person indexation}
\item 4-03 \textbf{Orientation and associated motion}
\item 4-04 \textbf{Non-finite verbal morphology}
\item 4-05a \textbf{Voice derivations (valency-increasing)}
\item 4-05b \textbf{Voice derivations (valency-decreasing)}
\item 4-05c \textbf{Other derivations} 
\item 4-06 \textbf{Denominal derivations}
\item 4-07 TAME
\item 5-01 Main clauses
\item 5-02 Relativization
\item 5-03 Complementation
\item 5-04 Temporal and conditional clauses
\item 5-05 Other subordinate clauses
\end{itemize}

 References to planned sections that have not yet been written are indicated by §XXX.
% Acknowledgements: Martin Haspelmath, Alexis Michaud, Marc Miyake, Roland Pooth, Theo Yeh, 